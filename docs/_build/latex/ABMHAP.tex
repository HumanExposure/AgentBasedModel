%% Generated by Sphinx.
\def\sphinxdocclass{report}
\documentclass[letterpaper,10pt,english]{sphinxmanual}
\ifdefined\pdfpxdimen
   \let\sphinxpxdimen\pdfpxdimen\else\newdimen\sphinxpxdimen
\fi \sphinxpxdimen=.75bp\relax

\PassOptionsToPackage{warn}{textcomp}
\usepackage[utf8]{inputenc}
\ifdefined\DeclareUnicodeCharacter
 \ifdefined\DeclareUnicodeCharacterAsOptional
  \DeclareUnicodeCharacter{"00A0}{\nobreakspace}
  \DeclareUnicodeCharacter{"2500}{\sphinxunichar{2500}}
  \DeclareUnicodeCharacter{"2502}{\sphinxunichar{2502}}
  \DeclareUnicodeCharacter{"2514}{\sphinxunichar{2514}}
  \DeclareUnicodeCharacter{"251C}{\sphinxunichar{251C}}
  \DeclareUnicodeCharacter{"2572}{\textbackslash}
 \else
  \DeclareUnicodeCharacter{00A0}{\nobreakspace}
  \DeclareUnicodeCharacter{2500}{\sphinxunichar{2500}}
  \DeclareUnicodeCharacter{2502}{\sphinxunichar{2502}}
  \DeclareUnicodeCharacter{2514}{\sphinxunichar{2514}}
  \DeclareUnicodeCharacter{251C}{\sphinxunichar{251C}}
  \DeclareUnicodeCharacter{2572}{\textbackslash}
 \fi
\fi
\usepackage{cmap}
\usepackage[T1]{fontenc}
\usepackage{amsmath,amssymb,amstext}
\usepackage{babel}
\usepackage{times}
\usepackage[Sonny]{fncychap}
\usepackage{sphinx}

\usepackage{geometry}

% Include hyperref last.
\usepackage{hyperref}
% Fix anchor placement for figures with captions.
\usepackage{hypcap}% it must be loaded after hyperref.
% Set up styles of URL: it should be placed after hyperref.
\urlstyle{same}

\addto\captionsenglish{\renewcommand{\figurename}{Fig.}}
\addto\captionsenglish{\renewcommand{\tablename}{Table}}
\addto\captionsenglish{\renewcommand{\literalblockname}{Listing}}

\addto\captionsenglish{\renewcommand{\literalblockcontinuedname}{continued from previous page}}
\addto\captionsenglish{\renewcommand{\literalblockcontinuesname}{continues on next page}}

\addto\extrasenglish{\def\pageautorefname{page}}

\setcounter{tocdepth}{3}
\setcounter{secnumdepth}{3}

% Jupyter Notebook prompt colors
\definecolor{nbsphinxin}{HTML}{303F9F}
\definecolor{nbsphinxout}{HTML}{D84315}
% ANSI colors for output streams and traceback highlighting
\definecolor{ansi-black}{HTML}{3E424D}
\definecolor{ansi-black-intense}{HTML}{282C36}
\definecolor{ansi-red}{HTML}{E75C58}
\definecolor{ansi-red-intense}{HTML}{B22B31}
\definecolor{ansi-green}{HTML}{00A250}
\definecolor{ansi-green-intense}{HTML}{007427}
\definecolor{ansi-yellow}{HTML}{DDB62B}
\definecolor{ansi-yellow-intense}{HTML}{B27D12}
\definecolor{ansi-blue}{HTML}{208FFB}
\definecolor{ansi-blue-intense}{HTML}{0065CA}
\definecolor{ansi-magenta}{HTML}{D160C4}
\definecolor{ansi-magenta-intense}{HTML}{A03196}
\definecolor{ansi-cyan}{HTML}{60C6C8}
\definecolor{ansi-cyan-intense}{HTML}{258F8F}
\definecolor{ansi-white}{HTML}{C5C1B4}
\definecolor{ansi-white-intense}{HTML}{A1A6B2}
\definecolor{ansi-default-inverse-fg}{HTML}{FFFFFF}
\definecolor{ansi-default-inverse-bg}{HTML}{000000}

% Define "notice" environment, which was removed in Sphinx 1.7.
% At some point, "notice" should be replaced by "sphinxadmonition",
% which is available since Sphinx 1.5.
\makeatletter
\@ifundefined{notice}{%
\newenvironment{notice}{\begin{sphinxadmonition}}{\end{sphinxadmonition}}%
}{}
\makeatother



\title{The Agent-Based Model of Human Activity Patterns   (ABMHAP): Documentation and Users Guide}
\date{Jun 25, 2018}
\release{2018.06}
\author{Namdi Brandon}
\newcommand{\sphinxlogo}{\vbox{}}
\renewcommand{\releasename}{Release}
\makeindex

\begin{document}

\maketitle
\sphinxtableofcontents
\phantomsection\label{\detokenize{index::doc}}


The Agent-Based Model of Human Activity Patterns (ABMHAP, pronounced “ab-map”) is one of the
modules for the Life Cycle Human Exposure Model (LC-HEM) project as described in the United States Environmental
Protection Agency (U.S. EPA) research plan, which may be found
\sphinxhref{https://19january2017snapshot.epa.gov/sites/production/files/2016-11/documents/css\_fy16-19\_strap.pdf}{here}.
ABMHAP is a model capable of creating agents that simulate longitudinal human behavior. The current version of
ABMHAP is able to simulate daily routines for the following behaviors:
\begin{enumerate}
\item {} 
Sleeping

\item {} 
Eating Breakfast

\item {} 
Eating Lunch

\item {} 
Eating Dinner

\item {} 
Working

\item {} 
Commuting to Work

\item {} 
Commuting from Work

\item {} 
Being Idle (i.e., time spent not doing the above activities)

\end{enumerate}

ABMHAP requires the user to input parameters that describe the longitudinal variation in behavior of a single individual. Information on the design of the model, modeling assumptions, and limitations of the model are described in the publications:
\begin{enumerate}
\item {} 
Brandon et al. \sphinxstyleemphasis{Simulating Exposure-Related Behaviors using Agent-Based Models Embedded with Needs-Based
Artificial Intelligence}, Journal of Exposure Science and Environmental Epidemiology, submitted 2018.

\item {} 
Brandon, N and Price, P. Simulating \sphinxstyleemphasis{Long-Term Patterns in Exposure-Related Behaviors using the Agent-Based
Model of Human Activity Patterns}, Journal of Exposure Science and Environmental Epidemiology, submitted 2018.

\end{enumerate}

The current version of ABMHAP is written in Python version 3.5.3. More information on the Python programming
language may be found \sphinxhref{https://www.python.org/}{here}. The Python libraries that must be installed in
order for ABMHAP to run are listed below.
\begin{itemize}
\item {} 
matplotlib

\item {} 
multiprocessing

\item {} 
numpy

\item {} 
pandas

\item {} 
scipy

\item {} 
sphinx

\item {} 
statsmodels

\end{itemize}

Some of the features within ABMHAP also use Jupyter Notebook. More information on Project Jupyter
may be found \sphinxhref{https://jupyter.org}{here}.

ABMHAP is written by Dr. Namdi Brandon (ORCID: 0000-0001-7050-1538).
\begin{description}
\item[{\sphinxstylestrong{Disclaimer}}] \leavevmode
The United States Environmental Protection Agency through its Office of Research and Development has
developed this software. The code is made publicly available to better communicate the research. All
input data used for a given application should be reviewed by the researcher so that the model results
are based on appropriate data for any given application. This model is under continued development. The
model and data included herein do not represent and should not be construed to represent any Agency
determination or policy.

\end{description}


\chapter{Running the ABMHAP Code}
\label{\detokenize{index:running-the-abmhap-code}}\label{\detokenize{index:welcome-to-the-documentation-for-the-agent-based-model-of-human-activity-patterns-abmhap}}
The current version of ABMHAP may be run in two different ways.
\begin{enumerate}
\item {} 
ABMHAP may run with user-defined parameters in order to simulate one household

\item {} 
ABMHAP may run with parameters derived from empirical data as a Monte-Carlo simulation in order
to simulate multiple households

\end{enumerate}


\section{How to Run the Code Using User-Defined Parameters}
\label{\detokenize{index:how-to-run-the-code-using-user-defined-parameters}}
The following describes how to run an ABMHAP simulation of \sphinxstylestrong{one household} (ABMHAP currenlty
simulates one agent per household). In order to do so, the user must do the following:
\begin{enumerate}
\item {} 
set the input parameters of the simulation in the file \sphinxcode{\sphinxupquote{\textbackslash{}run\textbackslash{}main\_params.py}}

\item {} 
run the executable using \sphinxcode{\sphinxupquote{\textbackslash{}run\textbackslash{}main.py}}

\end{enumerate}


\subsection{Setting the input parameters}
\label{\detokenize{index:setting-the-input-parameters}}
In order to run ABMHAP, the user must set the following types of input parameters in
\sphinxcode{\sphinxupquote{\textbackslash{}run\textbackslash{}main\_params.py}}:
\begin{enumerate}
\item {} 
input parameters that govern the general logistics of the simulation

\item {} 
input parameters that govern the the length of simulation duration

\item {} 
input parameters that define the behavior of the agent

\end{enumerate}

For illustrative purposes, what follows is a demonstration of how to parametrize a run for ABMHAP.

The below code does the following:
\begin{itemize}
\item {} 
informs the algorithm to not print the output to the screen

\item {} 
informs the algorithm to not plot the output

\item {} 
informs the algorithm to not save the output to a file

\item {} 
should the output file be saved, sets the output file to \sphinxcode{\sphinxupquote{\textbackslash{}output\_directory\textbackslash{}outputfile.csv}}

\end{itemize}

The user must set the input parameters that govern the general logistics of the simulation:

\fvset{hllines={, ,}}%
\begin{sphinxVerbatim}[commandchars=\\\{\}]
\PYG{c+c1}{\PYGZsh{} whether (if True) or not (if False) the output of the simulation should}
\PYG{c+c1}{\PYGZsh{} print a message to screen}
\PYG{n}{do\PYGZus{}print}    \PYG{o}{=} \PYG{k+kc}{False}

\PYG{c+c1}{\PYGZsh{} whether (if True) or not (if False) the output of the simulation should}
\PYG{c+c1}{\PYGZsh{} be plotted a message to screen}
\PYG{n}{do\PYGZus{}plot}     \PYG{o}{=} \PYG{k+kc}{False}

\PYG{c+c1}{\PYGZsh{} whether (if True) or not (if False) the output of the simulation should}
\PYG{c+c1}{\PYGZsh{} be saved in a file}
\PYG{n}{do\PYGZus{}save}     \PYG{o}{=} \PYG{k+kc}{False}

\PYG{c+c1}{\PYGZsh{} the name of the output file should the output be saved. The filename}
\PYG{c+c1}{\PYGZsh{} should have a \PYGZdq{}.csv\PYGZdq{} extension}
\PYG{n}{fname}       \PYG{o}{=} \PYG{l+s+s1}{\PYGZsq{}}\PYG{l+s+s1}{output\PYGZus{}directory}\PYG{l+s+se}{\PYGZbs{}\PYGZbs{}}\PYG{l+s+s1}{outputfile.csv}\PYG{l+s+s1}{\PYGZsq{}}
\end{sphinxVerbatim}

The following code shows how to set ABMHAP to run starting on Sunday, Day 0 starting from 16:00
and ending on Monday, Day 7 at 0:00. It’s recommended that the user start running the code on a Sunday or Saturday
at 16:00 in order to minimize potential activity conflicts at initiation.

The user must set the input parameters dealing with the duration of the simulation:

\fvset{hllines={, ,}}%
\begin{sphinxVerbatim}[commandchars=\\\{\}]
\PYG{c+c1}{\PYGZsh{} the number of days for the simulation}
\PYG{n}{num\PYGZus{}days}    \PYG{o}{=} \PYG{l+m+mi}{7}

\PYG{c+c1}{\PYGZsh{} the number of additional hours}
\PYG{n}{num\PYGZus{}hours}   \PYG{o}{=} \PYG{l+m+mi}{8}

\PYG{c+c1}{\PYGZsh{} the number of additional minutes}
\PYG{n}{num\PYGZus{}min}     \PYG{o}{=} \PYG{l+m+mi}{0}
\end{sphinxVerbatim}

The user must set the input parameters dealing with when in the simulation year the simulation should start:

\fvset{hllines={, ,}}%
\begin{sphinxVerbatim}[commandchars=\\\{\}]
\PYG{c+c1}{\PYGZsh{} start the simulation on Sunday, Day 0 at 16:00}
\PYG{n}{t\PYGZus{}start}     \PYG{o}{=} \PYG{n}{WINTER} \PYG{o}{*} \PYG{n}{SEASON\PYGZus{}2\PYGZus{}MIN} \PYG{o}{+} \PYG{l+m+mi}{0} \PYG{o}{*} \PYG{n}{WEEK\PYGZus{}2\PYGZus{}MIN} \PYGZbs{}
            \PYG{o}{+} \PYG{n}{SUNDAY} \PYG{o}{*} \PYG{n}{DAY\PYGZus{}2\PYGZus{}MIN} \PYG{o}{+} \PYG{l+m+mi}{16} \PYG{o}{*} \PYG{n}{HOUR\PYGZus{}2\PYGZus{}MIN}
\end{sphinxVerbatim}

where the following constants are useful in assigning input parameters that define
the start time of the simulation:

\fvset{hllines={, ,}}%
\begin{sphinxVerbatim}[commandchars=\\\{\}]
\PYG{c+c1}{\PYGZsh{} an agent\PYGZhy{}based model module with capabilities concerning time}
\PYG{k+kn}{import} \PYG{n+nn}{temporal}

\PYG{c+c1}{\PYGZsh{} the value of Sunday}
\PYG{n}{SUNDAY}         \PYG{o}{=} \PYG{n}{temporal}\PYG{o}{.}\PYG{n}{SUNDAY}

\PYG{c+c1}{\PYGZsh{} convert one day into minutes}
\PYG{n}{DAY\PYGZus{}2\PYGZus{}MIN}      \PYG{o}{=} \PYG{n}{temporal}\PYG{o}{.}\PYG{n}{DAY\PYGZus{}2\PYGZus{}MIN}

\PYG{c+c1}{\PYGZsh{} convert one hour into minutes}
\PYG{n}{HOUR\PYGZus{}2\PYGZus{}MIN}     \PYG{o}{=} \PYG{n}{temporal}\PYG{o}{.}\PYG{n}{HOUR\PYGZus{}2\PYGZus{}MIN}

\PYG{c+c1}{\PYGZsh{} the number of minutes in one season (13 weeks)}
\PYG{n}{SEASON\PYGZus{}2\PYGZus{}MIN}   \PYG{o}{=} \PYG{n}{temporal}\PYG{o}{.}\PYG{n}{SEASON\PYGZus{}2\PYGZus{}MIN}

\PYG{c+c1}{\PYGZsh{} the number of minutes in one week}
\PYG{n}{WEEK\PYGZus{}2\PYGZus{}MIN}     \PYG{o}{=} \PYG{n}{temporal}\PYG{o}{.}\PYG{n}{WEEK\PYGZus{}2\PYGZus{}MIN}

\PYG{c+c1}{\PYGZsh{} the winter season (has the value 0)}
\PYG{n}{WINTER}         \PYG{o}{=} \PYG{n}{temporal}\PYG{o}{.}\PYG{n}{WINTER}
\end{sphinxVerbatim}

The user must set the input parameters that govern the behavior of the agent. The input parameters will govern
the agent’s behavior for the following activities.
\begin{enumerate}
\item {} 
sleeping

\item {} 
eating breakfast

\item {} 
eating lunch

\item {} 
eating dinner

\item {} 
working

\item {} 
commuting to work

\item {} 
commuting from work

\end{enumerate}

In order to set the sleeping behavior, the user must set the the mean and standard deviation for the start time
and end time for the sleep activity. The agent’s behavior for sleeping is set as follows:

\fvset{hllines={, ,}}%
\begin{sphinxVerbatim}[commandchars=\\\{\}]
\PYG{c+c1}{\PYGZsh{} set the mean start time to be 22:00}
\PYG{n}{sleep\PYGZus{}start\PYGZus{}mean}     \PYG{o}{=} \PYG{n}{np}\PYG{o}{.}\PYG{n}{array}\PYG{p}{(} \PYG{p}{[}\PYG{l+m+mi}{22} \PYG{o}{*} \PYG{n}{HOUR\PYGZus{}2\PYGZus{}MIN}\PYG{p}{]} \PYG{p}{)}

\PYG{c+c1}{\PYGZsh{} set the standard deviation of the start time to be 30 minutes}
\PYG{n}{sleep\PYGZus{}start\PYGZus{}std}      \PYG{o}{=} \PYG{n}{np}\PYG{o}{.}\PYG{n}{array}\PYG{p}{(} \PYG{p}{[}\PYG{l+m+mi}{30}\PYG{p}{]} \PYG{p}{)}

\PYG{c+c1}{\PYGZsh{} set the mean end time to be 8:00}
\PYG{n}{sleep\PYGZus{}end\PYGZus{}mean}       \PYG{o}{=} \PYG{n}{np}\PYG{o}{.}\PYG{n}{array}\PYG{p}{(} \PYG{p}{[}\PYG{l+m+mi}{8} \PYG{o}{*} \PYG{n}{HOUR\PYGZus{}2\PYGZus{}MIN}\PYG{p}{]} \PYG{p}{)}

\PYG{c+c1}{\PYGZsh{} set the standard deviation of the end time to be 15 minutes}
\PYG{n}{sleep\PYGZus{}end\PYGZus{}std}        \PYG{o}{=} \PYG{n}{np}\PYG{o}{.}\PYG{n}{array}\PYG{p}{(} \PYG{p}{[}\PYG{l+m+mi}{15}\PYG{p}{]} \PYG{p}{)}
\end{sphinxVerbatim}

In order to set the eat breakfast behavior, the user must set the mean and standard deviation for the start time
and duration for the eat breakfast activity. The agent’s behavior for eating breakfast is set as follows:

\fvset{hllines={, ,}}%
\begin{sphinxVerbatim}[commandchars=\\\{\}]
\PYG{c+c1}{\PYGZsh{} set the mean start time to be 8:00}
\PYG{n}{bf\PYGZus{}start\PYGZus{}mean}       \PYG{o}{=} \PYG{n}{np}\PYG{o}{.}\PYG{n}{array}\PYG{p}{(} \PYG{p}{[}\PYG{l+m+mi}{8} \PYG{o}{*} \PYG{n}{HOUR\PYGZus{}2\PYGZus{}MIN}\PYG{p}{]} \PYG{p}{)}

\PYG{c+c1}{\PYGZsh{} set the standard deviation of the start time to be 10 minutes}
\PYG{n}{bf\PYGZus{}start\PYGZus{}std}        \PYG{o}{=} \PYG{n}{np}\PYG{o}{.}\PYG{n}{array}\PYG{p}{(} \PYG{p}{[}\PYG{l+m+mi}{10}\PYG{p}{]} \PYG{p}{)}

\PYG{c+c1}{\PYGZsh{} set the mean duration to be 15 minutes}
\PYG{n}{bf\PYGZus{}dt\PYGZus{}mean}          \PYG{o}{=} \PYG{n}{np}\PYG{o}{.}\PYG{n}{array}\PYG{p}{(} \PYG{p}{[}\PYG{l+m+mi}{15}\PYG{p}{]} \PYG{p}{)}

\PYG{c+c1}{\PYGZsh{} set the standard deviation of the duration to be 10 minutes}
\PYG{n}{bf\PYGZus{}dt\PYGZus{}std}           \PYG{o}{=} \PYG{n}{np}\PYG{o}{.}\PYG{n}{array}\PYG{p}{(} \PYG{p}{[}\PYG{l+m+mi}{10}\PYG{p}{]} \PYG{p}{)}
\end{sphinxVerbatim}

In order to set the eat lunch behavior, the user must set the mean and standard deviation for the start time
and duration for the eat lunch activity. The agent’s behavior for eating lunch is set as follows:

\fvset{hllines={, ,}}%
\begin{sphinxVerbatim}[commandchars=\\\{\}]
\PYG{c+c1}{\PYGZsh{} set the mean start time to be 12:000}
\PYG{n}{lunch\PYGZus{}start\PYGZus{}mean}       \PYG{o}{=} \PYG{n}{np}\PYG{o}{.}\PYG{n}{array}\PYG{p}{(} \PYG{p}{[}\PYG{l+m+mi}{12} \PYG{o}{*} \PYG{n}{HOUR\PYGZus{}2\PYGZus{}MIN}\PYG{p}{]} \PYG{p}{)}

\PYG{c+c1}{\PYGZsh{} set the standard deviation of start time to be 15 minutes}
\PYG{n}{lunch\PYGZus{}start\PYGZus{}std}        \PYG{o}{=} \PYG{n}{np}\PYG{o}{.}\PYG{n}{array}\PYG{p}{(} \PYG{p}{[}\PYG{l+m+mi}{15}\PYG{p}{]} \PYG{p}{)}

\PYG{c+c1}{\PYGZsh{} set the mean duration to be 30 minutes}
\PYG{n}{lunch\PYGZus{}dt\PYGZus{}mean}          \PYG{o}{=} \PYG{n}{np}\PYG{o}{.}\PYG{n}{array}\PYG{p}{(} \PYG{p}{[}\PYG{l+m+mi}{30}\PYG{p}{]} \PYG{p}{)}

\PYG{c+c1}{\PYGZsh{} set the standard deviation of duration to be 10 minutes}
\PYG{n}{lunch\PYGZus{}dt\PYGZus{}std}           \PYG{o}{=} \PYG{n}{np}\PYG{o}{.}\PYG{n}{array}\PYG{p}{(} \PYG{p}{[}\PYG{l+m+mi}{10}\PYG{p}{]} \PYG{p}{)}
\end{sphinxVerbatim}

In order to set the eat dinner behavior, the user must set the mean and standard deviation for the start time
and duration for the eat dinner activity. The agent’s behavior for eating dinner is set as follows:

\fvset{hllines={, ,}}%
\begin{sphinxVerbatim}[commandchars=\\\{\}]
\PYG{c+c1}{\PYGZsh{} set the mean start time to be 19:00}
\PYG{n}{dinner\PYGZus{}start\PYGZus{}mean}       \PYG{o}{=} \PYG{n}{np}\PYG{o}{.}\PYG{n}{array}\PYG{p}{(} \PYG{p}{[}\PYG{l+m+mi}{19} \PYG{o}{*} \PYG{n}{HOUR\PYGZus{}2\PYGZus{}MIN}\PYG{p}{]} \PYG{p}{)}

\PYG{c+c1}{\PYGZsh{} set the standard deviation of start time to be 10 minutes}
\PYG{n}{dinner\PYGZus{}start\PYGZus{}std}        \PYG{o}{=} \PYG{n}{np}\PYG{o}{.}\PYG{n}{array}\PYG{p}{(} \PYG{p}{[}\PYG{l+m+mi}{10}\PYG{p}{]} \PYG{p}{)}

\PYG{c+c1}{\PYGZsh{} set the mean of duration to be 45 minutes}
\PYG{n}{dinner\PYGZus{}dt\PYGZus{}mean}          \PYG{o}{=} \PYG{n}{np}\PYG{o}{.}\PYG{n}{array}\PYG{p}{(} \PYG{p}{[}\PYG{l+m+mi}{45}\PYG{p}{]} \PYG{p}{)}

\PYG{c+c1}{\PYGZsh{} set the standard deviation of duration to be 5 minutes}
\PYG{n}{dinner\PYGZus{}dt\PYGZus{}std}           \PYG{o}{=} \PYG{n}{np}\PYG{o}{.}\PYG{n}{array}\PYG{p}{(} \PYG{p}{[}\PYG{l+m+mi}{5}\PYG{p}{]} \PYG{p}{)}
\end{sphinxVerbatim}

In order to set the work behavior, the user must set the mean and standard deviation for the start time and
end time for the work activity. The agent’s behavior for working is set as follows:

\fvset{hllines={, ,}}%
\begin{sphinxVerbatim}[commandchars=\\\{\}]
\PYG{c+c1}{\PYGZsh{} set the mean start time to be 9:00}
\PYG{n}{work\PYGZus{}start\PYGZus{}mean}     \PYG{o}{=} \PYG{n}{np}\PYG{o}{.}\PYG{n}{array}\PYG{p}{(} \PYG{p}{[}\PYG{l+m+mi}{9} \PYG{o}{*} \PYG{n}{HOUR\PYGZus{}2\PYGZus{}MIN}\PYG{p}{]} \PYG{p}{)}

\PYG{c+c1}{\PYGZsh{} set the standard deviation of start time to be 15 minutes}
\PYG{n}{work\PYGZus{}start\PYGZus{}std}      \PYG{o}{=} \PYG{n}{np}\PYG{o}{.}\PYG{n}{array}\PYG{p}{(} \PYG{p}{[}\PYG{l+m+mi}{15}\PYG{p}{]} \PYG{p}{)}

\PYG{c+c1}{\PYGZsh{} set the mean end time to be 17:00}
\PYG{n}{work\PYGZus{}end\PYGZus{}mean}       \PYG{o}{=} \PYG{n}{np}\PYG{o}{.}\PYG{n}{array}\PYG{p}{(} \PYG{p}{[}\PYG{l+m+mi}{17} \PYG{o}{*} \PYG{n}{HOUR\PYGZus{}2\PYGZus{}MIN}\PYG{p}{]} \PYG{p}{)}

\PYG{c+c1}{\PYGZsh{} set the standard deviation of end time to be 5 minutes}
\PYG{n}{work\PYGZus{}end\PYGZus{}std}        \PYG{o}{=} \PYG{n}{np}\PYG{o}{.}\PYG{n}{array}\PYG{p}{(} \PYG{p}{[}\PYG{l+m+mi}{5}\PYG{p}{]} \PYG{p}{)}
\end{sphinxVerbatim}

The user must set the agent’s employment status. The agent’s employment status is set as follows:

\fvset{hllines={, ,}}%
\begin{sphinxVerbatim}[commandchars=\\\{\}]
\PYG{c+c1}{\PYGZsh{} an agent\PYGZhy{}based model module for functionality dealing with occupation}
\PYG{k+kn}{import} \PYG{n+nn}{occupation}

\PYG{c+c1}{\PYGZsh{} set the job identifier (job id) as standard job if the agent}
\PYG{c+c1}{\PYGZsh{} is supposed to work}
\PYG{n}{job\PYGZus{}id}   \PYG{o}{=} \PYG{n}{occupation}\PYG{o}{.}\PYG{n}{STANDARD\PYGZus{}JOB}

\PYG{c+c1}{\PYGZsh{} OR set the job identifier (job id) as not having a job if the agent}
\PYG{c+c1}{\PYGZsh{} is NOT supposed to work}
\PYG{n}{job\PYGZus{}id}   \PYG{o}{=} \PYG{n}{occupation}\PYG{o}{.}\PYG{n}{NO\PYGZus{}JOB}
\end{sphinxVerbatim}

In order to set the commute to work behavior, the user must set the mean and standard deviation for the duration
of the commute to work activity. The agent’s behavior for commuting to work is set as follows:

\fvset{hllines={, ,}}%
\begin{sphinxVerbatim}[commandchars=\\\{\}]
\PYG{c+c1}{\PYGZsh{} set the mean duration to be 30 minutes}
\PYG{n}{commute\PYGZus{}to\PYGZus{}work\PYGZus{}dt\PYGZus{}mean}     \PYG{o}{=} \PYG{n}{np}\PYG{o}{.}\PYG{n}{array}\PYG{p}{(} \PYG{p}{[}\PYG{l+m+mi}{30}\PYG{p}{]} \PYG{p}{)}

\PYG{c+c1}{\PYGZsh{} set the standard deviation to be 10 minutes}
\PYG{n}{commute\PYGZus{}to\PYGZus{}work\PYGZus{}dt\PYGZus{}std}      \PYG{o}{=} \PYG{n}{np}\PYG{o}{.}\PYG{n}{array}\PYG{p}{(} \PYG{p}{[}\PYG{l+m+mi}{10}\PYG{p}{]} \PYG{p}{)}
\end{sphinxVerbatim}

In order to set the commute from work behavior, the user must set the mean and standard deviation for the duration
of the commute from work activity. The agent’s behavior for commuting from work is set as follows:

\fvset{hllines={, ,}}%
\begin{sphinxVerbatim}[commandchars=\\\{\}]
\PYG{c+c1}{\PYGZsh{} set the mean duration to be 30 minutes}
\PYG{n}{commute\PYGZus{}from\PYGZus{}work\PYGZus{}dt\PYGZus{}mean}     \PYG{o}{=} \PYG{n}{np}\PYG{o}{.}\PYG{n}{array}\PYG{p}{(} \PYG{p}{[}\PYG{l+m+mi}{30}\PYG{p}{]} \PYG{p}{)}

\PYG{c+c1}{\PYGZsh{} set the standard deviation to be 10 minutes}
\PYG{n}{commute\PYGZus{}from\PYGZus{}work\PYGZus{}dt\PYGZus{}std}      \PYG{o}{=} \PYG{n}{np}\PYG{o}{.}\PYG{n}{array}\PYG{p}{(} \PYG{p}{[}\PYG{l+m+mi}{10}\PYG{p}{]} \PYG{p}{)}
\end{sphinxVerbatim}


\subsection{Running the simulation}
\label{\detokenize{index:running-the-simulation}}
After defining all of the parameters in the file \sphinxcode{\sphinxupquote{\textbackslash{}run\textbackslash{}main\_params.py}}, the code is run by doing
the following:
\begin{enumerate}
\item {} 
go to the \sphinxcode{\sphinxupquote{\textbackslash{}run}} directory.

\item {} 
enter \sphinxcode{\sphinxupquote{python main.py}} into the terminal (or command line)

\item {} 
press enter (or return)

\end{enumerate}


\subsection{Interpreting the output}
\label{\detokenize{index:interpreting-the-output}}
ABMHAP outputs the record of the activities that the agent did during the simulation. This record is called an
\sphinxstylestrong{activity diary}. An activity diary is a chronological record contains the following information about each
activity: day, start time, end time, duration, and location.

Below is an example of the output of ABMHAP. Recall that ABMHAP saves the activity diary in terms of a .csv file


\begin{savenotes}\sphinxattablestart
\centering
\begin{tabulary}{\linewidth}[t]{|T|T|T|T|T|T|}
\hline
\sphinxstyletheadfamily 
day
&\sphinxstyletheadfamily 
start
&\sphinxstyletheadfamily 
end
&\sphinxstyletheadfamily 
dt
&\sphinxstyletheadfamily 
act
&\sphinxstyletheadfamily 
loc
\\
\hline
0
&
16
&
19
&
3
&
-1
&
0
\\
\hline
0
&
19
&
19.75
&
0.75
&
4
&
0
\\
\hline
0
&
19.75
&
22
&
2.25
&
-1
&
0
\\
\hline
0
&
22
&
8
&
10
&
6
&
0
\\
\hline
1
&
8
&
8.25
&
0.25
&
3
&
0
\\
\hline
1
&
8.25
&
8.5
&
0.25
&
-1
&
0
\\
\hline
1
&
8.5
&
9
&
0.5
&
2
&
1
\\
\hline
1
&
9
&
12
&
3
&
7
&
3
\\
\hline
1
&
12
&
12.5
&
0.5
&
5
&
3
\\
\hline
1
&
12.5
&
17
&
4.5
&
7
&
3
\\
\hline
1
&
17
&
17.5
&
0.5
&
1
&
1
\\
\hline
1
&
17.5
&
19
&
1.5
&
-1
&
0
\\
\hline
1
&
19
&
19.75
&
0.75
&
4
&
0
\\
\hline
1
&
19.75
&
22
&
2.25
&
-1
&
0
\\
\hline
1
&
22
&
8
&
10
&
6
&
0
\\
\hline
\end{tabulary}
\par
\sphinxattableend\end{savenotes}

where day, start, end, dt, act, and loc represent the day the activity starts, the start time of the
activity (in hours), the end time of the activity (in hours), the duration of the activity (in hours), the
activity identifier, and the location identifier, respectively. In the results, the time of day 16:30 is
represented as 16.5.

The following table is an interpretation of the example output shown above. In the table, the duration is
expressed in minutes.


\begin{savenotes}\sphinxattablestart
\centering
\begin{tabulary}{\linewidth}[t]{|T|T|T|T|T|T|}
\hline
\sphinxstyletheadfamily 
Day
&\sphinxstyletheadfamily 
Start
&\sphinxstyletheadfamily 
End
&\sphinxstyletheadfamily 
Duration
&\sphinxstyletheadfamily 
Activity Code
&\sphinxstyletheadfamily 
Location Code
\\
\hline
0
&
16:00
&
19:00
&
180
&
Idle
&
Home
\\
\hline
0
&
19:00
&
19:45
&
45
&
Eat dinner
&
Home
\\
\hline
0
&
19:45
&
22:00
&
135
&
Idle
&
Home
\\
\hline
0
&
22:00
&
8:00
&
600
&
Sleep
&
Home
\\
\hline
1
&
8:00
&
8:15
&
15
&
Eat breakfast
&
Home
\\
\hline
1
&
8:15
&
8:30
&
15
&
Idle
&
Home
\\
\hline
1
&
8:30
&
9:00
&
30
&
Commute to work
&
Out of doors
\\
\hline
1
&
9:00
&
12:00
&
180
&
Work
&
Workplace
\\
\hline
1
&
12:00
&
12:30
&
30
&
Eat lunch
&
Workplace
\\
\hline
1
&
12:30
&
17:00
&
270
&
Work
&
Workplace
\\
\hline
1
&
17:00
&
17:30
&
30
&
Commute from work
&
Out of doors
\\
\hline
1
&
17:30
&
19:00
&
90
&
Idle
&
Home
\\
\hline
1
&
19:00
&
19:45
&
45
&
Eat dinner
&
Home
\\
\hline
1
&
19.45
&
22:00
&
135
&
Idle
&
Home
\\
\hline
1
&
22:00
&
8:00
&
600
&
Sleep
&
Home
\\
\hline
\end{tabulary}
\par
\sphinxattableend\end{savenotes}


\section{How to Run the Code Using Parameters Derived from an Empirical Dataset}
\label{\detokenize{index:how-to-run-the-code-using-parameters-derived-from-an-empirical-dataset}}
ABMHAP may be also used to simulate agents whose parametrization is derived from an empirical dataset, as opposed
to using user-defined parameters. Specifically, the current version of ABMHAP uses the Consolidated Human
Activity Database (CHAD) to parametrize the behavior of agents. More information about CHAD may be found
\sphinxhref{https://www.epa.gov/healthresearch/consolidated-human-activity-database-chad-use-human-exposure-and-health-studies-and}{here}.
Currently, ABMAHP uses data from CHAD to parametrize agents that simulate the behaviors of people that
represent the following demographic groups within the general United States population:
\begin{enumerate}
\item {} 
working adults (ages 18 and above)

\item {} 
non-working adults (ages 18 and above)

\item {} 
school-age children (ages 5 through 17)

\item {} 
preschool children (ages 1 through 4)

\end{enumerate}

To simulate a demographic, we run ABMHAP via a Monte-Carlo simulation that creates \sphinxstylestrong{multiple households}
(ABMHAP currently simulates one agent per household) and
randomly parametrizes their behavior based on empirical distributions from CHAD data. In order to decrease
the runtime of simulating multiple households, ABMHAP has the capability to run the Monte-Carlo simulations
in \sphinxstylestrong{parallel}.

In order to run the Monte-Carlo simulations, the user must do the following:
\begin{enumerate}
\item {} 
set the input parameters of the simulation in the file \sphinxcode{\sphinxupquote{\textbackslash{}run\_chad\textbackslash{}driver\_params.py}}

\item {} 
run the executable using \sphinxcode{\sphinxupquote{\textbackslash{}run\_chad\textbackslash{}driver.py}}

\end{enumerate}


\subsection{Setting the input parameters}
\label{\detokenize{index:id4}}
In order to run ABMHAP, the user must set the following types of input parameters in
\sphinxcode{\sphinxupquote{\textbackslash{}run\_chad\textbackslash{}driver\_params.py}}:
\begin{enumerate}
\item {} 
input parameters that govern the general logistics of the simulation

\item {} 
input parameters that govern the the length of simulation duration

\item {} 
input parameters that define the demographic of the agents

\end{enumerate}

For illustrative purposes, what follows is a demonstration of how to parametrize a run for ABMHAP. See,
the earlier section “Setting the input parameters” in “How to Run the Code Using User-Defined Parameters” to
understand how to set up the input parameters:

\fvset{hllines={, ,}}%
\begin{sphinxVerbatim}[commandchars=\\\{\}]
\PYG{c+c1}{\PYGZsh{} the number of days for the simulation}
\PYG{n}{num\PYGZus{}days}    \PYG{o}{=} \PYG{l+m+mi}{364}

\PYG{c+c1}{\PYGZsh{} the number of additional hours}
\PYG{n}{num\PYGZus{}hours}   \PYG{o}{=} \PYG{l+m+mi}{8}

\PYG{c+c1}{\PYGZsh{} the number of additional minutes}
\PYG{n}{num\PYGZus{}min}     \PYG{o}{=} \PYG{l+m+mi}{0}

\PYG{c+c1}{\PYGZsh{} should the simulation plot results at the end of the run}
\PYG{n}{do\PYGZus{}plot}         \PYG{o}{=} \PYG{k+kc}{False}

\PYG{c+c1}{\PYGZsh{} should the simulation print messages to the screen}
\PYG{n}{do\PYGZus{}print}        \PYG{o}{=} \PYG{k+kc}{True}

\PYG{c+c1}{\PYGZsh{} should the simulation save the results (both input and output)}
\PYG{c+c1}{\PYGZsh{} of the simulation}
\PYG{n}{do\PYGZus{}save}         \PYG{o}{=} \PYG{k+kc}{False}
\end{sphinxVerbatim}

In addition, the user must define the demographic of the agents
being simulated. This causes ABMHAP to use the empirical data from the respective demographic in CHAD to
parametrize the agent:

\fvset{hllines={, ,}}%
\begin{sphinxVerbatim}[commandchars=\\\{\}]
\PYG{c+c1}{\PYGZsh{} this causes ABMHAP to use empirical data from CHAD corresponding}
\PYG{c+c1}{\PYGZsh{} to the working adult demographic in order to parametrize the agents}
\PYG{n}{demographic}   \PYG{o}{=} \PYG{n}{dmg}\PYG{o}{.}\PYG{n}{ADULT\PYGZus{}WORK}

\PYG{c+c1}{\PYGZsh{} the path to the output directory where should the output results}
\PYG{c+c1}{\PYGZsh{} should be saved}
\PYG{n}{fpath}         \PYG{o}{=} \PYG{l+s+s1}{\PYGZsq{}}\PYG{l+s+se}{\PYGZbs{}\PYGZbs{}}\PYG{l+s+s1}{output\PYGZus{}directory}\PYG{l+s+s1}{\PYGZsq{}}

\PYG{c+c1}{\PYGZsh{} load input data from a previous simulation}
\PYG{n}{do\PYGZus{}load\PYGZus{}trials}  \PYG{o}{=} \PYG{k+kc}{False}

\PYG{c+c1}{\PYGZsh{} the file name for \PYGZdq{}trials\PYGZdq{} data without the .pkl extension,}
\PYG{c+c1}{\PYGZsh{} which will be used for saving the trial information}
\PYG{n}{fname\PYGZus{}load\PYGZus{}trials\PYGZus{}base} \PYG{o}{=} \PYG{k+kc}{None}
\end{sphinxVerbatim}


\subsection{Running the simulation}
\label{\detokenize{index:id5}}
After defining all of the parameters in the file \sphinxcode{\sphinxupquote{\textbackslash{}run\_chad\textbackslash{}driver\_params.py}}, the code is run by doing
the following:
\begin{enumerate}
\item {} 
go to the \sphinxcode{\sphinxupquote{\textbackslash{}run\_chad}} directory.

\item {} 
enter the following into the terminal (or command line):
\begin{quote}

\sphinxcode{\sphinxupquote{python driver.py num\_process num\_hhld num\_batch}}

where
\begin{itemize}
\item {} 
\sphinxcode{\sphinxupquote{num\_process}} is the total number of cores (i.e, processing units) used in the simulation

\item {} 
\sphinxcode{\sphinxupquote{num\_hhld}} is the number of housholds to run per batch

\item {} 
\sphinxcode{\sphinxupquote{num\_batch}} is the number of batches used per core

\end{itemize}
\end{quote}

\item {} 
press enter (or return)

\end{enumerate}

To run the code, do the following.

\begin{sphinxadmonition}{note}{Note:}
This code may be run in \sphinxstylestrong{batches} in order to run many households while conserving memory. That is,
instead of running 32 households at once (and keeping 32 households in memory), the program can
run 2 batches of 16 households (for a total of 32 households). This halves the amount of memory
used in the simulation compared to running the simulation of 1 batch of 32 households. We
will shown how to run the code using “batches” below.
\end{sphinxadmonition}


\subsubsection{Running in \sphinxstylestrong{serial}}
\label{\detokenize{index:running-in-serial}}
The following are examples on how to run the code in serial.

To run in serial with with 64 households per batch, 1 batch (implied)
\begin{quote}

\textgreater{} \sphinxcode{\sphinxupquote{python driver.py 1 64 1}}

\textgreater{} \sphinxcode{\sphinxupquote{python driver.py 1 64}}
\end{quote}

To run in serial using 2 batches with 1 thread with 32 households per batch, 2 batches
\begin{quote}

\textgreater{} \sphinxcode{\sphinxupquote{python driver.py 1 32 2}}
\end{quote}


\subsubsection{Running in \sphinxstylestrong{parallel}}
\label{\detokenize{index:running-in-parallel}}
The following are examples on how to run the code in parallel.

To run in parallel using 4 cores with 64 households total (16 household per core per batch), 1 batch (implied)
\begin{quote}

\textgreater{} \sphinxcode{\sphinxupquote{python driver.py 4 64 1}}

\textgreater{} \sphinxcode{\sphinxupquote{python driver.py 4 64}}
\end{quote}

To run in parallel using 4 cores with 32 households per batch, 2 batches(8 households per core per batch)
\begin{quote}

\textgreater{} \sphinxcode{\sphinxupquote{python driver.py 4 32 2}}
\end{quote}


\subsection{Interpreting the output}
\label{\detokenize{index:id6}}
ABMHAP outputs the record of the activities that \sphinxstylestrong{each} agent did during the simulation. Each agent’s
record is called an \sphinxstylestrong{activity diary}. An activity diary is a chronological record contains the following
information about each activity: day, start time, end time, duration, and location. The output is a combined
record of every agent simulated where each agent is given a unique integer as an identifier.

Below is an example of the output of ABMHAP. Recall that ABMHAP saves the activity diary in terms of a .csv file


\begin{savenotes}\sphinxattablestart
\centering
\begin{tabulary}{\linewidth}[t]{|T|T|T|T|T|T|T|}
\hline
\sphinxstyletheadfamily 
id
&\sphinxstyletheadfamily 
day
&\sphinxstyletheadfamily 
start
&\sphinxstyletheadfamily 
end
&\sphinxstyletheadfamily 
dt
&\sphinxstyletheadfamily 
act
&\sphinxstyletheadfamily 
loc
\\
\hline
0
&
0
&
16
&
19
&
3
&
-1
&
0
\\
\hline
0
&
0
&
19
&
19.75
&
0.75
&
4
&
0
\\
\hline
\(\vdots\)
&&&&&&\\
\hline
0
&
364
&
22
&
0
&
2
&
6
&
0
\\
\hline
1
&
0
&
16
&
20
&
4
&
-1
&
0
\\
\hline
1
&
0
&
20
&
20.5
&
0.5
&
4
&
0
\\
\hline
\(\vdots\)
&&&&&&\\
\hline
1
&
364
&
23
&
0
&
1
&
6
&
0
\\
\hline
\end{tabulary}
\par
\sphinxattableend\end{savenotes}

where id, day, start, end, dt, act, and loc represent the agent identifier, the day
the activity starts, the start time of the activity (in hours), the end time of the
activity (in hours), the duration of the activity (in hours), the activity identifier,
and the location identifier, respectively. In the results, the time of day 16:30 is
represented as 16.5. Each agent in the Mont-Carlo simulation is given a unique
identifier via “id”.

The following table is an interpretation of the example output shown above. In the table, the duration is
expressed in minutes.


\begin{savenotes}\sphinxattablestart
\centering
\begin{tabulary}{\linewidth}[t]{|T|T|T|T|T|T|T|}
\hline
\sphinxstyletheadfamily 
Identifier
&\sphinxstyletheadfamily 
Day
&\sphinxstyletheadfamily 
Start
&\sphinxstyletheadfamily 
End
&\sphinxstyletheadfamily 
Duration
&\sphinxstyletheadfamily 
Activity Code
&\sphinxstyletheadfamily 
Location Code
\\
\hline
0
&
0
&
16:00
&
19:00
&
180
&
Idle
&
Home
\\
\hline
0
&
0
&
19:00
&
19:45
&
45
&
Eat dinner
&
Home
\\
\hline
\(\vdots\)
&&&&&&\\
\hline
0
&
364
&
22:00
&
00:00
&
120
&
Sleep
&
Home
\\
\hline
1
&
0
&
16:00
&
20:00
&
240
&
Idle
&
Home
\\
\hline
1
&
0
&
20:00
&
20:30
&
30
&
Eat dinner
&
Home
\\
\hline
\(\vdots\)
&&&&&&\\
\hline
1
&
364
&
23:00
&
00:00
&
60
&
Sleep
&
Home
\\
\hline
\end{tabulary}
\par
\sphinxattableend\end{savenotes}

Given that ABMHAP is set to simulate working adults, the results from a ABMHAP
simulation are saved in the following files:
\begin{enumerate}
\item {} 
\sphinxcode{\sphinxupquote{\textbackslash{}output\_directory\textbackslash{}data\_adult\_work.csv}}

\item {} 
\sphinxcode{\sphinxupquote{\textbackslash{}output\_directory\textbackslash{}trials\_adult\_work.pkl}}

\item {} 
\sphinxcode{\sphinxupquote{\textbackslash{}output\_directory\textbackslash{}data\_adult\_work.pkl}}

\end{enumerate}

The \sphinxcode{\sphinxupquote{.csv}} file contains the activity diary of all of the simulations. The other files are
created for the following reason. Because these runs are capable of simulating large numbers
of agents, in order to save memory space, the results from ABMHAP are saved in a compressed
file format unique to python called “pickle” file format, which is denoted with a
\sphinxcode{\sphinxupquote{.pkl}} extension. The saved python objects correspond to
\begin{enumerate}
\item {} 
\sphinxcode{\sphinxupquote{\textbackslash{}output\_directory\textbackslash{}trials\_adult\_work.pkl}} contains the input data for each household being
simulated. The file contains a list of {\hyperref[\detokenize{trial:trial.Trial}]{\sphinxcrossref{\sphinxcode{\sphinxupquote{trial.Trial}}}}} objects.

\item {} 
\sphinxcode{\sphinxupquote{\textbackslash{}output\_directory\textbackslash{}data\_adult\_work.pkl}}  contains the output data (i.e., the activity
diaries) for each household being simulated. The file contains a {\hyperref[\detokenize{driver_result:driver_result.Driver_Result}]{\sphinxcrossref{\sphinxcode{\sphinxupquote{driver\_result.Driver\_Result}}}}}
object

\end{enumerate}


\chapter{Code Documentation}
\label{\detokenize{index:code-documentation}}
The following is the documentation of all of the files that are within the ABMHAP code. The
code is divided into the following directories:
\begin{itemize}
\item {} 
\sphinxcode{\sphinxupquote{\textbackslash{}source}}, which handles the key components for ABMHAP

\item {} 
\sphinxcode{\sphinxupquote{\textbackslash{}run}}, which handles code for running ABMHAP with user-defined parameters

\item {} 
\sphinxcode{\sphinxupquote{\textbackslash{}run\_chad}}, which handles running ABMHAP as a Monte-Carlo simulation with agents parameterized with empirical data from CHAD

\item {} 
\sphinxcode{\sphinxupquote{\textbackslash{}plotting}}, which handles some plotting capability

\item {} 
\sphinxcode{\sphinxupquote{\textbackslash{}processing}}, which handles parses the data from CHAD

\end{itemize}


\section{Source Directory}
\label{\detokenize{index:source-directory}}
These files are the key modules that are used to create the ABMHAP algorithm.

Contents:


\subsection{activity module}
\label{\detokenize{activity::doc}}\label{\detokenize{activity:module-activity}}\label{\detokenize{activity:activity-module}}\index{activity (module)}
This module contains code that governs the activities that the agent performs
in order to satisfy its needs.

This module contains the following class: {\hyperref[\detokenize{activity:activity.Activity}]{\sphinxcrossref{\sphinxcode{\sphinxupquote{activity.Activity}}}}}.
\index{Activity (class in activity)}

\begin{fulllineitems}
\phantomsection\label{\detokenize{activity:activity.Activity}}\pysigline{\sphinxbfcode{\sphinxupquote{class }}\sphinxcode{\sphinxupquote{activity.}}\sphinxbfcode{\sphinxupquote{Activity}}}
Bases: \sphinxcode{\sphinxupquote{object}}

An activity enables a {\hyperref[\detokenize{person:person.Person}]{\sphinxcrossref{\sphinxcode{\sphinxupquote{person.Person}}}}} to address its satiation
\(n(t)\). This class’s purpose is to encapsulate general
capabilities that specific instances of activity will derive from.
\begin{quote}\begin{description}
\item[{Variables}] \leavevmode\begin{itemize}
\item {} 
\sphinxstyleliteralstrong{\sphinxupquote{category}} (\sphinxstyleliteralemphasis{\sphinxupquote{int}}) \textendash{} an unique identifier naming the type of activity.

\item {} 
\sphinxstyleliteralstrong{\sphinxupquote{t\_end}} (\sphinxstyleliteralemphasis{\sphinxupquote{int}}) \textendash{} the end time of the activity {[}universal time, seconds{]}

\item {} 
\sphinxstyleliteralstrong{\sphinxupquote{t\_start}} (\sphinxstyleliteralemphasis{\sphinxupquote{int}}) \textendash{} the start time of the activity {[}universal time, seconds{]}

\item {} 
\sphinxstyleliteralstrong{\sphinxupquote{dt}} (\sphinxstyleliteralemphasis{\sphinxupquote{int}}) \textendash{} the duration of the activity {[}seconds{]}

\end{itemize}

\end{description}\end{quote}
\index{advertise() (activity.Activity method)}

\begin{fulllineitems}
\phantomsection\label{\detokenize{activity:activity.Activity.advertise}}\pysiglinewithargsret{\sphinxbfcode{\sphinxupquote{advertise}}}{\emph{the\_need}, \emph{dt}}{}
Calculates the advertised score of doing an activity. Let
\(\Omega\) be the set of all needs addressed by the activity.
The score \(S\) is calculated by doing the following
\begin{equation*}
\begin{split}S = \begin{cases}
    0  & n(t) > \lambda \\
    \sum_{j \in \Omega} W_j( n_j(t) ) - W_j( n_j(t + \Delta{t} )) & n(t) \le \lambda
\end{cases}\end{split}
\end{equation*}\begin{description}
\item[{where}] \leavevmode\begin{itemize}
\item {} 
\(t\) is the current time

\item {} 
\(\Delta{t}\) is the duration of the activity

\item {} 
\(n(t)\) is the satiation at time \(t\)

\item {} 
\(\lambda\) is the threshold value of the need

\item {} 
\(W(n)\) is the weight function for the particular need

\end{itemize}

\end{description}
\begin{quote}\begin{description}
\item[{Parameters}] \leavevmode\begin{itemize}
\item {} 
\sphinxstyleliteralstrong{\sphinxupquote{the\_need}} ({\hyperref[\detokenize{need:need.Need}]{\sphinxcrossref{\sphinxstyleliteralemphasis{\sphinxupquote{need.Need}}}}}) \textendash{} the primary need associated with the respective activity

\item {} 
\sphinxstyleliteralstrong{\sphinxupquote{dt}} (\sphinxstyleliteralemphasis{\sphinxupquote{int}}) \textendash{} the duration \(\Delta{t}\) of doing the activity {[}minutes{]}

\end{itemize}

\item[{Returns score}] \leavevmode
the score of the advertisement

\item[{Return type}] \leavevmode
float

\end{description}\end{quote}

\end{fulllineitems}

\index{advertise\_interruption() (activity.Activity method)}

\begin{fulllineitems}
\phantomsection\label{\detokenize{activity:activity.Activity.advertise_interruption}}\pysiglinewithargsret{\sphinxbfcode{\sphinxupquote{advertise\_interruption}}}{}{}
Advertise the score if this activity interrupts another activity.

\begin{sphinxadmonition}{note}{Note:}
This function should be overloaded in derived classes.
\end{sphinxadmonition}
\begin{quote}\begin{description}
\item[{Returns score}] \leavevmode
the advertised score

\item[{Return type}] \leavevmode
float

\end{description}\end{quote}

\end{fulllineitems}

\index{end() (activity.Activity method)}

\begin{fulllineitems}
\phantomsection\label{\detokenize{activity:activity.Activity.end}}\pysiglinewithargsret{\sphinxbfcode{\sphinxupquote{end}}}{\emph{p}}{}
This function handles some of the common logistics in ending a specific activity assuming         the activity ends without an interruption.

Currently the function does the following:
\begin{enumerate}
\item {} 
reset the {\hyperref[\detokenize{state:state.State}]{\sphinxcrossref{\sphinxcode{\sphinxupquote{state.State}}}}} variable’s start time to the current time

\item {} 
reset the {\hyperref[\detokenize{state:state.State}]{\sphinxcrossref{\sphinxcode{\sphinxupquote{state.State}}}}} variable’s end time to the current time

\end{enumerate}
\begin{quote}\begin{description}
\item[{Parameters}] \leavevmode
\sphinxstyleliteralstrong{\sphinxupquote{p}} ({\hyperref[\detokenize{person:person.Person}]{\sphinxcrossref{\sphinxstyleliteralemphasis{\sphinxupquote{person.Person}}}}}) \textendash{} the person whose activity is ending

\item[{Returns}] \leavevmode
None

\end{description}\end{quote}

\end{fulllineitems}

\index{halt() (activity.Activity method)}

\begin{fulllineitems}
\phantomsection\label{\detokenize{activity:activity.Activity.halt}}\pysiglinewithargsret{\sphinxbfcode{\sphinxupquote{halt}}}{\emph{p}}{}
This function handles some of the common logistics in ending a specific activity due to an         interruption.

Currently the function does the following:
\begin{enumerate}
\item {} 
reset the {\hyperref[\detokenize{state:state.State}]{\sphinxcrossref{\sphinxcode{\sphinxupquote{state.State}}}}} variable’s start time to the current time

\item {} 
reset the {\hyperref[\detokenize{state:state.State}]{\sphinxcrossref{\sphinxcode{\sphinxupquote{state.State}}}}} variable’s end time to the current time

\end{enumerate}
\begin{quote}\begin{description}
\item[{Parameters}] \leavevmode
\sphinxstyleliteralstrong{\sphinxupquote{p}} ({\hyperref[\detokenize{person:person.Person}]{\sphinxcrossref{\sphinxstyleliteralemphasis{\sphinxupquote{person.Person}}}}}) \textendash{} the person whose activity is being interrupted

\item[{Returns}] \leavevmode
None

\end{description}\end{quote}

\end{fulllineitems}

\index{print\_id() (activity.Activity method)}

\begin{fulllineitems}
\phantomsection\label{\detokenize{activity:activity.Activity.print_id}}\pysiglinewithargsret{\sphinxbfcode{\sphinxupquote{print\_id}}}{}{}
This function represents the activity category as a string.
\begin{quote}\begin{description}
\item[{Return msg}] \leavevmode
The string representation of the category

\item[{Return type}] \leavevmode
str

\end{description}\end{quote}

\end{fulllineitems}

\index{start() (activity.Activity method)}

\begin{fulllineitems}
\phantomsection\label{\detokenize{activity:activity.Activity.start}}\pysiglinewithargsret{\sphinxbfcode{\sphinxupquote{start}}}{}{}
This function starts a specific activity.

\begin{sphinxadmonition}{note}{Note:}
This function is meant to be overloaded by derived activity classes.
\end{sphinxadmonition}
\begin{quote}\begin{description}
\item[{Returns}] \leavevmode
None

\end{description}\end{quote}

\end{fulllineitems}

\index{toString() (activity.Activity method)}

\begin{fulllineitems}
\phantomsection\label{\detokenize{activity:activity.Activity.toString}}\pysiglinewithargsret{\sphinxbfcode{\sphinxupquote{toString}}}{}{}
This function represents the Activity object as a string.
\begin{quote}\begin{description}
\item[{Return msg}] \leavevmode
the string representation of the activity object

\item[{Return type}] \leavevmode
str

\end{description}\end{quote}

\end{fulllineitems}


\end{fulllineitems}



\subsection{asset module}
\label{\detokenize{asset::doc}}\label{\detokenize{asset:asset-module}}\label{\detokenize{asset:module-asset}}\index{asset (module)}
This module contains code that governs objects that enable access to activities ({\hyperref[\detokenize{activity:activity.Activity}]{\sphinxcrossref{\sphinxcode{\sphinxupquote{activity.Activity}}}}}) that an agent may use in order to address a need.

This module contains the following class: {\hyperref[\detokenize{asset:asset.Asset}]{\sphinxcrossref{\sphinxcode{\sphinxupquote{asset.Asset}}}}}.
\index{Asset (class in asset)}

\begin{fulllineitems}
\phantomsection\label{\detokenize{asset:asset.Asset}}\pysigline{\sphinxbfcode{\sphinxupquote{class }}\sphinxcode{\sphinxupquote{asset.}}\sphinxbfcode{\sphinxupquote{Asset}}}
Bases: \sphinxcode{\sphinxupquote{object}}

An asset is an object that allows the agent to perform an activity. Each asset     contains a list of activities that an agent can use to perform actions.
\begin{quote}\begin{description}
\item[{Variables}] \leavevmode\begin{itemize}
\item {} 
\sphinxstyleliteralstrong{\sphinxupquote{activities}} (\sphinxstyleliteralemphasis{\sphinxupquote{dict}}) \textendash{} a dictionary of all the activities associated with this asset

\item {} 
\sphinxstyleliteralstrong{\sphinxupquote{category}} (\sphinxstyleliteralemphasis{\sphinxupquote{int}}) \textendash{} a code that indicates the category type of asset

\item {} 
\sphinxstyleliteralstrong{\sphinxupquote{id}} (\sphinxstyleliteralemphasis{\sphinxupquote{int}}) \textendash{} an identifier number for the asset

\item {} 
\sphinxstyleliteralstrong{\sphinxupquote{'location'}} ({\hyperref[\detokenize{location:location.Location}]{\sphinxcrossref{\sphinxstyleliteralemphasis{\sphinxupquote{location.Location}}}}}) \textendash{} the location of the asset

\item {} 
\sphinxstyleliteralstrong{\sphinxupquote{max\_users}} (\sphinxstyleliteralemphasis{\sphinxupquote{int}}) \textendash{} the maximum number of users that can simultaneously access the asset

\item {} 
\sphinxstyleliteralstrong{\sphinxupquote{num\_users}} (\sphinxstyleliteralemphasis{\sphinxupquote{int}}) \textendash{} the current number of users for the asset

\item {} 
\sphinxstyleliteralstrong{\sphinxupquote{status}} (\sphinxstyleliteralemphasis{\sphinxupquote{int}}) \textendash{} the state of the asset

\end{itemize}

\end{description}\end{quote}
\index{free() (asset.Asset method)}

\begin{fulllineitems}
\phantomsection\label{\detokenize{asset:asset.Asset.free}}\pysiglinewithargsret{\sphinxbfcode{\sphinxupquote{free}}}{}{}
This function changes the state of an asset once it is freed by a Person by doing the following:
\begin{enumerate}
\item {} 
decreases the number of users of the asset by 1

\item {} 
if the number of users is zero, the status of the asset is set to idle (\sphinxcode{\sphinxupquote{state.IDLE}})

\end{enumerate}
\begin{quote}\begin{description}
\item[{Returns}] \leavevmode
None

\end{description}\end{quote}

\end{fulllineitems}

\index{initialize() (asset.Asset method)}

\begin{fulllineitems}
\phantomsection\label{\detokenize{asset:asset.Asset.initialize}}\pysiglinewithargsret{\sphinxbfcode{\sphinxupquote{initialize}}}{\emph{people}}{}
This function initializes the asset at the beginning of the simulation.

\begin{sphinxadmonition}{note}{Note:}
This function is meant to be overridden
\end{sphinxadmonition}
\begin{quote}\begin{description}
\item[{Parameters}] \leavevmode
\sphinxstyleliteralstrong{\sphinxupquote{people}} (\sphinxstyleliteralemphasis{\sphinxupquote{list}}\sphinxstyleliteralemphasis{\sphinxupquote{{[} }}{\hyperref[\detokenize{person:person.Person}]{\sphinxcrossref{\sphinxstyleliteralemphasis{\sphinxupquote{person.Person}}}}}\sphinxstyleliteralemphasis{\sphinxupquote{ {]}}}) \textendash{} the Person objects who could be using the asset.

\item[{Returns}] \leavevmode
None

\end{description}\end{quote}

\end{fulllineitems}

\index{print\_category() (asset.Asset method)}

\begin{fulllineitems}
\phantomsection\label{\detokenize{asset:asset.Asset.print_category}}\pysiglinewithargsret{\sphinxbfcode{\sphinxupquote{print\_category}}}{}{}~\begin{quote}

This function represents the category as a string.
\end{quote}
\begin{quote}\begin{description}
\item[{Returns}] \leavevmode
the string representation of the category

\item[{Return type}] \leavevmode
str

\end{description}\end{quote}

\end{fulllineitems}

\index{reset() (asset.Asset method)}

\begin{fulllineitems}
\phantomsection\label{\detokenize{asset:asset.Asset.reset}}\pysiglinewithargsret{\sphinxbfcode{\sphinxupquote{reset}}}{}{}
This function does the following:
\begin{enumerate}
\item {} 
sets the number of users to zero

\item {} 
sets the asset’s status to idle

\end{enumerate}
\begin{quote}\begin{description}
\item[{Returns}] \leavevmode
None

\end{description}\end{quote}

\end{fulllineitems}

\index{toString() (asset.Asset method)}

\begin{fulllineitems}
\phantomsection\label{\detokenize{asset:asset.Asset.toString}}\pysiglinewithargsret{\sphinxbfcode{\sphinxupquote{toString}}}{}{}~\begin{quote}

This function represents the asset as a string.
\end{quote}
\begin{quote}\begin{description}
\item[{Return msg}] \leavevmode
The string representation of the asset object.

\item[{Return type}] \leavevmode
str

\end{description}\end{quote}

\end{fulllineitems}

\index{update() (asset.Asset method)}

\begin{fulllineitems}
\phantomsection\label{\detokenize{asset:asset.Asset.update}}\pysiglinewithargsret{\sphinxbfcode{\sphinxupquote{update}}}{}{}
This function changes the state of the asset once it is used by a person. The update does         the following:
\begin{enumerate}
\item {} 
increases the number of people by 1

\item {} 
if the number of users is at the maximum number, set the asset’s status to busy

\item {} 
if the number of users is less than the maximum number, set the asset’s status to busy but         able to be used by another agent

\end{enumerate}
\begin{quote}\begin{description}
\item[{Returns}] \leavevmode
None

\end{description}\end{quote}

\end{fulllineitems}


\end{fulllineitems}



\subsection{bed module}
\label{\detokenize{bed::doc}}\label{\detokenize{bed:module-bed}}\label{\detokenize{bed:bed-module}}\index{bed (module)}
This module contains code that enables the agent to use a bed. This class allows access to the sleep ({\hyperref[\detokenize{sleep:sleep.Sleep}]{\sphinxcrossref{\sphinxcode{\sphinxupquote{sleep.Sleep}}}}}) activity.

This module contains the following class: {\hyperref[\detokenize{bed:bed.Bed}]{\sphinxcrossref{\sphinxcode{\sphinxupquote{bed.Bed}}}}}.
\index{Bed (class in bed)}

\begin{fulllineitems}
\phantomsection\label{\detokenize{bed:bed.Bed}}\pysigline{\sphinxbfcode{\sphinxupquote{class }}\sphinxcode{\sphinxupquote{bed.}}\sphinxbfcode{\sphinxupquote{Bed}}}
Bases: {\hyperref[\detokenize{asset:asset.Asset}]{\sphinxcrossref{\sphinxcode{\sphinxupquote{asset.Asset}}}}}

This asset models a bed. It allows the agent to address the Rest ({\hyperref[\detokenize{rest:rest.Rest}]{\sphinxcrossref{\sphinxcode{\sphinxupquote{rest.Rest}}}}}) need by doing the     sleep ({\hyperref[\detokenize{sleep:sleep.Sleep}]{\sphinxcrossref{\sphinxcode{\sphinxupquote{sleep.Sleep}}}}}) activity.

\end{fulllineitems}



\subsection{bio module}
\label{\detokenize{bio::doc}}\label{\detokenize{bio:module-bio}}\label{\detokenize{bio:bio-module}}\index{bio (module)}
This module contains information about a Person’s ({\hyperref[\detokenize{person:person.Person}]{\sphinxcrossref{\sphinxcode{\sphinxupquote{person.Person}}}}}) biology.

This module contains the following class: {\hyperref[\detokenize{bio:bio.Bio}]{\sphinxcrossref{\sphinxcode{\sphinxupquote{bio.Bio}}}}}.
\index{Bio (class in bio)}

\begin{fulllineitems}
\phantomsection\label{\detokenize{bio:bio.Bio}}\pysigline{\sphinxbfcode{\sphinxupquote{class }}\sphinxcode{\sphinxupquote{bio.}}\sphinxbfcode{\sphinxupquote{Bio}}}
Bases: \sphinxcode{\sphinxupquote{object}}

This class holds the biologically relevant information for a person. This information is:
\begin{itemize}
\item {} 
age

\item {} 
gender

\item {} 
mean / standard deviation of start time for sleeping

\item {} 
mean / standard deviation of end time for sleeping

\item {} 
probability distribution function sleep start time / end time

\end{itemize}
\begin{quote}\begin{description}
\item[{Variables}] \leavevmode\begin{itemize}
\item {} 
\sphinxstyleliteralstrong{\sphinxupquote{age}} (\sphinxstyleliteralemphasis{\sphinxupquote{int}}) \textendash{} the age {[}years{]}

\item {} 
\sphinxstyleliteralstrong{\sphinxupquote{gender}} (\sphinxstyleliteralemphasis{\sphinxupquote{int}}) \textendash{} the gender

\item {} 
\sphinxstyleliteralstrong{\sphinxupquote{sleep\_dt}} (\sphinxstyleliteralemphasis{\sphinxupquote{int}}) \textendash{} the duration of time for a sleep event {[}minutes{]}

\item {} 
\sphinxstyleliteralstrong{\sphinxupquote{sleep\_start\_mean}} (\sphinxstyleliteralemphasis{\sphinxupquote{int}}) \textendash{} the mean start time for a sleep event {[}minutes{]}

\item {} 
\sphinxstyleliteralstrong{\sphinxupquote{sleep\_start\_std}} (\sphinxstyleliteralemphasis{\sphinxupquote{int}}) \textendash{} the standard deviation for start time for a sleep event {[}minutes{]}

\item {} 
\sphinxstyleliteralstrong{\sphinxupquote{sleep\_start}} (\sphinxstyleliteralemphasis{\sphinxupquote{int}}) \textendash{} the start time for sleep {[}minutes, time of day{]}

\item {} 
\sphinxstyleliteralstrong{\sphinxupquote{sleep\_start\_univ}} (\sphinxstyleliteralemphasis{\sphinxupquote{int}}) \textendash{} the start time for sleep{[}minutes, universal time{]}

\item {} 
\sphinxstyleliteralstrong{\sphinxupquote{sleep\_end\_mean}} (\sphinxstyleliteralemphasis{\sphinxupquote{int}}) \textendash{} the mean end time for a sleep event {[}minutes{]}

\item {} 
\sphinxstyleliteralstrong{\sphinxupquote{sleep\_end\_std}} (\sphinxstyleliteralemphasis{\sphinxupquote{int}}) \textendash{} the standard deviation for end time for a sleep event {[}minutes{]}

\item {} 
\sphinxstyleliteralstrong{\sphinxupquote{sleep\_end}} (\sphinxstyleliteralemphasis{\sphinxupquote{int}}) \textendash{} the end time for sleep{[}minutes, time of day{]}

\item {} 
\sphinxstyleliteralstrong{\sphinxupquote{sleep\_end\_univ}} (\sphinxstyleliteralemphasis{\sphinxupquote{int}}) \textendash{} the end time for sleep {[}minutes, universal time{]}

\item {} 
\sphinxstyleliteralstrong{\sphinxupquote{start\_trunc}} (\sphinxstyleliteralemphasis{\sphinxupquote{int}}) \textendash{} the number of standard deviations to allow when sampling sleep the     truncated distribution for start time

\item {} 
\sphinxstyleliteralstrong{\sphinxupquote{end\_trunc}} (\sphinxstyleliteralemphasis{\sphinxupquote{int}}) \textendash{} the number of standard deviations to allow when sampling sleep the     truncated distribution for end time

\item {} 
\sphinxstyleliteralstrong{\sphinxupquote{f\_sleep\_start}} (\sphinxstyleliteralemphasis{\sphinxupquote{func}}) \textendash{} the distribution data for start time for sleep

\item {} 
\sphinxstyleliteralstrong{\sphinxupquote{f\_sleep\_end}} (\sphinxstyleliteralemphasis{\sphinxupquote{func}}) \textendash{} the distribution data for end time for sleep

\end{itemize}

\end{description}\end{quote}
\index{calc\_awake\_duration() (bio.Bio method)}

\begin{fulllineitems}
\phantomsection\label{\detokenize{bio:bio.Bio.calc_awake_duration}}\pysiglinewithargsret{\sphinxbfcode{\sphinxupquote{calc\_awake\_duration}}}{\emph{t}}{}
This function calculates the amount of time the person is expected to be awake.
\begin{quote}\begin{description}
\item[{Parameters}] \leavevmode
\sphinxstyleliteralstrong{\sphinxupquote{t}} (\sphinxstyleliteralemphasis{\sphinxupquote{int}}) \textendash{} time of day {[}minutes{]}

\item[{Returns}] \leavevmode
the duration {[}minutes{]} until the agent is expected to awaken

\end{description}\end{quote}

\end{fulllineitems}

\index{print\_gender() (bio.Bio method)}

\begin{fulllineitems}
\phantomsection\label{\detokenize{bio:bio.Bio.print_gender}}\pysiglinewithargsret{\sphinxbfcode{\sphinxupquote{print\_gender}}}{}{}
This function returns a string representation of gender
\begin{quote}\begin{description}
\item[{Returns}] \leavevmode
the string representation of gender

\item[{Return type}] \leavevmode
str

\end{description}\end{quote}

\end{fulllineitems}

\index{set\_sleep\_params() (bio.Bio method)}

\begin{fulllineitems}
\phantomsection\label{\detokenize{bio:bio.Bio.set_sleep_params}}\pysiglinewithargsret{\sphinxbfcode{\sphinxupquote{set\_sleep\_params}}}{\emph{start\_mean}, \emph{start\_std}, \emph{end\_mean}, \emph{end\_std}}{}
This function sets the biological sleep parameters themselves and the sleep parameter distribution functions.
\begin{quote}\begin{description}
\item[{Parameters}] \leavevmode\begin{itemize}
\item {} 
\sphinxstyleliteralstrong{\sphinxupquote{start\_mean}} (\sphinxstyleliteralemphasis{\sphinxupquote{int}}) \textendash{} the mean sleep start time {[}minutes{]}

\item {} 
\sphinxstyleliteralstrong{\sphinxupquote{start\_std}} (\sphinxstyleliteralemphasis{\sphinxupquote{int}}) \textendash{} the standard deviation of start time {[}minutes{]}

\item {} 
\sphinxstyleliteralstrong{\sphinxupquote{end\_mean}} (\sphinxstyleliteralemphasis{\sphinxupquote{int}}) \textendash{} the mean sleep end time {[}minutes{]}

\item {} 
\sphinxstyleliteralstrong{\sphinxupquote{end\_std}} (\sphinxstyleliteralemphasis{\sphinxupquote{int}}) \textendash{} the standard deviation of end time {[}minutes{]}

\end{itemize}

\item[{Returns}] \leavevmode
None

\end{description}\end{quote}

\end{fulllineitems}

\index{toString() (bio.Bio method)}

\begin{fulllineitems}
\phantomsection\label{\detokenize{bio:bio.Bio.toString}}\pysiglinewithargsret{\sphinxbfcode{\sphinxupquote{toString}}}{\emph{do\_decimal=False}}{}
This function represents the Bio object as a string.
\begin{quote}\begin{description}
\item[{Parameters}] \leavevmode
\sphinxstyleliteralstrong{\sphinxupquote{do\_decimal}} (\sphinxstyleliteralemphasis{\sphinxupquote{bool}}) \textendash{} This controls whether or not to represent the values in time in a                                 decimal (hours) format where {[}1:30pm is 13.5{]} if True or as the minutes                                 in the day if False {[}1:30pm is 13 * 60 + 30{]}.

\item[{Return msg}] \leavevmode
the string representation of the Bio object

\item[{Return type}] \leavevmode
string

\end{description}\end{quote}

\end{fulllineitems}

\index{update\_sleep\_dt() (bio.Bio method)}

\begin{fulllineitems}
\phantomsection\label{\detokenize{bio:bio.Bio.update_sleep_dt}}\pysiglinewithargsret{\sphinxbfcode{\sphinxupquote{update\_sleep\_dt}}}{}{}
This function sets the duration of sleep.
\begin{quote}\begin{description}
\item[{Returns}] \leavevmode
None

\end{description}\end{quote}

\end{fulllineitems}

\index{update\_sleep\_end() (bio.Bio method)}

\begin{fulllineitems}
\phantomsection\label{\detokenize{bio:bio.Bio.update_sleep_end}}\pysiglinewithargsret{\sphinxbfcode{\sphinxupquote{update\_sleep\_end}}}{}{}
This function samples the sleep end time distribution and sets the end time.
\begin{quote}\begin{description}
\item[{Returns}] \leavevmode
None

\end{description}\end{quote}

\end{fulllineitems}

\index{update\_sleep\_end\_univ() (bio.Bio method)}

\begin{fulllineitems}
\phantomsection\label{\detokenize{bio:bio.Bio.update_sleep_end_univ}}\pysiglinewithargsret{\sphinxbfcode{\sphinxupquote{update\_sleep\_end\_univ}}}{\emph{time\_of\_day}, \emph{t\_univ}}{}
This function sets the end time for sleep in terms of universal time.
\begin{quote}\begin{description}
\item[{Parameters}] \leavevmode\begin{itemize}
\item {} 
\sphinxstyleliteralstrong{\sphinxupquote{time\_of\_day}} (\sphinxstyleliteralemphasis{\sphinxupquote{int}}) \textendash{} the current time of day {[}minutes{]}

\item {} 
\sphinxstyleliteralstrong{\sphinxupquote{t\_univ}} (\sphinxstyleliteralemphasis{\sphinxupquote{int}}) \textendash{} the universal time {[}minutes{]}

\end{itemize}

\item[{Returns}] \leavevmode
None

\end{description}\end{quote}

\end{fulllineitems}

\index{update\_sleep\_start() (bio.Bio method)}

\begin{fulllineitems}
\phantomsection\label{\detokenize{bio:bio.Bio.update_sleep_start}}\pysiglinewithargsret{\sphinxbfcode{\sphinxupquote{update\_sleep\_start}}}{}{}
This function samples the sleep start time distribution and sets the start time.
\begin{quote}\begin{description}
\item[{Returns}] \leavevmode
None

\end{description}\end{quote}

\end{fulllineitems}

\index{update\_sleep\_start\_univ() (bio.Bio method)}

\begin{fulllineitems}
\phantomsection\label{\detokenize{bio:bio.Bio.update_sleep_start_univ}}\pysiglinewithargsret{\sphinxbfcode{\sphinxupquote{update\_sleep\_start\_univ}}}{\emph{time\_of\_day}, \emph{t\_univ}}{}
This function sets the start time for sleep in terms of universal time.
\begin{quote}\begin{description}
\item[{Parameters}] \leavevmode\begin{itemize}
\item {} 
\sphinxstyleliteralstrong{\sphinxupquote{time\_of\_day}} (\sphinxstyleliteralemphasis{\sphinxupquote{int}}) \textendash{} the current time of day {[}minutes{]}

\item {} 
\sphinxstyleliteralstrong{\sphinxupquote{t\_univ}} (\sphinxstyleliteralemphasis{\sphinxupquote{int}}) \textendash{} the universal time {[}minutes{]}

\end{itemize}

\item[{Returns}] \leavevmode
None

\end{description}\end{quote}

\end{fulllineitems}

\index{update\_time\_univ() (bio.Bio method)}

\begin{fulllineitems}
\phantomsection\label{\detokenize{bio:bio.Bio.update_time_univ}}\pysiglinewithargsret{\sphinxbfcode{\sphinxupquote{update\_time\_univ}}}{\emph{time\_of\_day}, \emph{t\_univ}, \emph{t}}{}
This function updates a time \(t\), which represents sleep start time or end time, to be in the         next occurrence
\begin{quote}\begin{description}
\item[{Parameters}] \leavevmode\begin{itemize}
\item {} 
\sphinxstyleliteralstrong{\sphinxupquote{time\_of\_day}} (\sphinxstyleliteralemphasis{\sphinxupquote{int}}) \textendash{} the current time of day {[}minutes{]}

\item {} 
\sphinxstyleliteralstrong{\sphinxupquote{t\_univ}} (\sphinxstyleliteralemphasis{\sphinxupquote{int}}) \textendash{} the universal time {[}minutes{]}

\item {} 
\sphinxstyleliteralstrong{\sphinxupquote{t}} (\sphinxstyleliteralemphasis{\sphinxupquote{int}}) \textendash{} the time to be set{[}minutes, time of day{]}

\end{itemize}

\item[{Return out}] \leavevmode
the time of the next event in universal time

\item[{Return type}] \leavevmode
int

\end{description}\end{quote}

\end{fulllineitems}


\end{fulllineitems}



\subsection{chad module}
\label{\detokenize{chad::doc}}\label{\detokenize{chad:module-chad}}\label{\detokenize{chad:chad-module}}\index{chad (module)}
This file contains data from the Consolidated Human Activity Database (CHAD). This module contains constants necessary to access various files in the CHAD.

This module contains the following classes:
\begin{enumerate}
\item {} 
{\hyperref[\detokenize{chad:chad.CHAD}]{\sphinxcrossref{\sphinxcode{\sphinxupquote{chad.CHAD}}}}}

\item {} 
{\hyperref[\detokenize{chad:chad.CHAD_RAW}]{\sphinxcrossref{\sphinxcode{\sphinxupquote{chad.CHAD\_RAW}}}}}.

\end{enumerate}
\index{CHAD (class in chad)}

\begin{fulllineitems}
\phantomsection\label{\detokenize{chad:chad.CHAD}}\pysiglinewithargsret{\sphinxbfcode{\sphinxupquote{class }}\sphinxcode{\sphinxupquote{chad.}}\sphinxbfcode{\sphinxupquote{CHAD}}}{\emph{fname}, \emph{mode='r'}}{}
Bases: \sphinxcode{\sphinxupquote{object}}

This object is in charge of accessing the compressed data files from CHAD.
\begin{quote}\begin{description}
\item[{Parameters}] \leavevmode\begin{itemize}
\item {} 
\sphinxstyleliteralstrong{\sphinxupquote{fname}} (\sphinxstyleliteralemphasis{\sphinxupquote{str}}) \textendash{} the directory to the respective compressed data files

\item {} 
\sphinxstyleliteralstrong{\sphinxupquote{mode}} (\sphinxstyleliteralemphasis{\sphinxupquote{str}}) \textendash{} the mode (read, write, or both) the zipfile will work under

\end{itemize}

\item[{Variables}] \leavevmode\begin{itemize}
\item {} 
\sphinxstyleliteralstrong{\sphinxupquote{fname\_zip}} (\sphinxstyleliteralemphasis{\sphinxupquote{str}}) \textendash{} the directory to the respective compressed data file (.zip)

\item {} 
\sphinxstyleliteralstrong{\sphinxupquote{mode}} (\sphinxstyleliteralemphasis{\sphinxupquote{int}}) \textendash{} the mode (read, write, or both) the zipfile will work under

\item {} 
\sphinxstyleliteralstrong{\sphinxupquote{z}} (\sphinxstyleliteralemphasis{\sphinxupquote{zipfile.Zipfile}}) \textendash{} object that holds the zipfile information

\end{itemize}

\end{description}\end{quote}
\index{activity\_times() (chad.CHAD method)}

\begin{fulllineitems}
\phantomsection\label{\detokenize{chad:chad.CHAD.activity_times}}\pysiglinewithargsret{\sphinxbfcode{\sphinxupquote{activity\_times}}}{\emph{df}, \emph{act\_codes}}{}
This function finds the  activity data (given by act\_codes) in the dataframe df.
\begin{quote}\begin{description}
\item[{Parameters}] \leavevmode\begin{itemize}
\item {} 
\sphinxstyleliteralstrong{\sphinxupquote{df}} (\sphinxstyleliteralemphasis{\sphinxupquote{pandas.core.frame.DataFrame}}) \textendash{} events data

\item {} 
\sphinxstyleliteralstrong{\sphinxupquote{act\_codes}} (\sphinxstyleliteralemphasis{\sphinxupquote{list}}) \textendash{} the list of CHAD activity codes specifying 1 general activity

\end{itemize}

\item[{Returns}] \leavevmode
the activity data for the selected activity codes

\item[{Return type}] \leavevmode
pandas.core.frame.DataFrame

\end{description}\end{quote}

\end{fulllineitems}

\index{get\_data() (chad.CHAD method)}

\begin{fulllineitems}
\phantomsection\label{\detokenize{chad:chad.CHAD.get_data}}\pysiglinewithargsret{\sphinxbfcode{\sphinxupquote{get\_data}}}{\emph{fname}}{}
Gets the decompressed data from the given file
\begin{quote}\begin{description}
\item[{Parameters}] \leavevmode
\sphinxstyleliteralstrong{\sphinxupquote{fname}} (\sphinxstyleliteralemphasis{\sphinxupquote{str}}) \textendash{} the name of the file to decompressed

\item[{Return data}] \leavevmode
the data

\item[{Return type}] \leavevmode
pandas.core.frame.DataFrame

\end{description}\end{quote}

\end{fulllineitems}

\index{sum\_time() (chad.CHAD method)}

\begin{fulllineitems}
\phantomsection\label{\detokenize{chad:chad.CHAD.sum_time}}\pysiglinewithargsret{\sphinxbfcode{\sphinxupquote{sum\_time}}}{\emph{df}}{}
This function merges two similar adjacent activities into one activity. This function is used normally for         the CHAD events data.
\begin{quote}\begin{description}
\item[{Parameters}] \leavevmode
\sphinxstyleliteralstrong{\sphinxupquote{df}} (\sphinxstyleliteralemphasis{\sphinxupquote{pandas.core.frame.DataFrame}}) \textendash{} the dataframe corresponding to a specific CHAD identifier

\item[{Returns}] \leavevmode
the dataframe where adjacent activities are merged into one activity

\item[{Return type}] \leavevmode
pandas.core.frame.DataFrame

\end{description}\end{quote}

\end{fulllineitems}

\index{toString() (chad.CHAD method)}

\begin{fulllineitems}
\phantomsection\label{\detokenize{chad:chad.CHAD.toString}}\pysiglinewithargsret{\sphinxbfcode{\sphinxupquote{toString}}}{}{}
Represent the contents of the compressed file.
\begin{quote}\begin{description}
\item[{Returns}] \leavevmode
a string representation of the CHAD object

\item[{Return type}] \leavevmode
string

\end{description}\end{quote}

\end{fulllineitems}


\end{fulllineitems}

\index{CHAD\_RAW (class in chad)}

\begin{fulllineitems}
\phantomsection\label{\detokenize{chad:chad.CHAD_RAW}}\pysiglinewithargsret{\sphinxbfcode{\sphinxupquote{class }}\sphinxcode{\sphinxupquote{chad.}}\sphinxbfcode{\sphinxupquote{CHAD\_RAW}}}{\emph{min\_age=18}, \emph{max\_age=130}}{}
Bases: {\hyperref[\detokenize{chad:chad.CHAD}]{\sphinxcrossref{\sphinxcode{\sphinxupquote{chad.CHAD}}}}}

This is a specific instance of {\hyperref[\detokenize{chad:chad.CHAD}]{\sphinxcrossref{\sphinxcode{\sphinxupquote{chad.CHAD}}}}} that is made for accessing the raw     CHAD data for accessing the questionnaire database and the events database.
\begin{quote}\begin{description}
\item[{Parameters}] \leavevmode\begin{itemize}
\item {} 
\sphinxstyleliteralstrong{\sphinxupquote{min\_age}} (\sphinxstyleliteralemphasis{\sphinxupquote{int}}) \textendash{} the minimum age {[}years{]} for the CHAD data age range

\item {} 
\sphinxstyleliteralstrong{\sphinxupquote{max\_age}} (\sphinxstyleliteralemphasis{\sphinxupquote{int}}) \textendash{} the maximum age {[}years{]} for the CHAD data age range

\end{itemize}

\item[{Variables}] \leavevmode\begin{itemize}
\item {} 
\sphinxstyleliteralstrong{\sphinxupquote{quest}} (\sphinxstyleliteralemphasis{\sphinxupquote{pandas.core.frame.DataFrame}}) \textendash{} the CHAD questionnaire data

\item {} 
\sphinxstyleliteralstrong{\sphinxupquote{events}} (\sphinxstyleliteralemphasis{\sphinxupquote{pandas.core.frame.DataFrame}}) \textendash{} the CHAD events data

\end{itemize}

\end{description}\end{quote}
\index{clean\_data() (chad.CHAD\_RAW method)}

\begin{fulllineitems}
\phantomsection\label{\detokenize{chad:chad.CHAD_RAW.clean_data}}\pysiglinewithargsret{\sphinxbfcode{\sphinxupquote{clean\_data}}}{}{}
This function cleans the data from the loaded CHAD .csv files for the format used for the ABM.
\begin{enumerate}
\item {} 
clean events

\item {} 
clean dates

\item {} 
set dates

\end{enumerate}
\begin{quote}\begin{description}
\item[{Returns}] \leavevmode
None

\end{description}\end{quote}

\end{fulllineitems}

\index{clean\_dates() (chad.CHAD\_RAW method)}

\begin{fulllineitems}
\phantomsection\label{\detokenize{chad:chad.CHAD_RAW.clean_dates}}\pysiglinewithargsret{\sphinxbfcode{\sphinxupquote{clean\_dates}}}{}{}
This function is needed in order to remove data where there are no dates from the dataframes that         represent the CHAD questionnaire data and the CHAD events data.
\begin{quote}\begin{description}
\item[{Returns}] \leavevmode
None

\end{description}\end{quote}

\end{fulllineitems}

\index{clean\_events() (chad.CHAD\_RAW method)}

\begin{fulllineitems}
\phantomsection\label{\detokenize{chad:chad.CHAD_RAW.clean_events}}\pysiglinewithargsret{\sphinxbfcode{\sphinxupquote{clean\_events}}}{}{}
This cleans the time information in the CHAD events data.
\begin{quote}\begin{description}
\item[{Returns}] \leavevmode
None

\end{description}\end{quote}

\end{fulllineitems}

\index{convert\_activity\_code() (chad.CHAD\_RAW method)}

\begin{fulllineitems}
\phantomsection\label{\detokenize{chad:chad.CHAD_RAW.convert_activity_code}}\pysiglinewithargsret{\sphinxbfcode{\sphinxupquote{convert\_activity\_code}}}{\emph{x}}{}
This function converts the activity code from a string into an integer. It also converts ‘X’         and ‘U’ into a numerical value.
\begin{quote}\begin{description}
\item[{Parameters}] \leavevmode
\sphinxstyleliteralstrong{\sphinxupquote{x}} (\sphinxstyleliteralemphasis{\sphinxupquote{string}}) \textendash{} the activity code that needs to be converted

\item[{Returns}] \leavevmode
None

\end{description}\end{quote}

\end{fulllineitems}

\index{convert\_military\_to\_decimal\_time() (chad.CHAD\_RAW method)}

\begin{fulllineitems}
\phantomsection\label{\detokenize{chad:chad.CHAD_RAW.convert_military_to_decimal_time}}\pysiglinewithargsret{\sphinxbfcode{\sphinxupquote{convert\_military\_to\_decimal\_time}}}{\emph{x}}{}
This function converts military time {[}00 00 - 23 59{]} to decimal time {[}0.0 - 24).
\begin{quote}\begin{description}
\item[{Parameters}] \leavevmode
\sphinxstyleliteralstrong{\sphinxupquote{x}} (\sphinxstyleliteralemphasis{\sphinxupquote{int}}) \textendash{} an integer representation of the military time where 09:00 is represented by 0900.

\item[{Returns}] \leavevmode
the time converted into decimal time

\item[{Return type}] \leavevmode
float

\end{description}\end{quote}

\end{fulllineitems}

\index{get\_PID() (chad.CHAD\_RAW method)}

\begin{fulllineitems}
\phantomsection\label{\detokenize{chad:chad.CHAD_RAW.get_PID}}\pysiglinewithargsret{\sphinxbfcode{\sphinxupquote{get\_PID}}}{\emph{x}}{}
Given a CHADID, this function returns the PID. The PID is the CHADID stripped of the last character,         which is a code for the day record.
\begin{quote}\begin{description}
\item[{Parameters}] \leavevmode
\sphinxstyleliteralstrong{\sphinxupquote{x}} (\sphinxstyleliteralemphasis{\sphinxupquote{string}}) \textendash{} the CHADID

\item[{Returns}] \leavevmode
the PID

\item[{Return type}] \leavevmode
numpy.ndarray

\end{description}\end{quote}

\end{fulllineitems}

\index{get\_data\_by\_age() (chad.CHAD\_RAW method)}

\begin{fulllineitems}
\phantomsection\label{\detokenize{chad:chad.CHAD_RAW.get_data_by_age}}\pysiglinewithargsret{\sphinxbfcode{\sphinxupquote{get\_data\_by\_age}}}{\emph{min\_age}, \emph{max\_age}}{}
This function samples the CHAD data by age via the age range inputs.
\begin{quote}\begin{description}
\item[{Parameters}] \leavevmode\begin{itemize}
\item {} 
\sphinxstyleliteralstrong{\sphinxupquote{min\_age}} (\sphinxstyleliteralemphasis{\sphinxupquote{int}}) \textendash{} the minimum age {[}years{]}

\item {} 
\sphinxstyleliteralstrong{\sphinxupquote{max\_age}} (\sphinxstyleliteralemphasis{\sphinxupquote{int}}) \textendash{} the maximum age {[}years{]}

\end{itemize}

\item[{Returns}] \leavevmode
None

\end{description}\end{quote}

\end{fulllineitems}

\index{get\_events() (chad.CHAD\_RAW method)}

\begin{fulllineitems}
\phantomsection\label{\detokenize{chad:chad.CHAD_RAW.get_events}}\pysiglinewithargsret{\sphinxbfcode{\sphinxupquote{get\_events}}}{}{}
This function gets the raw CHAD events data and returns it in the appropriate column order.
\begin{quote}\begin{description}
\item[{Returns}] \leavevmode
the CHAD events data

\item[{Return type}] \leavevmode
pandas.core.frame.DataFrame

\end{description}\end{quote}

\end{fulllineitems}

\index{get\_events\_raw() (chad.CHAD\_RAW method)}

\begin{fulllineitems}
\phantomsection\label{\detokenize{chad:chad.CHAD_RAW.get_events_raw}}\pysiglinewithargsret{\sphinxbfcode{\sphinxupquote{get\_events\_raw}}}{}{}
This function returns a data frame of the raw events data.
\begin{quote}\begin{description}
\item[{Return data}] \leavevmode
the raw CHAD events data

\item[{Return type}] \leavevmode
pandas.core.frame.DataFrame

\end{description}\end{quote}

\end{fulllineitems}

\index{get\_quest() (chad.CHAD\_RAW method)}

\begin{fulllineitems}
\phantomsection\label{\detokenize{chad:chad.CHAD_RAW.get_quest}}\pysiglinewithargsret{\sphinxbfcode{\sphinxupquote{get\_quest}}}{}{}
This function returns a data frame of the raw questionnaire data. However, the data must have the date         marked explicitly to be accepted.
\begin{quote}\begin{description}
\item[{Returns}] \leavevmode
the CHAD questionnaire data in the correct column order

\item[{Return type}] \leavevmode
pandas.core.frame.DataFrame

\end{description}\end{quote}

\end{fulllineitems}

\index{get\_quest\_raw() (chad.CHAD\_RAW method)}

\begin{fulllineitems}
\phantomsection\label{\detokenize{chad:chad.CHAD_RAW.get_quest_raw}}\pysiglinewithargsret{\sphinxbfcode{\sphinxupquote{get\_quest\_raw}}}{}{}
This function returns a data frame of the raw questionnaire data and add the PID information.
\begin{quote}\begin{description}
\item[{Returns}] \leavevmode
the raw CHAD questionnaire data

\item[{Return type}] \leavevmode
pandas.core.frame.DataFrame

\end{description}\end{quote}

\end{fulllineitems}

\index{set\_dates() (chad.CHAD\_RAW method)}

\begin{fulllineitems}
\phantomsection\label{\detokenize{chad:chad.CHAD_RAW.set_dates}}\pysiglinewithargsret{\sphinxbfcode{\sphinxupquote{set\_dates}}}{}{}
This function converts the date information in the CHAD questionnaire and CHAD events datafrmaes         from strings to python datetime objects.
\begin{quote}\begin{description}
\item[{Returns}] \leavevmode
None

\end{description}\end{quote}

\end{fulllineitems}

\index{set\_times() (chad.CHAD\_RAW method)}

\begin{fulllineitems}
\phantomsection\label{\detokenize{chad:chad.CHAD_RAW.set_times}}\pysiglinewithargsret{\sphinxbfcode{\sphinxupquote{set\_times}}}{}{}
This function handles setting the time information for formatting in the CHAD questionnaire and events         dataframes
\begin{enumerate}
\item {} 
converts the time from military time (0000 - 2359{]} to decimal time {[}0, 24) in the CHAD         evnets and questionnaire data for the start time and end time

\item {} 
converts the duration to hours

\item {} 
drops the military time (start time and end time) and duration (minutes) from the respective         data frames

\end{enumerate}
\begin{quote}\begin{description}
\item[{Returns}] \leavevmode
None

\end{description}\end{quote}

\end{fulllineitems}


\end{fulllineitems}

\index{sample\_stats() (in module chad)}

\begin{fulllineitems}
\phantomsection\label{\detokenize{chad:chad.sample_stats}}\pysiglinewithargsret{\sphinxcode{\sphinxupquote{chad.}}\sphinxbfcode{\sphinxupquote{sample\_stats}}}{\emph{sample}}{}
This function takes sample data and returns the mean and the standard deviation.
\begin{quote}\begin{description}
\item[{Parameters}] \leavevmode
\sphinxstyleliteralstrong{\sphinxupquote{sample}} (\sphinxstyleliteralemphasis{\sphinxupquote{pandas.core.frame.DataFrame}}) \textendash{} the data to analyze

\item[{Return s\_mean}] \leavevmode
the mean of the sample data

\item[{Return s\_std}] \leavevmode
the standard deviation of the sample data

\item[{Return type}] \leavevmode
pandas.core.frame.DataFrame

\item[{Return type}] \leavevmode
pandas.core.frame.DataFrame

\end{description}\end{quote}

\end{fulllineitems}



\subsection{chad\_code module}
\label{\detokenize{chad_code::doc}}\label{\detokenize{chad_code:chad-code-module}}\label{\detokenize{chad_code:module-chad_code}}\index{chad\_code (module)}
This module contains activity codes found in the Consolidated Human Activity Database (CHAD).

The following general {\hyperref[\detokenize{chad_code:module-chad_code}]{\sphinxcrossref{\sphinxcode{\sphinxupquote{chad\_code}}}}} constants consist of groupings of CHAD activity codes
\begin{enumerate}
\item {} 
sleep

\item {} 
eat

\item {} \begin{description}
\item[{work }] \leavevmode\begin{itemize}
\item {} 
work and income producing activities; work, general; work, income-related only; work, secondary     (income-related); work breaks

\end{itemize}

\end{description}

\item {} \begin{description}
\item[{education}] \leavevmode\begin{itemize}
\item {} 
general education and professional training, attending full-time school, attend day-care; attend school     kindergarten - 12th grade

\end{itemize}

\end{description}

\item {} \begin{description}
\item[{commute to/ from work}] \leavevmode\begin{itemize}
\item {} 
travel to/ from work general; travel to/ from work by bus; travel to/ from work by foot; travel to/ from via     motor vehicle; travel to/ from work via motor vehicle, by driving; travel to/from work via motor vehicle by     driving via motor vehicle, by riding; travel to/ from work waiting

\end{itemize}

\end{description}

\item {} \begin{description}
\item[{commute to/ from school}] \leavevmode\begin{itemize}
\item {} 
travel for education general; travel for education by bus; travel for education by foot; travel to/ from       school via motor vehicle; travel for education via motor vehicle, by driving; travel for education via motor       vehicle, by riding; travel for education, waiting

\end{itemize}

\end{description}

\item {} \begin{description}
\item[{All}] \leavevmode\begin{itemize}
\item {} 
sleep + eat + work + education + commute to/ from work + commute to/ from school

\end{itemize}

\end{description}

\end{enumerate}


\subsection{commute module}
\label{\detokenize{commute::doc}}\label{\detokenize{commute:module-commute}}\label{\detokenize{commute:commute-module}}\index{commute (module)}
This module contains about activities associated with commuting to and from work. This class is an Activity ({\hyperref[\detokenize{activity:activity.Activity}]{\sphinxcrossref{\sphinxcode{\sphinxupquote{activity.Activity}}}}}) that gives a Person ({\hyperref[\detokenize{person:person.Person}]{\sphinxcrossref{\sphinxcode{\sphinxupquote{person.Person}}}}}) the ability to commute to/ from work/ school and satisfy the need Travel ({\hyperref[\detokenize{travel:travel.Travel}]{\sphinxcrossref{\sphinxcode{\sphinxupquote{travel.Travel}}}}}).

This module contains the following classes:
\begin{enumerate}
\item {} 
{\hyperref[\detokenize{commute:commute.Commute}]{\sphinxcrossref{\sphinxcode{\sphinxupquote{commute.Commute}}}}} (general commuting capability)

\item {} 
{\hyperref[\detokenize{commute:commute.Commute_To_Work}]{\sphinxcrossref{\sphinxcode{\sphinxupquote{commute.Commute\_To\_Work}}}}} (commute to work/ school)

\item {} 
{\hyperref[\detokenize{commute:commute.Commute_From_Work}]{\sphinxcrossref{\sphinxcode{\sphinxupquote{commute.Commute\_From\_Work}}}}} (commute from work/ school)

\end{enumerate}
\index{Commute (class in commute)}

\begin{fulllineitems}
\phantomsection\label{\detokenize{commute:commute.Commute}}\pysigline{\sphinxbfcode{\sphinxupquote{class }}\sphinxcode{\sphinxupquote{commute.}}\sphinxbfcode{\sphinxupquote{Commute}}}
Bases: {\hyperref[\detokenize{activity:activity.Activity}]{\sphinxcrossref{\sphinxcode{\sphinxupquote{activity.Activity}}}}}

This class allows for commuting. This class is to be derived from.
\index{end() (commute.Commute method)}

\begin{fulllineitems}
\phantomsection\label{\detokenize{commute:commute.Commute.end}}\pysiglinewithargsret{\sphinxbfcode{\sphinxupquote{end}}}{\emph{p}, \emph{local}}{}
This function handles the end of an activity.
\begin{quote}\begin{description}
\item[{Parameters}] \leavevmode\begin{itemize}
\item {} 
\sphinxstyleliteralstrong{\sphinxupquote{p}} ({\hyperref[\detokenize{person:person.Person}]{\sphinxcrossref{\sphinxstyleliteralemphasis{\sphinxupquote{person.Person}}}}}) \textendash{} the person of interest

\item {} 
\sphinxstyleliteralstrong{\sphinxupquote{local}} (\sphinxstyleliteralemphasis{\sphinxupquote{int}}) \textendash{} the local location (work or home)

\end{itemize}

\item[{Returns}] \leavevmode
None

\end{description}\end{quote}

\end{fulllineitems}

\index{end\_commute() (commute.Commute method)}

\begin{fulllineitems}
\phantomsection\label{\detokenize{commute:commute.Commute.end_commute}}\pysiglinewithargsret{\sphinxbfcode{\sphinxupquote{end\_commute}}}{\emph{p}}{}
This function ends the commuting activity.

\begin{sphinxadmonition}{note}{Note:}
This function is to be overridden
\end{sphinxadmonition}
\begin{quote}\begin{description}
\item[{Parameters}] \leavevmode
\sphinxstyleliteralstrong{\sphinxupquote{p}} ({\hyperref[\detokenize{person:person.Person}]{\sphinxcrossref{\sphinxstyleliteralemphasis{\sphinxupquote{person.Person}}}}}) \textendash{} the person of interest

\item[{Returns}] \leavevmode
None

\end{description}\end{quote}

\end{fulllineitems}

\index{start() (commute.Commute method)}

\begin{fulllineitems}
\phantomsection\label{\detokenize{commute:commute.Commute.start}}\pysiglinewithargsret{\sphinxbfcode{\sphinxupquote{start}}}{\emph{p}}{}
This handles the start of the commute activity.
\begin{itemize}
\item {} 
If the current location of person is at home, the person is going to work, so set the         location to \sphinxcode{\sphinxupquote{location.OFF\_SITE}}.

\item {} 
If the current location of the person is off site, the person is going back home, so         set the location to \sphinxcode{\sphinxupquote{location.HOME}}.

\end{itemize}
\begin{quote}\begin{description}
\item[{Parameters}] \leavevmode
\sphinxstyleliteralstrong{\sphinxupquote{p}} ({\hyperref[\detokenize{person:person.Person}]{\sphinxcrossref{\sphinxstyleliteralemphasis{\sphinxupquote{person.Person}}}}}) \textendash{} the person of interest

\item[{Returns}] \leavevmode
None

\end{description}\end{quote}

\end{fulllineitems}

\index{start\_commute() (commute.Commute method)}

\begin{fulllineitems}
\phantomsection\label{\detokenize{commute:commute.Commute.start_commute}}\pysiglinewithargsret{\sphinxbfcode{\sphinxupquote{start\_commute}}}{\emph{p}}{}
This function sets the variables pertaining to starting the commute activity by doing         the following:
\begin{enumerate}
\item {} 
set the status of the person to \sphinxcode{\sphinxupquote{location.TRANSIT}}

\item {} 
set the location of the asset to \sphinxcode{\sphinxupquote{location.TRANSIT}}

\item {} 
set the person’s state start time of the commute

\item {} 
set the person’s state end time for the commute

\item {} 
update the asset

\item {} 
update the scheduler for the travel need for the end of the commute

\end{enumerate}
\begin{quote}\begin{description}
\item[{Parameters}] \leavevmode
\sphinxstyleliteralstrong{\sphinxupquote{p}} ({\hyperref[\detokenize{person:person.Person}]{\sphinxcrossref{\sphinxstyleliteralemphasis{\sphinxupquote{person.Person}}}}}) \textendash{} the person of interest

\item[{Returns}] \leavevmode
None

\end{description}\end{quote}

\end{fulllineitems}


\end{fulllineitems}

\index{Commute\_From\_Work (class in commute)}

\begin{fulllineitems}
\phantomsection\label{\detokenize{commute:commute.Commute_From_Work}}\pysigline{\sphinxbfcode{\sphinxupquote{class }}\sphinxcode{\sphinxupquote{commute.}}\sphinxbfcode{\sphinxupquote{Commute\_From\_Work}}}
Bases: {\hyperref[\detokenize{commute:commute.Commute}]{\sphinxcrossref{\sphinxcode{\sphinxupquote{commute.Commute}}}}}

This class allows for the activity: commuting from work.
\index{advertise() (commute.Commute\_From\_Work method)}

\begin{fulllineitems}
\phantomsection\label{\detokenize{commute:commute.Commute_From_Work.advertise}}\pysiglinewithargsret{\sphinxbfcode{\sphinxupquote{advertise}}}{\emph{p}}{}
This function calculates the score of to commute from work. It does this by doing the         following:
\begin{enumerate}
\item {} 
calculate advertisement only if the person is located at work (off-site)

\item {} 
calculate the score \(S\)
\begin{quote}
\begin{equation*}
\begin{split}S = \begin{cases}
0  & n_{travel}(t) > \lambda \\
W( n_{travel}(t) ) - W( n_{travel}(t + \Delta{t} )) & n_{travel}(t) \le \lambda
\end{cases}\end{split}
\end{equation*}\begin{description}
\item[{where}] \leavevmode\begin{itemize}
\item {} 
\(t\) is the current time

\item {} 
\(\Delta{t}\) is the duration of commuting from work {[}minutes{]}

\item {} 
\(n_{travel}(t)\) is the satiation for Travel at time \(t\)

\item {} 
\(\lambda\) is the threshold value of Travel

\item {} 
\(W(n)\) is the weight function for Travel

\end{itemize}

\end{description}
\end{quote}

\end{enumerate}
\begin{quote}\begin{description}
\item[{Parameters}] \leavevmode
\sphinxstyleliteralstrong{\sphinxupquote{p}} ({\hyperref[\detokenize{person:person.Person}]{\sphinxcrossref{\sphinxstyleliteralemphasis{\sphinxupquote{person.Person}}}}}) \textendash{} the person of interest

\item[{Returns}] \leavevmode
the advertised score

\item[{Return type}] \leavevmode
float

\end{description}\end{quote}

\end{fulllineitems}

\index{calc\_end\_time() (commute.Commute\_From\_Work method)}

\begin{fulllineitems}
\phantomsection\label{\detokenize{commute:commute.Commute_From_Work.calc_end_time}}\pysiglinewithargsret{\sphinxbfcode{\sphinxupquote{calc\_end\_time}}}{\emph{p}}{}~\begin{enumerate}
\item {} 
calculate the end time (minutes, universal time) of the commute

\item {} 
set the the end time in the person’s state

\end{enumerate}
\begin{quote}\begin{description}
\item[{Parameters}] \leavevmode
\sphinxstyleliteralstrong{\sphinxupquote{p}} ({\hyperref[\detokenize{person:person.Person}]{\sphinxcrossref{\sphinxstyleliteralemphasis{\sphinxupquote{person.Person}}}}}) \textendash{} the person of interest

\item[{Returns}] \leavevmode
None

\end{description}\end{quote}

\end{fulllineitems}

\index{end() (commute.Commute\_From\_Work method)}

\begin{fulllineitems}
\phantomsection\label{\detokenize{commute:commute.Commute_From_Work.end}}\pysiglinewithargsret{\sphinxbfcode{\sphinxupquote{end}}}{\emph{p}}{}
This function handles the end of an activity.
\begin{quote}\begin{description}
\item[{Parameters}] \leavevmode
\sphinxstyleliteralstrong{\sphinxupquote{p}} ({\hyperref[\detokenize{person:person.Person}]{\sphinxcrossref{\sphinxstyleliteralemphasis{\sphinxupquote{person.Person}}}}}) \textendash{} the person of interest

\item[{Returns}] \leavevmode
None

\end{description}\end{quote}

\end{fulllineitems}

\index{end\_commute() (commute.Commute\_From\_Work method)}

\begin{fulllineitems}
\phantomsection\label{\detokenize{commute:commute.Commute_From_Work.end_commute}}\pysiglinewithargsret{\sphinxbfcode{\sphinxupquote{end\_commute}}}{\emph{p}}{}
This function sets the variables pertaining to ending the commute activity.
\begin{enumerate}
\item {} 
Sets the person’s state to idle (\sphinxcode{\sphinxupquote{state.IDLE}})

\item {} 
Updates the asset’s state and number of users

\item {} 
Sets the travel magnitude

\item {} 
Sets the work magnitude to \(\eta_{work}\) to allow for work         to be the next activity, even if commute ends begin the work-start time

\item {} 
Sets the person’s state’s end time

\end{enumerate}
\begin{quote}\begin{description}
\item[{Parameters}] \leavevmode\begin{itemize}
\item {} 
\sphinxstyleliteralstrong{\sphinxupquote{p}} ({\hyperref[\detokenize{person:person.Person}]{\sphinxcrossref{\sphinxstyleliteralemphasis{\sphinxupquote{person.Person}}}}}) \textendash{} person of interest

\item {} 
\sphinxstyleliteralstrong{\sphinxupquote{destination}} (\sphinxstyleliteralemphasis{\sphinxupquote{int}}) \textendash{} a local location where the commute ends (home or workplace)

\end{itemize}

\item[{Returns}] \leavevmode
None

\end{description}\end{quote}

\end{fulllineitems}

\index{start() (commute.Commute\_From\_Work method)}

\begin{fulllineitems}
\phantomsection\label{\detokenize{commute:commute.Commute_From_Work.start}}\pysiglinewithargsret{\sphinxbfcode{\sphinxupquote{start}}}{\emph{p}}{}
This handles the start of the commute activity.
\begin{itemize}
\item {} 
If the current location of person is at home, the person is going to work, so set the         location to \sphinxcode{\sphinxupquote{location.OFF\_SITE}}

\item {} 
If the current location of the person is off site, the person is going back home, so         set the location to \sphinxcode{\sphinxupquote{location.HOME}}

\end{itemize}
\begin{quote}\begin{description}
\item[{Parameters}] \leavevmode
\sphinxstyleliteralstrong{\sphinxupquote{p}} ({\hyperref[\detokenize{person:person.Person}]{\sphinxcrossref{\sphinxstyleliteralemphasis{\sphinxupquote{person.Person}}}}}) \textendash{} the person of interest

\item[{Returns}] \leavevmode
None

\end{description}\end{quote}

\end{fulllineitems}


\end{fulllineitems}

\index{Commute\_To\_Work (class in commute)}

\begin{fulllineitems}
\phantomsection\label{\detokenize{commute:commute.Commute_To_Work}}\pysigline{\sphinxbfcode{\sphinxupquote{class }}\sphinxcode{\sphinxupquote{commute.}}\sphinxbfcode{\sphinxupquote{Commute\_To\_Work}}}
Bases: {\hyperref[\detokenize{commute:commute.Commute}]{\sphinxcrossref{\sphinxcode{\sphinxupquote{commute.Commute}}}}}

This class allows for the activity: commute to work
\index{advertise() (commute.Commute\_To\_Work method)}

\begin{fulllineitems}
\phantomsection\label{\detokenize{commute:commute.Commute_To_Work.advertise}}\pysiglinewithargsret{\sphinxbfcode{\sphinxupquote{advertise}}}{\emph{p}}{}
This function calculates the score of commuting to work by doing the following:
\begin{enumerate}
\item {} 
calculate advertisement only if the person is located at work (off-site)

\item {} 
calculate the score \(S\)
\begin{quote}
\begin{equation*}
\begin{split}S = \begin{cases}
0  & n_{travel}(t) > \lambda \\
W( n_{travel}(t) ) - W( n_{travel}(t + \Delta{t} )) & n_{travel}(t) \le \lambda
\end{cases}\end{split}
\end{equation*}\begin{description}
\item[{where}] \leavevmode\begin{itemize}
\item {} 
\(t\) is the current time

\item {} 
\(\Delta{t}\) is the duration of commuting to work {[}minutes{]}

\item {} 
\(n_{travel}(t)\) is the satiation for Travel at time \(t\)

\item {} 
\(\lambda\) is the threshold value of Travel

\item {} 
\(W(n)\) is the weight function for Travel

\end{itemize}

\end{description}
\end{quote}

\end{enumerate}
\begin{quote}\begin{description}
\item[{Parameters}] \leavevmode
\sphinxstyleliteralstrong{\sphinxupquote{p}} ({\hyperref[\detokenize{person:person.Person}]{\sphinxcrossref{\sphinxstyleliteralemphasis{\sphinxupquote{person.Person}}}}}) \textendash{} the person of interest

\item[{Return score}] \leavevmode
the advertisement score

\item[{Return type}] \leavevmode
float

\end{description}\end{quote}

\end{fulllineitems}

\index{calc\_end\_time() (commute.Commute\_To\_Work method)}

\begin{fulllineitems}
\phantomsection\label{\detokenize{commute:commute.Commute_To_Work.calc_end_time}}\pysiglinewithargsret{\sphinxbfcode{\sphinxupquote{calc\_end\_time}}}{\emph{p}}{}
Given the commute duration, store the end time. This function does the following:
\begin{enumerate}
\item {} 
calculate the end time {[}universal time{]} of the commute.

\item {} 
store the end time in the person.state

\end{enumerate}
\begin{quote}\begin{description}
\item[{Parameters}] \leavevmode
\sphinxstyleliteralstrong{\sphinxupquote{p}} ({\hyperref[\detokenize{person:person.Person}]{\sphinxcrossref{\sphinxstyleliteralemphasis{\sphinxupquote{person.Person}}}}}) \textendash{} the person of interest

\item[{Returns}] \leavevmode
None

\end{description}\end{quote}

\end{fulllineitems}

\index{end() (commute.Commute\_To\_Work method)}

\begin{fulllineitems}
\phantomsection\label{\detokenize{commute:commute.Commute_To_Work.end}}\pysiglinewithargsret{\sphinxbfcode{\sphinxupquote{end}}}{\emph{p}}{}
This function handles the logistics of ending the commute to work activity.
\begin{quote}\begin{description}
\item[{Parameters}] \leavevmode
\sphinxstyleliteralstrong{\sphinxupquote{p}} ({\hyperref[\detokenize{person:person.Person}]{\sphinxcrossref{\sphinxstyleliteralemphasis{\sphinxupquote{person.Person}}}}}) \textendash{} the person of interest

\item[{Returns}] \leavevmode
None

\end{description}\end{quote}

\end{fulllineitems}

\index{end\_commute() (commute.Commute\_To\_Work method)}

\begin{fulllineitems}
\phantomsection\label{\detokenize{commute:commute.Commute_To_Work.end_commute}}\pysiglinewithargsret{\sphinxbfcode{\sphinxupquote{end\_commute}}}{\emph{p}}{}
This function handles the logistics concerning ending the commute. Specifically,
this function does the following:
\begin{enumerate}
\item {} 
the asset is freed up from use

\item {} 
the magnitude of the travel need is set \(n_{travel}=1\)

\item {} 
the person’s state is set to idle (\sphinxcode{\sphinxupquote{state.IDLE}})

\item {} 
the person’s location is set to the location of the job

\item {} 
the asset’s location is set to the location of the job

\item {} 
the person’s income need is set to \(n_{income}=\eta_{work}\)

\item {} 
update the commute to work duration

\item {} 
calculate the time until the next leave work event

\item {} 
update the schedule for the travel need

\end{enumerate}
\begin{quote}\begin{description}
\item[{Parameters}] \leavevmode
\sphinxstyleliteralstrong{\sphinxupquote{p}} ({\hyperref[\detokenize{person:person.Person}]{\sphinxcrossref{\sphinxstyleliteralemphasis{\sphinxupquote{person.Person}}}}}) \textendash{} the person of interest

\item[{Returns}] \leavevmode


\end{description}\end{quote}

\end{fulllineitems}

\index{start() (commute.Commute\_To\_Work method)}

\begin{fulllineitems}
\phantomsection\label{\detokenize{commute:commute.Commute_To_Work.start}}\pysiglinewithargsret{\sphinxbfcode{\sphinxupquote{start}}}{\emph{p}}{}
This function handles the start of the commute to work activity. If the current location of person is         at home, the person is going to work, so set the location to workplace location (\sphinxcode{\sphinxupquote{location.OFF\_SITE}})
\begin{quote}\begin{description}
\item[{Parameters}] \leavevmode
\sphinxstyleliteralstrong{\sphinxupquote{p}} ({\hyperref[\detokenize{person:person.Person}]{\sphinxcrossref{\sphinxstyleliteralemphasis{\sphinxupquote{person.Person}}}}}) \textendash{} the person of interest

\item[{Returns}] \leavevmode
None

\end{description}\end{quote}

\end{fulllineitems}

\index{start\_commute() (commute.Commute\_To\_Work method)}

\begin{fulllineitems}
\phantomsection\label{\detokenize{commute:commute.Commute_To_Work.start_commute}}\pysiglinewithargsret{\sphinxbfcode{\sphinxupquote{start\_commute}}}{\emph{p}}{}
This function sets the variables pertaining to starting the commute to work activity. Specifically,         the function does the following:
\begin{enumerate}
\item {} 
set the person’s status to \sphinxcode{\sphinxupquote{state.TRANSIT}}

\item {} 
set the asset’s location to \sphinxcode{\sphinxupquote{location.TRANSIT}}

\item {} 
set the person’s state start time to the current time

\item {} 
calculate the end time of commute to work

\item {} 
update the asset’s update

\item {} 
update the scheduler for the travel need to take into account the end of the commute

\item {} 
update the scheduler for the income need to take into account the end of the commute

\end{enumerate}
\begin{quote}\begin{description}
\item[{Parameters}] \leavevmode
\sphinxstyleliteralstrong{\sphinxupquote{p}} ({\hyperref[\detokenize{person:person.Person}]{\sphinxcrossref{\sphinxstyleliteralemphasis{\sphinxupquote{person.Person}}}}}) \textendash{} the person of interest

\item[{Returns}] \leavevmode
None

\end{description}\end{quote}

\end{fulllineitems}


\end{fulllineitems}



\subsection{diary module}
\label{\detokenize{diary::doc}}\label{\detokenize{diary:diary-module}}\label{\detokenize{diary:module-diary}}\index{diary (module)}
This module contains code that governs the activity-diaries. Each activity diary contains
dataframes that store the activity-diaries for each person. The activity-diaries are the
output of the Agent-Based Model of Human Activity Patterns (ABMHAP) simulation.

This module contains class {\hyperref[\detokenize{diary:diary.Diary}]{\sphinxcrossref{\sphinxcode{\sphinxupquote{diary.Diary}}}}}.
\index{Diary (class in diary)}

\begin{fulllineitems}
\phantomsection\label{\detokenize{diary:diary.Diary}}\pysiglinewithargsret{\sphinxbfcode{\sphinxupquote{class }}\sphinxcode{\sphinxupquote{diary.}}\sphinxbfcode{\sphinxupquote{Diary}}}{\emph{t}, \emph{act}, \emph{local}}{}
Bases: \sphinxcode{\sphinxupquote{object}}

This class represents the activity-diaries for a person.
\begin{quote}\begin{description}
\item[{Parameters}] \leavevmode\begin{itemize}
\item {} 
\sphinxstyleliteralstrong{\sphinxupquote{t}} (\sphinxstyleliteralemphasis{\sphinxupquote{numpy.ndarray}}) \textendash{} the start times for each activity {[}universal time, minutes{]}

\item {} 
\sphinxstyleliteralstrong{\sphinxupquote{act}} (\sphinxstyleliteralemphasis{\sphinxupquote{numpy.ndarray}}) \textendash{} the activity code done at each time step {[}integer{]} (flattened array)

\item {} 
\sphinxstyleliteralstrong{\sphinxupquote{local}} (\sphinxstyleliteralemphasis{\sphinxupquote{numpy.ndarray}}) \textendash{} the history of location codes done by a person

\end{itemize}

\item[{Variables}] \leavevmode\begin{itemize}
\item {} 
\sphinxstyleliteralstrong{\sphinxupquote{colnames}} (\sphinxstyleliteralemphasis{\sphinxupquote{list}}) \textendash{} the column names for the activity diary in order

\item {} 
\sphinxstyleliteralstrong{\sphinxupquote{df}} (\sphinxstyleliteralemphasis{\sphinxupquote{pandas.core.frame.DataFrame}}) \textendash{} the activity-diary

\end{itemize}

\end{description}\end{quote}
\index{create\_activity\_diary() (diary.Diary method)}

\begin{fulllineitems}
\phantomsection\label{\detokenize{diary:diary.Diary.create_activity_diary}}\pysiglinewithargsret{\sphinxbfcode{\sphinxupquote{create\_activity\_diary}}}{\emph{t}, \emph{act}, \emph{local}}{}
This function creates the activity diary for a given agent in the simulation.

The activity diary contains:
\begin{enumerate}
\item {} 
the start-time and end-time for each activity

\item {} 
the activity code

\end{enumerate}
\begin{quote}\begin{description}
\item[{Parameters}] \leavevmode\begin{itemize}
\item {} 
\sphinxstyleliteralstrong{\sphinxupquote{t}} (\sphinxstyleliteralemphasis{\sphinxupquote{numpy.ndarray}}) \textendash{} the simulation times {[}universal time, minutes{]}

\item {} 
\sphinxstyleliteralstrong{\sphinxupquote{act}} (\sphinxstyleliteralemphasis{\sphinxupquote{numpy.ndarray}}) \textendash{} the activity code done at each time step {[}integer{]} (flattened array)

\end{itemize}

\item[{Returns}] \leavevmode
a tuple containing the following: the array indices for each activity grouping, the activity diaries         in a numerical format, the activity diary in a string format, and the column names for each data type

\end{description}\end{quote}

Each diary is a tuple that contains the following:
\begin{enumerate}
\item {} 
the day number of the start of the activity

\item {} 
the (start-time, end-time) for the activity event

\item {} 
the activity code for the activity event

\item {} 
the location of the event

\end{enumerate}

\end{fulllineitems}

\index{get\_day\_end() (diary.Diary method)}

\begin{fulllineitems}
\phantomsection\label{\detokenize{diary:diary.Diary.get_day_end}}\pysiglinewithargsret{\sphinxbfcode{\sphinxupquote{get\_day\_end}}}{\emph{day\_start}, \emph{start}, \emph{dt}}{}
This function gets the day that an activity ends.
\begin{quote}\begin{description}
\item[{Parameters}] \leavevmode\begin{itemize}
\item {} 
\sphinxstyleliteralstrong{\sphinxupquote{day\_start}} (\sphinxstyleliteralemphasis{\sphinxupquote{numpy.ndarray}}) \textendash{} the day an activity starts

\item {} 
\sphinxstyleliteralstrong{\sphinxupquote{start}} (\sphinxstyleliteralemphasis{\sphinxupquote{numpy.ndarray}}) \textendash{} the time an activity starts {[}hours{]}

\item {} 
\sphinxstyleliteralstrong{\sphinxupquote{dt}} (\sphinxstyleliteralemphasis{\sphinxupquote{numpy.ndarray}}) \textendash{} the duration for an activity {[}hours{]}

\end{itemize}

\item[{Returns}] \leavevmode
the day an activity ends

\item[{Return type}] \leavevmode
numpy.ndarray

\end{description}\end{quote}

\end{fulllineitems}

\index{get\_weekday\_data() (diary.Diary method)}

\begin{fulllineitems}
\phantomsection\label{\detokenize{diary:diary.Diary.get_weekday_data}}\pysiglinewithargsret{\sphinxbfcode{\sphinxupquote{get\_weekday\_data}}}{\emph{df=None}}{}
This function pulls out data that only corresponds to the weekday data.
\begin{quote}\begin{description}
\item[{Parameters}] \leavevmode
\sphinxstyleliteralstrong{\sphinxupquote{df}} (\sphinxstyleliteralemphasis{\sphinxupquote{pandas.core.frame.DataFrame}}) \textendash{} the activity-diary of interest. If df is None, then use the dataframe         associated with the diary object

\item[{Returns}] \leavevmode
the activity-diary of data that occur on weekdays

\end{description}\end{quote}

\end{fulllineitems}

\index{get\_weekday\_idx() (diary.Diary method)}

\begin{fulllineitems}
\phantomsection\label{\detokenize{diary:diary.Diary.get_weekday_idx}}\pysiglinewithargsret{\sphinxbfcode{\sphinxupquote{get\_weekday\_idx}}}{\emph{df=None}}{}
Get the indices of the data that occurs on weekdays. An activity is considered to be on the weekday if         the activity \sphinxstylestrong{ends} on Monday - Friday.
\begin{quote}\begin{description}
\item[{Parameters}] \leavevmode
\sphinxstyleliteralstrong{\sphinxupquote{df}} (\sphinxstyleliteralemphasis{\sphinxupquote{pandas.core.frame.DataFrame}}) \textendash{} the activity-diary of interest. If df is None, then use the dataframe         associated with the diary object

\item[{Returns}] \leavevmode
boolean indices of which activities end during the weekend

\item[{Return type}] \leavevmode
numpy.ndarray

\end{description}\end{quote}

\end{fulllineitems}

\index{get\_weekend\_data() (diary.Diary method)}

\begin{fulllineitems}
\phantomsection\label{\detokenize{diary:diary.Diary.get_weekend_data}}\pysiglinewithargsret{\sphinxbfcode{\sphinxupquote{get\_weekend\_data}}}{\emph{df=None}}{}
This function pulls out data that only corresponds to the weekend data.
\begin{quote}\begin{description}
\item[{Parameters}] \leavevmode
\sphinxstyleliteralstrong{\sphinxupquote{df}} (\sphinxstyleliteralemphasis{\sphinxupquote{pandas.core.frame.DataFrame}}) \textendash{} the activity-diary of interest. If df is None, the use  the dataframe         associated with the current diary object

\item[{Returns}] \leavevmode
an activity-diary of data that occurs on weekends

\end{description}\end{quote}

\end{fulllineitems}

\index{get\_weekend\_idx() (diary.Diary method)}

\begin{fulllineitems}
\phantomsection\label{\detokenize{diary:diary.Diary.get_weekend_idx}}\pysiglinewithargsret{\sphinxbfcode{\sphinxupquote{get\_weekend\_idx}}}{\emph{df=None}}{}
Get the indices of the data that occurs on weekend. An activity is considered to be on the weekend if         the activity \sphinxstylestrong{ends} on Saturday or Sunday.
\begin{quote}\begin{description}
\item[{Parameters}] \leavevmode
\sphinxstyleliteralstrong{\sphinxupquote{df}} (\sphinxstyleliteralemphasis{\sphinxupquote{pandas.core.frame.DataFrame}}) \textendash{} the activity-diary of interest. If df is None, then use the dataframe         associated with the diary object

\item[{Returns}] \leavevmode
boolean indices of which activities end during the weekend

\item[{Return type}] \leavevmode
numpy.ndarray

\end{description}\end{quote}

\end{fulllineitems}

\index{group\_activity() (diary.Diary method)}

\begin{fulllineitems}
\phantomsection\label{\detokenize{diary:diary.Diary.group_activity}}\pysiglinewithargsret{\sphinxbfcode{\sphinxupquote{group\_activity}}}{\emph{t}, \emph{y}}{}
This function groups activities in chronological order.
\begin{quote}\begin{description}
\item[{Parameters}] \leavevmode\begin{itemize}
\item {} 
\sphinxstyleliteralstrong{\sphinxupquote{t}} (\sphinxstyleliteralemphasis{\sphinxupquote{numpy.ndarray}}) \textendash{} the start time for activities

\item {} 
\sphinxstyleliteralstrong{\sphinxupquote{y}} (\sphinxstyleliteralemphasis{\sphinxupquote{numpy.ndarray}}) \textendash{} the activity code that corresponds with the respective time

\end{itemize}

\item[{Returns}] \leavevmode
a list of each unique group-lists. Each group-list contains a tuple         for (time step, activity code)

\end{description}\end{quote}

\end{fulllineitems}

\index{group\_activity\_indices() (diary.Diary method)}

\begin{fulllineitems}
\phantomsection\label{\detokenize{diary:diary.Diary.group_activity_indices}}\pysiglinewithargsret{\sphinxbfcode{\sphinxupquote{group\_activity\_indices}}}{\emph{groups}}{}
This function returns the indices for each continuous activity in chronological order.

\begin{sphinxadmonition}{note}{Note:}
The output is the time step number \sphinxstylestrong{not} the value of time
\end{sphinxadmonition}
\begin{quote}\begin{description}
\item[{Parameters}] \leavevmode
\sphinxstyleliteralstrong{\sphinxupquote{groups}} (\sphinxstyleliteralemphasis{\sphinxupquote{list}}) \textendash{} a list of tuples of (time step, activity code)

\item[{Returns}] \leavevmode


\end{description}\end{quote}

\end{fulllineitems}

\index{group\_activity\_key() (diary.Diary method)}

\begin{fulllineitems}
\phantomsection\label{\detokenize{diary:diary.Diary.group_activity_key}}\pysiglinewithargsret{\sphinxbfcode{\sphinxupquote{group\_activity\_key}}}{\emph{x}}{}
This is the key function used in groupby in order to group consecutive time-step-activity pairs. This is         necessary for creating an activity diary.
\begin{quote}\begin{description}
\item[{Parameters}] \leavevmode
\sphinxstyleliteralstrong{\sphinxupquote{x}} (\sphinxstyleliteralemphasis{\sphinxupquote{tuple}}) \textendash{} the data in the form of ( index, (time step, activity code) )

\item[{Returns}] \leavevmode
the key for sorting ( , activity code)

\item[{Return type}] \leavevmode
tuple

\end{description}\end{quote}

\end{fulllineitems}

\index{is\_weekend() (diary.Diary method)}

\begin{fulllineitems}
\phantomsection\label{\detokenize{diary:diary.Diary.is_weekend}}\pysiglinewithargsret{\sphinxbfcode{\sphinxupquote{is\_weekend}}}{\emph{day}}{}
This function returns true if a day is in the weekend and false if it’s in a weekday.
\begin{quote}\begin{description}
\item[{Parameters}] \leavevmode
\sphinxstyleliteralstrong{\sphinxupquote{day}} (\sphinxstyleliteralemphasis{\sphinxupquote{numpy.ndarray}}) \textendash{} the day of the weekd

\item[{Returns}] \leavevmode
boolean index of whether or not a day is in the weekend (True) or not (False)

\item[{Return type}] \leavevmode
numpy.ndarray

\end{description}\end{quote}

\end{fulllineitems}

\index{same\_day() (diary.Diary method)}

\begin{fulllineitems}
\phantomsection\label{\detokenize{diary:diary.Diary.same_day}}\pysiglinewithargsret{\sphinxbfcode{\sphinxupquote{same\_day}}}{\emph{start}, \emph{dt}}{}
This function returns true if the activity starts and ends on the same day.
\begin{quote}\begin{description}
\item[{Parameters}] \leavevmode\begin{itemize}
\item {} 
\sphinxstyleliteralstrong{\sphinxupquote{start}} (\sphinxstyleliteralemphasis{\sphinxupquote{numpy.ndarray}}) \textendash{} the time an activity starts {[}hours{]}

\item {} 
\sphinxstyleliteralstrong{\sphinxupquote{dt}} (\sphinxstyleliteralemphasis{\sphinxupquote{numpy.ndarray}}) \textendash{} the duration of an activity, \(\Delta{t}\) {[}hours{]}

\end{itemize}

\item[{Returns}] \leavevmode
a boolean index of whether or not an activity started and ended on the same day

\item[{Return type}] \leavevmode
numpy.ndarray

\end{description}\end{quote}

\end{fulllineitems}

\index{toString() (diary.Diary method)}

\begin{fulllineitems}
\phantomsection\label{\detokenize{diary:diary.Diary.toString}}\pysiglinewithargsret{\sphinxbfcode{\sphinxupquote{toString}}}{}{}
This function expresses the Diary object as a string
\begin{quote}\begin{description}
\item[{Returns}] \leavevmode
an expression of the diary as a string

\item[{Return type}] \leavevmode
string

\end{description}\end{quote}

\end{fulllineitems}


\end{fulllineitems}



\subsection{eat module}
\label{\detokenize{eat::doc}}\label{\detokenize{eat:module-eat}}\label{\detokenize{eat:eat-module}}\index{eat (module)}
This module contains information about the activities associated with eating. This class is an Activity ({\hyperref[\detokenize{activity:activity.Activity}]{\sphinxcrossref{\sphinxcode{\sphinxupquote{activity.Activity}}}}}) that gives a Person ({\hyperref[\detokenize{person:person.Person}]{\sphinxcrossref{\sphinxcode{\sphinxupquote{person.Person}}}}}) the ability to eat and satisfy the need Hunger ({\hyperref[\detokenize{hunger:hunger.Hunger}]{\sphinxcrossref{\sphinxcode{\sphinxupquote{hunger.Hunger}}}}}).

This module contains the following classes:
\begin{enumerate}
\item {} 
{\hyperref[\detokenize{eat:eat.Eat}]{\sphinxcrossref{\sphinxcode{\sphinxupquote{eat.Eat}}}}} (general eating capabilities)

\item {} 
{\hyperref[\detokenize{eat:eat.Eat_Breakfast}]{\sphinxcrossref{\sphinxcode{\sphinxupquote{eat.Eat\_Breakfast}}}}} (eating breakfast)

\item {} 
{\hyperref[\detokenize{eat:eat.Eat_Lunch}]{\sphinxcrossref{\sphinxcode{\sphinxupquote{eat.Eat\_Lunch}}}}} (eating lunch)

\item {} 
{\hyperref[\detokenize{eat:eat.Eat_Dinner}]{\sphinxcrossref{\sphinxcode{\sphinxupquote{eat.Eat\_Dinner}}}}} (eating dinner)

\end{enumerate}
\index{Eat (class in eat)}

\begin{fulllineitems}
\phantomsection\label{\detokenize{eat:eat.Eat}}\pysigline{\sphinxbfcode{\sphinxupquote{class }}\sphinxcode{\sphinxupquote{eat.}}\sphinxbfcode{\sphinxupquote{Eat}}}
Bases: {\hyperref[\detokenize{activity:activity.Activity}]{\sphinxcrossref{\sphinxcode{\sphinxupquote{activity.Activity}}}}}

This class has general capabilities that allow the person to eat in order to satisfy {\hyperref[\detokenize{hunger:hunger.Hunger}]{\sphinxcrossref{\sphinxcode{\sphinxupquote{hunger.Hunger}}}}}.     This class acts as a parent class and is expected to inherited.
\index{advertise() (eat.Eat method)}

\begin{fulllineitems}
\phantomsection\label{\detokenize{eat:eat.Eat.advertise}}\pysiglinewithargsret{\sphinxbfcode{\sphinxupquote{advertise}}}{\emph{p}}{}
This function handles advertising the score to an agent. This function returns 0.

\begin{sphinxadmonition}{note}{Note:}
This function should be overloaded when inherited.
\end{sphinxadmonition}
\begin{quote}\begin{description}
\item[{Parameters}] \leavevmode
\sphinxstyleliteralstrong{\sphinxupquote{p}} ({\hyperref[\detokenize{person:person.Person}]{\sphinxcrossref{\sphinxstyleliteralemphasis{\sphinxupquote{person.Person}}}}}) \textendash{} the person of interest

\item[{Returns}] \leavevmode
the score (0)

\item[{Return type}] \leavevmode
float

\end{description}\end{quote}

\end{fulllineitems}

\index{advertise\_help() (eat.Eat method)}

\begin{fulllineitems}
\phantomsection\label{\detokenize{eat:eat.Eat.advertise_help}}\pysiglinewithargsret{\sphinxbfcode{\sphinxupquote{advertise\_help}}}{\emph{p}, \emph{dt}}{}
This function does some of the logistics needed for {\hyperref[\detokenize{eat:eat.Eat.advertise}]{\sphinxcrossref{\sphinxcode{\sphinxupquote{advertise()}}}}}.

This function does the following:
\begin{enumerate}
\item {} 
sets the suggested recharge rate for hunger

\item {} 
calculates the score

\end{enumerate}
\begin{quote}\begin{description}
\item[{Parameters}] \leavevmode\begin{itemize}
\item {} 
\sphinxstyleliteralstrong{\sphinxupquote{p}} ({\hyperref[\detokenize{person:person.Person}]{\sphinxcrossref{\sphinxstyleliteralemphasis{\sphinxupquote{person.Person}}}}}) \textendash{} the person who is being advertised to

\item {} 
\sphinxstyleliteralstrong{\sphinxupquote{dt}} (\sphinxstyleliteralemphasis{\sphinxupquote{float}}) \textendash{} the duration of the activity

\end{itemize}

\item[{Returns}] \leavevmode
the score

\item[{Return type}] \leavevmode
float

\end{description}\end{quote}

\end{fulllineitems}

\index{advertise\_interruption() (eat.Eat method)}

\begin{fulllineitems}
\phantomsection\label{\detokenize{eat:eat.Eat.advertise_interruption}}\pysiglinewithargsret{\sphinxbfcode{\sphinxupquote{advertise\_interruption}}}{\emph{p}}{}
This function calculates the score of an activity advertisement when a person is going to interrupt an         ongoing activity in order to do an eating activity.

This function does the following:
\begin{enumerate}
\item {} 
temporarily sets the satiation of hunger  \(n_{hunger}(t) = \eta_{interruption}\)

\item {} 
calculate the score advertised for the potential eating activity that will interrupt a current activity

\item {} 
restores the the satiation for hunger to the original value

\end{enumerate}
\begin{quote}\begin{description}
\item[{Parameters}] \leavevmode
\sphinxstyleliteralstrong{\sphinxupquote{p}} ({\hyperref[\detokenize{person:person.Person}]{\sphinxcrossref{\sphinxstyleliteralemphasis{\sphinxupquote{person.Person}}}}}) \textendash{} the person of interest

\item[{Return score}] \leavevmode
the value of the advertisement

\item[{Return type}] \leavevmode
float

\end{description}\end{quote}

\end{fulllineitems}

\index{end() (eat.Eat method)}

\begin{fulllineitems}
\phantomsection\label{\detokenize{eat:eat.Eat.end}}\pysiglinewithargsret{\sphinxbfcode{\sphinxupquote{end}}}{\emph{p}}{}
This function ends the eat activity.
\begin{quote}\begin{description}
\item[{Parameters}] \leavevmode
\sphinxstyleliteralstrong{\sphinxupquote{p}} ({\hyperref[\detokenize{person:person.Person}]{\sphinxcrossref{\sphinxstyleliteralemphasis{\sphinxupquote{person.Person}}}}}) \textendash{} the person whose activity is ending

\item[{Returns}] \leavevmode
None

\end{description}\end{quote}

\end{fulllineitems}

\index{end\_meal() (eat.Eat method)}

\begin{fulllineitems}
\phantomsection\label{\detokenize{eat:eat.Eat.end_meal}}\pysiglinewithargsret{\sphinxbfcode{\sphinxupquote{end\_meal}}}{\emph{p}}{}
This function ends the eat activity by doing the following:
\begin{enumerate}
\item {} 
frees the person’s use of the asset

\item {} 
sets the state to idle (\sphinxcode{\sphinxupquote{state.IDLE}})

\item {} 
sets the satiation of hunger

\item {} 
set the current meal for the next day

\item {} 
set any skipped meals to be on the next day

\item {} 
find the the next meal

\item {} 
sets the decay rate of hunger

\item {} 
update the scheduler so that hunger will trigger the schedule to stop at the next meal

\item {} 
set the next meal to the current meal

\end{enumerate}
\begin{quote}\begin{description}
\item[{Parameters}] \leavevmode
\sphinxstyleliteralstrong{\sphinxupquote{p}} ({\hyperref[\detokenize{person:person.Person}]{\sphinxcrossref{\sphinxstyleliteralemphasis{\sphinxupquote{person.Person}}}}}) \textendash{} The person whose meal is ending.

\item[{Returns}] \leavevmode
None

\end{description}\end{quote}

\end{fulllineitems}

\index{set\_end\_time() (eat.Eat method)}

\begin{fulllineitems}
\phantomsection\label{\detokenize{eat:eat.Eat.set_end_time}}\pysiglinewithargsret{\sphinxbfcode{\sphinxupquote{set\_end\_time}}}{\emph{p}}{}
This function returns the end time of eating (universal time).
\begin{quote}\begin{description}
\item[{Parameters}] \leavevmode
\sphinxstyleliteralstrong{\sphinxupquote{p}} ({\hyperref[\detokenize{person:person.Person}]{\sphinxcrossref{\sphinxstyleliteralemphasis{\sphinxupquote{person.Person}}}}}) \textendash{} the person of interest.

\item[{Return t\_end}] \leavevmode
the end time of eating {[}minutes, universal time{]}

\item[{Return type}] \leavevmode
int

\end{description}\end{quote}

\end{fulllineitems}

\index{start() (eat.Eat method)}

\begin{fulllineitems}
\phantomsection\label{\detokenize{eat:eat.Eat.start}}\pysiglinewithargsret{\sphinxbfcode{\sphinxupquote{start}}}{\emph{p}}{}
This function starts the eating activity.
\begin{quote}\begin{description}
\item[{Parameters}] \leavevmode
\sphinxstyleliteralstrong{\sphinxupquote{p}} ({\hyperref[\detokenize{person:person.Person}]{\sphinxcrossref{\sphinxstyleliteralemphasis{\sphinxupquote{person.Person}}}}}) \textendash{} The person whose activity is starting.

\item[{Returns}] \leavevmode
None

\end{description}\end{quote}

\end{fulllineitems}

\index{start\_meal() (eat.Eat method)}

\begin{fulllineitems}
\phantomsection\label{\detokenize{eat:eat.Eat.start_meal}}\pysiglinewithargsret{\sphinxbfcode{\sphinxupquote{start\_meal}}}{\emph{p}}{}
This function starts the eat activity by doing the following:
\begin{enumerate}
\item {} 
sets the person’s state to busy (\sphinxcode{\sphinxupquote{state.BUSY}})

\item {} 
set the decay rate of hunger to 0

\item {} 
store the start time to the state

\item {} 
sets the end time

\item {} 
sets the hunger recharge rate

\item {} 
updates the asset’s state and number of users

\item {} 
update the schedule for the hunger need to trigger when the eat activity is scheduled to end

\end{enumerate}
\begin{quote}\begin{description}
\item[{Parameters}] \leavevmode
\sphinxstyleliteralstrong{\sphinxupquote{p}} ({\hyperref[\detokenize{person:person.Person}]{\sphinxcrossref{\sphinxstyleliteralemphasis{\sphinxupquote{person.Person}}}}}) \textendash{} the person who is starting the meal

\item[{Returns}] \leavevmode
None

\end{description}\end{quote}

\end{fulllineitems}

\index{test\_func() (eat.Eat method)}

\begin{fulllineitems}
\phantomsection\label{\detokenize{eat:eat.Eat.test_func}}\pysiglinewithargsret{\sphinxbfcode{\sphinxupquote{test\_func}}}{\emph{p}}{}~
\begin{sphinxadmonition}{note}{Note:}
This function is for debugging and has no practical purpose. This function will be             removed in the future.
\end{sphinxadmonition}
\begin{quote}\begin{description}
\item[{Parameters}] \leavevmode
\sphinxstyleliteralstrong{\sphinxupquote{p}} ({\hyperref[\detokenize{person:person.Person}]{\sphinxcrossref{\sphinxstyleliteralemphasis{\sphinxupquote{person.Person}}}}}) \textendash{} person of interest

\item[{Returns}] \leavevmode
None

\end{description}\end{quote}

\end{fulllineitems}


\end{fulllineitems}

\index{Eat\_Breakfast (class in eat)}

\begin{fulllineitems}
\phantomsection\label{\detokenize{eat:eat.Eat_Breakfast}}\pysigline{\sphinxbfcode{\sphinxupquote{class }}\sphinxcode{\sphinxupquote{eat.}}\sphinxbfcode{\sphinxupquote{Eat\_Breakfast}}}
Bases: {\hyperref[\detokenize{eat:eat.Eat}]{\sphinxcrossref{\sphinxcode{\sphinxupquote{eat.Eat}}}}}

This class is used to handle the logistics for eating breakfast.
\index{advertise() (eat.Eat\_Breakfast method)}

\begin{fulllineitems}
\phantomsection\label{\detokenize{eat:eat.Eat_Breakfast.advertise}}\pysiglinewithargsret{\sphinxbfcode{\sphinxupquote{advertise}}}{\emph{p}}{}
This function calculates the score of the activity’s advertisement to a person. The activity         advertise to the agent if the following conditions are met:
\begin{enumerate}
\item {} 
the current meal is breakfast

\item {} 
the agent’s location is at home (\sphinxcode{\sphinxupquote{location.HOME}})

\item {} 
calculate the score \(S\)

\end{enumerate}
\begin{equation*}
\begin{split}S = \begin{cases}
0  & n_{hunger}(t) > \lambda \\
W( n_{hunger}(t) ) - W( n_{hunger}(t + \Delta{t} )) & n_{hunger}(t) \le \lambda
\end{cases}\end{split}
\end{equation*}\begin{description}
\item[{where}] \leavevmode\begin{itemize}
\item {} 
\(t\) is the current time

\item {} 
\(\Delta{t}\) is the duration of eating breakfast {[}minutes{]}

\item {} 
\(n_{hunger}(t)\) is the satiation for Hunger at time \(t\)

\item {} 
\(\lambda\) is the threshold value of Hunger

\item {} 
\(W(n)\) is the weight function for Hunger

\end{itemize}

\end{description}
\begin{quote}\begin{description}
\item[{Parameters}] \leavevmode
\sphinxstyleliteralstrong{\sphinxupquote{p}} ({\hyperref[\detokenize{person:person.Person}]{\sphinxcrossref{\sphinxstyleliteralemphasis{\sphinxupquote{person.Person}}}}}) \textendash{} the person of interest

\item[{Return score}] \leavevmode
the advertised score of doing the eat breakfast activity

\item[{Return type}] \leavevmode
float

\end{description}\end{quote}

\end{fulllineitems}

\index{end\_meal() (eat.Eat\_Breakfast method)}

\begin{fulllineitems}
\phantomsection\label{\detokenize{eat:eat.Eat_Breakfast.end_meal}}\pysiglinewithargsret{\sphinxbfcode{\sphinxupquote{end\_meal}}}{\emph{p}}{}
This function handles the logistics for ending the eat breakfast activity by doing the following:
\begin{enumerate}
\item {} 
call \sphinxcode{\sphinxupquote{eat.end\_meal()}}

\item {} 
if planning to skip lunch, update the lunch event to be the next day

\end{enumerate}
\begin{quote}\begin{description}
\item[{Parameters}] \leavevmode
\sphinxstyleliteralstrong{\sphinxupquote{p}} ({\hyperref[\detokenize{person:person.Person}]{\sphinxcrossref{\sphinxstyleliteralemphasis{\sphinxupquote{person.Person}}}}}) \textendash{} the person who’s meal is ending

\item[{Returns}] \leavevmode


\end{description}\end{quote}

\end{fulllineitems}

\index{start\_meal() (eat.Eat\_Breakfast method)}

\begin{fulllineitems}
\phantomsection\label{\detokenize{eat:eat.Eat_Breakfast.start_meal}}\pysiglinewithargsret{\sphinxbfcode{\sphinxupquote{start\_meal}}}{\emph{p}}{}
This function handles the logistics for starting the eat activity by doing the following:
\begin{enumerate}
\item {} 
set the current meal to breakfast

\item {} 
call \sphinxcode{\sphinxupquote{eat.start\_meal()}}

\end{enumerate}
\begin{quote}\begin{description}
\item[{Parameters}] \leavevmode
\sphinxstyleliteralstrong{\sphinxupquote{p}} ({\hyperref[\detokenize{person:person.Person}]{\sphinxcrossref{\sphinxstyleliteralemphasis{\sphinxupquote{person.Person}}}}}) \textendash{} the person who is starting the eat activity

\item[{Returns}] \leavevmode


\end{description}\end{quote}

\end{fulllineitems}


\end{fulllineitems}

\index{Eat\_Dinner (class in eat)}

\begin{fulllineitems}
\phantomsection\label{\detokenize{eat:eat.Eat_Dinner}}\pysigline{\sphinxbfcode{\sphinxupquote{class }}\sphinxcode{\sphinxupquote{eat.}}\sphinxbfcode{\sphinxupquote{Eat\_Dinner}}}
Bases: {\hyperref[\detokenize{eat:eat.Eat}]{\sphinxcrossref{\sphinxcode{\sphinxupquote{eat.Eat}}}}}

This class is used to handle the logistics for eating dinner.
\index{advertise() (eat.Eat\_Dinner method)}

\begin{fulllineitems}
\phantomsection\label{\detokenize{eat:eat.Eat_Dinner.advertise}}\pysiglinewithargsret{\sphinxbfcode{\sphinxupquote{advertise}}}{\emph{p}}{}
This function calculates the score of an activities advertisement to a Person. This activity         advertises to the agent if the following conditions are met
\begin{enumerate}
\item {} 
the current meal is lunch

\item {} 
the agent’s location is at home (\sphinxcode{\sphinxupquote{location.HOME}})

\item {} 
calculate the score \(S\)

\end{enumerate}
\begin{equation*}
\begin{split}S = \begin{cases}
0  & n_{hunger}(t) > \lambda \\
W( n_{hunger}(t) ) - W( n_{hunger}(t + \Delta{t} )) & n_{hunger}(t) \le \lambda
\end{cases}\end{split}
\end{equation*}\begin{description}
\item[{where}] \leavevmode\begin{itemize}
\item {} 
\(t\) is the current time

\item {} 
\(\Delta{t}\) is the duration of eating dinner {[}minutes{]}

\item {} 
\(n_{hunger}(t)\) is the satiation for Hunger at time \(t\)

\item {} 
\(\lambda\) is the threshold value of Hunger

\item {} 
\(W(n)\) is the weight function for Hunger

\end{itemize}

\end{description}

If the threshold is not met, score is 0. The advertisements assume that the duration         of the activity is the mean duration.
\begin{quote}\begin{description}
\item[{Parameters}] \leavevmode
\sphinxstyleliteralstrong{\sphinxupquote{p}} ({\hyperref[\detokenize{person:person.Person}]{\sphinxcrossref{\sphinxstyleliteralemphasis{\sphinxupquote{person.Person}}}}}) \textendash{} the person of interest

\item[{Return score}] \leavevmode
the advertised score of doing the eat dinner activity

\item[{Return type}] \leavevmode
float

\end{description}\end{quote}

\end{fulllineitems}

\index{end\_meal() (eat.Eat\_Dinner method)}

\begin{fulllineitems}
\phantomsection\label{\detokenize{eat:eat.Eat_Dinner.end_meal}}\pysiglinewithargsret{\sphinxbfcode{\sphinxupquote{end\_meal}}}{\emph{p}}{}
This function goes through the logistics of ending the dinner meal by doing the following:
\begin{enumerate}
\item {} 
calls \sphinxcode{\sphinxupquote{end.end\_meal()}}

\item {} 
if breakfast will be skipped, update the lunch event to be 2 days from the current day

\end{enumerate}
\begin{quote}\begin{description}
\item[{Parameters}] \leavevmode
\sphinxstyleliteralstrong{\sphinxupquote{p}} ({\hyperref[\detokenize{person:person.Person}]{\sphinxcrossref{\sphinxstyleliteralemphasis{\sphinxupquote{person.Person}}}}}) \textendash{} the person who is finishing the eating dinner event

\item[{Returns}] \leavevmode
None

\end{description}\end{quote}

\end{fulllineitems}

\index{start\_meal() (eat.Eat\_Dinner method)}

\begin{fulllineitems}
\phantomsection\label{\detokenize{eat:eat.Eat_Dinner.start_meal}}\pysiglinewithargsret{\sphinxbfcode{\sphinxupquote{start\_meal}}}{\emph{p}}{}
This function goes through the logistics of starting the dinner meal by doing the following:
\begin{enumerate}
\item {} 
set the current meal to dinner

\item {} 
call \sphinxcode{\sphinxupquote{eat.start\_meal()}}

\end{enumerate}
\begin{quote}\begin{description}
\item[{Parameters}] \leavevmode
\sphinxstyleliteralstrong{\sphinxupquote{p}} ({\hyperref[\detokenize{person:person.Person}]{\sphinxcrossref{\sphinxstyleliteralemphasis{\sphinxupquote{person.Person}}}}}) \textendash{} the person who is starting the eat dinner event

\item[{Returns}] \leavevmode
None

\end{description}\end{quote}

\end{fulllineitems}


\end{fulllineitems}

\index{Eat\_Lunch (class in eat)}

\begin{fulllineitems}
\phantomsection\label{\detokenize{eat:eat.Eat_Lunch}}\pysigline{\sphinxbfcode{\sphinxupquote{class }}\sphinxcode{\sphinxupquote{eat.}}\sphinxbfcode{\sphinxupquote{Eat\_Lunch}}}
Bases: {\hyperref[\detokenize{eat:eat.Eat}]{\sphinxcrossref{\sphinxcode{\sphinxupquote{eat.Eat}}}}}

This class is used to handle the logistics for eating lunch.
\index{advertise() (eat.Eat\_Lunch method)}

\begin{fulllineitems}
\phantomsection\label{\detokenize{eat:eat.Eat_Lunch.advertise}}\pysiglinewithargsret{\sphinxbfcode{\sphinxupquote{advertise}}}{\emph{p}}{}
This function calculates the score of an activities advertisement to a person. The activity         advertise to the agent if the following conditions are met:
\begin{enumerate}
\item {} 
the current meal is lunch

\item {} 
the agent’s location is at home (\sphinxcode{\sphinxupquote{location.HOME}}) or the agent’s location is at the         workplace (\sphinxcode{\sphinxupquote{location.OFF\_SITE}})

\item {} 
calculate the score \(S\)

\end{enumerate}
\begin{equation*}
\begin{split}S = \begin{cases}
0  & n_{hunger}(t) > \lambda \\
W( n_{hunger}(t) ) - W( n_{hunger}(t + \Delta{t} )) & n_{hunger}(t) \le \lambda
\end{cases}\end{split}
\end{equation*}\begin{description}
\item[{where}] \leavevmode\begin{itemize}
\item {} 
\(t\) is the current time

\item {} 
\(\Delta{t}\) is the duration of eating lunch {[}minutes{]}

\item {} 
\(n_{hunger}(t)\) is the satiation for Hunger at time \(t\)

\item {} 
\(\lambda\) is the threshold value of Hunger

\item {} 
\(W(n)\) is the weight function for Hunger

\end{itemize}

\end{description}

If the threshold is not met, score is 0. The advertisements assume that the duration         of the activity is the mean duration.
\begin{quote}\begin{description}
\item[{Parameters}] \leavevmode
\sphinxstyleliteralstrong{\sphinxupquote{p}} ({\hyperref[\detokenize{person:person.Person}]{\sphinxcrossref{\sphinxstyleliteralemphasis{\sphinxupquote{person.Person}}}}}) \textendash{} the person of interest

\item[{Return score}] \leavevmode
the advertised score of doing the eat lunch activity

\item[{Return type}] \leavevmode
float

\end{description}\end{quote}

\end{fulllineitems}

\index{end\_meal() (eat.Eat\_Lunch method)}

\begin{fulllineitems}
\phantomsection\label{\detokenize{eat:eat.Eat_Lunch.end_meal}}\pysiglinewithargsret{\sphinxbfcode{\sphinxupquote{end\_meal}}}{\emph{p}}{}
This function ends the eat lunch activity by doing the following:
\begin{enumerate}
\item {} 
calls \sphinxcode{\sphinxupquote{eat.end\_meal()}}

\item {} 
if dinner is to be skipped, update the dinner event by doing the following:
\begin{itemize}
\item {} \begin{description}
\item[{if the lunch is an interrupting activity}] \leavevmode\begin{itemize}
\item {} 
set the time until the next lunch activity

\item {} 
update the schedule for the interruption to the next lunch activity

\item {} 
set the interruption state to False

\end{itemize}

\end{description}

\end{itemize}

\end{enumerate}
\begin{quote}\begin{description}
\item[{Parameters}] \leavevmode
\sphinxstyleliteralstrong{\sphinxupquote{p}} ({\hyperref[\detokenize{person:person.Person}]{\sphinxcrossref{\sphinxstyleliteralemphasis{\sphinxupquote{person.Person}}}}}) \textendash{} The person whose meal is ending.

\item[{Returns}] \leavevmode
None

\end{description}\end{quote}

\end{fulllineitems}

\index{start\_meal() (eat.Eat\_Lunch method)}

\begin{fulllineitems}
\phantomsection\label{\detokenize{eat:eat.Eat_Lunch.start_meal}}\pysiglinewithargsret{\sphinxbfcode{\sphinxupquote{start\_meal}}}{\emph{p}}{}
This function handles the logistics for starting the eat activity by doing the following:
\begin{enumerate}
\item {} 
sets the current meal to lunch

\item {} 
call \sphinxcode{\sphinxupquote{eat.start\_meal()}}

\end{enumerate}
\begin{quote}\begin{description}
\item[{Parameters}] \leavevmode
\sphinxstyleliteralstrong{\sphinxupquote{p}} ({\hyperref[\detokenize{person:person.Person}]{\sphinxcrossref{\sphinxstyleliteralemphasis{\sphinxupquote{person.Person}}}}}) \textendash{} the person starting the eat lunch event

\item[{Returns}] \leavevmode
None

\end{description}\end{quote}

\end{fulllineitems}


\end{fulllineitems}



\subsection{food module}
\label{\detokenize{food::doc}}\label{\detokenize{food:food-module}}\label{\detokenize{food:module-food}}\index{food (module)}
This module contains information about the asset that allows for the eating activity.

This module contains the following class: {\hyperref[\detokenize{food:food.Food}]{\sphinxcrossref{\sphinxcode{\sphinxupquote{food.Food}}}}}.
\index{Food (class in food)}

\begin{fulllineitems}
\phantomsection\label{\detokenize{food:food.Food}}\pysigline{\sphinxbfcode{\sphinxupquote{class }}\sphinxcode{\sphinxupquote{food.}}\sphinxbfcode{\sphinxupquote{Food}}}
Bases: {\hyperref[\detokenize{asset:asset.Asset}]{\sphinxcrossref{\sphinxcode{\sphinxupquote{asset.Asset}}}}}

This class represents an asset that allows the agent to eat breakfast, eat lunch, and eat dinner.

Activities in this asset:
\begin{enumerate}
\item {} 
{\hyperref[\detokenize{eat:eat.Eat_Breakfast}]{\sphinxcrossref{\sphinxcode{\sphinxupquote{eat.Eat\_Breakfast}}}}}

\item {} 
{\hyperref[\detokenize{eat:eat.Eat_Lunch}]{\sphinxcrossref{\sphinxcode{\sphinxupquote{eat.Eat\_Lunch}}}}}

\item {} 
{\hyperref[\detokenize{eat:eat.Eat_Dinner}]{\sphinxcrossref{\sphinxcode{\sphinxupquote{eat.Eat\_Dinner}}}}}

\end{enumerate}

\end{fulllineitems}



\subsection{home module}
\label{\detokenize{home::doc}}\label{\detokenize{home:home-module}}\label{\detokenize{home:module-home}}\index{home (module)}
This module governs the control of assets used in the simulation. Mainly, the home contains all of the assets used in the simulation for the current version of the code.

This module contains the following class: {\hyperref[\detokenize{home:home.Home}]{\sphinxcrossref{\sphinxcode{\sphinxupquote{home.Home}}}}}
\index{Home (class in home)}

\begin{fulllineitems}
\phantomsection\label{\detokenize{home:home.Home}}\pysiglinewithargsret{\sphinxbfcode{\sphinxupquote{class }}\sphinxcode{\sphinxupquote{home.}}\sphinxbfcode{\sphinxupquote{Home}}}{\emph{clock}}{}
Bases: \sphinxcode{\sphinxupquote{object}}

Contains all of the physical characteristics of a home/ residence. Currently, the home     contains all of the assets within the simulation.
\begin{quote}\begin{description}
\item[{Parameters}] \leavevmode
\sphinxstyleliteralstrong{\sphinxupquote{clock}} ({\hyperref[\detokenize{temporal:temporal.Temporal}]{\sphinxcrossref{\sphinxstyleliteralemphasis{\sphinxupquote{temporal.Temporal}}}}}) \textendash{} the time

\item[{Variables}] \leavevmode\begin{itemize}
\item {} 
\sphinxstyleliteralstrong{\sphinxupquote{assets}} (\sphinxstyleliteralemphasis{\sphinxupquote{dict}}) \textendash{} contains a list of all of the assets available in the home.

\item {} 
\sphinxstyleliteralstrong{\sphinxupquote{category}} (\sphinxstyleliteralemphasis{\sphinxupquote{int}}) \textendash{} the type of home

\item {} 
\sphinxstyleliteralstrong{\sphinxupquote{clock}} ({\hyperref[\detokenize{temporal:temporal.Temporal}]{\sphinxcrossref{\sphinxstyleliteralemphasis{\sphinxupquote{temporal.Temporal}}}}}) \textendash{} the time

\item {} 
\sphinxstyleliteralstrong{\sphinxupquote{id}} (\sphinxstyleliteralemphasis{\sphinxupquote{int}}) \textendash{} a unique home identification number

\item {} 
\sphinxstyleliteralstrong{\sphinxupquote{'location'}} ({\hyperref[\detokenize{location:location.Location}]{\sphinxcrossref{\sphinxstyleliteralemphasis{\sphinxupquote{location.Location}}}}}) \textendash{} the location of the home

\item {} 
\sphinxstyleliteralstrong{\sphinxupquote{population}} (\sphinxstyleliteralemphasis{\sphinxupquote{int}}) \textendash{} the number of people who reside in a home

\item {} 
\sphinxstyleliteralstrong{\sphinxupquote{revenue}} (\sphinxstyleliteralemphasis{\sphinxupquote{float}}) \textendash{} the household revenue

\end{itemize}

\end{description}\end{quote}
\index{advertise() (home.Home method)}

\begin{fulllineitems}
\phantomsection\label{\detokenize{home:home.Home.advertise}}\pysiglinewithargsret{\sphinxbfcode{\sphinxupquote{advertise}}}{\emph{p}, \emph{do\_interruption=False}, \emph{locale=None}}{}
This function handles all of the activities’ advertisements to a person. This occurs by looping         through each asset in the home and collecting a list of advertisements for each activity in each         asset. Specifically, the function does the following:
\begin{enumerate}
\item {} 
loop through each asset

\item {} 
if the asset is busy \sphinxstyleemphasis{and} is in the same location of the person
\begin{itemize}
\item {} \begin{description}
\item[{for each activity in the given asset}] \leavevmode\begin{enumerate}
\item {} 
advertise for interrupting activities

\item {} 
advertise for non interrupting activities

\item {} 
collect the advertisements

\end{enumerate}

\end{description}

\end{itemize}

\end{enumerate}
\begin{quote}\begin{description}
\item[{Parameters}] \leavevmode\begin{itemize}
\item {} 
\sphinxstyleliteralstrong{\sphinxupquote{p}} ({\hyperref[\detokenize{person:person.Person}]{\sphinxcrossref{\sphinxstyleliteralemphasis{\sphinxupquote{person.Person}}}}}) \textendash{} a person to whom the assets are advertising

\item {} 
\sphinxstyleliteralstrong{\sphinxupquote{do\_interruption}} (\sphinxstyleliteralemphasis{\sphinxupquote{bool}}) \textendash{} a flag that indicates whether or not we should advertise for interruptions

\item {} 
\sphinxstyleliteralstrong{\sphinxupquote{locale}} (\sphinxstyleliteralemphasis{\sphinxupquote{int}}) \textendash{} a local location identifier

\end{itemize}

\item[{Returns}] \leavevmode
the advertisements (score, asset, activity, person) containing the various data for                         each advertisement: (“score”, “asset”, “activity”, “person”) coupling of                         data type (float, {\hyperref[\detokenize{asset:asset.Asset}]{\sphinxcrossref{\sphinxcode{\sphinxupquote{asset.Asset}}}}}, {\hyperref[\detokenize{activity:activity.Activity}]{\sphinxcrossref{\sphinxcode{\sphinxupquote{activity.Activity}}}}},                         {\hyperref[\detokenize{person:person.Person}]{\sphinxcrossref{\sphinxcode{\sphinxupquote{person.Person}}}}})

\item[{Return type}] \leavevmode
dict

\end{description}\end{quote}

\end{fulllineitems}

\index{initialize() (home.Home method)}

\begin{fulllineitems}
\phantomsection\label{\detokenize{home:home.Home.initialize}}\pysiglinewithargsret{\sphinxbfcode{\sphinxupquote{initialize}}}{\emph{people}}{}
Initialize the assets in the home.
\begin{quote}\begin{description}
\item[{Parameters}] \leavevmode
\sphinxstyleliteralstrong{\sphinxupquote{people}} (\sphinxstyleliteralemphasis{\sphinxupquote{list}}) \textendash{} a list of people who reside in the home

\item[{Returns}] \leavevmode
None

\end{description}\end{quote}

\end{fulllineitems}

\index{print\_category() (home.Home method)}

\begin{fulllineitems}
\phantomsection\label{\detokenize{home:home.Home.print_category}}\pysiglinewithargsret{\sphinxbfcode{\sphinxupquote{print\_category}}}{}{}
This function expresses the category variable as a string.
\begin{quote}\begin{description}
\item[{Returns}] \leavevmode
string representation of the category

\item[{Return type}] \leavevmode
str

\end{description}\end{quote}

\end{fulllineitems}

\index{reset() (home.Home method)}

\begin{fulllineitems}
\phantomsection\label{\detokenize{home:home.Home.reset}}\pysiglinewithargsret{\sphinxbfcode{\sphinxupquote{reset}}}{}{}
This function resets the each asset in the home.
\begin{quote}\begin{description}
\item[{Returns}] \leavevmode
None

\end{description}\end{quote}

\end{fulllineitems}

\index{set\_population() (home.Home method)}

\begin{fulllineitems}
\phantomsection\label{\detokenize{home:home.Home.set_population}}\pysiglinewithargsret{\sphinxbfcode{\sphinxupquote{set\_population}}}{\emph{people}}{}
Set the population of the house.
\begin{quote}\begin{description}
\item[{Parameters}] \leavevmode
\sphinxstyleliteralstrong{\sphinxupquote{people}} (\sphinxstyleliteralemphasis{\sphinxupquote{list}}) \textendash{} the list of people living in the home

\item[{Returns}] \leavevmode
None

\end{description}\end{quote}

\end{fulllineitems}

\index{set\_revenue() (home.Home method)}

\begin{fulllineitems}
\phantomsection\label{\detokenize{home:home.Home.set_revenue}}\pysiglinewithargsret{\sphinxbfcode{\sphinxupquote{set\_revenue}}}{\emph{people}}{}
Sets the revenue of the home by adding the revenue of each person in the home.
\begin{quote}\begin{description}
\item[{Parameters}] \leavevmode
\sphinxstyleliteralstrong{\sphinxupquote{people}} (\sphinxstyleliteralemphasis{\sphinxupquote{list}}) \textendash{} the list of people living in the home

\item[{Returns}] \leavevmode
None

\end{description}\end{quote}

\end{fulllineitems}

\index{toString() (home.Home method)}

\begin{fulllineitems}
\phantomsection\label{\detokenize{home:home.Home.toString}}\pysiglinewithargsret{\sphinxbfcode{\sphinxupquote{toString}}}{}{}
This function expresses the Home object as a string
\begin{quote}\begin{description}
\item[{Return msg}] \leavevmode
the string representation of the home object

\item[{Return type}] \leavevmode
str

\end{description}\end{quote}

\end{fulllineitems}


\end{fulllineitems}



\subsection{hunger module}
\label{\detokenize{hunger::doc}}\label{\detokenize{hunger:hunger-module}}\label{\detokenize{hunger:module-hunger}}\index{hunger (module)}
This module contains information about governing the need Hunger.

This module contains the class Hunger ({\hyperref[\detokenize{hunger:hunger.Hunger}]{\sphinxcrossref{\sphinxcode{\sphinxupquote{hunger.Hunger}}}}}).
\index{Hunger (class in hunger)}

\begin{fulllineitems}
\phantomsection\label{\detokenize{hunger:hunger.Hunger}}\pysiglinewithargsret{\sphinxbfcode{\sphinxupquote{class }}\sphinxcode{\sphinxupquote{hunger.}}\sphinxbfcode{\sphinxupquote{Hunger}}}{\emph{clock}, \emph{num\_sample\_points}}{}
Bases: {\hyperref[\detokenize{need:need.Need}]{\sphinxcrossref{\sphinxcode{\sphinxupquote{need.Need}}}}}

This class governs the behavior of the need Hunger need. When Hunger is unstatisfied,
the agent feels compelled to eat a meal in order to satisfy the need. Mathematically     speaking, Hunger is modeled as linear-behaving need.
\begin{quote}\begin{description}
\item[{Parameters}] \leavevmode\begin{itemize}
\item {} 
\sphinxstyleliteralstrong{\sphinxupquote{clock}} ({\hyperref[\detokenize{temporal:temporal.Temporal}]{\sphinxcrossref{\sphinxstyleliteralemphasis{\sphinxupquote{temporal.Temporal}}}}}) \textendash{} the time

\item {} 
\sphinxstyleliteralstrong{\sphinxupquote{num\_sample\_points}} (\sphinxstyleliteralemphasis{\sphinxupquote{int}}) \textendash{} the number of temporal nodes in the simulation

\end{itemize}

\item[{Variables}] \leavevmode\begin{itemize}
\item {} 
\sphinxstyleliteralstrong{\sphinxupquote{category}} (\sphinxstyleliteralemphasis{\sphinxupquote{int}}) \textendash{} the category of the need

\item {} 
\sphinxstyleliteralstrong{\sphinxupquote{decay\_rate}} (\sphinxstyleliteralemphasis{\sphinxupquote{float}}) \textendash{} the decay rate of the Hunger need {[}need/minute{]}

\item {} 
\sphinxstyleliteralstrong{\sphinxupquote{recharge\_rate}} (\sphinxstyleliteralemphasis{\sphinxupquote{float}}) \textendash{} the recharge rate of the Hunger need {[}need/min{]}

\item {} 
\sphinxstyleliteralstrong{\sphinxupquote{suggested\_recharge\_rate}} (\sphinxstyleliteralemphasis{\sphinxupquote{float}}) \textendash{} an approximate recharge rate used to calculate the end time of an     event before rounding

\end{itemize}

\end{description}\end{quote}
\index{decay() (hunger.Hunger method)}

\begin{fulllineitems}
\phantomsection\label{\detokenize{hunger:hunger.Hunger.decay}}\pysiglinewithargsret{\sphinxbfcode{\sphinxupquote{decay}}}{\emph{status}}{}
This function decreases the satiation in Hunger by doing the following:
\begin{equation*}
\begin{split}n(t + 1) = n(t) + m_{decay}                        \end{split}
\end{equation*}
\begin{sphinxadmonition}{warning}{Warning:}
This function may be antiquated and \sphinxstylestrong{not used}
\end{sphinxadmonition}
\begin{quote}\begin{description}
\item[{Parameters}] \leavevmode
\sphinxstyleliteralstrong{\sphinxupquote{status}} (\sphinxstyleliteralemphasis{\sphinxupquote{int}}) \textendash{} indicates the current status of the person’s state (not-used)

\item[{Returns}] \leavevmode
None

\end{description}\end{quote}

\end{fulllineitems}

\index{decay\_new() (hunger.Hunger method)}

\begin{fulllineitems}
\phantomsection\label{\detokenize{hunger:hunger.Hunger.decay_new}}\pysiglinewithargsret{\sphinxbfcode{\sphinxupquote{decay\_new}}}{\emph{dt}}{}
This function sets the default decrease in the Hunger need.
\begin{equation*}
\begin{split}n(t + \Delta{t}) = n(t) + m_{decay}\,\Delta{t}\end{split}
\end{equation*}\begin{description}
\item[{where}] \leavevmode\begin{itemize}
\item {} 
\(t\) is the current time

\item {} 
\(\Delta{t}\) is the duration of time to decay the satiation {[}minutes{]}

\item {} 
\(n(t)\) is the satiation for Hunger at time \(t\)

\item {} 
\(m_{decay}\) the decay rate for Hunger

\end{itemize}

\end{description}
\begin{quote}\begin{description}
\item[{Parameters}] \leavevmode
\sphinxstyleliteralstrong{\sphinxupquote{dt}} (\sphinxstyleliteralemphasis{\sphinxupquote{int}}) \textendash{} the duration of time {[}minutes{]} \(\Delta{t}\) used to decay the need

\item[{Returns}] \leavevmode
None

\end{description}\end{quote}

\end{fulllineitems}

\index{initialize() (hunger.Hunger method)}

\begin{fulllineitems}
\phantomsection\label{\detokenize{hunger:hunger.Hunger.initialize}}\pysiglinewithargsret{\sphinxbfcode{\sphinxupquote{initialize}}}{\emph{p}}{}
This function initializes the the Hunger need at the first step of the simulation. The function         checks to see whether or not the current time implies that there should be an eating event. The         Hunger object is set to the respective state.

This function does the following exactly:
\begin{enumerate}
\item {} 
initialize all of the meals

\item {} 
check to see if a meal should be occurring at the current time

\item {} 
if no meals should be occurring
\begin{itemize}
\item {} 
figure out the next meal

\item {} 
calculate the decay rate for hunger until the next meal

\item {} 
calculate the amount of time until the next meal \(\Delta{t}\)

\item {} 
set the current meal

\item {} 
update the schedule for the hunger need to be the time the next meal starts

\end{itemize}

\item {} 
if a meal should be occurring
\begin{itemize}
\item {} 
get the index of the meal that should be occurring

\item {} 
set the current meal

\item {} 
calculate the final time of the meal

\item {} 
calculate the duration until the end of the next meal \(\Delta{t}\)

\item {} 
set the recharge rate

\item {} 
update the scheduler for the hunger need to be the time the current meal should end

\end{itemize}

\item {} 
initialize the start time for each meal

\end{enumerate}
\begin{quote}\begin{description}
\item[{Parameters}] \leavevmode
\sphinxstyleliteralstrong{\sphinxupquote{p}} ({\hyperref[\detokenize{person:person.Person}]{\sphinxcrossref{\sphinxstyleliteralemphasis{\sphinxupquote{person.Person}}}}}) \textendash{} the person whose hunger need is being initialized

\item[{Returns}] \leavevmode
None

\end{description}\end{quote}

\end{fulllineitems}

\index{is\_meal\_time() (hunger.Hunger method)}

\begin{fulllineitems}
\phantomsection\label{\detokenize{hunger:hunger.Hunger.is_meal_time}}\pysiglinewithargsret{\sphinxbfcode{\sphinxupquote{is\_meal\_time}}}{\emph{t}, \emph{the\_meal}}{}
This checks whether or not it is time for a meal.
\begin{quote}\begin{description}
\item[{Parameters}] \leavevmode\begin{itemize}
\item {} 
\sphinxstyleliteralstrong{\sphinxupquote{t}} (\sphinxstyleliteralemphasis{\sphinxupquote{int}}) \textendash{} time of day {[}minutes{]}

\item {} 
\sphinxstyleliteralstrong{\sphinxupquote{the\_meal}} ({\hyperref[\detokenize{meal:meal.Meal}]{\sphinxcrossref{\sphinxstyleliteralemphasis{\sphinxupquote{meal.Meal}}}}}) \textendash{} the respective meal to see whether the current time implies                             that an eating event should happen

\end{itemize}

\item[{Returns}] \leavevmode
True if the current time is within the time to eat. False, otherwise

\item[{Return type}] \leavevmode
bool

\end{description}\end{quote}

\end{fulllineitems}

\index{is\_meal\_time\_all() (hunger.Hunger method)}

\begin{fulllineitems}
\phantomsection\label{\detokenize{hunger:hunger.Hunger.is_meal_time_all}}\pysiglinewithargsret{\sphinxbfcode{\sphinxupquote{is\_meal\_time\_all}}}{\emph{t}, \emph{meals}}{}
This function checks every meal and sees whether or not the current time         implies that there should be an eventing event for a respective meal.
\begin{quote}\begin{description}
\item[{Parameters}] \leavevmode\begin{itemize}
\item {} 
\sphinxstyleliteralstrong{\sphinxupquote{t}} (\sphinxstyleliteralemphasis{\sphinxupquote{int}}) \textendash{} the current time of day {[}minutes{]}

\item {} 
\sphinxstyleliteralstrong{\sphinxupquote{meals}} (\sphinxstyleliteralemphasis{\sphinxupquote{list}}) \textendash{} a list of meals that a person has

\end{itemize}

\item[{Returns}] \leavevmode
a list of boolean flags indicating True or False, indicating whether or not an         eating event should occur for the respective meal

\item[{Return type}] \leavevmode
list

\end{description}\end{quote}

\end{fulllineitems}

\index{perceive() (hunger.Hunger method)}

\begin{fulllineitems}
\phantomsection\label{\detokenize{hunger:hunger.Hunger.perceive}}\pysiglinewithargsret{\sphinxbfcode{\sphinxupquote{perceive}}}{\emph{future\_clock}}{}
This gives the result if eat is done now until a later time corresponding to clock.
\begin{quote}\begin{description}
\item[{Parameters}] \leavevmode
\sphinxstyleliteralstrong{\sphinxupquote{future\_clock}} ({\hyperref[\detokenize{temporal:temporal.Temporal}]{\sphinxcrossref{\sphinxstyleliteralemphasis{\sphinxupquote{temporal.Temporal}}}}}) \textendash{} a clock at a future time

\item[{Return out}] \leavevmode
the perceived hunger need association level

\item[{Return type}] \leavevmode
float

\end{description}\end{quote}

\end{fulllineitems}

\index{reset() (hunger.Hunger method)}

\begin{fulllineitems}
\phantomsection\label{\detokenize{hunger:hunger.Hunger.reset}}\pysiglinewithargsret{\sphinxbfcode{\sphinxupquote{reset}}}{}{}
This function resets the values in order for the need to be used in the next simulation.
\begin{quote}\begin{description}
\item[{Returns}] \leavevmode


\end{description}\end{quote}

\end{fulllineitems}

\index{set\_decay\_rate() (hunger.Hunger method)}

\begin{fulllineitems}
\phantomsection\label{\detokenize{hunger:hunger.Hunger.set_decay_rate}}\pysiglinewithargsret{\sphinxbfcode{\sphinxupquote{set\_decay\_rate}}}{\emph{t\_start}}{}
This function calculates the decay rate of hunger to the next meal.
\begin{quote}\begin{description}
\item[{Parameters}] \leavevmode\begin{itemize}
\item {} 
\sphinxstyleliteralstrong{\sphinxupquote{dt}} (\sphinxstyleliteralemphasis{\sphinxupquote{int}}) \textendash{} the amount of time \(\Delta{t}\) to the next meal {[}minutes{]}

\item {} 
\sphinxstyleliteralstrong{\sphinxupquote{t\_start}} (\sphinxstyleliteralemphasis{\sphinxupquote{int}}) \textendash{} the start time {[}in minutes{]} of the next meal

\end{itemize}

\item[{Returns}] \leavevmode
None

\end{description}\end{quote}

\end{fulllineitems}

\index{set\_decay\_rate\_new() (hunger.Hunger method)}

\begin{fulllineitems}
\phantomsection\label{\detokenize{hunger:hunger.Hunger.set_decay_rate_new}}\pysiglinewithargsret{\sphinxbfcode{\sphinxupquote{set\_decay\_rate\_new}}}{\emph{dt}}{}
This function calculates the decay rate of hunger to the next meal.
\begin{quote}\begin{description}
\item[{Parameters}] \leavevmode
\sphinxstyleliteralstrong{\sphinxupquote{dt}} (\sphinxstyleliteralemphasis{\sphinxupquote{int}}) \textendash{} the amount of time \(\Delta{t}\) to the next meal {[}minutes{]}

\item[{Returns}] \leavevmode
None

\end{description}\end{quote}

\end{fulllineitems}

\index{set\_recharge\_rate() (hunger.Hunger method)}

\begin{fulllineitems}
\phantomsection\label{\detokenize{hunger:hunger.Hunger.set_recharge_rate}}\pysiglinewithargsret{\sphinxbfcode{\sphinxupquote{set\_recharge\_rate}}}{\emph{dt}}{}
This function calculates the recharge rate of hunger due to eating the current meal.
\begin{quote}\begin{description}
\item[{Parameters}] \leavevmode
\sphinxstyleliteralstrong{\sphinxupquote{dt}} (\sphinxstyleliteralemphasis{\sphinxupquote{int}}) \textendash{} the amount of time \(\Delta{t}\) it takes to finish a meal {[}minutes{]}

\item[{Returns}] \leavevmode
None

\end{description}\end{quote}

\end{fulllineitems}

\index{set\_suggested\_recharge\_rate() (hunger.Hunger method)}

\begin{fulllineitems}
\phantomsection\label{\detokenize{hunger:hunger.Hunger.set_suggested_recharge_rate}}\pysiglinewithargsret{\sphinxbfcode{\sphinxupquote{set\_suggested\_recharge\_rate}}}{\emph{dt}}{}
This function sets the suggested recharge rate assuming a \sphinxstylestrong{linear function} behavior

The suggested recharge rate is based on the duration of the sleeping event         and the threshold. The sleep duration is based on the biological data (no rounding).
\begin{quote}\begin{description}
\item[{Parameters}] \leavevmode
\sphinxstyleliteralstrong{\sphinxupquote{dt}} (\sphinxstyleliteralemphasis{\sphinxupquote{int}}) \textendash{} The duration of time \(\Delta{t}\) of the eating event {[}minutes{]}

\item[{Returns}] \leavevmode
None

\end{description}\end{quote}

\end{fulllineitems}

\index{toString() (hunger.Hunger method)}

\begin{fulllineitems}
\phantomsection\label{\detokenize{hunger:hunger.Hunger.toString}}\pysiglinewithargsret{\sphinxbfcode{\sphinxupquote{toString}}}{}{}~\begin{quote}

Represents the Hunger object as a string.
\end{quote}
\begin{quote}\begin{description}
\item[{Return msg}] \leavevmode
the string representation of the huger object

\item[{Return type}] \leavevmode
str

\end{description}\end{quote}

\end{fulllineitems}


\end{fulllineitems}



\subsection{income module}
\label{\detokenize{income::doc}}\label{\detokenize{income:module-income}}\label{\detokenize{income:income-module}}\index{income (module)}
This is module contains code for governing the need to work/ be schooled.

This module contains the class {\hyperref[\detokenize{income:income.Income}]{\sphinxcrossref{\sphinxcode{\sphinxupquote{income.Income}}}}}.
\index{Income (class in income)}

\begin{fulllineitems}
\phantomsection\label{\detokenize{income:income.Income}}\pysiglinewithargsret{\sphinxbfcode{\sphinxupquote{class }}\sphinxcode{\sphinxupquote{income.}}\sphinxbfcode{\sphinxupquote{Income}}}{\emph{clock}, \emph{num\_sample\_points}}{}
Bases: {\hyperref[\detokenize{need:need.Need}]{\sphinxcrossref{\sphinxcode{\sphinxupquote{need.Need}}}}}

This class governs the need dealing with work / school. Recall that income mathematically     resembles a step function.
\begin{quote}\begin{description}
\item[{Parameters}] \leavevmode\begin{itemize}
\item {} 
\sphinxstyleliteralstrong{\sphinxupquote{clock}} ({\hyperref[\detokenize{temporal:temporal.Temporal}]{\sphinxcrossref{\sphinxstyleliteralemphasis{\sphinxupquote{temporal.Temporal}}}}}) \textendash{} the time

\item {} 
\sphinxstyleliteralstrong{\sphinxupquote{num\_sample\_points}} (\sphinxstyleliteralemphasis{\sphinxupquote{int}}) \textendash{} the number of temporal node points in the simulation

\end{itemize}

\end{description}\end{quote}
\index{decay() (income.Income method)}

\begin{fulllineitems}
\phantomsection\label{\detokenize{income:income.Income.decay}}\pysiglinewithargsret{\sphinxbfcode{\sphinxupquote{decay}}}{\emph{p}}{}
This function decays the magnitude of the need. Income only decays after the job start time.
\begin{enumerate}
\item {} 
Find out if it is time to work

\item {} 
If it’s time to work, set the satiation \(n_{income} = \eta_{work}\)

\end{enumerate}
\begin{quote}\begin{description}
\item[{Parameters}] \leavevmode
\sphinxstyleliteralstrong{\sphinxupquote{p}} ({\hyperref[\detokenize{person:person.Person}]{\sphinxcrossref{\sphinxstyleliteralemphasis{\sphinxupquote{person.Person}}}}}) \textendash{} the person of interest

\item[{Returns}] \leavevmode
None

\end{description}\end{quote}

\end{fulllineitems}

\index{initialize() (income.Income method)}

\begin{fulllineitems}
\phantomsection\label{\detokenize{income:income.Income.initialize}}\pysiglinewithargsret{\sphinxbfcode{\sphinxupquote{initialize}}}{\emph{p}}{}
This function is used to initialize the agent’s income need at the beginning of the simulation.         This function initializes the Person to be at the workplace (\sphinxcode{\sphinxupquote{location.OFF\_SITE}}) if it is work time.         This function does the following:
\begin{enumerate}
\item {} 
decay the income satiation

\item {} 
if the person is supposed to be at work
\begin{itemize}
\item {} 
set the person to the workplace location

\item {} 
else, set the amount of time until the next work event

\end{itemize}

\item {} 
update the scheduler for the income need

\end{enumerate}
\begin{quote}\begin{description}
\item[{Parameters}] \leavevmode
\sphinxstyleliteralstrong{\sphinxupquote{p}} ({\hyperref[\detokenize{person:person.Person}]{\sphinxcrossref{\sphinxstyleliteralemphasis{\sphinxupquote{person.Person}}}}}) \textendash{} the person of interest

\item[{Returns}] \leavevmode
None

\end{description}\end{quote}

\end{fulllineitems}

\index{perceive() (income.Income method)}

\begin{fulllineitems}
\phantomsection\label{\detokenize{income:income.Income.perceive}}\pysiglinewithargsret{\sphinxbfcode{\sphinxupquote{perceive}}}{\emph{clock}, \emph{job}}{}
This gives the satiation of income \sphinxstylestrong{if} the income need is addressed now.
\begin{enumerate}
\item {} 
find out if the time associated with clock implies a work time for the person

\item {} \begin{description}
\item[{If it should be work time}] \leavevmode\begin{itemize}
\item {} 
the perceived satiation is \(\eta_{work} \le \lambda\)

\item {} 
else, the perceived satiation is \(1.0\)

\end{itemize}

\end{description}

\end{enumerate}
\begin{quote}\begin{description}
\item[{Parameters}] \leavevmode\begin{itemize}
\item {} 
\sphinxstyleliteralstrong{\sphinxupquote{clock}} ({\hyperref[\detokenize{temporal:temporal.Temporal}]{\sphinxcrossref{\sphinxstyleliteralemphasis{\sphinxupquote{temporal.Temporal}}}}}) \textendash{} the future time the activity the should be perceived to be done

\item {} 
\sphinxstyleliteralstrong{\sphinxupquote{job}} ({\hyperref[\detokenize{occupation:occupation.Occupation}]{\sphinxcrossref{\sphinxstyleliteralemphasis{\sphinxupquote{occupation.Occupation}}}}}) \textendash{} the job

\end{itemize}

\item[{Returns}] \leavevmode
the satiation at the perceived time

\item[{Return type}] \leavevmode
float

\end{description}\end{quote}

\end{fulllineitems}


\end{fulllineitems}



\subsection{interrupt module}
\label{\detokenize{interrupt::doc}}\label{\detokenize{interrupt:interrupt-module}}\label{\detokenize{interrupt:module-interrupt}}\index{interrupt (module)}
This module contains code for interrupting a current activity.

This module contains class {\hyperref[\detokenize{interrupt:interrupt.Interrupt}]{\sphinxcrossref{\sphinxcode{\sphinxupquote{interrupt.Interrupt}}}}}.
\index{Interrupt (class in interrupt)}

\begin{fulllineitems}
\phantomsection\label{\detokenize{interrupt:interrupt.Interrupt}}\pysigline{\sphinxbfcode{\sphinxupquote{class }}\sphinxcode{\sphinxupquote{interrupt.}}\sphinxbfcode{\sphinxupquote{Interrupt}}}
Bases: {\hyperref[\detokenize{activity:activity.Activity}]{\sphinxcrossref{\sphinxcode{\sphinxupquote{activity.Activity}}}}}

This class allows for the current activity to be interrupted by another activity.
\index{advertise() (interrupt.Interrupt method)}

\begin{fulllineitems}
\phantomsection\label{\detokenize{interrupt:interrupt.Interrupt.advertise}}\pysiglinewithargsret{\sphinxbfcode{\sphinxupquote{advertise}}}{\emph{p}, \emph{str\_need}, \emph{act}}{}
This function calculates the score of an activities advertisement to a Person. This function does the         the following:
\begin{enumerate}
\item {} 
temporarily sets the value of the Need that must be immediately addressed to a low level.

\item {} 
send an advertisement is is made from the potentially interrupting activity

\item {} 
calculate the score from the potentially interrupting activity

\end{enumerate}
\begin{quote}\begin{description}
\item[{Parameters}] \leavevmode\begin{itemize}
\item {} 
\sphinxstyleliteralstrong{\sphinxupquote{p}} ({\hyperref[\detokenize{person:person.Person}]{\sphinxcrossref{\sphinxstyleliteralemphasis{\sphinxupquote{person.Person}}}}}) \textendash{} the person who is being advertised to

\item {} 
\sphinxstyleliteralstrong{\sphinxupquote{str\_need}} (\sphinxstyleliteralemphasis{\sphinxupquote{int}}) \textendash{} the id of the Need that needs to be addressed, which                                 could potentially cause an interrupting event

\item {} 
\sphinxstyleliteralstrong{\sphinxupquote{act}} ({\hyperref[\detokenize{activity:activity.Activity}]{\sphinxcrossref{\sphinxstyleliteralemphasis{\sphinxupquote{activity.Activity}}}}}) \textendash{} the activity of interest that could immediately                             interrupt a current activity

\end{itemize}

\item[{Returns}] \leavevmode
the value of the advertisement

\item[{Return type}] \leavevmode
float

\end{description}\end{quote}

\end{fulllineitems}

\index{start() (interrupt.Interrupt method)}

\begin{fulllineitems}
\phantomsection\label{\detokenize{interrupt:interrupt.Interrupt.start}}\pysiglinewithargsret{\sphinxbfcode{\sphinxupquote{start}}}{\emph{p}}{}
This handles the start of an activity.
\begin{quote}\begin{description}
\item[{Parameters}] \leavevmode
\sphinxstyleliteralstrong{\sphinxupquote{p}} ({\hyperref[\detokenize{person:person.Person}]{\sphinxcrossref{\sphinxstyleliteralemphasis{\sphinxupquote{person.Person}}}}}) \textendash{} the person of interest

\item[{Returns}] \leavevmode
None

\end{description}\end{quote}

\end{fulllineitems}


\end{fulllineitems}



\subsection{interruption module}
\label{\detokenize{interruption::doc}}\label{\detokenize{interruption:module-interruption}}\label{\detokenize{interruption:interruption-module}}\index{interruption (module)}
This class gives an agent the ability to interrupt a current activity.

This module contains class {\hyperref[\detokenize{interruption:interruption.Interruption}]{\sphinxcrossref{\sphinxcode{\sphinxupquote{interruption.Interruption}}}}}.
\index{Interruption (class in interruption)}

\begin{fulllineitems}
\phantomsection\label{\detokenize{interruption:interruption.Interruption}}\pysiglinewithargsret{\sphinxbfcode{\sphinxupquote{class }}\sphinxcode{\sphinxupquote{interruption.}}\sphinxbfcode{\sphinxupquote{Interruption}}}{\emph{clock}, \emph{num\_sample\_points}}{}
Bases: {\hyperref[\detokenize{need:need.Need}]{\sphinxcrossref{\sphinxcode{\sphinxupquote{need.Need}}}}}

This class enables a Person to interrupt a current activity.
\begin{quote}\begin{description}
\item[{Parameters}] \leavevmode\begin{itemize}
\item {} 
\sphinxstyleliteralstrong{\sphinxupquote{clock}} ({\hyperref[\detokenize{temporal:temporal.Temporal}]{\sphinxcrossref{\sphinxstyleliteralemphasis{\sphinxupquote{temporal.Temporal}}}}}) \textendash{} the clock governing time in the simulation

\item {} 
\sphinxstyleliteralstrong{\sphinxupquote{num\_sample\_points}} (\sphinxstyleliteralemphasis{\sphinxupquote{int}}) \textendash{} the number of time nodes in the simulation

\end{itemize}

\item[{Variables}] \leavevmode\begin{itemize}
\item {} 
\sphinxstyleliteralstrong{\sphinxupquote{category}} (\sphinxstyleliteralemphasis{\sphinxupquote{int}}) \textendash{} the category of the interruption Need

\item {} 
\sphinxstyleliteralstrong{\sphinxupquote{activity\_start}} (\sphinxstyleliteralemphasis{\sphinxupquote{int}}) \textendash{} the category of the (interrupting) activity to start

\item {} 
\sphinxstyleliteralstrong{\sphinxupquote{activity\_stop}} (\sphinxstyleliteralemphasis{\sphinxupquote{int}}) \textendash{} the category of the (interrupted) activity to stop

\end{itemize}

\end{description}\end{quote}
\index{decay() (interruption.Interruption method)}

\begin{fulllineitems}
\phantomsection\label{\detokenize{interruption:interruption.Interruption.decay}}\pysiglinewithargsret{\sphinxbfcode{\sphinxupquote{decay}}}{\emph{p}}{}~\begin{quote}

This function sets the default decrease in the Interruption need
\end{quote}
\begin{quote}\begin{description}
\item[{Parameters}] \leavevmode
\sphinxstyleliteralstrong{\sphinxupquote{p}} ({\hyperref[\detokenize{person:person.Person}]{\sphinxcrossref{\sphinxstyleliteralemphasis{\sphinxupquote{person.Person}}}}}) \textendash{} the person of interest

\item[{Returns}] \leavevmode
None

\end{description}\end{quote}

\end{fulllineitems}

\index{get\_time\_to\_next\_work\_lunch() (interruption.Interruption method)}

\begin{fulllineitems}
\phantomsection\label{\detokenize{interruption:interruption.Interruption.get_time_to_next_work_lunch}}\pysiglinewithargsret{\sphinxbfcode{\sphinxupquote{get\_time\_to\_next\_work\_lunch}}}{\emph{p}}{}
This function calculates the amount of time {[}in minutes{]} until the agent should
eat lunch at work.
\begin{quote}\begin{description}
\item[{Parameters}] \leavevmode
\sphinxstyleliteralstrong{\sphinxupquote{p}} ({\hyperref[\detokenize{person:person.Person}]{\sphinxcrossref{\sphinxstyleliteralemphasis{\sphinxupquote{person.Person}}}}}) \textendash{} the person of interest

\item[{Returns}] \leavevmode
the amount of time {[}minutes{]} until the next time the agent should         eat lunch at work

\end{description}\end{quote}

\end{fulllineitems}

\index{initialize() (interruption.Interruption method)}

\begin{fulllineitems}
\phantomsection\label{\detokenize{interruption:interruption.Interruption.initialize}}\pysiglinewithargsret{\sphinxbfcode{\sphinxupquote{initialize}}}{\emph{p}}{}
Initializes the need at the beginning of the simulation.
\begin{quote}\begin{description}
\item[{Parameters}] \leavevmode
\sphinxstyleliteralstrong{\sphinxupquote{p}} ({\hyperref[\detokenize{person:person.Person}]{\sphinxcrossref{\sphinxstyleliteralemphasis{\sphinxupquote{person.Person}}}}}) \textendash{} the person of interest

\item[{Returns}] \leavevmode
None

\end{description}\end{quote}

\end{fulllineitems}

\index{is\_lunch\_time() (interruption.Interruption method)}

\begin{fulllineitems}
\phantomsection\label{\detokenize{interruption:interruption.Interruption.is_lunch_time}}\pysiglinewithargsret{\sphinxbfcode{\sphinxupquote{is\_lunch\_time}}}{\emph{time\_of\_day}, \emph{meals}}{}
This function indicates whether it is lunch time or not. This is used in the         interruption to stop the work activity and begin the eat lunch activity.
\begin{quote}\begin{description}
\item[{Parameters}] \leavevmode\begin{itemize}
\item {} 
\sphinxstyleliteralstrong{\sphinxupquote{time\_of\_day}} (\sphinxstyleliteralemphasis{\sphinxupquote{int}}) \textendash{} the time of day {[}minutes{]}

\item {} 
\sphinxstyleliteralstrong{\sphinxupquote{meals}} (\sphinxstyleliteralemphasis{\sphinxupquote{list}}) \textendash{} a list of the meals ({\hyperref[\detokenize{meal:meal.Meal}]{\sphinxcrossref{\sphinxcode{\sphinxupquote{meal.Meal}}}}}) for the agents; some of the         entries in the list may be None.

\end{itemize}

\item[{Return is\_lunch}] \leavevmode
a flag indicating whether it is lunch time

\end{description}\end{quote}

\end{fulllineitems}

\index{perceive() (interruption.Interruption method)}

\begin{fulllineitems}
\phantomsection\label{\detokenize{interruption:interruption.Interruption.perceive}}\pysiglinewithargsret{\sphinxbfcode{\sphinxupquote{perceive}}}{\emph{clock}}{}
This gives the result if sleep is done now until a later time corresponding to clock.
\begin{quote}\begin{description}
\item[{Parameters}] \leavevmode
\sphinxstyleliteralstrong{\sphinxupquote{clock}} ({\hyperref[\detokenize{temporal:temporal.Temporal}]{\sphinxcrossref{\sphinxstyleliteralemphasis{\sphinxupquote{temporal.Temporal}}}}}) \textendash{} a clock at a future time

\item[{Return out}] \leavevmode
the perceived interruption magnitude

\end{description}\end{quote}

\end{fulllineitems}

\index{reset() (interruption.Interruption method)}

\begin{fulllineitems}
\phantomsection\label{\detokenize{interruption:interruption.Interruption.reset}}\pysiglinewithargsret{\sphinxbfcode{\sphinxupquote{reset}}}{}{}
This function resets the Interruption need completely in order to re run         the simulation. In this reset the history is also reset.
\begin{quote}\begin{description}
\item[{Returns}] \leavevmode


\end{description}\end{quote}

\end{fulllineitems}

\index{reset\_minor() (interruption.Interruption method)}

\begin{fulllineitems}
\phantomsection\label{\detokenize{interruption:interruption.Interruption.reset_minor}}\pysiglinewithargsret{\sphinxbfcode{\sphinxupquote{reset\_minor}}}{}{}
This function resets the interruption need
\begin{quote}\begin{description}
\item[{Returns}] \leavevmode
None

\end{description}\end{quote}

\end{fulllineitems}

\index{stop\_work\_to\_eat() (interruption.Interruption method)}

\begin{fulllineitems}
\phantomsection\label{\detokenize{interruption:interruption.Interruption.stop_work_to_eat}}\pysiglinewithargsret{\sphinxbfcode{\sphinxupquote{stop\_work\_to\_eat}}}{\emph{p}}{}
This function checks to see if an interruption should occur to allow a person to         start the eating activity while doing the work activity

An agent may stop the work activity to eat lunch if the following conditions are met:
\begin{enumerate}
\item {} 
the agent is hungry

\item {} 
the current activity is work

\item {} 
it is lunch time

\end{enumerate}
\begin{quote}\begin{description}
\item[{Parameters}] \leavevmode
\sphinxstyleliteralstrong{\sphinxupquote{p}} ({\hyperref[\detokenize{person:person.Person}]{\sphinxcrossref{\sphinxstyleliteralemphasis{\sphinxupquote{person.Person}}}}}) \textendash{} the person of interest

\item[{Returns}] \leavevmode
None

\end{description}\end{quote}

\end{fulllineitems}


\end{fulllineitems}



\subsection{location module}
\label{\detokenize{location::doc}}\label{\detokenize{location:location-module}}\label{\detokenize{location:module-location}}\index{location (module)}
This module is responsible for containing information about the concept of location.

This module contains class {\hyperref[\detokenize{location:location.Location}]{\sphinxcrossref{\sphinxcode{\sphinxupquote{location.Location}}}}}.
\index{Location (class in location)}

\begin{fulllineitems}
\phantomsection\label{\detokenize{location:location.Location}}\pysiglinewithargsret{\sphinxbfcode{\sphinxupquote{class }}\sphinxcode{\sphinxupquote{location.}}\sphinxbfcode{\sphinxupquote{Location}}}{\emph{geography=1}, \emph{local=0}}{}
Bases: \sphinxcode{\sphinxupquote{object}}

This class holds information relevant to the location of persons and assets.
\begin{quote}\begin{description}
\item[{Parameters}] \leavevmode\begin{itemize}
\item {} 
\sphinxstyleliteralstrong{\sphinxupquote{geography}} (\sphinxstyleliteralemphasis{\sphinxupquote{int}}) \textendash{} the geographical location code

\item {} 
\sphinxstyleliteralstrong{\sphinxupquote{local}} (\sphinxstyleliteralemphasis{\sphinxupquote{int}}) \textendash{} the local location code

\end{itemize}

\item[{Variables}] \leavevmode\begin{itemize}
\item {} 
\sphinxstyleliteralstrong{\sphinxupquote{geo}} (\sphinxstyleliteralemphasis{\sphinxupquote{int}}) \textendash{} the geographical location code within the United States (e.g. north, south, eats, or west)

\item {} 
\sphinxstyleliteralstrong{\sphinxupquote{local}} (\sphinxstyleliteralemphasis{\sphinxupquote{int}}) \textendash{} the local location code (e.g. home, off site, etc)

\end{itemize}

\end{description}\end{quote}
\index{print\_geo() (location.Location method)}

\begin{fulllineitems}
\phantomsection\label{\detokenize{location:location.Location.print_geo}}\pysiglinewithargsret{\sphinxbfcode{\sphinxupquote{print\_geo}}}{}{}
Returns the geographical location in a string format
\begin{quote}\begin{description}
\item[{Returns}] \leavevmode
the string representation of the geographical location

\item[{Return type}] \leavevmode
str

\end{description}\end{quote}

\end{fulllineitems}

\index{print\_local() (location.Location method)}

\begin{fulllineitems}
\phantomsection\label{\detokenize{location:location.Location.print_local}}\pysiglinewithargsret{\sphinxbfcode{\sphinxupquote{print\_local}}}{}{}
Returns the local location in a string format
\begin{quote}\begin{description}
\item[{Returns}] \leavevmode
the string representation of the local location

\item[{Return type}] \leavevmode
str

\end{description}\end{quote}

\end{fulllineitems}

\index{reset() (location.Location method)}

\begin{fulllineitems}
\phantomsection\label{\detokenize{location:location.Location.reset}}\pysiglinewithargsret{\sphinxbfcode{\sphinxupquote{reset}}}{}{}
This function resets the location to the default value, (\sphinxcode{\sphinxupquote{location.HOME}}).
\begin{quote}\begin{description}
\item[{Returns}] \leavevmode
None

\end{description}\end{quote}

\end{fulllineitems}

\index{toString() (location.Location method)}

\begin{fulllineitems}
\phantomsection\label{\detokenize{location:location.Location.toString}}\pysiglinewithargsret{\sphinxbfcode{\sphinxupquote{toString}}}{}{}
This function represents the Location object as a string.
\begin{quote}\begin{description}
\item[{Return msg}] \leavevmode
the string representation of the Location object

\item[{Return type}] \leavevmode
str

\end{description}\end{quote}

\end{fulllineitems}


\end{fulllineitems}



\subsection{meal module}
\label{\detokenize{meal::doc}}\label{\detokenize{meal:module-meal}}\label{\detokenize{meal:meal-module}}\index{meal (module)}
This module contains contains information about various meals that an agent  would eat in.

This module contains code for class {\hyperref[\detokenize{meal:meal.Meal}]{\sphinxcrossref{\sphinxcode{\sphinxupquote{meal.Meal}}}}}.
\index{Meal (class in meal)}

\begin{fulllineitems}
\phantomsection\label{\detokenize{meal:meal.Meal}}\pysiglinewithargsret{\sphinxbfcode{\sphinxupquote{class }}\sphinxcode{\sphinxupquote{meal.}}\sphinxbfcode{\sphinxupquote{Meal}}}{\emph{id=0}, \emph{start\_mean=390}, \emph{start\_std=10}, \emph{start\_trunc=1}, \emph{dt\_mean=15}, \emph{dt\_std=5}, \emph{dt\_trunc=1}}{}
Bases: \sphinxcode{\sphinxupquote{object}}

This is class contains information about meals (breakfast, dinner, and lunch)
\begin{quote}\begin{description}
\item[{Variables}] \leavevmode\begin{itemize}
\item {} 
\sphinxstyleliteralstrong{\sphinxupquote{id}} (\sphinxstyleliteralemphasis{\sphinxupquote{int}}) \textendash{} the meal type (breakfast, lunch, or dinner)

\item {} 
\sphinxstyleliteralstrong{\sphinxupquote{dt}} (\sphinxstyleliteralemphasis{\sphinxupquote{int}}) \textendash{} the duration of a meal {[}minutes{]}

\item {} 
\sphinxstyleliteralstrong{\sphinxupquote{dt\_mean}} (\sphinxstyleliteralemphasis{\sphinxupquote{int}}) \textendash{} the mean duration of a meal {[}minutes{]}

\item {} 
\sphinxstyleliteralstrong{\sphinxupquote{dt\_std}} (\sphinxstyleliteralemphasis{\sphinxupquote{int}}) \textendash{} the standard deviation of meal duration {[}minutes{]}

\item {} 
\sphinxstyleliteralstrong{\sphinxupquote{dt\_trunc}} (\sphinxstyleliteralemphasis{\sphinxupquote{int}}) \textendash{} the number of standard deviation in the duration distribution

\item {} 
\sphinxstyleliteralstrong{\sphinxupquote{t\_start}} (\sphinxstyleliteralemphasis{\sphinxupquote{int}}) \textendash{} the start time of a meal {[}minutes, time of day{]}

\item {} 
\sphinxstyleliteralstrong{\sphinxupquote{t\_start\_univ}} (\sphinxstyleliteralemphasis{\sphinxupquote{int}}) \textendash{} the start time of a meals {[}minutes, universal time{]}

\item {} 
\sphinxstyleliteralstrong{\sphinxupquote{start\_mean}} (\sphinxstyleliteralemphasis{\sphinxupquote{int}}) \textendash{} the mean start time of a meal {[}minutes, time of day{]}

\item {} 
\sphinxstyleliteralstrong{\sphinxupquote{start\_std}} (\sphinxstyleliteralemphasis{\sphinxupquote{int}}) \textendash{} the standard deviation of start time of a meal {[}minutes{]}

\item {} 
\sphinxstyleliteralstrong{\sphinxupquote{start\_trunc}} (\sphinxstyleliteralemphasis{\sphinxupquote{int}}) \textendash{} the number of standard deviation of in the start time distribution

\item {} 
\sphinxstyleliteralstrong{\sphinxupquote{f\_start}} \textendash{} the start time distribution function

\item {} 
\sphinxstyleliteralstrong{\sphinxupquote{f\_dt}} \textendash{} the duration distribution function

\item {} 
\sphinxstyleliteralstrong{\sphinxupquote{day}} (\sphinxstyleliteralemphasis{\sphinxupquote{int}}) \textendash{} the day the meal should occur

\end{itemize}

\end{description}\end{quote}
\index{initialize() (meal.Meal method)}

\begin{fulllineitems}
\phantomsection\label{\detokenize{meal:meal.Meal.initialize}}\pysiglinewithargsret{\sphinxbfcode{\sphinxupquote{initialize}}}{\emph{t\_univ}}{}
At the beginning of the simulation, make sure that the meals are initialized to the proper times         (\sphinxcode{\sphinxupquote{t\_start\_univ}}) so that they relate to the simulation time (t\_univ)
\begin{quote}\begin{description}
\item[{Parameters}] \leavevmode
\sphinxstyleliteralstrong{\sphinxupquote{t\_univ}} (\sphinxstyleliteralemphasis{\sphinxupquote{int}}) \textendash{} the simulation time {[}minutes, universal time{]}

\item[{Returns}] \leavevmode
None

\end{description}\end{quote}

\end{fulllineitems}

\index{print\_id() (meal.Meal method)}

\begin{fulllineitems}
\phantomsection\label{\detokenize{meal:meal.Meal.print_id}}\pysiglinewithargsret{\sphinxbfcode{\sphinxupquote{print\_id}}}{}{}
This function returns a string representation of the meal id
\begin{quote}\begin{description}
\item[{Returns}] \leavevmode
a string representation of the meal id

\item[{Return type}] \leavevmode
str

\end{description}\end{quote}

\end{fulllineitems}

\index{set\_meal() (meal.Meal method)}

\begin{fulllineitems}
\phantomsection\label{\detokenize{meal:meal.Meal.set_meal}}\pysiglinewithargsret{\sphinxbfcode{\sphinxupquote{set\_meal}}}{\emph{id}, \emph{start\_mean}, \emph{start\_std}, \emph{start\_trunc}, \emph{dt\_mean}, \emph{dt\_std}, \emph{dt\_trunc}}{}
This function sets the values associated with the Meal object.
\begin{quote}\begin{description}
\item[{Parameters}] \leavevmode\begin{itemize}
\item {} 
\sphinxstyleliteralstrong{\sphinxupquote{id}} (\sphinxstyleliteralemphasis{\sphinxupquote{int}}) \textendash{} the meal type (breakfast, lunch, or dinner)

\item {} 
\sphinxstyleliteralstrong{\sphinxupquote{start\_mean}} (\sphinxstyleliteralemphasis{\sphinxupquote{int}}) \textendash{} the mean start time of the meal {[}minutes, time of day{]}

\item {} 
\sphinxstyleliteralstrong{\sphinxupquote{start\_std}} (\sphinxstyleliteralemphasis{\sphinxupquote{int}}) \textendash{} the standard deviation of start time {[}minutes{]}

\item {} 
\sphinxstyleliteralstrong{\sphinxupquote{start\_turnc}} (\sphinxstyleliteralemphasis{\sphinxupquote{int}}) \textendash{} the number of standard deviations in the start time distribution

\item {} 
\sphinxstyleliteralstrong{\sphinxupquote{dt\_mean}} (\sphinxstyleliteralemphasis{\sphinxupquote{int}}) \textendash{} the mean duration of a meal {[}minutes{]}

\item {} 
\sphinxstyleliteralstrong{\sphinxupquote{dt\_std}} (\sphinxstyleliteralemphasis{\sphinxupquote{int}}) \textendash{} the standard deviation of meal duration {[}minutes{]}

\item {} 
\sphinxstyleliteralstrong{\sphinxupquote{dt\_trunc}} (\sphinxstyleliteralemphasis{\sphinxupquote{int}}) \textendash{} the number of standard deviations in the duration distribution

\end{itemize}

\item[{Returns}] \leavevmode
None

\end{description}\end{quote}

\end{fulllineitems}

\index{toString() (meal.Meal method)}

\begin{fulllineitems}
\phantomsection\label{\detokenize{meal:meal.Meal.toString}}\pysiglinewithargsret{\sphinxbfcode{\sphinxupquote{toString}}}{}{}
This function returns a string representation of the Meal object.
\begin{quote}\begin{description}
\item[{Return msg}] \leavevmode
a string representation of the Meal object

\item[{Return type}] \leavevmode
str

\end{description}\end{quote}

\end{fulllineitems}

\index{update() (meal.Meal method)}

\begin{fulllineitems}
\phantomsection\label{\detokenize{meal:meal.Meal.update}}\pysiglinewithargsret{\sphinxbfcode{\sphinxupquote{update}}}{\emph{day}}{}
Given the day for the meal, update the meal. The following does the following:
\begin{enumerate}
\item {} 
Update the start time for the meal

\item {} 
Update the duration for the meal

\item {} 
Update the universal start time for the meal

\end{enumerate}
\begin{quote}\begin{description}
\item[{Parameters}] \leavevmode
\sphinxstyleliteralstrong{\sphinxupquote{day}} (\sphinxstyleliteralemphasis{\sphinxupquote{int}}) \textendash{} the day for the meal to occur

\item[{Returns}] \leavevmode
None

\end{description}\end{quote}

\end{fulllineitems}

\index{update\_day() (meal.Meal method)}

\begin{fulllineitems}
\phantomsection\label{\detokenize{meal:meal.Meal.update_day}}\pysiglinewithargsret{\sphinxbfcode{\sphinxupquote{update\_day}}}{}{}
Update the day for the next meal, given the universal start time for the meal (\sphinxcode{\sphinxupquote{t\_start\_univ}}).
\begin{quote}\begin{description}
\item[{Returns}] \leavevmode
None

\end{description}\end{quote}

\end{fulllineitems}

\index{update\_dt() (meal.Meal method)}

\begin{fulllineitems}
\phantomsection\label{\detokenize{meal:meal.Meal.update_dt}}\pysiglinewithargsret{\sphinxbfcode{\sphinxupquote{update\_dt}}}{}{}
Sample the duration distribution to get a duration.
\begin{quote}\begin{description}
\item[{Returns}] \leavevmode
None

\end{description}\end{quote}

\end{fulllineitems}

\index{update\_start() (meal.Meal method)}

\begin{fulllineitems}
\phantomsection\label{\detokenize{meal:meal.Meal.update_start}}\pysiglinewithargsret{\sphinxbfcode{\sphinxupquote{update\_start}}}{}{}
Sample the start time distribution to get a start time.
\begin{quote}\begin{description}
\item[{Returns}] \leavevmode
None

\end{description}\end{quote}

\end{fulllineitems}

\index{update\_start\_univ() (meal.Meal method)}

\begin{fulllineitems}
\phantomsection\label{\detokenize{meal:meal.Meal.update_start_univ}}\pysiglinewithargsret{\sphinxbfcode{\sphinxupquote{update\_start\_univ}}}{\emph{day}}{}
Given the day for the next meal, update the universal start time for the meal.
\begin{quote}\begin{description}
\item[{Parameters}] \leavevmode
\sphinxstyleliteralstrong{\sphinxupquote{day}} (\sphinxstyleliteralemphasis{\sphinxupquote{int}}) \textendash{} the day for the meal

\item[{Returns}] \leavevmode
None

\end{description}\end{quote}

\end{fulllineitems}


\end{fulllineitems}



\subsection{my\_globals module}
\label{\detokenize{my_globals::doc}}\label{\detokenize{my_globals:module-my_globals}}\label{\detokenize{my_globals:my-globals-module}}\index{my\_globals (module)}
This module contains constants and functions that are used for general use.

This module contains information about the following constants:
\begin{enumerate}
\item {} 
Identifiers for activity codes

\item {} 
File names file paths for saving figures for the different demographics

\item {} 
File names file paths for saving figures for the different activities

\end{enumerate}
\index{check\_filename\_extension() (in module my\_globals)}

\begin{fulllineitems}
\phantomsection\label{\detokenize{my_globals:my_globals.check_filename_extension}}\pysiglinewithargsret{\sphinxcode{\sphinxupquote{my\_globals.}}\sphinxbfcode{\sphinxupquote{check\_filename\_extension}}}{\emph{fname}, \emph{ext}}{}
This function returns whether or not the given file name has the given
filename extension.
\begin{quote}\begin{description}
\item[{Parameters}] \leavevmode\begin{itemize}
\item {} 
\sphinxstyleliteralstrong{\sphinxupquote{fname}} (\sphinxstyleliteralemphasis{\sphinxupquote{str}}) \textendash{} the file name

\item {} 
\sphinxstyleliteralstrong{\sphinxupquote{or str ext}} (\sphinxstyleliteralemphasis{\sphinxupquote{list}}) \textendash{} a single (or list) of acceptable filename extensions

\end{itemize}

\item[{Returns}] \leavevmode


\end{description}\end{quote}

\end{fulllineitems}

\index{fill\_out\_data() (in module my\_globals)}

\begin{fulllineitems}
\phantomsection\label{\detokenize{my_globals:my_globals.fill_out_data}}\pysiglinewithargsret{\sphinxcode{\sphinxupquote{my\_globals.}}\sphinxbfcode{\sphinxupquote{fill\_out\_data}}}{\emph{t}, \emph{y}}{}
This function takes an array of activity start times and activity codes from an activity diary and     fills out the activity, minute-by-minute in between two adjacent activities.
\begin{quote}\begin{description}
\item[{Parameters}] \leavevmode\begin{itemize}
\item {} 
\sphinxstyleliteralstrong{\sphinxupquote{t}} (\sphinxstyleliteralemphasis{\sphinxupquote{numpy.ndarray}}) \textendash{} the start times in an activity diary

\item {} 
\sphinxstyleliteralstrong{\sphinxupquote{y}} (\sphinxstyleliteralemphasis{\sphinxupquote{numpy.ndarray}}) \textendash{} the activity codes in an activity diary

\end{itemize}

\item[{Returns}] \leavevmode


\end{description}\end{quote}

\end{fulllineitems}

\index{fill\_out\_time() (in module my\_globals)}

\begin{fulllineitems}
\phantomsection\label{\detokenize{my_globals:my_globals.fill_out_time}}\pysiglinewithargsret{\sphinxcode{\sphinxupquote{my\_globals.}}\sphinxbfcode{\sphinxupquote{fill\_out\_time}}}{\emph{t}}{}
This function takes an array of activity start times from an activity diary and fills out the time,     minute-by-minute in between two adjacent activities

Example, if t = (0, 4, 7) (and \(t_{final}=10\)) we get the following:
\begin{itemize}
\item {} 
(0, 1, 2, 3)

\item {} 
(4, 5, 6)

\item {} 
(7, 8, 9, 10)

\end{itemize}
\begin{quote}\begin{description}
\item[{Parameters}] \leavevmode
\sphinxstyleliteralstrong{\sphinxupquote{t}} (\sphinxstyleliteralemphasis{\sphinxupquote{numpy.ndarray}}) \textendash{} the start times in the activity diary {[}minutes, universal time{]}

\item[{Returns}] \leavevmode
None

\end{description}\end{quote}

\end{fulllineitems}

\index{from\_periodic() (in module my\_globals)}

\begin{fulllineitems}
\phantomsection\label{\detokenize{my_globals:my_globals.from_periodic}}\pysiglinewithargsret{\sphinxcode{\sphinxupquote{my\_globals.}}\sphinxbfcode{\sphinxupquote{from\_periodic}}}{\emph{t}, \emph{do\_hours=True}}{}
This function returns the time of day in a 24 hour format. It takes the time \(t \in [-12, 12)\) and     expresses it at time \(x \in [0, 24)\) where 0 represents midnight. The same calculation can be     done to represent the time in minutes
\begin{quote}\begin{description}
\item[{Parameters}] \leavevmode\begin{itemize}
\item {} 
\sphinxstyleliteralstrong{\sphinxupquote{t}} (\sphinxstyleliteralemphasis{\sphinxupquote{float}}) \textendash{} the time in hours {[}-12, 12), or the respective minutes {[}-12 * 60, 12 * 60)

\item {} 
\sphinxstyleliteralstrong{\sphinxupquote{do\_hours}} (\sphinxstyleliteralemphasis{\sphinxupquote{bool}}) \textendash{} a flag to do the calculations in hours (if True)

\end{itemize}

\item[{Returns}] \leavevmode
the time in {[}0, 24) or in minutes {[}0, 24 * 60)

\end{description}\end{quote}

\end{fulllineitems}

\index{get\_ecdf() (in module my\_globals)}

\begin{fulllineitems}
\phantomsection\label{\detokenize{my_globals:my_globals.get_ecdf}}\pysiglinewithargsret{\sphinxcode{\sphinxupquote{my\_globals.}}\sphinxbfcode{\sphinxupquote{get\_ecdf}}}{\emph{data}, \emph{N=100}}{}
Given data, this function calculates the probability data from the empirical cumulative     distribution function (ECDF).
\begin{quote}\begin{description}
\item[{Parameters}] \leavevmode\begin{itemize}
\item {} 
\sphinxstyleliteralstrong{\sphinxupquote{data}} (\sphinxstyleliteralemphasis{\sphinxupquote{float}}) \textendash{} an array containing the relevant data to get the ECDF of

\item {} 
\sphinxstyleliteralstrong{\sphinxupquote{N}} (\sphinxstyleliteralemphasis{\sphinxupquote{int}}) \textendash{} the amount of samples in calculating the ECDF results

\end{itemize}

\item[{Return y}] \leavevmode
the ECDF

\item[{Return type}] \leavevmode
float array

\item[{Return x}] \leavevmode
the values sampled for the ECDF

\item[{Return type}] \leavevmode
float array

\end{description}\end{quote}

\end{fulllineitems}

\index{group\_time() (in module my\_globals)}

\begin{fulllineitems}
\phantomsection\label{\detokenize{my_globals:my_globals.group_time}}\pysiglinewithargsret{\sphinxcode{\sphinxupquote{my\_globals.}}\sphinxbfcode{\sphinxupquote{group\_time}}}{\emph{t}}{}
This function takes data from an activity diary and groups that activity diary into ,
minute by minute arrays from start to end for each activity (start, start + 1, … end-1, end)
\begin{quote}\begin{description}
\item[{Parameters}] \leavevmode
\sphinxstyleliteralstrong{\sphinxupquote{t}} (\sphinxstyleliteralemphasis{\sphinxupquote{numpy.ndarray}}) \textendash{} the start times from an activity diary {[}minutes, universal time{]}

\item[{Returns}] \leavevmode
the grouped start/end pairs for ech activitiy

\item[{Return type}] \leavevmode
list

\end{description}\end{quote}

\end{fulllineitems}

\index{hours\_to\_minutes() (in module my\_globals)}

\begin{fulllineitems}
\phantomsection\label{\detokenize{my_globals:my_globals.hours_to_minutes}}\pysiglinewithargsret{\sphinxcode{\sphinxupquote{my\_globals.}}\sphinxbfcode{\sphinxupquote{hours\_to\_minutes}}}{\emph{t}}{}
This function takes a duration of time in hours and converts the time rounded to the nearest minutes.
\begin{quote}\begin{description}
\item[{Parameters}] \leavevmode
\sphinxstyleliteralstrong{\sphinxupquote{t}} (\sphinxstyleliteralemphasis{\sphinxupquote{float}}) \textendash{} a duration of time {[}hours{]}

\item[{Returns}] \leavevmode
the time in minutes

\end{description}\end{quote}

\end{fulllineitems}

\index{initialize\_random\_number\_generator() (in module my\_globals)}

\begin{fulllineitems}
\phantomsection\label{\detokenize{my_globals:my_globals.initialize_random_number_generator}}\pysiglinewithargsret{\sphinxcode{\sphinxupquote{my\_globals.}}\sphinxbfcode{\sphinxupquote{initialize\_random\_number\_generator}}}{\emph{seed=None}}{}
This function initializes the random number generators with a specified seed, if given (i.e., that is if
seed is not None). Both the \sphinxstyleemphasis{random} module and the \sphinxstyleemphasis{numpy.random} module’s random number generator are     seeded. This is useful for reproducing results.
\begin{quote}\begin{description}
\item[{Parameters}] \leavevmode
\sphinxstyleliteralstrong{\sphinxupquote{seed}} (\sphinxstyleliteralemphasis{\sphinxupquote{int}}) \textendash{} the seed for the number generator

\item[{Returns}] \leavevmode


\end{description}\end{quote}

\end{fulllineitems}

\index{load() (in module my\_globals)}

\begin{fulllineitems}
\phantomsection\label{\detokenize{my_globals:my_globals.load}}\pysiglinewithargsret{\sphinxcode{\sphinxupquote{my\_globals.}}\sphinxbfcode{\sphinxupquote{load}}}{\emph{fname}}{}
This function loads data from a .pkl file.
\begin{quote}\begin{description}
\item[{Parameters}] \leavevmode
\sphinxstyleliteralstrong{\sphinxupquote{fname}} (\sphinxstyleliteralemphasis{\sphinxupquote{str}}) \textendash{} the file name to be loaded from

\item[{Returns}] \leavevmode
the data unpickled

\end{description}\end{quote}

\end{fulllineitems}

\index{sample() (in module my\_globals)}

\begin{fulllineitems}
\phantomsection\label{\detokenize{my_globals:my_globals.sample}}\pysiglinewithargsret{\sphinxcode{\sphinxupquote{my\_globals.}}\sphinxbfcode{\sphinxupquote{sample}}}{\emph{data}, \emph{N}}{}
This function creates N samples of the empirical distribution of the values in the     data array.
\begin{quote}\begin{description}
\item[{Parameters}] \leavevmode\begin{itemize}
\item {} 
\sphinxstyleliteralstrong{\sphinxupquote{data}} (\sphinxstyleliteralemphasis{\sphinxupquote{numpy.ndarray}}) \textendash{} the data to sample

\item {} 
\sphinxstyleliteralstrong{\sphinxupquote{N}} (\sphinxstyleliteralemphasis{\sphinxupquote{int}}) \textendash{} the number of data points to sample

\end{itemize}

\item[{Returns}] \leavevmode


\end{description}\end{quote}

\end{fulllineitems}

\index{sample\_normal() (in module my\_globals)}

\begin{fulllineitems}
\phantomsection\label{\detokenize{my_globals:my_globals.sample_normal}}\pysiglinewithargsret{\sphinxcode{\sphinxupquote{my\_globals.}}\sphinxbfcode{\sphinxupquote{sample\_normal}}}{\emph{std}, \emph{dx}}{}
This function samples a normal distribution centered at zero assuming a max and minimum acceptable value {[}dx, -dx{]}.
\begin{quote}\begin{description}
\item[{Parameters}] \leavevmode\begin{itemize}
\item {} 
\sphinxstyleliteralstrong{\sphinxupquote{std}} (\sphinxstyleliteralemphasis{\sphinxupquote{float}}) \textendash{} the standard deviation

\item {} 
\sphinxstyleliteralstrong{\sphinxupquote{dx}} (\sphinxstyleliteralemphasis{\sphinxupquote{float}}) \textendash{} the amount of total deviation from the mean allowd

\end{itemize}

\item[{Returns}] \leavevmode


\end{description}\end{quote}

\end{fulllineitems}

\index{save() (in module my\_globals)}

\begin{fulllineitems}
\phantomsection\label{\detokenize{my_globals:my_globals.save}}\pysiglinewithargsret{\sphinxcode{\sphinxupquote{my\_globals.}}\sphinxbfcode{\sphinxupquote{save}}}{\emph{x}, \emph{fname}}{}
This function saves a python variable by pickling it.
\begin{quote}\begin{description}
\item[{Parameters}] \leavevmode\begin{itemize}
\item {} 
\sphinxstyleliteralstrong{\sphinxupquote{x}} \textendash{} the data to be saved

\item {} 
\sphinxstyleliteralstrong{\sphinxupquote{fname}} (\sphinxstyleliteralemphasis{\sphinxupquote{str}}) \textendash{} the file name of the saved file. It must end with .pkl

\end{itemize}

\end{description}\end{quote}

\end{fulllineitems}

\index{save\_diary\_to\_csv() (in module my\_globals)}

\begin{fulllineitems}
\phantomsection\label{\detokenize{my_globals:my_globals.save_diary_to_csv}}\pysiglinewithargsret{\sphinxcode{\sphinxupquote{my\_globals.}}\sphinxbfcode{\sphinxupquote{save\_diary\_to\_csv}}}{\emph{df}, \emph{fname}}{}
This function saves an activity diary as a .csv file. The output is changed from the
original data in the following manor. We add  + 1 minute to the end time so that
16:00 - 16:59 (original version) becomes 16:00 - 17:00 (saved version).
\begin{quote}\begin{description}
\item[{Parameters}] \leavevmode\begin{itemize}
\item {} 
\sphinxstyleliteralstrong{\sphinxupquote{df}} (\sphinxstyleliteralemphasis{\sphinxupquote{pandas.core.frame.DataFrame}}) \textendash{} the activity-diary output of the simulation

\item {} 
\sphinxstyleliteralstrong{\sphinxupquote{fname}} (\sphinxstyleliteralemphasis{\sphinxupquote{str}}) \textendash{} the file name of the saved file. It must end with a .csv extension

\end{itemize}

\item[{Returns}] \leavevmode


\end{description}\end{quote}

\end{fulllineitems}

\index{save\_zip() (in module my\_globals)}

\begin{fulllineitems}
\phantomsection\label{\detokenize{my_globals:my_globals.save_zip}}\pysiglinewithargsret{\sphinxcode{\sphinxupquote{my\_globals.}}\sphinxbfcode{\sphinxupquote{save\_zip}}}{\emph{out\_file}, \emph{source\_dir}}{}
This function compresses an entire directory as a zip file.
\begin{quote}\begin{description}
\item[{Parameters}] \leavevmode\begin{itemize}
\item {} 
\sphinxstyleliteralstrong{\sphinxupquote{out\_file}} (\sphinxstyleliteralemphasis{\sphinxupquote{str}}) \textendash{} the filename of the save zip file without the .zip extension

\item {} 
\sphinxstyleliteralstrong{\sphinxupquote{source\_dir}} (\sphinxstyleliteralemphasis{\sphinxupquote{str}}) \textendash{} the directory to be compressed

\end{itemize}

\item[{Returns}] \leavevmode
the name of the compressed directory

\end{description}\end{quote}

\end{fulllineitems}

\index{set\_distribution() (in module my\_globals)}

\begin{fulllineitems}
\phantomsection\label{\detokenize{my_globals:my_globals.set_distribution}}\pysiglinewithargsret{\sphinxcode{\sphinxupquote{my\_globals.}}\sphinxbfcode{\sphinxupquote{set\_distribution}}}{\emph{lower}, \emph{upper}, \emph{mu}, \emph{std}}{}
This function sets the truncated normal probability distribution.
\begin{quote}\begin{description}
\item[{Parameters}] \leavevmode\begin{itemize}
\item {} 
\sphinxstyleliteralstrong{\sphinxupquote{lower}} (\sphinxstyleliteralemphasis{\sphinxupquote{int}}) \textendash{} the lower bound in number of standard deviation from the mean

\item {} 
\sphinxstyleliteralstrong{\sphinxupquote{upper}} (\sphinxstyleliteralemphasis{\sphinxupquote{int}}) \textendash{} the upper bound in number of standard deviation from the mean

\item {} 
\sphinxstyleliteralstrong{\sphinxupquote{mu}} (\sphinxstyleliteralemphasis{\sphinxupquote{int}}) \textendash{} the mean

\item {} 
\sphinxstyleliteralstrong{\sphinxupquote{std}} (\sphinxstyleliteralemphasis{\sphinxupquote{int}}) \textendash{} the standard deviation

\end{itemize}

\item[{Returns}] \leavevmode
the function for the truncated normal distribution

\end{description}\end{quote}

\end{fulllineitems}

\index{set\_distribution\_dt() (in module my\_globals)}

\begin{fulllineitems}
\phantomsection\label{\detokenize{my_globals:my_globals.set_distribution_dt}}\pysiglinewithargsret{\sphinxcode{\sphinxupquote{my\_globals.}}\sphinxbfcode{\sphinxupquote{set\_distribution\_dt}}}{\emph{lower}, \emph{upper}, \emph{mu}, \emph{std}, \emph{x\_min}}{}
This function set the truncated normal probability distribution subject to the fact that there     is an assigned lowest value.

If the lowest value of the normal distribution is lower than the lowest     allowed value, change the distribution so that the standard deviation allows the distribution to not be     lower than the lowest allowed value.
\begin{quote}\begin{description}
\item[{Parameters}] \leavevmode\begin{itemize}
\item {} 
\sphinxstyleliteralstrong{\sphinxupquote{lower}} (\sphinxstyleliteralemphasis{\sphinxupquote{int}}) \textendash{} the lower bound in number of standard deviation from the mean

\item {} 
\sphinxstyleliteralstrong{\sphinxupquote{upper}} (\sphinxstyleliteralemphasis{\sphinxupquote{int}}) \textendash{} the upper bound in number of standard deviation from the mean

\item {} 
\sphinxstyleliteralstrong{\sphinxupquote{mu}} (\sphinxstyleliteralemphasis{\sphinxupquote{int}}) \textendash{} the mean

\item {} 
\sphinxstyleliteralstrong{\sphinxupquote{std}} (\sphinxstyleliteralemphasis{\sphinxupquote{int}}) \textendash{} the standard deviation

\item {} 
\sphinxstyleliteralstrong{\sphinxupquote{x\_min}} (\sphinxstyleliteralemphasis{\sphinxupquote{int}}) \textendash{} the lowest allowed value

\end{itemize}

\item[{Returns}] \leavevmode
the function for the truncated normal distribution, the standard deviation of the distribution

\item[{Return type}] \leavevmode
tuple

\end{description}\end{quote}

\end{fulllineitems}

\index{to\_periodic() (in module my\_globals)}

\begin{fulllineitems}
\phantomsection\label{\detokenize{my_globals:my_globals.to_periodic}}\pysiglinewithargsret{\sphinxcode{\sphinxupquote{my\_globals.}}\sphinxbfcode{\sphinxupquote{to\_periodic}}}{\emph{t}, \emph{do\_hours=True}}{}
This function returns the time of day in a periodic format. It takes the time \(t \in [0, 24)\) and     expresses it at time \(x \in [-12, 12)\) where 0 represents midnight.
\begin{quote}\begin{description}
\item[{Parameters}] \leavevmode\begin{itemize}
\item {} 
\sphinxstyleliteralstrong{\sphinxupquote{t}} (\sphinxstyleliteralemphasis{\sphinxupquote{float}}) \textendash{} the time in hours {[}0, 24)

\item {} 
\sphinxstyleliteralstrong{\sphinxupquote{do\_hours}} (\sphinxstyleliteralemphasis{\sphinxupquote{bool}}) \textendash{} a flag to do the calculations in hours (if True) or minutes if False

\end{itemize}

\item[{Returns}] \leavevmode
the time in {[}-12, 12) or minutes {[}-12 * 60, 12 * 60)

\item[{Return type}] \leavevmode
float

\end{description}\end{quote}

\end{fulllineitems}



\subsection{need module}
\label{\detokenize{need::doc}}\label{\detokenize{need:module-need}}\label{\detokenize{need:need-module}}\index{need (module)}
This module contains information about governing the various needs that agents have in the simulation.

This module contains the class {\hyperref[\detokenize{need:need.Need}]{\sphinxcrossref{\sphinxcode{\sphinxupquote{need.Need}}}}}.
\index{Need (class in need)}

\begin{fulllineitems}
\phantomsection\label{\detokenize{need:need.Need}}\pysiglinewithargsret{\sphinxbfcode{\sphinxupquote{class }}\sphinxcode{\sphinxupquote{need.}}\sphinxbfcode{\sphinxupquote{Need}}}{\emph{clock}, \emph{num\_sample\_points}}{}
Bases: \sphinxcode{\sphinxupquote{object}}

This class holds general information about needs.
\begin{quote}\begin{description}
\item[{Parameters}] \leavevmode\begin{itemize}
\item {} 
\sphinxstyleliteralstrong{\sphinxupquote{clock}} ({\hyperref[\detokenize{temporal:temporal.Temporal}]{\sphinxcrossref{\sphinxstyleliteralemphasis{\sphinxupquote{temporal.Temporal}}}}}) \textendash{} the clock governing time in the simulation

\item {} 
\sphinxstyleliteralstrong{\sphinxupquote{num\_sample\_points}} (\sphinxstyleliteralemphasis{\sphinxupquote{int}}) \textendash{} the number of time nodes in the simulation

\end{itemize}

\item[{Variables}] \leavevmode\begin{itemize}
\item {} 
\sphinxstyleliteralstrong{\sphinxupquote{category}} (\sphinxstyleliteralemphasis{\sphinxupquote{int}}) \textendash{} the need- identifier

\item {} 
\sphinxstyleliteralstrong{\sphinxupquote{clock}} ({\hyperref[\detokenize{temporal:temporal.Temporal}]{\sphinxcrossref{\sphinxstyleliteralemphasis{\sphinxupquote{temporal.Temporal}}}}}) \textendash{} keeps track of the time

\item {} 
\sphinxstyleliteralstrong{\sphinxupquote{history}} (\sphinxstyleliteralemphasis{\sphinxupquote{float}}) \textendash{} an array containing the magnitude level \([0,1]\) of the need at all                         sample times.

\item {} 
\sphinxstyleliteralstrong{\sphinxupquote{magnitude}} (\sphinxstyleliteralemphasis{\sphinxupquote{float}}) \textendash{} the magnitude of the need (the satiation)

\item {} 
\sphinxstyleliteralstrong{\sphinxupquote{t0}} (\sphinxstyleliteralemphasis{\sphinxupquote{int}}) \textendash{} this keeps track of the last time the need was addressed

\item {} 
\sphinxstyleliteralstrong{\sphinxupquote{threshold}} (\sphinxstyleliteralemphasis{\sphinxupquote{float}}) \textendash{} the threshold of the need

\end{itemize}

\end{description}\end{quote}
\index{decay() (need.Need method)}

\begin{fulllineitems}
\phantomsection\label{\detokenize{need:need.Need.decay}}\pysiglinewithargsret{\sphinxbfcode{\sphinxupquote{decay}}}{}{}
This calculates the amount of decay over a time step.

\begin{sphinxadmonition}{note}{Note:}
This function is meant to be overridden.
\end{sphinxadmonition}
\begin{quote}\begin{description}
\item[{Returns}] \leavevmode
None

\end{description}\end{quote}

\end{fulllineitems}

\index{initialize() (need.Need method)}

\begin{fulllineitems}
\phantomsection\label{\detokenize{need:need.Need.initialize}}\pysiglinewithargsret{\sphinxbfcode{\sphinxupquote{initialize}}}{}{}
This function initializes the state of the need at the very beginning of simulation.

\begin{sphinxadmonition}{note}{Note:}
This function is meant to be overridden.
\end{sphinxadmonition}
\begin{quote}\begin{description}
\item[{Returns}] \leavevmode
None

\end{description}\end{quote}

\end{fulllineitems}

\index{print\_category() (need.Need method)}

\begin{fulllineitems}
\phantomsection\label{\detokenize{need:need.Need.print_category}}\pysiglinewithargsret{\sphinxbfcode{\sphinxupquote{print\_category}}}{}{}
This function represents the category as a string.
\begin{quote}\begin{description}
\item[{Returns}] \leavevmode
the string representation of the category

\item[{Return type}] \leavevmode
str

\end{description}\end{quote}

\end{fulllineitems}

\index{reset() (need.Need method)}

\begin{fulllineitems}
\phantomsection\label{\detokenize{need:need.Need.reset}}\pysiglinewithargsret{\sphinxbfcode{\sphinxupquote{reset}}}{}{}
This function resets the values in order for the need to be used in the next simulation. This function         does the following:
\begin{enumerate}
\item {} 
sets the satiation to 1.0

\item {} 
sets the history to zero

\end{enumerate}
\begin{quote}\begin{description}
\item[{Returns}] \leavevmode
None

\end{description}\end{quote}

\end{fulllineitems}

\index{toString() (need.Need method)}

\begin{fulllineitems}
\phantomsection\label{\detokenize{need:need.Need.toString}}\pysiglinewithargsret{\sphinxbfcode{\sphinxupquote{toString}}}{}{}
This function represents the Need object as a string.
\begin{quote}\begin{description}
\item[{Return msg}] \leavevmode
the string representation of the Need object

\item[{Return type}] \leavevmode
str

\end{description}\end{quote}

\end{fulllineitems}

\index{under\_threshold() (need.Need method)}

\begin{fulllineitems}
\phantomsection\label{\detokenize{need:need.Need.under_threshold}}\pysiglinewithargsret{\sphinxbfcode{\sphinxupquote{under\_threshold}}}{\emph{n}}{}
Compares the value of anNeed’s satiation to the threshold.
\begin{quote}\begin{description}
\item[{Parameters}] \leavevmode
\sphinxstyleliteralstrong{\sphinxupquote{n}} (\sphinxstyleliteralemphasis{\sphinxupquote{float}}) \textendash{} the satiation

\item[{Returns}] \leavevmode
True if the satiation is less than or equal to the threshold, False otherwise

\item[{Return type}] \leavevmode
bool

\end{description}\end{quote}

\end{fulllineitems}

\index{weight() (need.Need method)}

\begin{fulllineitems}
\phantomsection\label{\detokenize{need:need.Need.weight}}\pysiglinewithargsret{\sphinxbfcode{\sphinxupquote{weight}}}{\emph{n}}{}
This function calculates the weight function of a need.
\begin{quote}\begin{description}
\item[{Parameters}] \leavevmode
\sphinxstyleliteralstrong{\sphinxupquote{n}} (\sphinxstyleliteralemphasis{\sphinxupquote{float}}) \textendash{} the satiation

\item[{Returns}] \leavevmode
the weight due to the  need

\item[{Return type}] \leavevmode
float

\end{description}\end{quote}

\end{fulllineitems}


\end{fulllineitems}



\subsection{occupation module}
\label{\detokenize{occupation::doc}}\label{\detokenize{occupation:occupation-module}}\label{\detokenize{occupation:module-occupation}}\index{occupation (module)}
This module contains info needed for the occupation of a person. In addition, this file also contains functions useful for the module itself.This module contains constants relevant to the occupational information:
\begin{itemize}
\item {} 
job identifiers

\item {} 
job categories

\item {} 
default start time information

\item {} 
default end time information

\item {} 
default commuting to work information

\item {} 
default commuting from work information

\item {} 
default summer vacation (from school) information

\end{itemize}

This module contains class {\hyperref[\detokenize{occupation:occupation.Occupation}]{\sphinxcrossref{\sphinxcode{\sphinxupquote{occupation.Occupation}}}}}.
\index{Occupation (class in occupation)}

\begin{fulllineitems}
\phantomsection\label{\detokenize{occupation:occupation.Occupation}}\pysigline{\sphinxbfcode{\sphinxupquote{class }}\sphinxcode{\sphinxupquote{occupation.}}\sphinxbfcode{\sphinxupquote{Occupation}}}
Bases: \sphinxcode{\sphinxupquote{object}}

This class contains information relevant to an occupation of a Person.
\begin{quote}\begin{description}
\item[{Variables}] \leavevmode\begin{itemize}
\item {} 
\sphinxstyleliteralstrong{\sphinxupquote{category}} (\sphinxstyleliteralemphasis{\sphinxupquote{int}}) \textendash{} the category of the job

\item {} 
\sphinxstyleliteralstrong{\sphinxupquote{id}} (\sphinxstyleliteralemphasis{\sphinxupquote{int}}) \textendash{} the identifier for the job

\item {} 
\sphinxstyleliteralstrong{\sphinxupquote{commute\_to\_work\_dt\_mean}} (\sphinxstyleliteralemphasis{\sphinxupquote{int}}) \textendash{} the mean duration to commute to work {[}minutes{]}

\item {} 
\sphinxstyleliteralstrong{\sphinxupquote{commute\_to\_work\_dt\_std}} (\sphinxstyleliteralemphasis{\sphinxupquote{int}}) \textendash{} the standard deviation to commute to work {[}minutes{]}

\item {} 
\sphinxstyleliteralstrong{\sphinxupquote{commute\_to\_work\_dt}} (\sphinxstyleliteralemphasis{\sphinxupquote{int}}) \textendash{} the duration to commute to work {[}minutes{]}

\item {} 
\sphinxstyleliteralstrong{\sphinxupquote{commute\_to\_work\_dt\_trunc}} (\sphinxstyleliteralemphasis{\sphinxupquote{int}}) \textendash{} the number of standard deviation in the commute to work     duration distribution

\item {} 
\sphinxstyleliteralstrong{\sphinxupquote{commute\_to\_work\_start}} (\sphinxstyleliteralemphasis{\sphinxupquote{int}}) \textendash{} the start time for the commute to work activity {[}minutes{]}

\item {} 
\sphinxstyleliteralstrong{\sphinxupquote{dt\_commute}} (\sphinxstyleliteralemphasis{\sphinxupquote{int}}) \textendash{} the duration of the commute {[}minutes{]}

\item {} 
\sphinxstyleliteralstrong{\sphinxupquote{dt}} (\sphinxstyleliteralemphasis{\sphinxupquote{int}}) \textendash{} the duration of the work activity {[}minutes{]}

\item {} 
\sphinxstyleliteralstrong{\sphinxupquote{commute\_from\_work\_dt\_mean}} (\sphinxstyleliteralemphasis{\sphinxupquote{int}}) \textendash{} the mean duration to commute from work {[}minutes{]}

\item {} 
\sphinxstyleliteralstrong{\sphinxupquote{commute\_from\_work\_dt\_std}} (\sphinxstyleliteralemphasis{\sphinxupquote{int}}) \textendash{} the standard deviation to commute from work {[}minutes{]}

\item {} 
\sphinxstyleliteralstrong{\sphinxupquote{commute\_from\_work\_dt}} (\sphinxstyleliteralemphasis{\sphinxupquote{int}}) \textendash{} the duration to commute from work {[}minutes{]}

\item {} 
\sphinxstyleliteralstrong{\sphinxupquote{commute\_from\_work\_dt\_trunc}} (\sphinxstyleliteralemphasis{\sphinxupquote{int}}) \textendash{} the number of standard deviations in the commute from work     duration distribution

\item {} 
\sphinxstyleliteralstrong{\sphinxupquote{t\_start\_mean}} (\sphinxstyleliteralemphasis{\sphinxupquote{int}}) \textendash{} the mean start time for the job {[}minutes, time of day{]}

\item {} 
\sphinxstyleliteralstrong{\sphinxupquote{t\_start\_std}} (\sphinxstyleliteralemphasis{\sphinxupquote{int}}) \textendash{} the standard deviation of the start time for the job

\item {} 
\sphinxstyleliteralstrong{\sphinxupquote{t\_start}} (\sphinxstyleliteralemphasis{\sphinxupquote{int}}) \textendash{} the start time for the job {[}minutes, time of day{]}

\item {} 
\sphinxstyleliteralstrong{\sphinxupquote{t\_start\_univ}} (\sphinxstyleliteralemphasis{\sphinxupquote{int}}) \textendash{} the start time for the job {[}minutes, universal time{]}

\item {} 
\sphinxstyleliteralstrong{\sphinxupquote{work\_start\_trunc}} (\sphinxstyleliteralemphasis{\sphinxupquote{int}}) \textendash{} the number of standard deviations in the work start time distribution

\item {} 
\sphinxstyleliteralstrong{\sphinxupquote{day\_start}} (\sphinxstyleliteralemphasis{\sphinxupquote{int}}) \textendash{} the day the work activity start {[}minutes{]}

\item {} 
\sphinxstyleliteralstrong{\sphinxupquote{t\_end\_mean}} (\sphinxstyleliteralemphasis{\sphinxupquote{int}}) \textendash{} the mean end time for the job {[}minutes, time of day{]}

\item {} 
\sphinxstyleliteralstrong{\sphinxupquote{t\_end\_std}} (\sphinxstyleliteralemphasis{\sphinxupquote{int}}) \textendash{} the standard deviation of the end time for the job

\item {} 
\sphinxstyleliteralstrong{\sphinxupquote{t\_end}} (\sphinxstyleliteralemphasis{\sphinxupquote{int}}) \textendash{} the end time for the job {[}minutes, time of day{]}

\item {} 
\sphinxstyleliteralstrong{\sphinxupquote{t\_end\_univ}} (\sphinxstyleliteralemphasis{\sphinxupquote{int}}) \textendash{} the end time for the job {[}minutes, universal time{]}

\item {} 
\sphinxstyleliteralstrong{\sphinxupquote{work\_end\_trunc}} (\sphinxstyleliteralemphasis{\sphinxupquote{int}}) \textendash{} the number of standard deviations in the work end time distribution

\item {} 
\sphinxstyleliteralstrong{\sphinxupquote{is\_employed}} (\sphinxstyleliteralemphasis{\sphinxupquote{bool}}) \textendash{} a flag saying whether the agent is employed (True) or not (False)

\item {} 
\sphinxstyleliteralstrong{\sphinxupquote{is\_same\_day}} (\sphinxstyleliteralemphasis{\sphinxupquote{bool}}) \textendash{} a flag to see whether the start time and end time of a job are                             on the same day. If so, True. Else, False. If a person has \sphinxcode{\sphinxupquote{NO\_JOB}}, the flag                             is set to True

\item {} 
\sphinxstyleliteralstrong{\sphinxupquote{'location'}} ({\hyperref[\detokenize{location:location.Location}]{\sphinxcrossref{\sphinxstyleliteralemphasis{\sphinxupquote{location.Location}}}}}) \textendash{} the location of the Occupation

\item {} 
\sphinxstyleliteralstrong{\sphinxupquote{wage}} (\sphinxstyleliteralemphasis{\sphinxupquote{float}}) \textendash{} the yearly wage for that job {[}U.S. dollars{]}

\item {} 
\sphinxstyleliteralstrong{\sphinxupquote{work\_days}} (\sphinxstyleliteralemphasis{\sphinxupquote{list}}) \textendash{} a list of ints, giving the days the job starts

\item {} 
\sphinxstyleliteralstrong{\sphinxupquote{f\_commute\_to\_work\_dt}} \textendash{} the commute to work duration distribution

\item {} 
\sphinxstyleliteralstrong{\sphinxupquote{f\_commute\_from\_work\_dt}} \textendash{} the commute from work duration distribution

\item {} 
\sphinxstyleliteralstrong{\sphinxupquote{f\_work\_start}} \textendash{} the work start time distribution

\item {} 
\sphinxstyleliteralstrong{\sphinxupquote{f\_work\_end}} \textendash{} the work end time distribution

\end{itemize}

\end{description}\end{quote}
\index{is\_summer\_vacation() (occupation.Occupation method)}

\begin{fulllineitems}
\phantomsection\label{\detokenize{occupation:occupation.Occupation.is_summer_vacation}}\pysiglinewithargsret{\sphinxbfcode{\sphinxupquote{is\_summer\_vacation}}}{\emph{week\_of\_year}}{}
This function returns True if the agent should not go to school due to summer vacation. False, otherwise.
\begin{quote}\begin{description}
\item[{Parameters}] \leavevmode
\sphinxstyleliteralstrong{\sphinxupquote{week\_of\_year}} (\sphinxstyleliteralemphasis{\sphinxupquote{int}}) \textendash{} the week of the year

\item[{Returns}] \leavevmode


\end{description}\end{quote}

\end{fulllineitems}

\index{print\_category() (occupation.Occupation method)}

\begin{fulllineitems}
\phantomsection\label{\detokenize{occupation:occupation.Occupation.print_category}}\pysiglinewithargsret{\sphinxbfcode{\sphinxupquote{print\_category}}}{}{}
This function represents the Occupation category as a string
\begin{quote}\begin{description}
\item[{Returns}] \leavevmode
the string representation of a Occupation category

\item[{Return type}] \leavevmode
str

\end{description}\end{quote}

\end{fulllineitems}

\index{print\_id() (occupation.Occupation method)}

\begin{fulllineitems}
\phantomsection\label{\detokenize{occupation:occupation.Occupation.print_id}}\pysiglinewithargsret{\sphinxbfcode{\sphinxupquote{print\_id}}}{}{}
This function writes the Occupation id as a string
\begin{quote}\begin{description}
\item[{Returns}] \leavevmode
a string representation of the job ID

\item[{Return type}] \leavevmode
str

\end{description}\end{quote}

\end{fulllineitems}

\index{set\_commute\_distribution() (occupation.Occupation method)}

\begin{fulllineitems}
\phantomsection\label{\detokenize{occupation:occupation.Occupation.set_commute_distribution}}\pysiglinewithargsret{\sphinxbfcode{\sphinxupquote{set\_commute\_distribution}}}{}{}
This function sets the following:
\begin{itemize}
\item {} 
commute to work duration distribution

\item {} 
commute from work duration distribution.

\end{itemize}
\begin{quote}\begin{description}
\item[{Returns}] \leavevmode
None

\end{description}\end{quote}

\end{fulllineitems}

\index{set\_is\_job() (occupation.Occupation method)}

\begin{fulllineitems}
\phantomsection\label{\detokenize{occupation:occupation.Occupation.set_is_job}}\pysiglinewithargsret{\sphinxbfcode{\sphinxupquote{set\_is\_job}}}{}{}
This function checks to see if the current job is actually a job (eg. that it is not         \sphinxcode{\sphinxupquote{NO\_JOB}}).

Sets self.is\_job to True if the Occupation is \sphinxcode{\sphinxupquote{NO\_JOB}}, returns False otherwise
\begin{quote}\begin{description}
\item[{Returns}] \leavevmode
None

\end{description}\end{quote}

\end{fulllineitems}

\index{set\_is\_same\_day() (occupation.Occupation method)}

\begin{fulllineitems}
\phantomsection\label{\detokenize{occupation:occupation.Occupation.set_is_same_day}}\pysiglinewithargsret{\sphinxbfcode{\sphinxupquote{set\_is\_same\_day}}}{}{}
This function sets a flag indicating whether or not a job starts and ends         on the same day. The function sets \sphinxcode{\sphinxupquote{is\_same\_day}} to True if the Occupation start time and end time         are within the same day. False, otherwise.
\begin{quote}\begin{description}
\item[{Returns}] \leavevmode
None

\end{description}\end{quote}

\end{fulllineitems}

\index{set\_job\_params() (occupation.Occupation method)}

\begin{fulllineitems}
\phantomsection\label{\detokenize{occupation:occupation.Occupation.set_job_params}}\pysiglinewithargsret{\sphinxbfcode{\sphinxupquote{set\_job\_params}}}{\emph{id\_job}, \emph{start\_mean}, \emph{start\_std}, \emph{end\_mean}, \emph{end\_std}, \emph{commute\_to\_work\_dt\_mean}, \emph{commute\_to\_work\_dt\_std}, \emph{commute\_from\_work\_dt\_mean}, \emph{commute\_from\_work\_dt\_std}}{}
This function sets the Occupation parameters.
\begin{quote}\begin{description}
\item[{Parameters}] \leavevmode\begin{itemize}
\item {} 
\sphinxstyleliteralstrong{\sphinxupquote{id\_job}} (\sphinxstyleliteralemphasis{\sphinxupquote{int}}) \textendash{} the job identifier

\item {} 
\sphinxstyleliteralstrong{\sphinxupquote{start\_mean}} (\sphinxstyleliteralemphasis{\sphinxupquote{int}}) \textendash{} the mean start time for the occupation

\item {} 
\sphinxstyleliteralstrong{\sphinxupquote{start\_std}} (\sphinxstyleliteralemphasis{\sphinxupquote{int}}) \textendash{} the standard deviation of the start time for the occupation

\item {} 
\sphinxstyleliteralstrong{\sphinxupquote{end\_mean}} (\sphinxstyleliteralemphasis{\sphinxupquote{int}}) \textendash{} the mean end time for the occupation

\item {} 
\sphinxstyleliteralstrong{\sphinxupquote{end\_std}} (\sphinxstyleliteralemphasis{\sphinxupquote{int}}) \textendash{} the standard deviation for the end time

\item {} 
\sphinxstyleliteralstrong{\sphinxupquote{commute\_to\_work\_dt\_mean}} (\sphinxstyleliteralemphasis{\sphinxupquote{int}}) \textendash{} the mean commute to work duration

\item {} 
\sphinxstyleliteralstrong{\sphinxupquote{commute\_to\_work\_dt\_std}} (\sphinxstyleliteralemphasis{\sphinxupquote{int}}) \textendash{} the standard deviation of the commute to work duration

\item {} 
\sphinxstyleliteralstrong{\sphinxupquote{commute\_from\_work\_dt\_mean}} (\sphinxstyleliteralemphasis{\sphinxupquote{int}}) \textendash{} the mean commute from work duration

\item {} 
\sphinxstyleliteralstrong{\sphinxupquote{commute\_from\_work\_dt\_std}} (\sphinxstyleliteralemphasis{\sphinxupquote{int}}) \textendash{} the standard deviation to commute from work duration

\end{itemize}

\item[{Returns}] \leavevmode
None

\end{description}\end{quote}

\end{fulllineitems}

\index{set\_job\_preset() (occupation.Occupation method)}

\begin{fulllineitems}
\phantomsection\label{\detokenize{occupation:occupation.Occupation.set_job_preset}}\pysiglinewithargsret{\sphinxbfcode{\sphinxupquote{set\_job\_preset}}}{}{}
Sets Occupation to one of the following preset jobs:
\begin{itemize}
\item {} 
\sphinxcode{\sphinxupquote{NO\_JOB}}, the agent is unemployed

\item {} 
\sphinxcode{\sphinxupquote{STANDARD\_JOB}}, the agent has a job

\item {} 
\sphinxcode{\sphinxupquote{STUDENT}}, the agent attends school (not including college / university)

\end{itemize}
\begin{quote}\begin{description}
\item[{Returns}] \leavevmode
None

\end{description}\end{quote}

\end{fulllineitems}

\index{set\_no\_job() (occupation.Occupation method)}

\begin{fulllineitems}
\phantomsection\label{\detokenize{occupation:occupation.Occupation.set_no_job}}\pysiglinewithargsret{\sphinxbfcode{\sphinxupquote{set\_no\_job}}}{}{}
Set the Occupation to having no job.
\begin{quote}\begin{description}
\item[{Parameters}] \leavevmode
\sphinxstyleliteralstrong{\sphinxupquote{job}} ({\hyperref[\detokenize{occupation:occupation.Occupation}]{\sphinxcrossref{\sphinxstyleliteralemphasis{\sphinxupquote{occupation.Occupation}}}}}) \textendash{} the job of which to set the attributes

\item[{Returns}] \leavevmode
None

\end{description}\end{quote}

\end{fulllineitems}

\index{set\_standard\_job() (occupation.Occupation method)}

\begin{fulllineitems}
\phantomsection\label{\detokenize{occupation:occupation.Occupation.set_standard_job}}\pysiglinewithargsret{\sphinxbfcode{\sphinxupquote{set\_standard\_job}}}{}{}
This function sets the Occupation to the default job. The job has the following         characteristics:
\begin{itemize}
\item {} 
9:00 - 17:00

\item {} 
Monday through Friday

\item {} 
wage of \$40,000

\item {} 
30 minute commute to work

\item {} 
60 minute commute from work

\end{itemize}
\begin{quote}\begin{description}
\item[{Parameters}] \leavevmode
\sphinxstyleliteralstrong{\sphinxupquote{job}} ({\hyperref[\detokenize{occupation:occupation.Occupation}]{\sphinxcrossref{\sphinxstyleliteralemphasis{\sphinxupquote{occupation.Occupation}}}}}) \textendash{} the job of which to set the attributes

\item[{Returns}] \leavevmode
None

\end{description}\end{quote}

\end{fulllineitems}

\index{set\_student() (occupation.Occupation method)}

\begin{fulllineitems}
\phantomsection\label{\detokenize{occupation:occupation.Occupation.set_student}}\pysiglinewithargsret{\sphinxbfcode{\sphinxupquote{set\_student}}}{}{}
This function sets the Occupation to the default schooling behavior. This         has the following characteristics:
\begin{itemize}
\item {} 
8:00 - 15:00

\item {} 
Monday through Friday

\item {} 
wage of \$0

\item {} 
30 minute commute to school

\item {} 
60 minute commute from school

\end{itemize}
\begin{quote}\begin{description}
\item[{Parameters}] \leavevmode
\sphinxstyleliteralstrong{\sphinxupquote{job}} ({\hyperref[\detokenize{occupation:occupation.Occupation}]{\sphinxcrossref{\sphinxstyleliteralemphasis{\sphinxupquote{occupation.Occupation}}}}}) \textendash{} the job of which to set the attributes

\item[{Returns}] \leavevmode
None

\end{description}\end{quote}

\end{fulllineitems}

\index{set\_work\_distribution() (occupation.Occupation method)}

\begin{fulllineitems}
\phantomsection\label{\detokenize{occupation:occupation.Occupation.set_work_distribution}}\pysiglinewithargsret{\sphinxbfcode{\sphinxupquote{set\_work\_distribution}}}{}{}
This function sets the following distributions for work:
\begin{itemize}
\item {} 
work start time distribution

\item {} 
work end time distribution

\end{itemize}
\begin{quote}\begin{description}
\item[{Returns}] \leavevmode
None

\end{description}\end{quote}

\end{fulllineitems}

\index{toString() (occupation.Occupation method)}

\begin{fulllineitems}
\phantomsection\label{\detokenize{occupation:occupation.Occupation.toString}}\pysiglinewithargsret{\sphinxbfcode{\sphinxupquote{toString}}}{}{}
Represents the Occupation object as a string
\begin{quote}\begin{description}
\item[{Return msg}] \leavevmode
The representation of the Occupation object as a string

\item[{Return type}] \leavevmode
str

\end{description}\end{quote}

\end{fulllineitems}

\index{update\_commute\_from\_work\_dt() (occupation.Occupation method)}

\begin{fulllineitems}
\phantomsection\label{\detokenize{occupation:occupation.Occupation.update_commute_from_work_dt}}\pysiglinewithargsret{\sphinxbfcode{\sphinxupquote{update\_commute\_from\_work\_dt}}}{}{}
Pull a commute from work duration from the respective distribution.
\begin{quote}\begin{description}
\item[{Returns}] \leavevmode
None

\end{description}\end{quote}

\end{fulllineitems}

\index{update\_commute\_to\_work\_dt() (occupation.Occupation method)}

\begin{fulllineitems}
\phantomsection\label{\detokenize{occupation:occupation.Occupation.update_commute_to_work_dt}}\pysiglinewithargsret{\sphinxbfcode{\sphinxupquote{update\_commute\_to\_work\_dt}}}{}{}
Pull a commute to work duration from the respective distribution. Also, update the commute to work         start time place holder.
\begin{quote}\begin{description}
\item[{Returns}] \leavevmode
None

\end{description}\end{quote}

\end{fulllineitems}

\index{update\_commute\_to\_work\_start() (occupation.Occupation method)}

\begin{fulllineitems}
\phantomsection\label{\detokenize{occupation:occupation.Occupation.update_commute_to_work_start}}\pysiglinewithargsret{\sphinxbfcode{\sphinxupquote{update\_commute\_to\_work\_start}}}{}{}
Update the commute to work start time.
\begin{quote}\begin{description}
\item[{Returns}] \leavevmode
None

\end{description}\end{quote}

\end{fulllineitems}

\index{update\_work\_dt() (occupation.Occupation method)}

\begin{fulllineitems}
\phantomsection\label{\detokenize{occupation:occupation.Occupation.update_work_dt}}\pysiglinewithargsret{\sphinxbfcode{\sphinxupquote{update\_work\_dt}}}{}{}
Update the work duration
\begin{quote}\begin{description}
\item[{Returns}] \leavevmode
None

\end{description}\end{quote}

\end{fulllineitems}

\index{update\_work\_end() (occupation.Occupation method)}

\begin{fulllineitems}
\phantomsection\label{\detokenize{occupation:occupation.Occupation.update_work_end}}\pysiglinewithargsret{\sphinxbfcode{\sphinxupquote{update\_work\_end}}}{}{}
Update the work end time.
\begin{quote}\begin{description}
\item[{Returns}] \leavevmode
None

\end{description}\end{quote}

\end{fulllineitems}

\index{update\_work\_start() (occupation.Occupation method)}

\begin{fulllineitems}
\phantomsection\label{\detokenize{occupation:occupation.Occupation.update_work_start}}\pysiglinewithargsret{\sphinxbfcode{\sphinxupquote{update\_work\_start}}}{}{}
Update the work start time.
\begin{quote}\begin{description}
\item[{Returns}] \leavevmode
None

\end{description}\end{quote}

\end{fulllineitems}


\end{fulllineitems}

\index{is\_work\_time() (in module occupation)}

\begin{fulllineitems}
\phantomsection\label{\detokenize{occupation:occupation.is_work_time}}\pysiglinewithargsret{\sphinxcode{\sphinxupquote{occupation.}}\sphinxbfcode{\sphinxupquote{is\_work\_time}}}{\emph{clock}, \emph{job}, \emph{is\_commute\_to\_work=False}}{}
Given a clock and a job, this function says whether the clock’s time corresponds     to a time to be at work \sphinxstylestrong{or} a time to commute to work.

If \(\Delta{t} > 0\), it indicates when it’s time to commute to work.
\begin{quote}\begin{description}
\item[{Parameters}] \leavevmode\begin{itemize}
\item {} 
\sphinxstyleliteralstrong{\sphinxupquote{clock}} ({\hyperref[\detokenize{temporal:temporal.Temporal}]{\sphinxcrossref{\sphinxstyleliteralemphasis{\sphinxupquote{temporal.Temporal}}}}}) \textendash{} the time

\item {} 
\sphinxstyleliteralstrong{\sphinxupquote{job}} ({\hyperref[\detokenize{occupation:occupation.Occupation}]{\sphinxcrossref{\sphinxstyleliteralemphasis{\sphinxupquote{occupation.Occupation}}}}}) \textendash{} the job to inquiry

\item {} 
\sphinxstyleliteralstrong{\sphinxupquote{is\_commute\_to\_work}} (\sphinxstyleliteralemphasis{\sphinxupquote{bool}}) \textendash{} a flag indicating whether we are interested in calculating if it is                             time to commute to work

\end{itemize}

\item[{Returns}] \leavevmode
a flag indicating if it is / is not work time (or commute time if is\_commute\_to\_work is True)

\item[{Return type}] \leavevmode
bool

\end{description}\end{quote}

\end{fulllineitems}

\index{is\_work\_time\_help() (in module occupation)}

\begin{fulllineitems}
\phantomsection\label{\detokenize{occupation:occupation.is_work_time_help}}\pysiglinewithargsret{\sphinxcode{\sphinxupquote{occupation.}}\sphinxbfcode{\sphinxupquote{is\_work\_time\_help}}}{\emph{clock}, \emph{job}}{}
Given a clock and a job, this function says whether the clock’s time corresponds to
a time at work.
\begin{quote}\begin{description}
\item[{Parameters}] \leavevmode\begin{itemize}
\item {} 
\sphinxstyleliteralstrong{\sphinxupquote{clock}} ({\hyperref[\detokenize{temporal:temporal.Temporal}]{\sphinxcrossref{\sphinxstyleliteralemphasis{\sphinxupquote{temporal.Temporal}}}}}) \textendash{} the time

\item {} 
\sphinxstyleliteralstrong{\sphinxupquote{job}} ({\hyperref[\detokenize{occupation:occupation.Occupation}]{\sphinxcrossref{\sphinxstyleliteralemphasis{\sphinxupquote{occupation.Occupation}}}}}) \textendash{} the job to inquiry

\end{itemize}

\item[{Returns}] \leavevmode
is\_work\_time: a flag indicating if the time (clock) corresponds to a work time

\item[{Return type}] \leavevmode
bool

\end{description}\end{quote}

\end{fulllineitems}

\index{set\_grave\_shift() (in module occupation)}

\begin{fulllineitems}
\phantomsection\label{\detokenize{occupation:occupation.set_grave_shift}}\pysiglinewithargsret{\sphinxcode{\sphinxupquote{occupation.}}\sphinxbfcode{\sphinxupquote{set\_grave\_shift}}}{\emph{job}}{}
This function sets the Occupation to a grave shift.
\begin{itemize}
\item {} 
from  22:00 to 6:00

\item {} 
Monday through Friday

\item {} 
30 minute commute to work

\item {} 
60 minute commute from work

\item {} 
wage of \$40,0000.

\end{itemize}
\begin{quote}\begin{description}
\item[{Parameters}] \leavevmode
\sphinxstyleliteralstrong{\sphinxupquote{job}} ({\hyperref[\detokenize{occupation:occupation.Occupation}]{\sphinxcrossref{\sphinxstyleliteralemphasis{\sphinxupquote{occupation.Occupation}}}}}) \textendash{} the job of which to set the attributes

\item[{Returns}] \leavevmode
None

\end{description}\end{quote}

\end{fulllineitems}

\index{set\_no\_job() (in module occupation)}

\begin{fulllineitems}
\phantomsection\label{\detokenize{occupation:occupation.set_no_job}}\pysiglinewithargsret{\sphinxcode{\sphinxupquote{occupation.}}\sphinxbfcode{\sphinxupquote{set\_no\_job}}}{\emph{job}}{}
Set the Occupation to having no job.
\begin{quote}\begin{description}
\item[{Parameters}] \leavevmode
\sphinxstyleliteralstrong{\sphinxupquote{job}} ({\hyperref[\detokenize{occupation:occupation.Occupation}]{\sphinxcrossref{\sphinxstyleliteralemphasis{\sphinxupquote{occupation.Occupation}}}}}) \textendash{} the job of which to set the attributes

\item[{Returns}] \leavevmode
None

\end{description}\end{quote}

\end{fulllineitems}

\index{set\_standard\_job() (in module occupation)}

\begin{fulllineitems}
\phantomsection\label{\detokenize{occupation:occupation.set_standard_job}}\pysiglinewithargsret{\sphinxcode{\sphinxupquote{occupation.}}\sphinxbfcode{\sphinxupquote{set\_standard\_job}}}{\emph{job}}{}
This function sets the Occupation to the standard job.
\begin{itemize}
\item {} 
9:00 - 17:00

\item {} 
Monday through Friday

\item {} 
wage \$40,000

\item {} 
30 minute commute to work

\item {} 
60 minute commute from work

\end{itemize}
\begin{quote}\begin{description}
\item[{Parameters}] \leavevmode
\sphinxstyleliteralstrong{\sphinxupquote{job}} ({\hyperref[\detokenize{occupation:occupation.Occupation}]{\sphinxcrossref{\sphinxstyleliteralemphasis{\sphinxupquote{occupation.Occupation}}}}}) \textendash{} the job of which to set the attributes

\item[{Returns}] \leavevmode
None

\end{description}\end{quote}

\end{fulllineitems}

\index{set\_student() (in module occupation)}

\begin{fulllineitems}
\phantomsection\label{\detokenize{occupation:occupation.set_student}}\pysiglinewithargsret{\sphinxcode{\sphinxupquote{occupation.}}\sphinxbfcode{\sphinxupquote{set\_student}}}{\emph{job}}{}
This function sets a job to the preset values of student occupation.
\begin{itemize}
\item {} 
08:00 - 15:00

\item {} 
Monday through Friday

\item {} 
wage of \$0

\item {} 
30 minute commute to school

\item {} 
60 minute commute from school

\end{itemize}
\begin{quote}\begin{description}
\item[{Parameters}] \leavevmode
\sphinxstyleliteralstrong{\sphinxupquote{job}} ({\hyperref[\detokenize{occupation:occupation.Occupation}]{\sphinxcrossref{\sphinxstyleliteralemphasis{\sphinxupquote{occupation.Occupation}}}}}) \textendash{} the job to set

\item[{Returns}] \leavevmode
None

\end{description}\end{quote}

\end{fulllineitems}



\subsection{params module}
\label{\detokenize{params::doc}}\label{\detokenize{params:params-module}}\label{\detokenize{params:module-params}}\index{params (module)}
The purpose of this module is to assign parameters necessary to run the Agent-Based Model of Human Activity Patterns (ABMHAP). This module also have constants used in default runs.

This module contains class {\hyperref[\detokenize{params:params.Params}]{\sphinxcrossref{\sphinxcode{\sphinxupquote{params.Params}}}}}.
\index{Params (class in params)}

\begin{fulllineitems}
\phantomsection\label{\detokenize{params:params.Params}}\pysiglinewithargsret{\sphinxbfcode{\sphinxupquote{class }}\sphinxcode{\sphinxupquote{params.}}\sphinxbfcode{\sphinxupquote{Params}}}{\emph{num\_people}, \emph{num\_days}, \emph{num\_hours}, \emph{num\_min}, \emph{do\_minute\_by\_minute}, \emph{t\_start=960}, \emph{dt=1}, \emph{gender=None}, \emph{sleep\_start\_mean=None}, \emph{sleep\_start\_std=None}, \emph{sleep\_end\_mean=None}, \emph{sleep\_end\_std=None}, \emph{job\_id=None}, \emph{do\_alarm=None}, \emph{dt\_alarm=None}, \emph{bf\_start\_mean=None}, \emph{bf\_start\_std=None}, \emph{bf\_start\_trunc=None}, \emph{bf\_dt\_mean=None}, \emph{bf\_dt\_std=None}, \emph{bf\_dt\_trunc=None}, \emph{lunch\_start\_mean=None}, \emph{lunch\_start\_std=None}, \emph{lunch\_start\_trunc=None}, \emph{lunch\_dt\_mean=None}, \emph{lunch\_dt\_std=None}, \emph{lunch\_dt\_trunc=None}, \emph{dinner\_start\_mean=None}, \emph{dinner\_start\_std=None}, \emph{dinner\_start\_trunc=None}, \emph{dinner\_dt\_mean=None}, \emph{dinner\_dt\_std=None}, \emph{dinner\_dt\_trunc=None}, \emph{work\_start\_mean=None}, \emph{work\_start\_std=None}, \emph{work\_end\_mean=None}, \emph{work\_end\_std=None}, \emph{commute\_to\_work\_dt\_mean=None}, \emph{commute\_to\_work\_dt\_std=None}, \emph{commute\_from\_work\_dt\_mean=None}, \emph{commute\_from\_work\_dt\_std=None}}{}
Bases: \sphinxcode{\sphinxupquote{object}}

This class contains the parameters that are needed to parametrize a household.

\begin{sphinxadmonition}{note}{Note:}
Some of the class attributes are \sphinxstylestrong{not} really used and need to be phased out in future versions of the         model. Some of these attribtues are:
\begin{itemize}
\item {} 
\sphinxcode{\sphinxupquote{dt}}

\item {} 
\sphinxcode{\sphinxupquote{do\_alarm}}

\item {} 
\sphinxcode{\sphinxupquote{dt\_alarm}}

\end{itemize}
\end{sphinxadmonition}
\begin{quote}\begin{description}
\item[{Parameters}] \leavevmode\begin{itemize}
\item {} 
\sphinxstyleliteralstrong{\sphinxupquote{dt}} (\sphinxstyleliteralemphasis{\sphinxupquote{int}}) \textendash{} the step size {[}in minutes{]} in the simulation. \sphinxstylestrong{This is antiquated     and will be removed in future versions.}

\item {} 
\sphinxstyleliteralstrong{\sphinxupquote{num\_people}} (\sphinxstyleliteralemphasis{\sphinxupquote{int}}) \textendash{} the number of people in the household

\item {} 
\sphinxstyleliteralstrong{\sphinxupquote{num\_days}} (\sphinxstyleliteralemphasis{\sphinxupquote{int}}) \textendash{} the number of days in the simulation

\item {} 
\sphinxstyleliteralstrong{\sphinxupquote{num\_hours}} (\sphinxstyleliteralemphasis{\sphinxupquote{int}}) \textendash{} the number of additional hours in the simulation

\item {} 
\sphinxstyleliteralstrong{\sphinxupquote{num\_min}} (\sphinxstyleliteralemphasis{\sphinxupquote{int}}) \textendash{} the number of additional minutes in the simulation

\item {} 
\sphinxstyleliteralstrong{\sphinxupquote{t\_start}} (\sphinxstyleliteralemphasis{\sphinxupquote{int}}) \textendash{} the start time {[}in minutes{]} in the simulation

\item {} 
\sphinxstyleliteralstrong{\sphinxupquote{gender}} (\sphinxstyleliteralemphasis{\sphinxupquote{list}}) \textendash{} the gender of each person in the household

\item {} 
\sphinxstyleliteralstrong{\sphinxupquote{sleep\_start\_mean}} (\sphinxstyleliteralemphasis{\sphinxupquote{list}}) \textendash{} the mean sleep start time {[}in minutes, time of day{]} for each person in     the household

\item {} 
\sphinxstyleliteralstrong{\sphinxupquote{sleep\_start\_std}} (\sphinxstyleliteralemphasis{\sphinxupquote{list}}) \textendash{} the standard deviation of sleep start time {[}in minutes{]} for each person in the     household

\item {} 
\sphinxstyleliteralstrong{\sphinxupquote{sleep\_end\_mean}} (\sphinxstyleliteralemphasis{\sphinxupquote{list}}) \textendash{} the mean sleep end time {[}in minutes, time of day{]} for each person in the household

\item {} 
\sphinxstyleliteralstrong{\sphinxupquote{sleep\_end\_std}} (\sphinxstyleliteralemphasis{\sphinxupquote{list}}) \textendash{} the standard deviation of the sleep end time {[}in minutes{]} for each person in     the household

\item {} 
\sphinxstyleliteralstrong{\sphinxupquote{job\_id}} (\sphinxstyleliteralemphasis{\sphinxupquote{list}}) \textendash{} the occupation identifier for each person in the household

\item {} 
\sphinxstyleliteralstrong{\sphinxupquote{do\_alarm}} (\sphinxstyleliteralemphasis{\sphinxupquote{list}}) \textendash{} a flag indicating whether or not a person uses an alarm for each person in the household

\item {} 
\sphinxstyleliteralstrong{\sphinxupquote{dt\_alarm}} (\sphinxstyleliteralemphasis{\sphinxupquote{list}}) \textendash{} the duration of time {[}in minutes{]} before an alarm goes off before its respective event

\item {} 
\sphinxstyleliteralstrong{\sphinxupquote{bf\_start\_mean}} (\sphinxstyleliteralemphasis{\sphinxupquote{numpy.ndarray}}) \textendash{} the mean breakfast start time for each person in the household     {[}minutes, time of day{]}

\item {} 
\sphinxstyleliteralstrong{\sphinxupquote{bf\_start\_std}} (\sphinxstyleliteralemphasis{\sphinxupquote{numpy.ndarray}}) \textendash{} the standard deviation for breakfast start time for each person in the     household {[}minutes{]}

\item {} 
\sphinxstyleliteralstrong{\sphinxupquote{bf\_start\_trunc}} (\sphinxstyleliteralemphasis{\sphinxupquote{numpy.ndarray}}) \textendash{} the number of standard deviations used in the breakfast start time     distribution for each person

\item {} 
\sphinxstyleliteralstrong{\sphinxupquote{bf\_dt\_mean}} (\sphinxstyleliteralemphasis{\sphinxupquote{numpy.ndarray}}) \textendash{} the mean breakfast duration for each person in the household {[}minutes{]}

\item {} 
\sphinxstyleliteralstrong{\sphinxupquote{bf\_dt\_std}} (\sphinxstyleliteralemphasis{\sphinxupquote{numpy.ndarray}}) \textendash{} the standard deviation for breakfast duration for each person in the     household {[}minutes{]}

\item {} 
\sphinxstyleliteralstrong{\sphinxupquote{bf\_dt\_trunc}} (\sphinxstyleliteralemphasis{\sphinxupquote{numpy.ndarray}}) \textendash{} the number of standard deviations used in the breakfast duration     distribution for each person

\item {} 
\sphinxstyleliteralstrong{\sphinxupquote{lunch\_dt\_mean}} (\sphinxstyleliteralemphasis{\sphinxupquote{numpy.ndarray}}) \textendash{} the mean lunch duration for each person in the household {[}minutes{]}

\item {} 
\sphinxstyleliteralstrong{\sphinxupquote{lunch\_dt\_std}} (\sphinxstyleliteralemphasis{\sphinxupquote{numpy.ndarray}}) \textendash{} the standard deviation for lunch duration for each person in the     household {[}minutes{]}

\item {} 
\sphinxstyleliteralstrong{\sphinxupquote{lunch\_dt\_trunc}} (\sphinxstyleliteralemphasis{\sphinxupquote{numpy.ndarray}}) \textendash{} the number of standard deviations used in the lunch duration     distribution for each person

\item {} 
\sphinxstyleliteralstrong{\sphinxupquote{lunch\_start\_mean}} (\sphinxstyleliteralemphasis{\sphinxupquote{numpy.ndarray}}) \textendash{} the mean lunch start time for each person in the household     {[}minutes, time of day{]}

\item {} 
\sphinxstyleliteralstrong{\sphinxupquote{lunch\_start\_std}} (\sphinxstyleliteralemphasis{\sphinxupquote{numpy.ndarray}}) \textendash{} the standard deviation for lunch start time for each person in the     household {[}minutes{]}

\item {} 
\sphinxstyleliteralstrong{\sphinxupquote{lunch\_start\_trunc}} (\sphinxstyleliteralemphasis{\sphinxupquote{numpy.ndarray}}) \textendash{} the number of standard deviations used in the lunch start time     distribution for each person

\item {} 
\sphinxstyleliteralstrong{\sphinxupquote{dinner\_start\_mean}} (\sphinxstyleliteralemphasis{\sphinxupquote{numpy.ndarray}}) \textendash{} the mean dinner start time for each person in the household     {[}minutes, time of day{]}

\item {} 
\sphinxstyleliteralstrong{\sphinxupquote{dinner\_start\_std}} (\sphinxstyleliteralemphasis{\sphinxupquote{numpy.ndarray}}) \textendash{} the standard deviation for dinner start time for each person in the     household {[}minutes{]}

\item {} 
\sphinxstyleliteralstrong{\sphinxupquote{dinner\_start\_trunc}} (\sphinxstyleliteralemphasis{\sphinxupquote{numpy.ndarray}}) \textendash{} the number of standard deviations used in the dinner start time     distribution for each person

\item {} 
\sphinxstyleliteralstrong{\sphinxupquote{dinner\_dt\_mean}} (\sphinxstyleliteralemphasis{\sphinxupquote{numpy.ndarray}}) \textendash{} the mean dinner duration for each person in the household {[}minutes{]}

\item {} 
\sphinxstyleliteralstrong{\sphinxupquote{dinner\_dt\_std}} (\sphinxstyleliteralemphasis{\sphinxupquote{numpy.ndarray}}) \textendash{} the standard deviation for dinner duration for each person in the     household {[}minutes{]}

\item {} 
\sphinxstyleliteralstrong{\sphinxupquote{dinner\_dt\_trunc}} (\sphinxstyleliteralemphasis{\sphinxupquote{numpy.ndarray}}) \textendash{} the number of standard deviations used in the dinner duration     distribution for each person

\item {} 
\sphinxstyleliteralstrong{\sphinxupquote{work\_start\_mean}} (\sphinxstyleliteralemphasis{\sphinxupquote{numpy.ndarray}}) \textendash{} the mean work start time for each person in the household     {[}minutes, time of day{]}

\item {} 
\sphinxstyleliteralstrong{\sphinxupquote{work\_start\_std}} (\sphinxstyleliteralemphasis{\sphinxupquote{numpy.ndarray}}) \textendash{} the standard deviation of work start time for each person in the     household {[}minutes{]}

\item {} 
\sphinxstyleliteralstrong{\sphinxupquote{work\_end\_mean}} (\sphinxstyleliteralemphasis{\sphinxupquote{numpy.ndarray}}) \textendash{} the work end time for each person in the household {[}minutes, time of day{]}

\item {} 
\sphinxstyleliteralstrong{\sphinxupquote{work\_end\_std}} (\sphinxstyleliteralemphasis{\sphinxupquote{numpy.ndarray}}) \textendash{} the work standard deviation for each person in the household     {[}minutes, time of day{]}

\item {} 
\sphinxstyleliteralstrong{\sphinxupquote{commute\_to\_work\_dt\_mean}} (\sphinxstyleliteralemphasis{\sphinxupquote{numpy.ndarray}}) \textendash{} the mean duration for commuting to work {[}minutes{]} for each     person in the household

\item {} 
\sphinxstyleliteralstrong{\sphinxupquote{commute\_to\_work\_dt\_std}} (\sphinxstyleliteralemphasis{\sphinxupquote{numpy.ndarray}}) \textendash{} the standard deviation for commuting to work {[}minutes{]} for each     person in the household

\item {} 
\sphinxstyleliteralstrong{\sphinxupquote{commute\_from\_work\_dt\_mean}} (\sphinxstyleliteralemphasis{\sphinxupquote{numpy.ndarray}}) \textendash{} the mean duration for commuting from work {[}minutes{]} for each     person in the household

\item {} 
\sphinxstyleliteralstrong{\sphinxupquote{commute\_from\_work\_dt\_std}} (\sphinxstyleliteralemphasis{\sphinxupquote{numpy.ndarray}}) \textendash{} the standard deviation for commuting from work {[}minutes{]} for     each person in the household

\end{itemize}

\item[{Variables}] \leavevmode\begin{itemize}
\item {} 
\sphinxstyleliteralstrong{\sphinxupquote{dt}} (\sphinxstyleliteralemphasis{\sphinxupquote{int}}) \textendash{} the step size {[}in minutes{]} in the simulation \sphinxstylestrong{(this is antiquated and will be removed     in future versions)}

\item {} 
\sphinxstyleliteralstrong{\sphinxupquote{num\_people}} (\sphinxstyleliteralemphasis{\sphinxupquote{int}}) \textendash{} the number of people in the household

\item {} 
\sphinxstyleliteralstrong{\sphinxupquote{num\_days}} (\sphinxstyleliteralemphasis{\sphinxupquote{int}}) \textendash{} the number of days in the simulation

\item {} 
\sphinxstyleliteralstrong{\sphinxupquote{num\_hours}} (\sphinxstyleliteralemphasis{\sphinxupquote{int}}) \textendash{} the number of additional hours in the simulation

\item {} 
\sphinxstyleliteralstrong{\sphinxupquote{num\_min}} (\sphinxstyleliteralemphasis{\sphinxupquote{int}}) \textendash{} the number of additional minutes in the simulation

\item {} 
\sphinxstyleliteralstrong{\sphinxupquote{t\_start}} (\sphinxstyleliteralemphasis{\sphinxupquote{int}}) \textendash{} the start time {[}in minutes{]} in the simulation

\item {} 
\sphinxstyleliteralstrong{\sphinxupquote{gender}} (\sphinxstyleliteralemphasis{\sphinxupquote{list}}) \textendash{} the gender of each person in the household

\item {} 
\sphinxstyleliteralstrong{\sphinxupquote{sleep\_start\_mean}} (\sphinxstyleliteralemphasis{\sphinxupquote{list}}) \textendash{} the mean sleep start time {[}in minutes, time of day{]} for each person in     the household

\item {} 
\sphinxstyleliteralstrong{\sphinxupquote{sleep\_start\_std}} (\sphinxstyleliteralemphasis{\sphinxupquote{list}}) \textendash{} the standard deviation of sleep start time {[}in minutes{]} for each person in the     household

\item {} 
\sphinxstyleliteralstrong{\sphinxupquote{sleep\_end\_mean}} (\sphinxstyleliteralemphasis{\sphinxupquote{list}}) \textendash{} the mean sleep end time {[}in minutes, time of day{]} for each person in the household

\item {} 
\sphinxstyleliteralstrong{\sphinxupquote{sleep\_end\_std}} (\sphinxstyleliteralemphasis{\sphinxupquote{list}}) \textendash{} the standard deviation of the sleep end time {[}in minutes{]} for each person in     the household

\item {} 
\sphinxstyleliteralstrong{\sphinxupquote{job\_id}} (\sphinxstyleliteralemphasis{\sphinxupquote{list}}) \textendash{} the occupation identifier for each person in the household

\item {} 
\sphinxstyleliteralstrong{\sphinxupquote{do\_alarm}} (\sphinxstyleliteralemphasis{\sphinxupquote{list}}) \textendash{} a flag indicating whether or not a person uses an alarm for each person in the household

\item {} 
\sphinxstyleliteralstrong{\sphinxupquote{dt\_alarm}} (\sphinxstyleliteralemphasis{\sphinxupquote{list}}) \textendash{} the duration of time {[}in minutes{]} before an alarm goes off before its respective event

\item {} 
\sphinxstyleliteralstrong{\sphinxupquote{breakfasts}} (\sphinxstyleliteralemphasis{\sphinxupquote{list}}) \textendash{} the breakfast meal objects for each person in the household

\item {} 
\sphinxstyleliteralstrong{\sphinxupquote{lunches}} (\sphinxstyleliteralemphasis{\sphinxupquote{list}}) \textendash{} the lunch meal objects for each person in the household

\item {} 
\sphinxstyleliteralstrong{\sphinxupquote{dinners}} (\sphinxstyleliteralemphasis{\sphinxupquote{list}}) \textendash{} the dinner meal objects for each person in the household

\item {} 
\sphinxstyleliteralstrong{\sphinxupquote{work\_start\_mean}} (\sphinxstyleliteralemphasis{\sphinxupquote{numpy.ndarray}}) \textendash{} the mean work start time for each person in the household     {[}minutes, time of day{]}

\item {} 
\sphinxstyleliteralstrong{\sphinxupquote{work\_start\_std}} (\sphinxstyleliteralemphasis{\sphinxupquote{numpy.ndarray}}) \textendash{} the standard deviation of work start time for each person in the     household {[}minutes{]}

\item {} 
\sphinxstyleliteralstrong{\sphinxupquote{work\_end\_mean}} (\sphinxstyleliteralemphasis{\sphinxupquote{numpy.ndarray}}) \textendash{} the work end time for each person in the household {[}minutes, time of day{]}

\item {} 
\sphinxstyleliteralstrong{\sphinxupquote{work\_end\_std}} (\sphinxstyleliteralemphasis{\sphinxupquote{numpy.ndarray}}) \textendash{} the work standard deviation for each person in the household     {[}minutes, time of day{]}

\item {} 
\sphinxstyleliteralstrong{\sphinxupquote{commute\_to\_work\_dt\_mean}} (\sphinxstyleliteralemphasis{\sphinxupquote{numpy.ndarray}}) \textendash{} the mean duration for commuting to work {[}minutes{]} for each person     in the household

\item {} 
\sphinxstyleliteralstrong{\sphinxupquote{commute\_to\_work\_dt\_std}} (\sphinxstyleliteralemphasis{\sphinxupquote{numpy.ndarray}}) \textendash{} the standard deviation for commuting to work {[}minutes{]} for     each person in the household

\item {} 
\sphinxstyleliteralstrong{\sphinxupquote{commute\_from\_work\_dt\_mean}} (\sphinxstyleliteralemphasis{\sphinxupquote{numpy.ndarray}}) \textendash{} the mean duration for commuting from work {[}minutes{]} for     each person in the household

\item {} 
\sphinxstyleliteralstrong{\sphinxupquote{commute\_from\_work\_dt\_std}} (\sphinxstyleliteralemphasis{\sphinxupquote{numpy.ndarray}}) \textendash{} the standard deviation for commuting from work {[}minutes{]} for each     person in the household

\end{itemize}

\end{description}\end{quote}
\index{get\_info\_mean() (params.Params method)}

\begin{fulllineitems}
\phantomsection\label{\detokenize{params:params.Params.get_info_mean}}\pysiglinewithargsret{\sphinxbfcode{\sphinxupquote{get\_info\_mean}}}{}{}
This function stores information about the mean values of the activity-parameters.
\begin{quote}\begin{description}
\item[{Returns}] \leavevmode
a message about the the mean values

\item[{Return type}] \leavevmode
str

\item[{Returns}] \leavevmode
a list of activity codes in the chronological order in time of day

\item[{Return type}] \leavevmode
list

\end{description}\end{quote}

\end{fulllineitems}

\index{get\_info\_std() (params.Params method)}

\begin{fulllineitems}
\phantomsection\label{\detokenize{params:params.Params.get_info_std}}\pysiglinewithargsret{\sphinxbfcode{\sphinxupquote{get\_info\_std}}}{\emph{keys\_ordered=None}}{}
This function stores information about the standard deviation values of the activity-parameters.
\begin{quote}\begin{description}
\item[{Parameters}] \leavevmode
\sphinxstyleliteralstrong{\sphinxupquote{keys\_ordered}} (\sphinxstyleliteralemphasis{\sphinxupquote{list}}) \textendash{} this is a list of the names of the activities that are in the         same order as the information about the means

\item[{Returns}] \leavevmode
a message about the the standard deviation values

\item[{Return type}] \leavevmode
str

\end{description}\end{quote}

\end{fulllineitems}

\index{init\_help() (params.Params method)}

\begin{fulllineitems}
\phantomsection\label{\detokenize{params:params.Params.init_help}}\pysiglinewithargsret{\sphinxbfcode{\sphinxupquote{init\_help}}}{\emph{val}, \emph{default\_val}}{}
This function assigns a default value to an attribute in case it was not assigned already. This is,         function is particularly useful if the value to be assigned is an array depending on \sphinxcode{\sphinxupquote{num\_people}}.         More specifically,
\begin{itemize}
\item {} 
if val is not None, return val

\item {} 
if val is None, return the default value (default\_val)

\end{itemize}
\begin{quote}\begin{description}
\item[{Parameters}] \leavevmode\begin{itemize}
\item {} 
\sphinxstyleliteralstrong{\sphinxupquote{val}} \textendash{} the value to be assigned

\item {} 
\sphinxstyleliteralstrong{\sphinxupquote{default\_val}} \textendash{} the default value to assign in case val is None

\end{itemize}

\item[{Returns}] \leavevmode
the non-None value

\end{description}\end{quote}

\end{fulllineitems}

\index{init\_meal() (params.Params method)}

\begin{fulllineitems}
\phantomsection\label{\detokenize{params:params.Params.init_meal}}\pysiglinewithargsret{\sphinxbfcode{\sphinxupquote{init\_meal}}}{\emph{m\_id}, \emph{start\_mean=None}, \emph{start\_std=None}, \emph{start\_trunc=None}, \emph{dt\_mean=None}, \emph{dt\_std=None}, \emph{dt\_trunc=None}}{}
This function returns the data for each person in the household for the respective meal         given by “m\_id”:
\begin{itemize}
\item {} 
if specific parameters have been assigned, create meals with the respective parameters

\item {} 
if specific parameters have not been assigned, create meals with the default meal parameters for         each meal

\end{itemize}
\begin{quote}\begin{description}
\item[{Parameters}] \leavevmode\begin{itemize}
\item {} 
\sphinxstyleliteralstrong{\sphinxupquote{m\_id}} (\sphinxstyleliteralemphasis{\sphinxupquote{int}}) \textendash{} the identifier of meal type

\item {} 
\sphinxstyleliteralstrong{\sphinxupquote{start\_mean}} (\sphinxstyleliteralemphasis{\sphinxupquote{numpy.ndarray}}) \textendash{} the mean start time for the meal for each person in the household

\item {} 
\sphinxstyleliteralstrong{\sphinxupquote{start\_std}} (\sphinxstyleliteralemphasis{\sphinxupquote{numpy.ndarray}}) \textendash{} the standard deviation of start time for the meal for each person in         the household

\item {} 
\sphinxstyleliteralstrong{\sphinxupquote{start\_trunc}} (\sphinxstyleliteralemphasis{\sphinxupquote{numpy.ndarray}}) \textendash{} the amount of standard deviations allowed before truncating the         start time distribution for each person in the household

\item {} 
\sphinxstyleliteralstrong{\sphinxupquote{dt\_mean}} (\sphinxstyleliteralemphasis{\sphinxupquote{numpy.ndarray}}) \textendash{} the mean duration for the meal for each person in the household

\item {} 
\sphinxstyleliteralstrong{\sphinxupquote{dt\_std}} (\sphinxstyleliteralemphasis{\sphinxupquote{numpy.ndarray}}) \textendash{} the standard deviation for the meal for each person in the household

\item {} 
\sphinxstyleliteralstrong{\sphinxupquote{dt\_trunc}} (\sphinxstyleliteralemphasis{\sphinxupquote{numpy.ndarray}}) \textendash{} the amount of standard deviations allowed before truncating the         duration distribution for each person in the household

\end{itemize}

\item[{Returns}] \leavevmode
the meals for each person in the household

\item[{Return type}] \leavevmode
list

\end{description}\end{quote}

\end{fulllineitems}

\index{init\_meal\_old() (params.Params method)}

\begin{fulllineitems}
\phantomsection\label{\detokenize{params:params.Params.init_meal_old}}\pysiglinewithargsret{\sphinxbfcode{\sphinxupquote{init\_meal\_old}}}{\emph{id}, \emph{start\_mean=None}, \emph{start\_std=None}, \emph{dt\_mean=None}, \emph{dt\_std=None}}{}
This function returns the data for each person in the household for the respective meal given by “id”.

\begin{sphinxadmonition}{warning}{Warning:}
This function may be \sphinxstylestrong{not} used because it is antiquated.
\end{sphinxadmonition}
\begin{quote}\begin{description}
\item[{Parameters}] \leavevmode\begin{itemize}
\item {} 
\sphinxstyleliteralstrong{\sphinxupquote{id}} (\sphinxstyleliteralemphasis{\sphinxupquote{int}}) \textendash{} the id of meal type

\item {} 
\sphinxstyleliteralstrong{\sphinxupquote{start\_mean}} (\sphinxstyleliteralemphasis{\sphinxupquote{numpy.ndarray}}) \textendash{} the mean start time for the meal for each person in the household

\item {} 
\sphinxstyleliteralstrong{\sphinxupquote{dt\_mean}} (\sphinxstyleliteralemphasis{\sphinxupquote{numpy.ndarray}}) \textendash{} the mean duration for the meal for each person in the household

\item {} 
\sphinxstyleliteralstrong{\sphinxupquote{dt\_std}} (\sphinxstyleliteralemphasis{\sphinxupquote{numpy.ndarray}}) \textendash{} the mean standard deviation for the meal for each person in the household

\end{itemize}

\item[{Returns}] \leavevmode
the meals for each person in the household

\item[{Return type}] \leavevmode
list

\end{description}\end{quote}

\end{fulllineitems}

\index{set\_no\_variation() (params.Params method)}

\begin{fulllineitems}
\phantomsection\label{\detokenize{params:params.Params.set_no_variation}}\pysiglinewithargsret{\sphinxbfcode{\sphinxupquote{set\_no\_variation}}}{}{}
This function sets all of the standard deviations of the activity-parameters to zero for all
agents being simulated.
\begin{quote}\begin{description}
\item[{Returns}] \leavevmode


\end{description}\end{quote}

\end{fulllineitems}

\index{set\_num\_steps() (params.Params method)}

\begin{fulllineitems}
\phantomsection\label{\detokenize{params:params.Params.set_num_steps}}\pysiglinewithargsret{\sphinxbfcode{\sphinxupquote{set\_num\_steps}}}{}{}
This function calculates and sets the number of time steps ABMHAP will run.

\begin{sphinxadmonition}{note}{Note:}
This function may be antiquated.
\end{sphinxadmonition}
\begin{quote}\begin{description}
\item[{Return type}] \leavevmode
None

\end{description}\end{quote}

\end{fulllineitems}

\index{tester() (params.Params method)}

\begin{fulllineitems}
\phantomsection\label{\detokenize{params:params.Params.tester}}\pysiglinewithargsret{\sphinxbfcode{\sphinxupquote{tester}}}{}{}~
\begin{sphinxadmonition}{warning}{Warning:}
This function is just for testing. It checks to see whether the expected dinner time is before             the expected end time for work.
\end{sphinxadmonition}
\begin{quote}\begin{description}
\item[{Returns}] \leavevmode


\end{description}\end{quote}

\end{fulllineitems}

\index{toString() (params.Params method)}

\begin{fulllineitems}
\phantomsection\label{\detokenize{params:params.Params.toString}}\pysiglinewithargsret{\sphinxbfcode{\sphinxupquote{toString}}}{}{}
This function represents the Params object as a string. For now, it prints         the tuple (start time, duration, end time) in hours{[}0, 24{]} for the following activities:
\begin{enumerate}
\item {} 
eat breakfast

\item {} 
commute to work

\item {} 
work

\item {} 
eat lunch

\item {} 
commute from work

\item {} 
eat dinner

\item {} 
sleep

\end{enumerate}

in order of start time. The commute activities only have duration information.
\begin{quote}\begin{description}
\item[{Returns}] \leavevmode
the parameter information

\end{description}\end{quote}

\end{fulllineitems}


\end{fulllineitems}



\subsection{person module}
\label{\detokenize{person::doc}}\label{\detokenize{person:module-person}}\label{\detokenize{person:person-module}}\index{person (module)}
This module has code that governs information about the agent.

This module contains information about class {\hyperref[\detokenize{person:person.Person}]{\sphinxcrossref{\sphinxcode{\sphinxupquote{person.Person}}}}}.
\index{Person (class in person)}

\begin{fulllineitems}
\phantomsection\label{\detokenize{person:person.Person}}\pysiglinewithargsret{\sphinxbfcode{\sphinxupquote{class }}\sphinxcode{\sphinxupquote{person.}}\sphinxbfcode{\sphinxupquote{Person}}}{\emph{house}, \emph{clock}, \emph{schedule}}{}
Bases: \sphinxcode{\sphinxupquote{object}}

This class contains all of the information relevant for a Person.

A person is parametrized by the following
\begin{itemize}
\item {} 
a place of residence

\item {} 
a biology

\item {} 
social behavior

\item {} 
a location

\item {} 
a history of activities and states

\item {} 
Needs
\begin{enumerate}
\item {} 
Hunger

\item {} 
Rest

\item {} 
Income

\item {} 
Travel

\item {} 
Interruption

\end{enumerate}

\end{itemize}
\begin{quote}\begin{description}
\item[{Parameters}] \leavevmode\begin{itemize}
\item {} 
\sphinxstyleliteralstrong{\sphinxupquote{house}} ({\hyperref[\detokenize{home:home.Home}]{\sphinxcrossref{\sphinxstyleliteralemphasis{\sphinxupquote{home.Home}}}}}) \textendash{} the Home object the person resides in. (will need to remove this)

\item {} 
\sphinxstyleliteralstrong{\sphinxupquote{clock}} ({\hyperref[\detokenize{temporal:temporal.Temporal}]{\sphinxcrossref{\sphinxstyleliteralemphasis{\sphinxupquote{temporal.Temporal}}}}}) \textendash{} the time

\item {} 
\sphinxstyleliteralstrong{\sphinxupquote{schedule}} ({\hyperref[\detokenize{scheduler:scheduler.Scheduler}]{\sphinxcrossref{\sphinxstyleliteralemphasis{\sphinxupquote{scheduler.Scheduler}}}}}) \textendash{} the schedule

\end{itemize}

\item[{Variables}] \leavevmode\begin{itemize}
\item {} 
\sphinxstyleliteralstrong{\sphinxupquote{'bio'}} ({\hyperref[\detokenize{bio:bio.Bio}]{\sphinxcrossref{\sphinxstyleliteralemphasis{\sphinxupquote{bio.Bio}}}}}) \textendash{} the biological characteristics

\item {} 
\sphinxstyleliteralstrong{\sphinxupquote{clock}} ({\hyperref[\detokenize{temporal:temporal.Temporal}]{\sphinxcrossref{\sphinxstyleliteralemphasis{\sphinxupquote{temporal.Temporal}}}}}) \textendash{} keeps track of the current time. It is linked to the Universe clock

\item {} 
\sphinxstyleliteralstrong{\sphinxupquote{hist\_state}} (\sphinxstyleliteralemphasis{\sphinxupquote{numpy.ndarray}}) \textendash{} the state history {[}int{]} for each time step

\item {} 
\sphinxstyleliteralstrong{\sphinxupquote{hist\_activity}} (\sphinxstyleliteralemphasis{\sphinxupquote{numpy.ndarray}}) \textendash{} the activity history {[}int{]} for each time step

\item {} 
\sphinxstyleliteralstrong{\sphinxupquote{'home'}} ({\hyperref[\detokenize{home:home.Home}]{\sphinxcrossref{\sphinxstyleliteralemphasis{\sphinxupquote{home.Home}}}}}) \textendash{} this contains the place where the person resides

\item {} 
\sphinxstyleliteralstrong{\sphinxupquote{id}} (\sphinxstyleliteralemphasis{\sphinxupquote{int}}) \textendash{} unique person identifier

\item {} 
\sphinxstyleliteralstrong{\sphinxupquote{'income'}} ({\hyperref[\detokenize{income:income.Income}]{\sphinxcrossref{\sphinxstyleliteralemphasis{\sphinxupquote{income.Income}}}}}) \textendash{} the need that concerns itself with working/school

\item {} 
\sphinxstyleliteralstrong{\sphinxupquote{'interruption'}} ({\hyperref[\detokenize{interruption:interruption.Interruption}]{\sphinxcrossref{\sphinxstyleliteralemphasis{\sphinxupquote{interruption.Interruption}}}}}) \textendash{} the need that concerns itself with interrupting an ongoing activity

\item {} 
\sphinxstyleliteralstrong{\sphinxupquote{'location'}} ({\hyperref[\detokenize{location:location.Location}]{\sphinxcrossref{\sphinxstyleliteralemphasis{\sphinxupquote{location.Location}}}}}) \textendash{} the location data of a person

\item {} 
\sphinxstyleliteralstrong{\sphinxupquote{needs}} (\sphinxstyleliteralemphasis{\sphinxupquote{dict}}) \textendash{} a dictionary of all of the  needs

\item {} 
\sphinxstyleliteralstrong{\sphinxupquote{'rest'}} ({\hyperref[\detokenize{rest:rest.Rest}]{\sphinxcrossref{\sphinxstyleliteralemphasis{\sphinxupquote{rest.Rest}}}}}) \textendash{} the need that concerns itself with sleeping

\item {} 
\sphinxstyleliteralstrong{\sphinxupquote{socio}} ({\hyperref[\detokenize{social:social.Social}]{\sphinxcrossref{\sphinxstyleliteralemphasis{\sphinxupquote{social.Social}}}}}) \textendash{} the social characteristics of a Person

\item {} 
\sphinxstyleliteralstrong{\sphinxupquote{'state'}} ({\hyperref[\detokenize{state:state.State}]{\sphinxcrossref{\sphinxstyleliteralemphasis{\sphinxupquote{state.State}}}}}) \textendash{} information about a Person’s state

\item {} 
\sphinxstyleliteralstrong{\sphinxupquote{'travel'}} ({\hyperref[\detokenize{travel:travel.Travel}]{\sphinxcrossref{\sphinxstyleliteralemphasis{\sphinxupquote{travel.Travel}}}}}) \textendash{} the need that concerns itself with moving from one area to another

\item {} 
\sphinxstyleliteralstrong{\sphinxupquote{hist\_state}} \textendash{} the state of the person at each time step

\item {} 
\sphinxstyleliteralstrong{\sphinxupquote{hist\_activity}} \textendash{} the activity code of the person at each time step

\item {} 
\sphinxstyleliteralstrong{\sphinxupquote{hist\_local}} (\sphinxstyleliteralemphasis{\sphinxupquote{numpy.ndarray}}) \textendash{} the location code of the person at each time step

\item {} 
\sphinxstyleliteralstrong{\sphinxupquote{H}} (\sphinxstyleliteralemphasis{\sphinxupquote{numpy.ndarray}}) \textendash{} the satiation level for each need at each time step

\item {} 
\sphinxstyleliteralstrong{\sphinxupquote{need\_vector}} (\sphinxstyleliteralemphasis{\sphinxupquote{numpy.ndarray}}) \textendash{} the satiation level for each need at a given time step

\end{itemize}

\end{description}\end{quote}
\index{get\_diary() (person.Person method)}

\begin{fulllineitems}
\phantomsection\label{\detokenize{person:person.Person.get_diary}}\pysiglinewithargsret{\sphinxbfcode{\sphinxupquote{get\_diary}}}{}{}
This function output the result of the simulation in terms of an activity diary.
\begin{quote}\begin{description}
\item[{Returns}] \leavevmode
the activity diary describing the behavior of the agent

\item[{Return type}] \leavevmode
{\hyperref[\detokenize{diary:diary.Diary}]{\sphinxcrossref{diary.Diary}}}

\end{description}\end{quote}

\end{fulllineitems}

\index{print\_basic\_info() (person.Person method)}

\begin{fulllineitems}
\phantomsection\label{\detokenize{person:person.Person.print_basic_info}}\pysiglinewithargsret{\sphinxbfcode{\sphinxupquote{print\_basic\_info}}}{}{}
This function expresses basic information about the Person object as a string by         printing the following:
\begin{itemize}
\item {} 
person identifier

\item {} 
home identifier

\item {} 
age

\item {} 
gender

\end{itemize}
\begin{quote}\begin{description}
\item[{Returns}] \leavevmode
basic information about the Person

\item[{Return type}] \leavevmode
str

\end{description}\end{quote}

\end{fulllineitems}

\index{reset() (person.Person method)}

\begin{fulllineitems}
\phantomsection\label{\detokenize{person:person.Person.reset}}\pysiglinewithargsret{\sphinxbfcode{\sphinxupquote{reset}}}{}{}
This function rests the person at the beginning of a simulation by doing the following:
\begin{enumerate}
\item {} 
reset the history

\item {} 
reset the state

\item {} 
reset the location

\item {} 
reset the needs

\end{enumerate}

\begin{sphinxadmonition}{note}{Note:}
the clock needs to be set to the beginning of simulation
\end{sphinxadmonition}
\begin{quote}\begin{description}
\item[{Returns}] \leavevmode
None

\end{description}\end{quote}

\end{fulllineitems}

\index{reset\_history() (person.Person method)}

\begin{fulllineitems}
\phantomsection\label{\detokenize{person:person.Person.reset_history}}\pysiglinewithargsret{\sphinxbfcode{\sphinxupquote{reset\_history}}}{}{}
This function resets the variables:
\begin{enumerate}
\item {} 
history of the state

\item {} 
history of the activity

\item {} 
history of the location

\end{enumerate}
\begin{quote}\begin{description}
\item[{Returns}] \leavevmode
None

\end{description}\end{quote}

\end{fulllineitems}

\index{reset\_needs() (person.Person method)}

\begin{fulllineitems}
\phantomsection\label{\detokenize{person:person.Person.reset_needs}}\pysiglinewithargsret{\sphinxbfcode{\sphinxupquote{reset\_needs}}}{}{}
This function resets the needs.
\begin{quote}\begin{description}
\item[{Returns}] \leavevmode
None

\end{description}\end{quote}

\end{fulllineitems}

\index{toString() (person.Person method)}

\begin{fulllineitems}
\phantomsection\label{\detokenize{person:person.Person.toString}}\pysiglinewithargsret{\sphinxbfcode{\sphinxupquote{toString}}}{}{}
This function represents the Person object as a string.
\begin{quote}\begin{description}
\item[{Returns}] \leavevmode
information about the Person

\item[{Return type}] \leavevmode
str

\end{description}\end{quote}

\end{fulllineitems}

\index{update\_history() (person.Person method)}

\begin{fulllineitems}
\phantomsection\label{\detokenize{person:person.Person.update_history}}\pysiglinewithargsret{\sphinxbfcode{\sphinxupquote{update\_history}}}{}{}
This function updates the history of the following values with their current values:
\begin{itemize}
\item {} 
state history

\item {} 
location history

\item {} 
activity history

\item {} 
need (satiation) history

\end{itemize}
\begin{quote}\begin{description}
\item[{Returns}] \leavevmode


\end{description}\end{quote}

\end{fulllineitems}

\index{update\_history\_activity() (person.Person method)}

\begin{fulllineitems}
\phantomsection\label{\detokenize{person:person.Person.update_history_activity}}\pysiglinewithargsret{\sphinxbfcode{\sphinxupquote{update\_history\_activity}}}{}{}
This function updates the activity history with the current values.
\begin{quote}\begin{description}
\item[{Returns}] \leavevmode
None

\end{description}\end{quote}

\end{fulllineitems}

\index{update\_history\_needs() (person.Person method)}

\begin{fulllineitems}
\phantomsection\label{\detokenize{person:person.Person.update_history_needs}}\pysiglinewithargsret{\sphinxbfcode{\sphinxupquote{update\_history\_needs}}}{}{}
This function updates the needs (satiation) history with the current values.
\begin{quote}\begin{description}
\item[{Returns}] \leavevmode
None

\end{description}\end{quote}

\end{fulllineitems}


\end{fulllineitems}



\subsection{rest module}
\label{\detokenize{rest::doc}}\label{\detokenize{rest:rest-module}}\label{\detokenize{rest:module-rest}}\index{rest (module)}
This file contains information about the need dealing with Rest.

This module contains class {\hyperref[\detokenize{rest:rest.Rest}]{\sphinxcrossref{\sphinxcode{\sphinxupquote{rest.Rest}}}}}.
\index{Rest (class in rest)}

\begin{fulllineitems}
\phantomsection\label{\detokenize{rest:rest.Rest}}\pysiglinewithargsret{\sphinxbfcode{\sphinxupquote{class }}\sphinxcode{\sphinxupquote{rest.}}\sphinxbfcode{\sphinxupquote{Rest}}}{\emph{clock}, \emph{num\_sample\_points}}{}
Bases: {\hyperref[\detokenize{need:need.Need}]{\sphinxcrossref{\sphinxcode{\sphinxupquote{need.Need}}}}}

This class contains relevant information about the rest need.
\begin{quote}\begin{description}
\item[{Parameters}] \leavevmode\begin{itemize}
\item {} 
\sphinxstyleliteralstrong{\sphinxupquote{clock}} ({\hyperref[\detokenize{temporal:temporal.Temporal}]{\sphinxcrossref{\sphinxstyleliteralemphasis{\sphinxupquote{temporal.Temporal}}}}}) \textendash{} this keeps track of the current time. It is linked to the Universe clock.

\item {} 
\sphinxstyleliteralstrong{\sphinxupquote{num\_sample\_points}} (\sphinxstyleliteralemphasis{\sphinxupquote{int}}) \textendash{} the number of temporal nodes in the simulation

\end{itemize}

\end{description}\end{quote}
\index{decay() (rest.Rest method)}

\begin{fulllineitems}
\phantomsection\label{\detokenize{rest:rest.Rest.decay}}\pysiglinewithargsret{\sphinxbfcode{\sphinxupquote{decay}}}{\emph{status}}{}~
\begin{sphinxadmonition}{warning}{Warning:}
This function is old and antiquated.
\end{sphinxadmonition}

This function decays the Rest satiation. The satiation only decays if the person is         \sphinxstylestrong{not} asleep. The decay in sleep
\begin{equation*}
\begin{split}\delta &= m_{decay} \Delta{t} \\
n(t + \Delta{t}) &= n(t) + \delta\end{split}
\end{equation*}\begin{description}
\item[{where}] \leavevmode\begin{itemize}
\item {} 
\(m_{decay}\) is the decay rate

\item {} 
\(\Delta{t}\) is the duration of time in 1 time step of simulation {[}minutes{]}

\item {} 
\(\delta\) is the amount of decay of rest

\item {} 
\(n(t)\) is the satiation at time t

\end{itemize}

\end{description}
\begin{quote}\begin{description}
\item[{Parameters}] \leavevmode
\sphinxstyleliteralstrong{\sphinxupquote{status}} (\sphinxstyleliteralemphasis{\sphinxupquote{int}}) \textendash{} the current state of a person

\item[{Returns}] \leavevmode
None

\end{description}\end{quote}

\end{fulllineitems}

\index{decay\_new() (rest.Rest method)}

\begin{fulllineitems}
\phantomsection\label{\detokenize{rest:rest.Rest.decay_new}}\pysiglinewithargsret{\sphinxbfcode{\sphinxupquote{decay\_new}}}{\emph{status}, \emph{dt}}{}
This function decays Rests’ satiation. The satiation only decays if the         person is \sphinxstylestrong{not} asleep. The decay in sleep is calculated by
\begin{equation*}
\begin{split}\delta &= m_{decay} \Delta{t} \\
n(t + \Delta{t}) &= n(t) + \delta\end{split}
\end{equation*}\begin{description}
\item[{where}] \leavevmode\begin{itemize}
\item {} 
\(t\) the current time

\item {} 
\(\Delta{t}\) is the duration of time to decay the satiation {[}minutes{]}

\item {} 
\(\delta\) the change in the satiation for Rest

\item {} 
\(m_{decay}\) is the decay rate for Rest

\item {} 
\(n(t)\) is the satiation of Rest at time t

\end{itemize}

\end{description}
\begin{quote}\begin{description}
\item[{Parameters}] \leavevmode\begin{itemize}
\item {} 
\sphinxstyleliteralstrong{\sphinxupquote{status}} (\sphinxstyleliteralemphasis{\sphinxupquote{int}}) \textendash{} the current state of a person

\item {} 
\sphinxstyleliteralstrong{\sphinxupquote{dt}} (\sphinxstyleliteralemphasis{\sphinxupquote{int}}) \textendash{} the duration of time \(\Delta{t}\) {[}minutes{]} used to decay the need

\end{itemize}

\item[{Returns}] \leavevmode
None

\end{description}\end{quote}

\end{fulllineitems}

\index{initialize() (rest.Rest method)}

\begin{fulllineitems}
\phantomsection\label{\detokenize{rest:rest.Rest.initialize}}\pysiglinewithargsret{\sphinxbfcode{\sphinxupquote{initialize}}}{\emph{p}}{}
The purpose of this code is to help initialize Rest’s satiation and         whatever activity that goes with it, depending on any time the simulation begins.

\begin{sphinxadmonition}{note}{Note:}
This code is a work in progress.
\end{sphinxadmonition}
\begin{enumerate}
\item {} 
update the sleep start and end time

\item {} 
find out if the person should be asleep

\item {} 
if the Person is asleep,
\begin{itemize}
\item {} 
sets the appropriate duration of sleep left to do

\item {} 
sets the rest magnitude to threshold

\item {} 
sets the rest recharge rate

\item {} 
sets the schedule to trigger when when the person is scheduled to wake up

\end{itemize}

\item {} 
if the Person is not asleep,
\begin{itemize}
\item {} 
sets the decay rate

\item {} 
set the magnitude

\item {} 
sets the schedule to trigger when when the person is scheduled to start sleeping

\end{itemize}

\item {} 
update the schedule for the rest need

\end{enumerate}
\begin{quote}\begin{description}
\item[{Parameters}] \leavevmode
\sphinxstyleliteralstrong{\sphinxupquote{p}} ({\hyperref[\detokenize{person:person.Person}]{\sphinxcrossref{\sphinxstyleliteralemphasis{\sphinxupquote{person.Person}}}}}) \textendash{} the person of interest

\item[{Returns}] \leavevmode
None

\end{description}\end{quote}

\end{fulllineitems}

\index{is\_workday() (rest.Rest method)}

\begin{fulllineitems}
\phantomsection\label{\detokenize{rest:rest.Rest.is_workday}}\pysiglinewithargsret{\sphinxbfcode{\sphinxupquote{is\_workday}}}{\emph{p}}{}
This function indicates whether or not the sleep event resembles that from a person sleeping for         a workday.
\begin{quote}\begin{description}
\item[{Parameters}] \leavevmode
\sphinxstyleliteralstrong{\sphinxupquote{socio}} ({\hyperref[\detokenize{social:social.Social}]{\sphinxcrossref{\sphinxstyleliteralemphasis{\sphinxupquote{social.Social}}}}}) \textendash{} the social characteristics of the person of interest

\item[{Returns}] \leavevmode
True, if the sleep event resembles a workday. False, otherwise.

\end{description}\end{quote}

\end{fulllineitems}

\index{perceive() (rest.Rest method)}

\begin{fulllineitems}
\phantomsection\label{\detokenize{rest:rest.Rest.perceive}}\pysiglinewithargsret{\sphinxbfcode{\sphinxupquote{perceive}}}{\emph{future\_clock}}{}
This functions gives the updated rest magnitude if sleep is done from now until a later time         corresponding to clock.
\begin{equation*}
\begin{split}\delta = m_{suggested}\Delta{t}\end{split}
\end{equation*}\begin{description}
\item[{where}] \leavevmode\begin{itemize}
\item {} 
\(\delta\) is the amount of change in the satiation for Rest

\item {} 
\(m_{suggested}\) is the suggested recharge rate for Rest

\item {} \begin{description}
\item[{\(\Delta{t}\) is the duration of time from now until the future time}] \leavevmode
given by future\_clock

\end{description}

\end{itemize}

\end{description}
\begin{quote}\begin{description}
\item[{Parameters}] \leavevmode
\sphinxstyleliteralstrong{\sphinxupquote{future\_clock}} ({\hyperref[\detokenize{temporal:temporal.Temporal}]{\sphinxcrossref{\sphinxstyleliteralemphasis{\sphinxupquote{temporal.Temporal}}}}}) \textendash{} a clock corresponding to a future time

\item[{Returns}] \leavevmode
the perceived rest level

\item[{Return type}] \leavevmode
float

\end{description}\end{quote}

\end{fulllineitems}

\index{reset() (rest.Rest method)}

\begin{fulllineitems}
\phantomsection\label{\detokenize{rest:rest.Rest.reset}}\pysiglinewithargsret{\sphinxbfcode{\sphinxupquote{reset}}}{}{}
This function resets the values in order for the need to be used in the next simulation
\begin{quote}\begin{description}
\item[{Returns}] \leavevmode
None

\end{description}\end{quote}

\end{fulllineitems}

\index{set\_decay\_rate() (rest.Rest method)}

\begin{fulllineitems}
\phantomsection\label{\detokenize{rest:rest.Rest.set_decay_rate}}\pysiglinewithargsret{\sphinxbfcode{\sphinxupquote{set\_decay\_rate}}}{\emph{dt}}{}
This function sets the decay rate. The decay rate (\(m_{decay}\)) is assumed         to be the slope of a linear function.
\begin{equation*}
\begin{split}m_{decay} = -\frac{1 - \lambda}{\Delta{t}}\end{split}
\end{equation*}\begin{description}
\item[{where}] \leavevmode\begin{itemize}
\item {} 
\(\Delta{t}\) is the duration of time expected to be awake

\item {} 
\(\lambda\) is the Rest threshold

\item {} 
\(m_{decay}\) is the decay rate for Rest

\end{itemize}

\end{description}
\begin{quote}\begin{description}
\item[{Parameters}] \leavevmode
\sphinxstyleliteralstrong{\sphinxupquote{dt}} (\sphinxstyleliteralemphasis{\sphinxupquote{int}}) \textendash{} the duration of sleep \(\Delta{t}\) {[}minutes{]}

\item[{Returns}] \leavevmode
None

\end{description}\end{quote}

\end{fulllineitems}

\index{set\_recharge\_rate() (rest.Rest method)}

\begin{fulllineitems}
\phantomsection\label{\detokenize{rest:rest.Rest.set_recharge_rate}}\pysiglinewithargsret{\sphinxbfcode{\sphinxupquote{set\_recharge\_rate}}}{\emph{dt}}{}
This function sets the recharge rate. The recharge rate (\(m_{recharge}\))         is assumed to be the slope of a linear function.
\begin{equation*}
\begin{split}m_{recharge} = \frac{1 - \lambda}{ \Delta{t} }\end{split}
\end{equation*}\begin{description}
\item[{where}] \leavevmode\begin{itemize}
\item {} 
\(\Delta{t}\) is the duration of sleep

\item {} 
\(\lambda\) is the threshold for Rest

\item {} 
\(m_{recharge}\) is the recharge rate for Rest

\end{itemize}

\end{description}
\begin{quote}\begin{description}
\item[{Parameters}] \leavevmode
\sphinxstyleliteralstrong{\sphinxupquote{dt}} (\sphinxstyleliteralemphasis{\sphinxupquote{int}}) \textendash{} the duration of sleep after rounding \(\Delta{t}\) {[}minutes{]}

\item[{Returns}] \leavevmode
None

\end{description}\end{quote}

\end{fulllineitems}

\index{set\_suggested\_recharge\_rate() (rest.Rest method)}

\begin{fulllineitems}
\phantomsection\label{\detokenize{rest:rest.Rest.set_suggested_recharge_rate}}\pysiglinewithargsret{\sphinxbfcode{\sphinxupquote{set\_suggested\_recharge\_rate}}}{\emph{dt}}{}
This function sets the “suggested” recharge rate. That is, the rate of recharge assuming exact         arithmetic (there is no rounding in time, say to the nearest minute).
\begin{equation*}
\begin{split}m_{suggested} = \frac{ 1 - \lambda }{ \Delta{t} }\end{split}
\end{equation*}\begin{description}
\item[{where}] \leavevmode\begin{itemize}
\item {} 
\(\Delta{t}\) is the duration of sleep

\item {} 
\(\lambda\) is the Rest threshold

\item {} 
\(m_{suggested}\) is the suggested recharge rate for Rest

\end{itemize}

\end{description}
\begin{quote}\begin{description}
\item[{Parameters}] \leavevmode
\sphinxstyleliteralstrong{\sphinxupquote{dt}} (\sphinxstyleliteralemphasis{\sphinxupquote{int}}) \textendash{} the duration of sleep \(\Delta{t}\) {[}minutes{]}

\item[{Returns}] \leavevmode
None

\end{description}\end{quote}

\end{fulllineitems}

\index{should\_be\_asleep() (rest.Rest method)}

\begin{fulllineitems}
\phantomsection\label{\detokenize{rest:rest.Rest.should_be_asleep}}\pysiglinewithargsret{\sphinxbfcode{\sphinxupquote{should\_be\_asleep}}}{\emph{t\_start}, \emph{t\_end}}{}
This function finds out if the person should be asleep for the initialization         of the ABMHAP algorithm.
\begin{quote}\begin{description}
\item[{Parameters}] \leavevmode\begin{itemize}
\item {} 
\sphinxstyleliteralstrong{\sphinxupquote{t\_start}} (\sphinxstyleliteralemphasis{\sphinxupquote{int}}) \textendash{} start time of sleep {[}minutes, time of day{]}

\item {} 
\sphinxstyleliteralstrong{\sphinxupquote{t\_end}} (\sphinxstyleliteralemphasis{\sphinxupquote{int}}) \textendash{} end time of sleep {[}minutes, time of day{]}

\end{itemize}

\item[{Returns}] \leavevmode
a flag indicating whether a person should be asleep (if True) or awake (if False)

\item[{Return type}] \leavevmode
bool

\end{description}\end{quote}

\end{fulllineitems}

\index{toString() (rest.Rest method)}

\begin{fulllineitems}
\phantomsection\label{\detokenize{rest:rest.Rest.toString}}\pysiglinewithargsret{\sphinxbfcode{\sphinxupquote{toString}}}{}{}
Represent the Rest object as a string
\begin{quote}\begin{description}
\item[{Returns}] \leavevmode
the representation of the Rest object

\item[{Return type}] \leavevmode
str

\end{description}\end{quote}

\end{fulllineitems}


\end{fulllineitems}



\subsection{scheduler module}
\label{\detokenize{scheduler::doc}}\label{\detokenize{scheduler:module-scheduler}}\label{\detokenize{scheduler:scheduler-module}}\index{scheduler (module)}
This module contains code that is is responsible for controlling the scheduler for the simulation. Note that the simulation does \sphinxstylestrong{not} run continuously in from one adjacent time step to the next. Instead the simulation jumps forward in time (i.e. move across multiple time steps in time), stopping only at time steps in which an action could occur. The ability to jump forward in time is controlled by the scheduler.

The scheduler will trigger the simulation to stop skipping time steps for the following reasons:
\begin{enumerate}
\item {} 
an activity should start

\item {} 
an activity should end

\item {} 
a need is under threshold

\end{enumerate}

This module contains class {\hyperref[\detokenize{scheduler:scheduler.Scheduler}]{\sphinxcrossref{\sphinxcode{\sphinxupquote{scheduler.Scheduler}}}}}.
\index{Scheduler (class in scheduler)}

\begin{fulllineitems}
\phantomsection\label{\detokenize{scheduler:scheduler.Scheduler}}\pysiglinewithargsret{\sphinxbfcode{\sphinxupquote{class }}\sphinxcode{\sphinxupquote{scheduler.}}\sphinxbfcode{\sphinxupquote{Scheduler}}}{\emph{clock}, \emph{num\_people}, \emph{do\_minute\_by\_minute=False}}{}
Bases: \sphinxcode{\sphinxupquote{object}}

This class contains the code for the scheduler. The scheduler is in charge of jumping forward in time and     stopping at only potentially relevant time steps. The scheduler keeps track of the needs for every person in     in the household and stops at time steps where any person should have an action / need that needs to be     addressed.
\begin{quote}\begin{description}
\item[{Parameters}] \leavevmode\begin{itemize}
\item {} 
\sphinxstyleliteralstrong{\sphinxupquote{clock}} ({\hyperref[\detokenize{temporal:temporal.Temporal}]{\sphinxcrossref{\sphinxstyleliteralemphasis{\sphinxupquote{temporal.Temporal}}}}}) \textendash{} the time

\item {} 
\sphinxstyleliteralstrong{\sphinxupquote{num\_people}} (\sphinxstyleliteralemphasis{\sphinxupquote{int}}) \textendash{} the number of people in the household

\end{itemize}

\item[{Variables}] \leavevmode\begin{itemize}
\item {} 
\sphinxstyleliteralstrong{\sphinxupquote{clock}} ({\hyperref[\detokenize{temporal:temporal.Temporal}]{\sphinxcrossref{\sphinxstyleliteralemphasis{\sphinxupquote{temporal.Temporal}}}}}) \textendash{} the time

\item {} 
\sphinxstyleliteralstrong{\sphinxupquote{A}} (\sphinxstyleliteralemphasis{\sphinxupquote{numpy.ndarray}}) \textendash{} the schedule matrix of dimension (number of people x number of needs). This matrix     contains the times {[}minutes, universal time{]} that the simulation should not skip over

\item {} 
\sphinxstyleliteralstrong{\sphinxupquote{dt}} (\sphinxstyleliteralemphasis{\sphinxupquote{int}}) \textendash{} the duration of time between events

\item {} 
\sphinxstyleliteralstrong{\sphinxupquote{t\_old}} (\sphinxstyleliteralemphasis{\sphinxupquote{int}}) \textendash{} the time {[}minutes, universal time{]} of the prior event

\item {} 
\sphinxstyleliteralstrong{\sphinxupquote{do\_minute\_by\_minute}} (\sphinxstyleliteralemphasis{\sphinxupquote{bool}}) \textendash{} this flag controls whether the schedule should either     go through time minute by minute (if True) or jump forward in time (if False). The default     is to jump forward in time

\end{itemize}

\end{description}\end{quote}
\index{get\_next\_event\_time() (scheduler.Scheduler method)}

\begin{fulllineitems}
\phantomsection\label{\detokenize{scheduler:scheduler.Scheduler.get_next_event_time}}\pysiglinewithargsret{\sphinxbfcode{\sphinxupquote{get\_next\_event\_time}}}{}{}
This function searches the schedule matrix and finds the next time that that model should handle.

\begin{sphinxadmonition}{note}{Note:}
This function is only capable of handling \sphinxstylestrong{single-occupancy} households.
\end{sphinxadmonition}
\begin{quote}\begin{description}
\item[{Returns}] \leavevmode
the next time {[}minutes, time of day{]} that the model should address

\item[{Return type}] \leavevmode
int

\end{description}\end{quote}

\end{fulllineitems}

\index{toString() (scheduler.Scheduler method)}

\begin{fulllineitems}
\phantomsection\label{\detokenize{scheduler:scheduler.Scheduler.toString}}\pysiglinewithargsret{\sphinxbfcode{\sphinxupquote{toString}}}{}{}
This function presents the Scheduler object as a string.
\begin{quote}\begin{description}
\item[{Returns}] \leavevmode
a string representation of the object

\end{description}\end{quote}

\end{fulllineitems}

\index{update() (scheduler.Scheduler method)}

\begin{fulllineitems}
\phantomsection\label{\detokenize{scheduler:scheduler.Scheduler.update}}\pysiglinewithargsret{\sphinxbfcode{\sphinxupquote{update}}}{\emph{id\_person}, \emph{id\_need}, \emph{dt}}{}
This function updates the schedule matrix for a given person and need with the duration for the next event, 
for the respective person-need combination.
\begin{quote}\begin{description}
\item[{Parameters}] \leavevmode\begin{itemize}
\item {} 
\sphinxstyleliteralstrong{\sphinxupquote{id\_person}} (\sphinxstyleliteralemphasis{\sphinxupquote{int}}) \textendash{} the person identifier

\item {} 
\sphinxstyleliteralstrong{\sphinxupquote{id\_need}} (\sphinxstyleliteralemphasis{\sphinxupquote{int}}) \textendash{} the need identifier

\item {} 
\sphinxstyleliteralstrong{\sphinxupquote{dt}} (\sphinxstyleliteralemphasis{\sphinxupquote{int}}) \textendash{} the duration to the next event

\end{itemize}

\item[{Returns}] \leavevmode
None

\end{description}\end{quote}

\end{fulllineitems}


\end{fulllineitems}



\subsection{sleep module}
\label{\detokenize{sleep::doc}}\label{\detokenize{sleep:sleep-module}}\label{\detokenize{sleep:module-sleep}}\index{sleep (module)}
This module contains information about the activity dealing with sleeping. This class is an Activity ({\hyperref[\detokenize{activity:activity.Activity}]{\sphinxcrossref{\sphinxcode{\sphinxupquote{activity.Activity}}}}}) that gives a person ({\hyperref[\detokenize{person:person.Person}]{\sphinxcrossref{\sphinxcode{\sphinxupquote{person.Person}}}}}) the ability to eat and satisfy the need Rest ({\hyperref[\detokenize{rest:rest.Rest}]{\sphinxcrossref{\sphinxcode{\sphinxupquote{rest.Rest}}}}}).

This file contains class {\hyperref[\detokenize{sleep:sleep.Sleep}]{\sphinxcrossref{\sphinxcode{\sphinxupquote{sleep.Sleep}}}}}.
\index{Sleep (class in sleep)}

\begin{fulllineitems}
\phantomsection\label{\detokenize{sleep:sleep.Sleep}}\pysigline{\sphinxbfcode{\sphinxupquote{class }}\sphinxcode{\sphinxupquote{sleep.}}\sphinxbfcode{\sphinxupquote{Sleep}}}
Bases: {\hyperref[\detokenize{activity:activity.Activity}]{\sphinxcrossref{\sphinxcode{\sphinxupquote{activity.Activity}}}}}

This class is responsible for the act of sleeping, which satisfies the need {\hyperref[\detokenize{rest:rest.Rest}]{\sphinxcrossref{\sphinxcode{\sphinxupquote{rest.Rest}}}}}.
\index{advertise() (sleep.Sleep method)}

\begin{fulllineitems}
\phantomsection\label{\detokenize{sleep:sleep.Sleep.advertise}}\pysiglinewithargsret{\sphinxbfcode{\sphinxupquote{advertise}}}{\emph{p}}{}
This function calculates the score of an activity advertisement to a Person
\begin{quote}\begin{description}
\item[{Parameters}] \leavevmode
\sphinxstyleliteralstrong{\sphinxupquote{p}} ({\hyperref[\detokenize{person:person.Person}]{\sphinxcrossref{\sphinxstyleliteralemphasis{\sphinxupquote{person.Person}}}}}) \textendash{} the person being advertised to

\item[{Returns}] \leavevmode
the value of the advertisement

\item[{Return type}] \leavevmode
float

\end{description}\end{quote}

\end{fulllineitems}

\index{end() (sleep.Sleep method)}

\begin{fulllineitems}
\phantomsection\label{\detokenize{sleep:sleep.Sleep.end}}\pysiglinewithargsret{\sphinxbfcode{\sphinxupquote{end}}}{\emph{p}}{}
This handles the end of the sleep activity.
\begin{quote}\begin{description}
\item[{Parameters}] \leavevmode
\sphinxstyleliteralstrong{\sphinxupquote{p}} ({\hyperref[\detokenize{person:person.Person}]{\sphinxcrossref{\sphinxstyleliteralemphasis{\sphinxupquote{person.Person}}}}}) \textendash{} the person of interest

\item[{Returns}] \leavevmode
None

\end{description}\end{quote}

\end{fulllineitems}

\index{end\_sleep() (sleep.Sleep method)}

\begin{fulllineitems}
\phantomsection\label{\detokenize{sleep:sleep.Sleep.end_sleep}}\pysiglinewithargsret{\sphinxbfcode{\sphinxupquote{end\_sleep}}}{\emph{p}}{}
This function addresses logistics with a person waking up from sleep. More         specifically, the function does the following:
\begin{enumerate}
\item {} 
free the asset from use

\item {} 
set the state of the person to idle (\sphinxcode{\sphinxupquote{state.IDLE}})

\item {} 
update the satiation

\item {} 
update the start time and end time

\item {} 
set the decay rate

\item {} 
update the schedule for the rest need

\end{enumerate}
\begin{quote}\begin{description}
\item[{Parameters}] \leavevmode
\sphinxstyleliteralstrong{\sphinxupquote{p}} ({\hyperref[\detokenize{person:person.Person}]{\sphinxcrossref{\sphinxstyleliteralemphasis{\sphinxupquote{person.Person}}}}}) \textendash{} the person of interest

\item[{Returns}] \leavevmode
None

\end{description}\end{quote}

\end{fulllineitems}

\index{is\_workday() (sleep.Sleep method)}

\begin{fulllineitems}
\phantomsection\label{\detokenize{sleep:sleep.Sleep.is_workday}}\pysiglinewithargsret{\sphinxbfcode{\sphinxupquote{is\_workday}}}{\emph{p}}{}
This function indicates whether or not the sleep event resembles that from a person sleeping for         a workday.
\begin{quote}\begin{description}
\item[{Parameters}] \leavevmode
\sphinxstyleliteralstrong{\sphinxupquote{p}} ({\hyperref[\detokenize{person:person.Person}]{\sphinxcrossref{\sphinxstyleliteralemphasis{\sphinxupquote{person.Person}}}}}) \textendash{} the person of interest

\item[{Returns}] \leavevmode
True, if the sleep event resembles a workday. False, otherwise.

\end{description}\end{quote}

\end{fulllineitems}

\index{set\_end\_time() (sleep.Sleep method)}

\begin{fulllineitems}
\phantomsection\label{\detokenize{sleep:sleep.Sleep.set_end_time}}\pysiglinewithargsret{\sphinxbfcode{\sphinxupquote{set\_end\_time}}}{\emph{p}}{}
This function returns the end time of sleeping. The end time \(t_{end}\)         is set as follows:
\[
\begin{cases}
    \Delta{t} &= \frac{ 1 - n(t) }{ m_{suggested} } \\
    t_{end} &= t + \Delta{t}
\end{cases}
\]\begin{description}
\item[{where}] \leavevmode\begin{itemize}
\item {} 
\(\Delta{t}\) is the duration of sleep {[}minutes{]}

\item {} 
\(t_{end}\) is the end time of the sleep activity {[}universal time, minutes{]}

\item {} 
\(m_{suggested}\) is the suggested recharge rate for Rest

\item {} 
\(n(t)\) is the satiation of Rest at time t

\end{itemize}

\end{description}
\begin{quote}\begin{description}
\item[{Parameters}] \leavevmode
\sphinxstyleliteralstrong{\sphinxupquote{p}} ({\hyperref[\detokenize{person:person.Person}]{\sphinxcrossref{\sphinxstyleliteralemphasis{\sphinxupquote{person.Person}}}}}) \textendash{} the person of interest

\item[{Return t\_end}] \leavevmode
the end time of the sleep event {[}minutes, universal time{]}

\item[{Return type}] \leavevmode
int

\end{description}\end{quote}

\end{fulllineitems}

\index{start() (sleep.Sleep method)}

\begin{fulllineitems}
\phantomsection\label{\detokenize{sleep:sleep.Sleep.start}}\pysiglinewithargsret{\sphinxbfcode{\sphinxupquote{start}}}{\emph{p}}{}
This handles the start of the sleep activity.
\begin{quote}\begin{description}
\item[{Parameters}] \leavevmode
\sphinxstyleliteralstrong{\sphinxupquote{p}} ({\hyperref[\detokenize{person:person.Person}]{\sphinxcrossref{\sphinxstyleliteralemphasis{\sphinxupquote{person.Person}}}}}) \textendash{} the person of interest

\item[{Returns}] \leavevmode
None

\end{description}\end{quote}

\end{fulllineitems}

\index{start\_sleep() (sleep.Sleep method)}

\begin{fulllineitems}
\phantomsection\label{\detokenize{sleep:sleep.Sleep.start_sleep}}\pysiglinewithargsret{\sphinxbfcode{\sphinxupquote{start\_sleep}}}{\emph{p}}{}
This handles what happens when a person goes to sleep. Specifically, this function does the         following:
\begin{enumerate}
\item {} 
the asset’s status is updated.

\item {} 
the person’s state is set to the sleep state (\sphinxcode{\sphinxupquote{state.SLEEP}})

\item {} 
the end time is calculated

\item {} 
the recharge rate is set (according to whether or not it is a workday / non-workday)

\end{enumerate}
\begin{quote}\begin{description}
\item[{Parameters}] \leavevmode
\sphinxstyleliteralstrong{\sphinxupquote{p}} ({\hyperref[\detokenize{person:person.Person}]{\sphinxcrossref{\sphinxstyleliteralemphasis{\sphinxupquote{person.Person}}}}}) \textendash{} the person of interest

\item[{Returns}] \leavevmode
None

\end{description}\end{quote}

\end{fulllineitems}

\index{toString() (sleep.Sleep method)}

\begin{fulllineitems}
\phantomsection\label{\detokenize{sleep:sleep.Sleep.toString}}\pysiglinewithargsret{\sphinxbfcode{\sphinxupquote{toString}}}{}{}
This function represents the Sleep object as a string
\begin{quote}\begin{description}
\item[{Return msg}] \leavevmode
the representation of the Sleep object

\item[{Return type}] \leavevmode
str

\end{description}\end{quote}

\end{fulllineitems}


\end{fulllineitems}



\subsection{social module}
\label{\detokenize{social::doc}}\label{\detokenize{social:module-social}}\label{\detokenize{social:social-module}}\index{social (module)}
This module contains code that governs the social behavior/ characteristics relevant to a Person ({\hyperref[\detokenize{person:person.Person}]{\sphinxcrossref{\sphinxcode{\sphinxupquote{person.Person}}}}}).

This module contains class {\hyperref[\detokenize{social:social.Social}]{\sphinxcrossref{\sphinxcode{\sphinxupquote{social.Social}}}}}.
\index{Social (class in social)}

\begin{fulllineitems}
\phantomsection\label{\detokenize{social:social.Social}}\pysiglinewithargsret{\sphinxbfcode{\sphinxupquote{class }}\sphinxcode{\sphinxupquote{social.}}\sphinxbfcode{\sphinxupquote{Social}}}{\emph{age}, \emph{num\_meals=3}}{}
Bases: \sphinxcode{\sphinxupquote{object}}

This class contains all of the relevant information governing the person’s     social behavior.

\begin{sphinxadmonition}{note}{Note:}
The current version of ABMHAP does not have any “alarm” functionality / capability. The remnants of any         code that governs the use of an alarm  will be removed in future updates.
\end{sphinxadmonition}
\begin{quote}\begin{description}
\item[{Parameters}] \leavevmode\begin{itemize}
\item {} 
\sphinxstyleliteralstrong{\sphinxupquote{age}} (\sphinxstyleliteralemphasis{\sphinxupquote{int}}) \textendash{} the age of the person {[}years{]}

\item {} 
\sphinxstyleliteralstrong{\sphinxupquote{num\_meals}} (\sphinxstyleliteralemphasis{\sphinxupquote{int}}) \textendash{} the number of meals per day

\end{itemize}

\item[{Variables}] \leavevmode\begin{itemize}
\item {} 
\sphinxstyleliteralstrong{\sphinxupquote{is\_child}} (\sphinxstyleliteralemphasis{\sphinxupquote{bool}}) \textendash{} this flag is True if the person is a child, False otherwise

\item {} 
\sphinxstyleliteralstrong{\sphinxupquote{job}} ({\hyperref[\detokenize{occupation:occupation.Occupation}]{\sphinxcrossref{\sphinxstyleliteralemphasis{\sphinxupquote{occupation.Occupation}}}}}) \textendash{} the information pertaining the the job

\item {} 
\sphinxstyleliteralstrong{\sphinxupquote{num\_meals}} (\sphinxstyleliteralemphasis{\sphinxupquote{int}}) \textendash{} the number of meals per day a person will eat

\item {} 
\sphinxstyleliteralstrong{\sphinxupquote{meals}} (\sphinxstyleliteralemphasis{\sphinxupquote{list}}) \textendash{} a list of the meals that a person eats ({\hyperref[\detokenize{meal:meal.Meal}]{\sphinxcrossref{\sphinxcode{\sphinxupquote{meal.Meal}}}}})

\item {} 
\sphinxstyleliteralstrong{\sphinxupquote{current\_meal}} ({\hyperref[\detokenize{meal:meal.Meal}]{\sphinxcrossref{\sphinxstyleliteralemphasis{\sphinxupquote{meal.Meal}}}}}) \textendash{} the meal that is currently being eaten \sphinxstylestrong{or} if the person is not eating a meal,     it is the upcoming meal

\item {} 
\sphinxstyleliteralstrong{\sphinxupquote{next\_meal}} ({\hyperref[\detokenize{meal:meal.Meal}]{\sphinxcrossref{\sphinxstyleliteralemphasis{\sphinxupquote{meal.Meal}}}}}) \textendash{} the meal that is after the meal indicated by \sphinxcode{\sphinxupquote{current\_meal}}

\item {} 
\sphinxstyleliteralstrong{\sphinxupquote{uses\_alarm}} (\sphinxstyleliteralemphasis{\sphinxupquote{bool}}) \textendash{} indicates whether or not a person uses an alarm to wake up

\item {} 
\sphinxstyleliteralstrong{\sphinxupquote{is\_alarm\_set}} (\sphinxstyleliteralemphasis{\sphinxupquote{bool}}) \textendash{} indicates whether or not an alarm is set for the current day

\item {} 
\sphinxstyleliteralstrong{\sphinxupquote{t\_alarm}} (\sphinxstyleliteralemphasis{\sphinxupquote{int}}) \textendash{} the time an alarm is supposed to go off {[}minutes, time of day{]}

\end{itemize}

\end{description}\end{quote}
\index{duration\_to\_next\_commute\_event() (social.Social method)}

\begin{fulllineitems}
\phantomsection\label{\detokenize{social:social.Social.duration_to_next_commute_event}}\pysiglinewithargsret{\sphinxbfcode{\sphinxupquote{duration\_to\_next\_commute\_event}}}{\emph{clock}}{}
This function is called in in order to calculate the amount of time until the next commute event by         doing the following.
\begin{enumerate}
\item {} 
If the agent is unemployed, return infinity

\item {} 
If the time indicates that the agent should be currently working, set the duration to be the         length of time remaining at work

\item {} 
If the time indicates that the agent should be currently commuting to work, set the duration to be         the duration until the commute to work should start

\item {} 
If the time indicates that the agent should be currently commuting from work, set the duration to be         the amount of time until the commute from work should end

\item {} 
Else, calculate the amount of time until the next commute to work event

\end{enumerate}

\begin{sphinxadmonition}{note}{Note:}
The only reason this code is place here is because the work activity and the commute activity use it.
\end{sphinxadmonition}
\begin{quote}\begin{description}
\item[{Parameters}] \leavevmode
\sphinxstyleliteralstrong{\sphinxupquote{clock}} ({\hyperref[\detokenize{temporal:temporal.Temporal}]{\sphinxcrossref{\sphinxstyleliteralemphasis{\sphinxupquote{temporal.Temporal}}}}}) \textendash{} the current time

\item[{Returns}] \leavevmode
the duration in time {[}mintues{]} until the next commute event

\item[{Return type}] \leavevmode
int

\end{description}\end{quote}

\end{fulllineitems}

\index{duration\_to\_next\_meal() (social.Social method)}

\begin{fulllineitems}
\phantomsection\label{\detokenize{social:social.Social.duration_to_next_meal}}\pysiglinewithargsret{\sphinxbfcode{\sphinxupquote{duration\_to\_next\_meal}}}{\emph{t\_univ}}{}
This function calculates the amount of time until the next meal.
\begin{quote}\begin{description}
\item[{Parameters}] \leavevmode
\sphinxstyleliteralstrong{\sphinxupquote{t\_univ}} (\sphinxstyleliteralemphasis{\sphinxupquote{int}}) \textendash{} the current time {[}minutes, universal time{]}

\item[{Returns}] \leavevmode
the duration to the next meal {[}minutes{]}

\item[{Return type}] \leavevmode
int

\item[{Returns}] \leavevmode
the scheduled next meal

\item[{Return type}] \leavevmode
{\hyperref[\detokenize{meal:meal.Meal}]{\sphinxcrossref{meal.Meal}}}

\end{description}\end{quote}

\end{fulllineitems}

\index{duration\_to\_work\_event() (social.Social method)}

\begin{fulllineitems}
\phantomsection\label{\detokenize{social:social.Social.duration_to_work_event}}\pysiglinewithargsret{\sphinxbfcode{\sphinxupquote{duration\_to\_work\_event}}}{\emph{clock}}{}
This function is called in in order to calculate the amount of time until the next work event.
\begin{enumerate}
\item {} 
If the person is employed, the duration to the next meal is set to infinity

\item {} 
If the current time is a workday before the time work starts,
\begin{itemize}
\item {} 
set the duration to the amount of time until the start of work

\end{itemize}

\item {} 
Else,
\begin{itemize}
\item {} 
set the duration until the next work event

\end{itemize}

\end{enumerate}

\begin{sphinxadmonition}{note}{Note:}
The only reason this code is place here is because the work activity and the commute activity use it.
\end{sphinxadmonition}
\begin{quote}\begin{description}
\item[{Parameters}] \leavevmode
\sphinxstyleliteralstrong{\sphinxupquote{clock}} ({\hyperref[\detokenize{temporal:temporal.Temporal}]{\sphinxcrossref{\sphinxstyleliteralemphasis{\sphinxupquote{temporal.Temporal}}}}}) \textendash{} the current time

\item[{Returns}] \leavevmode
the duration {[}minutes{]} until the next minutes

\item[{Return type}] \leavevmode
int

\end{description}\end{quote}

\end{fulllineitems}

\index{get\_current\_meal() (social.Social method)}

\begin{fulllineitems}
\phantomsection\label{\detokenize{social:social.Social.get_current_meal}}\pysiglinewithargsret{\sphinxbfcode{\sphinxupquote{get\_current\_meal}}}{\emph{time\_of\_day}}{}
This function gets the closest meal to the time of day.
\begin{quote}\begin{description}
\item[{Parameters}] \leavevmode
\sphinxstyleliteralstrong{\sphinxupquote{time\_of\_day}} (\sphinxstyleliteralemphasis{\sphinxupquote{int}}) \textendash{} the time of day

\item[{Returns}] \leavevmode
return the meal

\item[{Return type}] \leavevmode
{\hyperref[\detokenize{meal:meal.Meal}]{\sphinxcrossref{meal.Meal}}}

\end{description}\end{quote}

\end{fulllineitems}

\index{get\_meal() (social.Social method)}

\begin{fulllineitems}
\phantomsection\label{\detokenize{social:social.Social.get_meal}}\pysiglinewithargsret{\sphinxbfcode{\sphinxupquote{get\_meal}}}{\emph{id\_meal}}{}
Get the specific meal given by a meal identifier.
\begin{quote}\begin{description}
\item[{Parameters}] \leavevmode
\sphinxstyleliteralstrong{\sphinxupquote{id\_meal}} (\sphinxstyleliteralemphasis{\sphinxupquote{int}}) \textendash{} the meal identifier

\item[{Returns}] \leavevmode
the meal given by the id

\item[{Return type}] \leavevmode
{\hyperref[\detokenize{meal:meal.Meal}]{\sphinxcrossref{meal.Meal}}}

\end{description}\end{quote}

\end{fulllineitems}

\index{get\_next\_meal() (social.Social method)}

\begin{fulllineitems}
\phantomsection\label{\detokenize{social:social.Social.get_next_meal}}\pysiglinewithargsret{\sphinxbfcode{\sphinxupquote{get\_next\_meal}}}{\emph{clock}}{}
This function gets the next meal. The meal must occur after the current time.
\begin{quote}\begin{description}
\item[{Parameters}] \leavevmode
\sphinxstyleliteralstrong{\sphinxupquote{clock}} ({\hyperref[\detokenize{temporal:temporal.Temporal}]{\sphinxcrossref{\sphinxstyleliteralemphasis{\sphinxupquote{temporal.Temporal}}}}}) \textendash{} the current time

\item[{Returns}] \leavevmode
the next meal

\item[{Return type}] \leavevmode
{\hyperref[\detokenize{meal:meal.Meal}]{\sphinxcrossref{meal.Meal}}}

\end{description}\end{quote}

\end{fulllineitems}

\index{print\_child\_status() (social.Social method)}

\begin{fulllineitems}
\phantomsection\label{\detokenize{social:social.Social.print_child_status}}\pysiglinewithargsret{\sphinxbfcode{\sphinxupquote{print\_child\_status}}}{}{}
This function represents the child status as a string.
\begin{quote}\begin{description}
\item[{Return msg}] \leavevmode
the child/ adult status

\item[{Return type}] \leavevmode
str

\end{description}\end{quote}

\end{fulllineitems}

\index{set\_child\_flag() (social.Social method)}

\begin{fulllineitems}
\phantomsection\label{\detokenize{social:social.Social.set_child_flag}}\pysiglinewithargsret{\sphinxbfcode{\sphinxupquote{set\_child\_flag}}}{\emph{age}}{}
Sets the flag indicating whether a person is a child.
\begin{quote}\begin{description}
\item[{Parameters}] \leavevmode
\sphinxstyleliteralstrong{\sphinxupquote{age}} (\sphinxstyleliteralemphasis{\sphinxupquote{int}}) \textendash{} the age of the person {[}years{]}

\item[{Returns}] \leavevmode
None

\end{description}\end{quote}

\end{fulllineitems}

\index{set\_job() (social.Social method)}

\begin{fulllineitems}
\phantomsection\label{\detokenize{social:social.Social.set_job}}\pysiglinewithargsret{\sphinxbfcode{\sphinxupquote{set\_job}}}{\emph{job\_id}, \emph{dt=0}}{}
This function sets the job and the alarm time (if used) that corresponds to the job. The alarm  is set,         if a person is using the alarm.

\begin{sphinxadmonition}{note}{Note:}
The current version of ABMHAP has no alarm capability.
\end{sphinxadmonition}
\begin{quote}\begin{description}
\item[{Parameters}] \leavevmode\begin{itemize}
\item {} 
\sphinxstyleliteralstrong{\sphinxupquote{job\_id}} (\sphinxstyleliteralemphasis{\sphinxupquote{int}}) \textendash{} job identifier

\item {} 
\sphinxstyleliteralstrong{\sphinxupquote{dt}} (\sphinxstyleliteralemphasis{\sphinxupquote{int}}) \textendash{} the amount of time before the job start.

\end{itemize}

\item[{Returns}] \leavevmode
None

\end{description}\end{quote}

\end{fulllineitems}

\index{set\_work\_alarm() (social.Social method)}

\begin{fulllineitems}
\phantomsection\label{\detokenize{social:social.Social.set_work_alarm}}\pysiglinewithargsret{\sphinxbfcode{\sphinxupquote{set\_work\_alarm}}}{\emph{dt=0}}{}
This sets the alarm time due to work. If a person uses an alarm, the alarm         is set to be “dt” minutes before work time.

\begin{sphinxadmonition}{note}{Note:}
The current version of ABMHAP has no alarm capability.
\end{sphinxadmonition}
\begin{quote}\begin{description}
\item[{Parameters}] \leavevmode
\sphinxstyleliteralstrong{\sphinxupquote{dt}} (\sphinxstyleliteralemphasis{\sphinxupquote{int}}) \textendash{} the amount of time to wake up before the work event {[}minutes{]}

\item[{Returns}] \leavevmode
None

\end{description}\end{quote}

\end{fulllineitems}

\index{test\_func() (social.Social method)}

\begin{fulllineitems}
\phantomsection\label{\detokenize{social:social.Social.test_func}}\pysiglinewithargsret{\sphinxbfcode{\sphinxupquote{test\_func}}}{\emph{time\_of\_day}, \emph{the\_meal}}{}
This is used for testing.

\begin{sphinxadmonition}{note}{Note:}
This function has no real purpose and will be deleted in future versions.
\end{sphinxadmonition}
\begin{quote}\begin{description}
\item[{Parameters}] \leavevmode\begin{itemize}
\item {} 
\sphinxstyleliteralstrong{\sphinxupquote{time\_of\_day}} (\sphinxstyleliteralemphasis{\sphinxupquote{int}}) \textendash{} the time of day in minutes

\item {} 
\sphinxstyleliteralstrong{\sphinxupquote{the\_meal}} ({\hyperref[\detokenize{meal:meal.Meal}]{\sphinxcrossref{\sphinxstyleliteralemphasis{\sphinxupquote{meal.Meal}}}}}) \textendash{} a meal object

\end{itemize}

\item[{Returns}] \leavevmode
None

\end{description}\end{quote}

\end{fulllineitems}

\index{toString() (social.Social method)}

\begin{fulllineitems}
\phantomsection\label{\detokenize{social:social.Social.toString}}\pysiglinewithargsret{\sphinxbfcode{\sphinxupquote{toString}}}{}{}
Represents the Social object as a string.
\begin{quote}\begin{description}
\item[{Returns}] \leavevmode
the representation of the Social object

\item[{Return type}] \leavevmode
str

\end{description}\end{quote}

\end{fulllineitems}


\end{fulllineitems}



\subsection{state module}
\label{\detokenize{state::doc}}\label{\detokenize{state:state-module}}\label{\detokenize{state:module-state}}\index{state (module)}
This module contains code that governs information relevant to a person’s state.

This module contains class {\hyperref[\detokenize{state:state.State}]{\sphinxcrossref{\sphinxcode{\sphinxupquote{state.State}}}}}.
\index{State (class in state)}

\begin{fulllineitems}
\phantomsection\label{\detokenize{state:state.State}}\pysiglinewithargsret{\sphinxbfcode{\sphinxupquote{class }}\sphinxcode{\sphinxupquote{state.}}\sphinxbfcode{\sphinxupquote{State}}}{\emph{status=0}}{}
Bases: \sphinxcode{\sphinxupquote{object}}

This class contains information relevant to a person state
\begin{quote}\begin{description}
\item[{Parameters}] \leavevmode
\sphinxstyleliteralstrong{\sphinxupquote{status}} (\sphinxstyleliteralemphasis{\sphinxupquote{int}}) \textendash{} the status of the person

\item[{Variables}] \leavevmode\begin{itemize}
\item {} 
\sphinxstyleliteralstrong{\sphinxupquote{'activity'}} ({\hyperref[\detokenize{activity:activity.Activity}]{\sphinxcrossref{\sphinxstyleliteralemphasis{\sphinxupquote{activity.Activity}}}}}) \textendash{} the particular activity of the asset

\item {} 
\sphinxstyleliteralstrong{\sphinxupquote{arg\_start}} (\sphinxstyleliteralemphasis{\sphinxupquote{list}}) \textendash{} the list of arguments for the start() function

\item {} 
\sphinxstyleliteralstrong{\sphinxupquote{arg\_end}} (\sphinxstyleliteralemphasis{\sphinxupquote{list}}) \textendash{} the list of arguments for the end() function

\item {} 
\sphinxstyleliteralstrong{\sphinxupquote{'asset'}} ({\hyperref[\detokenize{asset:asset.Asset}]{\sphinxcrossref{\sphinxstyleliteralemphasis{\sphinxupquote{asset.Asset}}}}}) \textendash{} the Asset that is being used

\item {} 
\sphinxstyleliteralstrong{\sphinxupquote{asset\_list}} (\sphinxstyleliteralemphasis{\sphinxupquote{list}}) \textendash{} 

\item {} 
\sphinxstyleliteralstrong{\sphinxupquote{is\_init}} (\sphinxstyleliteralemphasis{\sphinxupquote{bool}}) \textendash{} this is a flag indicating whether or not the agent is in the initialization state.     This state only occurs during the first step of the simulation.

\item {} 
\sphinxstyleliteralstrong{\sphinxupquote{status}} (\sphinxstyleliteralemphasis{\sphinxupquote{int}}) \textendash{} the status of a person

\item {} 
\sphinxstyleliteralstrong{\sphinxupquote{t\_end}} (\sphinxstyleliteralemphasis{\sphinxupquote{int}}) \textendash{} the end time of a state {[}minutes, universal time{]}

\item {} 
\sphinxstyleliteralstrong{\sphinxupquote{t\_start}} (\sphinxstyleliteralemphasis{\sphinxupquote{int}}) \textendash{} the start time of the current state {[}minutes, universal time{]}

\item {} 
\sphinxstyleliteralstrong{\sphinxupquote{round\_dt}} (\sphinxstyleliteralemphasis{\sphinxupquote{int}}) \textendash{} the amount of minutes {[}-1, 0, 1{]} to round an activity duration

\item {} 
\sphinxstyleliteralstrong{\sphinxupquote{dt\_frac}} (\sphinxstyleliteralemphasis{\sphinxupquote{float}}) \textendash{} the fraction of a minutes subtracted from rounding down from the true projected     activity duration

\item {} 
\sphinxstyleliteralstrong{\sphinxupquote{do\_interruption}} (\sphinxstyleliteralemphasis{\sphinxupquote{bool}}) \textendash{} a flag indicating whether the person is interrupting an ongoing activity

\end{itemize}

\end{description}\end{quote}
\index{end\_activity() (state.State method)}

\begin{fulllineitems}
\phantomsection\label{\detokenize{state:state.State.end_activity}}\pysiglinewithargsret{\sphinxbfcode{\sphinxupquote{end\_activity}}}{}{}
This function ends an activity.
\begin{quote}\begin{description}
\item[{Returns}] \leavevmode
None

\end{description}\end{quote}

\end{fulllineitems}

\index{halt\_activity() (state.State method)}

\begin{fulllineitems}
\phantomsection\label{\detokenize{state:state.State.halt_activity}}\pysiglinewithargsret{\sphinxbfcode{\sphinxupquote{halt\_activity}}}{\emph{p}}{}
This function runs the halt activity. The function is used by interruptions         to stop an activity \sphinxstylestrong{immediately} without giving benefits to the need that the         halted activity addressed.
\begin{quote}\begin{description}
\item[{Parameters}] \leavevmode
\sphinxstyleliteralstrong{\sphinxupquote{p}} ({\hyperref[\detokenize{person:person.Person}]{\sphinxcrossref{\sphinxstyleliteralemphasis{\sphinxupquote{person.Person}}}}}) \textendash{} the person of interest

\item[{Returns}] \leavevmode
None

\end{description}\end{quote}

\end{fulllineitems}

\index{print\_activity() (state.State method)}

\begin{fulllineitems}
\phantomsection\label{\detokenize{state:state.State.print_activity}}\pysiglinewithargsret{\sphinxbfcode{\sphinxupquote{print\_activity}}}{}{}
The string representation of the activity. This function handles the         possibility of the activity being None.
\begin{quote}\begin{description}
\item[{Returns}] \leavevmode
the representation of the activity

\item[{Return type}] \leavevmode
str

\end{description}\end{quote}

\end{fulllineitems}

\index{print\_asset() (state.State method)}

\begin{fulllineitems}
\phantomsection\label{\detokenize{state:state.State.print_asset}}\pysiglinewithargsret{\sphinxbfcode{\sphinxupquote{print\_asset}}}{}{}
This function represents the asset as a string. This function handles         the possibility of the asset being None.
\begin{quote}\begin{description}
\item[{Returns}] \leavevmode
the representation of the asset

\item[{Return type}] \leavevmode
str

\end{description}\end{quote}

\end{fulllineitems}

\index{print\_status() (state.State method)}

\begin{fulllineitems}
\phantomsection\label{\detokenize{state:state.State.print_status}}\pysiglinewithargsret{\sphinxbfcode{\sphinxupquote{print\_status}}}{}{}
This function represents the status as a string.
\begin{quote}\begin{description}
\item[{Returns}] \leavevmode
the representation of the status

\item[{Return type}] \leavevmode
str

\end{description}\end{quote}

\end{fulllineitems}

\index{reset() (state.State method)}

\begin{fulllineitems}
\phantomsection\label{\detokenize{state:state.State.reset}}\pysiglinewithargsret{\sphinxbfcode{\sphinxupquote{reset}}}{\emph{t\_univ}}{}
Reset the state object to the default behavior at the beginning of the simulation.
\begin{quote}\begin{description}
\item[{Parameters}] \leavevmode
\sphinxstyleliteralstrong{\sphinxupquote{t\_univ}} (\sphinxstyleliteralemphasis{\sphinxupquote{int}}) \textendash{} the time of the beginning of the simulation in universal time {[}seconds{]}

\item[{Returns}] \leavevmode
None

\end{description}\end{quote}

\end{fulllineitems}

\index{reset\_rounding\_parameters() (state.State method)}

\begin{fulllineitems}
\phantomsection\label{\detokenize{state:state.State.reset_rounding_parameters}}\pysiglinewithargsret{\sphinxbfcode{\sphinxupquote{reset\_rounding\_parameters}}}{}{}
This function resets the rounding parameters to zero.
\begin{quote}\begin{description}
\item[{Returns}] \leavevmode
None

\end{description}\end{quote}

\end{fulllineitems}

\index{reset\_time\_status() (state.State method)}

\begin{fulllineitems}
\phantomsection\label{\detokenize{state:state.State.reset_time_status}}\pysiglinewithargsret{\sphinxbfcode{\sphinxupquote{reset\_time\_status}}}{\emph{t\_start}, \emph{status=0}}{}
This function resets the time information to the current time and         sets the status. This function is usually used at the end of an activity.
\begin{quote}\begin{description}
\item[{Parameters}] \leavevmode\begin{itemize}
\item {} 
\sphinxstyleliteralstrong{\sphinxupquote{t\_start}} (\sphinxstyleliteralemphasis{\sphinxupquote{int}}) \textendash{} the start time {[}minutes, universal time{]}

\item {} 
\sphinxstyleliteralstrong{\sphinxupquote{status}} (\sphinxstyleliteralemphasis{\sphinxupquote{int}}) \textendash{} the status of the person

\end{itemize}

\item[{Returns}] \leavevmode
None

\end{description}\end{quote}

\end{fulllineitems}

\index{run\_activity() (state.State method)}

\begin{fulllineitems}
\phantomsection\label{\detokenize{state:state.State.run_activity}}\pysiglinewithargsret{\sphinxbfcode{\sphinxupquote{run\_activity}}}{\emph{arg}, \emph{func}}{}
This function allows an activity to start, end, or halt
\begin{quote}\begin{description}
\item[{Parameters}] \leavevmode\begin{itemize}
\item {} 
\sphinxstyleliteralstrong{\sphinxupquote{arg}} (\sphinxstyleliteralemphasis{\sphinxupquote{list}}) \textendash{} arguments for the func() function

\item {} 
\sphinxstyleliteralstrong{\sphinxupquote{func}} (\sphinxstyleliteralemphasis{\sphinxupquote{function}}) \textendash{} arguments for the func() function

\end{itemize}

\item[{Returns}] \leavevmode
None

\end{description}\end{quote}

\end{fulllineitems}

\index{start\_activity() (state.State method)}

\begin{fulllineitems}
\phantomsection\label{\detokenize{state:state.State.start_activity}}\pysiglinewithargsret{\sphinxbfcode{\sphinxupquote{start\_activity}}}{}{}
This function starts an activity
\begin{quote}\begin{description}
\item[{Returns}] \leavevmode
None

\end{description}\end{quote}

\end{fulllineitems}

\index{toString() (state.State method)}

\begin{fulllineitems}
\phantomsection\label{\detokenize{state:state.State.toString}}\pysiglinewithargsret{\sphinxbfcode{\sphinxupquote{toString}}}{}{}
This function represents the State object as a string.
\begin{quote}\begin{description}
\item[{Returns}] \leavevmode
the representation of the State object

\item[{Return type}] \leavevmode
str

\end{description}\end{quote}

\end{fulllineitems}


\end{fulllineitems}



\subsection{temporal module}
\label{\detokenize{temporal::doc}}\label{\detokenize{temporal:temporal-module}}\label{\detokenize{temporal:module-temporal}}\index{temporal (module)}
This file contains code that handles the time related aspects of this code.

This file contains code for class {\hyperref[\detokenize{temporal:temporal.Temporal}]{\sphinxcrossref{\sphinxcode{\sphinxupquote{temporal.Temporal}}}}}. This file also includes other functions that are accessed outside of the Temporal class.
\index{Temporal (class in temporal)}

\begin{fulllineitems}
\phantomsection\label{\detokenize{temporal:temporal.Temporal}}\pysiglinewithargsret{\sphinxbfcode{\sphinxupquote{class }}\sphinxcode{\sphinxupquote{temporal.}}\sphinxbfcode{\sphinxupquote{Temporal}}}{\emph{t\_univ=0}}{}
Bases: \sphinxcode{\sphinxupquote{object}}

This class handles all the time keeping responsibilities.

Universal time is the total amount of time in minutes elapsed from the start of the
calendar year.

Day 0 at 0:00 corresponds to a universal time of 0

Day 1 at 0:00 corresponds to a universal time of 1 * 24 * 60

Day 359 at 0:00 corresponds to a universal time of 359 * 24 * 60
\begin{quote}\begin{description}
\item[{Parameters}] \leavevmode
\sphinxstyleliteralstrong{\sphinxupquote{t\_univ}} (\sphinxstyleliteralemphasis{\sphinxupquote{int}}) \textendash{} the time in universal time {[}minutes{]}

\item[{Variables}] \leavevmode\begin{itemize}
\item {} 
\sphinxstyleliteralstrong{\sphinxupquote{day}} (\sphinxstyleliteralemphasis{\sphinxupquote{int}}) \textendash{} the day number in the simulation

\item {} 
\sphinxstyleliteralstrong{\sphinxupquote{day\_of\_week}} (\sphinxstyleliteralemphasis{\sphinxupquote{int}}) \textendash{} a number 0, 1, 2, … 6 corresponding to days of the week where 0 is Sunday, 1 is     Monday,  … 6 is Saturday

\item {} 
\sphinxstyleliteralstrong{\sphinxupquote{dt}} (\sphinxstyleliteralemphasis{\sphinxupquote{int}}) \textendash{} the step size in the simulation {[}minutes{]} (\sphinxstylestrong{antiquated})

\item {} 
\sphinxstyleliteralstrong{\sphinxupquote{hour\_of\_day}} (\sphinxstyleliteralemphasis{\sphinxupquote{int}}) \textendash{} the hour of the day {[}0, 23{]}

\item {} 
\sphinxstyleliteralstrong{\sphinxupquote{is\_weekday}} (\sphinxstyleliteralemphasis{\sphinxupquote{bool}}) \textendash{} a flag indicating if it’s a weekday (Monday-Friday) if True. False, otherwise.

\item {} 
\sphinxstyleliteralstrong{\sphinxupquote{is\_night}} (\sphinxstyleliteralemphasis{\sphinxupquote{bool}}) \textendash{} a flag indicating if the time of day is after \sphinxstylestrong{dusk} and before \sphinxstylestrong{dawn} if True.     False, otherwise.

\item {} 
\sphinxstyleliteralstrong{\sphinxupquote{min\_of\_day}} (\sphinxstyleliteralemphasis{\sphinxupquote{int}}) \textendash{} the minute of the day {[}0, 60 - 1{]}

\item {} 
\sphinxstyleliteralstrong{\sphinxupquote{t\_univ}} (\sphinxstyleliteralemphasis{\sphinxupquote{int}}) \textendash{} the universal time {[}minutes{]}

\item {} 
\sphinxstyleliteralstrong{\sphinxupquote{time\_of\_day}} (\sphinxstyleliteralemphasis{\sphinxupquote{int}}) \textendash{} the time of the day {[}minutes{]}, {[}0, 1, … 24 * 60 -1{]}

\item {} 
\sphinxstyleliteralstrong{\sphinxupquote{season}} (\sphinxstyleliteralemphasis{\sphinxupquote{int}}) \textendash{} the season

\item {} 
\sphinxstyleliteralstrong{\sphinxupquote{tic}} (\sphinxstyleliteralemphasis{\sphinxupquote{int}}) \textendash{} indicates that current tick (each tick corresponds to a step of size dt)

\item {} 
\sphinxstyleliteralstrong{\sphinxupquote{step}} (\sphinxstyleliteralemphasis{\sphinxupquote{int}}) \textendash{} indicates the current step in the simulation {[}0, … num\_steps-1{]}

\end{itemize}

\end{description}\end{quote}
\index{print\_day\_night() (temporal.Temporal method)}

\begin{fulllineitems}
\phantomsection\label{\detokenize{temporal:temporal.Temporal.print_day_night}}\pysiglinewithargsret{\sphinxbfcode{\sphinxupquote{print\_day\_night}}}{}{}
Represents whether it’s day or night as a string
\begin{quote}\begin{description}
\item[{Return msg}] \leavevmode
daytime / nightime status (or an error message, if there is an error)

\item[{Return type}] \leavevmode
str

\end{description}\end{quote}

\end{fulllineitems}

\index{print\_day\_of\_week() (temporal.Temporal method)}

\begin{fulllineitems}
\phantomsection\label{\detokenize{temporal:temporal.Temporal.print_day_of_week}}\pysiglinewithargsret{\sphinxbfcode{\sphinxupquote{print\_day\_of\_week}}}{}{}
Represents the day of the week as a string
\begin{quote}\begin{description}
\item[{Return msg}] \leavevmode
the day of the week (or an error message, if there is an error)

\item[{Return type}] \leavevmode
str

\end{description}\end{quote}

\end{fulllineitems}

\index{print\_season() (temporal.Temporal method)}

\begin{fulllineitems}
\phantomsection\label{\detokenize{temporal:temporal.Temporal.print_season}}\pysiglinewithargsret{\sphinxbfcode{\sphinxupquote{print\_season}}}{}{}
Represents the seasons as a string
\begin{quote}\begin{description}
\item[{Returns}] \leavevmode
the season (or an error message, if there is an error)

\item[{Return type}] \leavevmode
str

\end{description}\end{quote}

\end{fulllineitems}

\index{print\_time\_of\_day\_to\_military() (temporal.Temporal method)}

\begin{fulllineitems}
\phantomsection\label{\detokenize{temporal:temporal.Temporal.print_time_of_day_to_military}}\pysiglinewithargsret{\sphinxbfcode{\sphinxupquote{print\_time\_of\_day\_to\_military}}}{}{}
Represents the time of day as military time.
\begin{quote}\begin{description}
\item[{Returns}] \leavevmode
the time of day in military time

\item[{Return type}] \leavevmode
str

\end{description}\end{quote}

\end{fulllineitems}

\index{reset() (temporal.Temporal method)}

\begin{fulllineitems}
\phantomsection\label{\detokenize{temporal:temporal.Temporal.reset}}\pysiglinewithargsret{\sphinxbfcode{\sphinxupquote{reset}}}{\emph{t\_univ}}{}
Reset the temporal object to the initial state.
\begin{quote}\begin{description}
\item[{Parameters}] \leavevmode
\sphinxstyleliteralstrong{\sphinxupquote{t\_univ}} (\sphinxstyleliteralemphasis{\sphinxupquote{int}}) \textendash{} The time {[}seconds, universal time{]} that the time should be reset to

\item[{Returns}] \leavevmode


\end{description}\end{quote}

\end{fulllineitems}

\index{set\_day\_of\_week() (temporal.Temporal method)}

\begin{fulllineitems}
\phantomsection\label{\detokenize{temporal:temporal.Temporal.set_day_of_week}}\pysiglinewithargsret{\sphinxbfcode{\sphinxupquote{set\_day\_of\_week}}}{}{}
This function sets the day of the week. In addition, this function sets the day count,         the day of the week, and a flag indicating whether it is a weekday or not.
\begin{quote}\begin{description}
\item[{Returns}] \leavevmode
None

\end{description}\end{quote}

\end{fulllineitems}

\index{set\_season() (temporal.Temporal method)}

\begin{fulllineitems}
\phantomsection\label{\detokenize{temporal:temporal.Temporal.set_season}}\pysiglinewithargsret{\sphinxbfcode{\sphinxupquote{set\_season}}}{}{}
This function sets the season. Day 0 is the beginning of winter.
\begin{quote}\begin{description}
\item[{Returns}] \leavevmode
None

\end{description}\end{quote}

\end{fulllineitems}

\index{set\_time() (temporal.Temporal method)}

\begin{fulllineitems}
\phantomsection\label{\detokenize{temporal:temporal.Temporal.set_time}}\pysiglinewithargsret{\sphinxbfcode{\sphinxupquote{set\_time}}}{}{}
This function sets all the time variable due to the universal time. This function sets
\begin{enumerate}
\item {} 
the time of day

\item {} 
the day of the week

\item {} 
the season

\item {} 
the tic.

\end{enumerate}
\begin{quote}\begin{description}
\item[{Returns}] \leavevmode
None

\end{description}\end{quote}

\end{fulllineitems}

\index{set\_time\_of\_day() (temporal.Temporal method)}

\begin{fulllineitems}
\phantomsection\label{\detokenize{temporal:temporal.Temporal.set_time_of_day}}\pysiglinewithargsret{\sphinxbfcode{\sphinxupquote{set\_time\_of\_day}}}{}{}
Given the universal time, this function sets the time of day in minutes.
\begin{quote}\begin{description}
\item[{Returns}] \leavevmode
None

\end{description}\end{quote}

\end{fulllineitems}

\index{toString() (temporal.Temporal method)}

\begin{fulllineitems}
\phantomsection\label{\detokenize{temporal:temporal.Temporal.toString}}\pysiglinewithargsret{\sphinxbfcode{\sphinxupquote{toString}}}{}{}
This function represents the Temporal object as a string.
\begin{quote}\begin{description}
\item[{Return msg}] \leavevmode
the representation of the temporal object

\item[{Return type}] \leavevmode
str

\end{description}\end{quote}

\end{fulllineitems}

\index{update\_time() (temporal.Temporal method)}

\begin{fulllineitems}
\phantomsection\label{\detokenize{temporal:temporal.Temporal.update_time}}\pysiglinewithargsret{\sphinxbfcode{\sphinxupquote{update\_time}}}{}{}
Increments the time by 1 time step.

\begin{sphinxadmonition}{warning}{Warning:}
This function is outdated!
\end{sphinxadmonition}
\begin{quote}\begin{description}
\item[{Returns}] \leavevmode
None

\end{description}\end{quote}

\end{fulllineitems}


\end{fulllineitems}

\index{convert\_cyclical\_to\_decimal() (in module temporal)}

\begin{fulllineitems}
\phantomsection\label{\detokenize{temporal:temporal.convert_cyclical_to_decimal}}\pysiglinewithargsret{\sphinxcode{\sphinxupquote{temporal.}}\sphinxbfcode{\sphinxupquote{convert\_cyclical\_to\_decimal}}}{\emph{t}}{}
This function converts cyclical time to decimal time
\begin{quote}\begin{description}
\item[{Parameters}] \leavevmode
\sphinxstyleliteralstrong{\sphinxupquote{t}} (\sphinxstyleliteralemphasis{\sphinxupquote{int}}) \textendash{} the time of day {[}minutes{]}

\item[{Return out}] \leavevmode
the time of day in {[}hours{]}

\item[{Return type}] \leavevmode
float

\end{description}\end{quote}

\end{fulllineitems}

\index{convert\_cylical\_to\_universal() (in module temporal)}

\begin{fulllineitems}
\phantomsection\label{\detokenize{temporal:temporal.convert_cylical_to_universal}}\pysiglinewithargsret{\sphinxcode{\sphinxupquote{temporal.}}\sphinxbfcode{\sphinxupquote{convert\_cylical\_to\_universal}}}{\emph{day}, \emph{time\_of\_day}}{}
This function converts a cyclical time to the universal time.
\begin{quote}\begin{description}
\item[{Parameters}] \leavevmode\begin{itemize}
\item {} 
\sphinxstyleliteralstrong{\sphinxupquote{day}} (\sphinxstyleliteralemphasis{\sphinxupquote{int}}) \textendash{} the day of the year

\item {} 
\sphinxstyleliteralstrong{\sphinxupquote{time\_of\_day}} (\sphinxstyleliteralemphasis{\sphinxupquote{int}}) \textendash{} the time of day {[}minutes{]}

\end{itemize}

\item[{Return t}] \leavevmode
the time in universal time

\item[{Return type}] \leavevmode
int

\end{description}\end{quote}

\end{fulllineitems}

\index{convert\_decimal\_to\_min() (in module temporal)}

\begin{fulllineitems}
\phantomsection\label{\detokenize{temporal:temporal.convert_decimal_to_min}}\pysiglinewithargsret{\sphinxcode{\sphinxupquote{temporal.}}\sphinxbfcode{\sphinxupquote{convert\_decimal\_to\_min}}}{\emph{t}}{}
This function takes in the time of day as a decimal and outputs the time in minutes
\begin{quote}\begin{description}
\item[{Parameters}] \leavevmode
\sphinxstyleliteralstrong{\sphinxupquote{t}} (\sphinxstyleliteralemphasis{\sphinxupquote{float}}) \textendash{} the time of day {[}0, 24) {[}hours{]}

\item[{Return out}] \leavevmode
the time of day {[}minutes{]}

\item[{Return type}] \leavevmode
int

\end{description}\end{quote}

\end{fulllineitems}

\index{convert\_universal\_to\_decimal() (in module temporal)}

\begin{fulllineitems}
\phantomsection\label{\detokenize{temporal:temporal.convert_universal_to_decimal}}\pysiglinewithargsret{\sphinxcode{\sphinxupquote{temporal.}}\sphinxbfcode{\sphinxupquote{convert\_universal\_to\_decimal}}}{\emph{t\_univ}}{}
This function takes in the universal time and converts it to the time of day in decimal format {[}0, 24)
\begin{quote}\begin{description}
\item[{Parameters}] \leavevmode
\sphinxstyleliteralstrong{\sphinxupquote{t\_univ}} (\sphinxstyleliteralemphasis{\sphinxupquote{int}}) \textendash{} the universal time {[}minutes{]}

\item[{Return out}] \leavevmode
the universal time {[}hours{]}

\item[{Return type}] \leavevmode
float

\end{description}\end{quote}

\end{fulllineitems}

\index{print\_military\_time() (in module temporal)}

\begin{fulllineitems}
\phantomsection\label{\detokenize{temporal:temporal.print_military_time}}\pysiglinewithargsret{\sphinxcode{\sphinxupquote{temporal.}}\sphinxbfcode{\sphinxupquote{print\_military\_time}}}{\emph{t}}{}
Represents the time of day in military time  assume that time is in minutes format.
\begin{quote}\begin{description}
\item[{Parameters}] \leavevmode
\sphinxstyleliteralstrong{\sphinxupquote{t}} (\sphinxstyleliteralemphasis{\sphinxupquote{int}}) \textendash{} the time of day {[}minutes{]}

\item[{Return msg}] \leavevmode
the time of day in military time 00:00

\item[{Return type}] \leavevmode
str

\end{description}\end{quote}

\end{fulllineitems}



\subsection{transport module}
\label{\detokenize{transport::doc}}\label{\detokenize{transport:module-transport}}\label{\detokenize{transport:transport-module}}\index{transport (module)}
This module contains information about the asset that allows a person to do the following activities:
\begin{enumerate}
\item {} 
commute to work

\item {} 
commute from work

\end{enumerate}

This module contains code for {\hyperref[\detokenize{transport:transport.Transport}]{\sphinxcrossref{\sphinxcode{\sphinxupquote{transport.Transport}}}}}.
\index{Transport (class in transport)}

\begin{fulllineitems}
\phantomsection\label{\detokenize{transport:transport.Transport}}\pysigline{\sphinxbfcode{\sphinxupquote{class }}\sphinxcode{\sphinxupquote{transport.}}\sphinxbfcode{\sphinxupquote{Transport}}}
Bases: {\hyperref[\detokenize{asset:asset.Asset}]{\sphinxcrossref{\sphinxcode{\sphinxupquote{asset.Asset}}}}}

This class is an asset that allows for commuting.

Activities in this asset:
\begin{enumerate}
\item {} 
{\hyperref[\detokenize{commute:commute.Commute_To_Work}]{\sphinxcrossref{\sphinxcode{\sphinxupquote{commute.Commute\_To\_Work}}}}}

\item {} 
{\hyperref[\detokenize{commute:commute.Commute_From_Work}]{\sphinxcrossref{\sphinxcode{\sphinxupquote{commute.Commute\_From\_Work}}}}}

\end{enumerate}
\index{initialize() (transport.Transport method)}

\begin{fulllineitems}
\phantomsection\label{\detokenize{transport:transport.Transport.initialize}}\pysiglinewithargsret{\sphinxbfcode{\sphinxupquote{initialize}}}{\emph{people}}{}
This function sets the transport location according to whether or not the Person is commuting to or         from work.

\begin{sphinxadmonition}{note}{Note:}
This function just sets the transport object to be at the home
\end{sphinxadmonition}
\begin{quote}\begin{description}
\item[{Parameters}] \leavevmode
\sphinxstyleliteralstrong{\sphinxupquote{people}} (\sphinxstyleliteralemphasis{\sphinxupquote{list}}) \textendash{} a list of people in the simulation

\item[{Returns}] \leavevmode
None

\end{description}\end{quote}

\end{fulllineitems}


\end{fulllineitems}



\subsection{travel module}
\label{\detokenize{travel::doc}}\label{\detokenize{travel:module-travel}}\label{\detokenize{travel:travel-module}}\index{travel (module)}
This module contains code for the need associated with the desire to move from one environment to another.

This file contains code for {\hyperref[\detokenize{travel:travel.Travel}]{\sphinxcrossref{\sphinxcode{\sphinxupquote{travel.Travel}}}}}.
\index{Travel (class in travel)}

\begin{fulllineitems}
\phantomsection\label{\detokenize{travel:travel.Travel}}\pysiglinewithargsret{\sphinxbfcode{\sphinxupquote{class }}\sphinxcode{\sphinxupquote{travel.}}\sphinxbfcode{\sphinxupquote{Travel}}}{\emph{clock}, \emph{num\_sample\_points}}{}
Bases: {\hyperref[\detokenize{need:need.Need}]{\sphinxcrossref{\sphinxcode{\sphinxupquote{need.Need}}}}}

This class governs the need for traveling.
\begin{quote}\begin{description}
\item[{Parameters}] \leavevmode\begin{itemize}
\item {} 
\sphinxstyleliteralstrong{\sphinxupquote{clock}} ({\hyperref[\detokenize{temporal:temporal.Temporal}]{\sphinxcrossref{\sphinxstyleliteralemphasis{\sphinxupquote{temporal.Temporal}}}}}) \textendash{} the time

\item {} 
\sphinxstyleliteralstrong{\sphinxupquote{num\_sample\_points}} (\sphinxstyleliteralemphasis{\sphinxupquote{int}}) \textendash{} the number of temporal nodes in the simulation

\end{itemize}

\end{description}\end{quote}
\index{decay() (travel.Travel method)}

\begin{fulllineitems}
\phantomsection\label{\detokenize{travel:travel.Travel.decay}}\pysiglinewithargsret{\sphinxbfcode{\sphinxupquote{decay}}}{\emph{p}}{}
This function decays the satiation. Travel for commuting only decays when the work need is low
\begin{quote}\begin{description}
\item[{Parameters}] \leavevmode
\sphinxstyleliteralstrong{\sphinxupquote{p}} ({\hyperref[\detokenize{person:person.Person}]{\sphinxcrossref{\sphinxstyleliteralemphasis{\sphinxupquote{person.Person}}}}}) \textendash{} the person whose satiation is decaying

\item[{Returns}] \leavevmode
None

\end{description}\end{quote}

\end{fulllineitems}

\index{decay\_work\_commute() (travel.Travel method)}

\begin{fulllineitems}
\phantomsection\label{\detokenize{travel:travel.Travel.decay_work_commute}}\pysiglinewithargsret{\sphinxbfcode{\sphinxupquote{decay\_work\_commute}}}{\emph{p}}{}
This decays the satiation level in order to commute to work. For the satiation to decay the         person needs the following:
\begin{enumerate}
\item {} 
the agent should leave the home to go to work

\item {} 
the agent should leave work to go home

\end{enumerate}
\begin{quote}\begin{description}
\item[{Parameters}] \leavevmode
\sphinxstyleliteralstrong{\sphinxupquote{p}} ({\hyperref[\detokenize{person:person.Person}]{\sphinxcrossref{\sphinxstyleliteralemphasis{\sphinxupquote{person.Person}}}}}) \textendash{} the person of interest

\item[{Returns}] \leavevmode
None

\end{description}\end{quote}

\end{fulllineitems}

\index{initialize() (travel.Travel method)}

\begin{fulllineitems}
\phantomsection\label{\detokenize{travel:travel.Travel.initialize}}\pysiglinewithargsret{\sphinxbfcode{\sphinxupquote{initialize}}}{\emph{p}}{}
This function initializes the Travel by updating the {\hyperref[\detokenize{scheduler:scheduler.Scheduler}]{\sphinxcrossref{\sphinxcode{\sphinxupquote{scheduler.Scheduler}}}}} for Travel
\begin{quote}\begin{description}
\item[{Parameters}] \leavevmode
\sphinxstyleliteralstrong{\sphinxupquote{p}} ({\hyperref[\detokenize{person:person.Person}]{\sphinxcrossref{\sphinxstyleliteralemphasis{\sphinxupquote{person.Person}}}}}) \textendash{} the person of interest

\item[{Returns}] \leavevmode
None

\end{description}\end{quote}

\end{fulllineitems}

\index{perceive() (travel.Travel method)}

\begin{fulllineitems}
\phantomsection\label{\detokenize{travel:travel.Travel.perceive}}\pysiglinewithargsret{\sphinxbfcode{\sphinxupquote{perceive}}}{\emph{clock}, \emph{job}}{}
This function gives the satiation for Travel if the Travel need is addressed now.
\begin{quote}\begin{description}
\item[{Note}] \leavevmode
going to work can only happen according to work hours of the job.

\item[{Parameters}] \leavevmode\begin{itemize}
\item {} 
\sphinxstyleliteralstrong{\sphinxupquote{clock}} ({\hyperref[\detokenize{temporal:temporal.Temporal}]{\sphinxcrossref{\sphinxstyleliteralemphasis{\sphinxupquote{temporal.Temporal}}}}}) \textendash{} the time the need to travel is perceived

\item {} 
\sphinxstyleliteralstrong{\sphinxupquote{job}} ({\hyperref[\detokenize{occupation:occupation.Occupation}]{\sphinxcrossref{\sphinxstyleliteralemphasis{\sphinxupquote{occupation.Occupation}}}}}) \textendash{} the job of the person

\end{itemize}

\item[{Return mag}] \leavevmode
the perceived magnitude of the need

\item[{Return type}] \leavevmode
float

\end{description}\end{quote}

\end{fulllineitems}


\end{fulllineitems}



\subsection{universe module}
\label{\detokenize{universe::doc}}\label{\detokenize{universe:module-universe}}\label{\detokenize{universe:universe-module}}\index{universe (module)}
This module contains code that is responsible for running the simulation. This file contains {\hyperref[\detokenize{universe:universe.Universe}]{\sphinxcrossref{\sphinxcode{\sphinxupquote{universe.Universe}}}}}.
\index{Universe (class in universe)}

\begin{fulllineitems}
\phantomsection\label{\detokenize{universe:universe.Universe}}\pysiglinewithargsret{\sphinxbfcode{\sphinxupquote{class }}\sphinxcode{\sphinxupquote{universe.}}\sphinxbfcode{\sphinxupquote{Universe}}}{\emph{num\_steps}, \emph{dt}, \emph{t\_start}, \emph{num\_people}, \emph{do\_minute\_by\_minute=False}}{}
Bases: \sphinxcode{\sphinxupquote{object}}

The Universe is the governing engine of the simulation. The Universe contains all agents     and objects. The Universe is responsible for running the simulation itself.
\begin{quote}\begin{description}
\item[{Parameters}] \leavevmode\begin{itemize}
\item {} 
\sphinxstyleliteralstrong{\sphinxupquote{num\_steps}} (\sphinxstyleliteralemphasis{\sphinxupquote{int}}) \textendash{} the number of time steps in the simulation

\item {} 
\sphinxstyleliteralstrong{\sphinxupquote{dt}} (\sphinxstyleliteralemphasis{\sphinxupquote{int}}) \textendash{} the step size in the simulation {[}minutes{]}

\item {} 
\sphinxstyleliteralstrong{\sphinxupquote{t\_start}} (\sphinxstyleliteralemphasis{\sphinxupquote{int}}) \textendash{} the start time for the simulation {[}minutes, universal time{]}

\item {} 
\sphinxstyleliteralstrong{\sphinxupquote{num\_people}} (\sphinxstyleliteralemphasis{\sphinxupquote{int}}) \textendash{} the number of people in the household

\end{itemize}

\item[{Variables}] \leavevmode\begin{itemize}
\item {} 
\sphinxstyleliteralstrong{\sphinxupquote{clock}} ({\hyperref[\detokenize{temporal:temporal.Temporal}]{\sphinxcrossref{\sphinxstyleliteralemphasis{\sphinxupquote{temporal.Temporal}}}}}) \textendash{} does the timekeeping in the simulation

\item {} 
\sphinxstyleliteralstrong{\sphinxupquote{"home"}} ({\hyperref[\detokenize{home:home.Home}]{\sphinxcrossref{\sphinxstyleliteralemphasis{\sphinxupquote{home.Home}}}}}) \textendash{} the home the persons live in

\item {} 
\sphinxstyleliteralstrong{\sphinxupquote{people}} (\sphinxstyleliteralemphasis{\sphinxupquote{list}}) \textendash{} a list of all person objects created in the Universe object

\item {} 
\sphinxstyleliteralstrong{\sphinxupquote{t\_start}} (\sphinxstyleliteralemphasis{\sphinxupquote{int}}) \textendash{} the start time for the simulation {[}minutes, universal time{]}

\item {} 
\sphinxstyleliteralstrong{\sphinxupquote{t\_end}} (\sphinxstyleliteralemphasis{\sphinxupquote{int}}) \textendash{} the last time for the simulation {[}minutes, universal time{]}

\item {} 
\sphinxstyleliteralstrong{\sphinxupquote{schedule}} ({\hyperref[\detokenize{scheduler:scheduler.Scheduler}]{\sphinxcrossref{\sphinxstyleliteralemphasis{\sphinxupquote{scheduler.Scheduler}}}}}) \textendash{} the schedule governing each agent’s needs

\end{itemize}

\end{description}\end{quote}
\index{address\_needs() (universe.Universe method)}

\begin{fulllineitems}
\phantomsection\label{\detokenize{universe:universe.Universe.address_needs}}\pysiglinewithargsret{\sphinxbfcode{\sphinxupquote{address\_needs}}}{\emph{do\_interruption=False}}{}
This function checks the needs of the agents. The function uses a recursion loop to         choose activities.

The recursion:
\begin{enumerate}
\item {} 
gather all of the advertisements (object-person pairings)

\item {} 
assigns 1 activity to the Person with the highest score

\item {} 
that Person starts the activity, thereby updating the state of available activities in the home

\item {} 
the recursion starts again, where the Home advertises to all remaining Person(s)

\end{enumerate}
\begin{quote}\begin{description}
\item[{Note}] \leavevmode
If no activity will be done this time step to a person, a person is set to             the temporary status \sphinxcode{\sphinxupquote{state.IDLE\_TEMP}}, so that the home knows not to advertise             to that person.

\item[{Parameters}] \leavevmode
\sphinxstyleliteralstrong{\sphinxupquote{do\_interruption}} (\sphinxstyleliteralemphasis{\sphinxupquote{bool}}) \textendash{} this flag indicates whether or not advertisements should be made             for activities that will interrupt the current activity (if True). If False, the advertisements             are made for non-interrupting activities.

\item[{Returns}] \leavevmode
None

\end{description}\end{quote}

\end{fulllineitems}

\index{advertise() (universe.Universe method)}

\begin{fulllineitems}
\phantomsection\label{\detokenize{universe:universe.Universe.advertise}}\pysiglinewithargsret{\sphinxbfcode{\sphinxupquote{advertise}}}{\emph{do\_interruption=False}}{}
This function obtains a list of all of the possible activities each person could potentially start in         this time step.
\begin{quote}\begin{description}
\item[{Parameters}] \leavevmode
\sphinxstyleliteralstrong{\sphinxupquote{do\_interruption}} (\sphinxstyleliteralemphasis{\sphinxupquote{bool}}) \textendash{} this flag indicates whether to make advertisements due to an         interrupting activity (if True) or not (if False).

\item[{Return ads}] \leavevmode
ads is a list of dictionaries for advertisements:

Dictionary  (score, asset, activity, person) containing the various data for
each advertisement: (score, asset, activity, person) coupling where the data types                     are (float, {\hyperref[\detokenize{asset:asset.Asset}]{\sphinxcrossref{\sphinxcode{\sphinxupquote{asset.Asset}}}}}, {\hyperref[\detokenize{activity:activity.Activity}]{\sphinxcrossref{\sphinxcode{\sphinxupquote{activity.Activity}}}}}, {\hyperref[\detokenize{person:person.Person}]{\sphinxcrossref{\sphinxcode{\sphinxupquote{person.Person}}}}})

\item[{Return type}] \leavevmode
list

\end{description}\end{quote}

\end{fulllineitems}

\index{check\_expired\_activities() (universe.Universe method)}

\begin{fulllineitems}
\phantomsection\label{\detokenize{universe:universe.Universe.check_expired_activities}}\pysiglinewithargsret{\sphinxbfcode{\sphinxupquote{check\_expired\_activities}}}{}{}
This function checks for expired activities. If found, end the activities.
\begin{quote}\begin{description}
\item[{Returns}] \leavevmode
None

\end{description}\end{quote}

\end{fulllineitems}

\index{decay\_needs() (universe.Universe method)}

\begin{fulllineitems}
\phantomsection\label{\detokenize{universe:universe.Universe.decay_needs}}\pysiglinewithargsret{\sphinxbfcode{\sphinxupquote{decay\_needs}}}{\emph{dt=None}}{}
This function decays the needs according to the default behavior. That is, assume the needs are not         addressed earlier.
\begin{quote}\begin{description}
\item[{Parameters}] \leavevmode
\sphinxstyleliteralstrong{\sphinxupquote{dt}} (\sphinxstyleliteralemphasis{\sphinxupquote{int}}) \textendash{} the number of minutes to decay the needs by. The default behavior is to use the scheduler’s         time. If a number is specified, then it should be the number of minutes until the end of the simulation.

\item[{Returns}] \leavevmode
None

\end{description}\end{quote}

\end{fulllineitems}

\index{initial\_step() (universe.Universe method)}

\begin{fulllineitems}
\phantomsection\label{\detokenize{universe:universe.Universe.initial_step}}\pysiglinewithargsret{\sphinxbfcode{\sphinxupquote{initial\_step}}}{}{}
This function is supposed to run the first time step of the run() loop
\begin{enumerate}
\item {} 
store the current time

\item {} 
address the needs assuming interruption

\item {} 
address the needs assuming NO interruption

\item {} 
update the history

\item {} 
update the clock

\item {} 
decay the needs

\end{enumerate}

\begin{sphinxadmonition}{note}{Note:}
this function is \sphinxstylestrong{NOT} called on in the current implementation yet
\end{sphinxadmonition}
\begin{quote}\begin{description}
\item[{Returns}] \leavevmode
None

\end{description}\end{quote}

\end{fulllineitems}

\index{initialize\_needs() (universe.Universe method)}

\begin{fulllineitems}
\phantomsection\label{\detokenize{universe:universe.Universe.initialize_needs}}\pysiglinewithargsret{\sphinxbfcode{\sphinxupquote{initialize\_needs}}}{}{}
This function initializes the need state of each person at the beginning of simulation based on         the current time.

The needs are initialized in this order (the order matters)
\begin{enumerate}
\item {} 
Rest

\item {} 
Hunger

\item {} 
Income

\item {} 
Travel

\item {} 
Interruption

\end{enumerate}
\begin{quote}\begin{description}
\item[{Returns}] \leavevmode
None

\end{description}\end{quote}

\end{fulllineitems}

\index{print\_activity\_info() (universe.Universe method)}

\begin{fulllineitems}
\phantomsection\label{\detokenize{universe:universe.Universe.print_activity_info}}\pysiglinewithargsret{\sphinxbfcode{\sphinxupquote{print\_activity\_info}}}{\emph{p}}{}
This function stores activity info used for testing / developing/ debugging as a string.
\begin{quote}\begin{description}
\item[{Parameters}] \leavevmode
\sphinxstyleliteralstrong{\sphinxupquote{p}} ({\hyperref[\detokenize{person:person.Person}]{\sphinxcrossref{\sphinxstyleliteralemphasis{\sphinxupquote{person.Person}}}}}) \textendash{} the person of interest

\item[{Returns}] \leavevmode
None

\end{description}\end{quote}

\end{fulllineitems}

\index{reset() (universe.Universe method)}

\begin{fulllineitems}
\phantomsection\label{\detokenize{universe:universe.Universe.reset}}\pysiglinewithargsret{\sphinxbfcode{\sphinxupquote{reset}}}{\emph{t\_univ}}{}
This code resets the simulation by initializing the agents, home, and clock to the beginning status         of the simulation.

This code does the following:
\begin{enumerate}
\item {} 
reset the clock

\item {} 
reset the home

\item {} 
reset each person

\item {} 
initialize each person

\item {} 
initialize the home

\end{enumerate}
\begin{quote}\begin{description}
\item[{Parameters}] \leavevmode\begin{itemize}
\item {} 
\sphinxstyleliteralstrong{\sphinxupquote{p}} ({\hyperref[\detokenize{params:params.Params}]{\sphinxcrossref{\sphinxstyleliteralemphasis{\sphinxupquote{params.Params}}}}}) \textendash{} the parameters

\item {} 
\sphinxstyleliteralstrong{\sphinxupquote{t\_univ}} (\sphinxstyleliteralemphasis{\sphinxupquote{int}}) \textendash{} the time of the beginning of the simulation {[}seconds{]}

\end{itemize}

\item[{Returns}] \leavevmode


\end{description}\end{quote}

\end{fulllineitems}

\index{run() (universe.Universe method)}

\begin{fulllineitems}
\phantomsection\label{\detokenize{universe:universe.Universe.run}}\pysiglinewithargsret{\sphinxbfcode{\sphinxupquote{run}}}{}{}
This function is responsible for running the simulation. Instead of running the simulation minute-by-minute,         in an effort to reduce run-time, the simulation skips time steps and addresses the agent at times that         actions should occur. These times are dictated by the scheduler.

The function proceeds as following:

While the current time is less than the final time:
\begin{enumerate}
\item {} 
check for expired activities for all agents. If activities should have expired, tell the agent to end them

\item {} 
start new activities by addressing the needs for all agents (assuming no interruption)

\item {} 
decay the satiation for Interruption for all agents

\item {} 
start new activities by addressing the needs for all agents (assuming interruptions only)

\item {} 
update the history of the status of each agent

\item {} 
find the next time to jump to in the simulation according to the scheduler

\item {} 
update the clock to the new time

\item {} 
decay the needs for all agents

\item {} 
Repeat

\end{enumerate}

For the last time step:
\begin{enumerate}
\item {} 
update the clock

\item {} 
decay the needs for each agent

\item {} 
update the history of the status of each agent

\end{enumerate}
\begin{quote}\begin{description}
\item[{Returns}] \leavevmode


\end{description}\end{quote}

\end{fulllineitems}

\index{select\_activity() (universe.Universe method)}

\begin{fulllineitems}
\phantomsection\label{\detokenize{universe:universe.Universe.select_activity}}\pysiglinewithargsret{\sphinxbfcode{\sphinxupquote{select\_activity}}}{\emph{ads}}{}
Given a list of activity advertisements, this function selects the person
with the largest activity score and outputs the score, asset, activity, and person.
\begin{quote}\begin{description}
\item[{Parameters}] \leavevmode
\sphinxstyleliteralstrong{\sphinxupquote{ads}} (\sphinxstyleliteralemphasis{\sphinxupquote{list}}) \textendash{} a list of advertisements for this time step

\item[{Return chosen}] \leavevmode
the selected activity advertisement (score, asset, activity, person)

\item[{Return type}] \leavevmode
dict

\end{description}\end{quote}

\end{fulllineitems}

\index{set\_alarm() (universe.Universe method)}

\begin{fulllineitems}
\phantomsection\label{\detokenize{universe:universe.Universe.set_alarm}}\pysiglinewithargsret{\sphinxbfcode{\sphinxupquote{set\_alarm}}}{}{}
This function sets the alarm for those Person(s) who use an alarm

\begin{sphinxadmonition}{note}{Note:}
This function is \sphinxstylestrong{NOT} used. There is currently no alarm capability.
\end{sphinxadmonition}
\begin{quote}\begin{description}
\item[{Returns}] \leavevmode
None

\end{description}\end{quote}

\end{fulllineitems}

\index{test\_func() (universe.Universe method)}

\begin{fulllineitems}
\phantomsection\label{\detokenize{universe:universe.Universe.test_func}}\pysiglinewithargsret{\sphinxbfcode{\sphinxupquote{test\_func}}}{}{}~
\begin{sphinxadmonition}{note}{Note:}
This function is just for debugging. This has \sphinxstylestrong{no} use in the current version. This function             will be removed in future versions.
\end{sphinxadmonition}
\begin{quote}\begin{description}
\item[{Returns}] \leavevmode


\end{description}\end{quote}

\end{fulllineitems}

\index{toString() (universe.Universe method)}

\begin{fulllineitems}
\phantomsection\label{\detokenize{universe:universe.Universe.toString}}\pysiglinewithargsret{\sphinxbfcode{\sphinxupquote{toString}}}{}{}
Represent the Universe object as a string.

This function outputs the representation of:
\begin{enumerate}
\item {} 
the clock

\item {} 
the home

\item {} 
agent person residing in the home

\end{enumerate}
\begin{quote}\begin{description}
\item[{Return msg}] \leavevmode
a representation of the Universe object

\item[{Return type}] \leavevmode
str

\end{description}\end{quote}

\end{fulllineitems}

\index{update\_clock() (universe.Universe method)}

\begin{fulllineitems}
\phantomsection\label{\detokenize{universe:universe.Universe.update_clock}}\pysiglinewithargsret{\sphinxbfcode{\sphinxupquote{update\_clock}}}{\emph{t}}{}
This function updates the clock by
\begin{enumerate}
\item {} 
setting the clock to the given time

\item {} 
updating the step of the simulation

\item {} 
storing the history of the time nodes used in the simulation

\end{enumerate}
\begin{quote}\begin{description}
\item[{Parameters}] \leavevmode
\sphinxstyleliteralstrong{\sphinxupquote{t}} (\sphinxstyleliteralemphasis{\sphinxupquote{int}}) \textendash{} the time the clock should be set to

\item[{Returns}] \leavevmode


\end{description}\end{quote}

\end{fulllineitems}

\index{update\_history() (universe.Universe method)}

\begin{fulllineitems}
\phantomsection\label{\detokenize{universe:universe.Universe.update_history}}\pysiglinewithargsret{\sphinxbfcode{\sphinxupquote{update\_history}}}{\emph{step}}{}
Update the histories for each Person by storing the following:
\begin{enumerate}
\item {} 
the current state’s status

\item {} 
the current activity

\item {} 
the current satiation value for each needs

\item {} 
the current location

\end{enumerate}
\begin{quote}\begin{description}
\item[{Parameters}] \leavevmode
\sphinxstyleliteralstrong{\sphinxupquote{step}} (\sphinxstyleliteralemphasis{\sphinxupquote{int}}) \textendash{} the time step

\item[{Returns}] \leavevmode
None

\end{description}\end{quote}

\end{fulllineitems}

\index{update\_history\_new() (universe.Universe method)}

\begin{fulllineitems}
\phantomsection\label{\detokenize{universe:universe.Universe.update_history_new}}\pysiglinewithargsret{\sphinxbfcode{\sphinxupquote{update\_history\_new}}}{}{}
Update the histories of each person.
\begin{quote}\begin{description}
\item[{Returns}] \leavevmode
None

\end{description}\end{quote}

\end{fulllineitems}


\end{fulllineitems}



\subsection{work module}
\label{\detokenize{work::doc}}\label{\detokenize{work:module-work}}\label{\detokenize{work:work-module}}\index{work (module)}
This module contains code that governs the activity that gives a person the ability to go to work/ school.

This file contains {\hyperref[\detokenize{work:work.Work}]{\sphinxcrossref{\sphinxcode{\sphinxupquote{work.Work}}}}}.
\index{Work (class in work)}

\begin{fulllineitems}
\phantomsection\label{\detokenize{work:work.Work}}\pysigline{\sphinxbfcode{\sphinxupquote{class }}\sphinxcode{\sphinxupquote{work.}}\sphinxbfcode{\sphinxupquote{Work}}}
Bases: {\hyperref[\detokenize{activity:activity.Activity}]{\sphinxcrossref{\sphinxcode{\sphinxupquote{activity.Activity}}}}}

This class allows a person to work / go to school in order to satisfy the need     {\hyperref[\detokenize{income:income.Income}]{\sphinxcrossref{\sphinxcode{\sphinxupquote{income.Income}}}}}.
\index{advertise() (work.Work method)}

\begin{fulllineitems}
\phantomsection\label{\detokenize{work:work.Work.advertise}}\pysiglinewithargsret{\sphinxbfcode{\sphinxupquote{advertise}}}{\emph{p}}{}
This function calculates the score of the advertised work activity to a person
\begin{quote}\begin{description}
\item[{Parameters}] \leavevmode
\sphinxstyleliteralstrong{\sphinxupquote{p}} ({\hyperref[\detokenize{person:person.Person}]{\sphinxcrossref{\sphinxstyleliteralemphasis{\sphinxupquote{person.Person}}}}}) \textendash{} the person of interest

\item[{Return score}] \leavevmode
\item[{Return type}] \leavevmode
float

\end{description}\end{quote}

\end{fulllineitems}

\index{end() (work.Work method)}

\begin{fulllineitems}
\phantomsection\label{\detokenize{work:work.Work.end}}\pysiglinewithargsret{\sphinxbfcode{\sphinxupquote{end}}}{\emph{p}}{}
This function handles the end of an activity
\begin{quote}\begin{description}
\item[{Parameters}] \leavevmode
\sphinxstyleliteralstrong{\sphinxupquote{p}} ({\hyperref[\detokenize{person:person.Person}]{\sphinxcrossref{\sphinxstyleliteralemphasis{\sphinxupquote{person.Person}}}}}) \textendash{} the person of interest

\item[{Returns}] \leavevmode
None

\end{description}\end{quote}

\end{fulllineitems}

\index{end\_work() (work.Work method)}

\begin{fulllineitems}
\phantomsection\label{\detokenize{work:work.Work.end_work}}\pysiglinewithargsret{\sphinxbfcode{\sphinxupquote{end\_work}}}{\emph{p}}{}
This function sets the variables pertaining to coming back from work by doing the following:
\begin{enumerate}
\item {} 
free the asset from use

\item {} 
set the asset’s state to \sphinxcode{\sphinxupquote{state.IDLE}}

\item {} 
set the Income satiation to 1

\item {} 
decay the need Travel

\item {} 
sample the new work start time

\item {} 
sample the new work end time

\item {} 
update the scheduler to take into account the next work event

\end{enumerate}
\begin{quote}\begin{description}
\item[{Parameters}] \leavevmode
\sphinxstyleliteralstrong{\sphinxupquote{p}} ({\hyperref[\detokenize{person:person.Person}]{\sphinxcrossref{\sphinxstyleliteralemphasis{\sphinxupquote{person.Person}}}}}) \textendash{} the person of interest

\item[{Returns}] \leavevmode
None

\end{description}\end{quote}

\end{fulllineitems}

\index{halt() (work.Work method)}

\begin{fulllineitems}
\phantomsection\label{\detokenize{work:work.Work.halt}}\pysiglinewithargsret{\sphinxbfcode{\sphinxupquote{halt}}}{\emph{p}}{}
This function handles an interruption of an Activity.
\begin{quote}\begin{description}
\item[{Parameters}] \leavevmode
\sphinxstyleliteralstrong{\sphinxupquote{p}} ({\hyperref[\detokenize{person:person.Person}]{\sphinxcrossref{\sphinxstyleliteralemphasis{\sphinxupquote{person.Person}}}}}) \textendash{} the person of interest

\item[{Returns}] \leavevmode
None

\end{description}\end{quote}

\end{fulllineitems}

\index{halt\_work() (work.Work method)}

\begin{fulllineitems}
\phantomsection\label{\detokenize{work:work.Work.halt_work}}\pysiglinewithargsret{\sphinxbfcode{\sphinxupquote{halt\_work}}}{\emph{p}}{}
This function interrupts the work behavior by doing the following:
\begin{enumerate}
\item {} 
frees the current asset

\item {} 
the asset’s state is set to \sphinxcode{\sphinxupquote{state.IDLE}}

\item {} 
the Interruption satiation is set to 1.0

\item {} 
the Interruption’s activity start/ stop

\end{enumerate}
\begin{quote}\begin{description}
\item[{Note}] \leavevmode
No benefits of working are given while being interrupted

\item[{Parameters}] \leavevmode
\sphinxstyleliteralstrong{\sphinxupquote{p}} ({\hyperref[\detokenize{person:person.Person}]{\sphinxcrossref{\sphinxstyleliteralemphasis{\sphinxupquote{person.Person}}}}}) \textendash{} the person of interest

\item[{Returns}] \leavevmode
None

\end{description}\end{quote}

\end{fulllineitems}

\index{set\_end\_time() (work.Work method)}

\begin{fulllineitems}
\phantomsection\label{\detokenize{work:work.Work.set_end_time}}\pysiglinewithargsret{\sphinxbfcode{\sphinxupquote{set\_end\_time}}}{\emph{p}}{}
Calculates the end time of work.
\begin{quote}\begin{description}
\item[{Parameters}] \leavevmode
\sphinxstyleliteralstrong{\sphinxupquote{p}} ({\hyperref[\detokenize{person:person.Person}]{\sphinxcrossref{\sphinxstyleliteralemphasis{\sphinxupquote{person.Person}}}}}) \textendash{} the person of interest

\item[{Return t\_end}] \leavevmode
the end time {[}minutes, universal time{]}

\item[{Return type}] \leavevmode
int

\end{description}\end{quote}

\end{fulllineitems}

\index{start() (work.Work method)}

\begin{fulllineitems}
\phantomsection\label{\detokenize{work:work.Work.start}}\pysiglinewithargsret{\sphinxbfcode{\sphinxupquote{start}}}{\emph{p}}{}
This handles the start of an Activity
\begin{quote}\begin{description}
\item[{Parameters}] \leavevmode
\sphinxstyleliteralstrong{\sphinxupquote{p}} ({\hyperref[\detokenize{person:person.Person}]{\sphinxcrossref{\sphinxstyleliteralemphasis{\sphinxupquote{person.Person}}}}}) \textendash{} the person of interest

\item[{Returns}] \leavevmode
None

\end{description}\end{quote}

\end{fulllineitems}

\index{start\_work() (work.Work method)}

\begin{fulllineitems}
\phantomsection\label{\detokenize{work:work.Work.start_work}}\pysiglinewithargsret{\sphinxbfcode{\sphinxupquote{start\_work}}}{\emph{p}}{}
This function starts the work activity
\begin{itemize}
\item {} 
updates that asset’s status and number of users

\item {} 
changes the location of the Person

\item {} 
updates that person’s status

\item {} 
calculates the end time of the work activity

\item {} 
update the scheduler for the Income satiation

\item {} 
update the scheduler for the Travel satiation

\item {} 
set the day for the work period

\end{itemize}
\begin{quote}\begin{description}
\item[{Parameters}] \leavevmode
\sphinxstyleliteralstrong{\sphinxupquote{p}} ({\hyperref[\detokenize{person:person.Person}]{\sphinxcrossref{\sphinxstyleliteralemphasis{\sphinxupquote{person.Person}}}}}) \textendash{} the person of interest

\item[{Returns}] \leavevmode
None

\end{description}\end{quote}

\end{fulllineitems}

\index{test\_func() (work.Work method)}

\begin{fulllineitems}
\phantomsection\label{\detokenize{work:work.Work.test_func}}\pysiglinewithargsret{\sphinxbfcode{\sphinxupquote{test\_func}}}{\emph{p}}{}~
\begin{sphinxadmonition}{note}{Note:}
This function is \sphinxstylestrong{NOT} used.
\end{sphinxadmonition}
\begin{quote}\begin{description}
\item[{Parameters}] \leavevmode
\sphinxstyleliteralstrong{\sphinxupquote{p}} ({\hyperref[\detokenize{person:person.Person}]{\sphinxcrossref{\sphinxstyleliteralemphasis{\sphinxupquote{person.Person}}}}}) \textendash{} the person of interest

\item[{Returns}] \leavevmode


\end{description}\end{quote}

\end{fulllineitems}


\end{fulllineitems}



\subsection{workplace module}
\label{\detokenize{workplace::doc}}\label{\detokenize{workplace:workplace-module}}\label{\detokenize{workplace:module-workplace}}\index{workplace (module)}
This module contains code for the asset that allows a person to go to work / school.

This file contains {\hyperref[\detokenize{workplace:workplace.Workplace}]{\sphinxcrossref{\sphinxcode{\sphinxupquote{workplace.Workplace}}}}}.
\index{Workplace (class in workplace)}

\begin{fulllineitems}
\phantomsection\label{\detokenize{workplace:workplace.Workplace}}\pysigline{\sphinxbfcode{\sphinxupquote{class }}\sphinxcode{\sphinxupquote{workplace.}}\sphinxbfcode{\sphinxupquote{Workplace}}}
Bases: {\hyperref[\detokenize{asset:asset.Asset}]{\sphinxcrossref{\sphinxcode{\sphinxupquote{asset.Asset}}}}}

This class allows a Person to go to work / school.

Activities in this asset: {\hyperref[\detokenize{work:work.Work}]{\sphinxcrossref{\sphinxcode{\sphinxupquote{work.Work}}}}}

\end{fulllineitems}



\section{Run Directory}
\label{\detokenize{index:run-directory}}
These are the files needed to run an instance of ABMHAP with one agent parametrized by user-defined parameters.

The driver for these type of runs is main.py.

Contents:


\subsection{main module}
\label{\detokenize{main::doc}}\label{\detokenize{main:module-main}}\label{\detokenize{main:main-module}}\index{main (module)}
This is code is runs the simulation for the Agent-Based Model of Human Activity Patterns (ABMHAP) module of the Life Cycle Human Exposure Model (LC-HEM) project.

In order to run the code, do the following:
\begin{enumerate}
\item {} 
set the user-defined parameters of the simulation in \sphinxcode{\sphinxupquote{main\_params.py}}

\item {} 
run the code via
\begin{quote}

\textgreater{} \sphinxcode{\sphinxupquote{python main.py}}
\end{quote}

\end{enumerate}
\index{plot() (in module main)}

\begin{fulllineitems}
\phantomsection\label{\detokenize{main:main.plot}}\pysiglinewithargsret{\sphinxcode{\sphinxupquote{main.}}\sphinxbfcode{\sphinxupquote{plot}}}{\emph{p}, \emph{d=None}}{}
This function plots figures related to the results of the simulation. Specifically,     it does the following for the given agent:
\begin{enumerate}
\item {} 
plots the histograms about the activity data

\item {} 
plots cumulative distribution functions (CDFs) of the activity data

\item {} 
plots how the satiation changes over time for the all of the needs

\item {} 
plots how the weight function values change over time for all of the needs

\end{enumerate}

\begin{sphinxadmonition}{note}{Note:}
The satiation and weight function plots will \sphinxstylestrong{not} be correct unless the simulation         was set to run minute by minute. That is, main\_params.do\_minute\_by\_minute is set to \sphinxstylestrong{True}.
\end{sphinxadmonition}
\begin{quote}\begin{description}
\item[{Parameters}] \leavevmode\begin{itemize}
\item {} 
\sphinxstyleliteralstrong{\sphinxupquote{p}} ({\hyperref[\detokenize{person:person.Person}]{\sphinxcrossref{\sphinxstyleliteralemphasis{\sphinxupquote{person.Person}}}}}) \textendash{} the agent whose information is going to be plotted

\item {} 
\sphinxstyleliteralstrong{\sphinxupquote{d}} ({\hyperref[\detokenize{diary:diary.Diary}]{\sphinxcrossref{\sphinxstyleliteralemphasis{\sphinxupquote{diary.Diary}}}}}) \textendash{} the activity diary of the respected agent

\end{itemize}

\item[{Returns}] \leavevmode


\end{description}\end{quote}

\end{fulllineitems}



\subsection{main\_params module}
\label{\detokenize{main_params::doc}}\label{\detokenize{main_params:main-params-module}}\label{\detokenize{main_params:module-main_params}}\index{main\_params (module)}
This module is responsible for containing parameters that main.py uses to control the simulation. The user should set the parameters in this module \sphinxstylestrong{before} running the driver \sphinxcode{\sphinxupquote{main.py}}
\index{set\_no\_variation() (in module main\_params)}

\begin{fulllineitems}
\phantomsection\label{\detokenize{main_params:main_params.set_no_variation}}\pysiglinewithargsret{\sphinxcode{\sphinxupquote{main\_params.}}\sphinxbfcode{\sphinxupquote{set\_no\_variation}}}{\emph{num\_people}}{}
This function sets the standard deviations in all of the activity-parameters to zero.
\begin{quote}\begin{description}
\item[{Parameters}] \leavevmode
\sphinxstyleliteralstrong{\sphinxupquote{num\_people}} (\sphinxstyleliteralemphasis{\sphinxupquote{int}}) \textendash{} the number of people in the simulation

\item[{Returns}] \leavevmode
a tuple of the standard deviations of all of the activity-parameters

\end{description}\end{quote}

\end{fulllineitems}



\subsection{scenario module}
\label{\detokenize{scenario::doc}}\label{\detokenize{scenario:module-scenario}}\label{\detokenize{scenario:scenario-module}}\index{scenario (module)}
This file contains information to run the Agent-Based Model of Human Activity Patterns (ABMHAP) in in different simulation scenarios in which the agent has a user-defined parametrization.

The following classes are in this module
\begin{enumerate}
\item {} 
{\hyperref[\detokenize{scenario:scenario.Scenario}]{\sphinxcrossref{\sphinxcode{\sphinxupquote{scenario.Scenario}}}}}

\item {} 
{\hyperref[\detokenize{scenario:scenario.Solo}]{\sphinxcrossref{\sphinxcode{\sphinxupquote{scenario.Solo}}}}}

\item {} 
{\hyperref[\detokenize{scenario:scenario.Duo}]{\sphinxcrossref{\sphinxcode{\sphinxupquote{scenario.Duo}}}}}

\end{enumerate}
\index{Duo (class in scenario)}

\begin{fulllineitems}
\phantomsection\label{\detokenize{scenario:scenario.Duo}}\pysiglinewithargsret{\sphinxbfcode{\sphinxupquote{class }}\sphinxcode{\sphinxupquote{scenario.}}\sphinxbfcode{\sphinxupquote{Duo}}}{\emph{hhld\_params}}{}
Bases: {\hyperref[\detokenize{scenario:scenario.Scenario}]{\sphinxcrossref{\sphinxcode{\sphinxupquote{scenario.Scenario}}}}}

This class parametrizes / runs a simulation scenario for the cases where two Singleton     ({\hyperref[\detokenize{singleton:singleton.Singleton}]{\sphinxcrossref{\sphinxcode{\sphinxupquote{singleton.Singleton}}}}}) persons live in the same residence.

\begin{sphinxadmonition}{note}{Note:}
This scenario is used in order to check for activity conflicts among 2 agents living in         the same household. Currently it is used primarily as a debugging tool.
\end{sphinxadmonition}
\begin{quote}\begin{description}
\item[{Parameters}] \leavevmode
\sphinxstyleliteralstrong{\sphinxupquote{hhld\_params}} ({\hyperref[\detokenize{params:params.Params}]{\sphinxcrossref{\sphinxstyleliteralemphasis{\sphinxupquote{params.Params}}}}}) \textendash{} the parameters for the household that contain relevant information     for the simulation

\end{description}\end{quote}

\end{fulllineitems}

\index{Scenario (class in scenario)}

\begin{fulllineitems}
\phantomsection\label{\detokenize{scenario:scenario.Scenario}}\pysiglinewithargsret{\sphinxbfcode{\sphinxupquote{class }}\sphinxcode{\sphinxupquote{scenario.}}\sphinxbfcode{\sphinxupquote{Scenario}}}{\emph{hhld\_params}}{}
Bases: \sphinxcode{\sphinxupquote{object}}

This class governs what a simulation scenario consists of.
\begin{quote}\begin{description}
\item[{Parameters}] \leavevmode
\sphinxstyleliteralstrong{\sphinxupquote{hhld\_params}} ({\hyperref[\detokenize{params:params.Params}]{\sphinxcrossref{\sphinxstyleliteralemphasis{\sphinxupquote{params.Params}}}}}) \textendash{} the parameters for the household that contain relevant information     for the simulation

\item[{Variables}] \leavevmode\begin{itemize}
\item {} 
\sphinxstyleliteralstrong{\sphinxupquote{id}} (\sphinxstyleliteralemphasis{\sphinxupquote{int}}) \textendash{} the scenario identifier number

\item {} 
\sphinxstyleliteralstrong{\sphinxupquote{u}} ({\hyperref[\detokenize{universe:universe.Universe}]{\sphinxcrossref{\sphinxstyleliteralemphasis{\sphinxupquote{universe.Universe}}}}}) \textendash{} the universe object for the simulation

\item {} 
\sphinxstyleliteralstrong{\sphinxupquote{'params'}} ({\hyperref[\detokenize{params:params.Params}]{\sphinxcrossref{\sphinxstyleliteralemphasis{\sphinxupquote{params.Params}}}}}) \textendash{} the parameters needed that control the simulation

\end{itemize}

\end{description}\end{quote}
\index{activity\_diary() (scenario.Scenario method)}

\begin{fulllineitems}
\phantomsection\label{\detokenize{scenario:scenario.Scenario.activity_diary}}\pysiglinewithargsret{\sphinxbfcode{\sphinxupquote{activity\_diary}}}{}{}
This function returns the activity diary for each person

Each person will attain the following tuple
\begin{enumerate}
\item {} 
grouping of the index for each activity

\item {} 
the day, (start-time, end-time), activity code, and location for each activity-event, in a numeric format

\item {} 
the same as above in a string format

\end{enumerate}
\begin{quote}\begin{description}
\item[{Returns}] \leavevmode


\end{description}\end{quote}

\end{fulllineitems}

\index{default\_location() (scenario.Scenario method)}

\begin{fulllineitems}
\phantomsection\label{\detokenize{scenario:scenario.Scenario.default_location}}\pysiglinewithargsret{\sphinxbfcode{\sphinxupquote{default\_location}}}{}{}
Sets the default location for all Person’s to be be at the home. This location may         be overridden later in the initialization of persons.
\begin{quote}\begin{description}
\item[{Returns}] \leavevmode
None

\end{description}\end{quote}

\end{fulllineitems}

\index{initialize() (scenario.Scenario method)}

\begin{fulllineitems}
\phantomsection\label{\detokenize{scenario:scenario.Scenario.initialize}}\pysiglinewithargsret{\sphinxbfcode{\sphinxupquote{initialize}}}{}{}
This function initializes the scenario before the simulation scenario is run

More specifically, the function does the following:
\begin{enumerate}
\item {} 
Sets the state and location for each person

\item {} 
Sets the home

\item {} 
Initialize the initial need-association states for the Person(s) and Home

\end{enumerate}
\begin{quote}\begin{description}
\item[{Returns}] \leavevmode
None

\end{description}\end{quote}

\end{fulllineitems}

\index{run() (scenario.Scenario method)}

\begin{fulllineitems}
\phantomsection\label{\detokenize{scenario:scenario.Scenario.run}}\pysiglinewithargsret{\sphinxbfcode{\sphinxupquote{run}}}{}{}
This function initializes the scenario and then runs the ABMHAP simulation.
\begin{quote}\begin{description}
\item[{Returns}] \leavevmode
None

\end{description}\end{quote}

\end{fulllineitems}

\index{set\_home() (scenario.Scenario method)}

\begin{fulllineitems}
\phantomsection\label{\detokenize{scenario:scenario.Scenario.set_home}}\pysiglinewithargsret{\sphinxbfcode{\sphinxupquote{set\_home}}}{}{}
This function sets aspects of the home in order to run the simulation scenario.

More specifically, the function does the following
\begin{enumerate}
\item {} 
set the home revenue

\item {} 
set the home population

\end{enumerate}
\begin{quote}\begin{description}
\item[{Returns}] \leavevmode
None

\end{description}\end{quote}

\end{fulllineitems}

\index{set\_state() (scenario.Scenario method)}

\begin{fulllineitems}
\phantomsection\label{\detokenize{scenario:scenario.Scenario.set_state}}\pysiglinewithargsret{\sphinxbfcode{\sphinxupquote{set\_state}}}{}{}
This function initializes the scenario in order to run the simulation. More         specifically, this function does the following:
\begin{enumerate}
\item {} 
For each Person, the following is set:
\begin{enumerate}
\item {} 
identification number

\item {} 
the state

\end{enumerate}

\end{enumerate}
\begin{quote}\begin{description}
\item[{Returns}] \leavevmode
None

\end{description}\end{quote}

\end{fulllineitems}


\end{fulllineitems}

\index{Solo (class in scenario)}

\begin{fulllineitems}
\phantomsection\label{\detokenize{scenario:scenario.Solo}}\pysiglinewithargsret{\sphinxbfcode{\sphinxupquote{class }}\sphinxcode{\sphinxupquote{scenario.}}\sphinxbfcode{\sphinxupquote{Solo}}}{\emph{hhld\_params}}{}
Bases: {\hyperref[\detokenize{scenario:scenario.Scenario}]{\sphinxcrossref{\sphinxcode{\sphinxupquote{scenario.Scenario}}}}}

This class parametrizes / runs a simulation scenario for the Singleton ({\hyperref[\detokenize{singleton:singleton.Singleton}]{\sphinxcrossref{\sphinxcode{\sphinxupquote{singleton.Singleton}}}}}) person.
\begin{quote}\begin{description}
\item[{Parameters}] \leavevmode
\sphinxstyleliteralstrong{\sphinxupquote{hhld\_params}} ({\hyperref[\detokenize{params:params.Params}]{\sphinxcrossref{\sphinxstyleliteralemphasis{\sphinxupquote{params.Params}}}}}) \textendash{} the parameters for the household that contain relevant information     for the simulation

\end{description}\end{quote}

\end{fulllineitems}



\subsection{singleton module}
\label{\detokenize{singleton::doc}}\label{\detokenize{singleton:singleton-module}}\label{\detokenize{singleton:module-singleton}}\index{singleton (module)}
This file contains information for creating the default agent that represents a person that lives alone in the home. Singleton will be the name of this type of agent.

This module contains {\hyperref[\detokenize{singleton:singleton.Singleton}]{\sphinxcrossref{\sphinxcode{\sphinxupquote{singleton.Singleton}}}}}.
\index{Singleton (class in singleton)}

\begin{fulllineitems}
\phantomsection\label{\detokenize{singleton:singleton.Singleton}}\pysiglinewithargsret{\sphinxbfcode{\sphinxupquote{class }}\sphinxcode{\sphinxupquote{singleton.}}\sphinxbfcode{\sphinxupquote{Singleton}}}{\emph{house}, \emph{clock}, \emph{schedule}}{}
Bases: {\hyperref[\detokenize{person:person.Person}]{\sphinxcrossref{\sphinxcode{\sphinxupquote{person.Person}}}}}

Singleton default is a person that has the following characteristics
\begin{enumerate}
\item {} 
female

\item {} 
30 years old

\item {} 
goes to bed at 22:00 and sleeps for 8 hours

\item {} 
lives alone and has no children

\item {} 
works the Standard Job

\item {} 
eats breakfast at 7:30 for 15 minutes, lunch at 12:00 for 30 minutes,     and dinner at 19:00 for 45 minutes

\end{enumerate}
\begin{quote}\begin{description}
\item[{Parameters}] \leavevmode\begin{itemize}
\item {} 
\sphinxstyleliteralstrong{\sphinxupquote{house}} ({\hyperref[\detokenize{home:home.Home}]{\sphinxcrossref{\sphinxstyleliteralemphasis{\sphinxupquote{home.Home}}}}}) \textendash{} the place of residence

\item {} 
\sphinxstyleliteralstrong{\sphinxupquote{clock}} ({\hyperref[\detokenize{temporal:temporal.Temporal}]{\sphinxcrossref{\sphinxstyleliteralemphasis{\sphinxupquote{temporal.Temporal}}}}}) \textendash{} the clock running in the simulation

\item {} 
\sphinxstyleliteralstrong{\sphinxupquote{schedule}} ({\hyperref[\detokenize{scheduler:scheduler.Scheduler}]{\sphinxcrossref{\sphinxstyleliteralemphasis{\sphinxupquote{scheduler.Scheduler}}}}}) \textendash{} the schedule for the agent

\end{itemize}

\end{description}\end{quote}
\index{print\_params() (singleton.Singleton method)}

\begin{fulllineitems}
\phantomsection\label{\detokenize{singleton:singleton.Singleton.print_params}}\pysiglinewithargsret{\sphinxbfcode{\sphinxupquote{print\_params}}}{}{}
This function prints the activity-parameter means in chronological order of start time. This         results in the ability to print the mean daily routine.
\begin{quote}\begin{description}
\item[{Returns}] \leavevmode
a representation of the parameters of the agent in increasing values of         start time

\item[{Return type}] \leavevmode
str

\end{description}\end{quote}

\end{fulllineitems}

\index{set() (singleton.Singleton method)}

\begin{fulllineitems}
\phantomsection\label{\detokenize{singleton:singleton.Singleton.set}}\pysiglinewithargsret{\sphinxbfcode{\sphinxupquote{set}}}{\emph{param}, \emph{idx}}{}
This function sets the Singleton’s parameters.

The function does the following:
\begin{enumerate}
\item {} 
sets the biology

\item {} 
sets the job information

\item {} 
sets the alarm

\item {} 
sets the meal information

\end{enumerate}
\begin{quote}\begin{description}
\item[{Parameters}] \leavevmode\begin{itemize}
\item {} 
\sphinxstyleliteralstrong{\sphinxupquote{param}} ({\hyperref[\detokenize{params:params.Params}]{\sphinxcrossref{\sphinxstyleliteralemphasis{\sphinxupquote{params.Params}}}}}) \textendash{} parameters describing the household

\item {} 
\sphinxstyleliteralstrong{\sphinxupquote{idx}} (\sphinxstyleliteralemphasis{\sphinxupquote{int}}) \textendash{} the respective index number of the person of interest in the household

\end{itemize}

\item[{Returns}] \leavevmode
None

\end{description}\end{quote}

\end{fulllineitems}


\end{fulllineitems}



\section{Run\_chad Directory}
\label{\detokenize{index:run-chad-directory}}
These are the files needed to run an instance of ABMHAP as a monte-carlo simulation consisting of
multiple households of agents parametrized with the data from the Consolidated Human Activity Database
(CHAD).

These simulations may be run in parallel.The driver for these type of runs is driver.py.

Contents:


\subsection{analysis module}
\label{\detokenize{analysis::doc}}\label{\detokenize{analysis:analysis-module}}\label{\detokenize{analysis:module-analysis}}\index{analysis (module)}
This file contains capability for analyzing results from the comparisons between CHAD (Consolidated Human Activity Database) data and the performance of ABMHAP (Agent-Based Model of Human Activity Patterns).

\begin{sphinxadmonition}{warning}{Warning:}
This modules is old and may or may not be used.
\end{sphinxadmonition}
\index{get\_error() (in module analysis)}

\begin{fulllineitems}
\phantomsection\label{\detokenize{analysis:analysis.get_error}}\pysiglinewithargsret{\sphinxcode{\sphinxupquote{analysis.}}\sphinxbfcode{\sphinxupquote{get\_error}}}{\emph{chad\_raw}, \emph{chad\_stats}, \emph{col\_name}, \emph{abm\_all}, \emph{do\_cyclical=False}}{}~
\begin{sphinxadmonition}{warning}{Warning:}
I do not think this function is used.
\end{sphinxadmonition}
\begin{quote}\begin{description}
\item[{Parameters}] \leavevmode\begin{itemize}
\item {} 
\sphinxstyleliteralstrong{\sphinxupquote{chad\_raw}} (\sphinxstyleliteralemphasis{\sphinxupquote{pandas.core.frame.DataFrame}}) \textendash{} the CHAD activity data being compared to

\item {} 
\sphinxstyleliteralstrong{\sphinxupquote{chad\_stats}} (\sphinxstyleliteralemphasis{\sphinxupquote{pandas.core.frame.DataFrame}}) \textendash{} the relevant statistics for the CHAD activity
of the person (PID) being modeled

\item {} 
\sphinxstyleliteralstrong{\sphinxupquote{col\_name}} (\sphinxstyleliteralemphasis{\sphinxupquote{str}}) \textendash{} 
the name of the column of the CHAD data being compared
\begin{quote}\begin{description}
\item[{example}] \leavevmode
col\_name = “dt” would allow access for chad\_raw{[}“dt”{]}

\end{description}\end{quote}


\item {} 
\sphinxstyleliteralstrong{\sphinxupquote{abm\_all}} \textendash{} 
the ABM simulation data for the simulated person’s activity
with respected to the quantity from col\_name.
\begin{quote}\begin{description}
\item[{example}] \leavevmode
if col\_name = “dt”, then abm\_all should contain the duration data

\end{description}\end{quote}


\item {} 
\sphinxstyleliteralstrong{\sphinxupquote{do\_cyclical}} (\sphinxstyleliteralemphasis{\sphinxupquote{bool}}) \textendash{} indicates when to cast data in a “cyclical” form. As in,
{[}0, 24 * HOURS\_2\_MIN - 1{]} {[}minutes{]}

\end{itemize}

\item[{Returns}] \leavevmode
the L2 (sum of squares) absolute error for each agent,     the L2 (sum of squares) relative error for each agent

\item[{Return type}] \leavevmode
float array, float array

\end{description}\end{quote}

\end{fulllineitems}

\index{get\_moments() (in module analysis)}

\begin{fulllineitems}
\phantomsection\label{\detokenize{analysis:analysis.get_moments}}\pysiglinewithargsret{\sphinxcode{\sphinxupquote{analysis.}}\sphinxbfcode{\sphinxupquote{get\_moments}}}{\emph{abm\_data}}{}
This function takes in all of the ABMHAP simulation data {[}in minutes{]} for a particular activity and returns     the moments (mean and standard deviation) {[}hours{]} for each person in the simulation.
\begin{quote}\begin{description}
\item[{Parameters}] \leavevmode
\sphinxstyleliteralstrong{\sphinxupquote{abm\_data}} (\sphinxstyleliteralemphasis{\sphinxupquote{list}}) \textendash{} the list of ABMHAP of activity data in minutes per person

\item[{Returns}] \leavevmode
the mean and standard deviation for each person in the simulation

\item[{Return type}] \leavevmode
numpy.ndarray, numpy.ndarray

\end{description}\end{quote}

\end{fulllineitems}

\index{get\_proper\_data() (in module analysis)}

\begin{fulllineitems}
\phantomsection\label{\detokenize{analysis:analysis.get_proper_data}}\pysiglinewithargsret{\sphinxcode{\sphinxupquote{analysis.}}\sphinxbfcode{\sphinxupquote{get\_proper\_data}}}{\emph{df\_dt}, \emph{df\_start}, \emph{df\_record}, \emph{x}}{}
This function gets the duration, start time, and record data for a given activity

\begin{sphinxadmonition}{warning}{Warning:}
This function may not be  used.
\end{sphinxadmonition}
\begin{quote}\begin{description}
\item[{Parameters}] \leavevmode\begin{itemize}
\item {} 
\sphinxstyleliteralstrong{\sphinxupquote{df\_dt}} (\sphinxstyleliteralemphasis{\sphinxupquote{pandas.core.frame.DataFrame}}) \textendash{} the duration statistical data for a given activity

\item {} 
\sphinxstyleliteralstrong{\sphinxupquote{df\_start}} (\sphinxstyleliteralemphasis{\sphinxupquote{pandas.core.frame.DataFrame}}) \textendash{} the start time statistical data for a given activity

\item {} 
\sphinxstyleliteralstrong{\sphinxupquote{df\_record}} (\sphinxstyleliteralemphasis{\sphinxupquote{pandas.core.frame.DataFrame}}) \textendash{} the CHAD records for the given activity

\item {} 
\sphinxstyleliteralstrong{\sphinxupquote{x}} ({\hyperref[\detokenize{chad_params:chad_params.CHAD_params}]{\sphinxcrossref{\sphinxstyleliteralemphasis{\sphinxupquote{chad\_params.CHAD\_params}}}}}) \textendash{} the parameters that limit sampling the CHAD data

\end{itemize}

\item[{Returns}] \leavevmode


\end{description}\end{quote}

\end{fulllineitems}

\index{get\_verification\_info() (in module analysis)}

\begin{fulllineitems}
\phantomsection\label{\detokenize{analysis:analysis.get_verification_info}}\pysiglinewithargsret{\sphinxcode{\sphinxupquote{analysis.}}\sphinxbfcode{\sphinxupquote{get\_verification\_info}}}{\emph{demo}, \emph{key\_activity}, \emph{sampling\_params}, \emph{fname\_stats=None}}{}
This function gets the CHAD parameters for each household

\begin{sphinxadmonition}{note}{Note:}
Sometimes record dataframe can be null. I should remove the sampling of the record code \sphinxstylestrong{and}         output
\end{sphinxadmonition}
\begin{quote}\begin{description}
\item[{Parameters}] \leavevmode\begin{itemize}
\item {} 
\sphinxstyleliteralstrong{\sphinxupquote{demo}} (\sphinxstyleliteralemphasis{\sphinxupquote{int}}) \textendash{} the demographic identifier

\item {} 
\sphinxstyleliteralstrong{\sphinxupquote{key\_activity}} (\sphinxstyleliteralemphasis{\sphinxupquote{int}}) \textendash{} the identifier for the activity (from my\_globals)

\item {} 
\sphinxstyleliteralstrong{\sphinxupquote{sampling\_params}} (list of {\hyperref[\detokenize{chad_params:chad_params.CHAD_params}]{\sphinxcrossref{\sphinxcode{\sphinxupquote{chad\_params.CHAD\_params}}}}}) \textendash{} the parameters that the limit the     sampling of the CHAD data

\end{itemize}

\item[{Returns}] \leavevmode
the activity moments of the start time data from CHAD used to verify the ABMHAP,     the activity moments of the end time data from CHAD used to verify the ABMHAP,     the activity moments of the duration data from CHAD used to verify the ABMHAP,     the activity records data from CHAD used to verify the ABMHAP

\item[{Return type}] \leavevmode
pandas.core.frame.DataFrame, pandas.core.frame.DataFrame,     pandas.core.frame.DataFrame, pandas.core.frame.DataFrame

\end{description}\end{quote}

\end{fulllineitems}

\index{save\_figures() (in module analysis)}

\begin{fulllineitems}
\phantomsection\label{\detokenize{analysis:analysis.save_figures}}\pysiglinewithargsret{\sphinxcode{\sphinxupquote{analysis.}}\sphinxbfcode{\sphinxupquote{save\_figures}}}{\emph{figs}, \emph{fnames}}{}
This function saves figures in a python pickle file, so that the data     may be accessed again.
\begin{quote}\begin{description}
\item[{Parameters}] \leavevmode\begin{itemize}
\item {} 
\sphinxstyleliteralstrong{\sphinxupquote{figs}} (\sphinxstyleliteralemphasis{\sphinxupquote{list of figures}}) \textendash{} figures for duration and start time for an activity

\item {} 
\sphinxstyleliteralstrong{\sphinxupquote{fnames}} (\sphinxstyleliteralemphasis{\sphinxupquote{list of str}}) \textendash{} file names to save the data in figs

\item {} 
\sphinxstyleliteralstrong{\sphinxupquote{fdir}} (\sphinxstyleliteralemphasis{\sphinxupquote{str}}) \textendash{} the directory in which to save the files

\end{itemize}

\item[{Returns}] \leavevmode


\end{description}\end{quote}

\end{fulllineitems}



\subsection{analyzer module}
\label{\detokenize{analyzer::doc}}\label{\detokenize{analyzer:analyzer-module}}\label{\detokenize{analyzer:module-analyzer}}\index{analyzer (module)}
For a given activity, this code compares how well the ABMHAP matches the CHAD data. This is useful for the quality assurance and quality control (QAQC). This code compares the distributions of mean activity-start-time and mean activity duration for the respective activity.

\begin{sphinxadmonition}{warning}{Warning:}
I will need to update this definition
\end{sphinxadmonition}
\index{get\_activity\_data() (in module analyzer)}

\begin{fulllineitems}
\phantomsection\label{\detokenize{analyzer:analyzer.get_activity_data}}\pysiglinewithargsret{\sphinxcode{\sphinxupquote{analyzer.}}\sphinxbfcode{\sphinxupquote{get\_activity\_data}}}{\emph{df}, \emph{act}}{}
This function returns the activity data from an activity diary     of given respective agent.
\begin{quote}\begin{description}
\item[{Parameters}] \leavevmode\begin{itemize}
\item {} 
\sphinxstyleliteralstrong{\sphinxupquote{df}} (\sphinxstyleliteralemphasis{\sphinxupquote{pandas.core.frame.DataFrame}}) \textendash{} the activity diary

\item {} 
\sphinxstyleliteralstrong{\sphinxupquote{act}} (\sphinxstyleliteralemphasis{\sphinxupquote{int}}) \textendash{} ABMHAP activity code

\end{itemize}

\item[{Returns}] \leavevmode
an activity diary containing information from the given activity

\item[{Return type}] \leavevmode
pandas.core.frame.DataFrame

\end{description}\end{quote}

\end{fulllineitems}

\index{get\_moments() (in module analyzer)}

\begin{fulllineitems}
\phantomsection\label{\detokenize{analyzer:analyzer.get_moments}}\pysiglinewithargsret{\sphinxcode{\sphinxupquote{analyzer.}}\sphinxbfcode{\sphinxupquote{get\_moments}}}{\emph{abm\_list}, \emph{do\_periodic=False}}{}
This function calculates both the mean and the standard deviation for start time, end time, and     duration for a given activity for each agent simulated.
\begin{quote}\begin{description}
\item[{Parameters}] \leavevmode\begin{itemize}
\item {} 
\sphinxstyleliteralstrong{\sphinxupquote{abm\_list}} (\sphinxstyleliteralemphasis{\sphinxupquote{list of pandas.core.frame.DataFrame}}) \textendash{} the activity diary for a given activity for each agent

\item {} 
\sphinxstyleliteralstrong{\sphinxupquote{do\_periodic}} (\sphinxstyleliteralemphasis{\sphinxupquote{bool}}) \textendash{} this flag indicates whether (if True) or not (if False) to do the     analysis on a time scale that is {[}-12, 12). This is useful for activities that may occur     over midnight.

\end{itemize}

\item[{Returns}] \leavevmode
information containing the the mean and standard deviation information for start time,     end time, and duration for the simulated agents for a given activity

\item[{Return type}] \leavevmode
numpy.ndarray, numpy.ndarray, numpy.ndarray, numpy.ndarray,     numpy.ndarray, numpy.ndarray

\end{description}\end{quote}

\end{fulllineitems}

\index{get\_simulation\_data() (in module analyzer)}

\begin{fulllineitems}
\phantomsection\label{\detokenize{analyzer:analyzer.get_simulation_data}}\pysiglinewithargsret{\sphinxcode{\sphinxupquote{analyzer.}}\sphinxbfcode{\sphinxupquote{get\_simulation\_data}}}{\emph{df\_list}, \emph{act}}{}
This function obtains the simulation data for a given activity from each each agent in the
simulation.
\begin{quote}\begin{description}
\item[{Parameters}] \leavevmode\begin{itemize}
\item {} 
\sphinxstyleliteralstrong{\sphinxupquote{df\_list}} (\sphinxstyleliteralemphasis{\sphinxupquote{list of pandas.core.frame.DataFrame}}) \textendash{} the activity diaries for each agent in the entire simulation

\item {} 
\sphinxstyleliteralstrong{\sphinxupquote{act}} (\sphinxstyleliteralemphasis{\sphinxupquote{int}}) \textendash{} the ABMHAP activity code

\end{itemize}

\item[{Returns}] \leavevmode
the activity diary containing information for the respective activity     for each agent simulated

\item[{Returns}] \leavevmode
list of pandas.core.frame.DataFrame

\end{description}\end{quote}

\end{fulllineitems}

\index{get\_verify\_fpath() (in module analyzer)}

\begin{fulllineitems}
\phantomsection\label{\detokenize{analyzer:analyzer.get_verify_fpath}}\pysiglinewithargsret{\sphinxcode{\sphinxupquote{analyzer.}}\sphinxbfcode{\sphinxupquote{get\_verify\_fpath}}}{\emph{fdir}, \emph{act\_codes}}{}
This function returns the directories corresponding to the specified activity codes     where figures of activities will be stored for a specific simulation.
\begin{quote}\begin{description}
\item[{Parameters}] \leavevmode\begin{itemize}
\item {} 
\sphinxstyleliteralstrong{\sphinxupquote{fdir}} \textendash{} the file path to the directory of the figure data for a specific simulation.

\item {} 
\sphinxstyleliteralstrong{\sphinxupquote{act\_codes}} (\sphinxstyleliteralemphasis{\sphinxupquote{list of int}}) \textendash{} the ABMHAP activity codes for a given activity

\end{itemize}

\item[{Returns}] \leavevmode
the directories corresponding to the specified activity codeds where figures     of the activities will be stored

\item[{Return type}] \leavevmode
list of str

\end{description}\end{quote}

\end{fulllineitems}

\index{load\_plot\_data() (in module analyzer)}

\begin{fulllineitems}
\phantomsection\label{\detokenize{analyzer:analyzer.load_plot_data}}\pysiglinewithargsret{\sphinxcode{\sphinxupquote{analyzer.}}\sphinxbfcode{\sphinxupquote{load\_plot\_data}}}{\emph{fname}}{}
This function loads the data from pickled (.pkl) figures. This assumes that the figures plotted the
ABM data first and then the CHAD data.
\begin{quote}\begin{description}
\item[{Parameters}] \leavevmode
\sphinxstyleliteralstrong{\sphinxupquote{fname}} (\sphinxstyleliteralemphasis{\sphinxupquote{str}}) \textendash{} the filename of the saved figure (.pkl)

\item[{Returns}] \leavevmode
the x and y data for the ABM and CHAD data, respectively

\item[{Return type}] \leavevmode
numpy.ndarray, numpy.ndarray, numpy.ndarray, numpy.ndarray

\end{description}\end{quote}

\end{fulllineitems}

\index{plot\_cdf() (in module analyzer)}

\begin{fulllineitems}
\phantomsection\label{\detokenize{analyzer:analyzer.plot_cdf}}\pysiglinewithargsret{\sphinxcode{\sphinxupquote{analyzer.}}\sphinxbfcode{\sphinxupquote{plot\_cdf}}}{\emph{data\_abm}, \emph{data\_chad}, \emph{xlabel}, \emph{title}}{}
This function plots the CDF for data related to the ABM and CHAD
\begin{quote}\begin{description}
\item[{Parameters}] \leavevmode\begin{itemize}
\item {} 
\sphinxstyleliteralstrong{\sphinxupquote{data\_abm}} (\sphinxstyleliteralemphasis{\sphinxupquote{numpy.ndarray}}) \textendash{} the ABM data to be plotted

\item {} 
\sphinxstyleliteralstrong{\sphinxupquote{data\_chad}} (\sphinxstyleliteralemphasis{\sphinxupquote{numpy.ndarray}}) \textendash{} the CHAD data to be plotted

\item {} 
\sphinxstyleliteralstrong{\sphinxupquote{xlabel}} (\sphinxstyleliteralemphasis{\sphinxupquote{str}}) \textendash{} the x-axis label

\item {} 
\sphinxstyleliteralstrong{\sphinxupquote{title}} (\sphinxstyleliteralemphasis{\sphinxupquote{str}}) \textendash{} the title of the plot

\end{itemize}

\item[{Returns}] \leavevmode


\end{description}\end{quote}

\end{fulllineitems}

\index{plot\_cdf\_new() (in module analyzer)}

\begin{fulllineitems}
\phantomsection\label{\detokenize{analyzer:analyzer.plot_cdf_new}}\pysiglinewithargsret{\sphinxcode{\sphinxupquote{analyzer.}}\sphinxbfcode{\sphinxupquote{plot\_cdf\_new}}}{\emph{data\_abm}, \emph{data\_chad}, \emph{fid}, \emph{title}, \emph{xlabel}, \emph{do\_periodic=False}}{}
This function plots the cumulative distribution function (CDF) comparing     the ABM and CHAD data for a given activity
\begin{quote}\begin{description}
\item[{Parameters}] \leavevmode\begin{itemize}
\item {} 
\sphinxstyleliteralstrong{\sphinxupquote{data\_abm}} (\sphinxstyleliteralemphasis{\sphinxupquote{numpy.ndarray}}) \textendash{} 

\item {} 
\sphinxstyleliteralstrong{\sphinxupquote{data\_chad}} (\sphinxstyleliteralemphasis{\sphinxupquote{numpy.ndarray}}) \textendash{} 

\item {} 
\sphinxstyleliteralstrong{\sphinxupquote{fid}} (\sphinxstyleliteralemphasis{\sphinxupquote{int}}) \textendash{} the figure identifier

\item {} 
\sphinxstyleliteralstrong{\sphinxupquote{title}} (\sphinxstyleliteralemphasis{\sphinxupquote{str}}) \textendash{} the title of the figure

\item {} 
\sphinxstyleliteralstrong{\sphinxupquote{xlabel}} (\sphinxstyleliteralemphasis{\sphinxupquote{str}}) \textendash{} the label of the x-axis

\item {} 
\sphinxstyleliteralstrong{\sphinxupquote{do\_periodic}} (\sphinxstyleliteralemphasis{\sphinxupquote{bool}}) \textendash{} this flag indicates whether (if True) or not (if False) to convert     the data to a time scale that is {[}-12, 12). This is useful for activities that may occur     over midnight.

\end{itemize}

\item[{Returns}] \leavevmode
the figure of the CDF

\item[{Return type}] \leavevmode
matplotlib.figure.Figure

\end{description}\end{quote}

\end{fulllineitems}

\index{plot\_verify\_dt() (in module analyzer)}

\begin{fulllineitems}
\phantomsection\label{\detokenize{analyzer:analyzer.plot_verify_dt}}\pysiglinewithargsret{\sphinxcode{\sphinxupquote{analyzer.}}\sphinxbfcode{\sphinxupquote{plot\_verify\_dt}}}{\emph{act}, \emph{data\_abm}, \emph{data\_chad}, \emph{fid}, \emph{do\_save\_fig=False}, \emph{fpath=''}}{}
This function plots the cumulative distribution function (CDFs) in order to compare     the duration data for the given activity from the ABMHAP simulation to the CHAD data.
\begin{quote}\begin{description}
\item[{Parameters}] \leavevmode\begin{itemize}
\item {} 
\sphinxstyleliteralstrong{\sphinxupquote{act}} (\sphinxstyleliteralemphasis{\sphinxupquote{int}}) \textendash{} the ABMHAP activity data

\item {} 
\sphinxstyleliteralstrong{\sphinxupquote{data\_abm}} (\sphinxstyleliteralemphasis{\sphinxupquote{numpy.ndarray}}) \textendash{} the ABMHAP duration data

\item {} 
\sphinxstyleliteralstrong{\sphinxupquote{data\_chad}} (\sphinxstyleliteralemphasis{\sphinxupquote{numpy.ndarray}}) \textendash{} the CHAD duration data

\item {} 
\sphinxstyleliteralstrong{\sphinxupquote{fid}} (\sphinxstyleliteralemphasis{\sphinxupquote{int}}) \textendash{} the figure identifier

\item {} 
\sphinxstyleliteralstrong{\sphinxupquote{do\_save\_fig}} (\sphinxstyleliteralemphasis{\sphinxupquote{bool}}) \textendash{} a flag indicating whether (if True) or not (if False)     to save the figure

\item {} 
\sphinxstyleliteralstrong{\sphinxupquote{fpath}} (\sphinxstyleliteralemphasis{\sphinxupquote{str}}) \textendash{} the file path to the directory in which to save the figure

\end{itemize}

\item[{Returns}] \leavevmode


\end{description}\end{quote}

\end{fulllineitems}

\index{plot\_verify\_end() (in module analyzer)}

\begin{fulllineitems}
\phantomsection\label{\detokenize{analyzer:analyzer.plot_verify_end}}\pysiglinewithargsret{\sphinxcode{\sphinxupquote{analyzer.}}\sphinxbfcode{\sphinxupquote{plot\_verify\_end}}}{\emph{act}, \emph{data\_abm}, \emph{data\_chad}, \emph{fid}, \emph{do\_save\_fig=False}, \emph{fpath=''}}{}
This function plots the cumulative distribution function (CDFs) in order to compare     the end time data for the given activity from the ABMHAP simulation to the CHAD data.
\begin{quote}\begin{description}
\item[{Parameters}] \leavevmode\begin{itemize}
\item {} 
\sphinxstyleliteralstrong{\sphinxupquote{act}} (\sphinxstyleliteralemphasis{\sphinxupquote{int}}) \textendash{} the ABMHAP activity data

\item {} 
\sphinxstyleliteralstrong{\sphinxupquote{data\_abm}} (\sphinxstyleliteralemphasis{\sphinxupquote{numpy.ndarray}}) \textendash{} the ABMHAP end time data

\item {} 
\sphinxstyleliteralstrong{\sphinxupquote{data\_chad}} (\sphinxstyleliteralemphasis{\sphinxupquote{numpy.ndarray}}) \textendash{} the CHAD end time data

\item {} 
\sphinxstyleliteralstrong{\sphinxupquote{fid}} (\sphinxstyleliteralemphasis{\sphinxupquote{int}}) \textendash{} the figure identifier

\item {} 
\sphinxstyleliteralstrong{\sphinxupquote{do\_save\_fig}} (\sphinxstyleliteralemphasis{\sphinxupquote{bool}}) \textendash{} a flag indicating whether (if True) or not (if False)     to save the figure

\item {} 
\sphinxstyleliteralstrong{\sphinxupquote{fpath}} (\sphinxstyleliteralemphasis{\sphinxupquote{str}}) \textendash{} the file path to the directory in which to save the figure

\end{itemize}

\item[{Returns}] \leavevmode


\end{description}\end{quote}

\end{fulllineitems}

\index{plot\_verify\_start() (in module analyzer)}

\begin{fulllineitems}
\phantomsection\label{\detokenize{analyzer:analyzer.plot_verify_start}}\pysiglinewithargsret{\sphinxcode{\sphinxupquote{analyzer.}}\sphinxbfcode{\sphinxupquote{plot\_verify\_start}}}{\emph{act}, \emph{data\_abm}, \emph{data\_chad}, \emph{fid}, \emph{do\_save\_fig=False}, \emph{fpath=''}}{}
This function plots the cumulative distribution function (CDFs) in order to compare     the start time data for the given activity from the ABMHAP simulation to the CHAD data.
\begin{quote}\begin{description}
\item[{Parameters}] \leavevmode\begin{itemize}
\item {} 
\sphinxstyleliteralstrong{\sphinxupquote{act}} (\sphinxstyleliteralemphasis{\sphinxupquote{int}}) \textendash{} the ABMHAP activity data

\item {} 
\sphinxstyleliteralstrong{\sphinxupquote{data\_abm}} (\sphinxstyleliteralemphasis{\sphinxupquote{numpy.ndarray}}) \textendash{} the ABMHAP start time data

\item {} 
\sphinxstyleliteralstrong{\sphinxupquote{data\_chad}} (\sphinxstyleliteralemphasis{\sphinxupquote{numpy.ndarray}}) \textendash{} the CHAD start time data

\item {} 
\sphinxstyleliteralstrong{\sphinxupquote{fid}} (\sphinxstyleliteralemphasis{\sphinxupquote{int}}) \textendash{} the figure identifier

\item {} 
\sphinxstyleliteralstrong{\sphinxupquote{do\_save\_fig}} (\sphinxstyleliteralemphasis{\sphinxupquote{bool}}) \textendash{} a flag indicating whether (if True) or not (if False)     to save the figure

\item {} 
\sphinxstyleliteralstrong{\sphinxupquote{fpath}} (\sphinxstyleliteralemphasis{\sphinxupquote{str}}) \textendash{} the file path to the directory in which to save the figure

\end{itemize}

\item[{Returns}] \leavevmode


\end{description}\end{quote}

\end{fulllineitems}

\index{run() (in module analyzer)}

\begin{fulllineitems}
\phantomsection\label{\detokenize{analyzer:analyzer.run}}\pysiglinewithargsret{\sphinxcode{\sphinxupquote{analyzer.}}\sphinxbfcode{\sphinxupquote{run}}}{\emph{num\_process}, \emph{num\_hhld}, \emph{num\_batch}}{}
This function runs the simulations.
\begin{quote}\begin{description}
\item[{Parameters}] \leavevmode\begin{itemize}
\item {} 
\sphinxstyleliteralstrong{\sphinxupquote{num\_process}} (\sphinxstyleliteralemphasis{\sphinxupquote{int}}) \textendash{} the number of processors (cores)

\item {} 
\sphinxstyleliteralstrong{\sphinxupquote{num\_hhld}} (\sphinxstyleliteralemphasis{\sphinxupquote{int}}) \textendash{} the number of households per core per batch

\item {} 
\sphinxstyleliteralstrong{\sphinxupquote{num\_batch}} (\sphinxstyleliteralemphasis{\sphinxupquote{int}}) \textendash{} the number of batches

\end{itemize}

\item[{Returns}] \leavevmode
the results of the simulation

\item[{Return type}] \leavevmode
{\hyperref[\detokenize{driver_result:driver_result.Driver_Result}]{\sphinxcrossref{driver\_result.Driver\_Result}}}

\end{description}\end{quote}

\end{fulllineitems}

\index{verify() (in module analyzer)}

\begin{fulllineitems}
\phantomsection\label{\detokenize{analyzer:analyzer.verify}}\pysiglinewithargsret{\sphinxcode{\sphinxupquote{analyzer.}}\sphinxbfcode{\sphinxupquote{verify}}}{\emph{trial\_code}, \emph{demo}, \emph{chad\_param\_list}, \emph{df\_list}, \emph{do\_plot}, \emph{do\_print=False}, \emph{fdir=None}}{}
This code compares the results of the ABM to the CHAD data by comparing the cumulative distribution function     (CDF) of the duration and start times predicted by the ABM and that of respective CDFs from the CHAD data.
\begin{quote}\begin{description}
\item[{Parameters}] \leavevmode\begin{itemize}
\item {} 
\sphinxstyleliteralstrong{\sphinxupquote{trial\_code}} (\sphinxstyleliteralemphasis{\sphinxupquote{int}}) \textendash{} the trial code identifier

\item {} 
\sphinxstyleliteralstrong{\sphinxupquote{demo}} (\sphinxstyleliteralemphasis{\sphinxupquote{int}}) \textendash{} the demographic identifier

\item {} 
\sphinxstyleliteralstrong{\sphinxupquote{chad\_param\_list}} (list of {\hyperref[\detokenize{chad_params:chad_params.CHAD_params}]{\sphinxcrossref{\sphinxcode{\sphinxupquote{chad\_params.CHAD\_params}}}}}) \textendash{} that limit the CHAD parameters sampling in initializing the households

\item {} 
\sphinxstyleliteralstrong{\sphinxupquote{df\_list}} (\sphinxstyleliteralemphasis{\sphinxupquote{list of pandas.core.frame.DataFrame}}) \textendash{} contains the activity diaries for each household

\item {} 
\sphinxstyleliteralstrong{\sphinxupquote{do\_plot}} (\sphinxstyleliteralemphasis{\sphinxupquote{bool}}) \textendash{} a flag to indicate whether (True) or not (False) to plot

\item {} 
\sphinxstyleliteralstrong{\sphinxupquote{do\_print}} (\sphinxstyleliteralemphasis{\sphinxupquote{bool}}) \textendash{} a flag to indicate whether (True) or not (False) to print various messages to the screen

\item {} 
\sphinxstyleliteralstrong{\sphinxupquote{fdir}} (\sphinxstyleliteralemphasis{\sphinxupquote{list}}) \textendash{} a list of file directories needed to save the figures

\end{itemize}

\item[{Returns}] \leavevmode


\end{description}\end{quote}

\end{fulllineitems}



\subsection{chad\_demography module}
\label{\detokenize{chad_demography::doc}}\label{\detokenize{chad_demography:chad-demography-module}}\label{\detokenize{chad_demography:module-chad_demography}}\index{chad\_demography (module)}
This module contains code that handles accessing the Consolidated Human Activity Database (CHAD)
data for various demographics.

This module contains {\hyperref[\detokenize{chad_demography:chad_demography.CHAD_demography}]{\sphinxcrossref{\sphinxcode{\sphinxupquote{chad\_demography.CHAD\_demography}}}}}.
\index{CHAD\_demography (class in chad\_demography)}

\begin{fulllineitems}
\phantomsection\label{\detokenize{chad_demography:chad_demography.CHAD_demography}}\pysiglinewithargsret{\sphinxbfcode{\sphinxupquote{class }}\sphinxcode{\sphinxupquote{chad\_demography.}}\sphinxbfcode{\sphinxupquote{CHAD\_demography}}}{\emph{demographic}}{}
Bases: \sphinxcode{\sphinxupquote{object}}

This class contains the common functionality with accessing the CHAD data files
relevant to different demographics.
\begin{quote}\begin{description}
\item[{Parameters}] \leavevmode
\sphinxstyleliteralstrong{\sphinxupquote{demographic}} (\sphinxstyleliteralemphasis{\sphinxupquote{int}}) \textendash{} the demographic identifier

\item[{Variables}] \leavevmode\begin{itemize}
\item {} 
\sphinxstyleliteralstrong{\sphinxupquote{demographic}} (\sphinxstyleliteralemphasis{\sphinxupquote{int}}) \textendash{} the demographic identifier

\item {} 
\sphinxstyleliteralstrong{\sphinxupquote{fname\_zip}} (\sphinxstyleliteralemphasis{\sphinxupquote{str}}) \textendash{} the name of the file (.zip) that contains the CHAD data

\item {} 
\sphinxstyleliteralstrong{\sphinxupquote{fname\_stats\_commute\_to\_work}} (\sphinxstyleliteralemphasis{\sphinxupquote{dict}}) \textendash{} the file names for the CHAD longitudinal data     (start time, end time, duration, and records) to be sampled for commuting to work

\item {} 
\sphinxstyleliteralstrong{\sphinxupquote{fname\_stats\_commute\_from\_work}} (\sphinxstyleliteralemphasis{\sphinxupquote{dict}}) \textendash{} the file names for the CHAD longitudinal data     (start time, end time, duration, and records) to be sampled for commuting from work

\item {} 
\sphinxstyleliteralstrong{\sphinxupquote{fname\_stats\_eat\_breakfast}} (\sphinxstyleliteralemphasis{\sphinxupquote{dict}}) \textendash{} the file names for the CHAD longitudinal data     (start time, end time, duration, and records) to be sampled for eating breakfast

\item {} 
\sphinxstyleliteralstrong{\sphinxupquote{fname\_stats\_eat\_dinner}} (\sphinxstyleliteralemphasis{\sphinxupquote{dict}}) \textendash{} the file names for the CHAD longitudinal data     (start time, end time, duration, and records) to be sampled for eating dinner

\item {} 
\sphinxstyleliteralstrong{\sphinxupquote{fname\_stats\_eat\_lunch}} (\sphinxstyleliteralemphasis{\sphinxupquote{dict}}) \textendash{} the file names for the CHAD longitudinal data     (start time, end time, duration, and records) to be sampled for eating lunch

\item {} 
\sphinxstyleliteralstrong{\sphinxupquote{fname\_stats\_school}} (\sphinxstyleliteralemphasis{\sphinxupquote{dict}}) \textendash{} the file names for the CHAD longitudinal data     (start time, end time, duration, and records) to be sampled for schooling

\item {} 
\sphinxstyleliteralstrong{\sphinxupquote{fname\_stats\_sleep}} (\sphinxstyleliteralemphasis{\sphinxupquote{dict}}) \textendash{} the file names for the CHAD longitudinal data     (start time, end time, duration, and records) to be sampled for sleeping

\item {} 
\sphinxstyleliteralstrong{\sphinxupquote{fname\_stats\_work}} (\sphinxstyleliteralemphasis{\sphinxupquote{dict}}) \textendash{} the file names for the CHAD longitudinal data     (start time, end time, duration, and records) to be sampled for working

\item {} 
\sphinxstyleliteralstrong{\sphinxupquote{n\_commute\_from\_work}} (\sphinxstyleliteralemphasis{\sphinxupquote{int}}) \textendash{} the minimum number of events needed in sampling     from CHAD longitudinal data for commuting from work

\item {} 
\sphinxstyleliteralstrong{\sphinxupquote{n\_commute\_to\_work}} (\sphinxstyleliteralemphasis{\sphinxupquote{int}}) \textendash{} the minimum number of events needed in sampling     from CHAD longitudinal data for commuting to work

\item {} 
\sphinxstyleliteralstrong{\sphinxupquote{n\_eat\_breakfast}} (\sphinxstyleliteralemphasis{\sphinxupquote{int}}) \textendash{} the minimum number of events needed in sampling     from CHAD longitudinal data for eating breakfast

\item {} 
\sphinxstyleliteralstrong{\sphinxupquote{n\_eat\_dinner}} (\sphinxstyleliteralemphasis{\sphinxupquote{int}}) \textendash{} the minimum number of events needed in sampling     from CHAD longitudinal data for eating dinner

\item {} 
\sphinxstyleliteralstrong{\sphinxupquote{n\_eat\_lunch}} (\sphinxstyleliteralemphasis{\sphinxupquote{int}}) \textendash{} the minimum number of events needed in sampling     from CHAD longitudinal data for eating lunch

\item {} 
\sphinxstyleliteralstrong{\sphinxupquote{n\_school}} (\sphinxstyleliteralemphasis{\sphinxupquote{int}}) \textendash{} the minimum number of events needed in sampling     from CHAD longitudinal data for schooling

\item {} 
\sphinxstyleliteralstrong{\sphinxupquote{n\_sleep}} (\sphinxstyleliteralemphasis{\sphinxupquote{int}}) \textendash{} the minimum number of events needed in sampling     from CHAD longitudinal data for sleeping

\item {} 
\sphinxstyleliteralstrong{\sphinxupquote{n\_work}} (\sphinxstyleliteralemphasis{\sphinxupquote{int}}) \textendash{} the minimum number of events needed in sampling     from CHAD longitudinal data for working

\item {} 
\sphinxstyleliteralstrong{\sphinxupquote{work\_start\_mean\_min}} (\sphinxstyleliteralemphasis{\sphinxupquote{float}}) \textendash{} the minimum mean start time for working when sampling CHAD data

\item {} 
\sphinxstyleliteralstrong{\sphinxupquote{work\_start\_mean\_max}} (\sphinxstyleliteralemphasis{\sphinxupquote{float}}) \textendash{} the maximum mean start time for working when sampling CHAD data

\item {} 
\sphinxstyleliteralstrong{\sphinxupquote{work\_start\_std\_max}} (\sphinxstyleliteralemphasis{\sphinxupquote{float}}) \textendash{} the maximum standard deviation for start time for working when sampling CHAD data

\item {} 
\sphinxstyleliteralstrong{\sphinxupquote{work\_end\_mean\_min}} (\sphinxstyleliteralemphasis{\sphinxupquote{float}}) \textendash{} the minimum mean end time for working when sampling CHAD data

\item {} 
\sphinxstyleliteralstrong{\sphinxupquote{work\_end\_mean\_max}} (\sphinxstyleliteralemphasis{\sphinxupquote{float}}) \textendash{} the maximum mean end time for working when sampling CHAD data

\item {} 
\sphinxstyleliteralstrong{\sphinxupquote{work\_end\_std\_max}} (\sphinxstyleliteralemphasis{\sphinxupquote{float}}) \textendash{} the maximum standard deviating for end time for working when sampling CHAD data

\item {} 
\sphinxstyleliteralstrong{\sphinxupquote{work\_dt\_mean\_min}} (\sphinxstyleliteralemphasis{\sphinxupquote{float}}) \textendash{} the minimum mean duration for working when sampling CHAD data

\item {} 
\sphinxstyleliteralstrong{\sphinxupquote{work\_dt\_mean\_max}} (\sphinxstyleliteralemphasis{\sphinxupquote{float}}) \textendash{} the maximum mean duration for working when sampling CHAD data

\item {} 
\sphinxstyleliteralstrong{\sphinxupquote{work\_dt\_std\_max}} (\sphinxstyleliteralemphasis{\sphinxupquote{float}}) \textendash{} the maximum standard deviation for working when sampling CHAD data

\item {} 
\sphinxstyleliteralstrong{\sphinxupquote{school\_start\_mean\_min}} (\sphinxstyleliteralemphasis{\sphinxupquote{float}}) \textendash{} the minimum mean start time for schooling when sampling CHAD data

\item {} 
\sphinxstyleliteralstrong{\sphinxupquote{school\_start\_mean\_max}} (\sphinxstyleliteralemphasis{\sphinxupquote{float}}) \textendash{} the maximum mean start time for schooling when sampling CHAD data

\item {} 
\sphinxstyleliteralstrong{\sphinxupquote{school\_start\_std\_max}} (\sphinxstyleliteralemphasis{\sphinxupquote{float}}) \textendash{} the maximum standard deviation for start time for schooling when sampling CHAD data

\item {} 
\sphinxstyleliteralstrong{\sphinxupquote{school\_end\_mean\_min}} (\sphinxstyleliteralemphasis{\sphinxupquote{float}}) \textendash{} the minimum mean end time for schooling when sampling CHAD data

\item {} 
\sphinxstyleliteralstrong{\sphinxupquote{school\_end\_mean\_max}} (\sphinxstyleliteralemphasis{\sphinxupquote{float}}) \textendash{} the maximum mean end time for schooling when sampling CHAD data

\item {} 
\sphinxstyleliteralstrong{\sphinxupquote{school\_end\_std\_max}} (\sphinxstyleliteralemphasis{\sphinxupquote{float}}) \textendash{} the maximum standard deviating for end time for schooling when sampling CHAD data

\item {} 
\sphinxstyleliteralstrong{\sphinxupquote{school\_dt\_mean\_min}} (\sphinxstyleliteralemphasis{\sphinxupquote{float}}) \textendash{} the minimum mean duration for schooling when sampling CHAD data

\item {} 
\sphinxstyleliteralstrong{\sphinxupquote{school\_dt\_mean\_max}} (\sphinxstyleliteralemphasis{\sphinxupquote{float}}) \textendash{} the maximum mean duration for schooling when sampling CHAD data

\item {} 
\sphinxstyleliteralstrong{\sphinxupquote{school\_dt\_std\_max}} (\sphinxstyleliteralemphasis{\sphinxupquote{float}}) \textendash{} the maximum standard deviation for schooling when sampling CHAD data

\item {} 
\sphinxstyleliteralstrong{\sphinxupquote{commute\_to\_work\_start\_mean\_min}} (\sphinxstyleliteralemphasis{\sphinxupquote{float}}) \textendash{} the minimum mean start time for commuting to work when sampling CHAD data

\item {} 
\sphinxstyleliteralstrong{\sphinxupquote{commute\_to\_work\_start\_mean\_max}} (\sphinxstyleliteralemphasis{\sphinxupquote{float}}) \textendash{} the maximum mean start time for commuting to work when sampling CHAD data

\item {} 
\sphinxstyleliteralstrong{\sphinxupquote{commute\_to\_work\_start\_std\_max}} (\sphinxstyleliteralemphasis{\sphinxupquote{float}}) \textendash{} the maximum standard deviation for start time for commuting to work when sampling CHAD data

\item {} 
\sphinxstyleliteralstrong{\sphinxupquote{commute\_to\_work\_end\_mean\_min}} (\sphinxstyleliteralemphasis{\sphinxupquote{float}}) \textendash{} the minimum mean end time for commuting to work when sampling CHAD data

\item {} 
\sphinxstyleliteralstrong{\sphinxupquote{commute\_to\_work\_end\_mean\_max}} (\sphinxstyleliteralemphasis{\sphinxupquote{float}}) \textendash{} the maximum mean end time for commuting to work when sampling CHAD data

\item {} 
\sphinxstyleliteralstrong{\sphinxupquote{commute\_to\_work\_end\_std\_max}} (\sphinxstyleliteralemphasis{\sphinxupquote{float}}) \textendash{} the maximum standard deviating for end time for commuting to work when sampling CHAD data

\item {} 
\sphinxstyleliteralstrong{\sphinxupquote{commute\_to\_work\_dt\_mean\_min}} (\sphinxstyleliteralemphasis{\sphinxupquote{float}}) \textendash{} the minimum mean duration for commuting to work when sampling CHAD data

\item {} 
\sphinxstyleliteralstrong{\sphinxupquote{commute\_to\_work\_dt\_mean\_max}} (\sphinxstyleliteralemphasis{\sphinxupquote{float}}) \textendash{} the maximum mean duration for commuting to work when sampling CHAD data

\item {} 
\sphinxstyleliteralstrong{\sphinxupquote{commute\_to\_work\_dt\_std\_max}} (\sphinxstyleliteralemphasis{\sphinxupquote{float}}) \textendash{} the maximum standard deviation for commuting to work when sampling CHAD data

\item {} 
\sphinxstyleliteralstrong{\sphinxupquote{eat\_breakfast\_start\_mean\_min}} (\sphinxstyleliteralemphasis{\sphinxupquote{float}}) \textendash{} the minimum mean start time for eating breakfast when sampling CHAD data

\item {} 
\sphinxstyleliteralstrong{\sphinxupquote{eat\_breakfast\_start\_mean\_max}} (\sphinxstyleliteralemphasis{\sphinxupquote{float}}) \textendash{} the maximum mean start time for eating breakfast when sampling CHAD data

\item {} 
\sphinxstyleliteralstrong{\sphinxupquote{eat\_breakfast\_start\_std\_max}} (\sphinxstyleliteralemphasis{\sphinxupquote{float}}) \textendash{} the maximum standard deviation for start time for eating breakfast when sampling CHAD data

\item {} 
\sphinxstyleliteralstrong{\sphinxupquote{eat\_breakfast\_end\_mean\_min}} (\sphinxstyleliteralemphasis{\sphinxupquote{float}}) \textendash{} the minimum mean end time for eating breakfast when sampling CHAD data

\item {} 
\sphinxstyleliteralstrong{\sphinxupquote{eat\_breakfast\_end\_mean\_max}} (\sphinxstyleliteralemphasis{\sphinxupquote{float}}) \textendash{} the maximum mean end time for eating breakfast when sampling CHAD data

\item {} 
\sphinxstyleliteralstrong{\sphinxupquote{eat\_breakfast\_end\_std\_max}} (\sphinxstyleliteralemphasis{\sphinxupquote{float}}) \textendash{} the maximum standard deviating for end time for eating breakfast when sampling CHAD data

\item {} 
\sphinxstyleliteralstrong{\sphinxupquote{eat\_breakfast\_dt\_mean\_min}} (\sphinxstyleliteralemphasis{\sphinxupquote{float}}) \textendash{} the minimum mean duration for eating breakfast when sampling CHAD data

\item {} 
\sphinxstyleliteralstrong{\sphinxupquote{eat\_breakfast\_dt\_mean\_max}} (\sphinxstyleliteralemphasis{\sphinxupquote{float}}) \textendash{} the maximum mean duration for eating breakfast when sampling CHAD data

\item {} 
\sphinxstyleliteralstrong{\sphinxupquote{eat\_breakfast\_dt\_std\_max}} (\sphinxstyleliteralemphasis{\sphinxupquote{float}}) \textendash{} the maximum standard deviation for eating breakfast when sampling CHAD data

\item {} 
\sphinxstyleliteralstrong{\sphinxupquote{commute\_from\_work\_start\_mean\_min}} (\sphinxstyleliteralemphasis{\sphinxupquote{float}}) \textendash{} the minimum mean start time for commuting from work when sampling CHAD data

\item {} 
\sphinxstyleliteralstrong{\sphinxupquote{commute\_from\_work\_start\_mean\_max}} (\sphinxstyleliteralemphasis{\sphinxupquote{float}}) \textendash{} the maximum mean start time for commuting from work when sampling CHAD data

\item {} 
\sphinxstyleliteralstrong{\sphinxupquote{commute\_from\_work\_start\_std\_max}} (\sphinxstyleliteralemphasis{\sphinxupquote{float}}) \textendash{} the maximum standard deviation for start time for commuting from work when sampling CHAD data

\item {} 
\sphinxstyleliteralstrong{\sphinxupquote{commute\_from\_work\_end\_mean\_min}} (\sphinxstyleliteralemphasis{\sphinxupquote{float}}) \textendash{} the minimum mean end time for commuting from work when sampling CHAD data

\item {} 
\sphinxstyleliteralstrong{\sphinxupquote{commute\_from\_work\_end\_mean\_max}} (\sphinxstyleliteralemphasis{\sphinxupquote{float}}) \textendash{} the maximum mean end time for commuting from work when sampling CHAD data

\item {} 
\sphinxstyleliteralstrong{\sphinxupquote{commute\_from\_work\_end\_std\_max}} (\sphinxstyleliteralemphasis{\sphinxupquote{float}}) \textendash{} the maximum standard deviating for end time for commuting from work when sampling CHAD data

\item {} 
\sphinxstyleliteralstrong{\sphinxupquote{commute\_from\_work\_dt\_mean\_min}} (\sphinxstyleliteralemphasis{\sphinxupquote{float}}) \textendash{} the minimum mean duration for commuting from work when sampling CHAD data

\item {} 
\sphinxstyleliteralstrong{\sphinxupquote{commute\_from\_work\_dt\_mean\_max}} (\sphinxstyleliteralemphasis{\sphinxupquote{float}}) \textendash{} the maximum mean duration for commuting from work when sampling CHAD data

\item {} 
\sphinxstyleliteralstrong{\sphinxupquote{commute\_from\_work\_dt\_std\_max}} (\sphinxstyleliteralemphasis{\sphinxupquote{float}}) \textendash{} the maximum standard deviation for commuting from work when sampling CHAD data

\item {} 
\sphinxstyleliteralstrong{\sphinxupquote{sleep\_start\_mean\_min}} (\sphinxstyleliteralemphasis{\sphinxupquote{float}}) \textendash{} the minimum mean start time for sleeping when sampling CHAD data

\item {} 
\sphinxstyleliteralstrong{\sphinxupquote{sleep\_start\_mean\_max}} (\sphinxstyleliteralemphasis{\sphinxupquote{float}}) \textendash{} the maximum mean start time for sleeping when sampling CHAD data

\item {} 
\sphinxstyleliteralstrong{\sphinxupquote{sleep\_start\_std\_max}} (\sphinxstyleliteralemphasis{\sphinxupquote{float}}) \textendash{} the maximum standard deviation for start time for sleeping when sampling CHAD data

\item {} 
\sphinxstyleliteralstrong{\sphinxupquote{sleep\_end\_mean\_min}} (\sphinxstyleliteralemphasis{\sphinxupquote{float}}) \textendash{} the minimum mean end time for sleeping when sampling CHAD data

\item {} 
\sphinxstyleliteralstrong{\sphinxupquote{sleep\_end\_mean\_max}} (\sphinxstyleliteralemphasis{\sphinxupquote{float}}) \textendash{} the maximum mean end time for sleeping when sampling CHAD data

\item {} 
\sphinxstyleliteralstrong{\sphinxupquote{sleep\_end\_std\_max}} (\sphinxstyleliteralemphasis{\sphinxupquote{float}}) \textendash{} the maximum standard deviating for end time for sleeping when sampling CHAD data

\item {} 
\sphinxstyleliteralstrong{\sphinxupquote{sleep\_dt\_mean\_min}} (\sphinxstyleliteralemphasis{\sphinxupquote{float}}) \textendash{} the minimum mean duration for sleeping when sampling CHAD data

\item {} 
\sphinxstyleliteralstrong{\sphinxupquote{sleep\_dt\_mean\_max}} (\sphinxstyleliteralemphasis{\sphinxupquote{float}}) \textendash{} the maximum mean duration for sleeping when sampling CHAD data

\item {} 
\sphinxstyleliteralstrong{\sphinxupquote{sleep\_dt\_std\_max}} (\sphinxstyleliteralemphasis{\sphinxupquote{float}}) \textendash{} the maximum standard deviation for sleeping when sampling CHAD data

\item {} 
\sphinxstyleliteralstrong{\sphinxupquote{eat\_lunch\_start\_mean\_min}} (\sphinxstyleliteralemphasis{\sphinxupquote{float}}) \textendash{} the minimum mean start time for eating lunch when sampling CHAD data

\item {} 
\sphinxstyleliteralstrong{\sphinxupquote{eat\_lunch\_start\_mean\_max}} (\sphinxstyleliteralemphasis{\sphinxupquote{float}}) \textendash{} the maximum mean start time for eating lunch when sampling CHAD data

\item {} 
\sphinxstyleliteralstrong{\sphinxupquote{eat\_lunch\_start\_std\_max}} (\sphinxstyleliteralemphasis{\sphinxupquote{float}}) \textendash{} the maximum standard deviation for start time for eating lunch when sampling CHAD data

\item {} 
\sphinxstyleliteralstrong{\sphinxupquote{eat\_lunch\_end\_mean\_min}} (\sphinxstyleliteralemphasis{\sphinxupquote{float}}) \textendash{} the minimum mean end time for eating lunch when sampling CHAD data

\item {} 
\sphinxstyleliteralstrong{\sphinxupquote{eat\_lunch\_end\_mean\_max}} (\sphinxstyleliteralemphasis{\sphinxupquote{float}}) \textendash{} the maximum mean end time for eating lunch when sampling CHAD data

\item {} 
\sphinxstyleliteralstrong{\sphinxupquote{eat\_lunch\_end\_std\_max}} (\sphinxstyleliteralemphasis{\sphinxupquote{float}}) \textendash{} the maximum standard deviating for end time for eating lunch when sampling CHAD data

\item {} 
\sphinxstyleliteralstrong{\sphinxupquote{eat\_lunch\_dt\_mean\_min}} (\sphinxstyleliteralemphasis{\sphinxupquote{float}}) \textendash{} the minimum mean duration for eating lunch when sampling CHAD data

\item {} 
\sphinxstyleliteralstrong{\sphinxupquote{eat\_lunch\_dt\_mean\_max}} (\sphinxstyleliteralemphasis{\sphinxupquote{float}}) \textendash{} the maximum mean duration for eating lunch when sampling CHAD data

\item {} 
\sphinxstyleliteralstrong{\sphinxupquote{eat\_lunch\_dt\_std\_max}} (\sphinxstyleliteralemphasis{\sphinxupquote{float}}) \textendash{} the maximum standard deviation for eating lunch when sampling CHAD data

\item {} 
\sphinxstyleliteralstrong{\sphinxupquote{eat\_dinner\_start\_mean\_min}} (\sphinxstyleliteralemphasis{\sphinxupquote{float}}) \textendash{} the minimum mean start time for eating dinner when sampling CHAD data

\item {} 
\sphinxstyleliteralstrong{\sphinxupquote{eat\_dinner\_start\_mean\_max}} (\sphinxstyleliteralemphasis{\sphinxupquote{float}}) \textendash{} the maximum mean start time for eating dinner when sampling CHAD data

\item {} 
\sphinxstyleliteralstrong{\sphinxupquote{eat\_dinner\_start\_std\_max}} (\sphinxstyleliteralemphasis{\sphinxupquote{float}}) \textendash{} the maximum standard deviation for start time for eating dinner when sampling CHAD data

\item {} 
\sphinxstyleliteralstrong{\sphinxupquote{eat\_dinner\_end\_mean\_min}} (\sphinxstyleliteralemphasis{\sphinxupquote{float}}) \textendash{} the minimum mean end time for eating dinner when sampling CHAD data

\item {} 
\sphinxstyleliteralstrong{\sphinxupquote{eat\_dinner\_end\_mean\_max}} (\sphinxstyleliteralemphasis{\sphinxupquote{float}}) \textendash{} the maximum mean end time for eating dinner when sampling CHAD data

\item {} 
\sphinxstyleliteralstrong{\sphinxupquote{eat\_dinner\_end\_std\_max}} (\sphinxstyleliteralemphasis{\sphinxupquote{float}}) \textendash{} the maximum standard deviating for end time for eating dinner when sampling CHAD data

\item {} 
\sphinxstyleliteralstrong{\sphinxupquote{eat\_dinner\_dt\_mean\_min}} (\sphinxstyleliteralemphasis{\sphinxupquote{float}}) \textendash{} the minimum mean duration for eating dinner when sampling CHAD data

\item {} 
\sphinxstyleliteralstrong{\sphinxupquote{eat\_dinner\_dt\_mean\_max}} (\sphinxstyleliteralemphasis{\sphinxupquote{float}}) \textendash{} the maximum mean duration for eating dinner when sampling CHAD data

\item {} 
\sphinxstyleliteralstrong{\sphinxupquote{eat\_dinner\_dt\_std\_max}} (\sphinxstyleliteralemphasis{\sphinxupquote{float}}) \textendash{} the maximum standard deviation for eating dinner when sampling CHAD data

\end{itemize}

\end{description}\end{quote}
\index{set\_dt\_bounds() (chad\_demography.CHAD\_demography method)}

\begin{fulllineitems}
\phantomsection\label{\detokenize{chad_demography:chad_demography.CHAD_demography.set_dt_bounds}}\pysiglinewithargsret{\sphinxbfcode{\sphinxupquote{set\_dt\_bounds}}}{\emph{start\_min}, \emph{start\_max}, \emph{end\_min}, \emph{end\_max}}{}
This function calculates the bounds for duration time {[}expressed in hours\}
\begin{quote}\begin{description}
\item[{Parameters}] \leavevmode\begin{itemize}
\item {} 
\sphinxstyleliteralstrong{\sphinxupquote{start\_min}} (\sphinxstyleliteralemphasis{\sphinxupquote{float}}) \textendash{} the minimum start time {[}hours{]}

\item {} 
\sphinxstyleliteralstrong{\sphinxupquote{start\_max}} (\sphinxstyleliteralemphasis{\sphinxupquote{float}}) \textendash{} the maximum start time {[}hours{]}

\item {} 
\sphinxstyleliteralstrong{\sphinxupquote{end\_min}} (\sphinxstyleliteralemphasis{\sphinxupquote{float}}) \textendash{} the minimum end time {[}hours{]}

\item {} 
\sphinxstyleliteralstrong{\sphinxupquote{end\_max}} (\sphinxstyleliteralemphasis{\sphinxupquote{float}}) \textendash{} the maximum end time {[}hours{]}

\end{itemize}

\item[{Returns}] \leavevmode
the minimum duration, the maximum duration

\item[{Return type}] \leavevmode
float, float

\end{description}\end{quote}

\end{fulllineitems}

\index{set\_end\_bounds() (chad\_demography.CHAD\_demography method)}

\begin{fulllineitems}
\phantomsection\label{\detokenize{chad_demography:chad_demography.CHAD_demography.set_end_bounds}}\pysiglinewithargsret{\sphinxbfcode{\sphinxupquote{set\_end\_bounds}}}{\emph{start\_min}, \emph{start\_max}, \emph{dt\_min}, \emph{dt\_max}}{}
This function calculates the bounds for end time {[}expressed in hours{]}
\begin{quote}\begin{description}
\item[{Parameters}] \leavevmode\begin{itemize}
\item {} 
\sphinxstyleliteralstrong{\sphinxupquote{start\_min}} (\sphinxstyleliteralemphasis{\sphinxupquote{float}}) \textendash{} the minimum start time {[}hours{]}

\item {} 
\sphinxstyleliteralstrong{\sphinxupquote{start\_max}} (\sphinxstyleliteralemphasis{\sphinxupquote{float}}) \textendash{} the maximum start time {[}hours{]}

\item {} 
\sphinxstyleliteralstrong{\sphinxupquote{dt\_min}} (\sphinxstyleliteralemphasis{\sphinxupquote{float}}) \textendash{} the minimum duration {[}hours{]}

\item {} 
\sphinxstyleliteralstrong{\sphinxupquote{dt\_max}} (\sphinxstyleliteralemphasis{\sphinxupquote{float}}) \textendash{} the maximum duration {[}hours{]}

\end{itemize}

\item[{Returns}] \leavevmode
the minimum end time, the maximum end time

\item[{Return type}] \leavevmode
float, float

\end{description}\end{quote}

\end{fulllineitems}


\end{fulllineitems}



\subsection{chad\_demography\_adult\_non\_work module}
\label{\detokenize{chad_demography_adult_non_work::doc}}\label{\detokenize{chad_demography_adult_non_work:chad-demography-adult-non-work-module}}\label{\detokenize{chad_demography_adult_non_work:module-chad_demography_adult_non_work}}\index{chad\_demography\_adult\_non\_work (module)}
This module contains code that handles accessing the Consolidated Human Activity Database (CHAD)
data for the non-working adult demographic.

This module contains {\hyperref[\detokenize{chad_demography_adult_non_work:chad_demography_adult_non_work.CHAD_demography_adult_non_work}]{\sphinxcrossref{\sphinxcode{\sphinxupquote{chad\_demography\_adult\_non\_work.CHAD\_demography\_adult\_non\_work}}}}}.
\index{CHAD\_demography\_adult\_non\_work (class in chad\_demography\_adult\_non\_work)}

\begin{fulllineitems}
\phantomsection\label{\detokenize{chad_demography_adult_non_work:chad_demography_adult_non_work.CHAD_demography_adult_non_work}}\pysigline{\sphinxbfcode{\sphinxupquote{class }}\sphinxcode{\sphinxupquote{chad\_demography\_adult\_non\_work.}}\sphinxbfcode{\sphinxupquote{CHAD\_demography\_adult\_non\_work}}}
Bases: {\hyperref[\detokenize{chad_demography:chad_demography.CHAD_demography}]{\sphinxcrossref{\sphinxcode{\sphinxupquote{chad\_demography.CHAD\_demography}}}}}

This class contains the common functionality with accessing the CHAD data files
relevant to non-working adult demographic.
\begin{quote}\begin{description}
\item[{Variables}] \leavevmode\begin{itemize}
\item {} 
\sphinxstyleliteralstrong{\sphinxupquote{keys}} \textendash{} the ABMHAP activity codes for the activities simulated by the non-working adult demographic

\item {} 
\sphinxstyleliteralstrong{\sphinxupquote{fname\_stats}} (\sphinxstyleliteralemphasis{\sphinxupquote{dict}}) \textendash{} for a given ABMHAP activity code, access the file names for CHAD longitudinal data for     the respective activity

\item {} 
\sphinxstyleliteralstrong{\sphinxupquote{eat\_breakfast}} ({\hyperref[\detokenize{chad_params:chad_params.CHAD_params}]{\sphinxcrossref{\sphinxstyleliteralemphasis{\sphinxupquote{chad\_params.CHAD\_params}}}}}) \textendash{} sampling parameters for the eating breakfast activity within CHAD

\item {} 
\sphinxstyleliteralstrong{\sphinxupquote{eat\_dinner}} ({\hyperref[\detokenize{chad_params:chad_params.CHAD_params}]{\sphinxcrossref{\sphinxstyleliteralemphasis{\sphinxupquote{chad\_params.CHAD\_params}}}}}) \textendash{} sampling parameters for the eating dinner activity within CHAD

\item {} 
\sphinxstyleliteralstrong{\sphinxupquote{eat\_lunch}} ({\hyperref[\detokenize{chad_params:chad_params.CHAD_params}]{\sphinxcrossref{\sphinxstyleliteralemphasis{\sphinxupquote{chad\_params.CHAD\_params}}}}}) \textendash{} sampling parameters for the eating lunch activity within CHAD

\item {} 
\sphinxstyleliteralstrong{\sphinxupquote{'sleep'}} ({\hyperref[\detokenize{chad_params:chad_params.CHAD_params}]{\sphinxcrossref{\sphinxstyleliteralemphasis{\sphinxupquote{chad\_params.CHAD\_params}}}}}) \textendash{} CHAD sampling parameters for the sleep activity within CHAD

\item {} 
\sphinxstyleliteralstrong{\sphinxupquote{int\_2\_param}} (\sphinxstyleliteralemphasis{\sphinxupquote{dict}}) \textendash{} for a given activity code, choose the proper sampling parameters for the respective activity

\end{itemize}

\end{description}\end{quote}

\end{fulllineitems}



\subsection{chad\_demography\_adult\_work module}
\label{\detokenize{chad_demography_adult_work::doc}}\label{\detokenize{chad_demography_adult_work:chad-demography-adult-work-module}}\label{\detokenize{chad_demography_adult_work:module-chad_demography_adult_work}}\index{chad\_demography\_adult\_work (module)}
This module contains code that handles accessing the Consolidated Human Activity Database (CHAD)
data for the working adult demographic.

This module contains {\hyperref[\detokenize{chad_demography_adult_work:chad_demography_adult_work.CHAD_demography_adult_work}]{\sphinxcrossref{\sphinxcode{\sphinxupquote{chad\_demography\_adult\_work.CHAD\_demography\_adult\_work}}}}}.
\index{CHAD\_demography\_adult\_work (class in chad\_demography\_adult\_work)}

\begin{fulllineitems}
\phantomsection\label{\detokenize{chad_demography_adult_work:chad_demography_adult_work.CHAD_demography_adult_work}}\pysigline{\sphinxbfcode{\sphinxupquote{class }}\sphinxcode{\sphinxupquote{chad\_demography\_adult\_work.}}\sphinxbfcode{\sphinxupquote{CHAD\_demography\_adult\_work}}}
Bases: {\hyperref[\detokenize{chad_demography:chad_demography.CHAD_demography}]{\sphinxcrossref{\sphinxcode{\sphinxupquote{chad\_demography.CHAD\_demography}}}}}

This class contains the common functionality with accessing the CHAD data files
relevant to working adult demographic.
\begin{quote}\begin{description}
\item[{Variables}] \leavevmode\begin{itemize}
\item {} 
\sphinxstyleliteralstrong{\sphinxupquote{keys}} \textendash{} the ABMHAP activity codes for the activities simulated by the working adult demographic

\item {} 
\sphinxstyleliteralstrong{\sphinxupquote{fname\_stats}} (\sphinxstyleliteralemphasis{\sphinxupquote{dict}}) \textendash{} for a given ABMHAP activity code, access the file names for CHAD longitudinal data for     the respective activity

\item {} 
\sphinxstyleliteralstrong{\sphinxupquote{commute\_to\_work}} ({\hyperref[\detokenize{chad_params:chad_params.CHAD_params}]{\sphinxcrossref{\sphinxstyleliteralemphasis{\sphinxupquote{chad\_params.CHAD\_params}}}}}) \textendash{} sampling parameters for the commuting to work activity within CHAD

\item {} 
\sphinxstyleliteralstrong{\sphinxupquote{commute\_from\_work}} ({\hyperref[\detokenize{chad_params:chad_params.CHAD_params}]{\sphinxcrossref{\sphinxstyleliteralemphasis{\sphinxupquote{chad\_params.CHAD\_params}}}}}) \textendash{} sampling parameters for commuting from work activity within CHAD

\item {} 
\sphinxstyleliteralstrong{\sphinxupquote{'work'}} ({\hyperref[\detokenize{chad_params:chad_params.CHAD_params}]{\sphinxcrossref{\sphinxstyleliteralemphasis{\sphinxupquote{chad\_params.CHAD\_params}}}}}) \textendash{} sampling parameters for working activity within CHAD

\item {} 
\sphinxstyleliteralstrong{\sphinxupquote{eat\_breakfast}} ({\hyperref[\detokenize{chad_params:chad_params.CHAD_params}]{\sphinxcrossref{\sphinxstyleliteralemphasis{\sphinxupquote{chad\_params.CHAD\_params}}}}}) \textendash{} sampling parameters for the eating breakfast activity within CHAD

\item {} 
\sphinxstyleliteralstrong{\sphinxupquote{eat\_dinner}} ({\hyperref[\detokenize{chad_params:chad_params.CHAD_params}]{\sphinxcrossref{\sphinxstyleliteralemphasis{\sphinxupquote{chad\_params.CHAD\_params}}}}}) \textendash{} sampling parameters for the eating dinner activity within CHAD

\item {} 
\sphinxstyleliteralstrong{\sphinxupquote{eat\_lunch}} ({\hyperref[\detokenize{chad_params:chad_params.CHAD_params}]{\sphinxcrossref{\sphinxstyleliteralemphasis{\sphinxupquote{chad\_params.CHAD\_params}}}}}) \textendash{} sampling parameters for the eating lunch activity within CHAD

\item {} 
\sphinxstyleliteralstrong{\sphinxupquote{'sleep'}} ({\hyperref[\detokenize{chad_params:chad_params.CHAD_params}]{\sphinxcrossref{\sphinxstyleliteralemphasis{\sphinxupquote{chad\_params.CHAD\_params}}}}}) \textendash{} CHAD sampling parameters for the sleep activity within CHAD

\item {} 
\sphinxstyleliteralstrong{\sphinxupquote{int\_2\_param}} (\sphinxstyleliteralemphasis{\sphinxupquote{dict}}) \textendash{} for a given activity code, choose the proper sampling parameters for the respective activity

\end{itemize}

\end{description}\end{quote}

\end{fulllineitems}



\subsection{chad\_demography\_child\_school module}
\label{\detokenize{chad_demography_child_school::doc}}\label{\detokenize{chad_demography_child_school:chad-demography-child-school-module}}\label{\detokenize{chad_demography_child_school:module-chad_demography_child_school}}\index{chad\_demography\_child\_school (module)}
This module contains code that handles accessing the Consolidated Human Activity Database (CHAD)
data for the school-age children demographic.

This module contains {\hyperref[\detokenize{chad_demography_child_school:chad_demography_child_school.CHAD_demography_child_school}]{\sphinxcrossref{\sphinxcode{\sphinxupquote{chad\_demography\_child\_school.CHAD\_demography\_child\_school}}}}}.
\index{CHAD\_demography\_child\_school (class in chad\_demography\_child\_school)}

\begin{fulllineitems}
\phantomsection\label{\detokenize{chad_demography_child_school:chad_demography_child_school.CHAD_demography_child_school}}\pysigline{\sphinxbfcode{\sphinxupquote{class }}\sphinxcode{\sphinxupquote{chad\_demography\_child\_school.}}\sphinxbfcode{\sphinxupquote{CHAD\_demography\_child\_school}}}
Bases: {\hyperref[\detokenize{chad_demography:chad_demography.CHAD_demography}]{\sphinxcrossref{\sphinxcode{\sphinxupquote{chad\_demography.CHAD\_demography}}}}}

This class contains the common functionality with accessing the CHAD data files
relevant to school-age children demographic.
\begin{quote}\begin{description}
\item[{Variables}] \leavevmode\begin{itemize}
\item {} 
\sphinxstyleliteralstrong{\sphinxupquote{keys}} \textendash{} the ABMHAP activity codes for the activities simulated by the school-age children demographic

\item {} 
\sphinxstyleliteralstrong{\sphinxupquote{fname\_stats}} (\sphinxstyleliteralemphasis{\sphinxupquote{dict}}) \textendash{} for a given ABMHAP activity code, access the file names for CHAD longitudinal data for     the respective activity

\item {} 
\sphinxstyleliteralstrong{\sphinxupquote{commute\_to\_work}} ({\hyperref[\detokenize{chad_params:chad_params.CHAD_params}]{\sphinxcrossref{\sphinxstyleliteralemphasis{\sphinxupquote{chad\_params.CHAD\_params}}}}}) \textendash{} sampling parameters for the commuting to work activity within CHAD

\item {} 
\sphinxstyleliteralstrong{\sphinxupquote{commute\_from\_work}} ({\hyperref[\detokenize{chad_params:chad_params.CHAD_params}]{\sphinxcrossref{\sphinxstyleliteralemphasis{\sphinxupquote{chad\_params.CHAD\_params}}}}}) \textendash{} sampling parameters for commuting from work activity within CHAD

\item {} 
\sphinxstyleliteralstrong{\sphinxupquote{'work'}} ({\hyperref[\detokenize{chad_params:chad_params.CHAD_params}]{\sphinxcrossref{\sphinxstyleliteralemphasis{\sphinxupquote{chad\_params.CHAD\_params}}}}}) \textendash{} sampling parameters for schooling activity within CHAD

\item {} 
\sphinxstyleliteralstrong{\sphinxupquote{eat\_breakfast}} ({\hyperref[\detokenize{chad_params:chad_params.CHAD_params}]{\sphinxcrossref{\sphinxstyleliteralemphasis{\sphinxupquote{chad\_params.CHAD\_params}}}}}) \textendash{} sampling parameters for the eating breakfast activity within CHAD

\item {} 
\sphinxstyleliteralstrong{\sphinxupquote{eat\_dinner}} ({\hyperref[\detokenize{chad_params:chad_params.CHAD_params}]{\sphinxcrossref{\sphinxstyleliteralemphasis{\sphinxupquote{chad\_params.CHAD\_params}}}}}) \textendash{} sampling parameters for the eating dinner activity within CHAD

\item {} 
\sphinxstyleliteralstrong{\sphinxupquote{eat\_lunch}} ({\hyperref[\detokenize{chad_params:chad_params.CHAD_params}]{\sphinxcrossref{\sphinxstyleliteralemphasis{\sphinxupquote{chad\_params.CHAD\_params}}}}}) \textendash{} sampling parameters for the eating lunch activity within CHAD

\item {} 
\sphinxstyleliteralstrong{\sphinxupquote{'sleep'}} ({\hyperref[\detokenize{chad_params:chad_params.CHAD_params}]{\sphinxcrossref{\sphinxstyleliteralemphasis{\sphinxupquote{chad\_params.CHAD\_params}}}}}) \textendash{} CHAD sampling parameters for the sleep activity within CHAD

\item {} 
\sphinxstyleliteralstrong{\sphinxupquote{int\_2\_param}} (\sphinxstyleliteralemphasis{\sphinxupquote{dict}}) \textendash{} for a given activity code, choose the proper sampling parameters for the respective activity

\end{itemize}

\end{description}\end{quote}

\end{fulllineitems}



\subsection{chad\_demography\_child\_young module}
\label{\detokenize{chad_demography_child_young::doc}}\label{\detokenize{chad_demography_child_young:chad-demography-child-young-module}}\label{\detokenize{chad_demography_child_young:module-chad_demography_child_young}}\index{chad\_demography\_child\_young (module)}
This module contains code that handles accessing the Consolidated Human Activity Database (CHAD)
data for the pre-school children demographic.

This module contains {\hyperref[\detokenize{chad_demography_child_young:chad_demography_child_young.CHAD_demography_child_young}]{\sphinxcrossref{\sphinxcode{\sphinxupquote{chad\_demography\_child\_young.CHAD\_demography\_child\_young}}}}}.
\index{CHAD\_demography\_child\_young (class in chad\_demography\_child\_young)}

\begin{fulllineitems}
\phantomsection\label{\detokenize{chad_demography_child_young:chad_demography_child_young.CHAD_demography_child_young}}\pysigline{\sphinxbfcode{\sphinxupquote{class }}\sphinxcode{\sphinxupquote{chad\_demography\_child\_young.}}\sphinxbfcode{\sphinxupquote{CHAD\_demography\_child\_young}}}
Bases: {\hyperref[\detokenize{chad_demography:chad_demography.CHAD_demography}]{\sphinxcrossref{\sphinxcode{\sphinxupquote{chad\_demography.CHAD\_demography}}}}}

This class contains the common functionality with accessing the CHAD data files
relevant to preschool chidlren  demographic.
\begin{quote}\begin{description}
\item[{Variables}] \leavevmode\begin{itemize}
\item {} 
\sphinxstyleliteralstrong{\sphinxupquote{keys}} \textendash{} the ABMHAP activity codes for the activities simulated by the preschool children demographic

\item {} 
\sphinxstyleliteralstrong{\sphinxupquote{fname\_stats}} (\sphinxstyleliteralemphasis{\sphinxupquote{dict}}) \textendash{} for a given ABMHAP activity code, access the file names for CHAD longitudinal data for     the respective activity

\item {} 
\sphinxstyleliteralstrong{\sphinxupquote{eat\_breakfast}} ({\hyperref[\detokenize{chad_params:chad_params.CHAD_params}]{\sphinxcrossref{\sphinxstyleliteralemphasis{\sphinxupquote{chad\_params.CHAD\_params}}}}}) \textendash{} sampling parameters for the eating breakfast activity within CHAD

\item {} 
\sphinxstyleliteralstrong{\sphinxupquote{eat\_dinner}} ({\hyperref[\detokenize{chad_params:chad_params.CHAD_params}]{\sphinxcrossref{\sphinxstyleliteralemphasis{\sphinxupquote{chad\_params.CHAD\_params}}}}}) \textendash{} sampling parameters for the eating dinner activity within CHAD

\item {} 
\sphinxstyleliteralstrong{\sphinxupquote{eat\_lunch}} ({\hyperref[\detokenize{chad_params:chad_params.CHAD_params}]{\sphinxcrossref{\sphinxstyleliteralemphasis{\sphinxupquote{chad\_params.CHAD\_params}}}}}) \textendash{} sampling parameters for the eating lunch activity within CHAD

\item {} 
\sphinxstyleliteralstrong{\sphinxupquote{'sleep'}} ({\hyperref[\detokenize{chad_params:chad_params.CHAD_params}]{\sphinxcrossref{\sphinxstyleliteralemphasis{\sphinxupquote{chad\_params.CHAD\_params}}}}}) \textendash{} CHAD sampling parameters for the sleep activity within CHAD

\item {} 
\sphinxstyleliteralstrong{\sphinxupquote{int\_2\_param}} (\sphinxstyleliteralemphasis{\sphinxupquote{dict}}) \textendash{} for a given activity code, choose the proper sampling parameters for the respective activity

\end{itemize}

\end{description}\end{quote}

\end{fulllineitems}



\subsection{chad\_parameter\_figures notebook}
\label{\detokenize{chad_parameter_figures::doc}}\label{\detokenize{chad_parameter_figures:chad-parameter-figures-notebook}}
\fvset{hllines={, ,}}%
\begin{sphinxVerbatim}[commandchars=\\\{\}]
\PYG{c+c1}{\PYGZsh{} The United States Environmental Protection Agency through its Office of}
\PYG{c+c1}{\PYGZsh{} Research and Development has developed this software. The code is made}
\PYG{c+c1}{\PYGZsh{} publicly available to better communicate the research. All input data}
\PYG{c+c1}{\PYGZsh{} used fora given application should be reviewed by the researcher so}
\PYG{c+c1}{\PYGZsh{} that the model results are based on appropriate data for any given}
\PYG{c+c1}{\PYGZsh{} application. This model is under continued development. The model and}
\PYG{c+c1}{\PYGZsh{} data included herein do not represent and should not be construed to}
\PYG{c+c1}{\PYGZsh{} represent any Agency determination or policy.}
\PYG{c+c1}{\PYGZsh{}}
\PYG{c+c1}{\PYGZsh{} This file was written by Dr. Namdi Brandon}
\PYG{c+c1}{\PYGZsh{} ORCID: 0000\PYGZhy{}0001\PYGZhy{}7050\PYGZhy{}1538}
\PYG{c+c1}{\PYGZsh{} March 20, 2018}
\end{sphinxVerbatim}

WARNING:

this code may not be useful

This code plots the histograms of the distributions being sampled from
the CHAD data for each activity.

Import

\fvset{hllines={, ,}}%
\begin{sphinxVerbatim}[commandchars=\\\{\}]
\PYG{k+kn}{import} \PYG{n+nn}{sys}
\PYG{n}{sys}\PYG{o}{.}\PYG{n}{path}\PYG{o}{.}\PYG{n}{append}\PYG{p}{(}\PYG{l+s+s1}{\PYGZsq{}}\PYG{l+s+s1}{..}\PYG{l+s+se}{\PYGZbs{}\PYGZbs{}}\PYG{l+s+s1}{source}\PYG{l+s+s1}{\PYGZsq{}}\PYG{p}{)}
\PYG{n}{sys}\PYG{o}{.}\PYG{n}{path}\PYG{o}{.}\PYG{n}{append}\PYG{p}{(}\PYG{l+s+s1}{\PYGZsq{}}\PYG{l+s+s1}{..}\PYG{l+s+se}{\PYGZbs{}\PYGZbs{}}\PYG{l+s+s1}{processing}\PYG{l+s+s1}{\PYGZsq{}}\PYG{p}{)}

\PYG{c+c1}{\PYGZsh{} plotting capability}
\PYG{k+kn}{import} \PYG{n+nn}{matplotlib}\PYG{n+nn}{.}\PYG{n+nn}{pylab} \PYG{k}{as} \PYG{n+nn}{plt}

\PYG{c+c1}{\PYGZsh{} zipfile capability}
\PYG{k+kn}{import} \PYG{n+nn}{zipfile}

\PYG{c+c1}{\PYGZsh{} ABMHAP modules}

\PYG{c+c1}{\PYGZsh{} general capability}
\PYG{k+kn}{import} \PYG{n+nn}{my\PYGZus{}globals} \PYG{k}{as} \PYG{n+nn}{mg}
\PYG{k+kn}{import} \PYG{n+nn}{chad\PYGZus{}params} \PYG{k}{as} \PYG{n+nn}{cp}
\PYG{k+kn}{import} \PYG{n+nn}{demography} \PYG{k}{as} \PYG{n+nn}{dmg}

\PYG{k+kn}{import} \PYG{n+nn}{activity}\PYG{o}{,} \PYG{n+nn}{analysis}\PYG{o}{,} \PYG{n+nn}{chad}\PYG{o}{,} \PYG{n+nn}{omni\PYGZus{}trial}\PYG{o}{,} \PYG{n+nn}{params}
\end{sphinxVerbatim}

Run

\fvset{hllines={, ,}}%
\begin{sphinxVerbatim}[commandchars=\\\{\}]
\PYG{c+c1}{\PYGZsh{} the demographic}
\PYG{n}{demo} \PYG{o}{=} \PYG{n}{dmg}\PYG{o}{.}\PYG{n}{ADULT\PYGZus{}WORK}

\PYG{c+c1}{\PYGZsh{} sets of activities}
\PYG{n}{keys\PYGZus{}all} \PYG{o}{=} \PYG{n}{mg}\PYG{o}{.}\PYG{n}{KEYS\PYGZus{}ACTIVITIES}

\PYG{c+c1}{\PYGZsh{} the activity codes related to not eating}
\PYG{n}{keys\PYGZus{}not\PYGZus{}eat} \PYG{o}{=} \PYG{p}{[}\PYG{n}{mg}\PYG{o}{.}\PYG{n}{KEY\PYGZus{}SLEEP}\PYG{p}{,} \PYG{n}{mg}\PYG{o}{.}\PYG{n}{KEY\PYGZus{}WORK}\PYG{p}{,} \PYG{n}{mg}\PYG{o}{.}\PYG{n}{KEY\PYGZus{}COMMUTE\PYGZus{}TO\PYGZus{}WORK}\PYG{p}{,} \PYG{n}{mg}\PYG{o}{.}\PYG{n}{KEY\PYGZus{}COMMUTE\PYGZus{}FROM\PYGZus{}WORK}\PYG{p}{]}

\PYG{c+c1}{\PYGZsh{} the activity codes of the eating activities}
\PYG{n}{keys\PYGZus{}eat} \PYG{o}{=} \PYG{p}{[}\PYG{n}{mg}\PYG{o}{.}\PYG{n}{KEY\PYGZus{}EAT\PYGZus{}BREAKFAST}\PYG{p}{,} \PYG{n}{mg}\PYG{o}{.}\PYG{n}{KEY\PYGZus{}EAT\PYGZus{}LUNCH}\PYG{p}{,} \PYG{n}{mg}\PYG{o}{.}\PYG{n}{KEY\PYGZus{}EAT\PYGZus{}DINNER}\PYG{p}{]}

\PYG{c+c1}{\PYGZsh{} the chosen group of activities}
\PYG{n}{keys} \PYG{o}{=} \PYG{n}{keys\PYGZus{}all}
\end{sphinxVerbatim}

Loop through each activity and plot the histograms of start time, end
time, and duration. Note: the limitations for each activity depends on
which activity parameters are being sampled

\fvset{hllines={, ,}}%
\begin{sphinxVerbatim}[commandchars=\\\{\}]
\PYG{c+c1}{\PYGZsh{} loop through each activity and plot the histograms of start time, end time, and duration}
\PYG{c+c1}{\PYGZsh{} Note: the limitations for each activity depends on which activity parameters are being sampled}
\PYG{k}{for} \PYG{n}{k} \PYG{o+ow}{in} \PYG{n}{keys}\PYG{p}{:}

    \PYG{c+c1}{\PYGZsh{} the CHAD limiting parameters}
    \PYG{n}{s\PYGZus{}params} \PYG{o}{=} \PYG{n}{cp}\PYG{o}{.}\PYG{n}{OMNI}\PYG{p}{[}\PYG{n}{k}\PYG{p}{]}

    \PYG{c+c1}{\PYGZsh{} get the data}
    \PYG{n}{stats\PYGZus{}start}\PYG{p}{,} \PYG{n}{stats\PYGZus{}end}\PYG{p}{,} \PYG{n}{stats\PYGZus{}dt}\PYG{p}{,} \PYG{n}{record} \PYG{o}{=} \PYG{n}{analysis}\PYG{o}{.}\PYG{n}{get\PYGZus{}verification\PYGZus{}info}\PYG{p}{(}\PYG{n}{demo}\PYG{o}{=}\PYG{n}{demo}\PYG{p}{,} \PYG{n}{key\PYGZus{}activity}\PYG{o}{=}\PYG{n}{k}\PYG{p}{,}
                                                     \PYG{n}{sampling\PYGZus{}params}\PYG{o}{=}\PYG{p}{[}\PYG{n}{s\PYGZus{}params}\PYG{p}{]}\PYG{p}{)}
    \PYG{c+c1}{\PYGZsh{} number of the bins}
    \PYG{n}{num\PYGZus{}bins} \PYG{o}{=} \PYG{l+m+mi}{24}

    \PYG{c+c1}{\PYGZsh{} create subplots}
    \PYG{n}{fig}\PYG{p}{,} \PYG{n}{axes} \PYG{o}{=} \PYG{n}{plt}\PYG{o}{.}\PYG{n}{subplots}\PYG{p}{(}\PYG{l+m+mi}{2}\PYG{p}{,} \PYG{l+m+mi}{2}\PYG{p}{)}

    \PYG{c+c1}{\PYGZsh{} title}
    \PYG{n}{fig}\PYG{o}{.}\PYG{n}{suptitle}\PYG{p}{(} \PYG{n}{activity}\PYG{o}{.}\PYG{n}{INT\PYGZus{}2\PYGZus{}STR}\PYG{p}{[}\PYG{n}{k}\PYG{p}{]} \PYG{p}{)}

    \PYG{c+c1}{\PYGZsh{}}
    \PYG{c+c1}{\PYGZsh{} plot the mean start time distribution}
    \PYG{c+c1}{\PYGZsh{}}
    \PYG{n}{ax} \PYG{o}{=} \PYG{n}{axes}\PYG{p}{[}\PYG{l+m+mi}{0}\PYG{p}{,} \PYG{l+m+mi}{0}\PYG{p}{]}
    \PYG{k}{if} \PYG{n}{k} \PYG{o}{==} \PYG{n}{mg}\PYG{o}{.}\PYG{n}{KEY\PYGZus{}SLEEP}\PYG{p}{:}
        \PYG{n}{ax}\PYG{o}{.}\PYG{n}{hist}\PYG{p}{(}\PYG{n}{mg}\PYG{o}{.}\PYG{n}{to\PYGZus{}periodic}\PYG{p}{(}\PYG{n}{stats\PYGZus{}start}\PYG{o}{.}\PYG{n}{mu}\PYG{o}{.}\PYG{n}{values}\PYG{p}{,} \PYG{n}{do\PYGZus{}hours}\PYG{o}{=}\PYG{k+kc}{True}\PYG{p}{)}\PYG{p}{,} \PYG{n}{bins}\PYG{o}{=}\PYG{n}{num\PYGZus{}bins}\PYG{p}{,} \PYG{n}{color}\PYG{o}{=}\PYG{l+s+s1}{\PYGZsq{}}\PYG{l+s+s1}{blue}\PYG{l+s+s1}{\PYGZsq{}}\PYG{p}{,} \PYG{n}{label}\PYG{o}{=}\PYG{l+s+s1}{\PYGZsq{}}\PYG{l+s+s1}{start}\PYG{l+s+s1}{\PYGZsq{}}\PYG{p}{)}
    \PYG{k}{else}\PYG{p}{:}
        \PYG{n}{ax}\PYG{o}{.}\PYG{n}{hist}\PYG{p}{(}\PYG{n}{stats\PYGZus{}start}\PYG{o}{.}\PYG{n}{mu}\PYG{o}{.}\PYG{n}{values}\PYG{p}{,} \PYG{n}{bins}\PYG{o}{=}\PYG{n}{num\PYGZus{}bins}\PYG{p}{,} \PYG{n}{color}\PYG{o}{=}\PYG{l+s+s1}{\PYGZsq{}}\PYG{l+s+s1}{blue}\PYG{l+s+s1}{\PYGZsq{}}\PYG{p}{,} \PYG{n}{label}\PYG{o}{=}\PYG{l+s+s1}{\PYGZsq{}}\PYG{l+s+s1}{start}\PYG{l+s+s1}{\PYGZsq{}}\PYG{p}{)}
    \PYG{n}{ax}\PYG{o}{.}\PYG{n}{set\PYGZus{}xlabel}\PYG{p}{(}\PYG{l+s+s1}{\PYGZsq{}}\PYG{l+s+s1}{hours}\PYG{l+s+s1}{\PYGZsq{}}\PYG{p}{)}
    \PYG{n}{ax}\PYG{o}{.}\PYG{n}{legend}\PYG{p}{(}\PYG{n}{loc}\PYG{o}{=}\PYG{l+s+s1}{\PYGZsq{}}\PYG{l+s+s1}{best}\PYG{l+s+s1}{\PYGZsq{}}\PYG{p}{)}

    \PYG{c+c1}{\PYGZsh{}}
    \PYG{c+c1}{\PYGZsh{} plot the mean end time distribution}
    \PYG{c+c1}{\PYGZsh{}}
    \PYG{n}{ax} \PYG{o}{=} \PYG{n}{axes}\PYG{p}{[}\PYG{l+m+mi}{0}\PYG{p}{,} \PYG{l+m+mi}{1}\PYG{p}{]}
    \PYG{n}{ax}\PYG{o}{.}\PYG{n}{hist}\PYG{p}{(}\PYG{n}{stats\PYGZus{}end}\PYG{o}{.}\PYG{n}{mu}\PYG{o}{.}\PYG{n}{values}\PYG{p}{,} \PYG{n}{bins}\PYG{o}{=}\PYG{n}{num\PYGZus{}bins}\PYG{p}{,} \PYG{n}{color}\PYG{o}{=}\PYG{l+s+s1}{\PYGZsq{}}\PYG{l+s+s1}{green}\PYG{l+s+s1}{\PYGZsq{}}\PYG{p}{,} \PYG{n}{label}\PYG{o}{=}\PYG{l+s+s1}{\PYGZsq{}}\PYG{l+s+s1}{end}\PYG{l+s+s1}{\PYGZsq{}}\PYG{p}{)}
    \PYG{n}{ax}\PYG{o}{.}\PYG{n}{set\PYGZus{}xlabel}\PYG{p}{(}\PYG{l+s+s1}{\PYGZsq{}}\PYG{l+s+s1}{hours}\PYG{l+s+s1}{\PYGZsq{}}\PYG{p}{)}
    \PYG{n}{ax}\PYG{o}{.}\PYG{n}{legend}\PYG{p}{(}\PYG{n}{loc}\PYG{o}{=}\PYG{l+s+s1}{\PYGZsq{}}\PYG{l+s+s1}{best}\PYG{l+s+s1}{\PYGZsq{}}\PYG{p}{)}

    \PYG{c+c1}{\PYGZsh{}}
    \PYG{c+c1}{\PYGZsh{} plot the mean duration distribution}
    \PYG{c+c1}{\PYGZsh{}}
    \PYG{n}{ax} \PYG{o}{=} \PYG{n}{axes}\PYG{p}{[}\PYG{l+m+mi}{1}\PYG{p}{,} \PYG{l+m+mi}{0}\PYG{p}{]}
    \PYG{n}{ax}\PYG{o}{.}\PYG{n}{hist}\PYG{p}{(}\PYG{n}{stats\PYGZus{}dt}\PYG{o}{.}\PYG{n}{mu}\PYG{o}{.}\PYG{n}{values}\PYG{p}{,} \PYG{n}{bins}\PYG{o}{=}\PYG{n}{num\PYGZus{}bins}\PYG{p}{,} \PYG{n}{color}\PYG{o}{=}\PYG{l+s+s1}{\PYGZsq{}}\PYG{l+s+s1}{red}\PYG{l+s+s1}{\PYGZsq{}}\PYG{p}{,} \PYG{n}{label}\PYG{o}{=}\PYG{l+s+s1}{\PYGZsq{}}\PYG{l+s+s1}{duration}\PYG{l+s+s1}{\PYGZsq{}}\PYG{p}{)}
    \PYG{n}{ax}\PYG{o}{.}\PYG{n}{set\PYGZus{}xlabel}\PYG{p}{(}\PYG{l+s+s1}{\PYGZsq{}}\PYG{l+s+s1}{hours}\PYG{l+s+s1}{\PYGZsq{}}\PYG{p}{)}
    \PYG{n}{ax}\PYG{o}{.}\PYG{n}{legend}\PYG{p}{(}\PYG{n}{loc}\PYG{o}{=}\PYG{l+s+s1}{\PYGZsq{}}\PYG{l+s+s1}{best}\PYG{l+s+s1}{\PYGZsq{}}\PYG{p}{)}

\PYG{c+c1}{\PYGZsh{} show plots}
\PYG{n}{plt}\PYG{o}{.}\PYG{n}{show}\PYG{p}{(}\PYG{p}{)}
\end{sphinxVerbatim}


\subsection{chad\_params module}
\label{\detokenize{chad_params::doc}}\label{\detokenize{chad_params:chad-params-module}}\label{\detokenize{chad_params:module-chad_params}}\index{chad\_params (module)}
The purpose of this module is to assign parameters necessary to run the ABMHAP initialized with data from the Consolidated Human Activity Database (CHAD).

This module contains {\hyperref[\detokenize{chad_params:chad_params.CHAD_params}]{\sphinxcrossref{\sphinxcode{\sphinxupquote{chad\_params.CHAD\_params}}}}}.
\index{CHAD\_params (class in chad\_params)}

\begin{fulllineitems}
\phantomsection\label{\detokenize{chad_params:chad_params.CHAD_params}}\pysiglinewithargsret{\sphinxbfcode{\sphinxupquote{class }}\sphinxcode{\sphinxupquote{chad\_params.}}\sphinxbfcode{\sphinxupquote{CHAD\_params}}}{\emph{dt\_mean\_min=None}, \emph{dt\_mean\_max=None}, \emph{dt\_std\_max=None}, \emph{start\_mean\_min=None}, \emph{start\_mean\_max=None}, \emph{start\_std\_max=None}, \emph{end\_mean\_min=None}, \emph{end\_mean\_max=None}, \emph{end\_std\_max=None}, \emph{N=1}, \emph{do\_solo=False}, \emph{do\_dt=False}, \emph{do\_start=False}, \emph{do\_end=False}}{}
Bases: \sphinxcode{\sphinxupquote{object}}

This class holds sampling parameters for various activities in CHAD that are used to filter out what is     considered “good” data for a given activity.
\begin{quote}\begin{description}
\item[{Parameters}] \leavevmode\begin{itemize}
\item {} 
\sphinxstyleliteralstrong{\sphinxupquote{dt\_mean\_min}} (\sphinxstyleliteralemphasis{\sphinxupquote{float}}) \textendash{} the minimum mean duration to be sampled in hours {[}0, 24)

\item {} 
\sphinxstyleliteralstrong{\sphinxupquote{dt\_mean\_max}} (\sphinxstyleliteralemphasis{\sphinxupquote{float}}) \textendash{} the maximum mean duration to be sampled in hours {[}0, 24)

\item {} 
\sphinxstyleliteralstrong{\sphinxupquote{dt\_std\_max}} (\sphinxstyleliteralemphasis{\sphinxupquote{float}}) \textendash{} the maximum standard deviation of duration to be sampled in hours {[}0, 24)

\item {} 
\sphinxstyleliteralstrong{\sphinxupquote{start\_mean\_min}} (\sphinxstyleliteralemphasis{\sphinxupquote{float}}) \textendash{} the minimum mean start time to be sampled in hours {[}0, 24)

\item {} 
\sphinxstyleliteralstrong{\sphinxupquote{start\_mean\_max}} (\sphinxstyleliteralemphasis{\sphinxupquote{float}}) \textendash{} the maximum mean start time to be sampled in hours {[}0, 24)

\item {} 
\sphinxstyleliteralstrong{\sphinxupquote{start\_std\_max}} (\sphinxstyleliteralemphasis{\sphinxupquote{float}}) \textendash{} the maximum standard deviation of start time to be sampled in hours {[}0, 24)

\item {} 
\sphinxstyleliteralstrong{\sphinxupquote{end\_mean\_min}} (\sphinxstyleliteralemphasis{\sphinxupquote{float}}) \textendash{} the minimum mean end time to be sampled in hours {[}0, 24)

\item {} 
\sphinxstyleliteralstrong{\sphinxupquote{end\_mean\_max}} (\sphinxstyleliteralemphasis{\sphinxupquote{float}}) \textendash{} the maximum mean end time to be sampled in hours {[}0, 24)

\item {} 
\sphinxstyleliteralstrong{\sphinxupquote{end\_std\_max}} (\sphinxstyleliteralemphasis{\sphinxupquote{float}}) \textendash{} the maximum standard deviation of end time to be sampled in hours {[}0, 24)

\item {} 
\sphinxstyleliteralstrong{\sphinxupquote{N}} (\sphinxstyleliteralemphasis{\sphinxupquote{int}}) \textendash{} the minimum amount of activity-events needed in sampling

\item {} 
\sphinxstyleliteralstrong{\sphinxupquote{do\_solo}} (\sphinxstyleliteralemphasis{\sphinxupquote{bool}}) \textendash{} a flag indicating whether to take single activity-events only

\item {} 
\sphinxstyleliteralstrong{\sphinxupquote{do\_dt}} (\sphinxstyleliteralemphasis{\sphinxupquote{bool}}) \textendash{} a flag indicating whether (if True) or not (if False) to sample duration data from CHAD

\item {} 
\sphinxstyleliteralstrong{\sphinxupquote{do\_start}} (\sphinxstyleliteralemphasis{\sphinxupquote{bool}}) \textendash{} a flag indicating whether (if True) or not (if False) to sample start time data from CHAD

\item {} 
\sphinxstyleliteralstrong{\sphinxupquote{do\_end}} (\sphinxstyleliteralemphasis{\sphinxupquote{bool}}) \textendash{} a flag indicating whether (if True) or not (if False) to sample end time data from CHAD

\end{itemize}

\item[{Variables}] \leavevmode\begin{itemize}
\item {} 
\sphinxstyleliteralstrong{\sphinxupquote{dt\_mean\_min}} (\sphinxstyleliteralemphasis{\sphinxupquote{float}}) \textendash{} the minimum mean duration to be sampled in hours {[}0, 24)

\item {} 
\sphinxstyleliteralstrong{\sphinxupquote{dt\_mean\_max}} (\sphinxstyleliteralemphasis{\sphinxupquote{float}}) \textendash{} the maximum mean duration to be sampled in hours {[}0, 24)

\item {} 
\sphinxstyleliteralstrong{\sphinxupquote{dt\_std\_max}} (\sphinxstyleliteralemphasis{\sphinxupquote{float}}) \textendash{} the maximum standard deviation of  duration to be sampled in hours {[}0, 24)

\item {} 
\sphinxstyleliteralstrong{\sphinxupquote{start\_mean\_min}} (\sphinxstyleliteralemphasis{\sphinxupquote{float}}) \textendash{} the minimum mean start time to be sampled in hours {[}0, 24)

\item {} 
\sphinxstyleliteralstrong{\sphinxupquote{start\_mean\_max}} (\sphinxstyleliteralemphasis{\sphinxupquote{float}}) \textendash{} the maximum mean start time to be sampled in hours {[}0, 24)

\item {} 
\sphinxstyleliteralstrong{\sphinxupquote{start\_std\_max}} (\sphinxstyleliteralemphasis{\sphinxupquote{float}}) \textendash{} the maximum standard deviation of  start time to be sampled in hours {[}0, 24)

\item {} 
\sphinxstyleliteralstrong{\sphinxupquote{end\_mean\_min}} (\sphinxstyleliteralemphasis{\sphinxupquote{float}}) \textendash{} the minimum mean end time to be sampled in hours {[}0, 24)

\item {} 
\sphinxstyleliteralstrong{\sphinxupquote{end\_mean\_max}} (\sphinxstyleliteralemphasis{\sphinxupquote{float}}) \textendash{} the maximum mean end time to be sampled in hours {[}0, 24)

\item {} 
\sphinxstyleliteralstrong{\sphinxupquote{end\_std\_max}} (\sphinxstyleliteralemphasis{\sphinxupquote{float}}) \textendash{} the maximum standard deviation of end time to be sampled in hours {[}0, 24)

\item {} 
\sphinxstyleliteralstrong{\sphinxupquote{N}} (\sphinxstyleliteralemphasis{\sphinxupquote{int}}) \textendash{} the minimum amount of activity-events needed in sampling

\item {} 
\sphinxstyleliteralstrong{\sphinxupquote{do\_solo}} (\sphinxstyleliteralemphasis{\sphinxupquote{bool}}) \textendash{} a flag indicating whether to take single activity-events only

\item {} 
\sphinxstyleliteralstrong{\sphinxupquote{do\_dt}} (\sphinxstyleliteralemphasis{\sphinxupquote{bool}}) \textendash{} a flag indicating whether (if True) or not (if False) to sample duration data from CHAD

\item {} 
\sphinxstyleliteralstrong{\sphinxupquote{do\_start}} (\sphinxstyleliteralemphasis{\sphinxupquote{bool}}) \textendash{} a flag indicating whether (if True) or not (if False) to sample start time data from CHAD

\item {} 
\sphinxstyleliteralstrong{\sphinxupquote{do\_end}} (\sphinxstyleliteralemphasis{\sphinxupquote{bool}}) \textendash{} a flag indicating whether (if True) or not (if False) to sample end time data from CHAD

\end{itemize}

\end{description}\end{quote}
\index{get\_dt() (chad\_params.CHAD\_params method)}

\begin{fulllineitems}
\phantomsection\label{\detokenize{chad_params:chad_params.CHAD_params.get_dt}}\pysiglinewithargsret{\sphinxbfcode{\sphinxupquote{get\_dt}}}{\emph{df\_stats}}{}
This function samples CHAD data for duration.
\begin{quote}\begin{description}
\item[{Parameters}] \leavevmode
\sphinxstyleliteralstrong{\sphinxupquote{df\_stats}} (\sphinxstyleliteralemphasis{\sphinxupquote{pandas.core.frame.DataFrame}}) \textendash{} the duration data from CHAD for a given activity

\item[{Returns}] \leavevmode
the duration data from CHAD that satisfies statistical properties to use in ABMHAP.

\item[{Return type}] \leavevmode
pandas.core.frame.DataFrame

\end{description}\end{quote}

\end{fulllineitems}

\index{get\_end() (chad\_params.CHAD\_params method)}

\begin{fulllineitems}
\phantomsection\label{\detokenize{chad_params:chad_params.CHAD_params.get_end}}\pysiglinewithargsret{\sphinxbfcode{\sphinxupquote{get\_end}}}{\emph{df\_stats}}{}
This function samples CHAD data for end time.
\begin{quote}\begin{description}
\item[{Parameters}] \leavevmode
\sphinxstyleliteralstrong{\sphinxupquote{df\_stats}} (\sphinxstyleliteralemphasis{\sphinxupquote{pandas.core.frame.DataFrame}}) \textendash{} the end time data from CHAD for a given activity

\item[{Returns}] \leavevmode
the end time data from CHAD that satisfies statistical properties to use in ABMHAP.

\item[{Return type}] \leavevmode
pandas.core.frame.DataFrame

\end{description}\end{quote}

\end{fulllineitems}

\index{get\_record() (chad\_params.CHAD\_params method)}

\begin{fulllineitems}
\phantomsection\label{\detokenize{chad_params:chad_params.CHAD_params.get_record}}\pysiglinewithargsret{\sphinxbfcode{\sphinxupquote{get\_record}}}{\emph{df}, \emph{do\_periodic}}{}
Given a data frame of CHAD records, return the results where conditions are met according to the         chad\_param object.
\begin{quote}\begin{description}
\item[{Parameters}] \leavevmode\begin{itemize}
\item {} 
\sphinxstyleliteralstrong{\sphinxupquote{df}} (\sphinxstyleliteralemphasis{\sphinxupquote{pandas.core.frame.DataFrame}}) \textendash{} the CHAD records from participants for a given activity

\item {} 
\sphinxstyleliteralstrong{\sphinxupquote{do\_periodic}} (\sphinxstyleliteralemphasis{\sphinxupquote{bool}}) \textendash{} a flag indicating whether (if True) or not (if False) to convert time to a {[}-12, 12)         format due to an activity that could occur over midnight.

\end{itemize}

\item[{Returns}] \leavevmode
the records from CHAD that satisfy the statistical data for duration, start time, and end time.

\item[{Return type}] \leavevmode
pandas.core.frame.DataFrame

\end{description}\end{quote}

\end{fulllineitems}

\index{get\_record\_help() (chad\_params.CHAD\_params method)}

\begin{fulllineitems}
\phantomsection\label{\detokenize{chad_params:chad_params.CHAD_params.get_record_help}}\pysiglinewithargsret{\sphinxbfcode{\sphinxupquote{get\_record\_help}}}{\emph{x}, \emph{lower}, \emph{upper}, \emph{do\_periodic}}{}
This function finds the boolean indices of acceptable entries from an activity-parameter within         the CHAD data.
\begin{quote}\begin{description}
\item[{Parameters}] \leavevmode\begin{itemize}
\item {} 
\sphinxstyleliteralstrong{\sphinxupquote{x}} (\sphinxstyleliteralemphasis{\sphinxupquote{numpy.ndarray}}) \textendash{} data for a given activity-parameter (i.e., duration, start time, or end time)

\item {} 
\sphinxstyleliteralstrong{\sphinxupquote{lower}} (\sphinxstyleliteralemphasis{\sphinxupquote{float}}) \textendash{} the lower bound of acceptable values

\item {} 
\sphinxstyleliteralstrong{\sphinxupquote{upper}} (\sphinxstyleliteralemphasis{\sphinxupquote{float}}) \textendash{} the upper bound of acceptable values

\item {} 
\sphinxstyleliteralstrong{\sphinxupquote{do\_periodic}} (\sphinxstyleliteralemphasis{\sphinxupquote{bool}}) \textendash{} a flag indicating whether (if True) or not (if False) to convert time to a {[}-12, 12)         format due to an activity that could occur over midnight.

\end{itemize}

\item[{Returns}] \leavevmode
boolean indices of acceptable values, respectively

\item[{Return type}] \leavevmode
numpy.ndarray of int

\end{description}\end{quote}

\end{fulllineitems}

\index{get\_start() (chad\_params.CHAD\_params method)}

\begin{fulllineitems}
\phantomsection\label{\detokenize{chad_params:chad_params.CHAD_params.get_start}}\pysiglinewithargsret{\sphinxbfcode{\sphinxupquote{get\_start}}}{\emph{df\_stats}}{}
”
This function samples CHAD data for start time.
\begin{quote}\begin{description}
\item[{Parameters}] \leavevmode
\sphinxstyleliteralstrong{\sphinxupquote{df\_stats}} (\sphinxstyleliteralemphasis{\sphinxupquote{pandas.core.frame.DataFrame}}) \textendash{} the start time data from CHAD for a given activity

\item[{Returns}] \leavevmode
the start time data from CHAD that satisfies statistical properties to use in ABMHAP.

\item[{Return type}] \leavevmode
pandas.core.frame.DataFrame

\end{description}\end{quote}

\end{fulllineitems}

\index{get\_stats() (chad\_params.CHAD\_params method)}

\begin{fulllineitems}
\phantomsection\label{\detokenize{chad_params:chad_params.CHAD_params.get_stats}}\pysiglinewithargsret{\sphinxbfcode{\sphinxupquote{get\_stats}}}{\emph{df}, \emph{mean\_min}, \emph{mean\_max}, \emph{std\_max}, \emph{N}}{}
This function samples the CHAD longitudinal data and selects entries with the selected         characteristics: the mean within the given range, within the maximum standard deviation,         and having longitudinal data with at least N entries.
\begin{quote}\begin{description}
\item[{Parameters}] \leavevmode\begin{itemize}
\item {} 
\sphinxstyleliteralstrong{\sphinxupquote{df}} (\sphinxstyleliteralemphasis{\sphinxupquote{pandas.core.frame.DataFrame}}) \textendash{} the duration statistical data for a given activity

\item {} 
\sphinxstyleliteralstrong{\sphinxupquote{mean\_min}} (\sphinxstyleliteralemphasis{\sphinxupquote{float}}) \textendash{} 

\item {} 
\sphinxstyleliteralstrong{\sphinxupquote{mean\_max}} (\sphinxstyleliteralemphasis{\sphinxupquote{float}}) \textendash{} 

\item {} 
\sphinxstyleliteralstrong{\sphinxupquote{std\_max}} (\sphinxstyleliteralemphasis{\sphinxupquote{float}}) \textendash{} 

\item {} 
\sphinxstyleliteralstrong{\sphinxupquote{N}} (\sphinxstyleliteralemphasis{\sphinxupquote{int}}) \textendash{} 

\end{itemize}

\item[{Returns}] \leavevmode
the CHAD data that satisfies the given statistical constraints

\item[{Return type}] \leavevmode
pandas.core.frame.DataFrame

\end{description}\end{quote}

\end{fulllineitems}

\index{toString() (chad\_params.CHAD\_params method)}

\begin{fulllineitems}
\phantomsection\label{\detokenize{chad_params:chad_params.CHAD_params.toString}}\pysiglinewithargsret{\sphinxbfcode{\sphinxupquote{toString}}}{}{}
Represent the object as a string.
\begin{quote}\begin{description}
\item[{Returns}] \leavevmode
the representation of the object as a string

\item[{Return type}] \leavevmode
str

\end{description}\end{quote}

\end{fulllineitems}


\end{fulllineitems}



\subsection{commute\_from\_work\_trial module}
\label{\detokenize{commute_from_work_trial::doc}}\label{\detokenize{commute_from_work_trial:module-commute_from_work_trial}}\label{\detokenize{commute_from_work_trial:commute-from-work-trial-module}}\index{commute\_from\_work\_trial (module)}
This module contains code in order to run Monte-Carlo simulations to comparing the Agent-Based Model of Human Activity Patterns (ABMHAP) with the data from the Consolidated Human Activity Database (CHAD) for the \sphinxstylestrong{commute from work} activity.

This module contains class {\hyperref[\detokenize{commute_from_work_trial:commute_from_work_trial.Commute_From_Work_Trial}]{\sphinxcrossref{\sphinxcode{\sphinxupquote{commute\_from\_work\_trial.Commute\_From\_Work\_Trial}}}}}.
\index{Commute\_From\_Work\_Trial (class in commute\_from\_work\_trial)}

\begin{fulllineitems}
\phantomsection\label{\detokenize{commute_from_work_trial:commute_from_work_trial.Commute_From_Work_Trial}}\pysiglinewithargsret{\sphinxbfcode{\sphinxupquote{class }}\sphinxcode{\sphinxupquote{commute\_from\_work\_trial.}}\sphinxbfcode{\sphinxupquote{Commute\_From\_Work\_Trial}}}{\emph{parameters}, \emph{sampling\_params}, \emph{demographic}}{}
Bases: {\hyperref[\detokenize{trial:trial.Trial}]{\sphinxcrossref{\sphinxcode{\sphinxupquote{trial.Trial}}}}}

This class sets up runs for the ABMHAP initialized with data from CHAD to focus on the “commute     from work” activity.
\begin{quote}\begin{description}
\item[{Parameters}] \leavevmode\begin{itemize}
\item {} 
\sphinxstyleliteralstrong{\sphinxupquote{parameters}} ({\hyperref[\detokenize{params:params.Params}]{\sphinxcrossref{\sphinxstyleliteralemphasis{\sphinxupquote{params.Params}}}}}) \textendash{} the parameters that describe the household

\item {} 
\sphinxstyleliteralstrong{\sphinxupquote{sampling\_params}} ({\hyperref[\detokenize{chad_params:chad_params.CHAD_params}]{\sphinxcrossref{\sphinxstyleliteralemphasis{\sphinxupquote{chad\_params.CHAD\_params}}}}}) \textendash{} he sampling parameters used to filter “good” CHAD     commute from work data

\item {} 
\sphinxstyleliteralstrong{\sphinxupquote{demographic}} (\sphinxstyleliteralemphasis{\sphinxupquote{int}}) \textendash{} the demographic identifier

\end{itemize}

\end{description}\end{quote}
\index{adjust\_params() (commute\_from\_work\_trial.Commute\_From\_Work\_Trial method)}

\begin{fulllineitems}
\phantomsection\label{\detokenize{commute_from_work_trial:commute_from_work_trial.Commute_From_Work_Trial.adjust_params}}\pysiglinewithargsret{\sphinxbfcode{\sphinxupquote{adjust\_params}}}{\emph{commute\_dt\_mean}, \emph{commute\_dt\_std}, \emph{work\_start\_mean}, \emph{work\_start\_std}, \emph{work\_end\_mean}, \emph{work\_end\_std}}{}
This function adjusts the values for the mean and standard deviation of both commute from work         duration and start time in the key-word arguments based on the CHAD data         that was sampled. These new values will be used in the runs.
\begin{quote}\begin{description}
\item[{Parameters}] \leavevmode\begin{itemize}
\item {} 
\sphinxstyleliteralstrong{\sphinxupquote{commute\_dt\_mean}} (\sphinxstyleliteralemphasis{\sphinxupquote{numpy.ndarray}}) \textendash{} the commute duration mean {[}hours{]} for each person

\item {} 
\sphinxstyleliteralstrong{\sphinxupquote{commute\_dt\_std}} (\sphinxstyleliteralemphasis{\sphinxupquote{numpy.ndarray}}) \textendash{} the commute duration standard deviation {[}hours{]} for each person

\item {} 
\sphinxstyleliteralstrong{\sphinxupquote{work\_start\_mean}} (\sphinxstyleliteralemphasis{\sphinxupquote{numpy.ndarray}}) \textendash{} the mean work start time {[}hours{]} for each person

\item {} 
\sphinxstyleliteralstrong{\sphinxupquote{work\_start\_std}} (\sphinxstyleliteralemphasis{\sphinxupquote{numpy.ndarray}}) \textendash{} the standard deviation of work start time {[}hours{]} for each person

\item {} 
\sphinxstyleliteralstrong{\sphinxupquote{work\_end\_mean}} (\sphinxstyleliteralemphasis{\sphinxupquote{numpy.ndarray}}) \textendash{} the mean work end time {[}hours{]} for each person

\item {} 
\sphinxstyleliteralstrong{\sphinxupquote{work\_end\_std}} (\sphinxstyleliteralemphasis{\sphinxupquote{numpy.ndarray}}) \textendash{} the standard deviation of work end time {[}hours{]} for each person

\end{itemize}

\item[{Returns}] \leavevmode


\end{description}\end{quote}

\end{fulllineitems}

\index{create\_universe() (commute\_from\_work\_trial.Commute\_From\_Work\_Trial method)}

\begin{fulllineitems}
\phantomsection\label{\detokenize{commute_from_work_trial:commute_from_work_trial.Commute_From_Work_Trial.create_universe}}\pysiglinewithargsret{\sphinxbfcode{\sphinxupquote{create\_universe}}}{}{}
This function creates a universe object that simulations will run in. The only asset in this         simulation for an agent to use is a {\hyperref[\detokenize{transport:transport.Transport}]{\sphinxcrossref{\sphinxcode{\sphinxupquote{transport.Transport}}}}} and {\hyperref[\detokenize{workplace:workplace.Workplace}]{\sphinxcrossref{\sphinxcode{\sphinxupquote{workplace.Workplace}}}}}.
\begin{quote}\begin{description}
\item[{Returns}] \leavevmode
the universe

\item[{Return type}] \leavevmode
{\hyperref[\detokenize{universe:universe.Universe}]{\sphinxcrossref{universe.Universe}}}

\end{description}\end{quote}

\end{fulllineitems}

\index{initialize() (commute\_from\_work\_trial.Commute\_From\_Work\_Trial method)}

\begin{fulllineitems}
\phantomsection\label{\detokenize{commute_from_work_trial:commute_from_work_trial.Commute_From_Work_Trial.initialize}}\pysiglinewithargsret{\sphinxbfcode{\sphinxupquote{initialize}}}{}{}
This function sets up the trial.
\begin{enumerate}
\item {} 
gets the CHAD data for commuting from work under the appropriate conditions

\item {} 
gets N samples the CHAD data for workng and commuting for the N trials

\item {} 
updates the {\hyperref[\detokenize{params:module-params}]{\sphinxcrossref{\sphinxcode{\sphinxupquote{params}}}}} to reflect the newly assigned working and commuting parameters for the simulation

\end{enumerate}
\begin{quote}\begin{description}
\item[{Returns}] \leavevmode


\end{description}\end{quote}

\end{fulllineitems}


\end{fulllineitems}



\subsection{commute\_to\_work\_trial module}
\label{\detokenize{commute_to_work_trial::doc}}\label{\detokenize{commute_to_work_trial:commute-to-work-trial-module}}\label{\detokenize{commute_to_work_trial:module-commute_to_work_trial}}\index{commute\_to\_work\_trial (module)}
This module contains code in order to run Monte-Carlo simulations to comparing the Agent-Based Model of Human Activity Patterns (ABMHAP) with the data from the Consolidated Human Activity Database (CHAD) for the \sphinxstylestrong{commute to work} activity.

This module contains class {\hyperref[\detokenize{commute_to_work_trial:commute_to_work_trial.Commute_To_Work_Trial}]{\sphinxcrossref{\sphinxcode{\sphinxupquote{commute\_to\_work\_trial.Commute\_To\_Work\_Trial}}}}}.
\index{Commute\_To\_Work\_Trial (class in commute\_to\_work\_trial)}

\begin{fulllineitems}
\phantomsection\label{\detokenize{commute_to_work_trial:commute_to_work_trial.Commute_To_Work_Trial}}\pysiglinewithargsret{\sphinxbfcode{\sphinxupquote{class }}\sphinxcode{\sphinxupquote{commute\_to\_work\_trial.}}\sphinxbfcode{\sphinxupquote{Commute\_To\_Work\_Trial}}}{\emph{parameters}, \emph{sampling\_params}, \emph{demographic}}{}
Bases: {\hyperref[\detokenize{trial:trial.Trial}]{\sphinxcrossref{\sphinxcode{\sphinxupquote{trial.Trial}}}}}

This class sets up runs for the ABMHAP initialized with data from CHAD to focus on the “commute     to work” activity.
\begin{quote}\begin{description}
\item[{Parameters}] \leavevmode\begin{itemize}
\item {} 
\sphinxstyleliteralstrong{\sphinxupquote{parameters}} ({\hyperref[\detokenize{params:params.Params}]{\sphinxcrossref{\sphinxstyleliteralemphasis{\sphinxupquote{params.Params}}}}}) \textendash{} the parameters that describe the household

\item {} 
\sphinxstyleliteralstrong{\sphinxupquote{sampling\_params}} ({\hyperref[\detokenize{chad_params:chad_params.CHAD_params}]{\sphinxcrossref{\sphinxstyleliteralemphasis{\sphinxupquote{chad\_params.CHAD\_params}}}}}) \textendash{} he sampling parameters used to filter “good” CHAD     commute to work data

\item {} 
\sphinxstyleliteralstrong{\sphinxupquote{demographic}} (\sphinxstyleliteralemphasis{\sphinxupquote{int}}) \textendash{} the demographic identifier

\end{itemize}

\end{description}\end{quote}
\index{adjust\_params() (commute\_to\_work\_trial.Commute\_To\_Work\_Trial method)}

\begin{fulllineitems}
\phantomsection\label{\detokenize{commute_to_work_trial:commute_to_work_trial.Commute_To_Work_Trial.adjust_params}}\pysiglinewithargsret{\sphinxbfcode{\sphinxupquote{adjust\_params}}}{\emph{commute\_dt\_mean}, \emph{commute\_dt\_std}, \emph{work\_start\_mean}, \emph{work\_start\_std}, \emph{work\_end\_mean}, \emph{work\_end\_std}}{}
This function adjusts the values for the mean and standard deviation of both commute to work         duration and start time in the key-word arguments based on the CHAD data         that was sampled. These new values will be used in the runs.
\begin{quote}\begin{description}
\item[{Parameters}] \leavevmode\begin{itemize}
\item {} 
\sphinxstyleliteralstrong{\sphinxupquote{commute\_dt\_mean}} (\sphinxstyleliteralemphasis{\sphinxupquote{numpy.ndarray}}) \textendash{} the commute duration mean {[}hours{]} for each person

\item {} 
\sphinxstyleliteralstrong{\sphinxupquote{commute\_dt\_std}} (\sphinxstyleliteralemphasis{\sphinxupquote{numpy.ndarray}}) \textendash{} the commute duration standard deviation {[}hours{]} for each person

\item {} 
\sphinxstyleliteralstrong{\sphinxupquote{work\_start\_mean}} (\sphinxstyleliteralemphasis{\sphinxupquote{numpy.ndarray}}) \textendash{} the mean work start time {[}hours{]} for each person

\item {} 
\sphinxstyleliteralstrong{\sphinxupquote{work\_start\_std}} (\sphinxstyleliteralemphasis{\sphinxupquote{numpy.ndarray}}) \textendash{} the standard deviation of work start time {[}hours{]} for each person

\item {} 
\sphinxstyleliteralstrong{\sphinxupquote{work\_end\_mean}} (\sphinxstyleliteralemphasis{\sphinxupquote{numpy.ndarray}}) \textendash{} the mean work end time {[}hours{]} for each person

\item {} 
\sphinxstyleliteralstrong{\sphinxupquote{work\_end\_std}} (\sphinxstyleliteralemphasis{\sphinxupquote{numpy.ndarray}}) \textendash{} the standard deviation of work end time {[}hours{]} for each person

\end{itemize}

\item[{Returns}] \leavevmode


\end{description}\end{quote}

\end{fulllineitems}

\index{create\_universe() (commute\_to\_work\_trial.Commute\_To\_Work\_Trial method)}

\begin{fulllineitems}
\phantomsection\label{\detokenize{commute_to_work_trial:commute_to_work_trial.Commute_To_Work_Trial.create_universe}}\pysiglinewithargsret{\sphinxbfcode{\sphinxupquote{create\_universe}}}{}{}
This function creates a universe object that simulations will run in. The only asset in this         simulation for an agent to use is a {\hyperref[\detokenize{transport:transport.Transport}]{\sphinxcrossref{\sphinxcode{\sphinxupquote{transport.Transport}}}}} and {\hyperref[\detokenize{workplace:workplace.Workplace}]{\sphinxcrossref{\sphinxcode{\sphinxupquote{workplace.Workplace}}}}}.
\begin{quote}\begin{description}
\item[{Returns}] \leavevmode
the universe

\item[{Return type}] \leavevmode
{\hyperref[\detokenize{universe:universe.Universe}]{\sphinxcrossref{universe.Universe}}}

\end{description}\end{quote}

\end{fulllineitems}

\index{initialize() (commute\_to\_work\_trial.Commute\_To\_Work\_Trial method)}

\begin{fulllineitems}
\phantomsection\label{\detokenize{commute_to_work_trial:commute_to_work_trial.Commute_To_Work_Trial.initialize}}\pysiglinewithargsret{\sphinxbfcode{\sphinxupquote{initialize}}}{}{}
This function sets up the trial
\begin{enumerate}
\item {} 
gets the CHAD data for commuting to work under the appropriate conditions

\item {} 
gets N samples the CHAD data for workng and commuting for the N trials

\item {} 
updates the {\hyperref[\detokenize{params:module-params}]{\sphinxcrossref{\sphinxcode{\sphinxupquote{params}}}}} to reflect the newly assigned working and commuting parameters for the simulation

\end{enumerate}
\begin{quote}\begin{description}
\item[{Returns}] \leavevmode


\end{description}\end{quote}

\end{fulllineitems}


\end{fulllineitems}



\subsection{data\_counter notebook}
\label{\detokenize{data_counter::doc}}\label{\detokenize{data_counter:data-counter-notebook}}
\fvset{hllines={, ,}}%
\begin{sphinxVerbatim}[commandchars=\\\{\}]
\PYG{c+c1}{\PYGZsh{} The United States Environmental Protection Agency through its Office of}
\PYG{c+c1}{\PYGZsh{} Research and Development has developed this software. The code is made}
\PYG{c+c1}{\PYGZsh{} publicly available to better communicate the research. All input data}
\PYG{c+c1}{\PYGZsh{} used fora given application should be reviewed by the researcher so}
\PYG{c+c1}{\PYGZsh{} that the model results are based on appropriate data for any given}
\PYG{c+c1}{\PYGZsh{} application. This model is under continued development. The model and}
\PYG{c+c1}{\PYGZsh{} data included herein do not represent and should not be construed to}
\PYG{c+c1}{\PYGZsh{} represent any Agency determination or policy.}
\PYG{c+c1}{\PYGZsh{}}
\PYG{c+c1}{\PYGZsh{} This file was written by Dr. Namdi Brandon}
\PYG{c+c1}{\PYGZsh{} ORCID: 0000\PYGZhy{}0001\PYGZhy{}7050\PYGZhy{}1538}
\PYG{c+c1}{\PYGZsh{} March 20, 2018}
\end{sphinxVerbatim}

This file loads the activity-data assigned with each activity for the
respective demographic group. For each activity, then the file counts
the amount of Consolidated Human Acitivyt Databse (CHAD) individuals
from both the single day and the longitudinal entries.

Import

\fvset{hllines={, ,}}%
\begin{sphinxVerbatim}[commandchars=\\\{\}]
\PYG{k+kn}{import} \PYG{n+nn}{os}\PYG{o}{,} \PYG{n+nn}{sys}
\PYG{n}{sys}\PYG{o}{.}\PYG{n}{path}\PYG{o}{.}\PYG{n}{append}\PYG{p}{(}\PYG{l+s+s1}{\PYGZsq{}}\PYG{l+s+s1}{..}\PYG{l+s+se}{\PYGZbs{}\PYGZbs{}}\PYG{l+s+s1}{source}\PYG{l+s+s1}{\PYGZsq{}}\PYG{p}{)}
\PYG{n}{sys}\PYG{o}{.}\PYG{n}{path}\PYG{o}{.}\PYG{n}{append}\PYG{p}{(}\PYG{l+s+s1}{\PYGZsq{}}\PYG{l+s+s1}{..}\PYG{l+s+se}{\PYGZbs{}\PYGZbs{}}\PYG{l+s+s1}{processing}\PYG{l+s+s1}{\PYGZsq{}}\PYG{p}{)}

\PYG{c+c1}{\PYGZsh{} plotting capability}
\PYG{k+kn}{import} \PYG{n+nn}{matplotlib}\PYG{n+nn}{.}\PYG{n+nn}{pylab} \PYG{k}{as} \PYG{n+nn}{plt}

\PYG{c+c1}{\PYGZsh{} data frame capability}
\PYG{k+kn}{import} \PYG{n+nn}{pandas} \PYG{k}{as} \PYG{n+nn}{pd}

\PYG{c+c1}{\PYGZsh{} zipfile capability}
\PYG{k+kn}{import} \PYG{n+nn}{zipfile}

\PYG{c+c1}{\PYGZsh{} ABMHAP capability}
\PYG{k+kn}{import} \PYG{n+nn}{my\PYGZus{}globals} \PYG{k}{as} \PYG{n+nn}{mg}
\PYG{k+kn}{import} \PYG{n+nn}{chad\PYGZus{}demography\PYGZus{}adult\PYGZus{}non\PYGZus{}work} \PYG{k}{as} \PYG{n+nn}{cdanw}
\PYG{k+kn}{import} \PYG{n+nn}{chad\PYGZus{}demography\PYGZus{}adult\PYGZus{}work} \PYG{k}{as} \PYG{n+nn}{cdaw}
\PYG{k+kn}{import} \PYG{n+nn}{chad\PYGZus{}demography\PYGZus{}child\PYGZus{}school} \PYG{k}{as} \PYG{n+nn}{cdcs}
\PYG{k+kn}{import} \PYG{n+nn}{chad\PYGZus{}demography\PYGZus{}child\PYGZus{}young} \PYG{k}{as} \PYG{n+nn}{cdcy}
\PYG{k+kn}{import} \PYG{n+nn}{demography} \PYG{k}{as} \PYG{n+nn}{dmg}

\PYG{k+kn}{import} \PYG{n+nn}{activity}\PYG{o}{,} \PYG{n+nn}{chad}\PYG{o}{,} \PYG{n+nn}{datum}
\end{sphinxVerbatim}

\fvset{hllines={, ,}}%
\begin{sphinxVerbatim}[commandchars=\\\{\}]
\PYG{o}{\PYGZpc{}}\PYG{k}{matplotlib} auto
\end{sphinxVerbatim}

\fvset{hllines={, ,}}%
\begin{sphinxVerbatim}[commandchars=\\\{\}]
\PYG{n}{Using} \PYG{n}{matplotlib} \PYG{n}{backend}\PYG{p}{:} \PYG{n}{Qt5Agg}
\end{sphinxVerbatim}

Functions

\fvset{hllines={, ,}}%
\begin{sphinxVerbatim}[commandchars=\\\{\}]
\PYG{k}{def} \PYG{n+nf}{load\PYGZus{}data}\PYG{p}{(}\PYG{n}{z}\PYG{p}{,} \PYG{n}{fnames}\PYG{p}{)}\PYG{p}{:}

    \PYG{l+s+sd}{\PYGZdq{}\PYGZdq{}\PYGZdq{}}
\PYG{l+s+sd}{    This function loads the activity parameter data (start time, end time, \PYGZbs{}}
\PYG{l+s+sd}{    duration, and CHAD records) for an activity for the demographic.}

\PYG{l+s+sd}{    :param zipfile.Zipfile z: the ZipFile object for a given demographic group}
\PYG{l+s+sd}{    :param fnames: the file names for CHAD activity\PYGZhy{}moments data}
\PYG{l+s+sd}{    :type fnames: dict mapping int to str}

\PYG{l+s+sd}{    :return: the start time, end time, duration, and record data for a \PYGZbs{}}
\PYG{l+s+sd}{    given activity}
\PYG{l+s+sd}{    :rtype: numpy.ndarray, numpy.ndarray, numpy.ndarray, numpy.ndarray}
\PYG{l+s+sd}{    \PYGZdq{}\PYGZdq{}\PYGZdq{}}

    \PYG{n}{start} \PYG{o}{=} \PYG{n}{pd}\PYG{o}{.}\PYG{n}{read\PYGZus{}csv}\PYG{p}{(} \PYG{n}{z}\PYG{o}{.}\PYG{n}{open}\PYG{p}{(} \PYG{n}{fnames}\PYG{p}{[}\PYG{n}{chad}\PYG{o}{.}\PYG{n}{START}\PYG{p}{]}\PYG{p}{,} \PYG{n}{mode}\PYG{o}{=}\PYG{l+s+s1}{\PYGZsq{}}\PYG{l+s+s1}{r}\PYG{l+s+s1}{\PYGZsq{}} \PYG{p}{)} \PYG{p}{)}
    \PYG{n}{end} \PYG{o}{=} \PYG{n}{pd}\PYG{o}{.}\PYG{n}{read\PYGZus{}csv}\PYG{p}{(} \PYG{n}{z}\PYG{o}{.}\PYG{n}{open}\PYG{p}{(} \PYG{n}{fnames}\PYG{p}{[}\PYG{n}{chad}\PYG{o}{.}\PYG{n}{END}\PYG{p}{]}\PYG{p}{,} \PYG{n}{mode}\PYG{o}{=}\PYG{l+s+s1}{\PYGZsq{}}\PYG{l+s+s1}{r}\PYG{l+s+s1}{\PYGZsq{}} \PYG{p}{)} \PYG{p}{)}
    \PYG{n}{dt} \PYG{o}{=} \PYG{n}{pd}\PYG{o}{.}\PYG{n}{read\PYGZus{}csv}\PYG{p}{(} \PYG{n}{z}\PYG{o}{.}\PYG{n}{open}\PYG{p}{(} \PYG{n}{fnames}\PYG{p}{[}\PYG{n}{chad}\PYG{o}{.}\PYG{n}{DT}\PYG{p}{]}\PYG{p}{,} \PYG{n}{mode}\PYG{o}{=}\PYG{l+s+s1}{\PYGZsq{}}\PYG{l+s+s1}{r}\PYG{l+s+s1}{\PYGZsq{}} \PYG{p}{)} \PYG{p}{)}
    \PYG{n}{record} \PYG{o}{=} \PYG{n}{pd}\PYG{o}{.}\PYG{n}{read\PYGZus{}csv}\PYG{p}{(} \PYG{n}{z}\PYG{o}{.}\PYG{n}{open}\PYG{p}{(} \PYG{n}{fnames}\PYG{p}{[}\PYG{n}{chad}\PYG{o}{.}\PYG{n}{RECORD}\PYG{p}{]}\PYG{p}{,} \PYG{n}{mode}\PYG{o}{=}\PYG{l+s+s1}{\PYGZsq{}}\PYG{l+s+s1}{r}\PYG{l+s+s1}{\PYGZsq{}} \PYG{p}{)} \PYG{p}{)}

    \PYG{k}{return} \PYG{n}{start}\PYG{p}{,} \PYG{n}{end}\PYG{p}{,} \PYG{n}{dt}\PYG{p}{,} \PYG{n}{record}

\PYG{k}{def} \PYG{n+nf}{filter\PYGZus{}data}\PYG{p}{(}\PYG{n}{df}\PYG{p}{,} \PYG{n}{the\PYGZus{}filter}\PYG{p}{,} \PYG{n}{start\PYGZus{}periodic}\PYG{o}{=}\PYG{k+kc}{False}\PYG{p}{,} \PYG{n}{end\PYGZus{}periodic}\PYG{o}{=}\PYG{k+kc}{False}\PYG{p}{)}\PYG{p}{:}

    \PYG{l+s+sd}{\PYGZdq{}\PYGZdq{}\PYGZdq{}}
\PYG{l+s+sd}{    This function takes CHAD data for an activity and filters the CHAD data \PYGZbs{}}
\PYG{l+s+sd}{    the satisfy the sampling parameters. This function returns the CHAD data \PYGZbs{}}
\PYG{l+s+sd}{    suitable for use in parameterizing ABMHAP.}

\PYG{l+s+sd}{    :param pandas.core.frame.DataFrame df: the record data for a given activity}
\PYG{l+s+sd}{    :param the\PYGZus{}filter: for a given activity code, get the respective parameters \PYGZbs{}}
\PYG{l+s+sd}{    for sampling CHAD data}
\PYG{l+s+sd}{    :type the\PYGZus{}filter: dict mapping int to :class:{}`chad\PYGZus{}params.CHAD\PYGZus{}params{}`}
\PYG{l+s+sd}{    :param bool start\PYGZus{}periodic: whether (if True) or not (if False) the start \PYGZbs{}}
\PYG{l+s+sd}{    time should be in a [\PYGZhy{}12, 12) format}
\PYG{l+s+sd}{    :param bool end\PYGZus{}periodic: whether (if True) or not (if False) the end \PYGZbs{}}
\PYG{l+s+sd}{    time should be in a [\PYGZhy{}12, 12) format}

\PYG{l+s+sd}{    :return: the CHAD data that satisfy the sampling parameters for the following:}
\PYG{l+s+sd}{    start time moments, end time moments, duration momments, and records}
\PYG{l+s+sd}{    :rtype: pandas.core.frame.DataFrame, pandas.core.frame.DataFrame, \PYGZbs{}}
\PYG{l+s+sd}{    pandas.core.frame.DataFrame, pandas.core.frame.DataFrame}
\PYG{l+s+sd}{    \PYGZdq{}\PYGZdq{}\PYGZdq{}}

    \PYG{c+c1}{\PYGZsh{} the\PYGZus{}filter are the sampling paramters for the activity}

    \PYG{c+c1}{\PYGZsh{} the start time and end time data}
    \PYG{n}{x\PYGZus{}start}\PYG{p}{,} \PYG{n}{x\PYGZus{}end} \PYG{o}{=} \PYG{n}{df}\PYG{o}{.}\PYG{n}{start}\PYG{p}{,} \PYG{n}{df}\PYG{o}{.}\PYG{n}{end}

    \PYG{c+c1}{\PYGZsh{} change the start time data to a [\PYGZhy{}12, 12) format}
    \PYG{k}{if} \PYG{n}{start\PYGZus{}periodic}\PYG{p}{:}
        \PYG{n}{x\PYGZus{}start} \PYG{o}{=} \PYG{n}{mg}\PYG{o}{.}\PYG{n}{to\PYGZus{}periodic}\PYG{p}{(}\PYG{n}{x\PYGZus{}start}\PYG{p}{,} \PYG{n}{do\PYGZus{}hours}\PYG{o}{=}\PYG{k+kc}{True}\PYG{p}{)}

    \PYG{c+c1}{\PYGZsh{} change the start time data to a [\PYGZhy{}12, 12) format}
    \PYG{k}{if} \PYG{n}{end\PYGZus{}periodic}\PYG{p}{:}
        \PYG{n}{x\PYGZus{}end} \PYG{o}{=} \PYG{n}{mg}\PYG{o}{.}\PYG{n}{to\PYGZus{}periodic}\PYG{p}{(}\PYG{n}{x\PYGZus{}end}\PYG{p}{,} \PYG{n}{do\PYGZus{}hours}\PYG{o}{=}\PYG{k+kc}{True}\PYG{p}{)}

    \PYG{c+c1}{\PYGZsh{} the indices that satisfy the requirements for mean start time, end time, and}
    \PYG{c+c1}{\PYGZsh{} and duration respectively}
    \PYG{n}{idx} \PYG{o}{=} \PYG{p}{(} \PYG{n}{x\PYGZus{}start} \PYG{o}{\PYGZgt{}}\PYG{o}{=} \PYG{n}{the\PYGZus{}filter}\PYG{o}{.}\PYG{n}{start\PYGZus{}mean\PYGZus{}min} \PYG{p}{)} \PYG{o}{\PYGZam{}} \PYG{p}{(} \PYG{n}{x\PYGZus{}start} \PYG{o}{\PYGZlt{}}\PYG{o}{=} \PYG{n}{the\PYGZus{}filter}\PYG{o}{.}\PYG{n}{start\PYGZus{}mean\PYGZus{}max} \PYG{p}{)} \PYGZbs{}
    \PYG{o}{\PYGZam{}} \PYG{p}{(} \PYG{n}{df}\PYG{o}{.}\PYG{n}{end} \PYG{o}{\PYGZgt{}}\PYG{o}{=} \PYG{n}{the\PYGZus{}filter}\PYG{o}{.}\PYG{n}{end\PYGZus{}mean\PYGZus{}min} \PYG{p}{)} \PYG{o}{\PYGZam{}} \PYG{p}{(} \PYG{n}{df}\PYG{o}{.}\PYG{n}{end} \PYG{o}{\PYGZlt{}}\PYG{o}{=} \PYG{n}{the\PYGZus{}filter}\PYG{o}{.}\PYG{n}{end\PYGZus{}mean\PYGZus{}max} \PYG{p}{)} \PYGZbs{}
    \PYG{o}{\PYGZam{}} \PYG{p}{(} \PYG{n}{df}\PYG{o}{.}\PYG{n}{dt} \PYG{o}{\PYGZgt{}}\PYG{o}{=} \PYG{n}{the\PYGZus{}filter}\PYG{o}{.}\PYG{n}{dt\PYGZus{}mean\PYGZus{}min} \PYG{p}{)} \PYG{o}{\PYGZam{}} \PYG{p}{(} \PYG{n}{df}\PYG{o}{.}\PYG{n}{dt} \PYG{o}{\PYGZlt{}}\PYG{o}{=} \PYG{n}{the\PYGZus{}filter}\PYG{o}{.}\PYG{n}{dt\PYGZus{}mean\PYGZus{}max} \PYG{p}{)}

    \PYG{c+c1}{\PYGZsh{} get the record data that satisfy the proper sampling ranges}
    \PYG{n}{record} \PYG{o}{=} \PYG{n}{df}\PYG{p}{[}\PYG{n}{idx}\PYG{p}{]}

    \PYG{c+c1}{\PYGZsh{} the personal identifier values within the CHAD data}
    \PYG{n}{pid} \PYG{o}{=} \PYG{n}{record}\PYG{o}{.}\PYG{n}{PID}\PYG{o}{.}\PYG{n}{values}

    \PYG{c+c1}{\PYGZsh{} obtain the duraation, start time, and end time values from the filtered CHAD records}
    \PYG{n}{dt}\PYG{p}{,} \PYG{n}{start}\PYG{p}{,} \PYG{n}{end} \PYG{o}{=} \PYG{n}{record}\PYG{o}{.}\PYG{n}{dt}\PYG{o}{.}\PYG{n}{values}\PYG{p}{,} \PYG{n}{record}\PYG{o}{.}\PYG{n}{start}\PYG{o}{.}\PYG{n}{values}\PYG{p}{,} \PYG{n}{record}\PYG{o}{.}\PYG{n}{end}\PYG{o}{.}\PYG{n}{values}

    \PYG{c+c1}{\PYGZsh{} the CHAD data that satisfy the sampling parameters for the start time moments}
    \PYG{n}{stats\PYGZus{}start} \PYG{o}{=} \PYG{n}{datum}\PYG{o}{.}\PYG{n}{get\PYGZus{}stats}\PYG{p}{(}\PYG{n}{pid}\PYG{p}{,} \PYG{n}{start}\PYG{p}{,} \PYG{n}{do\PYGZus{}periodic}\PYG{o}{=}\PYG{n}{start\PYGZus{}periodic}\PYG{p}{)}

    \PYG{c+c1}{\PYGZsh{} the CHAD data that satisfy the sampling parameters for the end time moments}
    \PYG{n}{stats\PYGZus{}end}   \PYG{o}{=} \PYG{n}{datum}\PYG{o}{.}\PYG{n}{get\PYGZus{}stats}\PYG{p}{(}\PYG{n}{pid}\PYG{p}{,} \PYG{n}{end}\PYG{p}{,} \PYG{n}{do\PYGZus{}periodic}\PYG{o}{=}\PYG{n}{start\PYGZus{}periodic}\PYG{p}{)}

    \PYG{c+c1}{\PYGZsh{} the CHAD data that satisfy the sampling parameters for the duration moments}
    \PYG{n}{stats\PYGZus{}dt}    \PYG{o}{=} \PYG{n}{datum}\PYG{o}{.}\PYG{n}{get\PYGZus{}stats}\PYG{p}{(}\PYG{n}{pid}\PYG{p}{,} \PYG{n}{dt}\PYG{p}{)}

    \PYG{k}{return} \PYG{n}{stats\PYGZus{}start}\PYG{p}{,} \PYG{n}{stats\PYGZus{}end}\PYG{p}{,} \PYG{n}{stats\PYGZus{}dt}\PYG{p}{,} \PYG{n}{record}

\PYG{k}{def} \PYG{n+nf}{get\PYGZus{}activity\PYGZus{}data}\PYG{p}{(}\PYG{n}{z}\PYG{p}{,} \PYG{n}{fnames}\PYG{p}{,} \PYG{n}{the\PYGZus{}filter}\PYG{p}{,} \PYG{n}{start\PYGZus{}periodic}\PYG{o}{=}\PYG{k+kc}{False}\PYG{p}{,} \PYG{n}{end\PYGZus{}periodic}\PYG{o}{=}\PYG{k+kc}{False}\PYG{p}{)}\PYG{p}{:}

    \PYG{l+s+sd}{\PYGZdq{}\PYGZdq{}\PYGZdq{}}
\PYG{l+s+sd}{    This function loads CHAD data for an activity and filters the CHAD data \PYGZbs{}}
\PYG{l+s+sd}{    the satisfy the sampling parameters. This function returns the CHAD data \PYGZbs{}}
\PYG{l+s+sd}{    suitable for use in parameterizing ABMHAP.}

\PYG{l+s+sd}{    :param zipfile.Zipfile z: the ZipFile object for a given demographic group}
\PYG{l+s+sd}{    :param fnames: the file names for CHAD activity\PYGZhy{}moments data}
\PYG{l+s+sd}{    :type fnames: dict mapping int to str}
\PYG{l+s+sd}{    :param the\PYGZus{}filter: for a given activity code, get the respective parameters \PYGZbs{}}
\PYG{l+s+sd}{    for sampling CHAD data}
\PYG{l+s+sd}{    :type the\PYGZus{}filter: dict mapping int to :class:{}`chad\PYGZus{}params.CHAD\PYGZus{}params{}`}
\PYG{l+s+sd}{    :param bool start\PYGZus{}periodic: whether (if True) or not (if False) the start \PYGZbs{}}
\PYG{l+s+sd}{    time should be in a [\PYGZhy{}12, 12) format}
\PYG{l+s+sd}{    :param bool end\PYGZus{}periodic: whether (if True) or not (if False) the end \PYGZbs{}}
\PYG{l+s+sd}{    time should be in a [\PYGZhy{}12, 12) format}

\PYG{l+s+sd}{    :return: the CHAD data that satisfy the sampling parameters for the following:}
\PYG{l+s+sd}{    start time moments, end time moments, duration momments, and records}
\PYG{l+s+sd}{    :rtype: pandas.core.frame.DataFrame, pandas.core.frame.DataFrame, \PYGZbs{}}
\PYG{l+s+sd}{    pandas.core.frame.DataFrame, pandas.core.frame.DataFrame}
\PYG{l+s+sd}{    \PYGZdq{}\PYGZdq{}\PYGZdq{}}

    \PYG{c+c1}{\PYGZsh{} get the longitudinal data}
    \PYG{n}{start}\PYG{p}{,} \PYG{n}{end}\PYG{p}{,} \PYG{n}{dt}\PYG{p}{,} \PYG{n}{record} \PYG{o}{=} \PYG{n}{load\PYGZus{}data}\PYG{p}{(}\PYG{n}{z}\PYG{p}{,} \PYG{n}{fnames}\PYG{p}{)}

    \PYG{c+c1}{\PYGZsh{} filter the records and get the moments}
    \PYG{n}{stats\PYGZus{}start}\PYG{p}{,} \PYG{n}{stats\PYGZus{}end}\PYG{p}{,} \PYG{n}{stats\PYGZus{}dt}\PYG{p}{,} \PYG{n}{record} \PYG{o}{=} \PYGZbs{}
    \PYG{n}{filter\PYGZus{}data}\PYG{p}{(}\PYG{n}{record}\PYG{p}{,} \PYG{n}{the\PYGZus{}filter}\PYG{p}{,} \PYG{n}{start\PYGZus{}periodic}\PYG{o}{=}\PYG{n}{start\PYGZus{}periodic}\PYG{p}{,} \PYG{n}{end\PYGZus{}periodic}\PYG{o}{=}\PYG{n}{end\PYGZus{}periodic}\PYG{p}{)}

    \PYG{k}{return} \PYG{n}{stats\PYGZus{}start}\PYG{p}{,} \PYG{n}{stats\PYGZus{}end}\PYG{p}{,} \PYG{n}{stats\PYGZus{}dt}\PYG{p}{,} \PYG{n}{record}

\PYG{k}{def} \PYG{n+nf}{get\PYGZus{}fnames}\PYG{p}{(}\PYG{n}{demo}\PYG{p}{,} \PYG{n}{k}\PYG{p}{,} \PYG{n}{do\PYGZus{}long}\PYG{p}{)}\PYG{p}{:}

    \PYG{l+s+sd}{\PYGZdq{}\PYGZdq{}\PYGZdq{}}
\PYG{l+s+sd}{    For a demographic, this function obtains the file names of the \PYGZbs{}}
\PYG{l+s+sd}{    activity data for longitudinal or single\PYGZhy{}day data.}

\PYG{l+s+sd}{    :param demography.Demography demo: the demographic of choice to access the CHAD data}
\PYG{l+s+sd}{    :param int k: the activity code}
\PYG{l+s+sd}{    :param bool do\PYGZus{}long: whether (if True) to load the longitduinal data. If not (False), \PYGZbs{}}
\PYG{l+s+sd}{    load the single\PYGZhy{}day data.}

\PYG{l+s+sd}{    :return: the file names for CHAD activity\PYGZhy{}moments data for longitudinal data \PYGZbs{}}
\PYG{l+s+sd}{    or single\PYGZhy{}day data}
\PYG{l+s+sd}{    :rtype: dict of int to str}
\PYG{l+s+sd}{    \PYGZdq{}\PYGZdq{}\PYGZdq{}}

    \PYG{c+c1}{\PYGZsh{} get the file names of the longitudinal data}
    \PYG{n}{fnames} \PYG{o}{=} \PYG{n}{demo}\PYG{o}{.}\PYG{n}{fname\PYGZus{}stats}\PYG{p}{[}\PYG{n}{k}\PYG{p}{]}

    \PYG{k}{if} \PYG{o+ow}{not} \PYG{n}{do\PYGZus{}long}\PYG{p}{:}
        \PYG{c+c1}{\PYGZsh{} get the file names of the single\PYGZhy{}day data}
        \PYG{n}{x} \PYG{o}{=} \PYG{p}{[} \PYG{p}{(} \PYG{n}{key}\PYG{p}{,} \PYG{n}{value}\PYG{o}{.}\PYG{n}{replace}\PYG{p}{(}\PYG{l+s+s1}{\PYGZsq{}}\PYG{l+s+s1}{longitude}\PYG{l+s+s1}{\PYGZsq{}}\PYG{p}{,} \PYG{l+s+s1}{\PYGZsq{}}\PYG{l+s+s1}{solo}\PYG{l+s+s1}{\PYGZsq{}}\PYG{p}{)} \PYG{p}{)} \PYG{k}{for} \PYG{n}{key}\PYG{p}{,} \PYG{n}{value} \PYG{o+ow}{in} \PYG{n}{fnames}\PYG{o}{.}\PYG{n}{items}\PYG{p}{(}\PYG{p}{)} \PYG{p}{]}
        \PYG{n}{fnames} \PYG{o}{=} \PYG{n+nb}{dict}\PYG{p}{(} \PYG{n}{x} \PYG{p}{)}

    \PYG{k}{return} \PYG{n}{fnames}

\PYG{k}{def} \PYG{n+nf}{plot}\PYG{p}{(}\PYG{n}{data}\PYG{p}{,} \PYG{n}{ax}\PYG{p}{,} \PYG{n}{label}\PYG{p}{)}\PYG{p}{:}

    \PYG{l+s+sd}{\PYGZdq{}\PYGZdq{}\PYGZdq{}}
\PYG{l+s+sd}{    This function gets data and plots the empiricial cumulative dsitribution \PYGZbs{}}
\PYG{l+s+sd}{    function (CDF) of the data.}

\PYG{l+s+sd}{    :param numpy.ndarray data: the data to create a CDF of}
\PYG{l+s+sd}{    :param matplotlib.axes.\PYGZus{}subplots.AxesSubplot ax: the subplot that\PYGZsq{}s plotting}
\PYG{l+s+sd}{    :param str label: the label for the data}
\PYG{l+s+sd}{    \PYGZdq{}\PYGZdq{}\PYGZdq{}}

    \PYG{c+c1}{\PYGZsh{} get an empiricial CDF based on the data}
    \PYG{n}{x}\PYG{p}{,} \PYG{n}{y} \PYG{o}{=} \PYG{n}{mg}\PYG{o}{.}\PYG{n}{get\PYGZus{}ecdf}\PYG{p}{(}\PYG{n}{data}\PYG{p}{)}

    \PYG{c+c1}{\PYGZsh{} plot the CDF}
    \PYG{n}{ax}\PYG{o}{.}\PYG{n}{plot}\PYG{p}{(}\PYG{n}{x}\PYG{p}{,} \PYG{n}{y}\PYG{p}{,} \PYG{n}{label}\PYG{o}{=}\PYG{n}{label}\PYG{p}{)}

    \PYG{c+c1}{\PYGZsh{} show legend}
    \PYG{n}{ax}\PYG{o}{.}\PYG{n}{legend}\PYG{p}{(}\PYG{n}{loc}\PYG{o}{=}\PYG{l+s+s1}{\PYGZsq{}}\PYG{l+s+s1}{best}\PYG{l+s+s1}{\PYGZsq{}}\PYG{p}{)}

    \PYG{k}{return}
\end{sphinxVerbatim}

Run

Load data via demographic

\fvset{hllines={, ,}}%
\begin{sphinxVerbatim}[commandchars=\\\{\}]
\PYG{c+c1}{\PYGZsh{} map a demographic type to the respective CHAD\PYGZus{}demography object}
\PYG{n}{chooser} \PYG{o}{=} \PYG{p}{\PYGZob{}}\PYG{n}{dmg}\PYG{o}{.}\PYG{n}{ADULT\PYGZus{}WORK}\PYG{p}{:} \PYG{n}{cdaw}\PYG{o}{.}\PYG{n}{CHAD\PYGZus{}demography\PYGZus{}adult\PYGZus{}work}\PYG{p}{(}\PYG{p}{)}\PYG{p}{,}
           \PYG{n}{dmg}\PYG{o}{.}\PYG{n}{ADULT\PYGZus{}NON\PYGZus{}WORK}\PYG{p}{:} \PYG{n}{cdanw}\PYG{o}{.}\PYG{n}{CHAD\PYGZus{}demography\PYGZus{}adult\PYGZus{}non\PYGZus{}work}\PYG{p}{(}\PYG{p}{)}\PYG{p}{,}
           \PYG{n}{dmg}\PYG{o}{.}\PYG{n}{CHILD\PYGZus{}SCHOOL}\PYG{p}{:} \PYG{n}{cdcs}\PYG{o}{.}\PYG{n}{CHAD\PYGZus{}demography\PYGZus{}child\PYGZus{}school}\PYG{p}{(}\PYG{p}{)}\PYG{p}{,}
           \PYG{n}{dmg}\PYG{o}{.}\PYG{n}{CHILD\PYGZus{}YOUNG}\PYG{p}{:} \PYG{n}{cdcy}\PYG{o}{.}\PYG{n}{CHAD\PYGZus{}demography\PYGZus{}child\PYGZus{}young}\PYG{p}{(}\PYG{p}{)}\PYG{p}{\PYGZcb{}}
\end{sphinxVerbatim}

\fvset{hllines={, ,}}%
\begin{sphinxVerbatim}[commandchars=\\\{\}]
\PYG{c+c1}{\PYGZsh{} choose the demography}
\PYG{n}{demo\PYGZus{}type} \PYG{o}{=} \PYG{n}{dmg}\PYG{o}{.}\PYG{n}{CHILD\PYGZus{}SCHOOL}

\PYG{c+c1}{\PYGZsh{} get the name of the compressed data file}
\PYG{n}{fname\PYGZus{}zip} \PYG{o}{=} \PYG{n}{dmg}\PYG{o}{.}\PYG{n}{FNAME\PYGZus{}DEMOGRAPHY}\PYG{p}{[}\PYG{n}{demo\PYGZus{}type}\PYG{p}{]}

\PYG{c+c1}{\PYGZsh{} create the ZipFile object for the respective demographic group}
\PYG{n}{z} \PYG{o}{=} \PYG{n}{zipfile}\PYG{o}{.}\PYG{n}{ZipFile}\PYG{p}{(} \PYG{n}{fname\PYGZus{}zip} \PYG{p}{)}

\PYG{c+c1}{\PYGZsh{} set the demographic object}
\PYG{n}{demo} \PYG{o}{=} \PYG{n}{chooser}\PYG{p}{[}\PYG{n}{demo\PYGZus{}type}\PYG{p}{]}

\PYG{c+c1}{\PYGZsh{} store all of the activity\PYGZhy{}keys for the demographic}
\PYG{n}{keys} \PYG{o}{=} \PYG{n}{demo}\PYG{o}{.}\PYG{n}{keys}

\PYG{c+c1}{\PYGZsh{} print flag}
\PYG{n}{do\PYGZus{}print} \PYG{o}{=} \PYG{k+kc}{False}
\end{sphinxVerbatim}

Count the number of CHAD persons for each activity

\fvset{hllines={, ,}}%
\begin{sphinxVerbatim}[commandchars=\\\{\}]
\PYG{c+c1}{\PYGZsh{} if true, count the number of people with longitudinal data (at least 2 entries)}
\PYG{c+c1}{\PYGZsh{} if false, count the number of people with single data (only 1 entry)}
\PYG{n}{do\PYGZus{}long} \PYG{o}{=} \PYG{k+kc}{True}


\PYG{c+c1}{\PYGZsh{} for each activity in the demographic, count the amount of data}
\PYG{k}{for} \PYG{n}{k} \PYG{o+ow}{in} \PYG{n}{keys}\PYG{p}{:}

    \PYG{c+c1}{\PYGZsh{} set whether to set the time to periodic time [\PYGZhy{}12, 12) hours instead of [0, 24) hours}
    \PYG{n}{do\PYGZus{}periodic} \PYG{o}{=} \PYG{k+kc}{False}
    \PYG{k}{if} \PYG{n}{k} \PYG{o}{==} \PYG{n}{mg}\PYG{o}{.}\PYG{n}{KEY\PYGZus{}SLEEP}\PYG{p}{:}
        \PYG{n}{do\PYGZus{}periodic} \PYG{o}{=} \PYG{k+kc}{True}

    \PYG{c+c1}{\PYGZsh{} sampling / filtering params}
    \PYG{n}{the\PYGZus{}filter} \PYG{o}{=} \PYG{n}{demo}\PYG{o}{.}\PYG{n}{int\PYGZus{}2\PYGZus{}param}\PYG{p}{[}\PYG{n}{k}\PYG{p}{]}

    \PYG{c+c1}{\PYGZsh{} get the names of the statistics files}
    \PYG{n}{fnames} \PYG{o}{=} \PYG{n}{get\PYGZus{}fnames}\PYG{p}{(}\PYG{n}{demo}\PYG{p}{,} \PYG{n}{k}\PYG{p}{,} \PYG{n}{do\PYGZus{}long}\PYG{p}{)}

    \PYG{c+c1}{\PYGZsh{} load and filter data fitting for the demographic}
    \PYG{n}{start}\PYG{p}{,} \PYG{n}{end}\PYG{p}{,} \PYG{n}{dt}\PYG{p}{,} \PYG{n}{record} \PYG{o}{=} \PYG{n}{get\PYGZus{}activity\PYGZus{}data}\PYG{p}{(}\PYG{n}{z}\PYG{p}{,} \PYG{n}{fnames}\PYG{p}{,} \PYG{n}{the\PYGZus{}filter}\PYG{p}{,} \PYG{n}{start\PYGZus{}periodic}\PYG{o}{=}\PYG{n}{do\PYGZus{}periodic}\PYG{p}{)}

    \PYG{c+c1}{\PYGZsh{} print the activity}
    \PYG{k}{if} \PYG{n}{do\PYGZus{}print}\PYG{p}{:}
        \PYG{n+nb}{print}\PYG{p}{(} \PYG{n}{activity}\PYG{o}{.}\PYG{n}{INT\PYGZus{}2\PYGZus{}STR}\PYG{p}{[}\PYG{n}{k}\PYG{p}{]} \PYG{p}{)}

        \PYG{c+c1}{\PYGZsh{} count the number of longitudinal or single\PYGZhy{}day data, respectively}
        \PYG{k}{if} \PYG{n}{do\PYGZus{}long}\PYG{p}{:}
            \PYG{n+nb}{print}\PYG{p}{(} \PYG{n}{start}\PYG{p}{[}\PYG{n}{start}\PYG{o}{.}\PYG{n}{N} \PYG{o}{\PYGZgt{}} \PYG{l+m+mi}{1}\PYG{p}{]}\PYG{o}{.}\PYG{n}{shape}\PYG{p}{)}
        \PYG{k}{else}\PYG{p}{:}
            \PYG{n+nb}{print}\PYG{p}{(} \PYG{n}{start}\PYG{p}{[}\PYG{n}{start}\PYG{o}{.}\PYG{n}{N} \PYG{o}{==} \PYG{l+m+mi}{1}\PYG{p}{]}\PYG{o}{.}\PYG{n}{shape}\PYG{p}{)}
\end{sphinxVerbatim}
\begin{sphinxalltt}
..processingdatum.py:689: RuntimeWarning: invalid value encountered in double\_scalars
  cv  = std / np.abs(mu)
..processingdatum.py:689: RuntimeWarning: divide by zero encountered in double\_scalars
  cv  = std / np.abs(mu)
\end{sphinxalltt}

Plot the data

\fvset{hllines={, ,}}%
\begin{sphinxVerbatim}[commandchars=\\\{\}]
\PYG{c+c1}{\PYGZsh{} create the subplots}
\PYG{n}{fig}\PYG{p}{,} \PYG{n}{axes} \PYG{o}{=} \PYG{n}{plt}\PYG{o}{.}\PYG{n}{subplots}\PYG{p}{(}\PYG{l+m+mi}{3}\PYG{p}{)}

\PYG{c+c1}{\PYGZsh{} the title}
\PYG{n}{fig}\PYG{o}{.}\PYG{n}{suptitle}\PYG{p}{(}\PYG{n}{activity}\PYG{o}{.}\PYG{n}{INT\PYGZus{}2\PYGZus{}STR}\PYG{p}{[}\PYG{n}{k}\PYG{p}{]}\PYG{p}{)}

\PYG{c+c1}{\PYGZsh{}}
\PYG{c+c1}{\PYGZsh{} plot the start time data}
\PYG{c+c1}{\PYGZsh{}}

\PYG{c+c1}{\PYGZsh{} select the subplot}
\PYG{n}{ax} \PYG{o}{=} \PYG{n}{axes}\PYG{p}{[}\PYG{l+m+mi}{0}\PYG{p}{]}

\PYG{c+c1}{\PYGZsh{} the start time data}
\PYG{n}{plot}\PYG{p}{(}\PYG{n}{start}\PYG{o}{.}\PYG{n}{mu}\PYG{o}{.}\PYG{n}{values}\PYG{p}{,} \PYG{n}{ax}\PYG{p}{,} \PYG{l+s+s1}{\PYGZsq{}}\PYG{l+s+s1}{start}\PYG{l+s+s1}{\PYGZsq{}}\PYG{p}{)}

\PYG{c+c1}{\PYGZsh{}}
\PYG{c+c1}{\PYGZsh{} plot the end time data}
\PYG{c+c1}{\PYGZsh{}}

\PYG{c+c1}{\PYGZsh{} select the subplot}
\PYG{n}{ax} \PYG{o}{=} \PYG{n}{axes}\PYG{p}{[}\PYG{l+m+mi}{1}\PYG{p}{]}

\PYG{c+c1}{\PYGZsh{} the end time data}
\PYG{n}{plot}\PYG{p}{(}\PYG{n}{end}\PYG{o}{.}\PYG{n}{mu}\PYG{o}{.}\PYG{n}{values}\PYG{p}{,} \PYG{n}{ax}\PYG{p}{,} \PYG{l+s+s1}{\PYGZsq{}}\PYG{l+s+s1}{end}\PYG{l+s+s1}{\PYGZsq{}}\PYG{p}{)}

\PYG{c+c1}{\PYGZsh{}}
\PYG{c+c1}{\PYGZsh{} plot the duration data}
\PYG{c+c1}{\PYGZsh{}}

\PYG{c+c1}{\PYGZsh{} select the subplot}
\PYG{n}{ax} \PYG{o}{=} \PYG{n}{axes}\PYG{p}{[}\PYG{l+m+mi}{2}\PYG{p}{]}

\PYG{c+c1}{\PYGZsh{} the duration data}
\PYG{n}{plot}\PYG{p}{(}\PYG{n}{dt}\PYG{o}{.}\PYG{n}{mu}\PYG{o}{.}\PYG{n}{values}\PYG{p}{,} \PYG{n}{ax}\PYG{p}{,} \PYG{l+s+s1}{\PYGZsq{}}\PYG{l+s+s1}{duration}\PYG{l+s+s1}{\PYGZsq{}}\PYG{p}{)}

\PYG{c+c1}{\PYGZsh{} show plots}
\PYG{n}{plt}\PYG{o}{.}\PYG{n}{show}\PYG{p}{(}\PYG{p}{)}
\end{sphinxVerbatim}


\subsection{driver module}
\label{\detokenize{driver::doc}}\label{\detokenize{driver:module-driver}}\label{\detokenize{driver:driver-module}}\index{driver (module)}
This code runs the simulation for the Agent-Based Model of Human Activity Patterns (ABMHAP) module of the Life Cycle Human Exposure Model (LC-HEM) project. This code is the driver for seeing how well ABMHAP parameterized with empirical human behavior data from the Consolidated Human Activity Database (CHAD) compares to results seen in CHAD.

\begin{sphinxadmonition}{note}{Note:}
This code may be run in batches in order to run many households while conserving memory. That is,     instead of running 32 households at once (and keeping 32 households in memory), the program can     run 2 batches of 16 households (for a total of 32 household). This halves the amount of memory     used in the simulation compared to running the simulation of 1 batch of 32 households. We      will shown how to run the code using “batches” below.
\end{sphinxadmonition}

The driver can also be run in \sphinxstylestrong{parallel}. We will show how to do so below.

To run the code, do the following.
\begin{enumerate}
\item {} 
Set the simulation-centric parameters in driver\_params.py

\item {} \begin{description}
\item[{Run the code as}] \leavevmode
\textgreater{} \sphinxcode{\sphinxupquote{python driver.py num\_process num\_hhld num\_batch}}
where
\begin{itemize}
\item {} 
\sphinxcode{\sphinxupquote{num\_process}} is the total number of cores (i.e, processing units) used in the simulation

\item {} 
\sphinxcode{\sphinxupquote{num\_hhld}} is the number of simulations to run per batch

\item {} 
\sphinxcode{\sphinxupquote{num\_batch}} is the number of batches used per core

\end{itemize}

\end{description}

\end{enumerate}

The following are examples on how to run the code:

To run in \sphinxstylestrong{serial} with with 64 households per batch, 1 batch (implied)

\textgreater{} \sphinxcode{\sphinxupquote{python driver.py 1 64 1}}

\textgreater{} \sphinxcode{\sphinxupquote{python driver.py 1 64}}

To run in serial using 2 batches with 1 thread with 32 households per batch, 2 batches

\textgreater{} \sphinxcode{\sphinxupquote{python driver.py 1 32 2}}

To run in \sphinxstylestrong{parallel} using 4 cores with 64 households total (16 household per core per batch), 1 batch (implied)

\textgreater{} \sphinxcode{\sphinxupquote{python driver.py 4 64 1}}

\textgreater{} \sphinxcode{\sphinxupquote{python driver.py 4 64}}

To run in parallel using 4 cores with 32 households per batch, 2 batches(8 households per core per batch)

\textgreater{} \sphinxcode{\sphinxupquote{python driver.py 4 32 2}}
\index{create\_trials() (in module driver)}

\begin{fulllineitems}
\phantomsection\label{\detokenize{driver:driver.create_trials}}\pysiglinewithargsret{\sphinxcode{\sphinxupquote{driver.}}\sphinxbfcode{\sphinxupquote{create\_trials}}}{\emph{num\_hhld}, \emph{num\_days}, \emph{num\_hours}, \emph{num\_min}, \emph{trial\_code}, \emph{chad\_activity\_params}, \emph{demographic}, \emph{num\_people}, \emph{do\_minute\_by\_minute}, \emph{do\_print=False}}{}
This function creates the input data for each household in the simulation.
\begin{quote}\begin{description}
\item[{Parameters}] \leavevmode\begin{itemize}
\item {} 
\sphinxstyleliteralstrong{\sphinxupquote{num\_hhld}} (\sphinxstyleliteralemphasis{\sphinxupquote{int}}) \textendash{} the number of households simulated

\item {} 
\sphinxstyleliteralstrong{\sphinxupquote{num\_days}} (\sphinxstyleliteralemphasis{\sphinxupquote{int}}) \textendash{} the number of days in the simulation

\item {} 
\sphinxstyleliteralstrong{\sphinxupquote{num\_hours}} (\sphinxstyleliteralemphasis{\sphinxupquote{int}}) \textendash{} the number of additional hours

\item {} 
\sphinxstyleliteralstrong{\sphinxupquote{num\_min}} (\sphinxstyleliteralemphasis{\sphinxupquote{int}}) \textendash{} the number of additional minutes

\item {} 
\sphinxstyleliteralstrong{\sphinxupquote{trial\_code}} (\sphinxstyleliteralemphasis{\sphinxupquote{int}}) \textendash{} the trial identifier

\item {} 
\sphinxstyleliteralstrong{\sphinxupquote{chad\_activity\_params}} ({\hyperref[\detokenize{chad_params:chad_params.CHAD_params}]{\sphinxcrossref{\sphinxstyleliteralemphasis{\sphinxupquote{chad\_params.CHAD\_params}}}}}) \textendash{} the activity parameters     used to sample “good” CHAD data

\item {} 
\sphinxstyleliteralstrong{\sphinxupquote{demographic}} (\sphinxstyleliteralemphasis{\sphinxupquote{int}}) \textendash{} the demographic identifier

\item {} 
\sphinxstyleliteralstrong{\sphinxupquote{num\_people}} (\sphinxstyleliteralemphasis{\sphinxupquote{int}}) \textendash{} the number of people per household

\item {} 
\sphinxstyleliteralstrong{\sphinxupquote{do\_minute\_by\_minute}} (\sphinxstyleliteralemphasis{\sphinxupquote{bool}}) \textendash{} a flag for how the time steps progress in the scheduler

\item {} 
\sphinxstyleliteralstrong{\sphinxupquote{do\_print}} (\sphinxstyleliteralemphasis{\sphinxupquote{bool}}) \textendash{} flag whether to print messages to the console

\end{itemize}

\item[{Returns}] \leavevmode
input data where each entry corresponds to the input     for the respective household in the simulation

\item[{Return type}] \leavevmode
list of {\hyperref[\detokenize{trial:trial.Trial}]{\sphinxcrossref{\sphinxcode{\sphinxupquote{trial.Trial}}}}}

\end{description}\end{quote}

\end{fulllineitems}

\index{delete\_batch\_files() (in module driver)}

\begin{fulllineitems}
\phantomsection\label{\detokenize{driver:driver.delete_batch_files}}\pysiglinewithargsret{\sphinxcode{\sphinxupquote{driver.}}\sphinxbfcode{\sphinxupquote{delete\_batch\_files}}}{\emph{fname\_base}, \emph{num\_batch}}{}
This function deletes the batch files.
\begin{quote}\begin{description}
\item[{Parameters}] \leavevmode\begin{itemize}
\item {} 
\sphinxstyleliteralstrong{\sphinxupquote{fname\_base}} (\sphinxstyleliteralemphasis{\sphinxupquote{str}}) \textendash{} the file name for the files without the “.pkl”, that are the basis of the batch files     that will be deleted

\item {} 
\sphinxstyleliteralstrong{\sphinxupquote{num\_batch}} (\sphinxstyleliteralemphasis{\sphinxupquote{int}}) \textendash{} the number of batches used in the code run

\end{itemize}

\item[{Returns}] \leavevmode


\end{description}\end{quote}

\end{fulllineitems}

\index{get\_batch\_filenames() (in module driver)}

\begin{fulllineitems}
\phantomsection\label{\detokenize{driver:driver.get_batch_filenames}}\pysiglinewithargsret{\sphinxcode{\sphinxupquote{driver.}}\sphinxbfcode{\sphinxupquote{get\_batch\_filenames}}}{\emph{fpath}, \emph{fname}}{}
This file gets the file names for the batch saves.
\begin{quote}\begin{description}
\item[{Parameters}] \leavevmode\begin{itemize}
\item {} 
\sphinxstyleliteralstrong{\sphinxupquote{fpath}} (\sphinxstyleliteralemphasis{\sphinxupquote{str}}) \textendash{} the name of the directory that the batch file names are stored

\item {} 
\sphinxstyleliteralstrong{\sphinxupquote{fname}} (\sphinxstyleliteralemphasis{\sphinxupquote{str}}) \textendash{} the name of the file to save (.pkl)

\end{itemize}

\item[{Returns}] \leavevmode
the batched file names

\item[{Return type}] \leavevmode
list

\end{description}\end{quote}

\end{fulllineitems}

\index{get\_chad\_demo() (in module driver)}

\begin{fulllineitems}
\phantomsection\label{\detokenize{driver:driver.get_chad_demo}}\pysiglinewithargsret{\sphinxcode{\sphinxupquote{driver.}}\sphinxbfcode{\sphinxupquote{get\_chad\_demo}}}{\emph{demographic}}{}
Given the demographic, this function returns the respective CHAD\_demography object.
\begin{quote}\begin{description}
\item[{Parameters}] \leavevmode
\sphinxstyleliteralstrong{\sphinxupquote{demographic}} (\sphinxstyleliteralemphasis{\sphinxupquote{int}}) \textendash{} the demography identifier

\item[{Returns}] \leavevmode
the respective CHAD\_demography object

\end{description}\end{quote}

\end{fulllineitems}

\index{get\_cmd\_line\_params() (in module driver)}

\begin{fulllineitems}
\phantomsection\label{\detokenize{driver:driver.get_cmd_line_params}}\pysiglinewithargsret{\sphinxcode{\sphinxupquote{driver.}}\sphinxbfcode{\sphinxupquote{get\_cmd\_line\_params}}}{}{}
This function gets the parameters from the command line.

The order of arguments to be read on the command line in order:
\begin{enumerate}
\item {} 
the number of processors (threads)

\item {} 
the number of households per batch

\item {} 
the number of batches

\end{enumerate}
\begin{quote}\begin{description}
\item[{Returns}] \leavevmode
the number of processors, the, the total number of households to simulate, the number of batches

\item[{Return type}] \leavevmode
int, int, int

\end{description}\end{quote}

\end{fulllineitems}

\index{get\_current\_batch\_size() (in module driver)}

\begin{fulllineitems}
\phantomsection\label{\detokenize{driver:driver.get_current_batch_size}}\pysiglinewithargsret{\sphinxcode{\sphinxupquote{driver.}}\sphinxbfcode{\sphinxupquote{get\_current\_batch\_size}}}{\emph{num\_hhld}, \emph{idx}, \emph{max\_batch\_size}}{}
This function returns the number of households for the current batch if the total
number of households is a multiple of the number of batches. Each batch
contains max\_batch\_size amount of households.
However, if not, the last batch will be smaller than the number of the max\_batch\_size.
\begin{quote}\begin{description}
\item[{Parameters}] \leavevmode\begin{itemize}
\item {} 
\sphinxstyleliteralstrong{\sphinxupquote{num\_hhld}} (\sphinxstyleliteralemphasis{\sphinxupquote{int}}) \textendash{} the total amount of households in the simulation

\item {} 
\sphinxstyleliteralstrong{\sphinxupquote{idx}} (\sphinxstyleliteralemphasis{\sphinxupquote{int}}) \textendash{} the index of the current batch number

\item {} 
\sphinxstyleliteralstrong{\sphinxupquote{max\_batch\_size}} (\sphinxstyleliteralemphasis{\sphinxupquote{int}}) \textendash{} the maximum number of households per batch

\end{itemize}

\item[{Returns}] \leavevmode
the current batch size

\item[{Return type}] \leavevmode
int

\end{description}\end{quote}

\end{fulllineitems}

\index{get\_fnames() (in module driver)}

\begin{fulllineitems}
\phantomsection\label{\detokenize{driver:driver.get_fnames}}\pysiglinewithargsret{\sphinxcode{\sphinxupquote{driver.}}\sphinxbfcode{\sphinxupquote{get\_fnames}}}{\emph{fpath}, \emph{demographic}, \emph{num\_days}, \emph{N}, \emph{do\_print=False}}{}
Given a directory, this function creates the file names that will be used
to save the ABMHAP trials (input) and the ABMHAP data (output) according
to the respective demographic.
\begin{quote}\begin{description}
\item[{Parameters}] \leavevmode\begin{itemize}
\item {} 
\sphinxstyleliteralstrong{\sphinxupquote{fpath}} (\sphinxstyleliteralemphasis{\sphinxupquote{str}}) \textendash{} the directory in which to save the files

\item {} 
\sphinxstyleliteralstrong{\sphinxupquote{demographic}} (\sphinxstyleliteralemphasis{\sphinxupquote{int}}) \textendash{} the demography identifier

\item {} 
\sphinxstyleliteralstrong{\sphinxupquote{num\_days}} (\sphinxstyleliteralemphasis{\sphinxupquote{int}}) \textendash{} the number of days in the simulation

\item {} 
\sphinxstyleliteralstrong{\sphinxupquote{N}} (\sphinxstyleliteralemphasis{\sphinxupquote{int}}) \textendash{} the total number of households

\item {} 
\sphinxstyleliteralstrong{\sphinxupquote{do\_print}} (\sphinxstyleliteralemphasis{\sphinxupquote{bool}}) \textendash{} a flag to indicate whether (if True) or not     (if False) to print a message to the screen

\end{itemize}

\item[{Returns}] \leavevmode
the file name to save the trials data (“.pkl” extension);     the file name to save the data (“.pkl” extension”),     the file name to save the the basis (no “.pkl” extension) of the file     name to save the trials data,     the basis (no “.pkl” extension) of the file name to save the ABMHAP output data

\item[{Return type}] \leavevmode
str, str, str, str

\end{description}\end{quote}

\end{fulllineitems}

\index{get\_loaded\_trials\_for\_batch() (in module driver)}

\begin{fulllineitems}
\phantomsection\label{\detokenize{driver:driver.get_loaded_trials_for_batch}}\pysiglinewithargsret{\sphinxcode{\sphinxupquote{driver.}}\sphinxbfcode{\sphinxupquote{get\_loaded\_trials\_for\_batch}}}{\emph{loaded\_trials}, \emph{i}, \emph{batch\_size}}{}
This function extracts the household input information from the pre-loaded
input data for the respective batch.
\begin{quote}\begin{description}
\item[{Parameters}] \leavevmode\begin{itemize}
\item {} 
\sphinxstyleliteralstrong{\sphinxupquote{loaded\_trials}} (list of {\hyperref[\detokenize{trial:trial.Trial}]{\sphinxcrossref{\sphinxcode{\sphinxupquote{trial.Trial}}}}}) \textendash{} input data needed for the simulation

\item {} 
\sphinxstyleliteralstrong{\sphinxupquote{i}} (\sphinxstyleliteralemphasis{\sphinxupquote{int}}) \textendash{} the current batch number

\item {} 
\sphinxstyleliteralstrong{\sphinxupquote{batch\_size}} (\sphinxstyleliteralemphasis{\sphinxupquote{int}}) \textendash{} the number of households in the batch

\end{itemize}

\item[{Returns}] \leavevmode
the input data that corresponds to the batch number

\item[{Return type}] \leavevmode
list of {\hyperref[\detokenize{trial:trial.Trial}]{\sphinxcrossref{\sphinxcode{\sphinxupquote{trial.Trial}}}}}

\end{description}\end{quote}

\end{fulllineitems}

\index{get\_max\_batch\_size() (in module driver)}

\begin{fulllineitems}
\phantomsection\label{\detokenize{driver:driver.get_max_batch_size}}\pysiglinewithargsret{\sphinxcode{\sphinxupquote{driver.}}\sphinxbfcode{\sphinxupquote{get\_max\_batch\_size}}}{\emph{num\_hhld}, \emph{num\_batch}}{}
This function returns the maximum number of households
simulated per batch.
\begin{quote}\begin{description}
\item[{Parameters}] \leavevmode\begin{itemize}
\item {} 
\sphinxstyleliteralstrong{\sphinxupquote{num\_hhld}} (\sphinxstyleliteralemphasis{\sphinxupquote{int}}) \textendash{} the total number of households to simulate

\item {} 
\sphinxstyleliteralstrong{\sphinxupquote{num\_batch}} (\sphinxstyleliteralemphasis{\sphinxupquote{int}}) \textendash{} the number of batches

\end{itemize}

\item[{Returns}] \leavevmode
the number of households to simulate per batch

\item[{Return type}] \leavevmode
int

\end{description}\end{quote}

\end{fulllineitems}

\index{get\_results() (in module driver)}

\begin{fulllineitems}
\phantomsection\label{\detokenize{driver:driver.get_results}}\pysiglinewithargsret{\sphinxcode{\sphinxupquote{driver.}}\sphinxbfcode{\sphinxupquote{get\_results}}}{\emph{diaries}, \emph{trials}}{}
This function takes the output and input from the simulation and converts
the data into the appropriate output and input types.
\begin{quote}\begin{description}
\item[{Parameters}] \leavevmode\begin{itemize}
\item {} 
\sphinxstyleliteralstrong{\sphinxupquote{diaries}} (list of {\hyperref[\detokenize{diary:diary.Diary}]{\sphinxcrossref{\sphinxcode{\sphinxupquote{diary.Diary}}}}}) \textendash{} each activity diary (output) in the Monte-Carlo simulation

\item {} 
\sphinxstyleliteralstrong{\sphinxupquote{trials}} (list of {\hyperref[\detokenize{trial:trial.Trial}]{\sphinxcrossref{\sphinxcode{\sphinxupquote{trial.Trial}}}}}) \textendash{} the input data for each household simulation.

\end{itemize}

\item[{Returns}] \leavevmode
the output, the input

\item[{Return type}] \leavevmode
{\hyperref[\detokenize{driver_result:driver_result.Driver_Result}]{\sphinxcrossref{\sphinxcode{\sphinxupquote{driver\_result.Driver\_Result}}}}}, list of {\hyperref[\detokenize{params:params.Params}]{\sphinxcrossref{\sphinxcode{\sphinxupquote{params.Params}}}}}

\end{description}\end{quote}

\end{fulllineitems}

\index{initialize\_trials() (in module driver)}

\begin{fulllineitems}
\phantomsection\label{\detokenize{driver:driver.initialize_trials}}\pysiglinewithargsret{\sphinxcode{\sphinxupquote{driver.}}\sphinxbfcode{\sphinxupquote{initialize\_trials}}}{\emph{param\_list}, \emph{trial\_code}, \emph{chad\_activity\_params}, \emph{demographic}}{}
This function initializes the trials (input parameters) for the simulation.
\begin{quote}\begin{description}
\item[{Parameters}] \leavevmode\begin{itemize}
\item {} 
\sphinxstyleliteralstrong{\sphinxupquote{param\_list}} \textendash{} contains information on how to initialize the     simulation for each household.

\item {} 
\sphinxstyleliteralstrong{\sphinxupquote{trial\_code}} (\sphinxstyleliteralemphasis{\sphinxupquote{int}}) \textendash{} the code of what trial to run

\item {} 
\sphinxstyleliteralstrong{\sphinxupquote{chad\_activity\_params}} ({\hyperref[\detokenize{chad_params:chad_params.CHAD_params}]{\sphinxcrossref{\sphinxstyleliteralemphasis{\sphinxupquote{chad\_params.CHAD\_params}}}}}) \textendash{} the activity parameters used to sample “good” CHAD data

\item {} 
\sphinxstyleliteralstrong{\sphinxupquote{demographic}} (\sphinxstyleliteralemphasis{\sphinxupquote{int}}) \textendash{} this is the code for what demographic to run

\end{itemize}

\item[{Returns}] \leavevmode
the initialized simulation scenarios

\item[{Return type}] \leavevmode
list of {\hyperref[\detokenize{trial:trial.Trial}]{\sphinxcrossref{\sphinxcode{\sphinxupquote{trial.Trial}}}}}

\end{description}\end{quote}

\end{fulllineitems}

\index{is\_batch\_file() (in module driver)}

\begin{fulllineitems}
\phantomsection\label{\detokenize{driver:driver.is_batch_file}}\pysiglinewithargsret{\sphinxcode{\sphinxupquote{driver.}}\sphinxbfcode{\sphinxupquote{is\_batch\_file}}}{\emph{fname}, \emph{extensions}}{}
This function indicates whether or not the filename is a batch file. For example,
given a file name called filename\_b0000.pkl will return True. On the other hand,
filename.pkl will return False.
\begin{quote}\begin{description}
\item[{Parameters}] \leavevmode\begin{itemize}
\item {} 
\sphinxstyleliteralstrong{\sphinxupquote{fname}} (\sphinxstyleliteralemphasis{\sphinxupquote{str}}) \textendash{} the file name

\item {} 
\sphinxstyleliteralstrong{\sphinxupquote{extensions}} (\sphinxstyleliteralemphasis{\sphinxupquote{str}}\sphinxstyleliteralemphasis{\sphinxupquote{, }}\sphinxstyleliteralemphasis{\sphinxupquote{list of str}}) \textendash{} the file extensions for the file name

\end{itemize}

\item[{Returns}] \leavevmode
flag indicating whether the file name is a batch file

\item[{Return type}] \leavevmode
bool

\end{description}\end{quote}

\end{fulllineitems}

\index{load\_trials\_for\_batches() (in module driver)}

\begin{fulllineitems}
\phantomsection\label{\detokenize{driver:driver.load_trials_for_batches}}\pysiglinewithargsret{\sphinxcode{\sphinxupquote{driver.}}\sphinxbfcode{\sphinxupquote{load\_trials\_for\_batches}}}{\emph{fname\_load\_trials\_base}, \emph{num\_batch}, \emph{do\_print}}{}
This function loads pre-existing trials data.
\begin{quote}\begin{description}
\item[{Parameters}] \leavevmode\begin{itemize}
\item {} 
\sphinxstyleliteralstrong{\sphinxupquote{fname\_load\_trials}} (\sphinxstyleliteralemphasis{\sphinxupquote{str}}) \textendash{} the filename (.pkl) of the trials data to load

\item {} 
\sphinxstyleliteralstrong{\sphinxupquote{num\_batch}} (\sphinxstyleliteralemphasis{\sphinxupquote{int}}) \textendash{} the number of batches

\item {} 
\sphinxstyleliteralstrong{\sphinxupquote{do\_print}} (\sphinxstyleliteralemphasis{\sphinxupquote{bool}}) \textendash{} a flag to indicate whether or not to print a message to      screen about the logistics of loading the data

\end{itemize}

\item[{Returns}] \leavevmode
the input data that has been loaded,      the file name for the input data,      the batch size

\item[{Return type}] \leavevmode
list of {\hyperref[\detokenize{trial:trial.Trial}]{\sphinxcrossref{\sphinxcode{\sphinxupquote{trial.Trial}}}}}, str, int

\end{description}\end{quote}

\end{fulllineitems}

\index{print\_end() (in module driver)}

\begin{fulllineitems}
\phantomsection\label{\detokenize{driver:driver.print_end}}\pysiglinewithargsret{\sphinxcode{\sphinxupquote{driver.}}\sphinxbfcode{\sphinxupquote{print\_end}}}{\emph{elapsed\_time}}{}
Print the elapsed time for the simulation message.
\begin{quote}\begin{description}
\item[{Parameters}] \leavevmode
\sphinxstyleliteralstrong{\sphinxupquote{elapsed\_time}} (\sphinxstyleliteralemphasis{\sphinxupquote{float}}) \textendash{} the elapsed time for the simulation {[}seconds{]}

\item[{Returns}] \leavevmode


\end{description}\end{quote}

\end{fulllineitems}

\index{print\_start() (in module driver)}

\begin{fulllineitems}
\phantomsection\label{\detokenize{driver:driver.print_start}}\pysiglinewithargsret{\sphinxcode{\sphinxupquote{driver.}}\sphinxbfcode{\sphinxupquote{print\_start}}}{}{}
Print the message about starting the simulation.
\begin{quote}\begin{description}
\item[{Returns}] \leavevmode


\end{description}\end{quote}

\end{fulllineitems}

\index{print\_starting\_info() (in module driver)}

\begin{fulllineitems}
\phantomsection\label{\detokenize{driver:driver.print_starting_info}}\pysiglinewithargsret{\sphinxcode{\sphinxupquote{driver.}}\sphinxbfcode{\sphinxupquote{print\_starting\_info}}}{\emph{num\_hhld}, \emph{batch\_size}, \emph{num\_batch}, \emph{num\_days}, \emph{num\_process}, \emph{total\_cpus}}{}
Print information before the beginning of the simulation.
\begin{quote}\begin{description}
\item[{Parameters}] \leavevmode\begin{itemize}
\item {} 
\sphinxstyleliteralstrong{\sphinxupquote{num\_hhld}} (\sphinxstyleliteralemphasis{\sphinxupquote{int}}) \textendash{} the total number of households

\item {} 
\sphinxstyleliteralstrong{\sphinxupquote{batch\_size}} (\sphinxstyleliteralemphasis{\sphinxupquote{int}}) \textendash{} the maximum number of households to simulate per batch

\item {} 
\sphinxstyleliteralstrong{\sphinxupquote{num\_hhld\_per\_batch}} (\sphinxstyleliteralemphasis{\sphinxupquote{int}}) \textendash{} the number of households per batch

\item {} 
\sphinxstyleliteralstrong{\sphinxupquote{num\_days}} \textendash{} the number of days in the simulation

\item {} 
\sphinxstyleliteralstrong{\sphinxupquote{num\_process}} \textendash{} the number of processors used

\item {} 
\sphinxstyleliteralstrong{\sphinxupquote{total\_cpus}} \textendash{} the total amount of potential CPUs available.

\end{itemize}

\item[{Returns}] \leavevmode


\end{description}\end{quote}

\end{fulllineitems}

\index{run() (in module driver)}

\begin{fulllineitems}
\phantomsection\label{\detokenize{driver:driver.run}}\pysiglinewithargsret{\sphinxcode{\sphinxupquote{driver.}}\sphinxbfcode{\sphinxupquote{run}}}{\emph{num\_process}, \emph{trials}, \emph{do\_print=False}}{}
This function runs each simulation (in serial or parallel).
\begin{quote}\begin{description}
\item[{Parameters}] \leavevmode\begin{itemize}
\item {} 
\sphinxstyleliteralstrong{\sphinxupquote{num\_process}} (\sphinxstyleliteralemphasis{\sphinxupquote{int}}) \textendash{} the number of processors to use

\item {} 
\sphinxstyleliteralstrong{\sphinxupquote{trials}} (list of {\hyperref[\detokenize{trial:trial.Trial}]{\sphinxcrossref{\sphinxcode{\sphinxupquote{trial.Trial}}}}}) \textendash{} the input for each simulation

\item {} 
\sphinxstyleliteralstrong{\sphinxupquote{do\_print}} (\sphinxstyleliteralemphasis{\sphinxupquote{bool}}) \textendash{} a flag indicating whether to print (if True) or not (if False)

\end{itemize}

\item[{Returns}] \leavevmode
the results of the simulations, the input parameters

\item[{Return type}] \leavevmode
diary\_result.Diary\_result, list of {\hyperref[\detokenize{params:params.Params}]{\sphinxcrossref{\sphinxcode{\sphinxupquote{params.Params}}}}}

\end{description}\end{quote}

\end{fulllineitems}

\index{run\_batch() (in module driver)}

\begin{fulllineitems}
\phantomsection\label{\detokenize{driver:driver.run_batch}}\pysiglinewithargsret{\sphinxcode{\sphinxupquote{driver.}}\sphinxbfcode{\sphinxupquote{run\_batch}}}{\emph{num\_batch}, \emph{num\_hhld}, \emph{num\_process}, \emph{num\_days}, \emph{num\_hours}, \emph{num\_min}, \emph{trial\_code}, \emph{chad\_activity\_params}, \emph{demographic}, \emph{num\_people}, \emph{do\_minute\_by\_minute}, \emph{do\_print}, \emph{do\_save}, \emph{fpath}, \emph{do\_load\_trials=False}, \emph{fname\_load\_trials\_base=None}}{}
Run the simulation in batches.
\begin{quote}\begin{description}
\item[{Parameters}] \leavevmode\begin{itemize}
\item {} 
\sphinxstyleliteralstrong{\sphinxupquote{num\_batch}} (\sphinxstyleliteralemphasis{\sphinxupquote{int}}) \textendash{} the number of batches

\item {} 
\sphinxstyleliteralstrong{\sphinxupquote{num\_hhld}} (\sphinxstyleliteralemphasis{\sphinxupquote{int}}) \textendash{} the total number of households to simulate

\item {} 
\sphinxstyleliteralstrong{\sphinxupquote{num\_process}} (\sphinxstyleliteralemphasis{\sphinxupquote{int}}) \textendash{} the number of processors used

\item {} 
\sphinxstyleliteralstrong{\sphinxupquote{num\_days}} (\sphinxstyleliteralemphasis{\sphinxupquote{int}}) \textendash{} the number of days in the simulation

\item {} 
\sphinxstyleliteralstrong{\sphinxupquote{num\_hours}} (\sphinxstyleliteralemphasis{\sphinxupquote{int}}) \textendash{} the number of additional hours in the simulation

\item {} 
\sphinxstyleliteralstrong{\sphinxupquote{num\_min}} (\sphinxstyleliteralemphasis{\sphinxupquote{int}}) \textendash{} the number of additional minutes in the simulation

\item {} 
\sphinxstyleliteralstrong{\sphinxupquote{trial\_code}} (\sphinxstyleliteralemphasis{\sphinxupquote{int}}) \textendash{} the identifier for the trial being run

\item {} 
\sphinxstyleliteralstrong{\sphinxupquote{chad\_activity\_params}} ({\hyperref[\detokenize{chad_params:chad_params.CHAD_params}]{\sphinxcrossref{\sphinxstyleliteralemphasis{\sphinxupquote{chad\_params.CHAD\_params}}}}}) \textendash{} the activity parameters used to sample “good” CHAD data

\item {} 
\sphinxstyleliteralstrong{\sphinxupquote{demographic}} (\sphinxstyleliteralemphasis{\sphinxupquote{int}}) \textendash{} the demographic identifier

\item {} 
\sphinxstyleliteralstrong{\sphinxupquote{do\_print}} (\sphinxstyleliteralemphasis{\sphinxupquote{bool}}) \textendash{} a flag indicating whether to print (if True) or not (if False)

\item {} 
\sphinxstyleliteralstrong{\sphinxupquote{do\_save}} (\sphinxstyleliteralemphasis{\sphinxupquote{bool}}) \textendash{} flag to save the output

\item {} 
\sphinxstyleliteralstrong{\sphinxupquote{fname\_trials\_base}} (\sphinxstyleliteralemphasis{\sphinxupquote{str}}) \textendash{} the file name for the trials without the .pkl, which will be used for saving the     trial information (.pkl)

\item {} 
\sphinxstyleliteralstrong{\sphinxupquote{fname\_data\_base}} (\sphinxstyleliteralemphasis{\sphinxupquote{str}}) \textendash{} the file name for the ABMHAP without the .pkl, which will be used for saving the     trial information (.pkl)

\item {} 
\sphinxstyleliteralstrong{\sphinxupquote{do\_load\_trials}} (\sphinxstyleliteralemphasis{\sphinxupquote{bool}}) \textendash{} indicating whether (if True) or not (if False) to load trials from a saved     file instead of creating a new set of trials

\item {} 
\sphinxstyleliteralstrong{\sphinxupquote{fname\_load\_trials\_base}} (\sphinxstyleliteralemphasis{\sphinxupquote{str}}) \textendash{} the file name for the ABMHAP trials without the .pkl, which will be used for     saving the trial information (.pkl)

\end{itemize}

\item[{Returns}] \leavevmode
the file name of the input data,     the file name of the output data,     the file name of the input data (no “.pkl”),     the file name of the output data (no “.pkl”)

\item[{Return type}] \leavevmode
str, str, str, str

\end{description}\end{quote}

\end{fulllineitems}

\index{run\_everything() (in module driver)}

\begin{fulllineitems}
\phantomsection\label{\detokenize{driver:driver.run_everything}}\pysiglinewithargsret{\sphinxcode{\sphinxupquote{driver.}}\sphinxbfcode{\sphinxupquote{run\_everything}}}{\emph{num\_process}, \emph{num\_hhld}, \emph{num\_batch}}{}
This code runs the Monte-Carlo simulations. More specifically, it
\begin{enumerate}
\item {} 
creates / loads the input data

\item {} 
runs the simulations

\item {} 
saves both the input and output data

\end{enumerate}
\begin{quote}\begin{description}
\item[{Parameters}] \leavevmode\begin{itemize}
\item {} 
\sphinxstyleliteralstrong{\sphinxupquote{num\_process}} (\sphinxstyleliteralemphasis{\sphinxupquote{int}}) \textendash{} the number of processes

\item {} 
\sphinxstyleliteralstrong{\sphinxupquote{num\_hhld}} (\sphinxstyleliteralemphasis{\sphinxupquote{int}}) \textendash{} the number of households per core per batch

\item {} 
\sphinxstyleliteralstrong{\sphinxupquote{num\_batch}} (\sphinxstyleliteralemphasis{\sphinxupquote{int}}) \textendash{} the number of batches

\end{itemize}

\item[{Returns}] \leavevmode
the file name for the input data, the file name for the output data

\item[{Return type}] \leavevmode
str, str

\end{description}\end{quote}

\end{fulllineitems}

\index{run\_parallel() (in module driver)}

\begin{fulllineitems}
\phantomsection\label{\detokenize{driver:driver.run_parallel}}\pysiglinewithargsret{\sphinxcode{\sphinxupquote{driver.}}\sphinxbfcode{\sphinxupquote{run\_parallel}}}{\emph{num\_process}, \emph{trials}}{}
This function runs the simulation in parallel.
\begin{quote}\begin{description}
\item[{Parameters}] \leavevmode\begin{itemize}
\item {} 
\sphinxstyleliteralstrong{\sphinxupquote{num\_process}} (\sphinxstyleliteralemphasis{\sphinxupquote{int}}) \textendash{} the number of processors used

\item {} 
\sphinxstyleliteralstrong{\sphinxupquote{trials}} (list of {\hyperref[\detokenize{trial:trial.Trial}]{\sphinxcrossref{\sphinxcode{\sphinxupquote{trial.Trial}}}}}) \textendash{} the input data

\end{itemize}

\item[{Returns}] \leavevmode
the output of the simulations

\item[{Return type}] \leavevmode
list of {\hyperref[\detokenize{diary:diary.Diary}]{\sphinxcrossref{\sphinxcode{\sphinxupquote{diary.Diary}}}}}

\end{description}\end{quote}

\end{fulllineitems}

\index{run\_serial() (in module driver)}

\begin{fulllineitems}
\phantomsection\label{\detokenize{driver:driver.run_serial}}\pysiglinewithargsret{\sphinxcode{\sphinxupquote{driver.}}\sphinxbfcode{\sphinxupquote{run\_serial}}}{\emph{trials}, \emph{do\_print=False}}{}
This function runs the simulation in serial.
\begin{quote}\begin{description}
\item[{Parameters}] \leavevmode\begin{itemize}
\item {} 
\sphinxstyleliteralstrong{\sphinxupquote{trials}} (list of {\hyperref[\detokenize{trial:trial.Trial}]{\sphinxcrossref{\sphinxcode{\sphinxupquote{trial.Trial}}}}}) \textendash{} the input data

\item {} 
\sphinxstyleliteralstrong{\sphinxupquote{do\_print}} (\sphinxstyleliteralemphasis{\sphinxupquote{bool}}) \textendash{} a flag whether or not to print the trial number

\end{itemize}

\item[{Returns}] \leavevmode
the output of the simulations

\item[{Return type}] \leavevmode
list of {\hyperref[\detokenize{diary:diary.Diary}]{\sphinxcrossref{\sphinxcode{\sphinxupquote{diary.Diary}}}}}

\end{description}\end{quote}

\end{fulllineitems}

\index{run\_trials\_parallel() (in module driver)}

\begin{fulllineitems}
\phantomsection\label{\detokenize{driver:driver.run_trials_parallel}}\pysiglinewithargsret{\sphinxcode{\sphinxupquote{driver.}}\sphinxbfcode{\sphinxupquote{run\_trials\_parallel}}}{\emph{t}}{}
This function is called in order to run the trials in parallel.
\begin{quote}\begin{description}
\item[{Parameters}] \leavevmode
\sphinxstyleliteralstrong{\sphinxupquote{t}} ({\hyperref[\detokenize{trial:trial.Trial}]{\sphinxcrossref{\sphinxstyleliteralemphasis{\sphinxupquote{trial.Trial}}}}}) \textendash{} the trial to run

\item[{Returns}] \leavevmode
the results of the simulation

\item[{Return type}] \leavevmode
{\hyperref[\detokenize{diary:diary.Diary}]{\sphinxcrossref{diary.Diary}}}

\end{description}\end{quote}

\end{fulllineitems}

\index{save() (in module driver)}

\begin{fulllineitems}
\phantomsection\label{\detokenize{driver:driver.save}}\pysiglinewithargsret{\sphinxcode{\sphinxupquote{driver.}}\sphinxbfcode{\sphinxupquote{save}}}{\emph{fname\_data}, \emph{fname\_trials}, \emph{fname\_data\_base}, \emph{fname\_trials\_base}, \emph{num\_batch}, \emph{do\_print=False}}{}
This function saves the input and output from the simulation. It merges     the data from the batch save files into one file for not only the     trials data (input) but also the ABMHAP simulation data (output). Afterwards,     the individual batch files are deleted.
\begin{quote}\begin{description}
\item[{Parameters}] \leavevmode\begin{itemize}
\item {} 
\sphinxstyleliteralstrong{\sphinxupquote{fname\_data}} (\sphinxstyleliteralemphasis{\sphinxupquote{str}}) \textendash{} the file name in which to save the ABMHAP data (output)

\item {} 
\sphinxstyleliteralstrong{\sphinxupquote{fname\_trials}} (\sphinxstyleliteralemphasis{\sphinxupquote{str}}) \textendash{} the file name in which to save the ABMHAP trials (input)

\item {} 
\sphinxstyleliteralstrong{\sphinxupquote{fname\_data\_base}} (\sphinxstyleliteralemphasis{\sphinxupquote{str}}) \textendash{} the base (no “.pkl” extension) of the file name in     which to save the ABMHAP data (output)

\item {} 
\sphinxstyleliteralstrong{\sphinxupquote{fname\_trials\_base}} (\sphinxstyleliteralemphasis{\sphinxupquote{str}}) \textendash{} the base (no “.pkl” extension) of file name in     which to save the ABMHAP trials (input)

\item {} 
\sphinxstyleliteralstrong{\sphinxupquote{do\_print}} (\sphinxstyleliteralemphasis{\sphinxupquote{bool}}) \textendash{} print flag

\end{itemize}

\item[{Returns}] \leavevmode


\end{description}\end{quote}

\end{fulllineitems}

\index{save\_batch\_data\_as\_one\_file() (in module driver)}

\begin{fulllineitems}
\phantomsection\label{\detokenize{driver:driver.save_batch_data_as_one_file}}\pysiglinewithargsret{\sphinxcode{\sphinxupquote{driver.}}\sphinxbfcode{\sphinxupquote{save\_batch\_data\_as\_one\_file}}}{\emph{fname}, \emph{do\_print=False}}{}
The function combines the ABMHAP data from the individual batch saves and saves them in one file.
\begin{quote}\begin{description}
\item[{Parameters}] \leavevmode
\sphinxstyleliteralstrong{\sphinxupquote{fname}} (\sphinxstyleliteralemphasis{\sphinxupquote{str}}) \textendash{} the file name of the ABMHAP data (.pkl)

\item[{Returns}] \leavevmode


\end{description}\end{quote}

\end{fulllineitems}

\index{save\_batch\_trials\_as\_one\_file() (in module driver)}

\begin{fulllineitems}
\phantomsection\label{\detokenize{driver:driver.save_batch_trials_as_one_file}}\pysiglinewithargsret{\sphinxcode{\sphinxupquote{driver.}}\sphinxbfcode{\sphinxupquote{save\_batch\_trials\_as\_one\_file}}}{\emph{fname}, \emph{do\_print=False}}{}
This function combines the trial (input) data from the individual     batch saves and saves them in one file.
\begin{quote}\begin{description}
\item[{Parameters}] \leavevmode\begin{itemize}
\item {} 
\sphinxstyleliteralstrong{\sphinxupquote{fname}} (\sphinxstyleliteralemphasis{\sphinxupquote{str}}) \textendash{} the file name of the trials data (.pkl)

\item {} 
\sphinxstyleliteralstrong{\sphinxupquote{do\_print}} (\sphinxstyleliteralemphasis{\sphinxupquote{bool}}) \textendash{} print flag

\end{itemize}

\item[{Returns}] \leavevmode


\end{description}\end{quote}

\end{fulllineitems}

\index{save\_diary\_to\_csv() (in module driver)}

\begin{fulllineitems}
\phantomsection\label{\detokenize{driver:driver.save_diary_to_csv}}\pysiglinewithargsret{\sphinxcode{\sphinxupquote{driver.}}\sphinxbfcode{\sphinxupquote{save\_diary\_to\_csv}}}{\emph{fname}}{}
This function loads an activity diary from a compressed file format     and saves it as a .csv file.
\begin{quote}\begin{description}
\item[{Parameters}] \leavevmode
\sphinxstyleliteralstrong{\sphinxupquote{fname}} (\sphinxstyleliteralemphasis{\sphinxupquote{str}}) \textendash{} the pickle file that holds the activity diary file.

\item[{Returns}] \leavevmode


\end{description}\end{quote}

\end{fulllineitems}

\index{save\_for\_batch() (in module driver)}

\begin{fulllineitems}
\phantomsection\label{\detokenize{driver:driver.save_for_batch}}\pysiglinewithargsret{\sphinxcode{\sphinxupquote{driver.}}\sphinxbfcode{\sphinxupquote{save\_for\_batch}}}{\emph{result}, \emph{fname}, \emph{do\_print=False}}{}
Save the data for the current batch.
\begin{quote}\begin{description}
\item[{Parameters}] \leavevmode\begin{itemize}
\item {} 
\sphinxstyleliteralstrong{\sphinxupquote{result}} ({\hyperref[\detokenize{driver_result:driver_result.Driver_Result}]{\sphinxcrossref{\sphinxstyleliteralemphasis{\sphinxupquote{driver\_result.Driver\_Result}}}}}) \textendash{} the result of the simulation for the current batch

\item {} 
\sphinxstyleliteralstrong{\sphinxupquote{fname}} (\sphinxstyleliteralemphasis{\sphinxupquote{str}}) \textendash{} the file name to save the data for the current batch

\item {} 
\sphinxstyleliteralstrong{\sphinxupquote{do\_print}} (\sphinxstyleliteralemphasis{\sphinxupquote{bool}}) \textendash{} print flag

\end{itemize}

\item[{Returns}] \leavevmode


\end{description}\end{quote}

\end{fulllineitems}

\index{set\_save\_files\_for\_batch() (in module driver)}

\begin{fulllineitems}
\phantomsection\label{\detokenize{driver:driver.set_save_files_for_batch}}\pysiglinewithargsret{\sphinxcode{\sphinxupquote{driver.}}\sphinxbfcode{\sphinxupquote{set\_save\_files\_for\_batch}}}{\emph{fname\_trials\_base}, \emph{fname\_data\_base}, \emph{i}, \emph{do\_print=False}}{}
This function sets the save files (inputs and output files) for the current batch.
\begin{quote}\begin{description}
\item[{Parameters}] \leavevmode\begin{itemize}
\item {} 
\sphinxstyleliteralstrong{\sphinxupquote{fname\_trials\_base}} (\sphinxstyleliteralemphasis{\sphinxupquote{str}}) \textendash{} the file name for the ABMHAP data without the .pkl,     which will be used for saving the input data (.pkl)

\item {} 
\sphinxstyleliteralstrong{\sphinxupquote{fname\_data\_base}} (\sphinxstyleliteralemphasis{\sphinxupquote{str}}) \textendash{} the file name for the ABMHAP data without the .pkl,     which will be used for saving the output data (.pkl)

\item {} 
\sphinxstyleliteralstrong{\sphinxupquote{i}} (\sphinxstyleliteralemphasis{\sphinxupquote{int}}) \textendash{} the current batch index

\item {} 
\sphinxstyleliteralstrong{\sphinxupquote{do\_print}} (\sphinxstyleliteralemphasis{\sphinxupquote{bool}}) \textendash{} flag indicating whether or not to print relevant     information to the console

\end{itemize}

\item[{Returns}] \leavevmode
the save file name for the input data,     the save file name for the output data

\item[{Return type}] \leavevmode
str, str

\end{description}\end{quote}

\end{fulllineitems}

\index{set\_save\_path() (in module driver)}

\begin{fulllineitems}
\phantomsection\label{\detokenize{driver:driver.set_save_path}}\pysiglinewithargsret{\sphinxcode{\sphinxupquote{driver.}}\sphinxbfcode{\sphinxupquote{set\_save\_path}}}{\emph{fpath}, \emph{N}, \emph{num\_days}}{}
This function sets the save path for the data. Given a save path, the function appends it by adding an     extension of the current year, month, day, number of households, and number of days in the format.

For example, if this code is being run to simulate 64 households for 100 days on July 4, 2017 and the     file path is “output\_path”, the file path is set to the following: //output\_path//2017\_07\_04//n0064\_d100.
\begin{quote}\begin{description}
\item[{Parameters}] \leavevmode\begin{itemize}
\item {} 
\sphinxstyleliteralstrong{\sphinxupquote{fpath}} (\sphinxstyleliteralemphasis{\sphinxupquote{str}}) \textendash{} the file path in which to save the data

\item {} 
\sphinxstyleliteralstrong{\sphinxupquote{N}} (\sphinxstyleliteralemphasis{\sphinxupquote{int}}) \textendash{} the number of households

\item {} 
\sphinxstyleliteralstrong{\sphinxupquote{num\_days}} (\sphinxstyleliteralemphasis{\sphinxupquote{int}}) \textendash{} the number of days in the simulation

\end{itemize}

\item[{Returns}] \leavevmode
the file directory in the format ‘//output\_file\_path//YYYY\_MM\_DD//nXXXX\_dXXX’

\item[{Return type}] \leavevmode
str

\end{description}\end{quote}

\end{fulllineitems}



\subsection{driver\_params module}
\label{\detokenize{driver_params::doc}}\label{\detokenize{driver_params:driver-params-module}}\label{\detokenize{driver_params:module-driver_params}}\index{driver\_params (module)}
This module is responsible for containing parameters that driver.py uses to control the simulation. The user should set the parameters in this module \sphinxstylestrong{before} running the driver \sphinxcode{\sphinxupquote{driver.py}}.


\subsection{driver\_result module}
\label{\detokenize{driver_result::doc}}\label{\detokenize{driver_result:module-driver_result}}\label{\detokenize{driver_result:driver-result-module}}\index{driver\_result (module)}
This module holds the results from running the Monte-Carlo simulations.

This module contains class {\hyperref[\detokenize{driver_result:driver_result.Driver_Result}]{\sphinxcrossref{\sphinxcode{\sphinxupquote{driver\_result.Driver\_Result}}}}} and {\hyperref[\detokenize{driver_result:driver_result.Batch_Result}]{\sphinxcrossref{\sphinxcode{\sphinxupquote{driver\_result.Batch\_Result}}}}}.
\index{Batch\_Result (class in driver\_result)}

\begin{fulllineitems}
\phantomsection\label{\detokenize{driver_result:driver_result.Batch_Result}}\pysiglinewithargsret{\sphinxbfcode{\sphinxupquote{class }}\sphinxcode{\sphinxupquote{driver\_result.}}\sphinxbfcode{\sphinxupquote{Batch\_Result}}}{\emph{dr\_list}}{}
Bases: {\hyperref[\detokenize{driver_result:driver_result.Driver_Result}]{\sphinxcrossref{\sphinxcode{\sphinxupquote{driver\_result.Driver\_Result}}}}}

This class holds the results from batch runs from the driver in one object.
\begin{quote}\begin{description}
\item[{Parameters}] \leavevmode
\sphinxstyleliteralstrong{\sphinxupquote{dr\_list}} (list of {\hyperref[\detokenize{driver_result:driver_result.Driver_Result}]{\sphinxcrossref{\sphinxcode{\sphinxupquote{driver\_result.Driver\_Result}}}}}) \textendash{} the results from the simulation from each batch that was used.

\end{description}\end{quote}

\end{fulllineitems}

\index{Driver\_Result (class in driver\_result)}

\begin{fulllineitems}
\phantomsection\label{\detokenize{driver_result:driver_result.Driver_Result}}\pysiglinewithargsret{\sphinxbfcode{\sphinxupquote{class }}\sphinxcode{\sphinxupquote{driver\_result.}}\sphinxbfcode{\sphinxupquote{Driver\_Result}}}{\emph{diaries}, \emph{chad\_param\_list}, \emph{demographic}}{}
Bases: \sphinxcode{\sphinxupquote{object}}

This class holds the result of running driver.run().
\begin{quote}\begin{description}
\item[{Parameters}] \leavevmode\begin{itemize}
\item {} 
\sphinxstyleliteralstrong{\sphinxupquote{diaries}} (list of {\hyperref[\detokenize{diary:diary.Diary}]{\sphinxcrossref{\sphinxcode{\sphinxupquote{diary.Diary}}}}}) \textendash{} the activity diaries for each household in the simulation

\item {} 
\sphinxstyleliteralstrong{\sphinxupquote{chad\_param\_list}} (list of {\hyperref[\detokenize{chad_params:chad_params.CHAD_params}]{\sphinxcrossref{\sphinxcode{\sphinxupquote{chad\_params.CHAD\_params}}}}}) \textendash{} the CHAD parameters used for sampling the CHAD data

\item {} 
\sphinxstyleliteralstrong{\sphinxupquote{demographic}} (\sphinxstyleliteralemphasis{\sphinxupquote{int}}) \textendash{} the demography identifier

\end{itemize}

\item[{Variables}] \leavevmode\begin{itemize}
\item {} 
\sphinxstyleliteralstrong{\sphinxupquote{diaries}} \textendash{} the activity diaries for each household in the simulation

\item {} 
\sphinxstyleliteralstrong{\sphinxupquote{chad\_param\_list}} \textendash{} the CHAD parameters used for sampling the CHAD data

\item {} 
\sphinxstyleliteralstrong{\sphinxupquote{demographic}} (\sphinxstyleliteralemphasis{\sphinxupquote{int}}) \textendash{} the demography identifier

\item {} 
\sphinxstyleliteralstrong{\sphinxupquote{num\_hhld}} (\sphinxstyleliteralemphasis{\sphinxupquote{int}}) \textendash{} the number of households

\item {} 
\sphinxstyleliteralstrong{\sphinxupquote{num\_people}} (\sphinxstyleliteralemphasis{\sphinxupquote{int}}) \textendash{} the number of people in the simulation

\end{itemize}

\end{description}\end{quote}
\index{add\_id() (driver\_result.Driver\_Result method)}

\begin{fulllineitems}
\phantomsection\label{\detokenize{driver_result:driver_result.Driver_Result.add_id}}\pysiglinewithargsret{\sphinxbfcode{\sphinxupquote{add\_id}}}{\emph{df\_list}}{}
This function adds an integer identifier to each simulated agent’s activity diary.
\begin{quote}\begin{description}
\item[{Parameters}] \leavevmode
\sphinxstyleliteralstrong{\sphinxupquote{df\_list}} (\sphinxstyleliteralemphasis{\sphinxupquote{list of pandas.core.frame.DataFrame}}) \textendash{} the activity diaries for the simulated agents

\item[{Returns}] \leavevmode
the updated activity diaries for each agent

\item[{Return type}] \leavevmode
list of pandas.core.frame.DataFrame

\end{description}\end{quote}

\end{fulllineitems}

\index{get\_all\_data() (driver\_result.Driver\_Result method)}

\begin{fulllineitems}
\phantomsection\label{\detokenize{driver_result:driver_result.Driver_Result.get_all_data}}\pysiglinewithargsret{\sphinxbfcode{\sphinxupquote{get\_all\_data}}}{}{}
This function returns the diaries as a pandas data frame.
\begin{quote}\begin{description}
\item[{Returns}] \leavevmode
activity diaries for each person in the simulation

\end{description}\end{quote}

:rtype list of pandas.core.frame.DataFrame

\end{fulllineitems}

\index{get\_combined\_diary() (driver\_result.Driver\_Result method)}

\begin{fulllineitems}
\phantomsection\label{\detokenize{driver_result:driver_result.Driver_Result.get_combined_diary}}\pysiglinewithargsret{\sphinxbfcode{\sphinxupquote{get\_combined\_diary}}}{}{}
This function combines all of the activity diaries from the simulation into one.
\begin{quote}\begin{description}
\item[{Returns}] \leavevmode
all of the activity diaries from the simulated agents combine into one* dataframe

\item[{Return type}] \leavevmode
pandas.core.frame.DataFrame

\end{description}\end{quote}

\end{fulllineitems}


\end{fulllineitems}



\subsection{eat\_breakfast\_trial module}
\label{\detokenize{eat_breakfast_trial::doc}}\label{\detokenize{eat_breakfast_trial:module-eat_breakfast_trial}}\label{\detokenize{eat_breakfast_trial:eat-breakfast-trial-module}}\index{eat\_breakfast\_trial (module)}
This module contains code in order to run Monte-Carlo simulations to comparing the Agent-Based Model of Human Activity Patterns (ABMHAP) with the data from the Consolidated Human Activity Database (CHAD) for the \sphinxstylestrong{eat breakfast} activity.

This module contains class {\hyperref[\detokenize{eat_breakfast_trial:eat_breakfast_trial.Eat_Breakfast_Trial}]{\sphinxcrossref{\sphinxcode{\sphinxupquote{eat\_breakfast\_trial.Eat\_Breakfast\_Trial}}}}}.
\index{Eat\_Breakfast\_Trial (class in eat\_breakfast\_trial)}

\begin{fulllineitems}
\phantomsection\label{\detokenize{eat_breakfast_trial:eat_breakfast_trial.Eat_Breakfast_Trial}}\pysiglinewithargsret{\sphinxbfcode{\sphinxupquote{class }}\sphinxcode{\sphinxupquote{eat\_breakfast\_trial.}}\sphinxbfcode{\sphinxupquote{Eat\_Breakfast\_Trial}}}{\emph{parameters}, \emph{sampling\_params}, \emph{demographic}}{}
Bases: {\hyperref[\detokenize{trial:trial.Trial}]{\sphinxcrossref{\sphinxcode{\sphinxupquote{trial.Trial}}}}}

This class runs the ABMHAP simulations initialized with eat breakfast data from CHAD.
\begin{quote}\begin{description}
\item[{Parameters}] \leavevmode\begin{itemize}
\item {} 
\sphinxstyleliteralstrong{\sphinxupquote{paramters}} ({\hyperref[\detokenize{params:params.Params}]{\sphinxcrossref{\sphinxstyleliteralemphasis{\sphinxupquote{params.Params}}}}}) \textendash{} the parameters describing each person in the household

\item {} 
\sphinxstyleliteralstrong{\sphinxupquote{sampling\_params}} ({\hyperref[\detokenize{chad_params:chad_params.CHAD_params}]{\sphinxcrossref{\sphinxstyleliteralemphasis{\sphinxupquote{chad\_params.CHAD\_params}}}}}) \textendash{} the sampling parameters used to filter “good” CHAD     eat breakfast data

\item {} 
\sphinxstyleliteralstrong{\sphinxupquote{demographic}} (\sphinxstyleliteralemphasis{\sphinxupquote{int}}) \textendash{} the demographic identifier

\end{itemize}

\end{description}\end{quote}
\index{adjust\_params() (eat\_breakfast\_trial.Eat\_Breakfast\_Trial method)}

\begin{fulllineitems}
\phantomsection\label{\detokenize{eat_breakfast_trial:eat_breakfast_trial.Eat_Breakfast_Trial.adjust_params}}\pysiglinewithargsret{\sphinxbfcode{\sphinxupquote{adjust\_params}}}{\emph{start\_mean}, \emph{start\_std}, \emph{dt\_mean}, \emph{dt\_std}}{}
This function adjusts the values for the mean and standard deviation of both eat breakfast         duration and eat breakfast start time in the key-word arguments based on the CHAD data         that was sampled. These new values will be used in the runs.
\begin{quote}\begin{description}
\item[{Parameters}] \leavevmode\begin{itemize}
\item {} 
\sphinxstyleliteralstrong{\sphinxupquote{start\_mean}} (\sphinxstyleliteralemphasis{\sphinxupquote{numpy.ndarray}}) \textendash{} the mean eat breakfast start time {[}hours{]} for each person

\item {} 
\sphinxstyleliteralstrong{\sphinxupquote{start\_std}} (\sphinxstyleliteralemphasis{\sphinxupquote{numpy.ndarray}}) \textendash{} the standard deviation of eat breakfast start time {[}hours{]} for each person

\item {} 
\sphinxstyleliteralstrong{\sphinxupquote{dt\_mean}} (\sphinxstyleliteralemphasis{\sphinxupquote{numpy.ndarray}}) \textendash{} the eat breakfast mean duration {[}hours{]} for each person

\item {} 
\sphinxstyleliteralstrong{\sphinxupquote{dt\_std}} (\sphinxstyleliteralemphasis{\sphinxupquote{numpy.ndarray}}) \textendash{} the eat breakfast standard deviation of duration {[}hours{]} for each person

\end{itemize}

\item[{Returns}] \leavevmode


\end{description}\end{quote}

\end{fulllineitems}

\index{create\_universe() (eat\_breakfast\_trial.Eat\_Breakfast\_Trial method)}

\begin{fulllineitems}
\phantomsection\label{\detokenize{eat_breakfast_trial:eat_breakfast_trial.Eat_Breakfast_Trial.create_universe}}\pysiglinewithargsret{\sphinxbfcode{\sphinxupquote{create\_universe}}}{}{}
This function creates a universe object that simulations will run in. The only asset in this         simulation for an agent to use is a {\hyperref[\detokenize{food:food.Food}]{\sphinxcrossref{\sphinxcode{\sphinxupquote{food.Food}}}}}.
\begin{quote}\begin{description}
\item[{Returns}] \leavevmode
the universe

\item[{Return type}] \leavevmode
{\hyperref[\detokenize{universe:universe.Universe}]{\sphinxcrossref{universe.Universe}}}

\end{description}\end{quote}

\end{fulllineitems}

\index{initialize() (eat\_breakfast\_trial.Eat\_Breakfast\_Trial method)}

\begin{fulllineitems}
\phantomsection\label{\detokenize{eat_breakfast_trial:eat_breakfast_trial.Eat_Breakfast_Trial.initialize}}\pysiglinewithargsret{\sphinxbfcode{\sphinxupquote{initialize}}}{}{}
This function sets up the trial
\begin{enumerate}
\item {} 
gets the CHAD data for eat breakfast under the appropriate conditions for means and standard deviations         for both eat breakfast duration and eat breakfast start time

\item {} 
gets N samples the CHAD data for eat breakfast duration and eat breakfast start time for the N trials

\item {} 
updates the {\hyperref[\detokenize{params:module-params}]{\sphinxcrossref{\sphinxcode{\sphinxupquote{params}}}}} to reflect the newly assigned eat breakfast parameters for the simulation

\end{enumerate}
\begin{quote}\begin{description}
\item[{Parameters}] \leavevmode\begin{itemize}
\item {} 
\sphinxstyleliteralstrong{\sphinxupquote{fname\_dt}} (\sphinxstyleliteralemphasis{\sphinxupquote{str}}) \textendash{} the filename of the duration statistics

\item {} 
\sphinxstyleliteralstrong{\sphinxupquote{fname\_start}} (\sphinxstyleliteralemphasis{\sphinxupquote{str}}) \textendash{} the filename of the start time statistics

\end{itemize}

\item[{Returns}] \leavevmode


\end{description}\end{quote}

\end{fulllineitems}

\index{initialize\_person() (eat\_breakfast\_trial.Eat\_Breakfast\_Trial method)}

\begin{fulllineitems}
\phantomsection\label{\detokenize{eat_breakfast_trial:eat_breakfast_trial.Eat_Breakfast_Trial.initialize_person}}\pysiglinewithargsret{\sphinxbfcode{\sphinxupquote{initialize\_person}}}{\emph{u}, \emph{idx}}{}
This function creates and initializes a person with the proper parameters for the Eat Breakfast Trial        simulation. This is important because it changes the meal structure towards having only 1 meal per day.

More specifically, the function does
\begin{enumerate}
\item {} 
creates a person

\item {} 
initializes the person’s parameters to the respective values in {\hyperref[\detokenize{params:module-params}]{\sphinxcrossref{\sphinxcode{\sphinxupquote{params}}}}}

\end{enumerate}
\begin{quote}\begin{description}
\item[{Parameters}] \leavevmode\begin{itemize}
\item {} 
\sphinxstyleliteralstrong{\sphinxupquote{u}} ({\hyperref[\detokenize{universe:universe.Universe}]{\sphinxcrossref{\sphinxstyleliteralemphasis{\sphinxupquote{universe.Universe}}}}}) \textendash{} the universe the person will reside in

\item {} 
\sphinxstyleliteralstrong{\sphinxupquote{idx}} (\sphinxstyleliteralemphasis{\sphinxupquote{int}}) \textendash{} the index of the person’s parameters in {\hyperref[\detokenize{params:module-params}]{\sphinxcrossref{\sphinxcode{\sphinxupquote{params}}}}}

\end{itemize}

\item[{Return p}] \leavevmode
the agent to simulate

\item[{Return type}] \leavevmode
{\hyperref[\detokenize{person:person.Person}]{\sphinxcrossref{person.Person}}}

\end{description}\end{quote}

\end{fulllineitems}


\end{fulllineitems}



\subsection{eat\_dinner\_trial module}
\label{\detokenize{eat_dinner_trial::doc}}\label{\detokenize{eat_dinner_trial:module-eat_dinner_trial}}\label{\detokenize{eat_dinner_trial:eat-dinner-trial-module}}\index{eat\_dinner\_trial (module)}
This module contains code in order to run Monte-Carlo simulations to comparing the Agent-Based Model of Human Activity Patterns (ABMHAP) with the data from the Consolidated Human Activity Database (CHAD) for the \sphinxstylestrong{eat dinner} activity.

This module contains class {\hyperref[\detokenize{eat_dinner_trial:eat_dinner_trial.Eat_Dinner_Trial}]{\sphinxcrossref{\sphinxcode{\sphinxupquote{eat\_dinner\_trial.Eat\_Dinner\_Trial}}}}}.
\index{Eat\_Dinner\_Trial (class in eat\_dinner\_trial)}

\begin{fulllineitems}
\phantomsection\label{\detokenize{eat_dinner_trial:eat_dinner_trial.Eat_Dinner_Trial}}\pysiglinewithargsret{\sphinxbfcode{\sphinxupquote{class }}\sphinxcode{\sphinxupquote{eat\_dinner\_trial.}}\sphinxbfcode{\sphinxupquote{Eat\_Dinner\_Trial}}}{\emph{parameters}, \emph{sampling\_params}, \emph{demographic}}{}
Bases: {\hyperref[\detokenize{trial:trial.Trial}]{\sphinxcrossref{\sphinxcode{\sphinxupquote{trial.Trial}}}}}

This class runs the ABMHAP simulations initialized with eat dinner data from CHAD.
\begin{quote}\begin{description}
\item[{Parameters}] \leavevmode\begin{itemize}
\item {} 
\sphinxstyleliteralstrong{\sphinxupquote{paramters}} ({\hyperref[\detokenize{params:params.Params}]{\sphinxcrossref{\sphinxstyleliteralemphasis{\sphinxupquote{params.Params}}}}}) \textendash{} the parameters describing each person in the household

\item {} 
\sphinxstyleliteralstrong{\sphinxupquote{sampling\_params}} ({\hyperref[\detokenize{chad_params:chad_params.CHAD_params}]{\sphinxcrossref{\sphinxstyleliteralemphasis{\sphinxupquote{chad\_params.CHAD\_params}}}}}) \textendash{} the sampling parameters used to filter “good” CHAD     eat dinner data

\item {} 
\sphinxstyleliteralstrong{\sphinxupquote{demographic}} (\sphinxstyleliteralemphasis{\sphinxupquote{int}}) \textendash{} the demographic identifier

\end{itemize}

\end{description}\end{quote}
\index{adjust\_params() (eat\_dinner\_trial.Eat\_Dinner\_Trial method)}

\begin{fulllineitems}
\phantomsection\label{\detokenize{eat_dinner_trial:eat_dinner_trial.Eat_Dinner_Trial.adjust_params}}\pysiglinewithargsret{\sphinxbfcode{\sphinxupquote{adjust\_params}}}{\emph{start\_mean}, \emph{start\_std}, \emph{dt\_mean}, \emph{dt\_std}}{}
This function adjusts the values for the mean and standard deviation of both eat dinner         duration and eat dinner start time in the key-word arguments based on the CHAD data         that was sampled. These new values will be used in the runs.
\begin{quote}\begin{description}
\item[{Parameters}] \leavevmode\begin{itemize}
\item {} 
\sphinxstyleliteralstrong{\sphinxupquote{start\_mean}} (\sphinxstyleliteralemphasis{\sphinxupquote{numpy.ndarray}}) \textendash{} the mean eat dinner start time {[}hours{]} for each person

\item {} 
\sphinxstyleliteralstrong{\sphinxupquote{start\_std}} (\sphinxstyleliteralemphasis{\sphinxupquote{numpy.ndarray}}) \textendash{} the standard deviation of eat dinner start time {[}hours{]} for each person

\item {} 
\sphinxstyleliteralstrong{\sphinxupquote{dt\_mean}} (\sphinxstyleliteralemphasis{\sphinxupquote{numpy.ndarray}}) \textendash{} the eat dinner mean duration {[}hours{]} for each person

\item {} 
\sphinxstyleliteralstrong{\sphinxupquote{dt\_std}} (\sphinxstyleliteralemphasis{\sphinxupquote{numpy.ndarray}}) \textendash{} the eat dinner standard deviation of duration {[}hours{]} for each person

\end{itemize}

\item[{Returns}] \leavevmode


\end{description}\end{quote}

\end{fulllineitems}

\index{create\_universe() (eat\_dinner\_trial.Eat\_Dinner\_Trial method)}

\begin{fulllineitems}
\phantomsection\label{\detokenize{eat_dinner_trial:eat_dinner_trial.Eat_Dinner_Trial.create_universe}}\pysiglinewithargsret{\sphinxbfcode{\sphinxupquote{create\_universe}}}{}{}
This function creates a universe object that simulations will run in. The only asset in this         simulation for an agent to use is a {\hyperref[\detokenize{food:food.Food}]{\sphinxcrossref{\sphinxcode{\sphinxupquote{food.Food}}}}}.
\begin{quote}\begin{description}
\item[{Returns}] \leavevmode
the universe

\item[{Return type}] \leavevmode
{\hyperref[\detokenize{universe:universe.Universe}]{\sphinxcrossref{universe.Universe}}}

\end{description}\end{quote}

\end{fulllineitems}

\index{initialize() (eat\_dinner\_trial.Eat\_Dinner\_Trial method)}

\begin{fulllineitems}
\phantomsection\label{\detokenize{eat_dinner_trial:eat_dinner_trial.Eat_Dinner_Trial.initialize}}\pysiglinewithargsret{\sphinxbfcode{\sphinxupquote{initialize}}}{}{}
This function sets up the trial
\begin{enumerate}
\item {} 
gets the CHAD data for eat dinner under the appropriate conditions for means and standard deviations         for both eat dinner duration and eat dinner start time

\item {} 
gets N samples the CHAD data for eat dinner duration and eat dinner start time for the N trials

\item {} 
updates the {\hyperref[\detokenize{params:module-params}]{\sphinxcrossref{\sphinxcode{\sphinxupquote{params}}}}} to reflect the newly assigned eat dinner parameters for the simulation

\end{enumerate}
\begin{quote}\begin{description}
\item[{Returns}] \leavevmode


\end{description}\end{quote}

\end{fulllineitems}

\index{initialize\_person() (eat\_dinner\_trial.Eat\_Dinner\_Trial method)}

\begin{fulllineitems}
\phantomsection\label{\detokenize{eat_dinner_trial:eat_dinner_trial.Eat_Dinner_Trial.initialize_person}}\pysiglinewithargsret{\sphinxbfcode{\sphinxupquote{initialize\_person}}}{\emph{u}, \emph{idx}}{}
This function creates and initializes a person with the proper parameters for the Eat Dinner Trial        simulation. This is necessary because it changes the meal structure to having only one meal per day.

More specifically, the function does
\begin{enumerate}
\item {} 
creates a {\hyperref[\detokenize{singleton:singleton.Singleton}]{\sphinxcrossref{\sphinxcode{\sphinxupquote{singleton.Singleton}}}}} person

\item {} 
initializes the person’s parameters to the respective values in {\hyperref[\detokenize{params:module-params}]{\sphinxcrossref{\sphinxcode{\sphinxupquote{params}}}}}

\end{enumerate}
\begin{quote}\begin{description}
\item[{Parameters}] \leavevmode\begin{itemize}
\item {} 
\sphinxstyleliteralstrong{\sphinxupquote{u}} ({\hyperref[\detokenize{universe:universe.Universe}]{\sphinxcrossref{\sphinxstyleliteralemphasis{\sphinxupquote{universe.Universe}}}}}) \textendash{} the universe the person will reside in

\item {} 
\sphinxstyleliteralstrong{\sphinxupquote{idx}} (\sphinxstyleliteralemphasis{\sphinxupquote{int}}) \textendash{} the index of the person’s parameters in {\hyperref[\detokenize{params:module-params}]{\sphinxcrossref{\sphinxcode{\sphinxupquote{params}}}}}

\end{itemize}

\item[{Return p}] \leavevmode
the agent to simulate

\item[{Return type}] \leavevmode
{\hyperref[\detokenize{person:person.Person}]{\sphinxcrossref{person.Person}}}

\end{description}\end{quote}

\end{fulllineitems}


\end{fulllineitems}



\subsection{eat\_lunch\_trial module}
\label{\detokenize{eat_lunch_trial::doc}}\label{\detokenize{eat_lunch_trial:eat-lunch-trial-module}}\label{\detokenize{eat_lunch_trial:module-eat_lunch_trial}}\index{eat\_lunch\_trial (module)}
This module contains code in order to run Monte-Carlo simulations to comparing the Agent-Based Model of Human Activity Patterns (ABMHAP) with the data from the Consolidated Human Activity Database (CHAD) for the \sphinxstylestrong{eat lunch} activity.

This module contains class {\hyperref[\detokenize{eat_lunch_trial:eat_lunch_trial.Eat_Lunch_Trial}]{\sphinxcrossref{\sphinxcode{\sphinxupquote{eat\_lunch\_trial.Eat\_Lunch\_Trial}}}}}.
\index{Eat\_Lunch\_Trial (class in eat\_lunch\_trial)}

\begin{fulllineitems}
\phantomsection\label{\detokenize{eat_lunch_trial:eat_lunch_trial.Eat_Lunch_Trial}}\pysiglinewithargsret{\sphinxbfcode{\sphinxupquote{class }}\sphinxcode{\sphinxupquote{eat\_lunch\_trial.}}\sphinxbfcode{\sphinxupquote{Eat\_Lunch\_Trial}}}{\emph{parameters}, \emph{sampling\_params}, \emph{demographic}}{}
Bases: {\hyperref[\detokenize{trial:trial.Trial}]{\sphinxcrossref{\sphinxcode{\sphinxupquote{trial.Trial}}}}}

This class runs the ABMHAP simulations initialized with eat lunch data from CHAD.
\begin{quote}\begin{description}
\item[{Parameters}] \leavevmode\begin{itemize}
\item {} 
\sphinxstyleliteralstrong{\sphinxupquote{parameters}} ({\hyperref[\detokenize{params:params.Params}]{\sphinxcrossref{\sphinxstyleliteralemphasis{\sphinxupquote{params.Params}}}}}) \textendash{} the parameters describing each person in the household

\item {} 
\sphinxstyleliteralstrong{\sphinxupquote{sampling\_params}} ({\hyperref[\detokenize{chad_params:chad_params.CHAD_params}]{\sphinxcrossref{\sphinxstyleliteralemphasis{\sphinxupquote{chad\_params.CHAD\_params}}}}}) \textendash{} the sampling parameters used to filter “good” CHAD     eat lunch data

\item {} 
\sphinxstyleliteralstrong{\sphinxupquote{demographic}} (\sphinxstyleliteralemphasis{\sphinxupquote{int}}) \textendash{} the demographic identifier

\end{itemize}

\end{description}\end{quote}
\index{adjust\_params() (eat\_lunch\_trial.Eat\_Lunch\_Trial method)}

\begin{fulllineitems}
\phantomsection\label{\detokenize{eat_lunch_trial:eat_lunch_trial.Eat_Lunch_Trial.adjust_params}}\pysiglinewithargsret{\sphinxbfcode{\sphinxupquote{adjust\_params}}}{\emph{start\_mean}, \emph{start\_std}, \emph{dt\_mean}, \emph{dt\_std}}{}
This function adjusts the values for the mean and standard deviation of  eat lunch start time in the         key-word arguments based on the CHAD data that was sampled. These new values will be used in the runs.
\begin{quote}\begin{description}
\item[{Parameters}] \leavevmode\begin{itemize}
\item {} 
\sphinxstyleliteralstrong{\sphinxupquote{start\_mean}} (\sphinxstyleliteralemphasis{\sphinxupquote{numpy.ndarray}}) \textendash{} the mean eat lunch start time {[}hours{]} for each person

\item {} 
\sphinxstyleliteralstrong{\sphinxupquote{start\_std}} (\sphinxstyleliteralemphasis{\sphinxupquote{numpy.ndarray}}) \textendash{} the standard deviation of eat lunch start time {[}hours{]} for each person

\item {} 
\sphinxstyleliteralstrong{\sphinxupquote{dt\_mean}} (\sphinxstyleliteralemphasis{\sphinxupquote{numpy.ndarray}}) \textendash{} the eat lunch mean duration {[}hours{]} for each person

\item {} 
\sphinxstyleliteralstrong{\sphinxupquote{dt\_std}} (\sphinxstyleliteralemphasis{\sphinxupquote{numpy.ndarray}}) \textendash{} the eat lunch standard deviation of duration {[}hours{]} for each person

\end{itemize}

\item[{Returns}] \leavevmode


\end{description}\end{quote}

\end{fulllineitems}

\index{create\_universe() (eat\_lunch\_trial.Eat\_Lunch\_Trial method)}

\begin{fulllineitems}
\phantomsection\label{\detokenize{eat_lunch_trial:eat_lunch_trial.Eat_Lunch_Trial.create_universe}}\pysiglinewithargsret{\sphinxbfcode{\sphinxupquote{create\_universe}}}{}{}
This function creates a universe object that simulations will run in. The only asset in this         simulation for an agent to use is a {\hyperref[\detokenize{food:food.Food}]{\sphinxcrossref{\sphinxcode{\sphinxupquote{food.Food}}}}}.
\begin{quote}\begin{description}
\item[{Returns}] \leavevmode
the universe

\item[{Return type}] \leavevmode
{\hyperref[\detokenize{universe:universe.Universe}]{\sphinxcrossref{universe.Universe}}}

\end{description}\end{quote}

\end{fulllineitems}

\index{initialize() (eat\_lunch\_trial.Eat\_Lunch\_Trial method)}

\begin{fulllineitems}
\phantomsection\label{\detokenize{eat_lunch_trial:eat_lunch_trial.Eat_Lunch_Trial.initialize}}\pysiglinewithargsret{\sphinxbfcode{\sphinxupquote{initialize}}}{}{}
This function sets up the trial
\begin{enumerate}
\item {} 
gets the CHAD data for eat lunch under the appropriate conditions for means and standard deviations         for both eat lunch duration and eat lunch start time

\item {} 
gets N samples the CHAD data for eat lunch duration and eat lunch start time for the N trials

\item {} 
updates the {\hyperref[\detokenize{params:module-params}]{\sphinxcrossref{\sphinxcode{\sphinxupquote{params}}}}} to reflect the newly assigned eat lunch parameters for the simulation

\end{enumerate}
\begin{quote}\begin{description}
\item[{Parameters}] \leavevmode\begin{itemize}
\item {} 
\sphinxstyleliteralstrong{\sphinxupquote{fname\_dt}} (\sphinxstyleliteralemphasis{\sphinxupquote{str}}) \textendash{} the filename of the duration statistics

\item {} 
\sphinxstyleliteralstrong{\sphinxupquote{fname\_start}} (\sphinxstyleliteralemphasis{\sphinxupquote{str}}) \textendash{} the filename of the start time statistics

\end{itemize}

\item[{Returns}] \leavevmode


\end{description}\end{quote}

\end{fulllineitems}

\index{initialize\_person() (eat\_lunch\_trial.Eat\_Lunch\_Trial method)}

\begin{fulllineitems}
\phantomsection\label{\detokenize{eat_lunch_trial:eat_lunch_trial.Eat_Lunch_Trial.initialize_person}}\pysiglinewithargsret{\sphinxbfcode{\sphinxupquote{initialize\_person}}}{\emph{u}, \emph{idx}}{}
This function creates and initializes a person with the proper parameters for the Eat Lunch Trial        simulation.

More specifically, the function does
\begin{enumerate}
\item {} 
creates a {\hyperref[\detokenize{singleton:singleton.Singleton}]{\sphinxcrossref{\sphinxcode{\sphinxupquote{singleton.Singleton}}}}} person

\item {} 
initializes the person’s parameters to the respective values in {\hyperref[\detokenize{params:module-params}]{\sphinxcrossref{\sphinxcode{\sphinxupquote{params}}}}}

\end{enumerate}
\begin{quote}\begin{description}
\item[{Parameters}] \leavevmode\begin{itemize}
\item {} 
\sphinxstyleliteralstrong{\sphinxupquote{u}} ({\hyperref[\detokenize{universe:universe.Universe}]{\sphinxcrossref{\sphinxstyleliteralemphasis{\sphinxupquote{universe.Universe}}}}}) \textendash{} the universe the person will reside in

\item {} 
\sphinxstyleliteralstrong{\sphinxupquote{idx}} (\sphinxstyleliteralemphasis{\sphinxupquote{int}}) \textendash{} the index of the person’s parameters in {\hyperref[\detokenize{params:module-params}]{\sphinxcrossref{\sphinxcode{\sphinxupquote{params}}}}}

\end{itemize}

\item[{Return p}] \leavevmode
the agent to be simulated

\item[{Return type}] \leavevmode
{\hyperref[\detokenize{person:person.Person}]{\sphinxcrossref{person.Person}}}

\end{description}\end{quote}

\end{fulllineitems}


\end{fulllineitems}



\subsection{evaluation module}
\label{\detokenize{evaluation::doc}}\label{\detokenize{evaluation:module-evaluation}}\label{\detokenize{evaluation:evaluation-module}}\index{evaluation (module)}
This module is used to for evaluating the accuracy of the Agent-Based Model of Human Activity Patterns (ABMHAP) simulation results vs. Consolidated Human Activity Database (CHAD) data.
\index{compare\_abm\_to\_chad() (in module evaluation)}

\begin{fulllineitems}
\phantomsection\label{\detokenize{evaluation:evaluation.compare_abm_to_chad}}\pysiglinewithargsret{\sphinxcode{\sphinxupquote{evaluation.}}\sphinxbfcode{\sphinxupquote{compare\_abm\_to\_chad}}}{\emph{demo}, \emph{df\_list}, \emph{trial\_code}, \emph{fidx=100}, \emph{do\_save=False}, \emph{fpath=None}}{}
This function compares the results of the ABMHAP to the CHAD data by showing by
\begin{enumerate}
\item {} 
plotting cumulative distribution functions (CDF) of the predicted (ABMHAP) and     observed (CHAD) single-day data for each activity

\item {} 
plotting the residual (that difference between the CDFs) between the predicted     (ABMHAP) and observed (CHAD) data for each activity

\end{enumerate}
\begin{quote}\begin{description}
\item[{Parameters}] \leavevmode\begin{itemize}
\item {} 
\sphinxstyleliteralstrong{\sphinxupquote{demo}} (\sphinxstyleliteralemphasis{\sphinxupquote{int}}) \textendash{} the demographic identifier

\item {} 
\sphinxstyleliteralstrong{\sphinxupquote{df\_list}} (\sphinxstyleliteralemphasis{\sphinxupquote{list of  pandas.core.frame.DataFrame}}) \textendash{} the ABMHAP activity diaries to compare

\item {} 
\sphinxstyleliteralstrong{\sphinxupquote{trial\_code}} (\sphinxstyleliteralemphasis{\sphinxupquote{int}}) \textendash{} the trial identifier

\item {} 
\sphinxstyleliteralstrong{\sphinxupquote{fidx}} (\sphinxstyleliteralemphasis{\sphinxupquote{int}}) \textendash{} the figure identifier for the first figure in a series of figures

\item {} 
\sphinxstyleliteralstrong{\sphinxupquote{do\_save}} (\sphinxstyleliteralemphasis{\sphinxupquote{bool}}) \textendash{} a flag indicating whether (if True) or not (if False) to save     the figures

\item {} 
\sphinxstyleliteralstrong{\sphinxupquote{fpath}} (\sphinxstyleliteralemphasis{\sphinxupquote{str}}) \textendash{} the file path of the figures that are to be saved

\end{itemize}

\item[{Returns}] \leavevmode


\end{description}\end{quote}

\end{fulllineitems}

\index{compare\_abm\_to\_chad\_help() (in module evaluation)}

\begin{fulllineitems}
\phantomsection\label{\detokenize{evaluation:evaluation.compare_abm_to_chad_help}}\pysiglinewithargsret{\sphinxcode{\sphinxupquote{evaluation.}}\sphinxbfcode{\sphinxupquote{compare\_abm\_to\_chad\_help}}}{\emph{df\_abm}, \emph{df\_obs}, \emph{act\_code}, \emph{fidx}, \emph{do\_save}, \emph{fpath}}{}
This function compares the results of the ABMHAP to the CHAD data for a given activity by
\begin{enumerate}
\item {} 
plotting cumulative distribution functions (CDF) of the predicted (ABMHAP) and     observed (CHAD) single-day data for each activity

\item {} 
plotting the residual (that difference between the CDFs) between the predicted     (ABMHAP) and observed (CHAD) data for each activity

\end{enumerate}
\begin{quote}\begin{description}
\item[{Parameters}] \leavevmode\begin{itemize}
\item {} 
\sphinxstyleliteralstrong{\sphinxupquote{df\_abm}} (\sphinxstyleliteralemphasis{\sphinxupquote{pandas.core.frame.DataFrame}}) \textendash{} the predicted (ABMHAP) data for the respective activity

\item {} 
\sphinxstyleliteralstrong{\sphinxupquote{df\_obs}} (\sphinxstyleliteralemphasis{\sphinxupquote{pandas.core.frame.DataFrame}}) \textendash{} the single-day observed (CHAD) data for the respective activity

\item {} 
\sphinxstyleliteralstrong{\sphinxupquote{act\_code}} (\sphinxstyleliteralemphasis{\sphinxupquote{float}}) \textendash{} the activity code

\item {} 
\sphinxstyleliteralstrong{\sphinxupquote{fidx}} (\sphinxstyleliteralemphasis{\sphinxupquote{int}}) \textendash{} the figure identifier of the first figure

\item {} 
\sphinxstyleliteralstrong{\sphinxupquote{do\_save}} (\sphinxstyleliteralemphasis{\sphinxupquote{bool}}) \textendash{} a flag indicating whether (if True) or not (if False) to save     the figures

\item {} 
\sphinxstyleliteralstrong{\sphinxupquote{fpath}} (\sphinxstyleliteralemphasis{\sphinxupquote{str}}) \textendash{} the file path of the figures that are to be saved

\end{itemize}

\item[{Returns}] \leavevmode
the last figure identifier plotted

\item[{Return type}] \leavevmode
int

\end{description}\end{quote}

\end{fulllineitems}

\index{get\_solo\_data() (in module evaluation)}

\begin{fulllineitems}
\phantomsection\label{\detokenize{evaluation:evaluation.get_solo_data}}\pysiglinewithargsret{\sphinxcode{\sphinxupquote{evaluation.}}\sphinxbfcode{\sphinxupquote{get\_solo\_data}}}{\emph{z}, \emph{fname}}{}
This function gets the single-day data from individuals with only single-day records      within CHAD.
\begin{quote}\begin{description}
\item[{Parameters}] \leavevmode\begin{itemize}
\item {} 
\sphinxstyleliteralstrong{\sphinxupquote{z}} (\sphinxstyleliteralemphasis{\sphinxupquote{zipfile}}) \textendash{} the zipfile of the demographic data

\item {} 
\sphinxstyleliteralstrong{\sphinxupquote{fname}} (\sphinxstyleliteralemphasis{\sphinxupquote{str}}) \textendash{} the file name for the CHAD individual records data

\end{itemize}

\item[{Returns}] \leavevmode
the CHAD single-day data

\item[{Return type}] \leavevmode
pandas.core.frame.DataFrame

\end{description}\end{quote}

\end{fulllineitems}

\index{plot() (in module evaluation)}

\begin{fulllineitems}
\phantomsection\label{\detokenize{evaluation:evaluation.plot}}\pysiglinewithargsret{\sphinxcode{\sphinxupquote{evaluation.}}\sphinxbfcode{\sphinxupquote{plot}}}{\emph{x}, \emph{q}, \emph{cdf}, \emph{inv\_cdf}, \emph{act\_code}, \emph{fids}, \emph{do\_hours=True}, \emph{dname=None}}{}
This function plots the following results of cumulative distribution function (CDF):
\begin{enumerate}
\item {} 
CDFs comparing the predicted and observed values

\item {} 
CDFs showing the residual

\item {} 
CDFs showing the scaled residual

\item {} 
Inverted CDFs comparing the predicted and observed values

\item {} 
Inverted CDFs showing the residual

\item {} 
Inverted CDFs showing the scaled residual

\end{enumerate}
\begin{quote}\begin{description}
\item[{Parameters}] \leavevmode\begin{itemize}
\item {} 
\sphinxstyleliteralstrong{\sphinxupquote{x}} (\sphinxstyleliteralemphasis{\sphinxupquote{numpy.ndarray}}) \textendash{} the range of values of the data

\item {} 
\sphinxstyleliteralstrong{\sphinxupquote{q}} (\sphinxstyleliteralemphasis{\sphinxupquote{numpy.ndarray}}) \textendash{} the qunatiles

\item {} 
\sphinxstyleliteralstrong{\sphinxupquote{cdf}} (\sphinxstyleliteralemphasis{\sphinxupquote{numpy.ndarray}}) \textendash{} the cumulative distribution function in units of percentage

\item {} 
\sphinxstyleliteralstrong{\sphinxupquote{inv\_cdf}} (\sphinxstyleliteralemphasis{\sphinxupquote{numpy.ndarray}}) \textendash{} the cumulative distribution function in units of time

\item {} 
\sphinxstyleliteralstrong{\sphinxupquote{act\_code}} (\sphinxstyleliteralemphasis{\sphinxupquote{numpy.ndarray}}) \textendash{} the activity codes of the respective activities

\item {} 
\sphinxstyleliteralstrong{\sphinxupquote{fids}} (\sphinxstyleliteralemphasis{\sphinxupquote{numpy.ndarray}}) \textendash{} the figure identifiers

\item {} 
\sphinxstyleliteralstrong{\sphinxupquote{do\_hours}} (\sphinxstyleliteralemphasis{\sphinxupquote{bool}}) \textendash{} a flag indicating whether to plot the inverted CDF data in hours (if True)     or minutes (if false)

\item {} 
\sphinxstyleliteralstrong{\sphinxupquote{dname}} (\sphinxstyleliteralemphasis{\sphinxupquote{str}}) \textendash{} the name of the data to be plotted

\item {} 
\sphinxstyleliteralstrong{\sphinxupquote{off}} (\sphinxstyleliteralemphasis{\sphinxupquote{float}}) \textendash{} the percentage in which to put a vertical line indicating both the bottom and top     off-percentage of the data

\end{itemize}

\item[{Returns}] \leavevmode
a figure containing CDFs comparing the predicted and observed values,     a figure containing CDFs showing the residual,     a figure containing CDFs showing the scaled residual,     a figure containing Inverted CDFs comparing the predicted and observed values,     a figure containing Inverted CDFs showing the residual,     a figure containing Inverted CDFs showing the scaled residual

\item[{Return type}] \leavevmode
matplotlib.figure.Figure, matplotlib.figure.Figure, matplotlib.figure.Figure     matplotlib.figure.Figure, matplotlib.figure.Figure, matplotlib.figure.Figure

\end{description}\end{quote}

\end{fulllineitems}

\index{plot\_predicted\_observed() (in module evaluation)}

\begin{fulllineitems}
\phantomsection\label{\detokenize{evaluation:evaluation.plot_predicted_observed}}\pysiglinewithargsret{\sphinxcode{\sphinxupquote{evaluation.}}\sphinxbfcode{\sphinxupquote{plot\_predicted\_observed}}}{\emph{x}, \emph{pred}, \emph{obs}, \emph{xlabel}, \emph{ylabel}, \emph{title}}{}
Plot the predicted (ABMHAP) and observed (CHAD) data.
\begin{quote}\begin{description}
\item[{Parameters}] \leavevmode\begin{itemize}
\item {} 
\sphinxstyleliteralstrong{\sphinxupquote{x}} (\sphinxstyleliteralemphasis{\sphinxupquote{numpy.ndarray}}) \textendash{} the x-axis

\item {} 
\sphinxstyleliteralstrong{\sphinxupquote{pred}} (\sphinxstyleliteralemphasis{\sphinxupquote{numpy.ndarray}}) \textendash{} the predicted (ABMHAP) values

\item {} 
\sphinxstyleliteralstrong{\sphinxupquote{obs}} (\sphinxstyleliteralemphasis{\sphinxupquote{numpy.ndarray}}) \textendash{} the observed (CHAD) values from data

\item {} 
\sphinxstyleliteralstrong{\sphinxupquote{xlabel}} (\sphinxstyleliteralemphasis{\sphinxupquote{str}}) \textendash{} the x-axis label

\item {} 
\sphinxstyleliteralstrong{\sphinxupquote{ylabel}} (\sphinxstyleliteralemphasis{\sphinxupquote{str}}) \textendash{} the y-axis label

\item {} 
\sphinxstyleliteralstrong{\sphinxupquote{title}} (\sphinxstyleliteralemphasis{\sphinxupquote{str}}) \textendash{} the title of the figure

\end{itemize}

\item[{Returns}] \leavevmode


\end{description}\end{quote}

\end{fulllineitems}

\index{plot\_residual() (in module evaluation)}

\begin{fulllineitems}
\phantomsection\label{\detokenize{evaluation:evaluation.plot_residual}}\pysiglinewithargsret{\sphinxcode{\sphinxupquote{evaluation.}}\sphinxbfcode{\sphinxupquote{plot\_residual}}}{\emph{x}, \emph{res}, \emph{xlabel=''}, \emph{ylabel=''}, \emph{title=''}, \emph{color='r'}, \emph{label='Residual'}}{}
This function plots the residual between cumulative distribution functions (CDFs)     the ABMHAP and CHAD data.
\begin{quote}\begin{description}
\item[{Parameters}] \leavevmode\begin{itemize}
\item {} 
\sphinxstyleliteralstrong{\sphinxupquote{x}} (\sphinxstyleliteralemphasis{\sphinxupquote{numpy.ndarray}}) \textendash{} the x-axis data

\item {} 
\sphinxstyleliteralstrong{\sphinxupquote{res}} (\sphinxstyleliteralemphasis{\sphinxupquote{numpy.ndarray}}) \textendash{} the residual \(r(x)\)

\item {} 
\sphinxstyleliteralstrong{\sphinxupquote{xlabel}} (\sphinxstyleliteralemphasis{\sphinxupquote{str}}) \textendash{} the x-axis label

\item {} 
\sphinxstyleliteralstrong{\sphinxupquote{ylabel}} (\sphinxstyleliteralemphasis{\sphinxupquote{str}}) \textendash{} the y-axis label

\item {} 
\sphinxstyleliteralstrong{\sphinxupquote{title}} (\sphinxstyleliteralemphasis{\sphinxupquote{str}}) \textendash{} the title of the plot

\item {} 
\sphinxstyleliteralstrong{\sphinxupquote{color}} (\sphinxstyleliteralemphasis{\sphinxupquote{str}}) \textendash{} the color of the plot

\item {} 
\sphinxstyleliteralstrong{\sphinxupquote{label}} (\sphinxstyleliteralemphasis{\sphinxupquote{str}}) \textendash{} the label of the plot

\end{itemize}

\item[{Returns}] \leavevmode


\end{description}\end{quote}

\end{fulllineitems}

\index{residual() (in module evaluation)}

\begin{fulllineitems}
\phantomsection\label{\detokenize{evaluation:evaluation.residual}}\pysiglinewithargsret{\sphinxcode{\sphinxupquote{evaluation.}}\sphinxbfcode{\sphinxupquote{residual}}}{\emph{pred}, \emph{obs}, \emph{x}}{}
This function analyzes the residual between predicted values and observed values. Given the predicted and     observed values, this function does the following:
\begin{enumerate}
\item {} 
Compute the empirical cumulative distribution function (CDF) between the predicted and observed data     in units {[}quantile vs hours{]}

\item {} 
Compute the residual in the CDF between observed and predicted data
\begin{quote}
\begin{equation*}
\begin{split}r(x) = cdf_{observed}(x) - cdf_{predicted}(x)\end{split}
\end{equation*}\end{quote}

\item {} 
Invert the residual so that the CDFs and residuals are in units {[}minutes vs quantile{]}

\end{enumerate}
\begin{quote}\begin{description}
\item[{Parameters}] \leavevmode\begin{itemize}
\item {} 
\sphinxstyleliteralstrong{\sphinxupquote{pred}} (\sphinxstyleliteralemphasis{\sphinxupquote{numpy.ndarray}}) \textendash{} the predicted (ABMHAP) values used to make the empirical CDF

\item {} 
\sphinxstyleliteralstrong{\sphinxupquote{obs}} (\sphinxstyleliteralemphasis{\sphinxupquote{numpy.ndarray}}) \textendash{} the observed (CHAD) values used to make the empirical CDF

\item {} 
\sphinxstyleliteralstrong{\sphinxupquote{x}} (\sphinxstyleliteralemphasis{\sphinxupquote{numpy.ndarray}}) \textendash{} the x-values

\item {} 
\sphinxstyleliteralstrong{\sphinxupquote{do\_scaling}} (\sphinxstyleliteralemphasis{\sphinxupquote{bool}}) \textendash{} this scales the inverted cdf residual by the standard deviation of the observed values

\end{itemize}

\item[{Returns}] \leavevmode
the data for the cumulative distribution data (predicted, observed, residual, and scaled residual),     the data for the inverted cumulative distribution data (predicted, observed, residual, and scaled residual)

\item[{Return type}] \leavevmode
pandas.core.frame.DataFrame, pandas.core.frame.DataFrame

\end{description}\end{quote}

\end{fulllineitems}

\index{residual\_analysis() (in module evaluation)}

\begin{fulllineitems}
\phantomsection\label{\detokenize{evaluation:evaluation.residual_analysis}}\pysiglinewithargsret{\sphinxcode{\sphinxupquote{evaluation.}}\sphinxbfcode{\sphinxupquote{residual\_analysis}}}{\emph{pred}, \emph{obs}, \emph{N=1001}, \emph{do\_periodic=False}}{}
This function takes the predicted and observed values and computes the respective cumulative distribution     functions (CDFs) in units percentage and the inverted CDF which is the CDF in units of minutes.
\begin{quote}\begin{description}
\item[{Parameters}] \leavevmode\begin{itemize}
\item {} 
\sphinxstyleliteralstrong{\sphinxupquote{pred}} (\sphinxstyleliteralemphasis{\sphinxupquote{numpy.ndarray}}) \textendash{} the predicted values

\item {} 
\sphinxstyleliteralstrong{\sphinxupquote{obs}} (\sphinxstyleliteralemphasis{\sphinxupquote{numpy.ndarray}}) \textendash{} the observed values

\item {} 
\sphinxstyleliteralstrong{\sphinxupquote{N}} (\sphinxstyleliteralemphasis{\sphinxupquote{int}}) \textendash{} the number of points of the CDF vector

\item {} 
\sphinxstyleliteralstrong{\sphinxupquote{do\_periodic}} (\sphinxstyleliteralemphasis{\sphinxupquote{bool}}) \textendash{} a flag to see if the time data should be in a {[}-12, 12) hour format

\end{itemize}

\item[{Returns}] \leavevmode
the x values, CDF of residual, inverted CDF of residual

\item[{Return type}] \leavevmode
numpy.ndarray, pandas.core.frame.DataFrame, pandas.core.frame.DataFrame

\end{description}\end{quote}

\end{fulllineitems}

\index{sample\_activitiy\_abm\_work() (in module evaluation)}

\begin{fulllineitems}
\phantomsection\label{\detokenize{evaluation:evaluation.sample_activitiy_abm_work}}\pysiglinewithargsret{\sphinxcode{\sphinxupquote{evaluation.}}\sphinxbfcode{\sphinxupquote{sample\_activitiy\_abm\_work}}}{\emph{df}}{}
This function is used in order to sample a random day of work activity data from the ABM. This function takes     takes into account that 1 work “event” consists of multiple work activity-diary entries.

\begin{sphinxadmonition}{note}{Note:}
This function assumes that df only contains work activity data and is \sphinxstylestrong{NOT} empty
\end{sphinxadmonition}

\begin{sphinxadmonition}{note}{Note:}
The duration data here is the end of the last event - minus the start of the first event.         This is done to mimic how the duration data is stored in CHAD.
\end{sphinxadmonition}
\begin{quote}\begin{description}
\item[{Parameters}] \leavevmode
\sphinxstyleliteralstrong{\sphinxupquote{df}} (\sphinxstyleliteralemphasis{\sphinxupquote{pandas.core.frame.DataFrame}}) \textendash{} the diary of work activities for an individual

\end{description}\end{quote}

:return:the sampled work data
:rtype: pandas.core.frame.DataFrame

\end{fulllineitems}

\index{sample\_activity\_abm() (in module evaluation)}

\begin{fulllineitems}
\phantomsection\label{\detokenize{evaluation:evaluation.sample_activity_abm}}\pysiglinewithargsret{\sphinxcode{\sphinxupquote{evaluation.}}\sphinxbfcode{\sphinxupquote{sample\_activity\_abm}}}{\emph{df\_list}, \emph{act}}{}
Given an activity type, this function looks at each activity diary and samples 1 event of that activity type     should that diary have a matching activity-entry.

\begin{sphinxadmonition}{note}{Note:}
Because the work activity technically occurs twice (1 event before lunch and 1 event after lunch), the         activity needs to be merged as one event in order for the analysis to be correct.
\end{sphinxadmonition}
\begin{quote}\begin{description}
\item[{Parameters}] \leavevmode\begin{itemize}
\item {} 
\sphinxstyleliteralstrong{\sphinxupquote{df\_list}} (\sphinxstyleliteralemphasis{\sphinxupquote{list of pandas.core.frame.DataFrame}}) \textendash{} the activity diaries

\item {} 
\sphinxstyleliteralstrong{\sphinxupquote{act}} (\sphinxstyleliteralemphasis{\sphinxupquote{float}}) \textendash{} the activity code

\end{itemize}

\item[{Returns}] \leavevmode
the sampled activities

\item[{Return type}] \leavevmode
pandas.core.frame.DataFrame

\end{description}\end{quote}

\end{fulllineitems}

\index{save\_figs\_dt() (in module evaluation)}

\begin{fulllineitems}
\phantomsection\label{\detokenize{evaluation:evaluation.save_figs_dt}}\pysiglinewithargsret{\sphinxcode{\sphinxupquote{evaluation.}}\sphinxbfcode{\sphinxupquote{save\_figs\_dt}}}{\emph{figs}, \emph{fpath}}{}
This function save plots about the activity duration.
\begin{quote}\begin{description}
\item[{Parameters}] \leavevmode\begin{itemize}
\item {} 
\sphinxstyleliteralstrong{\sphinxupquote{figs}} (\sphinxstyleliteralemphasis{\sphinxupquote{tuple}}) \textendash{} a tuple of figures to save about activity duration data

\item {} 
\sphinxstyleliteralstrong{\sphinxupquote{fpath}} (\sphinxstyleliteralemphasis{\sphinxupquote{str}}) \textendash{} the specific file path in which to plot the data

\end{itemize}

\item[{Returns}] \leavevmode


\end{description}\end{quote}

\end{fulllineitems}

\index{save\_figs\_end() (in module evaluation)}

\begin{fulllineitems}
\phantomsection\label{\detokenize{evaluation:evaluation.save_figs_end}}\pysiglinewithargsret{\sphinxcode{\sphinxupquote{evaluation.}}\sphinxbfcode{\sphinxupquote{save\_figs\_end}}}{\emph{figs}, \emph{fpath}}{}
This function save plots about the activity end time.
\begin{quote}\begin{description}
\item[{Parameters}] \leavevmode\begin{itemize}
\item {} 
\sphinxstyleliteralstrong{\sphinxupquote{figs}} (\sphinxstyleliteralemphasis{\sphinxupquote{tuple}}) \textendash{} a tuple of figures to save about activity end time data

\item {} 
\sphinxstyleliteralstrong{\sphinxupquote{fpath}} (\sphinxstyleliteralemphasis{\sphinxupquote{str}}) \textendash{} the specific file path in which to plot the data

\end{itemize}

\item[{Returns}] \leavevmode


\end{description}\end{quote}

\end{fulllineitems}

\index{save\_figs\_start() (in module evaluation)}

\begin{fulllineitems}
\phantomsection\label{\detokenize{evaluation:evaluation.save_figs_start}}\pysiglinewithargsret{\sphinxcode{\sphinxupquote{evaluation.}}\sphinxbfcode{\sphinxupquote{save\_figs\_start}}}{\emph{figs}, \emph{fpath}}{}
This function save plots about the activity start time.
\begin{quote}\begin{description}
\item[{Parameters}] \leavevmode\begin{itemize}
\item {} 
\sphinxstyleliteralstrong{\sphinxupquote{figs}} (\sphinxstyleliteralemphasis{\sphinxupquote{tuple}}) \textendash{} a tuple of figures to save about activity start time data

\item {} 
\sphinxstyleliteralstrong{\sphinxupquote{fpath}} (\sphinxstyleliteralemphasis{\sphinxupquote{str}}) \textendash{} the specific file path in which to plot the data

\end{itemize}

\item[{Returns}] \leavevmode


\end{description}\end{quote}

\end{fulllineitems}

\index{save\_figures() (in module evaluation)}

\begin{fulllineitems}
\phantomsection\label{\detokenize{evaluation:evaluation.save_figures}}\pysiglinewithargsret{\sphinxcode{\sphinxupquote{evaluation.}}\sphinxbfcode{\sphinxupquote{save\_figures}}}{\emph{act}, \emph{figs\_start}, \emph{figs\_end}, \emph{figs\_dt}, \emph{fpath}}{}
This function saves the plotted figures about duration and start time data of the results from     {\hyperref[\detokenize{evaluation:evaluation.compare_abm_to_chad}]{\sphinxcrossref{\sphinxcode{\sphinxupquote{compare\_abm\_to\_chad()}}}}}.
\begin{quote}\begin{description}
\item[{Parameters}] \leavevmode\begin{itemize}
\item {} 
\sphinxstyleliteralstrong{\sphinxupquote{act}} (\sphinxstyleliteralemphasis{\sphinxupquote{int}}) \textendash{} the activity code

\item {} 
\sphinxstyleliteralstrong{\sphinxupquote{figs\_start}} (\sphinxstyleliteralemphasis{\sphinxupquote{tuple}}) \textendash{} a tuple of figures to save about activity start time data about the random day sampling

\item {} 
\sphinxstyleliteralstrong{\sphinxupquote{figs\_end}} (\sphinxstyleliteralemphasis{\sphinxupquote{tuple}}) \textendash{} a tuple of figures to save about activity end time data about the random day sampling

\item {} 
\sphinxstyleliteralstrong{\sphinxupquote{figs\_dt}} (\sphinxstyleliteralemphasis{\sphinxupquote{tuple}}) \textendash{} a tuple of figures to save about activity duration data about the random day sampling

\item {} 
\sphinxstyleliteralstrong{\sphinxupquote{fpath}} (\sphinxstyleliteralemphasis{\sphinxupquote{str}}) \textendash{} the general file path to plot the data

\end{itemize}

\item[{Returns}] \leavevmode


\end{description}\end{quote}

\end{fulllineitems}



\subsection{fig\_driver module}
\label{\detokenize{fig_driver::doc}}\label{\detokenize{fig_driver:fig-driver-module}}\label{\detokenize{fig_driver:module-fig_driver}}\index{fig\_driver (module)}
\begin{sphinxadmonition}{warning}{Warning:}
This module is antiquated and not used.
\end{sphinxadmonition}

This function is used to get the saved pickled plot data and plot them in subplots.

It is used to obtain cumulative distribution functions (CDFs) about the Agent-Based Model of Human Activity Patterns (ABMHAP) ABMHAP vs CHAD data for various activities and plot subplots with data from various activities instead of just 1
\index{compare\_single\_omni() (in module fig\_driver)}

\begin{fulllineitems}
\phantomsection\label{\detokenize{fig_driver:fig_driver.compare_single_omni}}\pysiglinewithargsret{\sphinxcode{\sphinxupquote{fig\_driver.}}\sphinxbfcode{\sphinxupquote{compare\_single\_omni}}}{\emph{fdirs\_single}, \emph{fdirs\_omni}, \emph{fid}, \emph{nrows}, \emph{ncols}, \emph{activity\_codes}, \emph{do\_chad=False}}{}
Plot a subplot of CDFs of single-activity and full simulation data for start time and duration.
\begin{quote}\begin{description}
\item[{Parameters}] \leavevmode\begin{itemize}
\item {} 
\sphinxstyleliteralstrong{\sphinxupquote{fdirs\_single}} (\sphinxstyleliteralemphasis{\sphinxupquote{list}}) \textendash{} the filenames of the pickled single-activity data

\item {} 
\sphinxstyleliteralstrong{\sphinxupquote{fdirs\_omni}} (\sphinxstyleliteralemphasis{\sphinxupquote{list}}) \textendash{} the filenames of the pickled for full-simulation data

\item {} 
\sphinxstyleliteralstrong{\sphinxupquote{fid}} (\sphinxstyleliteralemphasis{\sphinxupquote{int}}) \textendash{} figure identifier

\item {} 
\sphinxstyleliteralstrong{\sphinxupquote{nrows}} (\sphinxstyleliteralemphasis{\sphinxupquote{int}}) \textendash{} the number of rows in the suubplot

\item {} 
\sphinxstyleliteralstrong{\sphinxupquote{ncols}} (\sphinxstyleliteralemphasis{\sphinxupquote{int}}) \textendash{} the number of columns in the subplot

\item {} 
\sphinxstyleliteralstrong{\sphinxupquote{activity\_codes}} (\sphinxstyleliteralemphasis{\sphinxupquote{list}}) \textendash{} the activity codes to plot

\item {} 
\sphinxstyleliteralstrong{\sphinxupquote{do\_chad}} (\sphinxstyleliteralemphasis{\sphinxupquote{bool}}) \textendash{} flag indicating whether or not to plot the CHAD data

\end{itemize}

\item[{Returns}] \leavevmode


\end{description}\end{quote}

\end{fulllineitems}

\index{plot\_cdfs() (in module fig\_driver)}

\begin{fulllineitems}
\phantomsection\label{\detokenize{fig_driver:fig_driver.plot_cdfs}}\pysiglinewithargsret{\sphinxcode{\sphinxupquote{fig\_driver.}}\sphinxbfcode{\sphinxupquote{plot\_cdfs}}}{\emph{fdir}, \emph{fid}}{}
\end{fulllineitems}

\index{plot\_cdfs2() (in module fig\_driver)}

\begin{fulllineitems}
\phantomsection\label{\detokenize{fig_driver:fig_driver.plot_cdfs2}}\pysiglinewithargsret{\sphinxcode{\sphinxupquote{fig\_driver.}}\sphinxbfcode{\sphinxupquote{plot\_cdfs2}}}{\emph{fdirs}, \emph{fid}, \emph{nrows}, \emph{ncols}, \emph{activity\_codes}}{}
\end{fulllineitems}



\subsection{figure\_loader notebook}
\label{\detokenize{figure_loader::doc}}\label{\detokenize{figure_loader:figure-loader-notebook}}
\fvset{hllines={, ,}}%
\begin{sphinxVerbatim}[commandchars=\\\{\}]
\PYG{c+c1}{\PYGZsh{} The United States Environmental Protection Agency through its Office of}
\PYG{c+c1}{\PYGZsh{} Research and Development has developed this software. The code is made}
\PYG{c+c1}{\PYGZsh{} publicly available to better communicate the research. All input data}
\PYG{c+c1}{\PYGZsh{} used fora given application should be reviewed by the researcher so}
\PYG{c+c1}{\PYGZsh{} that the model results are based on appropriate data for any given}
\PYG{c+c1}{\PYGZsh{} application. This model is under continued development. The model and}
\PYG{c+c1}{\PYGZsh{} data included herein do not represent and should not be construed to}
\PYG{c+c1}{\PYGZsh{} represent any Agency determination or policy.}
\PYG{c+c1}{\PYGZsh{}}
\PYG{c+c1}{\PYGZsh{} This file was written by Dr. Namdi Brandon}
\PYG{c+c1}{\PYGZsh{} ORCID: 0000\PYGZhy{}0001\PYGZhy{}7050\PYGZhy{}1538}
\PYG{c+c1}{\PYGZsh{} March 20, 2018}
\end{sphinxVerbatim}

This notebook loads the individual data about the cumuluative
distribution functions (CDFs) comaparing the Agent-Based Model of Human
Activity Patterns (ABMHAP) results to the Consolidated Human Activity
Database (CHAD) data. The plots compare the distribution
activity-parameter data from ABMHAP to CHAD. More specifically, the
ABMAHP simulation data parameterized with CHAD longitduinal data are
comared to the single-day data from CHAD. The following is plotted: 1.
CDFs of ABMHAP vs. CHAD longitudianl data for activity-parameters 2.
CDFs of ABMHAP vs CHAD single-day data for activity-parameters 3.
Inverse CDFs of ABMHAP vs CHAD single-day data for ctivity-parameters 4.
Residual of the Inverse CDF of ABMHAP vs CHAD single-day data for
activity-parameters 5. Scaled Residual of the Quantile Functions of
ABMHAP vs CHAD single-day data for activity-parameters

Import

\fvset{hllines={, ,}}%
\begin{sphinxVerbatim}[commandchars=\\\{\}]
\PYG{k+kn}{import} \PYG{n+nn}{sys}
\PYG{n}{sys}\PYG{o}{.}\PYG{n}{path}\PYG{o}{.}\PYG{n}{append}\PYG{p}{(}\PYG{l+s+s1}{\PYGZsq{}}\PYG{l+s+s1}{..}\PYG{l+s+se}{\PYGZbs{}\PYGZbs{}}\PYG{l+s+s1}{source}\PYG{l+s+s1}{\PYGZsq{}}\PYG{p}{)}
\PYG{n}{sys}\PYG{o}{.}\PYG{n}{path}\PYG{o}{.}\PYG{n}{append}\PYG{p}{(}\PYG{l+s+s1}{\PYGZsq{}}\PYG{l+s+s1}{..}\PYG{l+s+se}{\PYGZbs{}\PYGZbs{}}\PYG{l+s+s1}{processing}\PYG{l+s+s1}{\PYGZsq{}}\PYG{p}{)}
\PYG{n}{sys}\PYG{o}{.}\PYG{n}{path}\PYG{o}{.}\PYG{n}{append}\PYG{p}{(}\PYG{l+s+s1}{\PYGZsq{}}\PYG{l+s+s1}{..}\PYG{l+s+se}{\PYGZbs{}\PYGZbs{}}\PYG{l+s+s1}{plotting}\PYG{l+s+s1}{\PYGZsq{}}\PYG{p}{)}

\PYG{c+c1}{\PYGZsh{} plotting capabilities}
\PYG{k+kn}{import} \PYG{n+nn}{matplotlib}\PYG{n+nn}{.}\PYG{n+nn}{pylab} \PYG{k}{as} \PYG{n+nn}{plt}
\PYG{k+kn}{import} \PYG{n+nn}{matplotlib}\PYG{n+nn}{.}\PYG{n+nn}{ticker} \PYG{k}{as} \PYG{n+nn}{ticker}

\PYG{c+c1}{\PYGZsh{} math capability}
\PYG{k+kn}{import} \PYG{n+nn}{numpy} \PYG{k}{as} \PYG{n+nn}{np}

\PYG{c+c1}{\PYGZsh{} data frame capability}
\PYG{k+kn}{import} \PYG{n+nn}{pandas} \PYG{k}{as} \PYG{n+nn}{pd}

\PYG{c+c1}{\PYGZsh{} python pickle capability}
\PYG{k+kn}{import} \PYG{n+nn}{pickle}

\PYG{c+c1}{\PYGZsh{} ABMHAP capability}
\PYG{k+kn}{import} \PYG{n+nn}{my\PYGZus{}globals} \PYG{k}{as} \PYG{n+nn}{mg}
\PYG{k+kn}{import} \PYG{n+nn}{chad\PYGZus{}demography\PYGZus{}adult\PYGZus{}work} \PYG{k}{as} \PYG{n+nn}{cdaw}
\PYG{k+kn}{import} \PYG{n+nn}{chad\PYGZus{}demography\PYGZus{}adult\PYGZus{}non\PYGZus{}work} \PYG{k}{as} \PYG{n+nn}{cdanw}
\PYG{k+kn}{import} \PYG{n+nn}{chad\PYGZus{}demography\PYGZus{}child\PYGZus{}school} \PYG{k}{as} \PYG{n+nn}{cdcs}
\PYG{k+kn}{import} \PYG{n+nn}{chad\PYGZus{}demography\PYGZus{}child\PYGZus{}young} \PYG{k}{as} \PYG{n+nn}{cdcy}
\PYG{k+kn}{import} \PYG{n+nn}{demography} \PYG{k}{as} \PYG{n+nn}{dmg}

\PYG{k+kn}{import} \PYG{n+nn}{activity}\PYG{o}{,} \PYG{n+nn}{analyzer}\PYG{o}{,} \PYG{n+nn}{plotter}\PYG{o}{,} \PYG{n+nn}{temporal}
\end{sphinxVerbatim}

\fvset{hllines={, ,}}%
\begin{sphinxVerbatim}[commandchars=\\\{\}]
\PYG{o}{\PYGZpc{}}\PYG{k}{matplotlib} auto
\end{sphinxVerbatim}

\fvset{hllines={, ,}}%
\begin{sphinxVerbatim}[commandchars=\\\{\}]
\PYG{n}{Using} \PYG{n}{matplotlib} \PYG{n}{backend}\PYG{p}{:} \PYG{n}{Qt5Agg}
\end{sphinxVerbatim}

define functions

\fvset{hllines={, ,}}%
\begin{sphinxVerbatim}[commandchars=\\\{\}]
\PYG{k}{def} \PYG{n+nf}{plot\PYGZus{}subplots}\PYG{p}{(}\PYG{n}{data\PYGZus{}list}\PYG{p}{,} \PYG{n}{do\PYGZus{}cdf}\PYG{p}{,} \PYG{n}{main\PYGZus{}title}\PYG{p}{,} \PYG{n}{legend}\PYG{p}{,} \PYG{n}{xlabels}\PYG{p}{,} \PYG{n}{ylabels}\PYG{p}{,} \PYG{n}{xunits}\PYG{p}{,} \PYG{n}{yunits}\PYG{p}{,} \PYG{n}{colors}\PYG{p}{,} \PYGZbs{}
                  \PYG{n}{do\PYGZus{}save}\PYG{o}{=}\PYG{k+kc}{False}\PYG{p}{,} \PYG{n}{fname}\PYG{o}{=}\PYG{k+kc}{None}\PYG{p}{,} \PYG{n}{linewidth}\PYG{o}{=}\PYG{l+m+mi}{1}\PYG{p}{)}\PYG{p}{:}

    \PYG{c+c1}{\PYGZsh{} the dimensions of a maximized figure. Base x Height [pixels]}
    \PYG{n}{b\PYGZus{}pixels}\PYG{p}{,} \PYG{n}{h\PYGZus{}pixels} \PYG{o}{=} \PYG{l+m+mi}{2400}\PYG{p}{,} \PYG{l+m+mi}{1255}
    \PYG{n}{my\PYGZus{}dpi} \PYG{o}{=} \PYG{l+m+mi}{800}

    \PYG{n}{b\PYGZus{}in} \PYG{o}{=} \PYG{n}{b\PYGZus{}pixels}\PYG{o}{/}\PYG{n}{my\PYGZus{}dpi}
    \PYG{n}{h\PYGZus{}in} \PYG{o}{=} \PYG{n}{h\PYGZus{}pixels}\PYG{o}{/}\PYG{n}{my\PYGZus{}dpi}


    \PYG{c+c1}{\PYGZsh{} set the figure size for saving to custom if savinig}
    \PYG{k}{if} \PYG{n}{do\PYGZus{}save}\PYG{p}{:}
        \PYG{n}{figsize}\PYG{p}{,} \PYG{n}{dpi} \PYG{o}{=} \PYG{p}{(}\PYG{n}{b\PYGZus{}in}\PYG{p}{,} \PYG{n}{h\PYGZus{}in}\PYG{p}{)}\PYG{p}{,} \PYG{n}{my\PYGZus{}dpi}
    \PYG{k}{else}\PYG{p}{:}
        \PYG{n}{figsize}\PYG{p}{,} \PYG{n}{dpi} \PYG{o}{=} \PYG{k+kc}{None}\PYG{p}{,} \PYG{k+kc}{None}

    \PYG{c+c1}{\PYGZsh{} data\PYGZus{}list is}
    \PYG{n}{nrows}\PYG{p}{,} \PYG{n}{ncols} \PYG{o}{=} \PYG{l+m+mi}{3}\PYG{p}{,} \PYG{n+nb}{len}\PYG{p}{(}\PYG{n}{data\PYGZus{}list}\PYG{p}{[}\PYG{l+m+mi}{0}\PYG{p}{]}\PYG{p}{)}

    \PYG{k}{if} \PYG{n}{do\PYGZus{}cdf}\PYG{p}{:}
        \PYG{n}{f}\PYG{p}{,} \PYG{n}{axes} \PYG{o}{=} \PYG{n}{plt}\PYG{o}{.}\PYG{n}{subplots}\PYG{p}{(}\PYG{n}{nrows}\PYG{p}{,} \PYG{n}{ncols}\PYG{p}{,} \PYG{n}{sharey}\PYG{o}{=}\PYG{k+kc}{True}\PYG{p}{,} \PYG{n}{figsize}\PYG{o}{=}\PYG{n}{figsize}\PYG{p}{,} \PYG{n}{dpi}\PYG{o}{=}\PYG{n}{dpi}\PYG{p}{)}
    \PYG{k}{else}\PYG{p}{:}
        \PYG{n}{f}\PYG{p}{,} \PYG{n}{axes} \PYG{o}{=} \PYG{n}{plt}\PYG{o}{.}\PYG{n}{subplots}\PYG{p}{(}\PYG{n}{nrows}\PYG{p}{,} \PYG{n}{ncols}\PYG{p}{,} \PYG{n}{sharex}\PYG{o}{=}\PYG{k+kc}{True}\PYG{p}{,} \PYG{n}{figsize}\PYG{o}{=}\PYG{n}{figsize}\PYG{p}{,} \PYG{n}{dpi}\PYG{o}{=}\PYG{n}{dpi}\PYG{p}{)}


    \PYG{c+c1}{\PYGZsh{}}
    \PYG{c+c1}{\PYGZsh{} plot}
    \PYG{c+c1}{\PYGZsh{}}
    \PYG{k}{for} \PYG{n}{i} \PYG{p}{,} \PYG{n}{ax} \PYG{o+ow}{in} \PYG{n+nb}{enumerate}\PYG{p}{(}\PYG{n}{f}\PYG{o}{.}\PYG{n}{axes}\PYG{p}{)}\PYG{p}{:}

        \PYG{c+c1}{\PYGZsh{} indices}
        \PYG{n}{irow} \PYG{o}{=} \PYG{n}{i} \PYG{o}{/}\PYG{o}{/} \PYG{n}{ncols}
        \PYG{n}{jcol} \PYG{o}{=} \PYG{n}{i} \PYG{o}{\PYGZpc{}} \PYG{n}{ncols}

        \PYG{c+c1}{\PYGZsh{} plot data}
        \PYG{n}{temp} \PYG{o}{=} \PYG{n}{data\PYGZus{}list}\PYG{p}{[}\PYG{n}{irow}\PYG{p}{]}\PYG{p}{[}\PYG{n}{jcol}\PYG{p}{]}

        \PYG{k}{for} \PYG{n}{t}\PYG{p}{,} \PYG{n}{color} \PYG{o+ow}{in} \PYG{n+nb}{zip}\PYG{p}{(}\PYG{n}{temp}\PYG{p}{,} \PYG{n}{colors}\PYG{p}{)}\PYG{p}{:}

            \PYG{n}{x\PYGZus{}data}\PYG{p}{,} \PYG{n}{y\PYGZus{}data} \PYG{o}{=} \PYG{n}{t}
            \PYG{k}{if} \PYG{n}{do\PYGZus{}cdf} \PYG{o+ow}{and} \PYG{n}{irow} \PYG{o}{==} \PYG{l+m+mi}{2}\PYG{p}{:}
                \PYG{n}{idx} \PYG{o}{=} \PYG{n}{x\PYGZus{}data} \PYG{o}{\PYGZgt{}}\PYG{o}{=} \PYG{l+m+mi}{0}
                \PYG{n}{ax}\PYG{o}{.}\PYG{n}{plot}\PYG{p}{(}\PYG{n}{x\PYGZus{}data}\PYG{p}{[}\PYG{n}{idx}\PYG{p}{]}\PYG{p}{,} \PYG{n}{y\PYGZus{}data}\PYG{p}{[}\PYG{n}{idx}\PYG{p}{]}\PYG{p}{,} \PYG{n}{color}\PYG{o}{=}\PYG{n}{color}\PYG{p}{,} \PYG{n}{linewidth}\PYG{o}{=}\PYG{n}{linewidth}\PYG{p}{)}
            \PYG{k}{else}\PYG{p}{:}
                \PYG{n}{ax}\PYG{o}{.}\PYG{n}{plot}\PYG{p}{(}\PYG{n}{x\PYGZus{}data}\PYG{p}{,} \PYG{n}{y\PYGZus{}data}\PYG{p}{,} \PYG{n}{color}\PYG{o}{=}\PYG{n}{color}\PYG{p}{,} \PYG{n}{linewidth}\PYG{o}{=}\PYG{n}{linewidth}\PYG{p}{)}

            \PYG{c+c1}{\PYGZsh{}}
            \PYG{c+c1}{\PYGZsh{} set the tick labels}
            \PYG{c+c1}{\PYGZsh{}}
            \PYG{n}{ticksize}\PYG{o}{=}\PYG{l+m+mi}{14}
            \PYG{n}{ax}\PYG{o}{.}\PYG{n}{tick\PYGZus{}params}\PYG{p}{(}\PYG{n}{axis}\PYG{o}{=}\PYG{l+s+s1}{\PYGZsq{}}\PYG{l+s+s1}{both}\PYG{l+s+s1}{\PYGZsq{}}\PYG{p}{,} \PYG{n}{labelsize}\PYG{o}{=}\PYG{n}{ticksize}\PYG{p}{)}

            \PYG{k}{if} \PYG{n}{irow} \PYG{o}{==} \PYG{l+m+mi}{2}\PYG{p}{:}
                \PYG{n}{ax}\PYG{o}{.}\PYG{n}{xaxis}\PYG{o}{.}\PYG{n}{set\PYGZus{}major\PYGZus{}locator}\PYG{p}{(}\PYG{n}{ticker}\PYG{o}{.}\PYG{n}{MaxNLocator}\PYG{p}{(}\PYG{n}{nbins}\PYG{o}{=}\PYG{l+m+mi}{5}\PYG{p}{)}\PYG{p}{)}

            \PYG{k}{if} \PYG{n}{do\PYGZus{}cdf} \PYG{o+ow}{and} \PYG{n}{irow} \PYG{o+ow}{in} \PYG{p}{[}\PYG{l+m+mi}{0}\PYG{p}{,} \PYG{l+m+mi}{1}\PYG{p}{]}\PYG{p}{:}
                \PYG{c+c1}{\PYGZsh{} limit the xaxis to integernumbers}
                \PYG{n}{x\PYGZus{}all} \PYG{o}{=} \PYG{p}{[}\PYG{n}{x}\PYG{o}{.}\PYG{n}{get\PYGZus{}xdata}\PYG{p}{(}\PYG{p}{)} \PYG{k}{for} \PYG{n}{x} \PYG{o+ow}{in} \PYG{n}{ax}\PYG{o}{.}\PYG{n}{lines}\PYG{p}{]}
                \PYG{n}{x\PYGZus{}all} \PYG{o}{=} \PYG{n}{np}\PYG{o}{.}\PYG{n}{hstack}\PYG{p}{(}\PYG{n}{x\PYGZus{}all}\PYG{p}{)}\PYG{o}{.}\PYG{n}{flatten}\PYG{p}{(}\PYG{p}{)}
                \PYG{n}{x\PYGZus{}min}\PYG{p}{,} \PYG{n}{x\PYGZus{}max} \PYG{o}{=} \PYG{n}{np}\PYG{o}{.}\PYG{n}{floor}\PYG{p}{(} \PYG{n}{np}\PYG{o}{.}\PYG{n}{min}\PYG{p}{(}\PYG{n}{x\PYGZus{}all}\PYG{p}{)} \PYG{p}{)}\PYG{p}{,} \PYG{n}{np}\PYG{o}{.}\PYG{n}{ceil}\PYG{p}{(} \PYG{n}{np}\PYG{o}{.}\PYG{n}{max}\PYG{p}{(}\PYG{n}{x\PYGZus{}all}\PYG{p}{)}\PYG{p}{)}
                \PYG{n}{dx} \PYG{o}{=} \PYG{n+nb}{abs}\PYG{p}{(}\PYG{n}{x\PYGZus{}min} \PYG{o}{\PYGZhy{}} \PYG{n}{x\PYGZus{}max}\PYG{p}{)} \PYG{o}{+} \PYG{l+m+mi}{1}
                \PYG{n}{nbins} \PYG{o}{=} \PYG{n}{np}\PYG{o}{.}\PYG{n}{ceil}\PYG{p}{(}\PYG{n}{dx}\PYG{o}{/}\PYG{l+m+mi}{2}\PYG{p}{)}
                \PYG{n}{ax}\PYG{o}{.}\PYG{n}{xaxis}\PYG{o}{.}\PYG{n}{set\PYGZus{}major\PYGZus{}locator}\PYG{p}{(}\PYG{n}{ticker}\PYG{o}{.}\PYG{n}{MaxNLocator}\PYG{p}{(}\PYG{n}{nbins}\PYG{p}{)}\PYG{p}{)}

                \PYG{n}{ax}\PYG{o}{.}\PYG{n}{set\PYGZus{}xlim}\PYG{p}{(}\PYG{n}{x\PYGZus{}min}\PYG{p}{,} \PYG{n}{x\PYGZus{}max}\PYG{p}{)}

                \PYG{c+c1}{\PYGZsh{} set the xticks}
                \PYG{c+c1}{\PYGZsh{} testing}
                \PYG{n}{x\PYGZus{}min} \PYG{o}{=} \PYG{n}{np}\PYG{o}{.}\PYG{n}{round}\PYG{p}{(}\PYG{n}{x\PYGZus{}min}\PYG{p}{)}\PYG{o}{.}\PYG{n}{astype}\PYG{p}{(}\PYG{n+nb}{int}\PYG{p}{)}
                \PYG{n}{x\PYGZus{}max} \PYG{o}{=} \PYG{n}{np}\PYG{o}{.}\PYG{n}{round}\PYG{p}{(}\PYG{n}{x\PYGZus{}max}\PYG{p}{)}\PYG{o}{.}\PYG{n}{astype}\PYG{p}{(}\PYG{n+nb}{int}\PYG{p}{)}
                \PYG{n}{dx} \PYG{o}{=} \PYG{p}{(}\PYG{n}{x\PYGZus{}max} \PYG{o}{\PYGZhy{}} \PYG{n}{x\PYGZus{}min}\PYG{p}{)} \PYG{o}{/} \PYG{p}{(}\PYG{l+m+mi}{5} \PYG{o}{\PYGZhy{}} \PYG{l+m+mi}{1}\PYG{p}{)}
                \PYG{n}{dx} \PYG{o}{=} \PYG{n}{np}\PYG{o}{.}\PYG{n}{floor}\PYG{p}{(}\PYG{n}{dx}\PYG{p}{)}\PYG{o}{.}\PYG{n}{astype}\PYG{p}{(}\PYG{n+nb}{int}\PYG{p}{)}
                \PYG{n}{xticks} \PYG{o}{=} \PYG{n}{np}\PYG{o}{.}\PYG{n}{arange}\PYG{p}{(}\PYG{n}{x\PYGZus{}min}\PYG{p}{,} \PYG{n}{x\PYGZus{}max}\PYG{p}{,} \PYG{n}{dx}\PYG{p}{)}
                \PYG{n}{ax}\PYG{o}{.}\PYG{n}{set\PYGZus{}xticks}\PYG{p}{(}\PYG{n}{xticks}\PYG{p}{)}


    \PYG{c+c1}{\PYGZsh{} main title}
    \PYG{n}{fontsize\PYGZus{}title} \PYG{o}{=} \PYG{l+m+mi}{18}
    \PYG{n}{f}\PYG{o}{.}\PYG{n}{suptitle}\PYG{p}{(}\PYG{n}{main\PYGZus{}title}\PYG{p}{,} \PYG{n}{fontsize}\PYG{o}{=}\PYG{n}{fontsize\PYGZus{}title}\PYG{p}{)}

    \PYG{c+c1}{\PYGZsh{} legend}
    \PYG{n}{f}\PYG{o}{.}\PYG{n}{legend}\PYG{p}{(} \PYG{n}{f}\PYG{o}{.}\PYG{n}{axes}\PYG{p}{[}\PYG{l+m+mi}{0}\PYG{p}{]}\PYG{o}{.}\PYG{n}{lines}\PYG{p}{,} \PYG{n}{legend}\PYG{p}{,} \PYG{l+s+s1}{\PYGZsq{}}\PYG{l+s+s1}{best}\PYG{l+s+s1}{\PYGZsq{}}\PYG{p}{)}

    \PYG{c+c1}{\PYGZsh{}}
    \PYG{c+c1}{\PYGZsh{} set the x\PYGZhy{}axis labels}
    \PYG{c+c1}{\PYGZsh{}}

    \PYG{n}{fontsize\PYGZus{}label} \PYG{o}{=} \PYG{l+m+mi}{18}
    \PYG{k}{for} \PYG{n}{ax}\PYG{p}{,} \PYG{n}{xlabel} \PYG{o+ow}{in} \PYG{n+nb}{zip}\PYG{p}{(} \PYG{n}{axes}\PYG{p}{[}\PYG{n}{nrows}\PYG{o}{\PYGZhy{}}\PYG{l+m+mi}{1}\PYG{p}{,}\PYG{p}{:}\PYG{p}{]}\PYG{p}{,} \PYG{n}{xlabels}\PYG{p}{)} \PYG{p}{:}
        \PYG{n}{ax}\PYG{o}{.}\PYG{n}{set\PYGZus{}xlabel}\PYG{p}{(}\PYG{n}{xlabel}\PYG{p}{,} \PYG{n}{fontsize}\PYG{o}{=}\PYG{n}{fontsize\PYGZus{}label}\PYG{p}{)}

        \PYG{k}{if} \PYG{o+ow}{not} \PYG{n}{do\PYGZus{}cdf}\PYG{p}{:}
            \PYG{n}{x\PYGZus{}min}\PYG{p}{,} \PYG{n}{x\PYGZus{}max} \PYG{o}{=} \PYG{l+m+mi}{0}\PYG{p}{,} \PYG{l+m+mi}{1}
            \PYG{n}{ax}\PYG{o}{.}\PYG{n}{set\PYGZus{}xlim}\PYG{p}{(}\PYG{n}{x\PYGZus{}min}\PYG{p}{,} \PYG{n}{x\PYGZus{}max}\PYG{p}{)}
            \PYG{n}{xticks} \PYG{o}{=} \PYG{n}{np}\PYG{o}{.}\PYG{n}{linspace}\PYG{p}{(}\PYG{n}{x\PYGZus{}min}\PYG{p}{,} \PYG{n}{x\PYGZus{}max}\PYG{p}{,} \PYG{l+m+mi}{3}\PYG{p}{)}
            \PYG{n}{ax}\PYG{o}{.}\PYG{n}{set\PYGZus{}xticks}\PYG{p}{(}\PYG{n}{xticks}\PYG{p}{)}
            \PYG{c+c1}{\PYGZsh{}\PYGZsh{}ax.set\PYGZus{}xticks(xticks, fontsize=20)}
            \PYG{c+c1}{\PYGZsh{}ax.set\PYGZus{}xticklabels(labels=[], fontsize=20)}

    \PYG{c+c1}{\PYGZsh{} set x titles}
    \PYG{k}{for} \PYG{n}{ax}\PYG{p}{,} \PYG{n}{key} \PYG{o+ow}{in} \PYG{n+nb}{zip}\PYG{p}{(}\PYG{n}{axes}\PYG{p}{[}\PYG{l+m+mi}{0}\PYG{p}{,}\PYG{p}{:}\PYG{p}{]}\PYG{p}{,} \PYG{n}{keys}\PYG{p}{)}\PYG{p}{:}
        \PYG{c+c1}{\PYGZsh{}ax.set\PYGZus{}title( activity.INT\PYGZus{}2\PYGZus{}STR[key], fontsize=fontsize\PYGZus{}title )}
        \PYG{n}{ax}\PYG{o}{.}\PYG{n}{set\PYGZus{}title}\PYG{p}{(} \PYG{n}{activity}\PYG{o}{.}\PYG{n}{INT\PYGZus{}2\PYGZus{}STR}\PYG{p}{[}\PYG{n}{key}\PYG{p}{]}\PYG{p}{,} \PYG{n}{fontsize}\PYG{o}{=}\PYG{l+m+mi}{14} \PYG{p}{)}

    \PYG{c+c1}{\PYGZsh{}}
    \PYG{c+c1}{\PYGZsh{} set the y\PYGZhy{}axis labels}
    \PYG{c+c1}{\PYGZsh{}}
    \PYG{k}{for} \PYG{n}{ax}\PYG{p}{,} \PYG{n}{ylabel} \PYG{o+ow}{in} \PYG{n+nb}{zip}\PYG{p}{(}\PYG{n}{axes}\PYG{p}{[}\PYG{p}{:}\PYG{p}{,} \PYG{n}{ncols}\PYG{o}{\PYGZhy{}}\PYG{l+m+mi}{1}\PYG{p}{]}\PYG{p}{,} \PYG{n}{ylabels}\PYG{p}{)}\PYG{p}{:}
        \PYG{n}{ax}\PYG{o}{.}\PYG{n}{yaxis}\PYG{o}{.}\PYG{n}{set\PYGZus{}label\PYGZus{}position}\PYG{p}{(}\PYG{l+s+s1}{\PYGZsq{}}\PYG{l+s+s1}{right}\PYG{l+s+s1}{\PYGZsq{}}\PYG{p}{)}
        \PYG{n}{ax}\PYG{o}{.}\PYG{n}{set\PYGZus{}ylabel}\PYG{p}{(}\PYG{n}{ylabel}\PYG{p}{,} \PYG{n}{fontsize}\PYG{o}{=}\PYG{n}{fontsize\PYGZus{}label}\PYG{p}{,} \PYG{n}{rotation}\PYG{o}{=}\PYG{l+m+mi}{270}\PYG{p}{,} \PYG{n}{labelpad}\PYG{o}{=}\PYG{l+m+mi}{20}\PYG{p}{)}

    \PYG{k}{for} \PYG{n}{i}\PYG{p}{,} \PYG{n}{ax} \PYG{o+ow}{in} \PYG{n+nb}{enumerate}\PYG{p}{(}\PYG{n}{axes}\PYG{p}{[}\PYG{p}{:}\PYG{p}{,}\PYG{l+m+mi}{0}\PYG{p}{]}\PYG{p}{)}\PYG{p}{:}
        \PYG{n}{ax}\PYG{o}{.}\PYG{n}{yaxis}\PYG{o}{.}\PYG{n}{set\PYGZus{}label\PYGZus{}position}\PYG{p}{(}\PYG{l+s+s1}{\PYGZsq{}}\PYG{l+s+s1}{left}\PYG{l+s+s1}{\PYGZsq{}}\PYG{p}{)}
        \PYG{n}{ax}\PYG{o}{.}\PYG{n}{set\PYGZus{}ylabel}\PYG{p}{(}\PYG{n}{yunits}\PYG{p}{[}\PYG{n}{i}\PYG{p}{]}\PYG{p}{,} \PYG{n}{fontsize}\PYG{o}{=}\PYG{n}{fontsize\PYGZus{}label}\PYG{p}{)}

        \PYG{k}{if} \PYG{n}{do\PYGZus{}cdf}\PYG{p}{:}
            \PYG{n}{y\PYGZus{}min}\PYG{p}{,} \PYG{n}{y\PYGZus{}max} \PYG{o}{=} \PYG{l+m+mi}{0}\PYG{p}{,} \PYG{l+m+mi}{1}
            \PYG{n}{ax}\PYG{o}{.}\PYG{n}{set\PYGZus{}ylim}\PYG{p}{(}\PYG{n}{y\PYGZus{}min}\PYG{p}{,} \PYG{n}{y\PYGZus{}max}\PYG{p}{)}

    \PYG{k}{if} \PYG{n}{do\PYGZus{}save} \PYG{o+ow}{and} \PYG{p}{(}\PYG{n}{fname} \PYG{o+ow}{is} \PYG{o+ow}{not} \PYG{k+kc}{None}\PYG{p}{)}\PYG{p}{:}
        \PYG{n}{f}\PYG{o}{.}\PYG{n}{savefig}\PYG{p}{(}\PYG{n}{fname}\PYG{p}{,} \PYG{n}{dpi}\PYG{o}{=}\PYG{n}{my\PYGZus{}dpi}\PYG{p}{)}

    \PYG{k}{return}
\end{sphinxVerbatim}

set up the parameters

\fvset{hllines={, ,}}%
\begin{sphinxVerbatim}[commandchars=\\\{\}]
\PYG{c+c1}{\PYGZsh{}}
\PYG{c+c1}{\PYGZsh{} choose the deomography}
\PYG{c+c1}{\PYGZsh{}}
\PYG{n}{demo} \PYG{o}{=} \PYG{n}{dmg}\PYG{o}{.}\PYG{n}{ADULT\PYGZus{}NON\PYGZus{}WORK}

\PYG{n}{chooser} \PYG{o}{=} \PYG{p}{\PYGZob{}}\PYG{n}{dmg}\PYG{o}{.}\PYG{n}{ADULT\PYGZus{}WORK}\PYG{p}{:} \PYG{n}{cdaw}\PYG{o}{.}\PYG{n}{CHAD\PYGZus{}demography\PYGZus{}adult\PYGZus{}work}\PYG{p}{(}\PYG{p}{)}\PYG{p}{,}
           \PYG{n}{dmg}\PYG{o}{.}\PYG{n}{ADULT\PYGZus{}NON\PYGZus{}WORK}\PYG{p}{:} \PYG{n}{cdanw}\PYG{o}{.}\PYG{n}{CHAD\PYGZus{}demography\PYGZus{}adult\PYGZus{}non\PYGZus{}work}\PYG{p}{(}\PYG{p}{)}\PYG{p}{,}
           \PYG{n}{dmg}\PYG{o}{.}\PYG{n}{CHILD\PYGZus{}SCHOOL}\PYG{p}{:} \PYG{n}{cdcs}\PYG{o}{.}\PYG{n}{CHAD\PYGZus{}demography\PYGZus{}child\PYGZus{}school}\PYG{p}{(}\PYG{p}{)}\PYG{p}{,}
           \PYG{n}{dmg}\PYG{o}{.}\PYG{n}{CHILD\PYGZus{}YOUNG}\PYG{p}{:} \PYG{n}{cdcy}\PYG{o}{.}\PYG{n}{CHAD\PYGZus{}demography\PYGZus{}child\PYGZus{}young}\PYG{p}{(}\PYG{p}{)}\PYG{p}{,}
           \PYG{p}{\PYGZcb{}}

\PYG{c+c1}{\PYGZsh{} the CHAD demogramphy object}
\PYG{n}{chad\PYGZus{}demo} \PYG{o}{=} \PYG{n}{chooser}\PYG{p}{[}\PYG{n}{demo}\PYG{p}{]}

\PYG{c+c1}{\PYGZsh{} the CHAD sampling parameters}
\PYG{n}{s\PYGZus{}params} \PYG{o}{=} \PYG{n}{chad\PYGZus{}demo}\PYG{o}{.}\PYG{n}{int\PYGZus{}2\PYGZus{}param}
\end{sphinxVerbatim}

\fvset{hllines={, ,}}%
\begin{sphinxVerbatim}[commandchars=\\\{\}]
\PYG{c+c1}{\PYGZsh{} save the figures}
\PYG{n}{do\PYGZus{}save\PYGZus{}fig} \PYG{o}{=} \PYG{k+kc}{False}

\PYG{c+c1}{\PYGZsh{} whether or not to show the plots}
\PYG{n}{do\PYGZus{}show} \PYG{o}{=} \PYG{k+kc}{True}

\PYG{c+c1}{\PYGZsh{} the linewidth}
\PYG{n}{linewidth} \PYG{o}{=} \PYG{l+m+mf}{0.5}
\end{sphinxVerbatim}

\fvset{hllines={, ,}}%
\begin{sphinxVerbatim}[commandchars=\\\{\}]
\PYG{c+c1}{\PYGZsh{} use a custom figure directory}
\PYG{n}{fpath} \PYG{o}{=} \PYG{n}{mg}\PYG{o}{.}\PYG{n}{FDIR\PYGZus{}SAVE\PYGZus{}FIG} \PYG{o}{+} \PYG{l+s+s1}{\PYGZsq{}}\PYG{l+s+se}{\PYGZbs{}\PYGZbs{}}\PYG{l+s+s1}{01\PYGZus{}16\PYGZus{}2018\PYGZus{}no\PYGZus{}variation}\PYG{l+s+se}{\PYGZbs{}\PYGZbs{}}\PYG{l+s+s1}{n8192\PYGZus{}d007}\PYG{l+s+s1}{\PYGZsq{}}

\PYG{n}{chooser\PYGZus{}fin} \PYG{o}{=} \PYG{p}{\PYGZob{}}\PYG{n}{dmg}\PYG{o}{.}\PYG{n}{ADULT\PYGZus{}WORK}\PYG{p}{:} \PYG{n}{fpath} \PYG{o}{+} \PYG{l+s+s1}{\PYGZsq{}}\PYG{l+s+se}{\PYGZbs{}\PYGZbs{}}\PYG{l+s+s1}{adult\PYGZus{}work}\PYG{l+s+s1}{\PYGZsq{}}\PYG{p}{,}
       \PYG{n}{dmg}\PYG{o}{.}\PYG{n}{ADULT\PYGZus{}NON\PYGZus{}WORK}\PYG{p}{:} \PYG{n}{fpath} \PYG{o}{+} \PYG{l+s+s1}{\PYGZsq{}}\PYG{l+s+se}{\PYGZbs{}\PYGZbs{}}\PYG{l+s+s1}{adult\PYGZus{}non\PYGZus{}work}\PYG{l+s+s1}{\PYGZsq{}}\PYG{p}{,}
       \PYG{n}{dmg}\PYG{o}{.}\PYG{n}{CHILD\PYGZus{}SCHOOL}\PYG{p}{:} \PYG{n}{fpath} \PYG{o}{+} \PYG{l+s+s1}{\PYGZsq{}}\PYG{l+s+se}{\PYGZbs{}\PYGZbs{}}\PYG{l+s+s1}{child\PYGZus{}school}\PYG{l+s+s1}{\PYGZsq{}}\PYG{p}{,}
       \PYG{n}{dmg}\PYG{o}{.}\PYG{n}{CHILD\PYGZus{}YOUNG}\PYG{p}{:} \PYG{n}{fpath} \PYG{o}{+} \PYG{l+s+s1}{\PYGZsq{}}\PYG{l+s+se}{\PYGZbs{}\PYGZbs{}}\PYG{l+s+s1}{child\PYGZus{}young}\PYG{l+s+s1}{\PYGZsq{}}\PYG{p}{,}
      \PYG{p}{\PYGZcb{}}

\PYG{n}{fpath\PYGZus{}figure\PYGZus{}save} \PYG{o}{=} \PYG{n}{chooser\PYGZus{}fin}\PYG{p}{[}\PYG{n}{demo}\PYG{p}{]}

\PYG{c+c1}{\PYGZsh{} print the save figure directory}
\PYG{n+nb}{print}\PYG{p}{(}\PYG{l+s+s1}{\PYGZsq{}}\PYG{l+s+s1}{the figure save path:}\PYG{l+s+se}{\PYGZbs{}t}\PYG{l+s+si}{\PYGZpc{}s}\PYG{l+s+s1}{\PYGZsq{}} \PYG{o}{\PYGZpc{}} \PYG{n}{fpath\PYGZus{}figure\PYGZus{}save}\PYG{p}{)}

\PYG{c+c1}{\PYGZsh{} different sets of activitiy data to plot}
\PYG{n}{keys\PYGZus{}all} \PYG{o}{=} \PYG{n}{chad\PYGZus{}demo}\PYG{o}{.}\PYG{n}{keys}

\PYG{c+c1}{\PYGZsh{} eating activities}
\PYG{n}{keys\PYGZus{}eat} \PYG{o}{=} \PYG{p}{[}\PYG{n}{mg}\PYG{o}{.}\PYG{n}{KEY\PYGZus{}EAT\PYGZus{}BREAKFAST}\PYG{p}{,} \PYG{n}{mg}\PYG{o}{.}\PYG{n}{KEY\PYGZus{}EAT\PYGZus{}LUNCH}\PYG{p}{,} \PYG{n}{mg}\PYG{o}{.}\PYG{n}{KEY\PYGZus{}EAT\PYGZus{}DINNER}\PYG{p}{]}

\PYG{c+c1}{\PYGZsh{} non\PYGZhy{}eating activities}
\PYG{n}{keys\PYGZus{}not\PYGZus{}eat} \PYG{o}{=} \PYG{p}{[} \PYG{n}{k} \PYG{k}{for} \PYG{n}{k} \PYG{o+ow}{in} \PYG{n}{keys\PYGZus{}all} \PYG{k}{if} \PYG{n}{k} \PYG{o+ow}{not} \PYG{o+ow}{in} \PYG{n}{keys\PYGZus{}eat} \PYG{p}{]}
\end{sphinxVerbatim}
\begin{sphinxalltt}
the figure save path:       ..my\_datafig01\_16\_2018\_no\_variationn8192\_d007adult\_non\_work
\end{sphinxalltt}

Plotting

\fvset{hllines={, ,}}%
\begin{sphinxVerbatim}[commandchars=\\\{\}]
\PYG{n}{DO\PYGZus{}ALL} \PYG{o}{=} \PYG{l+m+mi}{1}
\PYG{n}{DO\PYGZus{}MEALS} \PYG{o}{=} \PYG{l+m+mi}{2}
\PYG{n}{DO\PYGZus{}NOT\PYGZus{}MEALS} \PYG{o}{=} \PYG{l+m+mi}{3}

\PYG{c+c1}{\PYGZsh{} (the activites to plot, part of the file name that matches the keys)}
\PYG{n}{chooser\PYGZus{}keys} \PYG{o}{=} \PYG{p}{\PYGZob{}} \PYG{n}{DO\PYGZus{}ALL}\PYG{p}{:} \PYG{p}{(}\PYG{n}{keys\PYGZus{}all}\PYG{p}{,} \PYG{l+s+s1}{\PYGZsq{}}\PYG{l+s+s1}{all}\PYG{l+s+s1}{\PYGZsq{}}\PYG{p}{)}\PYG{p}{,} \PYGZbs{}
                \PYG{n}{DO\PYGZus{}MEALS}\PYG{p}{:} \PYG{p}{(}\PYG{n}{keys\PYGZus{}eat}\PYG{p}{,} \PYG{l+s+s1}{\PYGZsq{}}\PYG{l+s+s1}{meals}\PYG{l+s+s1}{\PYGZsq{}}\PYG{p}{)}\PYG{p}{,}\PYGZbs{}
                \PYG{n}{DO\PYGZus{}NOT\PYGZus{}MEALS}\PYG{p}{:} \PYG{p}{(}\PYG{n}{keys\PYGZus{}not\PYGZus{}eat}\PYG{p}{,} \PYG{l+s+s1}{\PYGZsq{}}\PYG{l+s+s1}{not\PYGZus{}meals}\PYG{l+s+s1}{\PYGZsq{}}\PYG{p}{)}\PYG{p}{,}
               \PYG{p}{\PYGZcb{}}
\end{sphinxVerbatim}

\fvset{hllines={, ,}}%
\begin{sphinxVerbatim}[commandchars=\\\{\}]
\PYG{c+c1}{\PYGZsh{}}
\PYG{c+c1}{\PYGZsh{} set the activities to plot}
\PYG{c+c1}{\PYGZsh{}}
\PYG{n}{plot\PYGZus{}keys} \PYG{o}{=} \PYG{n}{DO\PYGZus{}ALL}

\PYG{n}{keys}\PYG{p}{,} \PYG{n}{fname\PYGZus{}keys} \PYG{o}{=} \PYG{n}{chooser\PYGZus{}keys}\PYG{p}{[}\PYG{n}{plot\PYGZus{}keys}\PYG{p}{]}
\PYG{n}{name\PYGZus{}keys} \PYG{o}{=} \PYG{p}{[} \PYG{n}{activity}\PYG{o}{.}\PYG{n}{INT\PYGZus{}2\PYGZus{}STR}\PYG{p}{[}\PYG{n}{k}\PYG{p}{]} \PYG{k}{for} \PYG{n}{k} \PYG{o+ow}{in} \PYG{n}{keys}\PYG{p}{]}


\PYG{c+c1}{\PYGZsh{} labels on the right hand side of the plot}
\PYG{n}{ylabels} \PYG{o}{=} \PYG{p}{[}\PYG{l+s+s1}{\PYGZsq{}}\PYG{l+s+s1}{Start Time}\PYG{l+s+s1}{\PYGZsq{}}\PYG{p}{,} \PYG{l+s+s1}{\PYGZsq{}}\PYG{l+s+s1}{End Time}\PYG{l+s+s1}{\PYGZsq{}}\PYG{p}{,} \PYG{l+s+s1}{\PYGZsq{}}\PYG{l+s+s1}{Duration}\PYG{l+s+s1}{\PYGZsq{}}\PYG{p}{]}
\end{sphinxVerbatim}

Plot CDFs vs Longitudinal data

plot verification

\fvset{hllines={, ,}}%
\begin{sphinxVerbatim}[commandchars=\\\{\}]
\PYG{n}{fpaths} \PYG{o}{=} \PYG{n}{analyzer}\PYG{o}{.}\PYG{n}{get\PYGZus{}verify\PYGZus{}fpath}\PYG{p}{(}\PYG{n}{fpath\PYGZus{}figure\PYGZus{}save}\PYG{p}{,} \PYG{n}{keys}\PYG{p}{)}
\end{sphinxVerbatim}

\fvset{hllines={, ,}}%
\begin{sphinxVerbatim}[commandchars=\\\{\}]
\PYG{c+c1}{\PYGZsh{}}
\PYG{c+c1}{\PYGZsh{} plot the verification cdf}
\PYG{c+c1}{\PYGZsh{}}

\PYG{c+c1}{\PYGZsh{} load the data}
\PYG{n}{fname} \PYG{o}{=} \PYG{l+s+s1}{\PYGZsq{}}\PYG{l+s+se}{\PYGZbs{}\PYGZbs{}}\PYG{l+s+s1}{cdf\PYGZus{}}\PYG{l+s+s1}{\PYGZsq{}} \PYG{o}{+} \PYG{n}{fname\PYGZus{}keys} \PYG{o}{+} \PYG{l+s+s1}{\PYGZsq{}}\PYG{l+s+s1}{.png}\PYG{l+s+s1}{\PYGZsq{}}
\PYG{n}{data\PYGZus{}list\PYGZus{}all}\PYG{p}{,} \PYG{n}{fname\PYGZus{}subplot} \PYG{o}{=} \PYG{n}{plotter}\PYG{o}{.}\PYG{n}{get\PYGZus{}figure\PYGZus{}data}\PYG{p}{(}\PYG{n}{fpaths}\PYG{p}{,} \PYG{n}{fpath\PYGZus{}figure\PYGZus{}save}\PYG{p}{,} \PYG{n}{fname}\PYG{p}{)}

\PYG{c+c1}{\PYGZsh{}}
\PYG{c+c1}{\PYGZsh{} plotting parameters}
\PYG{c+c1}{\PYGZsh{}}
\PYG{n}{do\PYGZus{}cdf} \PYG{o}{=} \PYG{k+kc}{True}

\PYG{n}{colors} \PYG{o}{=} \PYG{p}{[}\PYG{l+s+s1}{\PYGZsq{}}\PYG{l+s+s1}{blue}\PYG{l+s+s1}{\PYGZsq{}}\PYG{p}{,} \PYG{l+s+s1}{\PYGZsq{}}\PYG{l+s+s1}{red}\PYG{l+s+s1}{\PYGZsq{}}\PYG{p}{]}
\PYG{n}{legend} \PYG{o}{=} \PYG{p}{[}\PYG{l+s+s1}{\PYGZsq{}}\PYG{l+s+s1}{Predicted}\PYG{l+s+s1}{\PYGZsq{}}\PYG{p}{,} \PYG{l+s+s1}{\PYGZsq{}}\PYG{l+s+s1}{Means (CHAD)}\PYG{l+s+s1}{\PYGZsq{}}\PYG{p}{]}

\PYG{n}{xunits} \PYG{o}{=} \PYG{l+s+s1}{\PYGZsq{}}\PYG{l+s+s1}{Hours}\PYG{l+s+s1}{\PYGZsq{}}
\PYG{n}{yunits} \PYG{o}{=} \PYG{p}{[}\PYG{l+s+s1}{\PYGZsq{}}\PYG{l+s+s1}{Quantile}\PYG{l+s+s1}{\PYGZsq{}}\PYG{p}{]} \PYG{o}{*} \PYG{l+m+mi}{3}

\PYG{n}{main\PYGZus{}title} \PYG{o}{=} \PYG{l+s+s1}{\PYGZsq{}}\PYG{l+s+s1}{CDFs of Activity\PYGZhy{}parameters}\PYG{l+s+s1}{\PYGZsq{}}

\PYG{n}{xlabels} \PYG{o}{=} \PYG{p}{[}\PYG{n}{xunits}\PYG{p}{]} \PYG{o}{*} \PYG{n+nb}{len}\PYG{p}{(}\PYG{n}{keys}\PYG{p}{)}

\PYG{c+c1}{\PYGZsh{}}
\PYG{c+c1}{\PYGZsh{} plot}
\PYG{c+c1}{\PYGZsh{}}

\PYG{n}{plot\PYGZus{}subplots}\PYG{p}{(}\PYG{n}{data\PYGZus{}list}\PYG{o}{=}\PYG{n}{data\PYGZus{}list\PYGZus{}all}\PYG{p}{,} \PYG{n}{do\PYGZus{}cdf}\PYG{o}{=}\PYG{n}{do\PYGZus{}cdf}\PYG{p}{,} \PYG{n}{main\PYGZus{}title}\PYG{o}{=}\PYG{n}{main\PYGZus{}title}\PYG{p}{,} \PYG{n}{legend}\PYG{o}{=}\PYG{n}{legend}\PYG{p}{,} \PYGZbs{}
                  \PYG{n}{xlabels}\PYG{o}{=}\PYG{n}{xlabels}\PYG{p}{,} \PYG{n}{ylabels}\PYG{o}{=}\PYG{n}{ylabels}\PYG{p}{,} \PYG{n}{xunits}\PYG{o}{=}\PYG{n}{xunits}\PYG{p}{,} \PYG{n}{yunits}\PYG{o}{=}\PYG{n}{yunits}\PYG{p}{,} \PYG{n}{colors}\PYG{o}{=}\PYG{n}{colors}\PYG{p}{,} \PYGZbs{}
                  \PYG{n}{do\PYGZus{}save}\PYG{o}{=}\PYG{n}{do\PYGZus{}save\PYGZus{}fig}\PYG{p}{,} \PYG{n}{fname}\PYG{o}{=}\PYG{n}{fname\PYGZus{}subplot}\PYG{p}{,} \PYG{n}{linewidth}\PYG{o}{=}\PYG{n}{linewidth}\PYG{p}{)}

\PYG{k}{if} \PYG{n}{do\PYGZus{}show}\PYG{p}{:}
    \PYG{n}{plt}\PYG{o}{.}\PYG{n}{show}\PYG{p}{(}\PYG{p}{)}
\PYG{k}{else}\PYG{p}{:}
    \PYG{n}{plt}\PYG{o}{.}\PYG{n}{close}\PYG{p}{(}\PYG{p}{)}
\end{sphinxVerbatim}
\begin{sphinxalltt}
C:UsersnbrandonAppDataLocalContinuumAnaconda3libsite-packagesmatplotliblegend.py:338: UserWarning: Automatic legend placement (loc="best") not implemented for figure legend. Falling back on "upper right".
  warnings.warn('Automatic legend placement (loc="best") not '
\end{sphinxalltt}

Plot CDFs vs random days

\fvset{hllines={, ,}}%
\begin{sphinxVerbatim}[commandchars=\\\{\}]
\PYG{c+c1}{\PYGZsh{} choose the activities to plot}
\PYG{c+c1}{\PYGZsh{} get the figure directories}
\PYG{n}{fpaths} \PYG{o}{=} \PYG{p}{[} \PYG{p}{(}\PYG{n}{fpath\PYGZus{}figure\PYGZus{}save} \PYG{o}{+} \PYG{n}{mg}\PYG{o}{.}\PYG{n}{KEY\PYGZus{}2\PYGZus{}FDIR\PYGZus{}SAVE\PYGZus{}FIG}\PYG{p}{[}\PYG{n}{k}\PYG{p}{]} \PYG{o}{+} \PYG{n}{mg}\PYG{o}{.}\PYG{n}{FDIR\PYGZus{}SAVE\PYGZus{}FIG\PYGZus{}RANDOM\PYGZus{}DAY}\PYG{p}{)} \PYG{k}{for} \PYG{n}{k} \PYG{o+ow}{in} \PYG{n}{keys}\PYG{p}{]}
\end{sphinxVerbatim}

plot the cdf

\fvset{hllines={, ,}}%
\begin{sphinxVerbatim}[commandchars=\\\{\}]
\PYG{c+c1}{\PYGZsh{}}
\PYG{c+c1}{\PYGZsh{} plot the CDF}
\PYG{c+c1}{\PYGZsh{}}

\PYG{n}{fname} \PYG{o}{=} \PYG{l+s+s1}{\PYGZsq{}}\PYG{l+s+se}{\PYGZbs{}\PYGZbs{}}\PYG{l+s+s1}{cdf\PYGZus{}}\PYG{l+s+s1}{\PYGZsq{}} \PYG{o}{+} \PYG{n}{fname\PYGZus{}keys} \PYG{o}{+} \PYG{l+s+s1}{\PYGZsq{}}\PYG{l+s+s1}{.png}\PYG{l+s+s1}{\PYGZsq{}}
\PYG{n}{fnames\PYGZus{}load} \PYG{o}{=} \PYG{p}{(}\PYG{l+s+s1}{\PYGZsq{}}\PYG{l+s+se}{\PYGZbs{}\PYGZbs{}}\PYG{l+s+s1}{cdf\PYGZus{}start.pkl}\PYG{l+s+s1}{\PYGZsq{}}\PYG{p}{,} \PYG{l+s+s1}{\PYGZsq{}}\PYG{l+s+se}{\PYGZbs{}\PYGZbs{}}\PYG{l+s+s1}{cdf\PYGZus{}end.pkl}\PYG{l+s+s1}{\PYGZsq{}}\PYG{p}{,} \PYG{l+s+s1}{\PYGZsq{}}\PYG{l+s+se}{\PYGZbs{}\PYGZbs{}}\PYG{l+s+s1}{cdf\PYGZus{}dt.pkl}\PYG{l+s+s1}{\PYGZsq{}}\PYG{p}{)}

\PYG{c+c1}{\PYGZsh{} load the data}
\PYG{n}{data\PYGZus{}list\PYGZus{}all}\PYG{p}{,} \PYG{n}{fname\PYGZus{}subplot} \PYG{o}{=} \PYG{n}{plotter}\PYG{o}{.}\PYG{n}{get\PYGZus{}figure\PYGZus{}data}\PYG{p}{(}\PYG{n}{fpaths}\PYG{p}{,} \PYG{n}{fpath\PYGZus{}figure\PYGZus{}save}\PYG{p}{,} \PYG{n}{fname}\PYG{p}{,} \PYG{n}{fnames\PYGZus{}load}\PYG{o}{=}\PYG{n}{fnames\PYGZus{}load}\PYG{p}{)}

\PYG{c+c1}{\PYGZsh{}}
\PYG{c+c1}{\PYGZsh{} plotting parameters}
\PYG{c+c1}{\PYGZsh{}}
\PYG{n}{do\PYGZus{}cdf} \PYG{o}{=} \PYG{k+kc}{True}

\PYG{n}{colors} \PYG{o}{=} \PYG{p}{[}\PYG{l+s+s1}{\PYGZsq{}}\PYG{l+s+s1}{blue}\PYG{l+s+s1}{\PYGZsq{}}\PYG{p}{,} \PYG{l+s+s1}{\PYGZsq{}}\PYG{l+s+s1}{red}\PYG{l+s+s1}{\PYGZsq{}}\PYG{p}{]}
\PYG{n}{legend} \PYG{o}{=} \PYG{p}{[}\PYG{l+s+s1}{\PYGZsq{}}\PYG{l+s+s1}{Predicted}\PYG{l+s+s1}{\PYGZsq{}}\PYG{p}{,} \PYG{l+s+s1}{\PYGZsq{}}\PYG{l+s+s1}{Observed}\PYG{l+s+s1}{\PYGZsq{}}\PYG{p}{]}

\PYG{n}{xunits} \PYG{o}{=} \PYG{l+s+s1}{\PYGZsq{}}\PYG{l+s+s1}{Hours}\PYG{l+s+s1}{\PYGZsq{}}
\PYG{n}{yunits} \PYG{o}{=} \PYG{p}{[}\PYG{l+s+s1}{\PYGZsq{}}\PYG{l+s+s1}{Quantile}\PYG{l+s+s1}{\PYGZsq{}}\PYG{p}{]} \PYG{o}{*} \PYG{l+m+mi}{3}

\PYG{n}{main\PYGZus{}title} \PYG{o}{=} \PYG{l+s+s1}{\PYGZsq{}}\PYG{l+s+s1}{CDFs of Activity\PYGZhy{}parameters}\PYG{l+s+s1}{\PYGZsq{}}

\PYG{n}{xlabels} \PYG{o}{=} \PYG{p}{[}\PYG{n}{xunits}\PYG{p}{]} \PYG{o}{*} \PYG{n+nb}{len}\PYG{p}{(}\PYG{n}{keys}\PYG{p}{)}

\PYG{c+c1}{\PYGZsh{}}
\PYG{c+c1}{\PYGZsh{} plot}
\PYG{c+c1}{\PYGZsh{}}

\PYG{n}{plot\PYGZus{}subplots}\PYG{p}{(}\PYG{n}{data\PYGZus{}list}\PYG{o}{=}\PYG{n}{data\PYGZus{}list\PYGZus{}all}\PYG{p}{,} \PYG{n}{do\PYGZus{}cdf}\PYG{o}{=}\PYG{n}{do\PYGZus{}cdf}\PYG{p}{,} \PYG{n}{main\PYGZus{}title}\PYG{o}{=}\PYG{n}{main\PYGZus{}title}\PYG{p}{,} \PYG{n}{legend}\PYG{o}{=}\PYG{n}{legend}\PYG{p}{,} \PYGZbs{}
                  \PYG{n}{xlabels}\PYG{o}{=}\PYG{n}{xlabels}\PYG{p}{,} \PYG{n}{ylabels}\PYG{o}{=}\PYG{n}{ylabels}\PYG{p}{,} \PYG{n}{xunits}\PYG{o}{=}\PYG{n}{xunits}\PYG{p}{,} \PYG{n}{yunits}\PYG{o}{=}\PYG{n}{yunits}\PYG{p}{,} \PYG{n}{colors}\PYG{o}{=}\PYG{n}{colors}\PYG{p}{,} \PYGZbs{}
                  \PYG{n}{do\PYGZus{}save}\PYG{o}{=}\PYG{n}{do\PYGZus{}save\PYGZus{}fig}\PYG{p}{,} \PYG{n}{fname}\PYG{o}{=}\PYG{n}{fname\PYGZus{}subplot}\PYG{p}{,} \PYG{n}{linewidth}\PYG{o}{=}\PYG{n}{linewidth}\PYG{p}{)}

\PYG{k}{if} \PYG{n}{do\PYGZus{}show}\PYG{p}{:}
    \PYG{n}{plt}\PYG{o}{.}\PYG{n}{show}\PYG{p}{(}\PYG{p}{)}
\PYG{k}{else}\PYG{p}{:}
    \PYG{n}{plt}\PYG{o}{.}\PYG{n}{close}\PYG{p}{(}\PYG{p}{)}
\end{sphinxVerbatim}
\begin{sphinxalltt}
C:UsersnbrandonAppDataLocalContinuumAnaconda3libsite-packagesmatplotliblegend.py:338: UserWarning: Automatic legend placement (loc="best") not implemented for figure legend. Falling back on "upper right".
  warnings.warn('Automatic legend placement (loc="best") not '
\end{sphinxalltt}

Plot the Inverse CDF

\fvset{hllines={, ,}}%
\begin{sphinxVerbatim}[commandchars=\\\{\}]
\PYG{c+c1}{\PYGZsh{}}
\PYG{c+c1}{\PYGZsh{} plot the Inverse CDF}
\PYG{c+c1}{\PYGZsh{}}

\PYG{n}{fname} \PYG{o}{=} \PYG{l+s+s1}{\PYGZsq{}}\PYG{l+s+se}{\PYGZbs{}\PYGZbs{}}\PYG{l+s+s1}{cdf\PYGZus{}inv\PYGZus{}}\PYG{l+s+s1}{\PYGZsq{}} \PYG{o}{+} \PYG{n}{fname\PYGZus{}keys} \PYG{o}{+} \PYG{l+s+s1}{\PYGZsq{}}\PYG{l+s+s1}{.png}\PYG{l+s+s1}{\PYGZsq{}}
\PYG{n}{fnames\PYGZus{}load} \PYG{o}{=} \PYG{p}{(}\PYG{l+s+s1}{\PYGZsq{}}\PYG{l+s+se}{\PYGZbs{}\PYGZbs{}}\PYG{l+s+s1}{cdf\PYGZus{}inv\PYGZus{}start.pkl}\PYG{l+s+s1}{\PYGZsq{}}\PYG{p}{,} \PYG{l+s+s1}{\PYGZsq{}}\PYG{l+s+se}{\PYGZbs{}\PYGZbs{}}\PYG{l+s+s1}{cdf\PYGZus{}inv\PYGZus{}end.pkl}\PYG{l+s+s1}{\PYGZsq{}}\PYG{p}{,} \PYG{l+s+s1}{\PYGZsq{}}\PYG{l+s+se}{\PYGZbs{}\PYGZbs{}}\PYG{l+s+s1}{cdf\PYGZus{}inv\PYGZus{}dt.pkl}\PYG{l+s+s1}{\PYGZsq{}}\PYG{p}{)}

\PYG{c+c1}{\PYGZsh{} load the data}
\PYG{n}{data\PYGZus{}list\PYGZus{}all}\PYG{p}{,} \PYG{n}{fname\PYGZus{}subplot} \PYG{o}{=} \PYG{n}{plotter}\PYG{o}{.}\PYG{n}{get\PYGZus{}figure\PYGZus{}data}\PYG{p}{(}\PYG{n}{fpaths}\PYG{p}{,} \PYG{n}{fpath\PYGZus{}figure\PYGZus{}save}\PYG{p}{,} \PYG{n}{fname}\PYG{p}{,} \PYG{n}{fnames\PYGZus{}load}\PYG{o}{=}\PYG{n}{fnames\PYGZus{}load}\PYG{p}{)}

\PYG{c+c1}{\PYGZsh{}}
\PYG{c+c1}{\PYGZsh{} plotting parameters}
\PYG{c+c1}{\PYGZsh{}}
\PYG{n}{do\PYGZus{}cdf} \PYG{o}{=} \PYG{k+kc}{True}

\PYG{n}{colors} \PYG{o}{=} \PYG{p}{[}\PYG{l+s+s1}{\PYGZsq{}}\PYG{l+s+s1}{blue}\PYG{l+s+s1}{\PYGZsq{}}\PYG{p}{,} \PYG{l+s+s1}{\PYGZsq{}}\PYG{l+s+s1}{red}\PYG{l+s+s1}{\PYGZsq{}}\PYG{p}{]}
\PYG{n}{legend} \PYG{o}{=} \PYG{p}{[}\PYG{l+s+s1}{\PYGZsq{}}\PYG{l+s+s1}{Predicted}\PYG{l+s+s1}{\PYGZsq{}}\PYG{p}{,} \PYG{l+s+s1}{\PYGZsq{}}\PYG{l+s+s1}{Observed}\PYG{l+s+s1}{\PYGZsq{}}\PYG{p}{]}

\PYG{n}{xunits} \PYG{o}{=} \PYG{l+s+s1}{\PYGZsq{}}\PYG{l+s+s1}{Hours}\PYG{l+s+s1}{\PYGZsq{}}
\PYG{n}{yunits} \PYG{o}{=} \PYG{p}{[}\PYG{l+s+s1}{\PYGZsq{}}\PYG{l+s+s1}{Quantile}\PYG{l+s+s1}{\PYGZsq{}}\PYG{p}{]} \PYG{o}{*} \PYG{l+m+mi}{3}

\PYG{n}{main\PYGZus{}title} \PYG{o}{=} \PYG{l+s+s1}{\PYGZsq{}}\PYG{l+s+s1}{Inverse CDFs of Activity\PYGZhy{}parameters}\PYG{l+s+s1}{\PYGZsq{}}

\PYG{n}{xlabels} \PYG{o}{=} \PYG{p}{[}\PYG{n}{xunits}\PYG{p}{]} \PYG{o}{*} \PYG{n+nb}{len}\PYG{p}{(}\PYG{n}{keys}\PYG{p}{)}

\PYG{c+c1}{\PYGZsh{}}
\PYG{c+c1}{\PYGZsh{} plot}
\PYG{c+c1}{\PYGZsh{}}

\PYG{n}{plot\PYGZus{}subplots}\PYG{p}{(}\PYG{n}{data\PYGZus{}list}\PYG{o}{=}\PYG{n}{data\PYGZus{}list\PYGZus{}all}\PYG{p}{,} \PYG{n}{do\PYGZus{}cdf}\PYG{o}{=}\PYG{n}{do\PYGZus{}cdf}\PYG{p}{,} \PYG{n}{main\PYGZus{}title}\PYG{o}{=}\PYG{n}{main\PYGZus{}title}\PYG{p}{,} \PYG{n}{legend}\PYG{o}{=}\PYG{n}{legend}\PYG{p}{,} \PYGZbs{}
                  \PYG{n}{xlabels}\PYG{o}{=}\PYG{n}{xlabels}\PYG{p}{,} \PYG{n}{ylabels}\PYG{o}{=}\PYG{n}{ylabels}\PYG{p}{,} \PYG{n}{xunits}\PYG{o}{=}\PYG{n}{xunits}\PYG{p}{,} \PYG{n}{yunits}\PYG{o}{=}\PYG{n}{yunits}\PYG{p}{,} \PYG{n}{colors}\PYG{o}{=}\PYG{n}{colors}\PYG{p}{,} \PYGZbs{}
                  \PYG{n}{do\PYGZus{}save}\PYG{o}{=}\PYG{n}{do\PYGZus{}save\PYGZus{}fig}\PYG{p}{,} \PYG{n}{fname}\PYG{o}{=}\PYG{n}{fname\PYGZus{}subplot}\PYG{p}{,} \PYG{n}{linewidth}\PYG{o}{=}\PYG{n}{linewidth}\PYG{p}{)}

\PYG{k}{if} \PYG{n}{do\PYGZus{}show}\PYG{p}{:}
    \PYG{n}{plt}\PYG{o}{.}\PYG{n}{show}\PYG{p}{(}\PYG{p}{)}
\PYG{k}{else}\PYG{p}{:}
    \PYG{n}{plt}\PYG{o}{.}\PYG{n}{close}\PYG{p}{(}\PYG{p}{)}
\end{sphinxVerbatim}
\begin{sphinxalltt}
C:UsersnbrandonAppDataLocalContinuumAnaconda3libsite-packagesipykernel\_launcher.py:73: RuntimeWarning: divide by zero encountered in long\_scalars
\end{sphinxalltt}

\fvset{hllines={, ,}}%
\begin{sphinxVerbatim}[commandchars=\\\{\}]
\PYG{o}{\PYGZhy{}}\PYG{o}{\PYGZhy{}}\PYG{o}{\PYGZhy{}}\PYG{o}{\PYGZhy{}}\PYG{o}{\PYGZhy{}}\PYG{o}{\PYGZhy{}}\PYG{o}{\PYGZhy{}}\PYG{o}{\PYGZhy{}}\PYG{o}{\PYGZhy{}}\PYG{o}{\PYGZhy{}}\PYG{o}{\PYGZhy{}}\PYG{o}{\PYGZhy{}}\PYG{o}{\PYGZhy{}}\PYG{o}{\PYGZhy{}}\PYG{o}{\PYGZhy{}}\PYG{o}{\PYGZhy{}}\PYG{o}{\PYGZhy{}}\PYG{o}{\PYGZhy{}}\PYG{o}{\PYGZhy{}}\PYG{o}{\PYGZhy{}}\PYG{o}{\PYGZhy{}}\PYG{o}{\PYGZhy{}}\PYG{o}{\PYGZhy{}}\PYG{o}{\PYGZhy{}}\PYG{o}{\PYGZhy{}}\PYG{o}{\PYGZhy{}}\PYG{o}{\PYGZhy{}}\PYG{o}{\PYGZhy{}}\PYG{o}{\PYGZhy{}}\PYG{o}{\PYGZhy{}}\PYG{o}{\PYGZhy{}}\PYG{o}{\PYGZhy{}}\PYG{o}{\PYGZhy{}}\PYG{o}{\PYGZhy{}}\PYG{o}{\PYGZhy{}}\PYG{o}{\PYGZhy{}}\PYG{o}{\PYGZhy{}}\PYG{o}{\PYGZhy{}}\PYG{o}{\PYGZhy{}}\PYG{o}{\PYGZhy{}}\PYG{o}{\PYGZhy{}}\PYG{o}{\PYGZhy{}}\PYG{o}{\PYGZhy{}}\PYG{o}{\PYGZhy{}}\PYG{o}{\PYGZhy{}}\PYG{o}{\PYGZhy{}}\PYG{o}{\PYGZhy{}}\PYG{o}{\PYGZhy{}}\PYG{o}{\PYGZhy{}}\PYG{o}{\PYGZhy{}}\PYG{o}{\PYGZhy{}}\PYG{o}{\PYGZhy{}}\PYG{o}{\PYGZhy{}}\PYG{o}{\PYGZhy{}}\PYG{o}{\PYGZhy{}}\PYG{o}{\PYGZhy{}}\PYG{o}{\PYGZhy{}}\PYG{o}{\PYGZhy{}}\PYG{o}{\PYGZhy{}}\PYG{o}{\PYGZhy{}}\PYG{o}{\PYGZhy{}}\PYG{o}{\PYGZhy{}}\PYG{o}{\PYGZhy{}}\PYG{o}{\PYGZhy{}}\PYG{o}{\PYGZhy{}}\PYG{o}{\PYGZhy{}}\PYG{o}{\PYGZhy{}}\PYG{o}{\PYGZhy{}}\PYG{o}{\PYGZhy{}}\PYG{o}{\PYGZhy{}}\PYG{o}{\PYGZhy{}}\PYG{o}{\PYGZhy{}}\PYG{o}{\PYGZhy{}}\PYG{o}{\PYGZhy{}}\PYG{o}{\PYGZhy{}}

\PYG{n+ne}{ValueError}                                \PYG{n}{Traceback} \PYG{p}{(}\PYG{n}{most} \PYG{n}{recent} \PYG{n}{call} \PYG{n}{last}\PYG{p}{)}

\PYG{o}{\PYGZlt{}}\PYG{n}{ipython}\PYG{o}{\PYGZhy{}}\PYG{n+nb}{input}\PYG{o}{\PYGZhy{}}\PYG{l+m+mi}{39}\PYG{o}{\PYGZhy{}}\PYG{l+m+mi}{2}\PYG{n}{c25156c1693}\PYG{o}{\PYGZgt{}} \PYG{o+ow}{in} \PYG{o}{\PYGZlt{}}\PYG{n}{module}\PYG{o}{\PYGZgt{}}\PYG{p}{(}\PYG{p}{)}
     \PYG{l+m+mi}{28} \PYG{c+c1}{\PYGZsh{}}
     \PYG{l+m+mi}{29}
\PYG{o}{\PYGZhy{}}\PYG{o}{\PYGZhy{}}\PYG{o}{\PYGZhy{}}\PYG{o}{\PYGZgt{}} \PYG{l+m+mi}{30} \PYG{n}{plot\PYGZus{}subplots}\PYG{p}{(}\PYG{n}{data\PYGZus{}list}\PYG{o}{=}\PYG{n}{data\PYGZus{}list\PYGZus{}all}\PYG{p}{,} \PYG{n}{do\PYGZus{}cdf}\PYG{o}{=}\PYG{n}{do\PYGZus{}cdf}\PYG{p}{,} \PYG{n}{main\PYGZus{}title}\PYG{o}{=}\PYG{n}{main\PYGZus{}title}\PYG{p}{,} \PYG{n}{legend}\PYG{o}{=}\PYG{n}{legend}\PYG{p}{,}                   \PYG{n}{xlabels}\PYG{o}{=}\PYG{n}{xlabels}\PYG{p}{,} \PYG{n}{ylabels}\PYG{o}{=}\PYG{n}{ylabels}\PYG{p}{,} \PYG{n}{xunits}\PYG{o}{=}\PYG{n}{xunits}\PYG{p}{,} \PYG{n}{yunits}\PYG{o}{=}\PYG{n}{yunits}\PYG{p}{,} \PYG{n}{colors}\PYG{o}{=}\PYG{n}{colors}\PYG{p}{,}                   \PYG{n}{do\PYGZus{}save}\PYG{o}{=}\PYG{n}{do\PYGZus{}save\PYGZus{}fig}\PYG{p}{,} \PYG{n}{fname}\PYG{o}{=}\PYG{n}{fname\PYGZus{}subplot}\PYG{p}{,} \PYG{n}{linewidth}\PYG{o}{=}\PYG{n}{linewidth}\PYG{p}{)}
     \PYG{l+m+mi}{31}
     \PYG{l+m+mi}{32} \PYG{k}{if} \PYG{n}{do\PYGZus{}show}\PYG{p}{:}


\PYG{o}{\PYGZlt{}}\PYG{n}{ipython}\PYG{o}{\PYGZhy{}}\PYG{n+nb}{input}\PYG{o}{\PYGZhy{}}\PYG{l+m+mi}{3}\PYG{o}{\PYGZhy{}}\PYG{l+m+mi}{8}\PYG{n}{a15175d88ba}\PYG{o}{\PYGZgt{}} \PYG{o+ow}{in} \PYG{n}{plot\PYGZus{}subplots}\PYG{p}{(}\PYG{n}{data\PYGZus{}list}\PYG{p}{,} \PYG{n}{do\PYGZus{}cdf}\PYG{p}{,} \PYG{n}{main\PYGZus{}title}\PYG{p}{,} \PYG{n}{legend}\PYG{p}{,} \PYG{n}{xlabels}\PYG{p}{,} \PYG{n}{ylabels}\PYG{p}{,} \PYG{n}{xunits}\PYG{p}{,} \PYG{n}{yunits}\PYG{p}{,} \PYG{n}{colors}\PYG{p}{,} \PYG{n}{do\PYGZus{}save}\PYG{p}{,} \PYG{n}{fname}\PYG{p}{,} \PYG{n}{linewidth}\PYG{p}{)}
     \PYG{l+m+mi}{71}                 \PYG{n}{dx} \PYG{o}{=} \PYG{p}{(}\PYG{n}{x\PYGZus{}max} \PYG{o}{\PYGZhy{}} \PYG{n}{x\PYGZus{}min}\PYG{p}{)} \PYG{o}{/} \PYG{p}{(}\PYG{l+m+mi}{5} \PYG{o}{\PYGZhy{}} \PYG{l+m+mi}{1}\PYG{p}{)}
     \PYG{l+m+mi}{72}                 \PYG{n}{dx} \PYG{o}{=} \PYG{n}{np}\PYG{o}{.}\PYG{n}{floor}\PYG{p}{(}\PYG{n}{dx}\PYG{p}{)}\PYG{o}{.}\PYG{n}{astype}\PYG{p}{(}\PYG{n+nb}{int}\PYG{p}{)}
\PYG{o}{\PYGZhy{}}\PYG{o}{\PYGZhy{}}\PYG{o}{\PYGZhy{}}\PYG{o}{\PYGZgt{}} \PYG{l+m+mi}{73}                 \PYG{n}{xticks} \PYG{o}{=} \PYG{n}{np}\PYG{o}{.}\PYG{n}{arange}\PYG{p}{(}\PYG{n}{x\PYGZus{}min}\PYG{p}{,} \PYG{n}{x\PYGZus{}max}\PYG{p}{,} \PYG{n}{dx}\PYG{p}{)}
     \PYG{l+m+mi}{74}                 \PYG{n}{ax}\PYG{o}{.}\PYG{n}{set\PYGZus{}xticks}\PYG{p}{(}\PYG{n}{xticks}\PYG{p}{)}
     \PYG{l+m+mi}{75}


\PYG{n+ne}{ValueError}\PYG{p}{:} \PYG{n}{Maximum} \PYG{n}{allowed} \PYG{n}{size} \PYG{n}{exceeded}
\end{sphinxVerbatim}

plot residuals

\fvset{hllines={, ,}}%
\begin{sphinxVerbatim}[commandchars=\\\{\}]
\PYG{c+c1}{\PYGZsh{}}
\PYG{c+c1}{\PYGZsh{} plot the residuals ICDF}
\PYG{c+c1}{\PYGZsh{}}


\PYG{c+c1}{\PYGZsh{} recall that the residuals should be multiplied by \PYGZhy{}1}
\PYG{n}{fname} \PYG{o}{=} \PYG{l+s+s1}{\PYGZsq{}}\PYG{l+s+se}{\PYGZbs{}\PYGZbs{}}\PYG{l+s+s1}{res\PYGZus{}inv\PYGZus{}}\PYG{l+s+s1}{\PYGZsq{}} \PYG{o}{+} \PYG{n}{fname\PYGZus{}keys} \PYG{o}{+} \PYG{l+s+s1}{\PYGZsq{}}\PYG{l+s+s1}{.png}\PYG{l+s+s1}{\PYGZsq{}}
\PYG{n}{fnames\PYGZus{}load} \PYG{o}{=} \PYG{p}{(}\PYG{l+s+s1}{\PYGZsq{}}\PYG{l+s+se}{\PYGZbs{}\PYGZbs{}}\PYG{l+s+s1}{res\PYGZus{}inv\PYGZus{}start.pkl}\PYG{l+s+s1}{\PYGZsq{}}\PYG{p}{,} \PYG{l+s+s1}{\PYGZsq{}}\PYG{l+s+se}{\PYGZbs{}\PYGZbs{}}\PYG{l+s+s1}{res\PYGZus{}inv\PYGZus{}end.pkl}\PYG{l+s+s1}{\PYGZsq{}}\PYG{p}{,} \PYG{l+s+s1}{\PYGZsq{}}\PYG{l+s+se}{\PYGZbs{}\PYGZbs{}}\PYG{l+s+s1}{res\PYGZus{}inv\PYGZus{}dt.pkl}\PYG{l+s+s1}{\PYGZsq{}}\PYG{p}{)}

\PYG{n}{data\PYGZus{}list\PYGZus{}all}\PYG{p}{,} \PYG{n}{fname\PYGZus{}subplot} \PYG{o}{=} \PYG{n}{plotter}\PYG{o}{.}\PYG{n}{get\PYGZus{}figure\PYGZus{}data}\PYG{p}{(}\PYG{n}{fpaths}\PYG{p}{,} \PYG{n}{fpath\PYGZus{}figure\PYGZus{}save}\PYG{p}{,} \PYG{n}{fname}\PYG{p}{,} \PYG{n}{fnames\PYGZus{}load}\PYG{o}{=}\PYG{n}{fnames\PYGZus{}load}\PYG{p}{)}
\PYG{c+c1}{\PYGZsh{}}
\PYG{c+c1}{\PYGZsh{} plotting parameters}
\PYG{c+c1}{\PYGZsh{}}

\PYG{c+c1}{\PYGZsh{} residual plot (inverse CDF)}
\PYG{n}{do\PYGZus{}cdf} \PYG{o}{=} \PYG{k+kc}{False}
\PYG{n}{legend} \PYG{o}{=} \PYG{p}{[}\PYG{l+s+s1}{\PYGZsq{}}\PYG{l+s+s1}{Residual}\PYG{l+s+s1}{\PYGZsq{}}\PYG{p}{]}
\PYG{n}{colors} \PYG{o}{=} \PYG{p}{[}\PYG{l+s+s1}{\PYGZsq{}}\PYG{l+s+s1}{Red}\PYG{l+s+s1}{\PYGZsq{}}\PYG{p}{]}

\PYG{n}{xunits} \PYG{o}{=} \PYG{l+s+s1}{\PYGZsq{}}\PYG{l+s+s1}{Quantile}\PYG{l+s+s1}{\PYGZsq{}}
\PYG{n}{yunits} \PYG{o}{=} \PYG{p}{[}\PYG{l+s+s1}{\PYGZsq{}}\PYG{l+s+s1}{Hours}\PYG{l+s+s1}{\PYGZsq{}}\PYG{p}{,} \PYG{l+s+s1}{\PYGZsq{}}\PYG{l+s+s1}{Hours}\PYG{l+s+s1}{\PYGZsq{}}\PYG{p}{,} \PYG{l+s+s1}{\PYGZsq{}}\PYG{l+s+s1}{Minutes}\PYG{l+s+s1}{\PYGZsq{}}\PYG{p}{]}

\PYG{n}{main\PYGZus{}title} \PYG{o}{=} \PYG{l+s+s1}{\PYGZsq{}}\PYG{l+s+s1}{Residual of the Inverse CDF}\PYG{l+s+s1}{\PYGZsq{}}

\PYG{n}{xlabels} \PYG{o}{=} \PYG{p}{[}\PYG{n}{xunits}\PYG{p}{]} \PYG{o}{*} \PYG{n+nb}{len}\PYG{p}{(}\PYG{n}{keys}\PYG{p}{)}

\PYG{c+c1}{\PYGZsh{}}
\PYG{c+c1}{\PYGZsh{} plot the data}
\PYG{c+c1}{\PYGZsh{}}
\PYG{n}{plot\PYGZus{}subplots}\PYG{p}{(}\PYG{n}{data\PYGZus{}list}\PYG{o}{=}\PYG{n}{data\PYGZus{}list\PYGZus{}all}\PYG{p}{,} \PYG{n}{do\PYGZus{}cdf}\PYG{o}{=}\PYG{n}{do\PYGZus{}cdf}\PYG{p}{,} \PYG{n}{main\PYGZus{}title}\PYG{o}{=}\PYG{n}{main\PYGZus{}title}\PYG{p}{,} \PYG{n}{legend}\PYG{o}{=}\PYG{n}{legend}\PYG{p}{,} \PYGZbs{}
                  \PYG{n}{xlabels}\PYG{o}{=}\PYG{n}{xlabels}\PYG{p}{,} \PYG{n}{ylabels}\PYG{o}{=}\PYG{n}{ylabels}\PYG{p}{,} \PYG{n}{xunits}\PYG{o}{=}\PYG{n}{xunits}\PYG{p}{,} \PYG{n}{yunits}\PYG{o}{=}\PYG{n}{yunits}\PYG{p}{,} \PYG{n}{colors}\PYG{o}{=}\PYG{n}{colors}\PYG{p}{,} \PYGZbs{}
                  \PYG{n}{do\PYGZus{}save}\PYG{o}{=}\PYG{n}{do\PYGZus{}save\PYGZus{}fig}\PYG{p}{,} \PYG{n}{fname}\PYG{o}{=}\PYG{n}{fname\PYGZus{}subplot}\PYG{p}{,} \PYG{n}{linewidth}\PYG{o}{=}\PYG{n}{linewidth}\PYG{p}{)}

\PYG{k}{if} \PYG{n}{do\PYGZus{}show}\PYG{p}{:}
    \PYG{n}{plt}\PYG{o}{.}\PYG{n}{show}\PYG{p}{(}\PYG{p}{)}
\PYG{k}{else}\PYG{p}{:}
    \PYG{n}{plt}\PYG{o}{.}\PYG{n}{close}\PYG{p}{(}\PYG{p}{)}
\end{sphinxVerbatim}
\begin{sphinxalltt}
C:UsersnbrandonAppDataLocalContinuumAnaconda3libsite-packagesmatplotliblegend.py:338: UserWarning: Automatic legend placement (loc="best") not implemented for figure legend. Falling back on "upper right".
  warnings.warn('Automatic legend placement (loc="best") not '
\end{sphinxalltt}

plot the scaled residuals

\fvset{hllines={, ,}}%
\begin{sphinxVerbatim}[commandchars=\\\{\}]
\PYG{c+c1}{\PYGZsh{}}
\PYG{c+c1}{\PYGZsh{} plot the residuals ICDF scaled}
\PYG{c+c1}{\PYGZsh{}}

\PYG{c+c1}{\PYGZsh{} recall that the residuals should be multiplied by \PYGZhy{}1}
\PYG{n}{fnames} \PYG{o}{=} \PYG{l+s+s1}{\PYGZsq{}}\PYG{l+s+se}{\PYGZbs{}\PYGZbs{}}\PYG{l+s+s1}{res\PYGZus{}inv\PYGZus{}scaled}\PYG{l+s+s1}{\PYGZsq{}} \PYG{o}{+} \PYG{n}{fname\PYGZus{}keys} \PYG{o}{+} \PYG{l+s+s1}{\PYGZsq{}}\PYG{l+s+s1}{.png}\PYG{l+s+s1}{\PYGZsq{}}

\PYG{n}{fnames\PYGZus{}load} \PYG{o}{=} \PYG{p}{(}\PYG{l+s+s1}{\PYGZsq{}}\PYG{l+s+se}{\PYGZbs{}\PYGZbs{}}\PYG{l+s+s1}{res\PYGZus{}inv\PYGZus{}scaled\PYGZus{}start.pkl}\PYG{l+s+s1}{\PYGZsq{}}\PYG{p}{,} \PYG{l+s+s1}{\PYGZsq{}}\PYG{l+s+se}{\PYGZbs{}\PYGZbs{}}\PYG{l+s+s1}{res\PYGZus{}inv\PYGZus{}scaled\PYGZus{}end.pkl}\PYG{l+s+s1}{\PYGZsq{}}\PYG{p}{,} \PYGZbs{}
               \PYG{l+s+s1}{\PYGZsq{}}\PYG{l+s+se}{\PYGZbs{}\PYGZbs{}}\PYG{l+s+s1}{res\PYGZus{}inv\PYGZus{}scaled\PYGZus{}dt.pkl}\PYG{l+s+s1}{\PYGZsq{}}\PYG{p}{)}

\PYG{n}{data\PYGZus{}list\PYGZus{}all}\PYG{p}{,} \PYG{n}{fname\PYGZus{}subplot} \PYG{o}{=} \PYG{n}{plotter}\PYG{o}{.}\PYG{n}{get\PYGZus{}figure\PYGZus{}data}\PYG{p}{(}\PYG{n}{fpaths}\PYG{p}{,} \PYG{n}{fpath\PYGZus{}figure\PYGZus{}save}\PYG{p}{,} \PYG{n}{fname}\PYG{p}{,} \PYG{n}{fnames\PYGZus{}load}\PYG{o}{=}\PYG{n}{fnames\PYGZus{}load}\PYG{p}{)}

\PYG{c+c1}{\PYGZsh{}}
\PYG{c+c1}{\PYGZsh{} plotting parameters}
\PYG{c+c1}{\PYGZsh{}Q}
\PYG{n}{do\PYGZus{}cdf} \PYG{o}{=} \PYG{k+kc}{False}

\PYG{n}{legend} \PYG{o}{=} \PYG{p}{[}\PYG{l+s+s1}{\PYGZsq{}}\PYG{l+s+s1}{Residual}\PYG{l+s+s1}{\PYGZsq{}}\PYG{p}{]}
\PYG{n}{colors} \PYG{o}{=} \PYG{p}{[}\PYG{l+s+s1}{\PYGZsq{}}\PYG{l+s+s1}{Red}\PYG{l+s+s1}{\PYGZsq{}}\PYG{p}{]}
\PYG{n}{xunits} \PYG{o}{=} \PYG{l+s+s1}{\PYGZsq{}}\PYG{l+s+s1}{Quantitle}\PYG{l+s+s1}{\PYGZsq{}}
\PYG{n}{yunits} \PYG{o}{=} \PYG{p}{[}\PYG{l+s+s1}{\PYGZsq{}}\PYG{l+s+s1}{Standard Deviations}\PYG{l+s+s1}{\PYGZsq{}}\PYG{p}{]} \PYG{o}{*} \PYG{l+m+mi}{3}

\PYG{n}{main\PYGZus{}title} \PYG{o}{=} \PYG{l+s+s1}{\PYGZsq{}}\PYG{l+s+s1}{Scaled Residual of the Quantile Functions}\PYG{l+s+s1}{\PYGZsq{}}

\PYG{n}{xlabels} \PYG{o}{=} \PYG{p}{[}\PYG{n}{xunits}\PYG{p}{]} \PYG{o}{*} \PYG{n+nb}{len}\PYG{p}{(}\PYG{n}{keys}\PYG{p}{)}

\PYG{c+c1}{\PYGZsh{}}
\PYG{c+c1}{\PYGZsh{} plot the data}
\PYG{c+c1}{\PYGZsh{}}

\PYG{n}{plot\PYGZus{}subplots}\PYG{p}{(}\PYG{n}{data\PYGZus{}list}\PYG{o}{=}\PYG{n}{data\PYGZus{}list\PYGZus{}all}\PYG{p}{,} \PYG{n}{do\PYGZus{}cdf}\PYG{o}{=}\PYG{n}{do\PYGZus{}cdf}\PYG{p}{,} \PYG{n}{main\PYGZus{}title}\PYG{o}{=}\PYG{n}{main\PYGZus{}title}\PYG{p}{,} \PYG{n}{legend}\PYG{o}{=}\PYG{n}{legend}\PYG{p}{,} \PYGZbs{}
                  \PYG{n}{xlabels}\PYG{o}{=}\PYG{n}{xlabels}\PYG{p}{,} \PYG{n}{ylabels}\PYG{o}{=}\PYG{n}{ylabels}\PYG{p}{,} \PYG{n}{xunits}\PYG{o}{=}\PYG{n}{xunits}\PYG{p}{,} \PYG{n}{yunits}\PYG{o}{=}\PYG{n}{yunits}\PYG{p}{,} \PYG{n}{colors}\PYG{o}{=}\PYG{n}{colors}\PYG{p}{,} \PYGZbs{}
                  \PYG{n}{do\PYGZus{}save}\PYG{o}{=}\PYG{n}{do\PYGZus{}save\PYGZus{}fig}\PYG{p}{,} \PYG{n}{fname}\PYG{o}{=}\PYG{n}{fname\PYGZus{}subplot}\PYG{p}{,} \PYG{n}{linewidth}\PYG{o}{=}\PYG{n}{linewidth}\PYG{p}{)}

\PYG{k}{if} \PYG{n}{do\PYGZus{}show}\PYG{p}{:}
    \PYG{n}{plt}\PYG{o}{.}\PYG{n}{show}\PYG{p}{(}\PYG{p}{)}
\PYG{k}{else}\PYG{p}{:}
    \PYG{n}{plt}\PYG{o}{.}\PYG{n}{close}\PYG{p}{(}\PYG{p}{)}
\end{sphinxVerbatim}
\begin{sphinxalltt}
C:UsersnbrandonAppDataLocalContinuumAnaconda3libsite-packagesmatplotliblegend.py:338: UserWarning: Automatic legend placement (loc="best") not implemented for figure legend. Falling back on "upper right".
  warnings.warn('Automatic legend placement (loc="best") not '
\end{sphinxalltt}


\subsection{figure\_loader\_with\_without\_variation notebook}
\label{\detokenize{figure_loader_with_without_variation::doc}}\label{\detokenize{figure_loader_with_without_variation:figure-loader-with-without-variation-notebook}}
\fvset{hllines={, ,}}%
\begin{sphinxVerbatim}[commandchars=\\\{\}]
\PYG{c+c1}{\PYGZsh{} The United States Environmental Protection Agency through its Office of}
\PYG{c+c1}{\PYGZsh{} Research and Development has developed this software. The code is made}
\PYG{c+c1}{\PYGZsh{} publicly available to better communicate the research. All input data}
\PYG{c+c1}{\PYGZsh{} used fora given application should be reviewed by the researcher so}
\PYG{c+c1}{\PYGZsh{} that the model results are based on appropriate data for any given}
\PYG{c+c1}{\PYGZsh{} application. This model is under continued development. The model and}
\PYG{c+c1}{\PYGZsh{} data included herein do not represent and should not be construed to}
\PYG{c+c1}{\PYGZsh{} represent any Agency determination or policy.}
\PYG{c+c1}{\PYGZsh{}}
\PYG{c+c1}{\PYGZsh{} This file was written by Dr. Namdi Brandon}
\PYG{c+c1}{\PYGZsh{} ORCID: 0000\PYGZhy{}0001\PYGZhy{}7050\PYGZhy{}1538}
\PYG{c+c1}{\PYGZsh{} March 20, 2018}
\end{sphinxVerbatim}

This notebook loads the individual data about the cumuluative
distribution functions (CDFs) comaparing the Agent-Based Model of Human
Activity Patterns (ABMHAP) results to the Consolidated Human Activity
Database (CHAD) data. The plots compare the distribution
activity-parameter data from ABMHAP to CHAD. More specifically, the we
compare the ABMHAP with intra-individual variation, ABMHAP without
intra-individual variation, and CHAD single-day data.

This module loads and plots a figure with the following:
\begin{enumerate}
\item {} 
CDFs of ABMHAP with intra-individual variation vs. ABMHAP without
intra-individual variation vs. CHAD longitudinal data for
activity-parameters

\end{enumerate}

Import

\fvset{hllines={, ,}}%
\begin{sphinxVerbatim}[commandchars=\\\{\}]
\PYG{k+kn}{import} \PYG{n+nn}{sys}
\PYG{n}{sys}\PYG{o}{.}\PYG{n}{path}\PYG{o}{.}\PYG{n}{append}\PYG{p}{(}\PYG{l+s+s1}{\PYGZsq{}}\PYG{l+s+s1}{..}\PYG{l+s+se}{\PYGZbs{}\PYGZbs{}}\PYG{l+s+s1}{source}\PYG{l+s+s1}{\PYGZsq{}}\PYG{p}{)}
\PYG{n}{sys}\PYG{o}{.}\PYG{n}{path}\PYG{o}{.}\PYG{n}{append}\PYG{p}{(}\PYG{l+s+s1}{\PYGZsq{}}\PYG{l+s+s1}{..}\PYG{l+s+se}{\PYGZbs{}\PYGZbs{}}\PYG{l+s+s1}{processing}\PYG{l+s+s1}{\PYGZsq{}}\PYG{p}{)}
\PYG{n}{sys}\PYG{o}{.}\PYG{n}{path}\PYG{o}{.}\PYG{n}{append}\PYG{p}{(}\PYG{l+s+s1}{\PYGZsq{}}\PYG{l+s+s1}{..}\PYG{l+s+se}{\PYGZbs{}\PYGZbs{}}\PYG{l+s+s1}{plotting}\PYG{l+s+s1}{\PYGZsq{}}\PYG{p}{)}

\PYG{c+c1}{\PYGZsh{} plotting capability}
\PYG{k+kn}{import} \PYG{n+nn}{matplotlib}\PYG{n+nn}{.}\PYG{n+nn}{pylab} \PYG{k}{as} \PYG{n+nn}{plt}
\PYG{k+kn}{import} \PYG{n+nn}{matplotlib}\PYG{n+nn}{.}\PYG{n+nn}{ticker} \PYG{k}{as} \PYG{n+nn}{ticker}

\PYG{c+c1}{\PYGZsh{} math capability}
\PYG{k+kn}{import} \PYG{n+nn}{numpy} \PYG{k}{as} \PYG{n+nn}{np}

\PYG{c+c1}{\PYGZsh{} data frame capability}
\PYG{k+kn}{import} \PYG{n+nn}{pandas} \PYG{k}{as} \PYG{n+nn}{pd}

\PYG{c+c1}{\PYGZsh{} pickling capability}
\PYG{k+kn}{import} \PYG{n+nn}{pickle}

\PYG{c+c1}{\PYGZsh{} ABMHAP modules}
\PYG{k+kn}{import} \PYG{n+nn}{my\PYGZus{}globals} \PYG{k}{as} \PYG{n+nn}{mg}
\PYG{k+kn}{import} \PYG{n+nn}{demography} \PYG{k}{as} \PYG{n+nn}{dmg}
\PYG{k+kn}{import} \PYG{n+nn}{activity}\PYG{o}{,} \PYG{n+nn}{analyzer}\PYG{o}{,} \PYG{n+nn}{plotter}\PYG{o}{,} \PYG{n+nn}{temporal}

\PYG{k+kn}{import} \PYG{n+nn}{chad\PYGZus{}demography\PYGZus{}adult\PYGZus{}work} \PYG{k}{as} \PYG{n+nn}{cdaw}
\PYG{k+kn}{import} \PYG{n+nn}{chad\PYGZus{}demography\PYGZus{}adult\PYGZus{}non\PYGZus{}work} \PYG{k}{as} \PYG{n+nn}{cdanw}
\PYG{k+kn}{import} \PYG{n+nn}{chad\PYGZus{}demography\PYGZus{}child\PYGZus{}school} \PYG{k}{as} \PYG{n+nn}{cdcs}
\PYG{k+kn}{import} \PYG{n+nn}{chad\PYGZus{}demography\PYGZus{}child\PYGZus{}young} \PYG{k}{as} \PYG{n+nn}{cdcy}
\end{sphinxVerbatim}

\fvset{hllines={, ,}}%
\begin{sphinxVerbatim}[commandchars=\\\{\}]
\PYG{o}{\PYGZpc{}}\PYG{k}{matplotlib} auto
\end{sphinxVerbatim}

\fvset{hllines={, ,}}%
\begin{sphinxVerbatim}[commandchars=\\\{\}]
\PYG{n}{Using} \PYG{n}{matplotlib} \PYG{n}{backend}\PYG{p}{:} \PYG{n}{Qt5Agg}
\end{sphinxVerbatim}

define functions

\fvset{hllines={, ,}}%
\begin{sphinxVerbatim}[commandchars=\\\{\}]
\PYG{c+c1}{\PYGZsh{} plot subplots}

\PYG{k}{def} \PYG{n+nf}{plot\PYGZus{}subplots}\PYG{p}{(}\PYG{n}{data\PYGZus{}list1}\PYG{p}{,} \PYG{n}{data\PYGZus{}list2}\PYG{p}{,} \PYG{n}{data\PYGZus{}list3}\PYG{p}{,} \PYG{n}{do\PYGZus{}cdf}\PYG{p}{,} \PYG{n}{main\PYGZus{}title}\PYG{p}{,} \PYG{n}{legend}\PYG{p}{,} \PYG{n}{xlabels}\PYG{p}{,} \PYG{n}{ylabels}\PYG{p}{,} \PYGZbs{}
                       \PYG{n}{xunits}\PYG{p}{,} \PYG{n}{yunits}\PYG{p}{,} \PYG{n}{colors}\PYG{p}{,} \PYG{n}{do\PYGZus{}save}\PYG{o}{=}\PYG{k+kc}{False}\PYG{p}{,} \PYG{n}{fname}\PYG{o}{=}\PYG{k+kc}{None}\PYG{p}{,} \PYG{n}{linewidth}\PYG{o}{=}\PYG{l+m+mi}{1}\PYG{p}{)}\PYG{p}{:}

    \PYG{c+c1}{\PYGZsh{} the dimensions of a maximized figure. Base x Height [pixels]}
    \PYG{n}{b\PYGZus{}pixels}\PYG{p}{,} \PYG{n}{h\PYGZus{}pixels} \PYG{o}{=} \PYG{l+m+mi}{2400}\PYG{p}{,} \PYG{l+m+mi}{1255}
    \PYG{n}{my\PYGZus{}dpi} \PYG{o}{=} \PYG{l+m+mi}{800}

    \PYG{n}{b\PYGZus{}in} \PYG{o}{=} \PYG{n}{b\PYGZus{}pixels}\PYG{o}{/}\PYG{n}{my\PYGZus{}dpi}
    \PYG{n}{h\PYGZus{}in} \PYG{o}{=} \PYG{n}{h\PYGZus{}pixels}\PYG{o}{/}\PYG{n}{my\PYGZus{}dpi}


    \PYG{c+c1}{\PYGZsh{} set the figure size for saving to custom if savinig}
    \PYG{k}{if} \PYG{n}{do\PYGZus{}save}\PYG{p}{:}
        \PYG{n}{figsize}\PYG{p}{,} \PYG{n}{dpi} \PYG{o}{=} \PYG{p}{(}\PYG{n}{b\PYGZus{}in}\PYG{p}{,} \PYG{n}{h\PYGZus{}in}\PYG{p}{)}\PYG{p}{,} \PYG{n}{my\PYGZus{}dpi}
    \PYG{k}{else}\PYG{p}{:}
        \PYG{n}{figsize}\PYG{p}{,} \PYG{n}{dpi} \PYG{o}{=} \PYG{k+kc}{None}\PYG{p}{,} \PYG{k+kc}{None}

    \PYG{c+c1}{\PYGZsh{} data\PYGZus{}list is}
    \PYG{n}{nrows}\PYG{p}{,} \PYG{n}{ncols} \PYG{o}{=} \PYG{l+m+mi}{3}\PYG{p}{,} \PYG{n+nb}{len}\PYG{p}{(}\PYG{n}{data\PYGZus{}list1}\PYG{p}{[}\PYG{l+m+mi}{0}\PYG{p}{]}\PYG{p}{)}

    \PYG{k}{if} \PYG{n}{do\PYGZus{}cdf}\PYG{p}{:}
        \PYG{n}{f}\PYG{p}{,} \PYG{n}{axes} \PYG{o}{=} \PYG{n}{plt}\PYG{o}{.}\PYG{n}{subplots}\PYG{p}{(}\PYG{n}{nrows}\PYG{p}{,} \PYG{n}{ncols}\PYG{p}{,} \PYG{n}{sharey}\PYG{o}{=}\PYG{k+kc}{True}\PYG{p}{,} \PYG{n}{figsize}\PYG{o}{=}\PYG{n}{figsize}\PYG{p}{,} \PYG{n}{dpi}\PYG{o}{=}\PYG{n}{dpi}\PYG{p}{)}
    \PYG{k}{else}\PYG{p}{:}
        \PYG{n}{f}\PYG{p}{,} \PYG{n}{axes} \PYG{o}{=} \PYG{n}{plt}\PYG{o}{.}\PYG{n}{subplots}\PYG{p}{(}\PYG{n}{nrows}\PYG{p}{,} \PYG{n}{ncols}\PYG{p}{,} \PYG{n}{sharex}\PYG{o}{=}\PYG{k+kc}{True}\PYG{p}{,} \PYG{n}{figsize}\PYG{o}{=}\PYG{n}{figsize}\PYG{p}{,} \PYG{n}{dpi}\PYG{o}{=}\PYG{n}{dpi}\PYG{p}{)}


    \PYG{c+c1}{\PYGZsh{}}
    \PYG{c+c1}{\PYGZsh{} plot}
    \PYG{c+c1}{\PYGZsh{}}
    \PYG{n}{alpha} \PYG{o}{=} \PYG{l+m+mf}{0.7}
    \PYG{k}{for} \PYG{n}{i} \PYG{p}{,} \PYG{n}{ax} \PYG{o+ow}{in} \PYG{n+nb}{enumerate}\PYG{p}{(}\PYG{n}{f}\PYG{o}{.}\PYG{n}{axes}\PYG{p}{)}\PYG{p}{:}

        \PYG{c+c1}{\PYGZsh{} indices}
        \PYG{n}{irow} \PYG{o}{=} \PYG{n}{i} \PYG{o}{/}\PYG{o}{/} \PYG{n}{ncols}
        \PYG{n}{jcol} \PYG{o}{=} \PYG{n}{i} \PYG{o}{\PYGZpc{}} \PYG{n}{ncols}

        \PYG{c+c1}{\PYGZsh{} plot data}
        \PYG{n}{temp1} \PYG{o}{=} \PYG{n}{data\PYGZus{}list1}\PYG{p}{[}\PYG{n}{irow}\PYG{p}{]}\PYG{p}{[}\PYG{n}{jcol}\PYG{p}{]}
        \PYG{n}{temp2} \PYG{o}{=} \PYG{n}{data\PYGZus{}list2}\PYG{p}{[}\PYG{n}{irow}\PYG{p}{]}\PYG{p}{[}\PYG{n}{jcol}\PYG{p}{]}
        \PYG{n}{temp3} \PYG{o}{=} \PYG{n}{data\PYGZus{}list3}\PYG{p}{[}\PYG{n}{irow}\PYG{p}{]}\PYG{p}{[}\PYG{n}{jcol}\PYG{p}{]}

        \PYG{n}{counter} \PYG{o}{=} \PYG{l+m+mi}{0}

        \PYG{c+c1}{\PYGZsh{} ii for testing if}
        \PYG{n}{ii} \PYG{o}{=} \PYG{l+m+mi}{0}

        \PYG{k}{for} \PYG{n}{t1}\PYG{p}{,} \PYG{n}{t2}\PYG{p}{,} \PYG{n}{color} \PYG{o+ow}{in} \PYG{n+nb}{zip}\PYG{p}{(}\PYG{n}{temp1}\PYG{p}{,} \PYG{n}{temp2}\PYG{p}{,} \PYG{n}{colors}\PYG{p}{)}\PYG{p}{:}

            \PYG{k}{if} \PYG{n}{ii} \PYG{o}{==} \PYG{l+m+mi}{0}\PYG{p}{:}
                \PYG{n}{x\PYGZus{}data1}\PYG{p}{,} \PYG{n}{y\PYGZus{}data1} \PYG{o}{=} \PYG{n}{t1}
                \PYG{n}{x\PYGZus{}data2}\PYG{p}{,} \PYG{n}{y\PYGZus{}data2} \PYG{o}{=} \PYG{n}{t2}

                \PYG{k}{if} \PYG{n}{counter} \PYG{o}{==} \PYG{l+m+mi}{0}\PYG{p}{:}
                    \PYG{n}{c1} \PYG{o}{=} \PYG{l+s+s1}{\PYGZsq{}}\PYG{l+s+s1}{blue}\PYG{l+s+s1}{\PYGZsq{}}
                    \PYG{n}{c2} \PYG{o}{=} \PYG{l+s+s1}{\PYGZsq{}}\PYG{l+s+s1}{k}\PYG{l+s+s1}{\PYGZsq{}}
                    \PYG{c+c1}{\PYGZsh{}c2 = \PYGZsq{}green\PYGZsq{}}
                \PYG{k}{else}\PYG{p}{:}
                    \PYG{n}{c1} \PYG{o}{=} \PYG{l+s+s1}{\PYGZsq{}}\PYG{l+s+s1}{red}\PYG{l+s+s1}{\PYGZsq{}}
                    \PYG{n}{c2} \PYG{o}{=} \PYG{l+s+s1}{\PYGZsq{}}\PYG{l+s+s1}{red}\PYG{l+s+s1}{\PYGZsq{}}

                \PYG{k}{if} \PYG{n}{do\PYGZus{}cdf} \PYG{o+ow}{and} \PYG{n}{irow} \PYG{o}{==} \PYG{l+m+mi}{2}\PYG{p}{:}
                    \PYG{n}{idx} \PYG{o}{=} \PYG{n}{x\PYGZus{}data1} \PYG{o}{\PYGZgt{}}\PYG{o}{=} \PYG{l+m+mi}{0}

                    \PYG{n}{ax}\PYG{o}{.}\PYG{n}{plot}\PYG{p}{(}\PYG{n}{x\PYGZus{}data1}\PYG{p}{[}\PYG{n}{idx}\PYG{p}{]}\PYG{p}{,} \PYG{n}{y\PYGZus{}data1}\PYG{p}{[}\PYG{n}{idx}\PYG{p}{]}\PYG{p}{,} \PYG{n}{color}\PYG{o}{=}\PYG{n}{c1}\PYG{p}{,} \PYG{n}{linewidth}\PYG{o}{=}\PYG{n}{linewidth}\PYG{p}{,} \PYG{n}{alpha}\PYG{o}{=}\PYG{n}{alpha}\PYG{p}{)}
                    \PYG{n}{ax}\PYG{o}{.}\PYG{n}{plot}\PYG{p}{(}\PYG{n}{x\PYGZus{}data2}\PYG{p}{[}\PYG{n}{idx}\PYG{p}{]}\PYG{p}{,} \PYG{n}{y\PYGZus{}data2}\PYG{p}{[}\PYG{n}{idx}\PYG{p}{]}\PYG{p}{,} \PYG{n}{color}\PYG{o}{=}\PYG{n}{c2}\PYG{p}{,} \PYG{n}{ls}\PYG{o}{=}\PYG{l+s+s1}{\PYGZsq{}}\PYG{l+s+s1}{\PYGZhy{}\PYGZhy{}}\PYG{l+s+s1}{\PYGZsq{}}\PYG{p}{,} \PYG{n}{linewidth}\PYG{o}{=}\PYG{n}{linewidth}\PYG{p}{,} \PYG{n}{alpha}\PYG{o}{=}\PYG{n}{alpha}\PYG{p}{)}
                \PYG{k}{else}\PYG{p}{:}
                    \PYG{n}{ax}\PYG{o}{.}\PYG{n}{plot}\PYG{p}{(}\PYG{n}{x\PYGZus{}data1}\PYG{p}{,} \PYG{n}{y\PYGZus{}data1}\PYG{p}{,} \PYG{n}{color}\PYG{o}{=}\PYG{n}{c1}\PYG{p}{,} \PYG{n}{linewidth}\PYG{o}{=}\PYG{n}{linewidth}\PYG{p}{,} \PYG{n}{alpha}\PYG{o}{=}\PYG{n}{alpha}\PYG{p}{)}
                    \PYG{n}{ax}\PYG{o}{.}\PYG{n}{plot}\PYG{p}{(}\PYG{n}{x\PYGZus{}data2}\PYG{p}{,} \PYG{n}{y\PYGZus{}data2}\PYG{p}{,} \PYG{n}{color}\PYG{o}{=}\PYG{n}{c2}\PYG{p}{,} \PYG{n}{ls}\PYG{o}{=}\PYG{l+s+s1}{\PYGZsq{}}\PYG{l+s+s1}{\PYGZhy{}\PYGZhy{}}\PYG{l+s+s1}{\PYGZsq{}}\PYG{p}{,} \PYG{n}{linewidth}\PYG{o}{=}\PYG{n}{linewidth}\PYG{p}{,} \PYG{n}{alpha}\PYG{o}{=}\PYG{n}{alpha}\PYG{p}{)}

                \PYG{c+c1}{\PYGZsh{} access the CHAD data}
                \PYG{n}{x\PYGZus{}data3}\PYG{p}{,} \PYG{n}{y\PYGZus{}data3} \PYG{o}{=} \PYG{n}{temp3}\PYG{p}{[}\PYG{l+m+mi}{1}\PYG{p}{]}

                \PYG{k}{if} \PYG{p}{(}\PYG{n}{irow} \PYG{o+ow}{in} \PYG{p}{[}\PYG{l+m+mi}{0}\PYG{p}{,} \PYG{l+m+mi}{1}\PYG{p}{]}\PYG{p}{)} \PYG{o+ow}{and} \PYG{n}{jcol} \PYG{o+ow}{in} \PYG{p}{[}\PYG{l+m+mi}{1}\PYG{p}{,} \PYG{l+m+mi}{4}\PYG{p}{]}\PYG{p}{:}
                    \PYG{n}{x\PYGZus{}data3} \PYG{o}{=} \PYG{n}{mg}\PYG{o}{.}\PYG{n}{from\PYGZus{}periodic}\PYG{p}{(}\PYG{n}{x\PYGZus{}data3}\PYG{p}{,} \PYG{n}{do\PYGZus{}hours}\PYG{o}{=}\PYG{k+kc}{True}\PYG{p}{)}

                \PYG{n}{ax}\PYG{o}{.}\PYG{n}{plot}\PYG{p}{(}\PYG{n}{x\PYGZus{}data3}\PYG{p}{,} \PYG{n}{y\PYGZus{}data3}\PYG{p}{,} \PYG{n}{color}\PYG{o}{=}\PYG{l+s+s1}{\PYGZsq{}}\PYG{l+s+s1}{r}\PYG{l+s+s1}{\PYGZsq{}}\PYG{p}{,} \PYG{n}{linewidth}\PYG{o}{=}\PYG{n}{linewidth}\PYG{p}{,} \PYG{n}{alpha}\PYG{o}{=}\PYG{n}{alpha}\PYG{p}{)}

                \PYG{n}{counter} \PYG{o}{=} \PYG{n}{counter} \PYG{o}{+} \PYG{l+m+mi}{1}
                \PYG{n}{ii} \PYG{o}{=} \PYG{n}{ii} \PYG{o}{+} \PYG{l+m+mi}{1}
            \PYG{c+c1}{\PYGZsh{}}
            \PYG{c+c1}{\PYGZsh{} set the tick labels}
            \PYG{c+c1}{\PYGZsh{}}
            \PYG{n}{ticksize}\PYG{o}{=}\PYG{l+m+mi}{14}
            \PYG{n}{ax}\PYG{o}{.}\PYG{n}{tick\PYGZus{}params}\PYG{p}{(}\PYG{n}{axis}\PYG{o}{=}\PYG{l+s+s1}{\PYGZsq{}}\PYG{l+s+s1}{both}\PYG{l+s+s1}{\PYGZsq{}}\PYG{p}{,} \PYG{n}{labelsize}\PYG{o}{=}\PYG{n}{ticksize}\PYG{p}{)}

            \PYG{k}{if} \PYG{n}{irow} \PYG{o}{==} \PYG{l+m+mi}{2}\PYG{p}{:}
                \PYG{n}{ax}\PYG{o}{.}\PYG{n}{xaxis}\PYG{o}{.}\PYG{n}{set\PYGZus{}major\PYGZus{}locator}\PYG{p}{(}\PYG{n}{ticker}\PYG{o}{.}\PYG{n}{MaxNLocator}\PYG{p}{(}\PYG{n}{nbins}\PYG{o}{=}\PYG{l+m+mi}{5}\PYG{p}{)}\PYG{p}{)}

            \PYG{k}{if} \PYG{n}{do\PYGZus{}cdf} \PYG{o+ow}{and} \PYG{n}{irow} \PYG{o+ow}{in} \PYG{p}{[}\PYG{l+m+mi}{0}\PYG{p}{,} \PYG{l+m+mi}{1}\PYG{p}{]}\PYG{p}{:}
                \PYG{c+c1}{\PYGZsh{} limit the xaxis to integernumbers}
                \PYG{n}{x\PYGZus{}all} \PYG{o}{=} \PYG{p}{[}\PYG{n}{x}\PYG{o}{.}\PYG{n}{get\PYGZus{}xdata}\PYG{p}{(}\PYG{p}{)} \PYG{k}{for} \PYG{n}{x} \PYG{o+ow}{in} \PYG{n}{ax}\PYG{o}{.}\PYG{n}{lines}\PYG{p}{]}
                \PYG{n}{x\PYGZus{}all} \PYG{o}{=} \PYG{n}{np}\PYG{o}{.}\PYG{n}{hstack}\PYG{p}{(}\PYG{n}{x\PYGZus{}all}\PYG{p}{)}\PYG{o}{.}\PYG{n}{flatten}\PYG{p}{(}\PYG{p}{)}
                \PYG{n}{x\PYGZus{}min}\PYG{p}{,} \PYG{n}{x\PYGZus{}max} \PYG{o}{=} \PYG{n}{np}\PYG{o}{.}\PYG{n}{floor}\PYG{p}{(} \PYG{n}{np}\PYG{o}{.}\PYG{n}{min}\PYG{p}{(}\PYG{n}{x\PYGZus{}all}\PYG{p}{)} \PYG{p}{)}\PYG{p}{,} \PYG{n}{np}\PYG{o}{.}\PYG{n}{ceil}\PYG{p}{(} \PYG{n}{np}\PYG{o}{.}\PYG{n}{max}\PYG{p}{(}\PYG{n}{x\PYGZus{}all}\PYG{p}{)}\PYG{p}{)}
                \PYG{n}{dx} \PYG{o}{=} \PYG{n+nb}{abs}\PYG{p}{(}\PYG{n}{x\PYGZus{}min} \PYG{o}{\PYGZhy{}} \PYG{n}{x\PYGZus{}max}\PYG{p}{)} \PYG{o}{+} \PYG{l+m+mi}{1}
                \PYG{n}{nbins} \PYG{o}{=} \PYG{n}{np}\PYG{o}{.}\PYG{n}{ceil}\PYG{p}{(}\PYG{n}{dx}\PYG{o}{/}\PYG{l+m+mi}{2}\PYG{p}{)}
                \PYG{n}{ax}\PYG{o}{.}\PYG{n}{xaxis}\PYG{o}{.}\PYG{n}{set\PYGZus{}major\PYGZus{}locator}\PYG{p}{(}\PYG{n}{ticker}\PYG{o}{.}\PYG{n}{MaxNLocator}\PYG{p}{(}\PYG{n}{nbins}\PYG{p}{)}\PYG{p}{)}

                \PYG{n}{ax}\PYG{o}{.}\PYG{n}{set\PYGZus{}xlim}\PYG{p}{(}\PYG{n}{x\PYGZus{}min}\PYG{p}{,} \PYG{n}{x\PYGZus{}max}\PYG{p}{)}

                \PYG{c+c1}{\PYGZsh{} set the xticks}
                \PYG{c+c1}{\PYGZsh{} testing}
                \PYG{n}{x\PYGZus{}min} \PYG{o}{=} \PYG{n}{np}\PYG{o}{.}\PYG{n}{round}\PYG{p}{(}\PYG{n}{x\PYGZus{}min}\PYG{p}{)}\PYG{o}{.}\PYG{n}{astype}\PYG{p}{(}\PYG{n+nb}{int}\PYG{p}{)}
                \PYG{n}{x\PYGZus{}max} \PYG{o}{=} \PYG{n}{np}\PYG{o}{.}\PYG{n}{round}\PYG{p}{(}\PYG{n}{x\PYGZus{}max}\PYG{p}{)}\PYG{o}{.}\PYG{n}{astype}\PYG{p}{(}\PYG{n+nb}{int}\PYG{p}{)}
                \PYG{n}{dx} \PYG{o}{=} \PYG{p}{(}\PYG{n}{x\PYGZus{}max} \PYG{o}{\PYGZhy{}} \PYG{n}{x\PYGZus{}min}\PYG{p}{)} \PYG{o}{/} \PYG{p}{(}\PYG{l+m+mi}{5} \PYG{o}{\PYGZhy{}} \PYG{l+m+mi}{1}\PYG{p}{)}
                \PYG{n}{dx} \PYG{o}{=} \PYG{n}{np}\PYG{o}{.}\PYG{n}{floor}\PYG{p}{(}\PYG{n}{dx}\PYG{p}{)}\PYG{o}{.}\PYG{n}{astype}\PYG{p}{(}\PYG{n+nb}{int}\PYG{p}{)}
                \PYG{n}{xticks} \PYG{o}{=} \PYG{n}{np}\PYG{o}{.}\PYG{n}{arange}\PYG{p}{(}\PYG{n}{x\PYGZus{}min}\PYG{p}{,} \PYG{n}{x\PYGZus{}max}\PYG{p}{,} \PYG{n}{dx}\PYG{p}{)}
                \PYG{n}{ax}\PYG{o}{.}\PYG{n}{set\PYGZus{}xticks}\PYG{p}{(}\PYG{n}{xticks}\PYG{p}{)}


    \PYG{c+c1}{\PYGZsh{} main title}
    \PYG{n}{fontsize\PYGZus{}title} \PYG{o}{=} \PYG{l+m+mi}{18}
    \PYG{n}{f}\PYG{o}{.}\PYG{n}{suptitle}\PYG{p}{(}\PYG{n}{main\PYGZus{}title}\PYG{p}{,} \PYG{n}{fontsize}\PYG{o}{=}\PYG{n}{fontsize\PYGZus{}title}\PYG{p}{)}

    \PYG{c+c1}{\PYGZsh{} legend}
    \PYG{n}{f}\PYG{o}{.}\PYG{n}{legend}\PYG{p}{(} \PYG{n}{f}\PYG{o}{.}\PYG{n}{axes}\PYG{p}{[}\PYG{l+m+mi}{0}\PYG{p}{]}\PYG{o}{.}\PYG{n}{lines}\PYG{p}{,} \PYG{n}{legend}\PYG{p}{,} \PYG{l+s+s1}{\PYGZsq{}}\PYG{l+s+s1}{best}\PYG{l+s+s1}{\PYGZsq{}}\PYG{p}{)}

    \PYG{c+c1}{\PYGZsh{}}
    \PYG{c+c1}{\PYGZsh{} set the x\PYGZhy{}axis labels}
    \PYG{c+c1}{\PYGZsh{}}

    \PYG{n}{fontsize\PYGZus{}label} \PYG{o}{=} \PYG{l+m+mi}{18}
    \PYG{k}{for} \PYG{n}{ax}\PYG{p}{,} \PYG{n}{xlabel} \PYG{o+ow}{in} \PYG{n+nb}{zip}\PYG{p}{(} \PYG{n}{axes}\PYG{p}{[}\PYG{n}{nrows}\PYG{o}{\PYGZhy{}}\PYG{l+m+mi}{1}\PYG{p}{,}\PYG{p}{:}\PYG{p}{]}\PYG{p}{,} \PYG{n}{xlabels}\PYG{p}{)} \PYG{p}{:}
        \PYG{n}{ax}\PYG{o}{.}\PYG{n}{set\PYGZus{}xlabel}\PYG{p}{(}\PYG{n}{xlabel}\PYG{p}{,} \PYG{n}{fontsize}\PYG{o}{=}\PYG{n}{fontsize\PYGZus{}label}\PYG{p}{)}

        \PYG{k}{if} \PYG{o+ow}{not} \PYG{n}{do\PYGZus{}cdf}\PYG{p}{:}
            \PYG{n}{x\PYGZus{}min}\PYG{p}{,} \PYG{n}{x\PYGZus{}max} \PYG{o}{=} \PYG{l+m+mi}{0}\PYG{p}{,} \PYG{l+m+mi}{1}
            \PYG{n}{ax}\PYG{o}{.}\PYG{n}{set\PYGZus{}xlim}\PYG{p}{(}\PYG{n}{x\PYGZus{}min}\PYG{p}{,} \PYG{n}{x\PYGZus{}max}\PYG{p}{)}
            \PYG{n}{xticks} \PYG{o}{=} \PYG{n}{np}\PYG{o}{.}\PYG{n}{linspace}\PYG{p}{(}\PYG{n}{x\PYGZus{}min}\PYG{p}{,} \PYG{n}{x\PYGZus{}max}\PYG{p}{,} \PYG{l+m+mi}{3}\PYG{p}{)}
            \PYG{n}{ax}\PYG{o}{.}\PYG{n}{set\PYGZus{}xticks}\PYG{p}{(}\PYG{n}{xticks}\PYG{p}{)}
            \PYG{c+c1}{\PYGZsh{}\PYGZsh{}ax.set\PYGZus{}xticks(xticks, fontsize=20)}
            \PYG{c+c1}{\PYGZsh{}ax.set\PYGZus{}xticklabels(labels=[], fontsize=20)}

    \PYG{c+c1}{\PYGZsh{} set x titles}
    \PYG{k}{for} \PYG{n}{ax}\PYG{p}{,} \PYG{n}{key} \PYG{o+ow}{in} \PYG{n+nb}{zip}\PYG{p}{(}\PYG{n}{axes}\PYG{p}{[}\PYG{l+m+mi}{0}\PYG{p}{,}\PYG{p}{:}\PYG{p}{]}\PYG{p}{,} \PYG{n}{keys}\PYG{p}{)}\PYG{p}{:}
        \PYG{c+c1}{\PYGZsh{}ax.set\PYGZus{}title( activity.INT\PYGZus{}2\PYGZus{}STR[key], fontsize=fontsize\PYGZus{}title )}
        \PYG{n}{ax}\PYG{o}{.}\PYG{n}{set\PYGZus{}title}\PYG{p}{(} \PYG{n}{activity}\PYG{o}{.}\PYG{n}{INT\PYGZus{}2\PYGZus{}STR}\PYG{p}{[}\PYG{n}{key}\PYG{p}{]}\PYG{p}{,} \PYG{n}{fontsize}\PYG{o}{=}\PYG{l+m+mi}{14} \PYG{p}{)}

    \PYG{c+c1}{\PYGZsh{}}
    \PYG{c+c1}{\PYGZsh{} set the y\PYGZhy{}axis labels}
    \PYG{c+c1}{\PYGZsh{}}
    \PYG{k}{for} \PYG{n}{ax}\PYG{p}{,} \PYG{n}{ylabel} \PYG{o+ow}{in} \PYG{n+nb}{zip}\PYG{p}{(}\PYG{n}{axes}\PYG{p}{[}\PYG{p}{:}\PYG{p}{,} \PYG{n}{ncols}\PYG{o}{\PYGZhy{}}\PYG{l+m+mi}{1}\PYG{p}{]}\PYG{p}{,} \PYG{n}{ylabels}\PYG{p}{)}\PYG{p}{:}
        \PYG{n}{ax}\PYG{o}{.}\PYG{n}{yaxis}\PYG{o}{.}\PYG{n}{set\PYGZus{}label\PYGZus{}position}\PYG{p}{(}\PYG{l+s+s1}{\PYGZsq{}}\PYG{l+s+s1}{right}\PYG{l+s+s1}{\PYGZsq{}}\PYG{p}{)}
        \PYG{n}{ax}\PYG{o}{.}\PYG{n}{set\PYGZus{}ylabel}\PYG{p}{(}\PYG{n}{ylabel}\PYG{p}{,} \PYG{n}{fontsize}\PYG{o}{=}\PYG{n}{fontsize\PYGZus{}label}\PYG{p}{,} \PYG{n}{rotation}\PYG{o}{=}\PYG{l+m+mi}{270}\PYG{p}{,} \PYG{n}{labelpad}\PYG{o}{=}\PYG{l+m+mi}{20}\PYG{p}{)}

    \PYG{k}{for} \PYG{n}{i}\PYG{p}{,} \PYG{n}{ax} \PYG{o+ow}{in} \PYG{n+nb}{enumerate}\PYG{p}{(}\PYG{n}{axes}\PYG{p}{[}\PYG{p}{:}\PYG{p}{,}\PYG{l+m+mi}{0}\PYG{p}{]}\PYG{p}{)}\PYG{p}{:}
        \PYG{n}{ax}\PYG{o}{.}\PYG{n}{yaxis}\PYG{o}{.}\PYG{n}{set\PYGZus{}label\PYGZus{}position}\PYG{p}{(}\PYG{l+s+s1}{\PYGZsq{}}\PYG{l+s+s1}{left}\PYG{l+s+s1}{\PYGZsq{}}\PYG{p}{)}
        \PYG{n}{ax}\PYG{o}{.}\PYG{n}{set\PYGZus{}ylabel}\PYG{p}{(}\PYG{n}{yunits}\PYG{p}{[}\PYG{n}{i}\PYG{p}{]}\PYG{p}{,} \PYG{n}{fontsize}\PYG{o}{=}\PYG{n}{fontsize\PYGZus{}label}\PYG{p}{)}

        \PYG{k}{if} \PYG{n}{do\PYGZus{}cdf}\PYG{p}{:}
            \PYG{n}{y\PYGZus{}min}\PYG{p}{,} \PYG{n}{y\PYGZus{}max} \PYG{o}{=} \PYG{l+m+mi}{0}\PYG{p}{,} \PYG{l+m+mi}{1}
            \PYG{n}{ax}\PYG{o}{.}\PYG{n}{set\PYGZus{}ylim}\PYG{p}{(}\PYG{n}{y\PYGZus{}min}\PYG{p}{,} \PYG{n}{y\PYGZus{}max}\PYG{p}{)}

    \PYG{k}{if} \PYG{n}{do\PYGZus{}save} \PYG{o+ow}{and} \PYG{p}{(}\PYG{n}{fname} \PYG{o+ow}{is} \PYG{o+ow}{not} \PYG{k+kc}{None}\PYG{p}{)}\PYG{p}{:}
        \PYG{n}{f}\PYG{o}{.}\PYG{n}{savefig}\PYG{p}{(}\PYG{n}{fname}\PYG{p}{,} \PYG{n}{dpi}\PYG{o}{=}\PYG{n}{my\PYGZus{}dpi}\PYG{p}{)}

    \PYG{k}{return}
\end{sphinxVerbatim}

set up the parameters

\fvset{hllines={, ,}}%
\begin{sphinxVerbatim}[commandchars=\\\{\}]
\PYG{c+c1}{\PYGZsh{}}
\PYG{c+c1}{\PYGZsh{} choose the deomography}
\PYG{c+c1}{\PYGZsh{}}
\PYG{n}{demo} \PYG{o}{=} \PYG{n}{dmg}\PYG{o}{.}\PYG{n}{CHILD\PYGZus{}YOUNG}

\PYG{n}{chooser} \PYG{o}{=} \PYG{p}{\PYGZob{}}\PYG{n}{dmg}\PYG{o}{.}\PYG{n}{ADULT\PYGZus{}WORK}\PYG{p}{:} \PYG{n}{cdaw}\PYG{o}{.}\PYG{n}{CHAD\PYGZus{}demography\PYGZus{}adult\PYGZus{}work}\PYG{p}{(}\PYG{p}{)}\PYG{p}{,}
           \PYG{n}{dmg}\PYG{o}{.}\PYG{n}{ADULT\PYGZus{}NON\PYGZus{}WORK}\PYG{p}{:} \PYG{n}{cdanw}\PYG{o}{.}\PYG{n}{CHAD\PYGZus{}demography\PYGZus{}adult\PYGZus{}non\PYGZus{}work}\PYG{p}{(}\PYG{p}{)}\PYG{p}{,}
           \PYG{n}{dmg}\PYG{o}{.}\PYG{n}{CHILD\PYGZus{}SCHOOL}\PYG{p}{:} \PYG{n}{cdcs}\PYG{o}{.}\PYG{n}{CHAD\PYGZus{}demography\PYGZus{}child\PYGZus{}school}\PYG{p}{(}\PYG{p}{)}\PYG{p}{,}
           \PYG{n}{dmg}\PYG{o}{.}\PYG{n}{CHILD\PYGZus{}YOUNG}\PYG{p}{:} \PYG{n}{cdcy}\PYG{o}{.}\PYG{n}{CHAD\PYGZus{}demography\PYGZus{}child\PYGZus{}young}\PYG{p}{(}\PYG{p}{)}\PYG{p}{,}
           \PYG{p}{\PYGZcb{}}

\PYG{c+c1}{\PYGZsh{} the CHAD demogramphy object}
\PYG{n}{chad\PYGZus{}demo} \PYG{o}{=} \PYG{n}{chooser}\PYG{p}{[}\PYG{n}{demo}\PYG{p}{]}

\PYG{c+c1}{\PYGZsh{} the CHAD sampling parameters}
\PYG{n}{s\PYGZus{}params} \PYG{o}{=} \PYG{n}{chad\PYGZus{}demo}\PYG{o}{.}\PYG{n}{int\PYGZus{}2\PYGZus{}param}
\end{sphinxVerbatim}

\fvset{hllines={, ,}}%
\begin{sphinxVerbatim}[commandchars=\\\{\}]
\PYG{c+c1}{\PYGZsh{} save the figures}
\PYG{n}{do\PYGZus{}save\PYGZus{}fig} \PYG{o}{=} \PYG{k+kc}{False}

\PYG{c+c1}{\PYGZsh{} whether or not to show the plots}
\PYG{n}{do\PYGZus{}show} \PYG{o}{=} \PYG{k+kc}{True}

\PYG{c+c1}{\PYGZsh{} the linewidth}
\PYG{n}{linewidth} \PYG{o}{=} \PYG{l+m+mi}{1}
\end{sphinxVerbatim}

\fvset{hllines={, ,}}%
\begin{sphinxVerbatim}[commandchars=\\\{\}]
\PYG{c+c1}{\PYGZsh{}fpath1 = mg.FDIR\PYGZus{}SAVE\PYGZus{}FIG + \PYGZsq{}\PYGZbs{}\PYGZbs{}11\PYGZus{}21\PYGZus{}2017\PYGZbs{}\PYGZbs{}n8192\PYGZus{}d364\PYGZsq{} \PYGZsh{} with variation}
\PYG{c+c1}{\PYGZsh{}fpath2 = mg.FDIR\PYGZus{}SAVE\PYGZus{}FIG + \PYGZsq{}\PYGZbs{}\PYGZbs{}01\PYGZus{}11\PYGZus{}2018\PYGZbs{}\PYGZbs{}n8192\PYGZus{}d007\PYGZus{}no\PYGZus{}variation\PYGZsq{} \PYGZsh{} no variation}

\PYG{n}{fpath1} \PYG{o}{=} \PYG{n}{mg}\PYG{o}{.}\PYG{n}{FDIR\PYGZus{}SAVE\PYGZus{}FIG} \PYG{o}{+} \PYG{l+s+s1}{\PYGZsq{}}\PYG{l+s+se}{\PYGZbs{}\PYGZbs{}}\PYG{l+s+s1}{12\PYGZus{}07\PYGZus{}2017}\PYG{l+s+se}{\PYGZbs{}\PYGZbs{}}\PYG{l+s+s1}{n8192\PYGZus{}d364}\PYG{l+s+s1}{\PYGZsq{}} \PYG{c+c1}{\PYGZsh{} with variation}
\PYG{n}{fpath2} \PYG{o}{=} \PYG{n}{mg}\PYG{o}{.}\PYG{n}{FDIR\PYGZus{}SAVE\PYGZus{}FIG} \PYG{o}{+} \PYG{l+s+s1}{\PYGZsq{}}\PYG{l+s+se}{\PYGZbs{}\PYGZbs{}}\PYG{l+s+s1}{01\PYGZus{}16\PYGZus{}2018\PYGZus{}no\PYGZus{}variation}\PYG{l+s+se}{\PYGZbs{}\PYGZbs{}}\PYG{l+s+s1}{n8192\PYGZus{}d007}\PYG{l+s+s1}{\PYGZsq{}} \PYG{c+c1}{\PYGZsh{} no variation}

\PYG{c+c1}{\PYGZsh{}fpath\PYGZus{}temp = mg.FDIR\PYGZus{}SAVE\PYGZus{}FIG + \PYGZsq{}\PYGZbs{}\PYGZbs{}with\PYGZus{}without\PYGZus{}variation\PYGZsq{}}
\PYG{c+c1}{\PYGZsh{}fpath1 = fpath\PYGZus{}temp + \PYGZsq{}\PYGZbs{}\PYGZbs{}n8192\PYGZus{}d007\PYGZus{}with\PYGZus{}variation\PYGZsq{}}
\PYG{c+c1}{\PYGZsh{}fpath2 = fpath\PYGZus{}temp + \PYGZsq{}\PYGZbs{}\PYGZbs{}n8192\PYGZus{}d364\PYGZus{}no\PYGZus{}variation\PYGZsq{}}

\PYG{n}{fpath\PYGZus{}figure\PYGZus{}save1} \PYG{o}{=} \PYG{n}{fpath1} \PYG{o}{+} \PYG{l+s+s1}{\PYGZsq{}}\PYG{l+s+se}{\PYGZbs{}\PYGZbs{}}\PYG{l+s+s1}{child\PYGZus{}young}\PYG{l+s+s1}{\PYGZsq{}}
\PYG{n}{fpath\PYGZus{}figure\PYGZus{}save2} \PYG{o}{=} \PYG{n}{fpath2} \PYG{o}{+} \PYG{l+s+s1}{\PYGZsq{}}\PYG{l+s+se}{\PYGZbs{}\PYGZbs{}}\PYG{l+s+s1}{child\PYGZus{}young}\PYG{l+s+s1}{\PYGZsq{}}

\PYG{c+c1}{\PYGZsh{} print the save figure directory}
\PYG{n+nb}{print}\PYG{p}{(}\PYG{l+s+s1}{\PYGZsq{}}\PYG{l+s+s1}{the figure save path 1:}\PYG{l+s+se}{\PYGZbs{}t}\PYG{l+s+si}{\PYGZpc{}s}\PYG{l+s+s1}{\PYGZsq{}} \PYG{o}{\PYGZpc{}} \PYG{n}{fpath\PYGZus{}figure\PYGZus{}save1}\PYG{p}{)}
\PYG{n+nb}{print}\PYG{p}{(}\PYG{l+s+s1}{\PYGZsq{}}\PYG{l+s+s1}{the figure save path 2:}\PYG{l+s+se}{\PYGZbs{}t}\PYG{l+s+si}{\PYGZpc{}s}\PYG{l+s+s1}{\PYGZsq{}} \PYG{o}{\PYGZpc{}} \PYG{n}{fpath\PYGZus{}figure\PYGZus{}save2}\PYG{p}{)}

\PYG{c+c1}{\PYGZsh{} different sets of activitiy data to plot}
\PYG{n}{keys\PYGZus{}all} \PYG{o}{=} \PYG{n}{chad\PYGZus{}demo}\PYG{o}{.}\PYG{n}{keys}

\PYG{n}{keys\PYGZus{}eat} \PYG{o}{=} \PYG{p}{[}\PYG{n}{mg}\PYG{o}{.}\PYG{n}{KEY\PYGZus{}EAT\PYGZus{}BREAKFAST}\PYG{p}{,} \PYG{n}{mg}\PYG{o}{.}\PYG{n}{KEY\PYGZus{}EAT\PYGZus{}LUNCH}\PYG{p}{,} \PYG{n}{mg}\PYG{o}{.}\PYG{n}{KEY\PYGZus{}EAT\PYGZus{}DINNER}\PYG{p}{]}

\PYG{n}{keys\PYGZus{}not\PYGZus{}eat} \PYG{o}{=} \PYG{p}{[} \PYG{n}{k} \PYG{k}{for} \PYG{n}{k} \PYG{o+ow}{in} \PYG{n}{keys\PYGZus{}all} \PYG{k}{if} \PYG{n}{k} \PYG{o+ow}{not} \PYG{o+ow}{in} \PYG{n}{keys\PYGZus{}eat} \PYG{p}{]}
\end{sphinxVerbatim}
\begin{sphinxalltt}
the figure save path 1:     ..my\_datafig12\_07\_2017n8192\_d364child\_young
the figure save path 2:     ..my\_datafig01\_16\_2018\_no\_variationn8192\_d007child\_young
\end{sphinxalltt}

Plotting

\fvset{hllines={, ,}}%
\begin{sphinxVerbatim}[commandchars=\\\{\}]
\PYG{n}{DO\PYGZus{}ALL} \PYG{o}{=} \PYG{l+m+mi}{1}
\PYG{n}{DO\PYGZus{}MEALS} \PYG{o}{=} \PYG{l+m+mi}{2}
\PYG{n}{DO\PYGZus{}NOT\PYGZus{}MEALS} \PYG{o}{=} \PYG{l+m+mi}{3}

\PYG{c+c1}{\PYGZsh{} (the activites to plot, part of the file name that matches the keys)}
\PYG{n}{chooser\PYGZus{}keys} \PYG{o}{=} \PYG{p}{\PYGZob{}} \PYG{n}{DO\PYGZus{}ALL}\PYG{p}{:} \PYG{p}{(}\PYG{n}{keys\PYGZus{}all}\PYG{p}{,} \PYG{l+s+s1}{\PYGZsq{}}\PYG{l+s+s1}{all}\PYG{l+s+s1}{\PYGZsq{}}\PYG{p}{)}\PYG{p}{,} \PYGZbs{}
                \PYG{n}{DO\PYGZus{}MEALS}\PYG{p}{:} \PYG{p}{(}\PYG{n}{keys\PYGZus{}eat}\PYG{p}{,} \PYG{l+s+s1}{\PYGZsq{}}\PYG{l+s+s1}{meals}\PYG{l+s+s1}{\PYGZsq{}}\PYG{p}{)}\PYG{p}{,}\PYGZbs{}
                \PYG{n}{DO\PYGZus{}NOT\PYGZus{}MEALS}\PYG{p}{:} \PYG{p}{(}\PYG{n}{keys\PYGZus{}not\PYGZus{}eat}\PYG{p}{,} \PYG{l+s+s1}{\PYGZsq{}}\PYG{l+s+s1}{not\PYGZus{}meals}\PYG{l+s+s1}{\PYGZsq{}}\PYG{p}{)}\PYG{p}{,}
               \PYG{p}{\PYGZcb{}}
\end{sphinxVerbatim}

\fvset{hllines={, ,}}%
\begin{sphinxVerbatim}[commandchars=\\\{\}]
\PYG{c+c1}{\PYGZsh{}}
\PYG{c+c1}{\PYGZsh{} set the activities to plot}
\PYG{c+c1}{\PYGZsh{}}
\PYG{n}{plot\PYGZus{}keys} \PYG{o}{=} \PYG{n}{DO\PYGZus{}ALL}

\PYG{n}{keys}\PYG{p}{,} \PYG{n}{fname\PYGZus{}keys} \PYG{o}{=} \PYG{n}{chooser\PYGZus{}keys}\PYG{p}{[}\PYG{n}{plot\PYGZus{}keys}\PYG{p}{]}
\PYG{n}{name\PYGZus{}keys} \PYG{o}{=} \PYG{p}{[} \PYG{n}{activity}\PYG{o}{.}\PYG{n}{INT\PYGZus{}2\PYGZus{}STR}\PYG{p}{[}\PYG{n}{k}\PYG{p}{]} \PYG{k}{for} \PYG{n}{k} \PYG{o+ow}{in} \PYG{n}{keys}\PYG{p}{]}


\PYG{c+c1}{\PYGZsh{} labels on the right hand side of the plot}
\PYG{n}{ylabels} \PYG{o}{=} \PYG{p}{[}\PYG{l+s+s1}{\PYGZsq{}}\PYG{l+s+s1}{Start Time}\PYG{l+s+s1}{\PYGZsq{}}\PYG{p}{,} \PYG{l+s+s1}{\PYGZsq{}}\PYG{l+s+s1}{End Time}\PYG{l+s+s1}{\PYGZsq{}}\PYG{p}{,} \PYG{l+s+s1}{\PYGZsq{}}\PYG{l+s+s1}{Duration}\PYG{l+s+s1}{\PYGZsq{}}\PYG{p}{]}
\end{sphinxVerbatim}

Plot CDFs vs Longitudinal data

plot verification

\fvset{hllines={, ,}}%
\begin{sphinxVerbatim}[commandchars=\\\{\}]
\PYG{c+c1}{\PYGZsh{} get the figure directory of ABMHAP runs with intra\PYGZhy{}individual variation}
\PYG{n}{fpaths1} \PYG{o}{=} \PYG{n}{analyzer}\PYG{o}{.}\PYG{n}{get\PYGZus{}verify\PYGZus{}fpath}\PYG{p}{(}\PYG{n}{fpath\PYGZus{}figure\PYGZus{}save1}\PYG{p}{,} \PYG{n}{keys}\PYG{p}{)}

\PYG{c+c1}{\PYGZsh{} get the figure directory of ABMHAP runs with no intra\PYGZhy{}individual variation}
\PYG{n}{fpaths2} \PYG{o}{=} \PYG{n}{analyzer}\PYG{o}{.}\PYG{n}{get\PYGZus{}verify\PYGZus{}fpath}\PYG{p}{(}\PYG{n}{fpath\PYGZus{}figure\PYGZus{}save2}\PYG{p}{,} \PYG{n}{keys}\PYG{p}{)}
\end{sphinxVerbatim}

\fvset{hllines={, ,}}%
\begin{sphinxVerbatim}[commandchars=\\\{\}]
\PYG{c+c1}{\PYGZsh{} load figure data with longitudinal data}

\PYG{c+c1}{\PYGZsh{} file names}
\PYG{n}{fname} \PYG{o}{=} \PYG{l+s+s1}{\PYGZsq{}}\PYG{l+s+se}{\PYGZbs{}\PYGZbs{}}\PYG{l+s+s1}{cdf\PYGZus{}}\PYG{l+s+s1}{\PYGZsq{}} \PYG{o}{+} \PYG{n}{fname\PYGZus{}keys} \PYG{o}{+} \PYG{l+s+s1}{\PYGZsq{}}\PYG{l+s+s1}{.png}\PYG{l+s+s1}{\PYGZsq{}}

\PYG{c+c1}{\PYGZsh{} load figure data}
\PYG{n}{data\PYGZus{}list\PYGZus{}all1}\PYG{p}{,} \PYG{n}{fname\PYGZus{}subplot1} \PYG{o}{=} \PYG{n}{plotter}\PYG{o}{.}\PYG{n}{get\PYGZus{}figure\PYGZus{}data}\PYG{p}{(}\PYG{n}{fpaths1}\PYG{p}{,} \PYG{n}{fpath\PYGZus{}figure\PYGZus{}save1}\PYG{p}{,} \PYG{n}{fname}\PYG{p}{)}
\PYG{n}{data\PYGZus{}list\PYGZus{}all2}\PYG{p}{,} \PYG{n}{fname\PYGZus{}subplot2} \PYG{o}{=} \PYG{n}{plotter}\PYG{o}{.}\PYG{n}{get\PYGZus{}figure\PYGZus{}data}\PYG{p}{(}\PYG{n}{fpaths2}\PYG{p}{,} \PYG{n}{fpath\PYGZus{}figure\PYGZus{}save2}\PYG{p}{,} \PYG{n}{fname}\PYG{p}{)}
\end{sphinxVerbatim}

Get the data for a random single day

\fvset{hllines={, ,}}%
\begin{sphinxVerbatim}[commandchars=\\\{\}]
\PYG{c+c1}{\PYGZsh{} load figure data of sinlge\PYGZhy{}day data}

\PYG{c+c1}{\PYGZsh{} file names}
\PYG{n}{fname} \PYG{o}{=} \PYG{l+s+s1}{\PYGZsq{}}\PYG{l+s+se}{\PYGZbs{}\PYGZbs{}}\PYG{l+s+s1}{cdf\PYGZus{}}\PYG{l+s+s1}{\PYGZsq{}} \PYG{o}{+} \PYG{n}{fname\PYGZus{}keys} \PYG{o}{+} \PYG{l+s+s1}{\PYGZsq{}}\PYG{l+s+s1}{.png}\PYG{l+s+s1}{\PYGZsq{}}

\PYG{n}{fnames\PYGZus{}load} \PYG{o}{=} \PYG{p}{(}\PYG{l+s+s1}{\PYGZsq{}}\PYG{l+s+se}{\PYGZbs{}\PYGZbs{}}\PYG{l+s+s1}{cdf\PYGZus{}start.pkl}\PYG{l+s+s1}{\PYGZsq{}}\PYG{p}{,} \PYG{l+s+s1}{\PYGZsq{}}\PYG{l+s+se}{\PYGZbs{}\PYGZbs{}}\PYG{l+s+s1}{cdf\PYGZus{}end.pkl}\PYG{l+s+s1}{\PYGZsq{}}\PYG{p}{,} \PYG{l+s+s1}{\PYGZsq{}}\PYG{l+s+se}{\PYGZbs{}\PYGZbs{}}\PYG{l+s+s1}{cdf\PYGZus{}dt.pkl}\PYG{l+s+s1}{\PYGZsq{}}\PYG{p}{)}

\PYG{c+c1}{\PYGZsh{} load figure data from ABMHAP figures with intra\PYGZhy{}individual variation}
\PYG{n}{data\PYGZus{}list\PYGZus{}all\PYGZus{}single\PYGZus{}day1}\PYG{p}{,} \PYG{n}{fname\PYGZus{}subplot1} \PYG{o}{=} \PYGZbs{}
\PYG{n}{plotter}\PYG{o}{.}\PYG{n}{get\PYGZus{}figure\PYGZus{}data}\PYG{p}{(}\PYG{n}{fpaths1}\PYG{p}{,} \PYG{n}{fpath\PYGZus{}figure\PYGZus{}save1}\PYG{p}{,} \PYG{n}{fname}\PYG{p}{,} \PYG{n}{fnames\PYGZus{}load}\PYG{o}{=}\PYG{n}{fnames\PYGZus{}load}\PYG{p}{,} \PYG{n}{do\PYGZus{}single\PYGZus{}day}\PYG{o}{=}\PYG{k+kc}{True}\PYG{p}{)}

\PYG{c+c1}{\PYGZsh{} load figure data from ABMHAP figures with no intra\PYGZhy{}individual variation}
\PYG{n}{data\PYGZus{}list\PYGZus{}all\PYGZus{}single\PYGZus{}day2}\PYG{p}{,} \PYG{n}{fname\PYGZus{}subplot2} \PYG{o}{=} \PYGZbs{}
\PYG{n}{plotter}\PYG{o}{.}\PYG{n}{get\PYGZus{}figure\PYGZus{}data}\PYG{p}{(}\PYG{n}{fpaths2}\PYG{p}{,} \PYG{n}{fpath\PYGZus{}figure\PYGZus{}save2}\PYG{p}{,} \PYG{n}{fname}\PYG{p}{,} \PYG{n}{fnames\PYGZus{}load}\PYG{o}{=}\PYG{n}{fnames\PYGZus{}load}\PYG{p}{,} \PYG{n}{do\PYGZus{}single\PYGZus{}day}\PYG{o}{=}\PYG{k+kc}{True}\PYG{p}{)}
\end{sphinxVerbatim}

\fvset{hllines={, ,}}%
\begin{sphinxVerbatim}[commandchars=\\\{\}]
\PYG{n}{fpath\PYGZus{}figure\PYGZus{}save2}
\end{sphinxVerbatim}
\begin{sphinxalltt}
'..\textbackslash{}my\_data\textbackslash{}fig\textbackslash{}01\_16\_2018\_no\_variation\textbackslash{}n8192\_d007\textbackslash{}child\_young'
\end{sphinxalltt}

plot the cdf

\fvset{hllines={, ,}}%
\begin{sphinxVerbatim}[commandchars=\\\{\}]
\PYG{c+c1}{\PYGZsh{}}
\PYG{c+c1}{\PYGZsh{} plot the verification cdf}
\PYG{c+c1}{\PYGZsh{}}

\PYG{c+c1}{\PYGZsh{}}
\PYG{c+c1}{\PYGZsh{} plotting parameters}
\PYG{c+c1}{\PYGZsh{}}
\PYG{n}{do\PYGZus{}cdf} \PYG{o}{=} \PYG{k+kc}{True}

\PYG{n}{colors} \PYG{o}{=} \PYG{p}{[}\PYG{l+s+s1}{\PYGZsq{}}\PYG{l+s+s1}{blue}\PYG{l+s+s1}{\PYGZsq{}}\PYG{p}{,} \PYG{l+s+s1}{\PYGZsq{}}\PYG{l+s+s1}{red}\PYG{l+s+s1}{\PYGZsq{}}\PYG{p}{]}
\PYG{n}{legend} \PYG{o}{=} \PYG{p}{[}\PYG{l+s+s1}{\PYGZsq{}}\PYG{l+s+s1}{With Intra}\PYG{l+s+s1}{\PYGZsq{}}\PYG{p}{,}\PYG{l+s+s1}{\PYGZsq{}}\PYG{l+s+s1}{No Intra}\PYG{l+s+s1}{\PYGZsq{}}\PYG{p}{,} \PYG{l+s+s1}{\PYGZsq{}}\PYG{l+s+s1}{CHAD single day}\PYG{l+s+s1}{\PYGZsq{}}\PYG{p}{,} \PYG{l+s+s1}{\PYGZsq{}}\PYG{l+s+s1}{CHAD means}\PYG{l+s+s1}{\PYGZsq{}}\PYG{p}{]}

\PYG{n}{xunits} \PYG{o}{=} \PYG{l+s+s1}{\PYGZsq{}}\PYG{l+s+s1}{Hours}\PYG{l+s+s1}{\PYGZsq{}}
\PYG{n}{yunits} \PYG{o}{=} \PYG{p}{[}\PYG{l+s+s1}{\PYGZsq{}}\PYG{l+s+s1}{Quantile}\PYG{l+s+s1}{\PYGZsq{}}\PYG{p}{]} \PYG{o}{*} \PYG{l+m+mi}{3}

\PYG{n}{main\PYGZus{}title} \PYG{o}{=} \PYG{l+s+s1}{\PYGZsq{}}\PYG{l+s+s1}{CDFs of Activity\PYGZhy{}parameters}\PYG{l+s+s1}{\PYGZsq{}}

\PYG{n}{xlabels} \PYG{o}{=} \PYG{p}{[}\PYG{n}{xunits}\PYG{p}{]} \PYG{o}{*} \PYG{n+nb}{len}\PYG{p}{(}\PYG{n}{keys}\PYG{p}{)}

\PYG{c+c1}{\PYGZsh{}}
\PYG{c+c1}{\PYGZsh{} plot}
\PYG{c+c1}{\PYGZsh{}}

\PYG{c+c1}{\PYGZsh{} set the data}
\PYG{n}{data\PYGZus{}list1} \PYG{o}{=} \PYG{n}{data\PYGZus{}list\PYGZus{}all\PYGZus{}single\PYGZus{}day1} \PYG{c+c1}{\PYGZsh{} with variaiton}
\PYG{n}{data\PYGZus{}list2} \PYG{o}{=} \PYG{n}{data\PYGZus{}list\PYGZus{}all\PYGZus{}single\PYGZus{}day2} \PYG{c+c1}{\PYGZsh{} no variation}
\PYG{n}{data\PYGZus{}list3} \PYG{o}{=} \PYG{n}{data\PYGZus{}list\PYGZus{}all\PYGZus{}single\PYGZus{}day1} \PYG{c+c1}{\PYGZsh{} acesses the CHAD random day data which is encapsulated within}
                                        \PYG{c+c1}{\PYGZsh{} data\PYGZus{}list[irow][icol][1]}

\PYG{c+c1}{\PYGZsh{} plot the data}
\PYG{n}{plot\PYGZus{}subplots}\PYG{p}{(}\PYG{n}{data\PYGZus{}list1}\PYG{o}{=}\PYG{n}{data\PYGZus{}list1}\PYG{p}{,} \PYG{n}{data\PYGZus{}list2}\PYG{o}{=}\PYG{n}{data\PYGZus{}list2}\PYG{p}{,} \PYG{n}{data\PYGZus{}list3}\PYG{o}{=}\PYG{n}{data\PYGZus{}list3}\PYG{p}{,} \PYGZbs{}
                   \PYG{n}{do\PYGZus{}cdf}\PYG{o}{=}\PYG{n}{do\PYGZus{}cdf}\PYG{p}{,} \PYG{n}{main\PYGZus{}title}\PYG{o}{=}\PYG{n}{main\PYGZus{}title}\PYG{p}{,} \PYGZbs{}
                   \PYG{n}{legend}\PYG{o}{=}\PYG{n}{legend}\PYG{p}{,} \PYG{n}{xlabels}\PYG{o}{=}\PYG{n}{xlabels}\PYG{p}{,} \PYG{n}{ylabels}\PYG{o}{=}\PYG{n}{ylabels}\PYG{p}{,} \PYG{n}{xunits}\PYG{o}{=}\PYG{n}{xunits}\PYG{p}{,} \PYG{n}{yunits}\PYG{o}{=}\PYG{n}{yunits}\PYG{p}{,} \PYG{n}{colors}\PYG{o}{=}\PYG{n}{colors}\PYG{p}{,} \PYGZbs{}
                   \PYG{n}{do\PYGZus{}save}\PYG{o}{=}\PYG{n}{do\PYGZus{}save\PYGZus{}fig}\PYG{p}{,} \PYG{n}{fname}\PYG{o}{=}\PYG{n}{fname\PYGZus{}subplot1}\PYG{p}{,} \PYG{n}{linewidth}\PYG{o}{=}\PYG{l+m+mf}{0.5}\PYG{p}{)}

\PYG{k}{if} \PYG{n}{do\PYGZus{}show}\PYG{p}{:}
    \PYG{n}{plt}\PYG{o}{.}\PYG{n}{show}\PYG{p}{(}\PYG{p}{)}
\PYG{k}{else}\PYG{p}{:}
    \PYG{n}{plt}\PYG{o}{.}\PYG{n}{close}\PYG{p}{(}\PYG{p}{)}
\end{sphinxVerbatim}
\begin{sphinxalltt}
C:UsersnbrandonAppDataLocalContinuumAnaconda3libsite-packagesmatplotliblegend.py:338: UserWarning: Automatic legend placement (loc="best") not implemented for figure legend. Falling back on "upper right".
  warnings.warn('Automatic legend placement (loc="best") not '
\end{sphinxalltt}


\subsection{figure\_residuals notebook}
\label{\detokenize{figure_residuals::doc}}\label{\detokenize{figure_residuals:figure-residuals-notebook}}
\fvset{hllines={, ,}}%
\begin{sphinxVerbatim}[commandchars=\\\{\}]
\PYG{c+c1}{\PYGZsh{} The United States Environmental Protection Agency through its Office of}
\PYG{c+c1}{\PYGZsh{} Research and Development has developed this software. The code is made}
\PYG{c+c1}{\PYGZsh{} publicly available to better communicate the research. All input data}
\PYG{c+c1}{\PYGZsh{} used fora given application should be reviewed by the researcher so}
\PYG{c+c1}{\PYGZsh{} that the model results are based on appropriate data for any given}
\PYG{c+c1}{\PYGZsh{} application. This model is under continued development. The model and}
\PYG{c+c1}{\PYGZsh{} data included herein do not represent and should not be construed to}
\PYG{c+c1}{\PYGZsh{} represent any Agency determination or policy.}
\PYG{c+c1}{\PYGZsh{}}
\PYG{c+c1}{\PYGZsh{} This file was written by Dr. Namdi Brandon}
\PYG{c+c1}{\PYGZsh{} ORCID: 0000\PYGZhy{}0001\PYGZhy{}7050\PYGZhy{}1538}
\PYG{c+c1}{\PYGZsh{} March 20, 2018}
\end{sphinxVerbatim}

This file calculates the residuals in the cumaltive distribution
functions (CDFs) for the activities in each demographic.

The file calculates the residuals = \textbar{}cdf\_predicted - cdf\_observed\textbar{}
as a function of percentile from 0 to 1. Then the mean value for the
residual plot is calculated which represents the expected deviation from
the data for each percentile

Import

\fvset{hllines={, ,}}%
\begin{sphinxVerbatim}[commandchars=\\\{\}]
\PYG{k+kn}{import} \PYG{n+nn}{sys}
\PYG{n}{sys}\PYG{o}{.}\PYG{n}{path}\PYG{o}{.}\PYG{n}{append}\PYG{p}{(}\PYG{l+s+s1}{\PYGZsq{}}\PYG{l+s+s1}{..}\PYG{l+s+se}{\PYGZbs{}\PYGZbs{}}\PYG{l+s+s1}{source}\PYG{l+s+s1}{\PYGZsq{}}\PYG{p}{)}
\PYG{n}{sys}\PYG{o}{.}\PYG{n}{path}\PYG{o}{.}\PYG{n}{append}\PYG{p}{(}\PYG{l+s+s1}{\PYGZsq{}}\PYG{l+s+s1}{..}\PYG{l+s+se}{\PYGZbs{}\PYGZbs{}}\PYG{l+s+s1}{processing}\PYG{l+s+s1}{\PYGZsq{}}\PYG{p}{)}
\PYG{n}{sys}\PYG{o}{.}\PYG{n}{path}\PYG{o}{.}\PYG{n}{append}\PYG{p}{(}\PYG{l+s+s1}{\PYGZsq{}}\PYG{l+s+s1}{..}\PYG{l+s+se}{\PYGZbs{}\PYGZbs{}}\PYG{l+s+s1}{plotting}\PYG{l+s+s1}{\PYGZsq{}}\PYG{p}{)}

\PYG{c+c1}{\PYGZsh{} plotting capability analysis}
\PYG{k+kn}{import} \PYG{n+nn}{matplotlib}\PYG{n+nn}{.}\PYG{n+nn}{pylab} \PYG{k}{as} \PYG{n+nn}{plt}

\PYG{c+c1}{\PYGZsh{} math capability}
\PYG{k+kn}{import} \PYG{n+nn}{numpy} \PYG{k}{as} \PYG{n+nn}{np}

\PYG{c+c1}{\PYGZsh{} python data compression}
\PYG{k+kn}{import} \PYG{n+nn}{pickle}

\PYG{c+c1}{\PYGZsh{} ABMHAP modules}
\PYG{k+kn}{import} \PYG{n+nn}{my\PYGZus{}globals} \PYG{k}{as} \PYG{n+nn}{mg}
\PYG{k+kn}{import} \PYG{n+nn}{chad\PYGZus{}demography\PYGZus{}adult\PYGZus{}work} \PYG{k}{as} \PYG{n+nn}{cdaw}
\PYG{k+kn}{import} \PYG{n+nn}{chad\PYGZus{}demography\PYGZus{}adult\PYGZus{}non\PYGZus{}work} \PYG{k}{as} \PYG{n+nn}{cdanw}
\PYG{k+kn}{import} \PYG{n+nn}{chad\PYGZus{}demography\PYGZus{}child\PYGZus{}school} \PYG{k}{as} \PYG{n+nn}{cdcs}
\PYG{k+kn}{import} \PYG{n+nn}{chad\PYGZus{}demography\PYGZus{}child\PYGZus{}young} \PYG{k}{as} \PYG{n+nn}{cdcy}
\PYG{k+kn}{import} \PYG{n+nn}{demography} \PYG{k}{as} \PYG{n+nn}{dmg}

\PYG{k+kn}{import} \PYG{n+nn}{activity}\PYG{o}{,} \PYG{n+nn}{plotter}
\end{sphinxVerbatim}

\fvset{hllines={, ,}}%
\begin{sphinxVerbatim}[commandchars=\\\{\}]
\PYG{o}{\PYGZpc{}}\PYG{k}{matplotlib} auto
\end{sphinxVerbatim}

\fvset{hllines={, ,}}%
\begin{sphinxVerbatim}[commandchars=\\\{\}]
\PYG{n}{Using} \PYG{n}{matplotlib} \PYG{n}{backend}\PYG{p}{:} \PYG{n}{Qt5Agg}
\end{sphinxVerbatim}

define functions

\fvset{hllines={, ,}}%
\begin{sphinxVerbatim}[commandchars=\\\{\}]
\PYG{k}{def} \PYG{n+nf}{f}\PYG{p}{(}\PYG{n}{data}\PYG{p}{,} \PYG{n}{alpha}\PYG{o}{=}\PYG{l+m+mi}{0}\PYG{p}{)}\PYG{p}{:}

    \PYG{c+c1}{\PYGZsh{} create the residuals between the prediction (ABMHAP) and observation (CHAD)}
    \PYG{c+c1}{\PYGZsh{} data. Plot the quantiles of the data [alpha, 1 \PYGZhy{} alpha] percentiles of the data.}

    \PYG{c+c1}{\PYGZsh{} predicted data and observed data}
    \PYG{n}{pred}\PYG{p}{,} \PYG{n}{obs} \PYG{o}{=} \PYG{n}{data}

    \PYG{c+c1}{\PYGZsh{} the x and y values for the predicted data and observed data}
    \PYG{n}{x\PYGZus{}pred}\PYG{p}{,} \PYG{n}{y\PYGZus{}pred} \PYG{o}{=} \PYG{n}{pred}
    \PYG{n}{x\PYGZus{}obs}\PYG{p}{,} \PYG{n}{y\PYGZus{}obs} \PYG{o}{=} \PYG{n}{obs}

    \PYG{c+c1}{\PYGZsh{} residual}
    \PYG{n}{r} \PYG{o}{=} \PYG{n}{np}\PYG{o}{.}\PYG{n}{abs}\PYG{p}{(}\PYG{n}{y\PYGZus{}pred} \PYG{o}{\PYGZhy{}} \PYG{n}{y\PYGZus{}obs}\PYG{p}{)}

    \PYG{c+c1}{\PYGZsh{} the number of data points}
    \PYG{n}{m} \PYG{o}{=} \PYG{n+nb}{len}\PYG{p}{(}\PYG{n}{r}\PYG{p}{)}

    \PYG{c+c1}{\PYGZsh{} the bottom and top percentile}
    \PYG{n}{bot}\PYG{p}{,} \PYG{n}{top} \PYG{o}{=} \PYG{n}{alpha}\PYG{o}{/}\PYG{l+m+mi}{2}\PYG{p}{,} \PYG{l+m+mi}{1} \PYG{o}{\PYGZhy{}} \PYG{n}{alpha}\PYG{o}{/}\PYG{l+m+mi}{2}

    \PYG{c+c1}{\PYGZsh{} get the percentiles within range}
    \PYG{n}{x} \PYG{o}{=} \PYG{n}{x\PYGZus{}pred}
    \PYG{n}{idx} \PYG{o}{=} \PYG{p}{(}\PYG{n}{x} \PYG{o}{\PYGZgt{}}\PYG{o}{=} \PYG{n}{bot}\PYG{p}{)} \PYG{o}{\PYGZam{}} \PYG{p}{(}\PYG{n}{x} \PYG{o}{\PYGZlt{}}\PYG{o}{=} \PYG{n}{top}\PYG{p}{)}

    \PYG{k}{return} \PYG{n}{x}\PYG{p}{[}\PYG{n}{idx}\PYG{p}{]}\PYG{p}{,} \PYG{n}{r}\PYG{p}{[}\PYG{n}{idx}\PYG{p}{]}

\PYG{c+c1}{\PYGZsh{} get the moments}
\PYG{k}{def} \PYG{n+nf}{get\PYGZus{}moments}\PYG{p}{(}\PYG{n}{x}\PYG{p}{)}\PYG{p}{:}

    \PYG{c+c1}{\PYGZsh{} the mean data}
    \PYG{n}{mu} \PYG{o}{=} \PYG{n}{x}\PYG{o}{.}\PYG{n}{mean}\PYG{p}{(}\PYG{p}{)}

    \PYG{c+c1}{\PYGZsh{} the standard deviation data}
    \PYG{n}{std} \PYG{o}{=} \PYG{n}{x}\PYG{o}{.}\PYG{n}{std}\PYG{p}{(}\PYG{p}{)}

    \PYG{k}{return} \PYG{n}{mu}\PYG{p}{,} \PYG{n}{std}
\end{sphinxVerbatim}

set up the parameters

\fvset{hllines={, ,}}%
\begin{sphinxVerbatim}[commandchars=\\\{\}]
\PYG{c+c1}{\PYGZsh{}}
\PYG{c+c1}{\PYGZsh{} choose the deomography}
\PYG{c+c1}{\PYGZsh{}}
\PYG{n}{demo} \PYG{o}{=} \PYG{n}{dmg}\PYG{o}{.}\PYG{n}{CHILD\PYGZus{}YOUNG}

\PYG{n}{chooser} \PYG{o}{=} \PYG{p}{\PYGZob{}}\PYG{n}{dmg}\PYG{o}{.}\PYG{n}{ADULT\PYGZus{}WORK}\PYG{p}{:} \PYG{n}{cdaw}\PYG{o}{.}\PYG{n}{CHAD\PYGZus{}demography\PYGZus{}adult\PYGZus{}work}\PYG{p}{(}\PYG{p}{)}\PYG{p}{,}
           \PYG{n}{dmg}\PYG{o}{.}\PYG{n}{ADULT\PYGZus{}NON\PYGZus{}WORK}\PYG{p}{:} \PYG{n}{cdanw}\PYG{o}{.}\PYG{n}{CHAD\PYGZus{}demography\PYGZus{}adult\PYGZus{}non\PYGZus{}work}\PYG{p}{(}\PYG{p}{)}\PYG{p}{,}
           \PYG{n}{dmg}\PYG{o}{.}\PYG{n}{CHILD\PYGZus{}SCHOOL}\PYG{p}{:} \PYG{n}{cdcs}\PYG{o}{.}\PYG{n}{CHAD\PYGZus{}demography\PYGZus{}child\PYGZus{}school}\PYG{p}{(}\PYG{p}{)}\PYG{p}{,}
           \PYG{n}{dmg}\PYG{o}{.}\PYG{n}{CHILD\PYGZus{}YOUNG}\PYG{p}{:} \PYG{n}{cdcy}\PYG{o}{.}\PYG{n}{CHAD\PYGZus{}demography\PYGZus{}child\PYGZus{}young}\PYG{p}{(}\PYG{p}{)}\PYG{p}{,}
           \PYG{p}{\PYGZcb{}}

\PYG{c+c1}{\PYGZsh{} the CHAD demogramphy object}
\PYG{n}{chad\PYGZus{}demo} \PYG{o}{=} \PYG{n}{chooser}\PYG{p}{[}\PYG{n}{demo}\PYG{p}{]}

\PYG{c+c1}{\PYGZsh{} the CHAD sampling parameters}
\PYG{n}{s\PYGZus{}params} \PYG{o}{=} \PYG{n}{chad\PYGZus{}demo}\PYG{o}{.}\PYG{n}{int\PYGZus{}2\PYGZus{}param}
\end{sphinxVerbatim}

\fvset{hllines={, ,}}%
\begin{sphinxVerbatim}[commandchars=\\\{\}]
\PYG{c+c1}{\PYGZsh{} save the figures}
\PYG{n}{do\PYGZus{}save\PYGZus{}fig} \PYG{o}{=} \PYG{k+kc}{False}

\PYG{c+c1}{\PYGZsh{} whether or not to show the plots}
\PYG{n}{do\PYGZus{}show} \PYG{o}{=} \PYG{k+kc}{True}

\PYG{c+c1}{\PYGZsh{} the linewidth}
\PYG{n}{linewidth} \PYG{o}{=} \PYG{l+m+mf}{1.5}
\end{sphinxVerbatim}

\fvset{hllines={, ,}}%
\begin{sphinxVerbatim}[commandchars=\\\{\}]
\PYG{c+c1}{\PYGZsh{} choose the appropriate figure directory}
\PYG{n}{fpath} \PYG{o}{=} \PYG{n}{mg}\PYG{o}{.}\PYG{n}{FDIR\PYGZus{}SAVE\PYGZus{}FIG} \PYG{o}{+} \PYG{l+s+s1}{\PYGZsq{}}\PYG{l+s+se}{\PYGZbs{}\PYGZbs{}}\PYG{l+s+s1}{12\PYGZus{}07\PYGZus{}2017}\PYG{l+s+se}{\PYGZbs{}\PYGZbs{}}\PYG{l+s+s1}{n8192\PYGZus{}d364}\PYG{l+s+s1}{\PYGZsq{}}

\PYG{n}{chooser\PYGZus{}fin} \PYG{o}{=} \PYG{p}{\PYGZob{}}\PYG{n}{dmg}\PYG{o}{.}\PYG{n}{ADULT\PYGZus{}WORK}\PYG{p}{:} \PYG{n}{fpath} \PYG{o}{+} \PYG{l+s+s1}{\PYGZsq{}}\PYG{l+s+se}{\PYGZbs{}\PYGZbs{}}\PYG{l+s+s1}{adult\PYGZus{}work}\PYG{l+s+s1}{\PYGZsq{}}\PYG{p}{,}
       \PYG{n}{dmg}\PYG{o}{.}\PYG{n}{ADULT\PYGZus{}NON\PYGZus{}WORK}\PYG{p}{:} \PYG{n}{fpath} \PYG{o}{+} \PYG{l+s+s1}{\PYGZsq{}}\PYG{l+s+se}{\PYGZbs{}\PYGZbs{}}\PYG{l+s+s1}{adult\PYGZus{}non\PYGZus{}work}\PYG{l+s+s1}{\PYGZsq{}}\PYG{p}{,}
       \PYG{n}{dmg}\PYG{o}{.}\PYG{n}{CHILD\PYGZus{}SCHOOL}\PYG{p}{:} \PYG{n}{fpath} \PYG{o}{+} \PYG{l+s+s1}{\PYGZsq{}}\PYG{l+s+se}{\PYGZbs{}\PYGZbs{}}\PYG{l+s+s1}{child\PYGZus{}school}\PYG{l+s+s1}{\PYGZsq{}}\PYG{p}{,}
       \PYG{n}{dmg}\PYG{o}{.}\PYG{n}{CHILD\PYGZus{}YOUNG}\PYG{p}{:} \PYG{n}{fpath} \PYG{o}{+} \PYG{l+s+s1}{\PYGZsq{}}\PYG{l+s+se}{\PYGZbs{}\PYGZbs{}}\PYG{l+s+s1}{child\PYGZus{}young}\PYG{l+s+s1}{\PYGZsq{}}\PYG{p}{,}
      \PYG{p}{\PYGZcb{}}

\PYG{n}{fpath\PYGZus{}figure\PYGZus{}save} \PYG{o}{=} \PYG{n}{chooser\PYGZus{}fin}\PYG{p}{[}\PYG{n}{demo}\PYG{p}{]}

\PYG{c+c1}{\PYGZsh{} print the save figure directory}
\PYG{n+nb}{print}\PYG{p}{(}\PYG{l+s+s1}{\PYGZsq{}}\PYG{l+s+s1}{the figure save path:}\PYG{l+s+se}{\PYGZbs{}t}\PYG{l+s+si}{\PYGZpc{}s}\PYG{l+s+s1}{\PYGZsq{}} \PYG{o}{\PYGZpc{}} \PYG{n}{fpath\PYGZus{}figure\PYGZus{}save}\PYG{p}{)}

\PYG{c+c1}{\PYGZsh{} different sets of activitiy data to plot}
\PYG{n}{keys\PYGZus{}all} \PYG{o}{=} \PYG{n}{chad\PYGZus{}demo}\PYG{o}{.}\PYG{n}{keys}

\PYG{c+c1}{\PYGZsh{} eating activities}
\PYG{n}{keys\PYGZus{}eat} \PYG{o}{=} \PYG{p}{[}\PYG{n}{mg}\PYG{o}{.}\PYG{n}{KEY\PYGZus{}EAT\PYGZus{}BREAKFAST}\PYG{p}{,} \PYG{n}{mg}\PYG{o}{.}\PYG{n}{KEY\PYGZus{}EAT\PYGZus{}LUNCH}\PYG{p}{,} \PYG{n}{mg}\PYG{o}{.}\PYG{n}{KEY\PYGZus{}EAT\PYGZus{}DINNER}\PYG{p}{]}

\PYG{c+c1}{\PYGZsh{} non eating activities}
\PYG{n}{keys\PYGZus{}not\PYGZus{}eat} \PYG{o}{=} \PYG{p}{[} \PYG{n}{k} \PYG{k}{for} \PYG{n}{k} \PYG{o+ow}{in} \PYG{n}{keys\PYGZus{}all} \PYG{k}{if} \PYG{n}{k} \PYG{o+ow}{not} \PYG{o+ow}{in} \PYG{n}{keys\PYGZus{}eat} \PYG{p}{]}
\end{sphinxVerbatim}
\begin{sphinxalltt}
the figure save path:       ..my\_datafig12\_07\_2017n8192\_d364child\_young
\end{sphinxalltt}

Load plotting data

\fvset{hllines={, ,}}%
\begin{sphinxVerbatim}[commandchars=\\\{\}]
\PYG{n}{DO\PYGZus{}ALL} \PYG{o}{=} \PYG{l+m+mi}{1}
\PYG{n}{DO\PYGZus{}MEALS} \PYG{o}{=} \PYG{l+m+mi}{2}
\PYG{n}{DO\PYGZus{}NOT\PYGZus{}MEALS} \PYG{o}{=} \PYG{l+m+mi}{3}

\PYG{c+c1}{\PYGZsh{} (the activites to plot, part of the file name that matches the keys)}
\PYG{n}{chooser\PYGZus{}keys} \PYG{o}{=} \PYG{p}{\PYGZob{}} \PYG{n}{DO\PYGZus{}ALL}\PYG{p}{:} \PYG{p}{(}\PYG{n}{keys\PYGZus{}all}\PYG{p}{,} \PYG{l+s+s1}{\PYGZsq{}}\PYG{l+s+s1}{all}\PYG{l+s+s1}{\PYGZsq{}}\PYG{p}{)}\PYG{p}{,} \PYGZbs{}
                \PYG{n}{DO\PYGZus{}MEALS}\PYG{p}{:} \PYG{p}{(}\PYG{n}{keys\PYGZus{}eat}\PYG{p}{,} \PYG{l+s+s1}{\PYGZsq{}}\PYG{l+s+s1}{meals}\PYG{l+s+s1}{\PYGZsq{}}\PYG{p}{)}\PYG{p}{,}\PYGZbs{}
                \PYG{n}{DO\PYGZus{}NOT\PYGZus{}MEALS}\PYG{p}{:} \PYG{p}{(}\PYG{n}{keys\PYGZus{}not\PYGZus{}eat}\PYG{p}{,} \PYG{l+s+s1}{\PYGZsq{}}\PYG{l+s+s1}{not\PYGZus{}meals}\PYG{l+s+s1}{\PYGZsq{}}\PYG{p}{)}\PYG{p}{,}
               \PYG{p}{\PYGZcb{}}
\end{sphinxVerbatim}

\fvset{hllines={, ,}}%
\begin{sphinxVerbatim}[commandchars=\\\{\}]
\PYG{c+c1}{\PYGZsh{}}
\PYG{c+c1}{\PYGZsh{} set the activities to plot}
\PYG{c+c1}{\PYGZsh{}}
\PYG{n}{plot\PYGZus{}keys} \PYG{o}{=} \PYG{n}{DO\PYGZus{}ALL}

\PYG{n}{keys}\PYG{p}{,} \PYG{n}{fname\PYGZus{}keys} \PYG{o}{=} \PYG{n}{chooser\PYGZus{}keys}\PYG{p}{[}\PYG{n}{plot\PYGZus{}keys}\PYG{p}{]}
\PYG{n}{name\PYGZus{}keys} \PYG{o}{=} \PYG{p}{[} \PYG{n}{activity}\PYG{o}{.}\PYG{n}{INT\PYGZus{}2\PYGZus{}STR}\PYG{p}{[}\PYG{n}{k}\PYG{p}{]} \PYG{k}{for} \PYG{n}{k} \PYG{o+ow}{in} \PYG{n}{keys}\PYG{p}{]}


\PYG{c+c1}{\PYGZsh{} labels on the right hand side of the plot}
\PYG{n}{ylabels} \PYG{o}{=} \PYG{p}{[}\PYG{l+s+s1}{\PYGZsq{}}\PYG{l+s+s1}{Start Time}\PYG{l+s+s1}{\PYGZsq{}}\PYG{p}{,} \PYG{l+s+s1}{\PYGZsq{}}\PYG{l+s+s1}{End Time}\PYG{l+s+s1}{\PYGZsq{}}\PYG{p}{,} \PYG{l+s+s1}{\PYGZsq{}}\PYG{l+s+s1}{Duration}\PYG{l+s+s1}{\PYGZsq{}}\PYG{p}{]}
\end{sphinxVerbatim}

Load all data

\fvset{hllines={, ,}}%
\begin{sphinxVerbatim}[commandchars=\\\{\}]
\PYG{c+c1}{\PYGZsh{} choose the activities to plot}

\PYG{c+c1}{\PYGZsh{} get the figure directories}
\PYG{n}{fpaths} \PYG{o}{=} \PYG{p}{[} \PYG{p}{(}\PYG{n}{fpath\PYGZus{}figure\PYGZus{}save} \PYG{o}{+} \PYG{n}{mg}\PYG{o}{.}\PYG{n}{KEY\PYGZus{}2\PYGZus{}FDIR\PYGZus{}SAVE\PYGZus{}FIG}\PYG{p}{[}\PYG{n}{k}\PYG{p}{]} \PYG{o}{+} \PYG{n}{mg}\PYG{o}{.}\PYG{n}{FDIR\PYGZus{}SAVE\PYGZus{}FIG\PYGZus{}RANDOM\PYGZus{}DAY}\PYG{p}{)} \PYG{k}{for} \PYG{n}{k} \PYG{o+ow}{in} \PYG{n}{keys}\PYG{p}{]}

\PYG{c+c1}{\PYGZsh{} the file name (no file path) of the data to save}
\PYG{n}{fname} \PYG{o}{=} \PYG{n}{fpath\PYGZus{}figure\PYGZus{}save} \PYG{o}{+} \PYG{l+s+s1}{\PYGZsq{}}\PYG{l+s+se}{\PYGZbs{}\PYGZbs{}}\PYG{l+s+s1}{cdf\PYGZus{}inv\PYGZus{}}\PYG{l+s+s1}{\PYGZsq{}} \PYG{o}{+} \PYG{n}{fname\PYGZus{}keys} \PYG{o}{+} \PYG{l+s+s1}{\PYGZsq{}}\PYG{l+s+s1}{.png}\PYG{l+s+s1}{\PYGZsq{}}

\PYG{c+c1}{\PYGZsh{} file name to load}
\PYG{n}{fnames\PYGZus{}load} \PYG{o}{=} \PYG{p}{(}\PYG{l+s+s1}{\PYGZsq{}}\PYG{l+s+se}{\PYGZbs{}\PYGZbs{}}\PYG{l+s+s1}{cdf\PYGZus{}inv\PYGZus{}start.pkl}\PYG{l+s+s1}{\PYGZsq{}}\PYG{p}{,} \PYG{l+s+s1}{\PYGZsq{}}\PYG{l+s+se}{\PYGZbs{}\PYGZbs{}}\PYG{l+s+s1}{cdf\PYGZus{}inv\PYGZus{}end.pkl}\PYG{l+s+s1}{\PYGZsq{}}\PYG{p}{,} \PYG{l+s+s1}{\PYGZsq{}}\PYG{l+s+se}{\PYGZbs{}\PYGZbs{}}\PYG{l+s+s1}{cdf\PYGZus{}inv\PYGZus{}dt.pkl}\PYG{l+s+s1}{\PYGZsq{}}\PYG{p}{)}

\PYG{c+c1}{\PYGZsh{} load the data}
\PYG{n}{data\PYGZus{}list\PYGZus{}all}\PYG{p}{,} \PYG{n}{fname\PYGZus{}subplot} \PYG{o}{=} \PYG{n}{plotter}\PYG{o}{.}\PYG{n}{get\PYGZus{}figure\PYGZus{}data}\PYG{p}{(}\PYG{n}{fpaths}\PYG{p}{,} \PYG{n}{fpath\PYGZus{}figure\PYGZus{}save}\PYG{p}{,} \PYG{n}{fname}\PYG{p}{,} \PYG{n}{fnames\PYGZus{}load}\PYG{o}{=}\PYG{n}{fnames\PYGZus{}load}\PYG{p}{)}
\end{sphinxVerbatim}

Load the data for a specific activity-data

\fvset{hllines={, ,}}%
\begin{sphinxVerbatim}[commandchars=\\\{\}]
\PYG{n}{idx} \PYG{o}{=} \PYG{o}{\PYGZhy{}}\PYG{l+m+mi}{1}
\PYG{n}{start} \PYG{o}{=} \PYG{n}{data\PYGZus{}list\PYGZus{}start}\PYG{p}{[}\PYG{n}{idx}\PYG{p}{]}
\PYG{n}{end} \PYG{o}{=} \PYG{n}{data\PYGZus{}list\PYGZus{}end}\PYG{p}{[}\PYG{n}{idx}\PYG{p}{]}
\PYG{n}{dt} \PYG{o}{=} \PYG{n}{data\PYGZus{}list\PYGZus{}dt}\PYG{p}{[}\PYG{n}{idx}\PYG{p}{]}

\PYG{n}{f\PYGZus{}end} \PYG{o}{=} \PYG{n}{fnames\PYGZus{}end}\PYG{p}{[}\PYG{n}{idx}\PYG{p}{]}
\PYG{n}{f\PYGZus{}start} \PYG{o}{=} \PYG{n}{fnames\PYGZus{}start}\PYG{p}{[}\PYG{n}{idx}\PYG{p}{]}
\PYG{n}{f\PYGZus{}dt} \PYG{o}{=} \PYG{n}{fnames\PYGZus{}dt}\PYG{p}{[}\PYG{n}{idx}\PYG{p}{]}

\PYG{n+nb}{print}\PYG{p}{(}\PYG{n}{f\PYGZus{}start}\PYG{p}{)}
\PYG{n+nb}{print}\PYG{p}{(}\PYG{n}{f\PYGZus{}end}\PYG{p}{)}
\PYG{n+nb}{print}\PYG{p}{(}\PYG{n}{f\PYGZus{}dt}\PYG{p}{)}
\end{sphinxVerbatim}
\begin{sphinxalltt}
..my\_datafig12\_07\_2017n8192\_d364child\_youngsleeprandom\_daycdf\_inv\_start.pkl
..my\_datafig12\_07\_2017n8192\_d364child\_youngsleeprandom\_daycdf\_inv\_end.pkl
..my\_datafig12\_07\_2017n8192\_d364child\_youngsleeprandom\_daycdf\_inv\_dt.pkl
\end{sphinxalltt}

plot the residuals

\fvset{hllines={, ,}}%
\begin{sphinxVerbatim}[commandchars=\\\{\}]
\PYG{c+c1}{\PYGZsh{}}
\PYG{c+c1}{\PYGZsh{} plot the residuals}
\PYG{c+c1}{\PYGZsh{}}

\PYG{n}{alpha} \PYG{o}{=} \PYG{l+m+mf}{0.05}
\PYG{n}{plt}\PYG{o}{.}\PYG{n}{close}\PYG{p}{(}\PYG{l+s+s1}{\PYGZsq{}}\PYG{l+s+s1}{all}\PYG{l+s+s1}{\PYGZsq{}}\PYG{p}{)}

\PYG{k}{for} \PYG{n}{idx}\PYG{p}{,} \PYG{n}{k} \PYG{o+ow}{in} \PYG{n+nb}{enumerate}\PYG{p}{(}\PYG{n}{keys}\PYG{p}{)}\PYG{p}{:}

    \PYG{n+nb}{print}\PYG{p}{(} \PYG{n}{activity}\PYG{o}{.}\PYG{n}{INT\PYGZus{}2\PYGZus{}STR}\PYG{p}{[}\PYG{n}{k}\PYG{p}{]} \PYG{p}{)}

    \PYG{c+c1}{\PYGZsh{} load the start time, end time, and duration data}
    \PYG{n}{start} \PYG{o}{=} \PYG{n}{data\PYGZus{}list\PYGZus{}start}\PYG{p}{[}\PYG{n}{idx}\PYG{p}{]}
    \PYG{n}{end} \PYG{o}{=} \PYG{n}{data\PYGZus{}list\PYGZus{}end}\PYG{p}{[}\PYG{n}{idx}\PYG{p}{]}
    \PYG{n}{dt} \PYG{o}{=} \PYG{n}{data\PYGZus{}list\PYGZus{}dt}\PYG{p}{[}\PYG{n}{idx}\PYG{p}{]}

    \PYG{c+c1}{\PYGZsh{} quantile, and residual data}
    \PYG{n}{x\PYGZus{}start}\PYG{p}{,} \PYG{n}{r\PYGZus{}start} \PYG{o}{=} \PYG{n}{f}\PYG{p}{(}\PYG{n}{start}\PYG{p}{,} \PYG{n}{alpha}\PYG{o}{=}\PYG{n}{alpha}\PYG{p}{)}
    \PYG{n}{x\PYGZus{}end}\PYG{p}{,} \PYG{n}{r\PYGZus{}end} \PYG{o}{=} \PYG{n}{f}\PYG{p}{(}\PYG{n}{end}\PYG{p}{,} \PYG{n}{alpha}\PYG{o}{=}\PYG{n}{alpha}\PYG{p}{)}
    \PYG{n}{x\PYGZus{}dt}\PYG{p}{,} \PYG{n}{r\PYGZus{}dt} \PYG{o}{=} \PYG{n}{f}\PYG{p}{(}\PYG{n}{dt}\PYG{p}{,} \PYG{n}{alpha}\PYG{o}{=}\PYG{n}{alpha}\PYG{p}{)}

    \PYG{c+c1}{\PYGZsh{} covert the residuals into minutes}
    \PYG{n}{r\PYGZus{}start} \PYG{o}{=} \PYG{n}{r\PYGZus{}start} \PYG{o}{*} \PYG{l+m+mi}{60}
    \PYG{n}{r\PYGZus{}end} \PYG{o}{=} \PYG{n}{r\PYGZus{}end} \PYG{o}{*} \PYG{l+m+mi}{60}
    \PYG{n}{r\PYGZus{}dt} \PYG{o}{=} \PYG{n}{r\PYGZus{}dt}

    \PYG{c+c1}{\PYGZsh{} get the moments on the residuals for start time, end time, and duration}
    \PYG{n}{mu\PYGZus{}start}\PYG{p}{,} \PYG{n}{std\PYGZus{}start} \PYG{o}{=} \PYG{n}{get\PYGZus{}moments}\PYG{p}{(}\PYG{n}{r\PYGZus{}start}\PYG{p}{)}
    \PYG{n}{mu\PYGZus{}end}\PYG{p}{,} \PYG{n}{std\PYGZus{}end} \PYG{o}{=} \PYG{n}{get\PYGZus{}moments}\PYG{p}{(}\PYG{n}{r\PYGZus{}end}\PYG{p}{)}
    \PYG{n}{mu\PYGZus{}dt}\PYG{p}{,} \PYG{n}{std\PYGZus{}dt} \PYG{o}{=} \PYG{n}{get\PYGZus{}moments}\PYG{p}{(}\PYG{n}{r\PYGZus{}dt}\PYG{p}{)}

    \PYG{n+nb}{print}\PYG{p}{(}\PYG{l+s+s1}{\PYGZsq{}}\PYG{l+s+s1}{mu start: }\PYG{l+s+si}{\PYGZpc{}.2f}\PYG{l+s+se}{\PYGZbs{}t}\PYG{l+s+se}{\PYGZbs{}t}\PYG{l+s+s1}{std start: }\PYG{l+s+si}{\PYGZpc{}.2f}\PYG{l+s+s1}{\PYGZsq{}} \PYG{o}{\PYGZpc{}} \PYG{p}{(}\PYG{n}{mu\PYGZus{}start}\PYG{p}{,} \PYG{n}{std\PYGZus{}start}\PYG{p}{)}\PYG{p}{)}
    \PYG{n+nb}{print}\PYG{p}{(}\PYG{l+s+s1}{\PYGZsq{}}\PYG{l+s+s1}{mu end: }\PYG{l+s+si}{\PYGZpc{}.2f}\PYG{l+s+se}{\PYGZbs{}t}\PYG{l+s+se}{\PYGZbs{}t}\PYG{l+s+s1}{std end: }\PYG{l+s+si}{\PYGZpc{}.2f}\PYG{l+s+s1}{\PYGZsq{}} \PYG{o}{\PYGZpc{}} \PYG{p}{(}\PYG{n}{mu\PYGZus{}end}\PYG{p}{,} \PYG{n}{std\PYGZus{}end}\PYG{p}{)}\PYG{p}{)}
    \PYG{n+nb}{print}\PYG{p}{(}\PYG{l+s+s1}{\PYGZsq{}}\PYG{l+s+s1}{mu dt: }\PYG{l+s+si}{\PYGZpc{}.2f}\PYG{l+s+se}{\PYGZbs{}t}\PYG{l+s+se}{\PYGZbs{}t}\PYG{l+s+s1}{std dt: }\PYG{l+s+si}{\PYGZpc{}.2f}\PYG{l+s+se}{\PYGZbs{}n}\PYG{l+s+s1}{\PYGZsq{}} \PYG{o}{\PYGZpc{}} \PYG{p}{(}\PYG{n}{mu\PYGZus{}dt}\PYG{p}{,} \PYG{n}{std\PYGZus{}dt}\PYG{p}{)}\PYG{p}{)}

    \PYG{c+c1}{\PYGZsh{} create subplots}
    \PYG{n}{fig}\PYG{p}{,} \PYG{n}{axes} \PYG{o}{=} \PYG{n}{plt}\PYG{o}{.}\PYG{n}{subplots}\PYG{p}{(}\PYG{l+m+mi}{3}\PYG{p}{)}

    \PYG{c+c1}{\PYGZsh{} create title}
    \PYG{n}{fig}\PYG{o}{.}\PYG{n}{suptitle}\PYG{p}{(} \PYG{n}{activity}\PYG{o}{.}\PYG{n}{INT\PYGZus{}2\PYGZus{}STR}\PYG{p}{[}\PYG{n}{k}\PYG{p}{]} \PYG{p}{)}

    \PYG{c+c1}{\PYGZsh{} plot data about start time}
    \PYG{n}{ax} \PYG{o}{=} \PYG{n}{axes}\PYG{p}{[}\PYG{l+m+mi}{0}\PYG{p}{]}
    \PYG{n}{ax}\PYG{o}{.}\PYG{n}{plot}\PYG{p}{(}\PYG{n}{x\PYGZus{}start}\PYG{p}{,} \PYG{n}{r\PYGZus{}start}\PYG{p}{,} \PYG{n}{label}\PYG{o}{=}\PYG{l+s+s1}{\PYGZsq{}}\PYG{l+s+s1}{start}\PYG{l+s+s1}{\PYGZsq{}}\PYG{p}{)}
    \PYG{n}{ax}\PYG{o}{.}\PYG{n}{axhline}\PYG{p}{(}\PYG{n}{mu\PYGZus{}start}\PYG{p}{,} \PYG{n}{ls}\PYG{o}{=}\PYG{l+s+s1}{\PYGZsq{}}\PYG{l+s+s1}{\PYGZhy{}\PYGZhy{}}\PYG{l+s+s1}{\PYGZsq{}}\PYG{p}{)}
    \PYG{n}{ax}\PYG{o}{.}\PYG{n}{legend}\PYG{p}{(}\PYG{n}{loc}\PYG{o}{=}\PYG{l+s+s1}{\PYGZsq{}}\PYG{l+s+s1}{best}\PYG{l+s+s1}{\PYGZsq{}}\PYG{p}{)}

    \PYG{c+c1}{\PYGZsh{} plot data about end time}
    \PYG{n}{ax} \PYG{o}{=} \PYG{n}{axes}\PYG{p}{[}\PYG{l+m+mi}{1}\PYG{p}{]}
    \PYG{n}{ax}\PYG{o}{.}\PYG{n}{plot}\PYG{p}{(}\PYG{n}{x\PYGZus{}end}\PYG{p}{,} \PYG{n}{r\PYGZus{}end}\PYG{p}{,} \PYG{n}{label}\PYG{o}{=}\PYG{l+s+s1}{\PYGZsq{}}\PYG{l+s+s1}{end}\PYG{l+s+s1}{\PYGZsq{}}\PYG{p}{)}
    \PYG{n}{ax}\PYG{o}{.}\PYG{n}{axhline}\PYG{p}{(}\PYG{n}{mu\PYGZus{}end}\PYG{p}{,} \PYG{n}{ls}\PYG{o}{=}\PYG{l+s+s1}{\PYGZsq{}}\PYG{l+s+s1}{\PYGZhy{}\PYGZhy{}}\PYG{l+s+s1}{\PYGZsq{}}\PYG{p}{)}
    \PYG{n}{ax}\PYG{o}{.}\PYG{n}{legend}\PYG{p}{(}\PYG{n}{loc}\PYG{o}{=}\PYG{l+s+s1}{\PYGZsq{}}\PYG{l+s+s1}{best}\PYG{l+s+s1}{\PYGZsq{}}\PYG{p}{)}

    \PYG{c+c1}{\PYGZsh{} plot data about duration}
    \PYG{n}{ax} \PYG{o}{=} \PYG{n}{axes}\PYG{p}{[}\PYG{l+m+mi}{2}\PYG{p}{]}
    \PYG{n}{ax}\PYG{o}{.}\PYG{n}{plot}\PYG{p}{(}\PYG{n}{x\PYGZus{}dt}\PYG{p}{,} \PYG{n}{r\PYGZus{}dt}\PYG{p}{,} \PYG{n}{label}\PYG{o}{=}\PYG{l+s+s1}{\PYGZsq{}}\PYG{l+s+s1}{dt}\PYG{l+s+s1}{\PYGZsq{}}\PYG{p}{)}
    \PYG{n}{ax}\PYG{o}{.}\PYG{n}{axhline}\PYG{p}{(}\PYG{n}{mu\PYGZus{}dt}\PYG{p}{,} \PYG{n}{ls}\PYG{o}{=}\PYG{l+s+s1}{\PYGZsq{}}\PYG{l+s+s1}{\PYGZhy{}\PYGZhy{}}\PYG{l+s+s1}{\PYGZsq{}}\PYG{p}{)}
    \PYG{n}{ax}\PYG{o}{.}\PYG{n}{legend}\PYG{p}{(}\PYG{n}{loc}\PYG{o}{=}\PYG{l+s+s1}{\PYGZsq{}}\PYG{l+s+s1}{best}\PYG{l+s+s1}{\PYGZsq{}}\PYG{p}{)}

\PYG{n}{plt}\PYG{o}{.}\PYG{n}{show}\PYG{p}{(}\PYG{p}{)}
\end{sphinxVerbatim}

\fvset{hllines={, ,}}%
\begin{sphinxVerbatim}[commandchars=\\\{\}]
\PYG{n}{Eat} \PYG{n}{Breakfast}
\PYG{n}{mu} \PYG{n}{start}\PYG{p}{:} \PYG{l+m+mf}{11.83}             \PYG{n}{std} \PYG{n}{start}\PYG{p}{:} \PYG{l+m+mf}{8.87}
\PYG{n}{mu} \PYG{n}{end}\PYG{p}{:} \PYG{l+m+mf}{8.20}                \PYG{n}{std} \PYG{n}{end}\PYG{p}{:} \PYG{l+m+mf}{9.31}
\PYG{n}{mu} \PYG{n}{dt}\PYG{p}{:} \PYG{l+m+mf}{3.79}         \PYG{n}{std} \PYG{n}{dt}\PYG{p}{:} \PYG{l+m+mf}{4.17}

\PYG{n}{Eat} \PYG{n}{Lunch}
\PYG{n}{mu} \PYG{n}{start}\PYG{p}{:} \PYG{l+m+mf}{12.39}             \PYG{n}{std} \PYG{n}{start}\PYG{p}{:} \PYG{l+m+mf}{8.78}
\PYG{n}{mu} \PYG{n}{end}\PYG{p}{:} \PYG{l+m+mf}{14.46}               \PYG{n}{std} \PYG{n}{end}\PYG{p}{:} \PYG{l+m+mf}{7.60}
\PYG{n}{mu} \PYG{n}{dt}\PYG{p}{:} \PYG{l+m+mf}{2.10}         \PYG{n}{std} \PYG{n}{dt}\PYG{p}{:} \PYG{l+m+mf}{1.56}

\PYG{n}{Eat} \PYG{n}{Dinner}
\PYG{n}{mu} \PYG{n}{start}\PYG{p}{:} \PYG{l+m+mf}{7.21}              \PYG{n}{std} \PYG{n}{start}\PYG{p}{:} \PYG{l+m+mf}{5.18}
\PYG{n}{mu} \PYG{n}{end}\PYG{p}{:} \PYG{l+m+mf}{8.86}                \PYG{n}{std} \PYG{n}{end}\PYG{p}{:} \PYG{l+m+mf}{4.73}
\PYG{n}{mu} \PYG{n}{dt}\PYG{p}{:} \PYG{l+m+mf}{3.24}         \PYG{n}{std} \PYG{n}{dt}\PYG{p}{:} \PYG{l+m+mf}{2.95}

\PYG{n}{Sleep}
\PYG{n}{mu} \PYG{n}{start}\PYG{p}{:} \PYG{l+m+mf}{5.94}              \PYG{n}{std} \PYG{n}{start}\PYG{p}{:} \PYG{l+m+mf}{4.78}
\PYG{n}{mu} \PYG{n}{end}\PYG{p}{:} \PYG{l+m+mf}{5.88}                \PYG{n}{std} \PYG{n}{end}\PYG{p}{:} \PYG{l+m+mf}{5.57}
\PYG{n}{mu} \PYG{n}{dt}\PYG{p}{:} \PYG{l+m+mf}{13.44}                \PYG{n}{std} \PYG{n}{dt}\PYG{p}{:} \PYG{l+m+mf}{10.27}
\end{sphinxVerbatim}


\subsection{longitude\_plot notebook}
\label{\detokenize{longitude_plot::doc}}\label{\detokenize{longitude_plot:longitude-plot-notebook}}
\fvset{hllines={, ,}}%
\begin{sphinxVerbatim}[commandchars=\\\{\}]
\PYG{c+c1}{\PYGZsh{} The United States Environmental Protection Agency through its Office of}
\PYG{c+c1}{\PYGZsh{} Research and Development has developed this software. The code is made}
\PYG{c+c1}{\PYGZsh{} publicly available to better communicate the research. All input data}
\PYG{c+c1}{\PYGZsh{} used fora given application should be reviewed by the researcher so}
\PYG{c+c1}{\PYGZsh{} that the model results are based on appropriate data for any given}
\PYG{c+c1}{\PYGZsh{} application. This model is under continued development. The model and}
\PYG{c+c1}{\PYGZsh{} data included herein do not represent and should not be construed to}
\PYG{c+c1}{\PYGZsh{} represent any Agency determination or policy.}
\PYG{c+c1}{\PYGZsh{}}
\PYG{c+c1}{\PYGZsh{} This file was written by Dr. Namdi Brandon}
\PYG{c+c1}{\PYGZsh{} ORCID: 0000\PYGZhy{}0001\PYGZhy{}7050\PYGZhy{}1538}
\PYG{c+c1}{\PYGZsh{} March 20, 2018}
\end{sphinxVerbatim}

This module plots the daily activity-duration for each activity over
time done by an agent in an Agent-Based Module of Human Activity
Patterns (ABMHAP) simulation. An agent representing each demographic are
shown in a combined subplot:
\begin{enumerate}
\item {} 
An agent representing a respective demographic has its activity
behavior is plotted in a log10 scale over time

\item {} 
This function plots a histogram showing the amount of times each
activity was done in an ABMHAP simulation.

\end{enumerate}

import

\fvset{hllines={, ,}}%
\begin{sphinxVerbatim}[commandchars=\\\{\}]
\PYG{k+kn}{import} \PYG{n+nn}{os}\PYG{o}{,} \PYG{n+nn}{sys}
\PYG{n}{sys}\PYG{o}{.}\PYG{n}{path}\PYG{o}{.}\PYG{n}{append}\PYG{p}{(}\PYG{l+s+s1}{\PYGZsq{}}\PYG{l+s+s1}{..}\PYG{l+s+se}{\PYGZbs{}\PYGZbs{}}\PYG{l+s+s1}{source}\PYG{l+s+s1}{\PYGZsq{}}\PYG{p}{)}
\PYG{n}{sys}\PYG{o}{.}\PYG{n}{path}\PYG{o}{.}\PYG{n}{append}\PYG{p}{(}\PYG{l+s+s1}{\PYGZsq{}}\PYG{l+s+s1}{..}\PYG{l+s+se}{\PYGZbs{}\PYGZbs{}}\PYG{l+s+s1}{processing}\PYG{l+s+s1}{\PYGZsq{}}\PYG{p}{)}
\PYG{n}{sys}\PYG{o}{.}\PYG{n}{path}\PYG{o}{.}\PYG{n}{append}\PYG{p}{(}\PYG{l+s+s1}{\PYGZsq{}}\PYG{l+s+s1}{..}\PYG{l+s+se}{\PYGZbs{}\PYGZbs{}}\PYG{l+s+s1}{plotting}\PYG{l+s+s1}{\PYGZsq{}}\PYG{p}{)}

\PYG{c+c1}{\PYGZsh{} plotting capability}
\PYG{k+kn}{import} \PYG{n+nn}{matplotlib}\PYG{n+nn}{.}\PYG{n+nn}{pylab} \PYG{k}{as} \PYG{n+nn}{plt}

\PYG{c+c1}{\PYGZsh{} math capabilitiy}
\PYG{k+kn}{import} \PYG{n+nn}{numpy} \PYG{k}{as} \PYG{n+nn}{np}

\PYG{c+c1}{\PYGZsh{} dataframe capability}
\PYG{k+kn}{import} \PYG{n+nn}{pandas} \PYG{k}{as} \PYG{n+nn}{pd}

\PYG{c+c1}{\PYGZsh{} ABMHAP capability}
\PYG{k+kn}{import} \PYG{n+nn}{my\PYGZus{}globals} \PYG{k}{as} \PYG{n+nn}{mg}
\PYG{k+kn}{import} \PYG{n+nn}{chad\PYGZus{}demography\PYGZus{}adult\PYGZus{}non\PYGZus{}work} \PYG{k}{as} \PYG{n+nn}{cdanw}
\PYG{k+kn}{import} \PYG{n+nn}{chad\PYGZus{}demography\PYGZus{}adult\PYGZus{}work} \PYG{k}{as} \PYG{n+nn}{cdaw}
\PYG{k+kn}{import} \PYG{n+nn}{chad\PYGZus{}demography\PYGZus{}child\PYGZus{}school} \PYG{k}{as} \PYG{n+nn}{cdcs}
\PYG{k+kn}{import} \PYG{n+nn}{chad\PYGZus{}demography\PYGZus{}child\PYGZus{}young} \PYG{k}{as} \PYG{n+nn}{cdcy}
\PYG{k+kn}{import} \PYG{n+nn}{demography} \PYG{k}{as} \PYG{n+nn}{dmg}

\PYG{k+kn}{import} \PYG{n+nn}{activity}\PYG{o}{,} \PYG{n+nn}{plotter}\PYG{o}{,} \PYG{n+nn}{temporal}
\end{sphinxVerbatim}

\fvset{hllines={, ,}}%
\begin{sphinxVerbatim}[commandchars=\\\{\}]
\PYG{o}{\PYGZpc{}}\PYG{k}{matplotlib} auto
\end{sphinxVerbatim}

\fvset{hllines={, ,}}%
\begin{sphinxVerbatim}[commandchars=\\\{\}]
\PYG{n}{Using} \PYG{n}{matplotlib} \PYG{n}{backend}\PYG{p}{:} \PYG{n}{Qt5Agg}
\end{sphinxVerbatim}

run

\fvset{hllines={, ,}}%
\begin{sphinxVerbatim}[commandchars=\\\{\}]
\PYG{c+c1}{\PYGZsh{}}
\PYG{c+c1}{\PYGZsh{} get the file name}
\PYG{c+c1}{\PYGZsh{}}

\PYG{c+c1}{\PYGZsh{} variation}
\PYG{n}{fpath} \PYG{o}{=} \PYG{n}{mg}\PYG{o}{.}\PYG{n}{FDIR\PYGZus{}MY\PYGZus{}DATA}

\PYG{c+c1}{\PYGZsh{} file paths for each demographic}
\PYG{n}{fpath\PYGZus{}adult\PYGZus{}work} \PYG{o}{=} \PYG{n}{fpath} \PYG{o}{+} \PYG{l+s+s1}{\PYGZsq{}}\PYG{l+s+se}{\PYGZbs{}\PYGZbs{}}\PYG{l+s+s1}{11\PYGZus{}21\PYGZus{}2017}\PYG{l+s+se}{\PYGZbs{}\PYGZbs{}}\PYG{l+s+s1}{n8192\PYGZus{}d364}\PYG{l+s+s1}{\PYGZsq{}}
\PYG{n}{fpath\PYGZus{}adult\PYGZus{}non\PYGZus{}work} \PYG{o}{=} \PYG{n}{fpath} \PYG{o}{+} \PYG{l+s+s1}{\PYGZsq{}}\PYG{l+s+se}{\PYGZbs{}\PYGZbs{}}\PYG{l+s+s1}{11\PYGZus{}27\PYGZus{}2017}\PYG{l+s+se}{\PYGZbs{}\PYGZbs{}}\PYG{l+s+s1}{n8192\PYGZus{}d364}\PYG{l+s+s1}{\PYGZsq{}}
\PYG{n}{fpath\PYGZus{}child\PYGZus{}school} \PYG{o}{=} \PYG{n}{fpath} \PYG{o}{+} \PYG{l+s+s1}{\PYGZsq{}}\PYG{l+s+se}{\PYGZbs{}\PYGZbs{}}\PYG{l+s+s1}{11\PYGZus{}29\PYGZus{}2017}\PYG{l+s+se}{\PYGZbs{}\PYGZbs{}}\PYG{l+s+s1}{n8192\PYGZus{}d364}\PYG{l+s+s1}{\PYGZsq{}}
\PYG{n}{fpath\PYGZus{}child\PYGZus{}young} \PYG{o}{=} \PYG{n}{fpath} \PYG{o}{+} \PYG{l+s+s1}{\PYGZsq{}}\PYG{l+s+se}{\PYGZbs{}\PYGZbs{}}\PYG{l+s+s1}{12\PYGZus{}07\PYGZus{}2017}\PYG{l+s+se}{\PYGZbs{}\PYGZbs{}}\PYG{l+s+s1}{n8192\PYGZus{}d364}\PYG{l+s+s1}{\PYGZsq{}}

\PYG{c+c1}{\PYGZsh{} full file names for each demographic}
\PYG{n}{fname\PYGZus{}adult\PYGZus{}work} \PYG{o}{=} \PYG{n}{fpath\PYGZus{}adult\PYGZus{}work} \PYG{o}{+} \PYG{l+s+s1}{\PYGZsq{}}\PYG{l+s+se}{\PYGZbs{}\PYGZbs{}}\PYG{l+s+s1}{data\PYGZus{}adult\PYGZus{}work.pkl}\PYG{l+s+s1}{\PYGZsq{}}
\PYG{n}{fname\PYGZus{}adult\PYGZus{}non\PYGZus{}work} \PYG{o}{=} \PYG{n}{fpath\PYGZus{}adult\PYGZus{}non\PYGZus{}work} \PYG{o}{+} \PYG{l+s+s1}{\PYGZsq{}}\PYG{l+s+se}{\PYGZbs{}\PYGZbs{}}\PYG{l+s+s1}{data\PYGZus{}adult\PYGZus{}non\PYGZus{}work.pkl}\PYG{l+s+s1}{\PYGZsq{}}
\PYG{n}{fname\PYGZus{}child\PYGZus{}school} \PYG{o}{=} \PYG{n}{fpath\PYGZus{}child\PYGZus{}school} \PYG{o}{+} \PYG{l+s+s1}{\PYGZsq{}}\PYG{l+s+se}{\PYGZbs{}\PYGZbs{}}\PYG{l+s+s1}{data\PYGZus{}child\PYGZus{}school.pkl}\PYG{l+s+s1}{\PYGZsq{}}
\PYG{n}{fname\PYGZus{}child\PYGZus{}young}  \PYG{o}{=} \PYG{n}{fpath\PYGZus{}child\PYGZus{}young} \PYG{o}{+} \PYG{l+s+s1}{\PYGZsq{}}\PYG{l+s+se}{\PYGZbs{}\PYGZbs{}}\PYG{l+s+s1}{data\PYGZus{}child\PYGZus{}young.pkl}\PYG{l+s+s1}{\PYGZsq{}}

\PYG{c+c1}{\PYGZsh{} demographic chooser}
\PYG{n}{chooser} \PYG{o}{=} \PYG{p}{\PYGZob{}}\PYG{n}{dmg}\PYG{o}{.}\PYG{n}{ADULT\PYGZus{}WORK}\PYG{p}{:} \PYG{n}{cdaw}\PYG{o}{.}\PYG{n}{CHAD\PYGZus{}demography\PYGZus{}adult\PYGZus{}work}\PYG{p}{(}\PYG{p}{)}\PYG{p}{,}
           \PYG{n}{dmg}\PYG{o}{.}\PYG{n}{ADULT\PYGZus{}NON\PYGZus{}WORK}\PYG{p}{:} \PYG{n}{cdanw}\PYG{o}{.}\PYG{n}{CHAD\PYGZus{}demography\PYGZus{}adult\PYGZus{}non\PYGZus{}work}\PYG{p}{(}\PYG{p}{)}\PYG{p}{,}
           \PYG{n}{dmg}\PYG{o}{.}\PYG{n}{CHILD\PYGZus{}SCHOOL}\PYG{p}{:} \PYG{n}{cdcs}\PYG{o}{.}\PYG{n}{CHAD\PYGZus{}demography\PYGZus{}child\PYGZus{}school}\PYG{p}{(}\PYG{p}{)}\PYG{p}{,}
           \PYG{n}{dmg}\PYG{o}{.}\PYG{n}{CHILD\PYGZus{}YOUNG}\PYG{p}{:} \PYG{n}{cdcy}\PYG{o}{.}\PYG{n}{CHAD\PYGZus{}demography\PYGZus{}child\PYGZus{}young}\PYG{p}{(}\PYG{p}{)}\PYG{p}{,}
          \PYG{p}{\PYGZcb{}}
\end{sphinxVerbatim}

\fvset{hllines={, ,}}%
\begin{sphinxVerbatim}[commandchars=\\\{\}]
\PYG{c+c1}{\PYGZsh{}}
\PYG{c+c1}{\PYGZsh{} load demographic information}
\PYG{c+c1}{\PYGZsh{}}
\PYG{n}{adult\PYGZus{}work}     \PYG{o}{=} \PYG{n}{mg}\PYG{o}{.}\PYG{n}{load}\PYG{p}{(}\PYG{n}{fname\PYGZus{}adult\PYGZus{}work}\PYG{p}{)}
\PYG{n}{adult\PYGZus{}non\PYGZus{}work} \PYG{o}{=} \PYG{n}{mg}\PYG{o}{.}\PYG{n}{load}\PYG{p}{(}\PYG{n}{fname\PYGZus{}adult\PYGZus{}non\PYGZus{}work}\PYG{p}{)}
\PYG{n}{child\PYGZus{}school}   \PYG{o}{=} \PYG{n}{mg}\PYG{o}{.}\PYG{n}{load}\PYG{p}{(}\PYG{n}{fname\PYGZus{}child\PYGZus{}school}\PYG{p}{)}
\PYG{n}{child\PYGZus{}young}    \PYG{o}{=} \PYG{n}{mg}\PYG{o}{.}\PYG{n}{load}\PYG{p}{(}\PYG{n}{fname\PYGZus{}child\PYGZus{}young}\PYG{p}{)}
\end{sphinxVerbatim}

\fvset{hllines={, ,}}%
\begin{sphinxVerbatim}[commandchars=\\\{\}]
\PYG{c+c1}{\PYGZsh{} set the data}
\PYG{n}{data\PYGZus{}all} \PYG{o}{=} \PYG{p}{(}\PYG{n}{adult\PYGZus{}work}\PYG{p}{,} \PYG{n}{adult\PYGZus{}non\PYGZus{}work}\PYG{p}{,} \PYG{n}{child\PYGZus{}school}\PYG{p}{,} \PYG{n}{child\PYGZus{}young}\PYG{p}{)}

\PYG{c+c1}{\PYGZsh{} set the titles of the data}
\PYG{n}{titles}   \PYG{o}{=} \PYG{p}{(}\PYG{l+s+s1}{\PYGZsq{}}\PYG{l+s+s1}{Working Adults}\PYG{l+s+s1}{\PYGZsq{}}\PYG{p}{,} \PYG{l+s+s1}{\PYGZsq{}}\PYG{l+s+s1}{Non\PYGZhy{}working Adults}\PYG{l+s+s1}{\PYGZsq{}}\PYG{p}{,} \PYG{l+s+s1}{\PYGZsq{}}\PYG{l+s+s1}{School\PYGZhy{}age Children}\PYG{l+s+s1}{\PYGZsq{}}\PYG{p}{,} \PYG{l+s+s1}{\PYGZsq{}}\PYG{l+s+s1}{Pre\PYGZhy{}school Children}\PYG{l+s+s1}{\PYGZsq{}}\PYG{p}{)}
\end{sphinxVerbatim}

\fvset{hllines={, ,}}%
\begin{sphinxVerbatim}[commandchars=\\\{\}]
\PYG{c+c1}{\PYGZsh{} th index of the agent whose chosen for each demgoraphic, respectively}
\PYG{n}{idx} \PYG{o}{=} \PYG{l+m+mi}{2}

\PYG{c+c1}{\PYGZsh{} full simulation data}
\PYG{n}{diary\PYGZus{}demo\PYGZus{}full} \PYG{o}{=} \PYG{p}{[} \PYG{n}{xx}\PYG{o}{.}\PYG{n}{diaries}\PYG{p}{[}\PYG{n}{idx}\PYG{p}{]}\PYG{p}{[}\PYG{l+m+mi}{0}\PYG{p}{]}\PYG{o}{.}\PYG{n}{df} \PYG{k}{for} \PYG{n}{xx} \PYG{o+ow}{in} \PYG{n}{data\PYGZus{}all}\PYG{p}{]}

\PYG{c+c1}{\PYGZsh{} simulation data set to 14 days}
\PYG{n}{diary\PYGZus{}demo\PYGZus{}week} \PYG{o}{=} \PYG{p}{[}\PYG{p}{]}
\PYG{k}{for} \PYG{n}{xx} \PYG{o+ow}{in} \PYG{n}{data\PYGZus{}all}\PYG{p}{:}
    \PYG{n}{df} \PYG{o}{=} \PYG{n}{xx}\PYG{o}{.}\PYG{n}{diaries}\PYG{p}{[}\PYG{n}{idx}\PYG{p}{]}\PYG{p}{[}\PYG{l+m+mi}{0}\PYG{p}{]}\PYG{o}{.}\PYG{n}{df}
    \PYG{n}{diary\PYGZus{}demo\PYGZus{}week}\PYG{o}{.}\PYG{n}{append}\PYG{p}{(} \PYG{n}{df}\PYG{p}{[}\PYG{n}{df}\PYG{o}{.}\PYG{n}{day} \PYG{o}{\PYGZlt{}}\PYG{o}{=} \PYG{l+m+mi}{14}\PYG{p}{]}\PYG{p}{)}
\end{sphinxVerbatim}

plot

\fvset{hllines={, ,}}%
\begin{sphinxVerbatim}[commandchars=\\\{\}]
\PYG{c+c1}{\PYGZsh{}}
\PYG{c+c1}{\PYGZsh{} plot longitudinal plots of the daily activities}
\PYG{c+c1}{\PYGZsh{}}
\PYG{n}{linewidth} \PYG{o}{=} \PYG{l+m+mf}{0.5}
\PYG{n}{data} \PYG{o}{=} \PYG{n}{diary\PYGZus{}demo\PYGZus{}week}
\PYG{n}{plotter}\PYG{o}{.}\PYG{n}{plot\PYGZus{}longitude}\PYG{p}{(}\PYG{n}{data}\PYG{o}{=}\PYG{n}{data}\PYG{p}{,} \PYG{n}{titles}\PYG{o}{=}\PYG{n}{titles}\PYG{p}{,} \PYG{n}{linewidth}\PYG{o}{=}\PYG{n}{linewidth}\PYG{p}{)}
\PYG{n}{linewidth} \PYG{o}{=} \PYG{k+kc}{None}

\PYG{n}{plt}\PYG{o}{.}\PYG{n}{show}\PYG{p}{(}\PYG{p}{)}
\end{sphinxVerbatim}

\fvset{hllines={, ,}}%
\begin{sphinxVerbatim}[commandchars=\\\{\}]
\PYG{c+c1}{\PYGZsh{}}
\PYG{c+c1}{\PYGZsh{} plot the distribution of how many times each activity was done}
\PYG{c+c1}{\PYGZsh{}}

\PYG{k}{for} \PYG{n}{data}\PYG{p}{,} \PYG{n}{title} \PYG{o+ow}{in} \PYG{n+nb}{zip}\PYG{p}{(}\PYG{n}{data\PYGZus{}all}\PYG{p}{,} \PYG{n}{titles}\PYG{p}{)}\PYG{p}{:}
    \PYG{n}{plotter}\PYG{o}{.}\PYG{n}{plot\PYGZus{}count}\PYG{p}{(}\PYG{n}{data}\PYG{p}{,} \PYG{n}{chooser}\PYG{p}{[}\PYG{n}{data}\PYG{o}{.}\PYG{n}{demographic}\PYG{p}{]}\PYG{o}{.}\PYG{n}{keys}\PYG{p}{,} \PYG{n}{do\PYGZus{}abs}\PYG{o}{=}\PYG{k+kc}{True}\PYG{p}{,} \PYG{n}{title}\PYG{o}{=}\PYG{n}{title}\PYG{p}{)}
    \PYG{n}{plotter}\PYG{o}{.}\PYG{n}{plot\PYGZus{}count}\PYG{p}{(}\PYG{n}{data}\PYG{p}{,} \PYG{n}{chooser}\PYG{p}{[}\PYG{n}{data}\PYG{o}{.}\PYG{n}{demographic}\PYG{p}{]}\PYG{o}{.}\PYG{n}{keys}\PYG{p}{,} \PYG{n}{do\PYGZus{}abs}\PYG{o}{=}\PYG{k+kc}{False}\PYG{p}{,} \PYG{n}{title}\PYG{o}{=}\PYG{n}{title}\PYG{p}{)}

\PYG{n}{plt}\PYG{o}{.}\PYG{n}{show}\PYG{p}{(}\PYG{p}{)}
\end{sphinxVerbatim}


\subsection{plot\_graphs notebook}
\label{\detokenize{plot_graphs::doc}}\label{\detokenize{plot_graphs:plot-graphs-notebook}}
\fvset{hllines={, ,}}%
\begin{sphinxVerbatim}[commandchars=\\\{\}]
\PYG{c+c1}{\PYGZsh{} The United States Environmental Protection Agency through its Office of}
\PYG{c+c1}{\PYGZsh{} Research and Development has developed this software. The code is made}
\PYG{c+c1}{\PYGZsh{} publicly available to better communicate the research. All input data}
\PYG{c+c1}{\PYGZsh{} used fora given application should be reviewed by the researcher so}
\PYG{c+c1}{\PYGZsh{} that the model results are based on appropriate data for any given}
\PYG{c+c1}{\PYGZsh{} application. This model is under continued development. The model and}
\PYG{c+c1}{\PYGZsh{} data included herein do not represent and should not be construed to}
\PYG{c+c1}{\PYGZsh{} represent any Agency determination or policy.}
\PYG{c+c1}{\PYGZsh{}}
\PYG{c+c1}{\PYGZsh{} This file was written by Dr. Namdi Brandon}
\PYG{c+c1}{\PYGZsh{} ORCID: 0000\PYGZhy{}0001\PYGZhy{}7050\PYGZhy{}1538}
\PYG{c+c1}{\PYGZsh{} March 20, 2018}
\end{sphinxVerbatim}

This notebook plots graphs comparing results from the Agent-Based Model
of Human Activity Patterns (ABMHAP) to the data from the Consolidated
Human Activity Database (CHAD).
\begin{enumerate}
\item {} 
plots the graphs of a distribution of the mean values of the agent
and compares it to the distribution of CHAD mean values from the
longitudinaal data for each activity start time, end time, and
duration. The plots are the following:

\item {} 
plots the graphs of a distribution of 1 randomly chosen day from each
agent and compares it to the distribution of CHAD single-day data for
each activity start time, end time, and duration. The plots are the
following:
\begin{enumerate}
\item {} 
the CDF plots of the ABMHAP distribution and CHAD distribution

\item {} 
the inveted CDF plots of the ABMHAP distribution and CHAD
distribution

\item {} 
the inverted residual plots of the ABMHAP distribution and CHAD
distribution

\item {} 
the scaled inverted residual plots of the ABMHAP distribution and
CHAD distribution

\end{enumerate}

\item {} 
The results of the figures are saved in a suite of .pkl files

\end{enumerate}

Import

\fvset{hllines={, ,}}%
\begin{sphinxVerbatim}[commandchars=\\\{\}]
\PYG{k+kn}{import} \PYG{n+nn}{os}\PYG{o}{,} \PYG{n+nn}{sys}
\PYG{n}{sys}\PYG{o}{.}\PYG{n}{path}\PYG{o}{.}\PYG{n}{append}\PYG{p}{(}\PYG{l+s+s1}{\PYGZsq{}}\PYG{l+s+s1}{..}\PYG{l+s+se}{\PYGZbs{}\PYGZbs{}}\PYG{l+s+s1}{source}\PYG{l+s+s1}{\PYGZsq{}}\PYG{p}{)}
\PYG{n}{sys}\PYG{o}{.}\PYG{n}{path}\PYG{o}{.}\PYG{n}{append}\PYG{p}{(}\PYG{l+s+s1}{\PYGZsq{}}\PYG{l+s+s1}{..}\PYG{l+s+se}{\PYGZbs{}\PYGZbs{}}\PYG{l+s+s1}{processing}\PYG{l+s+s1}{\PYGZsq{}}\PYG{p}{)}

\PYG{c+c1}{\PYGZsh{} plotting capbailities}
\PYG{k+kn}{import} \PYG{n+nn}{matplotlib}\PYG{n+nn}{.}\PYG{n+nn}{pylab} \PYG{k}{as} \PYG{n+nn}{plt}

\PYG{c+c1}{\PYGZsh{} ABMHAP capabilities}
\PYG{k+kn}{import} \PYG{n+nn}{my\PYGZus{}globals} \PYG{k}{as} \PYG{n+nn}{mg}
\PYG{k+kn}{import} \PYG{n+nn}{chad\PYGZus{}demography\PYGZus{}adult\PYGZus{}non\PYGZus{}work} \PYG{k}{as} \PYG{n+nn}{cdanw}
\PYG{k+kn}{import} \PYG{n+nn}{chad\PYGZus{}demography\PYGZus{}adult\PYGZus{}work} \PYG{k}{as} \PYG{n+nn}{cdaw}
\PYG{k+kn}{import} \PYG{n+nn}{chad\PYGZus{}demography\PYGZus{}child\PYGZus{}school} \PYG{k}{as} \PYG{n+nn}{cdcs}
\PYG{k+kn}{import} \PYG{n+nn}{chad\PYGZus{}demography\PYGZus{}child\PYGZus{}young} \PYG{k}{as} \PYG{n+nn}{cdcy}
\PYG{k+kn}{import} \PYG{n+nn}{demography} \PYG{k}{as} \PYG{n+nn}{dmg}
\PYG{k+kn}{import} \PYG{n+nn}{evaluation} \PYG{k}{as} \PYG{n+nn}{ev}

\PYG{k+kn}{import} \PYG{n+nn}{activity}\PYG{o}{,} \PYG{n+nn}{analysis}\PYG{o}{,} \PYG{n+nn}{analyzer}\PYG{o}{,} \PYG{n+nn}{zipfile}
\end{sphinxVerbatim}

\fvset{hllines={, ,}}%
\begin{sphinxVerbatim}[commandchars=\\\{\}]
\PYG{o}{\PYGZpc{}}\PYG{k}{matplotlib} auto
\end{sphinxVerbatim}

load the data

\fvset{hllines={, ,}}%
\begin{sphinxVerbatim}[commandchars=\\\{\}]
\PYG{c+c1}{\PYGZsh{}}
\PYG{c+c1}{\PYGZsh{} load the data}
\PYG{c+c1}{\PYGZsh{}}

\PYG{c+c1}{\PYGZsh{}}
\PYG{c+c1}{\PYGZsh{} Get filename to load the data}
\PYG{c+c1}{\PYGZsh{}}

\PYG{c+c1}{\PYGZsh{} get the file name}
\PYG{n}{f\PYGZus{}data\PYGZus{}ending} \PYG{o}{=} \PYG{l+s+s1}{\PYGZsq{}}\PYG{l+s+se}{\PYGZbs{}\PYGZbs{}}\PYG{l+s+s1}{12\PYGZus{}07\PYGZus{}2017}\PYG{l+s+se}{\PYGZbs{}\PYGZbs{}}\PYG{l+s+s1}{n8192\PYGZus{}d364}\PYG{l+s+s1}{\PYGZsq{}}

\PYG{c+c1}{\PYGZsh{} the file path directory to load the data}
\PYG{n}{fpath} \PYG{o}{=} \PYG{n}{mg}\PYG{o}{.}\PYG{n}{FDIR\PYGZus{}MY\PYGZus{}DATA} \PYG{o}{+} \PYG{n}{f\PYGZus{}data\PYGZus{}ending}

\PYG{c+c1}{\PYGZsh{} the full file name for loading the data}
\PYG{n}{fname\PYGZus{}load\PYGZus{}data} \PYG{o}{=} \PYG{n}{fpath} \PYG{o}{+} \PYG{l+s+s1}{\PYGZsq{}}\PYG{l+s+se}{\PYGZbs{}\PYGZbs{}}\PYG{l+s+s1}{data\PYGZus{}child\PYGZus{}young.pkl}\PYG{l+s+s1}{\PYGZsq{}}

\PYG{n+nb}{print}\PYG{p}{(}\PYG{l+s+s1}{\PYGZsq{}}\PYG{l+s+s1}{Loading data from:}\PYG{l+s+se}{\PYGZbs{}t}\PYG{l+s+si}{\PYGZpc{}s}\PYG{l+s+s1}{\PYGZsq{}} \PYG{o}{\PYGZpc{}} \PYG{n}{fname\PYGZus{}load\PYGZus{}data}\PYG{p}{)}

\PYG{c+c1}{\PYGZsh{} clear variables}
\PYG{n}{fname}\PYG{p}{,} \PYG{n}{fpath} \PYG{o}{=} \PYG{k+kc}{None}\PYG{p}{,} \PYG{k+kc}{None}

\PYG{c+c1}{\PYGZsh{} load the data}
\PYG{n}{x} \PYG{o}{=} \PYG{n}{mg}\PYG{o}{.}\PYG{n}{load}\PYG{p}{(}\PYG{n}{fname\PYGZus{}load\PYGZus{}data}\PYG{p}{)}

\PYG{c+c1}{\PYGZsh{} get all of the data frames}
\PYG{n}{df\PYGZus{}list} \PYG{o}{=} \PYG{n}{x}\PYG{o}{.}\PYG{n}{get\PYGZus{}all\PYGZus{}data}\PYG{p}{(}\PYG{p}{)}

\PYG{c+c1}{\PYGZsh{} demographic}
\PYG{n}{demo} \PYG{o}{=} \PYG{n}{x}\PYG{o}{.}\PYG{n}{demographic}
\end{sphinxVerbatim}

parameters for saving the data

\fvset{hllines={, ,}}%
\begin{sphinxVerbatim}[commandchars=\\\{\}]
\PYG{c+c1}{\PYGZsh{}}
\PYG{c+c1}{\PYGZsh{} Get directory to save the figrues in}
\PYG{c+c1}{\PYGZsh{}}

\PYG{c+c1}{\PYGZsh{} file directory for saving the data}
\PYG{n}{fpath} \PYG{o}{=} \PYG{n}{mg}\PYG{o}{.}\PYG{n}{FDIR\PYGZus{}SAVE\PYGZus{}FIG} \PYG{o}{+} \PYG{n}{f\PYGZus{}data\PYGZus{}ending}

\PYG{c+c1}{\PYGZsh{} map the demographic to the correct file directory}
\PYG{n}{chooser\PYGZus{}fout} \PYG{o}{=} \PYG{p}{\PYGZob{}}\PYG{n}{dmg}\PYG{o}{.}\PYG{n}{ADULT\PYGZus{}WORK}\PYG{p}{:} \PYG{n}{fpath} \PYG{o}{+} \PYG{l+s+s1}{\PYGZsq{}}\PYG{l+s+se}{\PYGZbs{}\PYGZbs{}}\PYG{l+s+s1}{adult\PYGZus{}work}\PYG{l+s+s1}{\PYGZsq{}}\PYG{p}{,}
       \PYG{n}{dmg}\PYG{o}{.}\PYG{n}{ADULT\PYGZus{}NON\PYGZus{}WORK}\PYG{p}{:} \PYG{n}{fpath} \PYG{o}{+} \PYG{l+s+s1}{\PYGZsq{}}\PYG{l+s+se}{\PYGZbs{}\PYGZbs{}}\PYG{l+s+s1}{adult\PYGZus{}non\PYGZus{}work}\PYG{l+s+s1}{\PYGZsq{}}\PYG{p}{,}
       \PYG{n}{dmg}\PYG{o}{.}\PYG{n}{CHILD\PYGZus{}SCHOOL}\PYG{p}{:} \PYG{n}{fpath} \PYG{o}{+} \PYG{l+s+s1}{\PYGZsq{}}\PYG{l+s+se}{\PYGZbs{}\PYGZbs{}}\PYG{l+s+s1}{child\PYGZus{}school}\PYG{l+s+s1}{\PYGZsq{}}\PYG{p}{,}
       \PYG{n}{dmg}\PYG{o}{.}\PYG{n}{CHILD\PYGZus{}YOUNG}\PYG{p}{:} \PYG{n}{fpath} \PYG{o}{+} \PYG{l+s+s1}{\PYGZsq{}}\PYG{l+s+se}{\PYGZbs{}\PYGZbs{}}\PYG{l+s+s1}{child\PYGZus{}young}\PYG{l+s+s1}{\PYGZsq{}}\PYG{p}{,}
      \PYG{p}{\PYGZcb{}}

\PYG{c+c1}{\PYGZsh{} get the file directory to save the data}
\PYG{n}{fpath\PYGZus{}save\PYGZus{}fig} \PYG{o}{=} \PYG{n}{chooser\PYGZus{}fout}\PYG{p}{[}\PYG{n}{demo}\PYG{p}{]}

\PYG{n+nb}{print}\PYG{p}{(}\PYG{l+s+s1}{\PYGZsq{}}\PYG{l+s+s1}{The directory to save the data:}\PYG{l+s+se}{\PYGZbs{}t}\PYG{l+s+si}{\PYGZpc{}s}\PYG{l+s+s1}{\PYGZsq{}} \PYG{o}{\PYGZpc{}} \PYG{n}{fpath\PYGZus{}save\PYGZus{}fig}\PYG{p}{)}

\PYG{c+c1}{\PYGZsh{} clear variables}
\PYG{n}{fpath} \PYG{o}{=} \PYG{k+kc}{None}
\end{sphinxVerbatim}

the plotting parameters

\fvset{hllines={, ,}}%
\begin{sphinxVerbatim}[commandchars=\\\{\}]
\PYG{c+c1}{\PYGZsh{}}
\PYG{c+c1}{\PYGZsh{} plotting flags}
\PYG{c+c1}{\PYGZsh{}}

\PYG{c+c1}{\PYGZsh{} calculates the plots}
\PYG{n}{do\PYGZus{}plot} \PYG{o}{=} \PYG{k+kc}{True}

\PYG{c+c1}{\PYGZsh{} save the figures}
\PYG{n}{do\PYGZus{}save\PYGZus{}fig} \PYG{o}{=} \PYG{k+kc}{False}

\PYG{c+c1}{\PYGZsh{} show the plots}
\PYG{n}{do\PYGZus{}show} \PYG{o}{=} \PYG{k+kc}{False}

\PYG{c+c1}{\PYGZsh{} show extra print messages}
\PYG{n}{do\PYGZus{}print} \PYG{o}{=} \PYG{k+kc}{False}
\end{sphinxVerbatim}

\fvset{hllines={, ,}}%
\begin{sphinxVerbatim}[commandchars=\\\{\}]
\PYG{c+c1}{\PYGZsh{}}
\PYG{c+c1}{\PYGZsh{} demography}
\PYG{c+c1}{\PYGZsh{}}

\PYG{c+c1}{\PYGZsh{} map the demograph;y identifiyer to the demographics object}
\PYG{n}{chooser} \PYG{o}{=} \PYG{p}{\PYGZob{}}\PYG{n}{dmg}\PYG{o}{.}\PYG{n}{ADULT\PYGZus{}WORK}\PYG{p}{:} \PYG{n}{cdaw}\PYG{o}{.}\PYG{n}{CHAD\PYGZus{}demography\PYGZus{}adult\PYGZus{}work}\PYG{p}{(}\PYG{p}{)}\PYG{p}{,}
           \PYG{n}{dmg}\PYG{o}{.}\PYG{n}{ADULT\PYGZus{}NON\PYGZus{}WORK}\PYG{p}{:} \PYG{n}{cdanw}\PYG{o}{.}\PYG{n}{CHAD\PYGZus{}demography\PYGZus{}adult\PYGZus{}non\PYGZus{}work}\PYG{p}{(}\PYG{p}{)}\PYG{p}{,}
           \PYG{n}{dmg}\PYG{o}{.}\PYG{n}{CHILD\PYGZus{}SCHOOL}\PYG{p}{:} \PYG{n}{cdcs}\PYG{o}{.}\PYG{n}{CHAD\PYGZus{}demography\PYGZus{}child\PYGZus{}school}\PYG{p}{(}\PYG{p}{)}\PYG{p}{,}
           \PYG{n}{dmg}\PYG{o}{.}\PYG{n}{CHILD\PYGZus{}YOUNG}\PYG{p}{:} \PYG{n}{cdcy}\PYG{o}{.}\PYG{n}{CHAD\PYGZus{}demography\PYGZus{}child\PYGZus{}young}\PYG{p}{(}\PYG{p}{)}\PYG{p}{,}
          \PYG{p}{\PYGZcb{}}

\PYG{c+c1}{\PYGZsh{} choose the demography}
\PYG{n}{chad\PYGZus{}demo} \PYG{o}{=} \PYG{n}{chooser}\PYG{p}{[}\PYG{n}{demo}\PYG{p}{]}
\end{sphinxVerbatim}

plot

\fvset{hllines={, ,}}%
\begin{sphinxVerbatim}[commandchars=\\\{\}]
\PYG{c+c1}{\PYGZsh{} CHAD parameters}
\PYG{n}{chad\PYGZus{}param\PYGZus{}list} \PYG{o}{=} \PYG{n}{chad\PYGZus{}demo}\PYG{o}{.}\PYG{n}{int\PYGZus{}2\PYGZus{}param}

\PYG{c+c1}{\PYGZsh{} get the activity codes for a given trial}
\PYG{n}{act\PYGZus{}codes} \PYG{o}{=} \PYG{n}{chad\PYGZus{}demo}\PYG{o}{.}\PYG{n}{keys}

\PYG{c+c1}{\PYGZsh{} the directories for the respective activities. This is used for saving the figures}
\PYG{n}{fdirs} \PYG{o}{=} \PYG{n}{analyzer}\PYG{o}{.}\PYG{n}{get\PYGZus{}verify\PYGZus{}fpath}\PYG{p}{(}\PYG{n}{fpath\PYGZus{}save\PYGZus{}fig}\PYG{p}{,} \PYG{n}{act\PYGZus{}codes}\PYG{p}{)}

\PYG{k}{if} \PYG{n}{fpath\PYGZus{}save\PYGZus{}fig} \PYG{o+ow}{is} \PYG{k+kc}{None}\PYG{p}{:}
    \PYG{n}{do\PYGZus{}save\PYGZus{}fig} \PYG{o}{=} \PYG{k+kc}{False}

\PYG{c+c1}{\PYGZsh{} offset, used for figure identifiers}
\PYG{n}{off} \PYG{o}{=} \PYG{l+m+mi}{0}

\PYG{c+c1}{\PYGZsh{} number of days in the simulation}
\PYG{n}{n\PYGZus{}days} \PYG{o}{=} \PYG{n+nb}{len}\PYG{p}{(} \PYG{n}{df\PYGZus{}list}\PYG{p}{[}\PYG{l+m+mi}{0}\PYG{p}{]}\PYG{o}{.}\PYG{n}{day}\PYG{o}{.}\PYG{n}{unique}\PYG{p}{(}\PYG{p}{)} \PYG{p}{)}

\PYG{n}{fid} \PYG{o}{=} \PYG{l+m+mi}{0}

\PYG{k}{for} \PYG{n}{act}\PYG{p}{,} \PYG{n}{fpath} \PYG{o+ow}{in} \PYG{n+nb}{zip}\PYG{p}{(}\PYG{n}{act\PYGZus{}codes}\PYG{p}{,} \PYG{n}{fdirs}\PYG{p}{)}\PYG{p}{:}

    \PYG{n+nb}{print}\PYG{p}{(} \PYG{n}{activity}\PYG{o}{.}\PYG{n}{INT\PYGZus{}2\PYGZus{}STR}\PYG{p}{[}\PYG{n}{act}\PYG{p}{]}\PYG{p}{)}
    \PYG{k}{if} \PYG{p}{(}\PYG{n}{do\PYGZus{}print}\PYG{p}{)}\PYG{p}{:}
        \PYG{n}{msg} \PYG{o}{=} \PYG{l+s+s1}{\PYGZsq{}}\PYG{l+s+s1}{starting analysis for the }\PYG{l+s+s1}{\PYGZsq{}} \PYG{o}{+} \PYG{n}{activity}\PYG{o}{.}\PYG{n}{INT\PYGZus{}2\PYGZus{}STR}\PYG{p}{[}\PYG{n}{act}\PYG{p}{]} \PYG{o}{+} \PYG{l+s+s1}{\PYGZsq{}}\PYG{l+s+s1}{ activity ...}\PYG{l+s+s1}{\PYGZsq{}}
        \PYG{n+nb}{print}\PYG{p}{(}\PYG{n}{msg}\PYG{p}{)}

    \PYG{c+c1}{\PYGZsh{} this is to see if the analysis of the moments for start time needs to be in [\PYGZhy{}12, 12)}
    \PYG{c+c1}{\PYGZsh{} instead of [0, 24) format}
    \PYG{n}{chooser}     \PYG{o}{=} \PYG{p}{\PYGZob{}}\PYG{n}{activity}\PYG{o}{.}\PYG{n}{SLEEP}\PYG{p}{:} \PYG{k+kc}{True}\PYG{p}{,} \PYG{p}{\PYGZcb{}}
    \PYG{n}{do\PYGZus{}periodic} \PYG{o}{=} \PYG{n}{chooser}\PYG{o}{.}\PYG{n}{get}\PYG{p}{(}\PYG{n}{act}\PYG{p}{,} \PYG{k+kc}{False}\PYG{p}{)}

    \PYG{c+c1}{\PYGZsh{} get the CHAD data}
    \PYG{c+c1}{\PYGZsh{} this is here to access the data frames from t.initialize()}
    \PYG{n}{f\PYGZus{}stats} \PYG{o}{=} \PYG{n}{chad\PYGZus{}demo}\PYG{o}{.}\PYG{n}{fname\PYGZus{}stats}\PYG{p}{[}\PYG{n}{act}\PYG{p}{]}

    \PYG{c+c1}{\PYGZsh{} the sampling parameters for 1 household}
    \PYG{n}{s\PYGZus{}params} \PYG{o}{=} \PYG{n}{chad\PYGZus{}demo}\PYG{o}{.}\PYG{n}{int\PYGZus{}2\PYGZus{}param}\PYG{p}{[}\PYG{n}{act}\PYG{p}{]}

    \PYG{c+c1}{\PYGZsh{} get the CHAD data}
    \PYG{n}{chad\PYGZus{}start}\PYG{p}{,} \PYG{n}{chad\PYGZus{}end}\PYG{p}{,} \PYG{n}{chad\PYGZus{}dt}\PYG{p}{,} \PYG{n}{chad\PYGZus{}record} \PYG{o}{=} \PYGZbs{}
        \PYG{n}{analysis}\PYG{o}{.}\PYG{n}{get\PYGZus{}verification\PYGZus{}info}\PYG{p}{(}\PYG{n}{demo}\PYG{o}{=}\PYG{n}{demo}\PYG{p}{,} \PYG{n}{key\PYGZus{}activity}\PYG{o}{=}\PYG{n}{act}\PYG{p}{,} \PYG{n}{fname\PYGZus{}stats}\PYG{o}{=}\PYG{n}{f\PYGZus{}stats}\PYG{p}{,} \PYGZbs{}
                                       \PYG{n}{sampling\PYGZus{}params}\PYG{o}{=}\PYG{p}{[}\PYG{n}{s\PYGZus{}params}\PYG{p}{]} \PYG{p}{)}

    \PYG{c+c1}{\PYGZsh{} plot the ABMHAP data}
    \PYG{n}{df\PYGZus{}abm}         \PYG{o}{=} \PYG{n}{ev}\PYG{o}{.}\PYG{n}{sample\PYGZus{}activity\PYGZus{}abm}\PYG{p}{(}\PYG{n}{df\PYGZus{}list}\PYG{p}{,} \PYG{n}{act}\PYG{p}{)}
    \PYG{n}{abm\PYGZus{}start\PYGZus{}mean} \PYG{o}{=} \PYG{n}{df\PYGZus{}abm}\PYG{o}{.}\PYG{n}{start}\PYG{o}{.}\PYG{n}{values}
    \PYG{n}{abm\PYGZus{}end\PYGZus{}mean}   \PYG{o}{=} \PYG{n}{df\PYGZus{}abm}\PYG{o}{.}\PYG{n}{end}\PYG{o}{.}\PYG{n}{values}
    \PYG{n}{abm\PYGZus{}dt\PYGZus{}mean}    \PYG{o}{=} \PYG{n}{df\PYGZus{}abm}\PYG{o}{.}\PYG{n}{dt}\PYG{o}{.}\PYG{n}{values}

    \PYG{c+c1}{\PYGZsh{} create the plots}
    \PYG{k}{if} \PYG{p}{(}\PYG{n}{do\PYGZus{}plot}\PYG{p}{)}\PYG{p}{:}

        \PYG{n+nb}{print}\PYG{p}{(}\PYG{n}{fpath}\PYG{p}{)}
        \PYG{c+c1}{\PYGZsh{}if s\PYGZus{}params.do\PYGZus{}start:}
        \PYG{n}{fid} \PYG{o}{=} \PYG{n}{fid} \PYG{o}{+} \PYG{l+m+mi}{1}
        \PYG{n}{analyzer}\PYG{o}{.}\PYG{n}{plot\PYGZus{}verify\PYGZus{}start}\PYG{p}{(}\PYG{n}{act}\PYG{p}{,} \PYG{n}{abm\PYGZus{}start\PYGZus{}mean}\PYG{p}{,} \PYG{n}{chad\PYGZus{}start}\PYG{p}{[}\PYG{l+s+s1}{\PYGZsq{}}\PYG{l+s+s1}{mu}\PYG{l+s+s1}{\PYGZsq{}}\PYG{p}{]}\PYG{o}{.}\PYG{n}{values}\PYG{p}{,} \PYG{n}{fid}\PYG{o}{=}\PYG{n}{fid}\PYG{p}{,} \PYGZbs{}
                                   \PYG{n}{do\PYGZus{}save\PYGZus{}fig}\PYG{o}{=}\PYG{n}{do\PYGZus{}save\PYGZus{}fig}\PYG{p}{,} \PYG{n}{fpath}\PYG{o}{=}\PYG{n}{fpath}\PYG{p}{)}

        \PYG{c+c1}{\PYGZsh{}if s\PYGZus{}params.do\PYGZus{}end:}
        \PYG{n}{fid} \PYG{o}{=} \PYG{n}{fid} \PYG{o}{+} \PYG{l+m+mi}{1}
        \PYG{n}{analyzer}\PYG{o}{.}\PYG{n}{plot\PYGZus{}verify\PYGZus{}end}\PYG{p}{(}\PYG{n}{act}\PYG{p}{,} \PYG{n}{abm\PYGZus{}end\PYGZus{}mean}\PYG{p}{,} \PYG{n}{chad\PYGZus{}end}\PYG{p}{[}\PYG{l+s+s1}{\PYGZsq{}}\PYG{l+s+s1}{mu}\PYG{l+s+s1}{\PYGZsq{}}\PYG{p}{]}\PYG{o}{.}\PYG{n}{values}\PYG{p}{,} \PYG{n}{fid}\PYG{o}{=}\PYG{n}{fid}\PYG{p}{,} \PYGZbs{}
                                 \PYG{n}{do\PYGZus{}save\PYGZus{}fig}\PYG{o}{=}\PYG{n}{do\PYGZus{}save\PYGZus{}fig}\PYG{p}{,} \PYG{n}{fpath}\PYG{o}{=}\PYG{n}{fpath}\PYG{p}{)}

        \PYG{c+c1}{\PYGZsh{}if s\PYGZus{}params.do\PYGZus{}dt:}
        \PYG{n}{fid} \PYG{o}{=} \PYG{n}{fid} \PYG{o}{+} \PYG{l+m+mi}{1}
        \PYG{n}{analyzer}\PYG{o}{.}\PYG{n}{plot\PYGZus{}verify\PYGZus{}dt}\PYG{p}{(}\PYG{n}{act}\PYG{p}{,} \PYG{n}{abm\PYGZus{}dt\PYGZus{}mean}\PYG{p}{,} \PYG{n}{chad\PYGZus{}dt}\PYG{p}{[}\PYG{l+s+s1}{\PYGZsq{}}\PYG{l+s+s1}{mu}\PYG{l+s+s1}{\PYGZsq{}}\PYG{p}{]}\PYG{o}{.}\PYG{n}{values}\PYG{p}{,} \PYG{n}{fid}\PYG{o}{=}\PYG{n}{fid}\PYG{p}{,} \PYGZbs{}
                                 \PYG{n}{do\PYGZus{}save\PYGZus{}fig}\PYG{o}{=}\PYG{n}{do\PYGZus{}save\PYGZus{}fig}\PYG{p}{,} \PYG{n}{fpath}\PYG{o}{=}\PYG{n}{fpath}\PYG{p}{)}

\PYG{k}{if} \PYG{n}{do\PYGZus{}show}\PYG{p}{:}
    \PYG{n}{plt}\PYG{o}{.}\PYG{n}{show}\PYG{p}{(}\PYG{p}{)}
\PYG{k}{else}\PYG{p}{:}
    \PYG{n}{plt}\PYG{o}{.}\PYG{n}{close}\PYG{p}{(}\PYG{l+s+s1}{\PYGZsq{}}\PYG{l+s+s1}{all}\PYG{l+s+s1}{\PYGZsq{}}\PYG{p}{)}
\end{sphinxVerbatim}

Validation

\fvset{hllines={, ,}}%
\begin{sphinxVerbatim}[commandchars=\\\{\}]
\PYG{c+c1}{\PYGZsh{} get the CHAD sampling parameters for the given demographioc}
\PYG{n}{chad\PYGZus{}param\PYGZus{}list} \PYG{o}{=} \PYG{n}{x}\PYG{o}{.}\PYG{n}{chad\PYGZus{}param\PYGZus{}list}

\PYG{c+c1}{\PYGZsh{} get the sampling parameters}
\PYG{n}{s\PYGZus{}params} \PYG{o}{=} \PYG{n}{chad\PYGZus{}param\PYGZus{}list}\PYG{p}{[}\PYG{l+m+mi}{0}\PYG{p}{]}

\PYG{c+c1}{\PYGZsh{} get the figure index}
\PYG{n}{fidx} \PYG{o}{=} \PYG{l+m+mi}{100}

\PYG{c+c1}{\PYGZsh{} save flag}
\PYG{n}{do\PYGZus{}save} \PYG{o}{=} \PYG{k+kc}{False}

\PYG{n+nb}{print}\PYG{p}{(}\PYG{n}{fpath\PYGZus{}save\PYGZus{}fig}\PYG{p}{)}
\end{sphinxVerbatim}

Compare random events

\fvset{hllines={, ,}}%
\begin{sphinxVerbatim}[commandchars=\\\{\}]
\PYG{c+c1}{\PYGZsh{} the activity codes}
\PYG{n}{act\PYGZus{}codes} \PYG{o}{=} \PYG{n}{chad\PYGZus{}demo}\PYG{o}{.}\PYG{n}{keys}
\PYG{c+c1}{\PYGZsh{}act\PYGZus{}codes = [mg.KEY\PYGZus{}WORK]}

\PYG{c+c1}{\PYGZsh{} open the data}
\PYG{n}{z} \PYG{o}{=} \PYG{n}{zipfile}\PYG{o}{.}\PYG{n}{ZipFile}\PYG{p}{(}\PYG{n}{chad\PYGZus{}demo}\PYG{o}{.}\PYG{n}{fname\PYGZus{}zip}\PYG{p}{,} \PYG{n}{mode}\PYG{o}{=}\PYG{l+s+s1}{\PYGZsq{}}\PYG{l+s+s1}{r}\PYG{l+s+s1}{\PYGZsq{}}\PYG{p}{)}

\PYG{c+c1}{\PYGZsh{} this flag allows the code to pick a random record from the longitudinal data (if True)}
\PYG{c+c1}{\PYGZsh{} or single\PYGZhy{}day data (if False)}
\PYG{n}{do\PYGZus{}random\PYGZus{}long} \PYG{o}{=} \PYG{k+kc}{False}

\PYG{c+c1}{\PYGZsh{} for each activity, plot the corresponding plots}
\PYG{k}{for} \PYG{n}{act} \PYG{o+ow}{in} \PYG{n}{act\PYGZus{}codes}\PYG{p}{:}

    \PYG{n+nb}{print}\PYG{p}{(} \PYG{n}{activity}\PYG{o}{.}\PYG{n}{INT\PYGZus{}2\PYGZus{}STR}\PYG{p}{[}\PYG{n}{act}\PYG{p}{]} \PYG{p}{)}


    \PYG{c+c1}{\PYGZsh{} periodic time flag [\PYGZhy{}12, 12)}
    \PYG{n}{do\PYGZus{}periodic} \PYG{o}{=} \PYG{k+kc}{False}

    \PYG{c+c1}{\PYGZsh{} if the activity occurs over midnight (if True), set the}
    \PYG{c+c1}{\PYGZsh{}}
    \PYG{k}{if} \PYG{n}{act} \PYG{o}{==} \PYG{n}{activity}\PYG{o}{.}\PYG{n}{SLEEP}\PYG{p}{:}
        \PYG{n}{do\PYGZus{}periodic} \PYG{o}{=} \PYG{k+kc}{True}

    \PYG{c+c1}{\PYGZsh{} sample the ABM data}
    \PYG{n}{df\PYGZus{}abm}  \PYG{o}{=} \PYG{n}{ev}\PYG{o}{.}\PYG{n}{sample\PYGZus{}activity\PYGZus{}abm}\PYG{p}{(}\PYG{n}{df\PYGZus{}list}\PYG{p}{,} \PYG{n}{act}\PYG{p}{)}

    \PYG{c+c1}{\PYGZsh{} get the CHAD data}
    \PYG{c+c1}{\PYGZsh{} this is here to access the data frames from t.initialize()}
    \PYG{n}{f\PYGZus{}stats} \PYG{o}{=} \PYG{n}{chad\PYGZus{}demo}\PYG{o}{.}\PYG{n}{fname\PYGZus{}stats}\PYG{p}{[}\PYG{n}{act}\PYG{p}{]}

    \PYG{c+c1}{\PYGZsh{} get the file name data of the single name data}
    \PYG{k}{if} \PYG{n}{do\PYGZus{}random\PYGZus{}long} \PYG{o}{==} \PYG{k+kc}{False}\PYG{p}{:}
        \PYG{k}{for} \PYG{n}{k} \PYG{o+ow}{in} \PYG{n}{f\PYGZus{}stats}\PYG{o}{.}\PYG{n}{keys}\PYG{p}{(}\PYG{p}{)}\PYG{p}{:}
            \PYG{n}{f\PYGZus{}stats}\PYG{p}{[}\PYG{n}{k}\PYG{p}{]} \PYG{o}{=} \PYG{n}{f\PYGZus{}stats}\PYG{p}{[}\PYG{n}{k}\PYG{p}{]}\PYG{o}{.}\PYG{n}{replace}\PYG{p}{(}\PYG{l+s+s1}{\PYGZsq{}}\PYG{l+s+s1}{longitude}\PYG{l+s+s1}{\PYGZsq{}}\PYG{p}{,} \PYG{l+s+s1}{\PYGZsq{}}\PYG{l+s+s1}{solo}\PYG{l+s+s1}{\PYGZsq{}}\PYG{p}{)}

    \PYG{c+c1}{\PYGZsh{} the sampling parameters for 1 household}
    \PYG{n}{s\PYGZus{}params} \PYG{o}{=} \PYG{n}{chad\PYGZus{}demo}\PYG{o}{.}\PYG{n}{int\PYGZus{}2\PYGZus{}param}\PYG{p}{[}\PYG{n}{act}\PYG{p}{]}

    \PYG{c+c1}{\PYGZsh{} get the CHAD data}
    \PYG{n}{stats\PYGZus{}start}\PYG{p}{,} \PYG{n}{stats\PYGZus{}end}\PYG{p}{,} \PYG{n}{stats\PYGZus{}dt}\PYG{p}{,} \PYG{n}{record} \PYG{o}{=} \PYGZbs{}
        \PYG{n}{analysis}\PYG{o}{.}\PYG{n}{get\PYGZus{}verification\PYGZus{}info}\PYG{p}{(}\PYG{n}{demo}\PYG{o}{=}\PYG{n}{demo}\PYG{p}{,} \PYG{n}{key\PYGZus{}activity}\PYG{o}{=}\PYG{n}{act}\PYG{p}{,} \PYG{n}{fname\PYGZus{}stats}\PYG{o}{=}\PYG{n}{f\PYGZus{}stats}\PYG{p}{,} \PYGZbs{}
                                       \PYG{n}{sampling\PYGZus{}params}\PYG{o}{=}\PYG{p}{[}\PYG{n}{s\PYGZus{}params}\PYG{p}{]}\PYG{p}{)}

    \PYG{c+c1}{\PYGZsh{} grouby the CHAD records by identifier}
    \PYG{n}{gb}  \PYG{o}{=} \PYG{n}{record}\PYG{o}{.}\PYG{n}{groupby}\PYG{p}{(}\PYG{l+s+s1}{\PYGZsq{}}\PYG{l+s+s1}{PID}\PYG{l+s+s1}{\PYGZsq{}}\PYG{p}{)}
    \PYG{n}{pid} \PYG{o}{=} \PYG{n}{record}\PYG{o}{.}\PYG{n}{PID}\PYG{o}{.}\PYG{n}{unique}\PYG{p}{(}\PYG{p}{)}

    \PYG{c+c1}{\PYGZsh{} return true if x is in pid}
    \PYG{n}{f} \PYG{o}{=} \PYG{k}{lambda} \PYG{n}{x}\PYG{p}{:} \PYG{n}{x} \PYG{o+ow}{in} \PYG{n}{pid}

    \PYG{c+c1}{\PYGZsh{} indices of records within \PYGZsq{}pid\PYGZsq{}}
    \PYG{n}{i} \PYG{o}{=} \PYG{n}{record}\PYG{o}{.}\PYG{n}{PID}\PYG{o}{.}\PYG{n}{apply}\PYG{p}{(}\PYG{n}{f}\PYG{p}{)}

    \PYG{c+c1}{\PYGZsh{} get the CHAD observations}
    \PYG{n}{df\PYGZus{}obs} \PYG{o}{=} \PYG{n}{record}\PYG{p}{[}\PYG{n}{i}\PYG{p}{]}

    \PYG{c+c1}{\PYGZsh{} get teh CHAD records that satisfy the sampling parameters for the given activity}
    \PYG{n}{df\PYGZus{}obs\PYGZus{}new} \PYG{o}{=} \PYG{n}{s\PYGZus{}params}\PYG{o}{.}\PYG{n}{get\PYGZus{}record}\PYG{p}{(}\PYG{n}{df\PYGZus{}obs}\PYG{p}{,} \PYG{n}{do\PYGZus{}periodic}\PYG{p}{)}

    \PYG{c+c1}{\PYGZsh{} get the single day observations}
    \PYG{n+nb}{print}\PYG{p}{(}\PYG{n}{fpath\PYGZus{}save\PYGZus{}fig}\PYG{p}{)}
    \PYG{n}{fid\PYGZus{}last}    \PYG{o}{=} \PYG{n}{ev}\PYG{o}{.}\PYG{n}{compare\PYGZus{}abm\PYGZus{}to\PYGZus{}chad\PYGZus{}help}\PYG{p}{(}\PYG{n}{df\PYGZus{}abm}\PYG{o}{=}\PYG{n}{df\PYGZus{}abm}\PYG{p}{,} \PYG{n}{df\PYGZus{}obs}\PYG{o}{=}\PYG{n}{df\PYGZus{}obs\PYGZus{}new}\PYG{p}{,} \PYG{n}{act\PYGZus{}code}\PYG{o}{=}\PYG{n}{act}\PYG{p}{,} \PYG{n}{fidx}\PYG{o}{=}\PYG{n}{fidx}\PYG{p}{,} \PYGZbs{}
                                              \PYG{n}{do\PYGZus{}save}\PYG{o}{=}\PYG{n}{do\PYGZus{}save}\PYG{p}{,} \PYG{n}{fpath}\PYG{o}{=}\PYG{n}{fpath\PYGZus{}save\PYGZus{}fig}\PYG{p}{)}
    \PYG{n}{fidx}        \PYG{o}{=} \PYG{n}{fid\PYGZus{}last} \PYG{o}{+} \PYG{l+m+mi}{1}

\PYG{n}{z}\PYG{o}{.}\PYG{n}{close}\PYG{p}{(}\PYG{p}{)}

\PYG{n+nb}{print}\PYG{p}{(}\PYG{l+s+s1}{\PYGZsq{}}\PYG{l+s+s1}{finished plotting...}\PYG{l+s+s1}{\PYGZsq{}}\PYG{p}{)}

\PYG{c+c1}{\PYGZsh{} show the plots}
\PYG{k}{if} \PYG{n}{do\PYGZus{}show}\PYG{p}{:}
    \PYG{n}{plt}\PYG{o}{.}\PYG{n}{show}\PYG{p}{(}\PYG{p}{)}
\PYG{k}{else}\PYG{p}{:}
    \PYG{c+c1}{\PYGZsh{} clear all of the plots}
    \PYG{n}{plt}\PYG{o}{.}\PYG{n}{close}\PYG{p}{(}\PYG{l+s+s1}{\PYGZsq{}}\PYG{l+s+s1}{all}\PYG{l+s+s1}{\PYGZsq{}}\PYG{p}{)}

\PYG{n}{fpath} \PYG{o}{=} \PYG{k+kc}{None}
\end{sphinxVerbatim}


\subsection{my\_debug module}
\label{\detokenize{my_debug::doc}}\label{\detokenize{my_debug:module-my_debug}}\label{\detokenize{my_debug:my-debug-module}}\index{my\_debug (module)}
\begin{sphinxadmonition}{warning}{Warning:}
This is not used as part of the ABMHAP module. This should be removed.
\end{sphinxadmonition}


\subsection{omni\_trial module}
\label{\detokenize{omni_trial::doc}}\label{\detokenize{omni_trial:module-omni_trial}}\label{\detokenize{omni_trial:omni-trial-module}}\index{omni\_trial (module)}
This is the module that is in charge of running simulations comparing the Agent-Based Model of Human Activity Patterns (ABMHAP) with the data from the Consolidated Human Activity Database (CHAD) comparing the performance of ABMHAP with all of the activity data.

This module contains class {\hyperref[\detokenize{omni_trial:omni_trial.Omni_Trial}]{\sphinxcrossref{\sphinxcode{\sphinxupquote{omni\_trial.Omni\_Trial}}}}}.
\index{Omni\_Trial (class in omni\_trial)}

\begin{fulllineitems}
\phantomsection\label{\detokenize{omni_trial:omni_trial.Omni_Trial}}\pysiglinewithargsret{\sphinxbfcode{\sphinxupquote{class }}\sphinxcode{\sphinxupquote{omni\_trial.}}\sphinxbfcode{\sphinxupquote{Omni\_Trial}}}{\emph{parameters}, \emph{sampling\_params}, \emph{demographic}}{}
Bases: {\hyperref[\detokenize{trial:trial.Trial}]{\sphinxcrossref{\sphinxcode{\sphinxupquote{trial.Trial}}}}}

This class runs the ABMHAP simulations initialized with all of the activity data from CHAD for     a given demographic. For the respective demographic, the following activity-data from CHAD     are used:
\begin{itemize}
\item {} 
commute from work

\item {} 
commute to work

\item {} 
eat breakfast

\item {} 
eat dinner

\item {} 
eat lunch

\item {} 
sleep

\item {} 
work

\end{itemize}
\begin{quote}\begin{description}
\item[{Parameters}] \leavevmode\begin{itemize}
\item {} 
\sphinxstyleliteralstrong{\sphinxupquote{params}} ({\hyperref[\detokenize{params:params.Params}]{\sphinxcrossref{\sphinxstyleliteralemphasis{\sphinxupquote{params.Params}}}}}) \textendash{} the parameters that describe the household:

\item {} 
\sphinxstyleliteralstrong{\sphinxupquote{sampling\_parameters}} (dict of activity code - {\hyperref[\detokenize{chad_params:chad_params.CHAD_params}]{\sphinxcrossref{\sphinxcode{\sphinxupquote{chad\_params.CHAD\_params}}}}}) \textendash{} maps an activity code to the sampling parameters     to the CHAD data for the respective activity

\item {} 
\sphinxstyleliteralstrong{\sphinxupquote{demographic}} (\sphinxstyleliteralemphasis{\sphinxupquote{int}}) \textendash{} the demographic identifier

\end{itemize}

\end{description}\end{quote}
\index{adjust\_commute\_from\_work() (omni\_trial.Omni\_Trial method)}

\begin{fulllineitems}
\phantomsection\label{\detokenize{omni_trial:omni_trial.Omni_Trial.adjust_commute_from_work}}\pysiglinewithargsret{\sphinxbfcode{\sphinxupquote{adjust\_commute\_from\_work}}}{\emph{data}, \emph{no\_variation=False}}{}
This function adjusts the household parameters to reflect the sampled parameters         (mean and standard deviation of start time, end time, and         duration, respectively), from the CHAD data for the commuting from work         activity.
\begin{quote}\begin{description}
\item[{Parameters}] \leavevmode\begin{itemize}
\item {} 
\sphinxstyleliteralstrong{\sphinxupquote{data}} (\sphinxstyleliteralemphasis{\sphinxupquote{tuple of numpy.ndarray}}\sphinxstyleliteralemphasis{\sphinxupquote{, }}\sphinxstyleliteralemphasis{\sphinxupquote{numpy.ndarray}}\sphinxstyleliteralemphasis{\sphinxupquote{, }}\sphinxstyleliteralemphasis{\sphinxupquote{numpy.ndarray}}\sphinxstyleliteralemphasis{\sphinxupquote{,         }}\sphinxstyleliteralemphasis{\sphinxupquote{numpy.ndarray}}\sphinxstyleliteralemphasis{\sphinxupquote{, }}\sphinxstyleliteralemphasis{\sphinxupquote{numpy.ndarray}}\sphinxstyleliteralemphasis{\sphinxupquote{, }}\sphinxstyleliteralemphasis{\sphinxupquote{numpy.ndarray}}) \textendash{} relevant parameters for each person in the household for         commuting from work. The tuple contains the following: mean start time, standard         deviation of start time, mean end time, standard deviation of end time, mean duration,         and standard deviation of duration for each person in the household.

\item {} 
\sphinxstyleliteralstrong{\sphinxupquote{no\_variation}} (\sphinxstyleliteralemphasis{\sphinxupquote{bool}}) \textendash{} whether (if True) or not (if False) intra-individual         variation is set to zero among the activities

\end{itemize}

\item[{Returns}] \leavevmode


\end{description}\end{quote}

\end{fulllineitems}

\index{adjust\_commute\_to\_work() (omni\_trial.Omni\_Trial method)}

\begin{fulllineitems}
\phantomsection\label{\detokenize{omni_trial:omni_trial.Omni_Trial.adjust_commute_to_work}}\pysiglinewithargsret{\sphinxbfcode{\sphinxupquote{adjust\_commute\_to\_work}}}{\emph{data}, \emph{no\_variation=False}}{}
This function adjusts the household parameters to reflect the sampled parameters         (mean and standard deviation of start time, end time, and         duration, respectively), from the CHAD data for the commuting to work         activity.
\begin{quote}\begin{description}
\item[{Parameters}] \leavevmode\begin{itemize}
\item {} 
\sphinxstyleliteralstrong{\sphinxupquote{data}} (\sphinxstyleliteralemphasis{\sphinxupquote{tuple of numpy.ndarray}}\sphinxstyleliteralemphasis{\sphinxupquote{, }}\sphinxstyleliteralemphasis{\sphinxupquote{numpy.ndarray}}\sphinxstyleliteralemphasis{\sphinxupquote{, }}\sphinxstyleliteralemphasis{\sphinxupquote{numpy.ndarray}}\sphinxstyleliteralemphasis{\sphinxupquote{,         }}\sphinxstyleliteralemphasis{\sphinxupquote{numpy.ndarray}}\sphinxstyleliteralemphasis{\sphinxupquote{, }}\sphinxstyleliteralemphasis{\sphinxupquote{numpy.ndarray}}\sphinxstyleliteralemphasis{\sphinxupquote{, }}\sphinxstyleliteralemphasis{\sphinxupquote{numpy.ndarray}}) \textendash{} relevant parameters for each person in the household for         commuting to work. The tuple contains the following: mean start time, standard         deviation of start time, mean end time, standard deviation of end time, mean duration,         and standard deviation of duration for each person in the household.

\item {} 
\sphinxstyleliteralstrong{\sphinxupquote{no\_variation}} (\sphinxstyleliteralemphasis{\sphinxupquote{bool}}) \textendash{} whether (if True) or not (if False) intra-individual         variation is set to zero among the activities

\end{itemize}

\item[{Returns}] \leavevmode


\end{description}\end{quote}

\end{fulllineitems}

\index{adjust\_eat\_breakfast() (omni\_trial.Omni\_Trial method)}

\begin{fulllineitems}
\phantomsection\label{\detokenize{omni_trial:omni_trial.Omni_Trial.adjust_eat_breakfast}}\pysiglinewithargsret{\sphinxbfcode{\sphinxupquote{adjust\_eat\_breakfast}}}{\emph{data}, \emph{no\_variation=False}}{}
This function adjusts the household parameters to reflect the sampled parameters         (mean and standard deviation of start time, end time, and         duration, respectively), from the CHAD data for the eating breakfast         activity.
\begin{quote}\begin{description}
\item[{Parameters}] \leavevmode\begin{itemize}
\item {} 
\sphinxstyleliteralstrong{\sphinxupquote{data}} (\sphinxstyleliteralemphasis{\sphinxupquote{tuple of numpy.ndarray}}\sphinxstyleliteralemphasis{\sphinxupquote{, }}\sphinxstyleliteralemphasis{\sphinxupquote{numpy.ndarray}}\sphinxstyleliteralemphasis{\sphinxupquote{, }}\sphinxstyleliteralemphasis{\sphinxupquote{numpy.ndarray}}\sphinxstyleliteralemphasis{\sphinxupquote{,         }}\sphinxstyleliteralemphasis{\sphinxupquote{numpy.ndarray}}\sphinxstyleliteralemphasis{\sphinxupquote{, }}\sphinxstyleliteralemphasis{\sphinxupquote{numpy.ndarray}}\sphinxstyleliteralemphasis{\sphinxupquote{, }}\sphinxstyleliteralemphasis{\sphinxupquote{numpy.ndarray}}) \textendash{} relevant parameters for each person in the household for         eating breakfast. The tuple contains the following: mean start time, standard         deviation of start time, mean end time, standard deviation of end time, mean duration,         and standard deviation of duration for each person in the household.

\item {} 
\sphinxstyleliteralstrong{\sphinxupquote{no\_variation}} (\sphinxstyleliteralemphasis{\sphinxupquote{bool}}) \textendash{} whether (if True) or not (if False) intra-individual         variation is set to zero among the activities

\end{itemize}

\item[{Returns}] \leavevmode


\end{description}\end{quote}

\end{fulllineitems}

\index{adjust\_eat\_dinner() (omni\_trial.Omni\_Trial method)}

\begin{fulllineitems}
\phantomsection\label{\detokenize{omni_trial:omni_trial.Omni_Trial.adjust_eat_dinner}}\pysiglinewithargsret{\sphinxbfcode{\sphinxupquote{adjust\_eat\_dinner}}}{\emph{data}, \emph{no\_variation=False}}{}
This function adjusts the household parameters to reflect the sampled parameters         (mean and standard deviation of start time, end time, and         duration, respectively), from the CHAD data for the eating dinner         activity.
\begin{quote}\begin{description}
\item[{Parameters}] \leavevmode\begin{itemize}
\item {} 
\sphinxstyleliteralstrong{\sphinxupquote{data}} (\sphinxstyleliteralemphasis{\sphinxupquote{tuple of numpy.ndarray}}\sphinxstyleliteralemphasis{\sphinxupquote{, }}\sphinxstyleliteralemphasis{\sphinxupquote{numpy.ndarray}}\sphinxstyleliteralemphasis{\sphinxupquote{, }}\sphinxstyleliteralemphasis{\sphinxupquote{numpy.ndarray}}\sphinxstyleliteralemphasis{\sphinxupquote{,         }}\sphinxstyleliteralemphasis{\sphinxupquote{numpy.ndarray}}\sphinxstyleliteralemphasis{\sphinxupquote{, }}\sphinxstyleliteralemphasis{\sphinxupquote{numpy.ndarray}}\sphinxstyleliteralemphasis{\sphinxupquote{, }}\sphinxstyleliteralemphasis{\sphinxupquote{numpy.ndarray}}) \textendash{} relevant parameters for each person in the household for         eating dinner. The tuple contains the following: mean start time, standard         deviation of start time, mean end time, standard deviation of end time, mean duration,         and standard deviation of duration for each person in the household.

\item {} 
\sphinxstyleliteralstrong{\sphinxupquote{no\_variation}} (\sphinxstyleliteralemphasis{\sphinxupquote{bool}}) \textendash{} whether (if True) or not (if False) intra-individual         variation is set to zero among the activities

\end{itemize}

\item[{Returns}] \leavevmode


\end{description}\end{quote}

\end{fulllineitems}

\index{adjust\_eat\_lunch() (omni\_trial.Omni\_Trial method)}

\begin{fulllineitems}
\phantomsection\label{\detokenize{omni_trial:omni_trial.Omni_Trial.adjust_eat_lunch}}\pysiglinewithargsret{\sphinxbfcode{\sphinxupquote{adjust\_eat\_lunch}}}{\emph{data}, \emph{no\_variation=False}}{}
This function adjusts the household parameters to reflect the sampled parameters         (mean and standard deviation of start time, end time, and         duration, respectively), from the CHAD data for the eating lunch         activity.
\begin{quote}\begin{description}
\item[{Parameters}] \leavevmode\begin{itemize}
\item {} 
\sphinxstyleliteralstrong{\sphinxupquote{data}} (\sphinxstyleliteralemphasis{\sphinxupquote{tuple of numpy.ndarray}}\sphinxstyleliteralemphasis{\sphinxupquote{, }}\sphinxstyleliteralemphasis{\sphinxupquote{numpy.ndarray}}\sphinxstyleliteralemphasis{\sphinxupquote{, }}\sphinxstyleliteralemphasis{\sphinxupquote{numpy.ndarray}}\sphinxstyleliteralemphasis{\sphinxupquote{,         }}\sphinxstyleliteralemphasis{\sphinxupquote{numpy.ndarray}}\sphinxstyleliteralemphasis{\sphinxupquote{, }}\sphinxstyleliteralemphasis{\sphinxupquote{numpy.ndarray}}\sphinxstyleliteralemphasis{\sphinxupquote{, }}\sphinxstyleliteralemphasis{\sphinxupquote{numpy.ndarray}}) \textendash{} relevant parameters for each person in the household for         eating lunch. The tuple contains the following: mean start time, standard         deviation of start time, mean end time, standard deviation of end time, mean duration,         and standard deviation of duration for each person in the household.

\item {} 
\sphinxstyleliteralstrong{\sphinxupquote{no\_variation}} (\sphinxstyleliteralemphasis{\sphinxupquote{bool}}) \textendash{} whether (if True) or not (if False) intra-individual         variation is set to zero among the activities

\end{itemize}

\item[{Returns}] \leavevmode


\end{description}\end{quote}

\end{fulllineitems}

\index{adjust\_params() (omni\_trial.Omni\_Trial method)}

\begin{fulllineitems}
\phantomsection\label{\detokenize{omni_trial:omni_trial.Omni_Trial.adjust_params}}\pysiglinewithargsret{\sphinxbfcode{\sphinxupquote{adjust\_params}}}{\emph{x}}{}
This function adjusts the household parameters to reflect the sampled parameters         (mean and standard deviation of start time, end time, and         duration, respectively), from the CHAD data for simulating the respective demographic         in ABMHAP.
\begin{quote}\begin{description}
\item[{Parameters}] \leavevmode
\sphinxstyleliteralstrong{\sphinxupquote{x}} (\sphinxstyleliteralemphasis{\sphinxupquote{dict that maps int to a tuple: numpy.ndarray}}\sphinxstyleliteralemphasis{\sphinxupquote{, }}\sphinxstyleliteralemphasis{\sphinxupquote{numpy.ndarray}}\sphinxstyleliteralemphasis{\sphinxupquote{, }}\sphinxstyleliteralemphasis{\sphinxupquote{numpy.ndarray}}\sphinxstyleliteralemphasis{\sphinxupquote{,         }}\sphinxstyleliteralemphasis{\sphinxupquote{numpy.ndarray}}\sphinxstyleliteralemphasis{\sphinxupquote{, }}\sphinxstyleliteralemphasis{\sphinxupquote{numpy.ndarray}}\sphinxstyleliteralemphasis{\sphinxupquote{, }}\sphinxstyleliteralemphasis{\sphinxupquote{numpy.ndarray}}) \textendash{} maps an activity code to the parameterizing CHAD data for each activity,         respectively. The CHAD data are the mean and standard deviation of the start time, end time,         and duration.

\item[{Returns}] \leavevmode


\end{description}\end{quote}

\end{fulllineitems}

\index{adjust\_params\_adult\_non\_work() (omni\_trial.Omni\_Trial method)}

\begin{fulllineitems}
\phantomsection\label{\detokenize{omni_trial:omni_trial.Omni_Trial.adjust_params_adult_non_work}}\pysiglinewithargsret{\sphinxbfcode{\sphinxupquote{adjust\_params\_adult\_non\_work}}}{\emph{x}, \emph{no\_variation=False}}{}
For the non-working adult demographic, this function adjusts the         household parameters to reflect the sampled         parameters (mean and standard deviation of the activity start time, end         time, and duration, respectively, for the following activities:
\begin{enumerate}
\item {} 
eat breakfast

\item {} 
eat lunch

\item {} 
eat dinner

\item {} 
sleep

\end{enumerate}
\begin{quote}\begin{description}
\item[{Parameters}] \leavevmode\begin{itemize}
\item {} 
\sphinxstyleliteralstrong{\sphinxupquote{x}} (\sphinxstyleliteralemphasis{\sphinxupquote{dict that maps int to a tuple: numpy.ndarray}}\sphinxstyleliteralemphasis{\sphinxupquote{, }}\sphinxstyleliteralemphasis{\sphinxupquote{numpy.ndarray}}\sphinxstyleliteralemphasis{\sphinxupquote{, }}\sphinxstyleliteralemphasis{\sphinxupquote{numpy.ndarray}}\sphinxstyleliteralemphasis{\sphinxupquote{,         }}\sphinxstyleliteralemphasis{\sphinxupquote{numpy.ndarray}}\sphinxstyleliteralemphasis{\sphinxupquote{, }}\sphinxstyleliteralemphasis{\sphinxupquote{numpy.ndarray}}\sphinxstyleliteralemphasis{\sphinxupquote{, }}\sphinxstyleliteralemphasis{\sphinxupquote{numpy.ndarray}}) \textendash{} maps an activity code to the parameterizing CHAD data for each activity,         respectively. The CHAD data are the mean and standard deviation of the start time, end time,         and duration.

\item {} 
\sphinxstyleliteralstrong{\sphinxupquote{no\_variation}} (\sphinxstyleliteralemphasis{\sphinxupquote{bool}}) \textendash{} off or on intra-individual variation among the activities

\end{itemize}

\item[{Returns}] \leavevmode


\end{description}\end{quote}

\end{fulllineitems}

\index{adjust\_params\_adult\_work() (omni\_trial.Omni\_Trial method)}

\begin{fulllineitems}
\phantomsection\label{\detokenize{omni_trial:omni_trial.Omni_Trial.adjust_params_adult_work}}\pysiglinewithargsret{\sphinxbfcode{\sphinxupquote{adjust\_params\_adult\_work}}}{\emph{x}, \emph{no\_variation=False}}{}
For the working adult demographic, this function adjusts the         household parameters to reflect the sampled         parameters (mean and standard deviation of the activity start time, end         time, and duration, respectively, for the following activities:
\begin{enumerate}
\item {} 
sleep

\item {} 
eat breakfast

\item {} 
eat lunch

\item {} 
eat dinner

\item {} 
commute to work

\item {} 
commute from work

\item {} 
work

\end{enumerate}
\begin{quote}\begin{description}
\item[{Parameters}] \leavevmode\begin{itemize}
\item {} 
\sphinxstyleliteralstrong{\sphinxupquote{x}} (\sphinxstyleliteralemphasis{\sphinxupquote{dict that maps int to a tuple: numpy.ndarray}}\sphinxstyleliteralemphasis{\sphinxupquote{, }}\sphinxstyleliteralemphasis{\sphinxupquote{numpy.ndarray}}\sphinxstyleliteralemphasis{\sphinxupquote{, }}\sphinxstyleliteralemphasis{\sphinxupquote{numpy.ndarray}}\sphinxstyleliteralemphasis{\sphinxupquote{,         }}\sphinxstyleliteralemphasis{\sphinxupquote{numpy.ndarray}}\sphinxstyleliteralemphasis{\sphinxupquote{, }}\sphinxstyleliteralemphasis{\sphinxupquote{numpy.ndarray}}\sphinxstyleliteralemphasis{\sphinxupquote{, }}\sphinxstyleliteralemphasis{\sphinxupquote{numpy.ndarray}}) \textendash{} maps an activity code to the parameterizing CHAD data for each activity,         respectively. The CHAD data are the mean and standard deviation of the start time, end time,         and duration.

\item {} 
\sphinxstyleliteralstrong{\sphinxupquote{no\_variation}} (\sphinxstyleliteralemphasis{\sphinxupquote{bool}}) \textendash{} off or on intra-individual variation among the activities

\end{itemize}

\item[{Returns}] \leavevmode


\end{description}\end{quote}

\end{fulllineitems}

\index{adjust\_params\_child\_school() (omni\_trial.Omni\_Trial method)}

\begin{fulllineitems}
\phantomsection\label{\detokenize{omni_trial:omni_trial.Omni_Trial.adjust_params_child_school}}\pysiglinewithargsret{\sphinxbfcode{\sphinxupquote{adjust\_params\_child\_school}}}{\emph{x}, \emph{no\_variation=False}}{}
For the school-age children demographic, this function adjusts the         household parameters to reflect the sampled         parameters (mean and standard deviation of the activity start time, end         time, and duration, respectively, for the following activities:
\begin{enumerate}
\item {} 
sleep

\item {} 
eat breakfast

\item {} 
eat lunch

\item {} 
eat dinner

\item {} 
commute To work

\item {} 
commute From work

\item {} 
work

\end{enumerate}
\begin{quote}\begin{description}
\item[{Parameters}] \leavevmode\begin{itemize}
\item {} 
\sphinxstyleliteralstrong{\sphinxupquote{x}} (\sphinxstyleliteralemphasis{\sphinxupquote{dict that maps int to a tuple: numpy.ndarray}}\sphinxstyleliteralemphasis{\sphinxupquote{, }}\sphinxstyleliteralemphasis{\sphinxupquote{numpy.ndarray}}\sphinxstyleliteralemphasis{\sphinxupquote{, }}\sphinxstyleliteralemphasis{\sphinxupquote{numpy.ndarray}}\sphinxstyleliteralemphasis{\sphinxupquote{,         }}\sphinxstyleliteralemphasis{\sphinxupquote{numpy.ndarray}}\sphinxstyleliteralemphasis{\sphinxupquote{, }}\sphinxstyleliteralemphasis{\sphinxupquote{numpy.ndarray}}\sphinxstyleliteralemphasis{\sphinxupquote{, }}\sphinxstyleliteralemphasis{\sphinxupquote{numpy.ndarray}}) \textendash{} maps an activity code to the parameterizing CHAD data for each activity,         respectively. The CHAD data are the mean and standard deviation of the start time, end time,         and duration.

\item {} 
\sphinxstyleliteralstrong{\sphinxupquote{no\_variation}} (\sphinxstyleliteralemphasis{\sphinxupquote{bool}}) \textendash{} off or on intra-individual variation among the activities

\end{itemize}

\item[{Returns}] \leavevmode


\end{description}\end{quote}

\end{fulllineitems}

\index{adjust\_params\_child\_young() (omni\_trial.Omni\_Trial method)}

\begin{fulllineitems}
\phantomsection\label{\detokenize{omni_trial:omni_trial.Omni_Trial.adjust_params_child_young}}\pysiglinewithargsret{\sphinxbfcode{\sphinxupquote{adjust\_params\_child\_young}}}{\emph{x}, \emph{no\_variation=False}}{}
For the preschool children demographic, this function adjusts the         household parameters to reflect the sampled         parameters (mean and standard deviation of the activity start time, end         time, and duration, respectively, for the following activities:
\begin{enumerate}
\item {} 
eat breakfast

\item {} 
eat lunch

\item {} 
eat dinner

\item {} 
sleep

\end{enumerate}
\begin{quote}\begin{description}
\item[{Parameters}] \leavevmode\begin{itemize}
\item {} 
\sphinxstyleliteralstrong{\sphinxupquote{x}} (\sphinxstyleliteralemphasis{\sphinxupquote{dict that maps int to a tuple: numpy.ndarray}}\sphinxstyleliteralemphasis{\sphinxupquote{, }}\sphinxstyleliteralemphasis{\sphinxupquote{numpy.ndarray}}\sphinxstyleliteralemphasis{\sphinxupquote{, }}\sphinxstyleliteralemphasis{\sphinxupquote{numpy.ndarray}}\sphinxstyleliteralemphasis{\sphinxupquote{,         }}\sphinxstyleliteralemphasis{\sphinxupquote{numpy.ndarray}}\sphinxstyleliteralemphasis{\sphinxupquote{, }}\sphinxstyleliteralemphasis{\sphinxupquote{numpy.ndarray}}\sphinxstyleliteralemphasis{\sphinxupquote{, }}\sphinxstyleliteralemphasis{\sphinxupquote{numpy.ndarray}}) \textendash{} maps an activity code to the parameterizing CHAD data for each activity,         respectively. The CHAD data are the mean and standard deviation of the start time, end time,         and duration.

\item {} 
\sphinxstyleliteralstrong{\sphinxupquote{no\_variation}} (\sphinxstyleliteralemphasis{\sphinxupquote{bool}}) \textendash{} off or on intra-individual variation among the activities

\end{itemize}

\item[{Returns}] \leavevmode


\end{description}\end{quote}

\end{fulllineitems}

\index{adjust\_sleep() (omni\_trial.Omni\_Trial method)}

\begin{fulllineitems}
\phantomsection\label{\detokenize{omni_trial:omni_trial.Omni_Trial.adjust_sleep}}\pysiglinewithargsret{\sphinxbfcode{\sphinxupquote{adjust\_sleep}}}{\emph{data}, \emph{no\_variation=False}}{}
This function adjusts the household parameters to reflect the sampled parameters         (mean and standard deviation of start time, end time, and         duration, respectively), from the CHAD data for the sleeping         activity.
\begin{quote}\begin{description}
\item[{Parameters}] \leavevmode\begin{itemize}
\item {} 
\sphinxstyleliteralstrong{\sphinxupquote{data}} (\sphinxstyleliteralemphasis{\sphinxupquote{tuple of numpy.ndarray}}\sphinxstyleliteralemphasis{\sphinxupquote{, }}\sphinxstyleliteralemphasis{\sphinxupquote{numpy.ndarray}}\sphinxstyleliteralemphasis{\sphinxupquote{, }}\sphinxstyleliteralemphasis{\sphinxupquote{numpy.ndarray}}\sphinxstyleliteralemphasis{\sphinxupquote{,         }}\sphinxstyleliteralemphasis{\sphinxupquote{numpy.ndarray}}\sphinxstyleliteralemphasis{\sphinxupquote{, }}\sphinxstyleliteralemphasis{\sphinxupquote{numpy.ndarray}}\sphinxstyleliteralemphasis{\sphinxupquote{, }}\sphinxstyleliteralemphasis{\sphinxupquote{numpy.ndarray}}) \textendash{} relevant parameters for each person in the household for         sleeping. The tuple contains the following: mean start time, standard         deviation of start time, mean end time, standard deviation of end time, mean duration,         and standard deviation of duration for each person in the household.

\item {} 
\sphinxstyleliteralstrong{\sphinxupquote{no\_variation}} (\sphinxstyleliteralemphasis{\sphinxupquote{bool}}) \textendash{} whether (if True) or not (if False) intra-individual         variation is set to zero among the activities

\end{itemize}

\item[{Returns}] \leavevmode


\end{description}\end{quote}

\end{fulllineitems}

\index{adjust\_work() (omni\_trial.Omni\_Trial method)}

\begin{fulllineitems}
\phantomsection\label{\detokenize{omni_trial:omni_trial.Omni_Trial.adjust_work}}\pysiglinewithargsret{\sphinxbfcode{\sphinxupquote{adjust\_work}}}{\emph{data}, \emph{no\_variation=False}}{}
This function adjusts the household parameters to reflect the sampled parameters         (mean and standard deviation of start time, end time, and         duration, respectively), from the CHAD data for the working         activity.
\begin{quote}\begin{description}
\item[{Parameters}] \leavevmode\begin{itemize}
\item {} 
\sphinxstyleliteralstrong{\sphinxupquote{data}} (\sphinxstyleliteralemphasis{\sphinxupquote{tuple of numpy.ndarray}}\sphinxstyleliteralemphasis{\sphinxupquote{, }}\sphinxstyleliteralemphasis{\sphinxupquote{numpy.ndarray}}\sphinxstyleliteralemphasis{\sphinxupquote{, }}\sphinxstyleliteralemphasis{\sphinxupquote{numpy.ndarray}}\sphinxstyleliteralemphasis{\sphinxupquote{,         }}\sphinxstyleliteralemphasis{\sphinxupquote{numpy.ndarray}}\sphinxstyleliteralemphasis{\sphinxupquote{, }}\sphinxstyleliteralemphasis{\sphinxupquote{numpy.ndarray}}\sphinxstyleliteralemphasis{\sphinxupquote{, }}\sphinxstyleliteralemphasis{\sphinxupquote{numpy.ndarray}}) \textendash{} relevant parameters for each person in the household for         working. The tuple contains the following: mean start time, standard         deviation of start time, mean end time, standard deviation of end time, mean duration,         and standard deviation of duration for each person in the household.

\item {} 
\sphinxstyleliteralstrong{\sphinxupquote{no\_variation}} (\sphinxstyleliteralemphasis{\sphinxupquote{bool}}) \textendash{} whether (if True) or not (if False) intra-individual         variation is set to zero among the activities

\end{itemize}

\item[{Returns}] \leavevmode


\end{description}\end{quote}

\end{fulllineitems}

\index{initialize() (omni\_trial.Omni\_Trial method)}

\begin{fulllineitems}
\phantomsection\label{\detokenize{omni_trial:omni_trial.Omni_Trial.initialize}}\pysiglinewithargsret{\sphinxbfcode{\sphinxupquote{initialize}}}{}{}
This function initializes the parameters for the ABMHAP simulation based on the         CHAD data for the given demographic.
\begin{quote}\begin{description}
\item[{Returns}] \leavevmode


\end{description}\end{quote}

\end{fulllineitems}


\end{fulllineitems}



\subsection{sleep\_trial module}
\label{\detokenize{sleep_trial::doc}}\label{\detokenize{sleep_trial:module-sleep_trial}}\label{\detokenize{sleep_trial:sleep-trial-module}}\index{sleep\_trial (module)}
This module contains code in order to run Monte-Carlo simulations to comparing the Agent-Based Model of Human Activity Patterns (ABMHAP) with the data from the Consolidated Human Activity Database (CHAD) for the \sphinxstylestrong{sleep} activity.

This module contains class {\hyperref[\detokenize{sleep_trial:sleep_trial.Sleep_Trial}]{\sphinxcrossref{\sphinxcode{\sphinxupquote{sleep\_trial.Sleep\_Trial}}}}}.
\index{Sleep\_Trial (class in sleep\_trial)}

\begin{fulllineitems}
\phantomsection\label{\detokenize{sleep_trial:sleep_trial.Sleep_Trial}}\pysiglinewithargsret{\sphinxbfcode{\sphinxupquote{class }}\sphinxcode{\sphinxupquote{sleep\_trial.}}\sphinxbfcode{\sphinxupquote{Sleep\_Trial}}}{\emph{parameters}, \emph{sampling\_params}, \emph{demographic}}{}
Bases: {\hyperref[\detokenize{trial:trial.Trial}]{\sphinxcrossref{\sphinxcode{\sphinxupquote{trial.Trial}}}}}

This class runs the ABMHAP simulations initialized with sleep data from CHAD.
\begin{quote}\begin{description}
\item[{Parameters}] \leavevmode\begin{itemize}
\item {} 
\sphinxstyleliteralstrong{\sphinxupquote{parameters}} ({\hyperref[\detokenize{params:params.Params}]{\sphinxcrossref{\sphinxstyleliteralemphasis{\sphinxupquote{params.Params}}}}}) \textendash{} the parameters describing each person in the household

\item {} 
\sphinxstyleliteralstrong{\sphinxupquote{sampling\_params}} ({\hyperref[\detokenize{chad_params:chad_params.CHAD_params}]{\sphinxcrossref{\sphinxstyleliteralemphasis{\sphinxupquote{chad\_params.CHAD\_params}}}}}) \textendash{} the sampling parameters used to filter “good” CHAD     sleep data

\item {} 
\sphinxstyleliteralstrong{\sphinxupquote{demographic}} (\sphinxstyleliteralemphasis{\sphinxupquote{int}}) \textendash{} the demographic identifier

\end{itemize}

\end{description}\end{quote}
\index{adjust\_params() (sleep\_trial.Sleep\_Trial method)}

\begin{fulllineitems}
\phantomsection\label{\detokenize{sleep_trial:sleep_trial.Sleep_Trial.adjust_params}}\pysiglinewithargsret{\sphinxbfcode{\sphinxupquote{adjust\_params}}}{\emph{start\_mean}, \emph{start\_std}, \emph{end\_mean}, \emph{end\_std}}{}
This function adjusts the values for the mean and standard deviation of both sleep         duration and sleep start time in the key-word arguments based on the CHAD data         that was sampled. These new values will be used in the runs.
\begin{quote}\begin{description}
\item[{Parameters}] \leavevmode\begin{itemize}
\item {} 
\sphinxstyleliteralstrong{\sphinxupquote{start\_mean}} (\sphinxstyleliteralemphasis{\sphinxupquote{numpy.ndarray}}) \textendash{} the mean sleep start time {[}hours{]} for each person

\item {} 
\sphinxstyleliteralstrong{\sphinxupquote{start\_std}} (\sphinxstyleliteralemphasis{\sphinxupquote{numpy.ndarray}}) \textendash{} the standard deviation of sleep start time {[}hours{]} for each person

\item {} 
\sphinxstyleliteralstrong{\sphinxupquote{end\_mean}} (\sphinxstyleliteralemphasis{\sphinxupquote{numpy.ndarray}}) \textendash{} the sleep mean end time {[}hours{]} for each person

\item {} 
\sphinxstyleliteralstrong{\sphinxupquote{end\_std}} (\sphinxstyleliteralemphasis{\sphinxupquote{numpy.ndarray}}) \textendash{} the sleep standard deviation of end time {[}hours{]} for each person

\end{itemize}

\item[{Returns}] \leavevmode


\end{description}\end{quote}

\end{fulllineitems}

\index{create\_universe() (sleep\_trial.Sleep\_Trial method)}

\begin{fulllineitems}
\phantomsection\label{\detokenize{sleep_trial:sleep_trial.Sleep_Trial.create_universe}}\pysiglinewithargsret{\sphinxbfcode{\sphinxupquote{create\_universe}}}{}{}
This function creates a universe object that simulations will run in. The only         asset in this simulation for an agent to use is a {\hyperref[\detokenize{bed:bed.Bed}]{\sphinxcrossref{\sphinxcode{\sphinxupquote{bed.Bed}}}}}.
\begin{quote}\begin{description}
\item[{Returns}] \leavevmode
the universe

\item[{Return type}] \leavevmode
{\hyperref[\detokenize{universe:universe.Universe}]{\sphinxcrossref{universe.Universe}}}

\end{description}\end{quote}

\end{fulllineitems}

\index{initialize() (sleep\_trial.Sleep\_Trial method)}

\begin{fulllineitems}
\phantomsection\label{\detokenize{sleep_trial:sleep_trial.Sleep_Trial.initialize}}\pysiglinewithargsret{\sphinxbfcode{\sphinxupquote{initialize}}}{}{}
This function sets up the trial
\begin{enumerate}
\item {} 
gets the CHAD data for sleep under the appropriate conditions for means and standard deviations         for both sleep duration and sleep start time

\item {} 
gets N samples the CHAD data for sleep duration and sleep start time for the N trials

\item {} 
updates the {\hyperref[\detokenize{params:module-params}]{\sphinxcrossref{\sphinxcode{\sphinxupquote{params}}}}} to reflect the newly assigned sleep parameters for the simulation

\end{enumerate}
\begin{quote}\begin{description}
\item[{Returns}] \leavevmode


\end{description}\end{quote}

\end{fulllineitems}

\index{sample\_start() (sleep\_trial.Sleep\_Trial method)}

\begin{fulllineitems}
\phantomsection\label{\detokenize{sleep_trial:sleep_trial.Sleep_Trial.sample_start}}\pysiglinewithargsret{\sphinxbfcode{\sphinxupquote{sample\_start}}}{\emph{df}, \emph{s\_params}}{}
This function is used for sampling mean and standard deviation data from start times.
\begin{quote}\begin{description}
\item[{Parameters}] \leavevmode\begin{itemize}
\item {} 
\sphinxstyleliteralstrong{\sphinxupquote{df}} (\sphinxstyleliteralemphasis{\sphinxupquote{pandas.core.frame.DataFrame}}) \textendash{} the statistical start time data

\item {} 
\sphinxstyleliteralstrong{\sphinxupquote{s\_params}} ({\hyperref[\detokenize{chad_params:chad_params.CHAD_params}]{\sphinxcrossref{\sphinxstyleliteralemphasis{\sphinxupquote{chad\_params.CHAD\_params}}}}}) \textendash{} the parameters the limit the sampling of CHAD data

\end{itemize}

\item[{Returns}] \leavevmode
the start time time data in the range {[}-12, 12) {[}hours{]}

\item[{Return type}] \leavevmode
pandas.core.frame.DataFrame

\end{description}\end{quote}

\end{fulllineitems}


\end{fulllineitems}



\subsection{trial module}
\label{\detokenize{trial::doc}}\label{\detokenize{trial:module-trial}}\label{\detokenize{trial:trial-module}}\index{trial (module)}
This is the module that is in charge of running simulations comparing the Agent-Based Model of Human Activity Patterns (ABMHAP) with the data from the Consolidated Human Activity Database (CHAD).

This module contains the class {\hyperref[\detokenize{trial:trial.Trial}]{\sphinxcrossref{\sphinxcode{\sphinxupquote{trial.Trial}}}}}.
\index{Trial (class in trial)}

\begin{fulllineitems}
\phantomsection\label{\detokenize{trial:trial.Trial}}\pysiglinewithargsret{\sphinxbfcode{\sphinxupquote{class }}\sphinxcode{\sphinxupquote{trial.}}\sphinxbfcode{\sphinxupquote{Trial}}}{\emph{parameters}, \emph{sampling\_params}, \emph{demographic}}{}
Bases: \sphinxcode{\sphinxupquote{object}}

This class is sets up runs for the ABMHAP initialized with data from CHAD.

This is how to run a trial
\begin{enumerate}
\item {} 
create the Trial object via \_\_init\_\_()

\item {} 
initialize the Trial. That is, one must set up the distribution for sampling means and standard deviations)      via initialize(). This is usually done by sending the appropriate files names to the function for the respective      distributions.

\item {} 
create the universe for the simulation

\item {} 
add the people to the household

\item {} 
run the simulation

\end{enumerate}
\begin{quote}\begin{description}
\item[{Parameters}] \leavevmode\begin{itemize}
\item {} 
\sphinxstyleliteralstrong{\sphinxupquote{params}} ({\hyperref[\detokenize{params:params.Params}]{\sphinxcrossref{\sphinxstyleliteralemphasis{\sphinxupquote{params.Params}}}}}) \textendash{} the parameters that describe the household

\item {} 
\sphinxstyleliteralstrong{\sphinxupquote{sampling\_params}} ({\hyperref[\detokenize{chad_params:chad_params.CHAD_params}]{\sphinxcrossref{\sphinxstyleliteralemphasis{\sphinxupquote{chad\_params.CHAD\_params}}}}}) \textendash{} the sampling parameters used to filter “good” CHAD activity data

\item {} 
\sphinxstyleliteralstrong{\sphinxupquote{demographic}} (\sphinxstyleliteralemphasis{\sphinxupquote{int}}) \textendash{} the demographic identifier used to parametrize the agent

\end{itemize}

\item[{Variables}] \leavevmode\begin{itemize}
\item {} 
\sphinxstyleliteralstrong{\sphinxupquote{id}} (\sphinxstyleliteralemphasis{\sphinxupquote{int}}) \textendash{} the trial identifier

\item {} 
\sphinxstyleliteralstrong{\sphinxupquote{'params'}} ({\hyperref[\detokenize{params:params.Params}]{\sphinxcrossref{\sphinxstyleliteralemphasis{\sphinxupquote{params.Params}}}}}) \textendash{} the parameters that describe the household

\item {} 
\sphinxstyleliteralstrong{\sphinxupquote{sampling\_params}} ({\hyperref[\detokenize{chad_params:chad_params.CHAD_params}]{\sphinxcrossref{\sphinxstyleliteralemphasis{\sphinxupquote{chad\_params.CHAD\_params}}}}}) \textendash{} the sampling parameters used to filter “good” CHAD data

\item {} 
\sphinxstyleliteralstrong{\sphinxupquote{num\_samples}} (\sphinxstyleliteralemphasis{\sphinxupquote{int}}) \textendash{} the number of ABMHAP samples (or trials) to be run

\item {} 
\sphinxstyleliteralstrong{\sphinxupquote{demographic}} (\sphinxstyleliteralemphasis{\sphinxupquote{int}}) \textendash{} the demographic identifier used to parametrize the agent

\item {} 
\sphinxstyleliteralstrong{\sphinxupquote{fname}} (\sphinxstyleliteralemphasis{\sphinxupquote{str}}) \textendash{} the name of the zipfile for the CHAD data

\end{itemize}

\end{description}\end{quote}
\index{add\_person\_to\_universe() (trial.Trial method)}

\begin{fulllineitems}
\phantomsection\label{\detokenize{trial:trial.Trial.add_person_to_universe}}\pysiglinewithargsret{\sphinxbfcode{\sphinxupquote{add\_person\_to\_universe}}}{\emph{u}, \emph{idx}}{}
This function creates a person and sets up the universe for simulation.

\begin{sphinxadmonition}{note}{Note:}
This function currently only assumes that each simulation has only 1 person / household.             This will need to be changed later. There will be conflicts with the idx and id.
\end{sphinxadmonition}
\begin{quote}\begin{description}
\item[{Parameters}] \leavevmode\begin{itemize}
\item {} 
\sphinxstyleliteralstrong{\sphinxupquote{u}} ({\hyperref[\detokenize{universe:universe.Universe}]{\sphinxcrossref{\sphinxstyleliteralemphasis{\sphinxupquote{universe.Universe}}}}}) \textendash{} the universe the simulation will run in

\item {} 
\sphinxstyleliteralstrong{\sphinxupquote{idx}} (\sphinxstyleliteralemphasis{\sphinxupquote{int}}) \textendash{} the index for {\hyperref[\detokenize{params:module-params}]{\sphinxcrossref{\sphinxcode{\sphinxupquote{params}}}}} to access to parametrize this person.

\end{itemize}

\item[{Return u}] \leavevmode
the updated/ initialized universe

\item[{Return type}] \leavevmode
{\hyperref[\detokenize{universe:universe.Universe}]{\sphinxcrossref{universe.Universe}}}

\end{description}\end{quote}

\end{fulllineitems}

\index{assign\_chad\_params() (trial.Trial method)}

\begin{fulllineitems}
\phantomsection\label{\detokenize{trial:trial.Trial.assign_chad_params}}\pysiglinewithargsret{\sphinxbfcode{\sphinxupquote{assign\_chad\_params}}}{\emph{z}, \emph{f\_stats}, \emph{s\_params}}{}
Assign the CHAD statistical parameters for a given activity to the agent.
\begin{quote}\begin{description}
\item[{Parameters}] \leavevmode\begin{itemize}
\item {} 
\sphinxstyleliteralstrong{\sphinxupquote{z}} (\sphinxstyleliteralemphasis{\sphinxupquote{zipfile.ZipFile}}) \textendash{} the file name (.zip) for the demographic data

\item {} 
\sphinxstyleliteralstrong{\sphinxupquote{f\_stats}} (\sphinxstyleliteralemphasis{\sphinxupquote{a dictionary of int - str}}) \textendash{} the file names of the statistical data relevant to the start time,         end time, duration, and CHAD records for a given activity

\item {} 
\sphinxstyleliteralstrong{\sphinxupquote{s\_params}} ({\hyperref[\detokenize{chad_params:chad_params.CHAD_params}]{\sphinxcrossref{\sphinxstyleliteralemphasis{\sphinxupquote{chad\_params.CHAD\_params}}}}}) \textendash{} the parameters that limit the sampling of         respective statistical data

\end{itemize}

\item[{Returns}] \leavevmode
relevant parameters for each person in the household for         a given activity. The tuple contains the following {[}in hours{]}: mean start time, standard         deviation of start time, mean end time, standard deviation of end time, mean duration,         and standard deviation of duration for each person in the household.

\item[{Rtype data}] \leavevmode
tuple of numpy.ndarray, numpy.ndarray, numpy.ndarray,         numpy.ndarray, numpy.ndarray, numpy.ndarray

\end{description}\end{quote}

\end{fulllineitems}

\index{check\_spacing() (trial.Trial method)}

\begin{fulllineitems}
\phantomsection\label{\detokenize{trial:trial.Trial.check_spacing}}\pysiglinewithargsret{\sphinxbfcode{\sphinxupquote{check\_spacing}}}{\emph{start\_mean}, \emph{start\_std}, \emph{end\_mean}, \emph{end\_std}, \emph{spacing}}{}
This is done to make sure the minimum end time does not overlap with plausible start times. The         function returns the indices of agents with a parametrization that causes this overlap. This is         a concern for activities like sleeping where the agent can be assigned to end too early after         starting the sleep too quickly.
\begin{quote}\begin{description}
\item[{Parameters}] \leavevmode\begin{itemize}
\item {} 
\sphinxstyleliteralstrong{\sphinxupquote{start\_mean}} (\sphinxstyleliteralemphasis{\sphinxupquote{numpy.ndarray}}) \textendash{} the mean start time for the given activity for each person         in the household

\item {} 
\sphinxstyleliteralstrong{\sphinxupquote{start\_std}} (\sphinxstyleliteralemphasis{\sphinxupquote{numpy.ndarray}}) \textendash{} the standard deviation of start time for the given activity         for each person in the household

\item {} 
\sphinxstyleliteralstrong{\sphinxupquote{end\_mean}} (\sphinxstyleliteralemphasis{\sphinxupquote{numpy.ndarray}}) \textendash{} the mean end time for the given activity for each person in         the household

\item {} 
\sphinxstyleliteralstrong{\sphinxupquote{end\_std}} (\sphinxstyleliteralemphasis{\sphinxupquote{numpy.ndarray}}) \textendash{} the standard deviation of end time for the given activity for         each person in the household

\item {} 
\sphinxstyleliteralstrong{\sphinxupquote{spacing}} (\sphinxstyleliteralemphasis{\sphinxupquote{float}}) \textendash{} the minimum amount

\end{itemize}

\item[{Returns}] \leavevmode
the indices of the agents with improper parametrization

\item[{Return type}] \leavevmode
numpy.ndarray

\end{description}\end{quote}

\end{fulllineitems}

\index{create\_universe() (trial.Trial method)}

\begin{fulllineitems}
\phantomsection\label{\detokenize{trial:trial.Trial.create_universe}}\pysiglinewithargsret{\sphinxbfcode{\sphinxupquote{create\_universe}}}{}{}
This function creates a universe object that simulations will run in.
\begin{quote}\begin{description}
\item[{Return u}] \leavevmode
the universe for the simulation to run in

\item[{Return type}] \leavevmode
{\hyperref[\detokenize{universe:universe.Universe}]{\sphinxcrossref{universe.Universe}}}

\end{description}\end{quote}

\end{fulllineitems}

\index{get\_chad\_stats\_data\_dt() (trial.Trial method)}

\begin{fulllineitems}
\phantomsection\label{\detokenize{trial:trial.Trial.get_chad_stats_data_dt}}\pysiglinewithargsret{\sphinxbfcode{\sphinxupquote{get\_chad\_stats\_data\_dt}}}{\emph{z}, \emph{fname}, \emph{s\_params}}{}
This function obtains the CHAD data for activity duration data that are         suitable for ABMHAP simulation.
\begin{quote}\begin{description}
\item[{Parameters}] \leavevmode\begin{itemize}
\item {} 
\sphinxstyleliteralstrong{\sphinxupquote{z}} (\sphinxstyleliteralemphasis{\sphinxupquote{zifpile.Zipfile}}) \textendash{} the zipfile of the activity data

\item {} 
\sphinxstyleliteralstrong{\sphinxupquote{fname}} (\sphinxstyleliteralemphasis{\sphinxupquote{str}}) \textendash{} the file name for the data file for activity duration

\item {} 
\sphinxstyleliteralstrong{\sphinxupquote{s\_params}} ({\hyperref[\detokenize{chad_params:chad_params.CHAD_params}]{\sphinxcrossref{\sphinxstyleliteralemphasis{\sphinxupquote{chad\_params.CHAD\_params}}}}}) \textendash{} the parameters that limit the sampling of         respective statistical data for a given activity

\end{itemize}

\item[{Returns}] \leavevmode
the CHAD data for activity duration suitable for ABMHAP simulation

\item[{Return type}] \leavevmode
pandas.core.frame.DataFrame

\end{description}\end{quote}

\end{fulllineitems}

\index{get\_chad\_stats\_data\_end() (trial.Trial method)}

\begin{fulllineitems}
\phantomsection\label{\detokenize{trial:trial.Trial.get_chad_stats_data_end}}\pysiglinewithargsret{\sphinxbfcode{\sphinxupquote{get\_chad\_stats\_data\_end}}}{\emph{z}, \emph{fname}, \emph{s\_params}}{}
This function obtains the CHAD data for activity end time data that are         suitable for ABMHAP simulation.
\begin{quote}\begin{description}
\item[{Parameters}] \leavevmode\begin{itemize}
\item {} 
\sphinxstyleliteralstrong{\sphinxupquote{z}} (\sphinxstyleliteralemphasis{\sphinxupquote{zifpile.Zipfile}}) \textendash{} the zipfile of the activity data

\item {} 
\sphinxstyleliteralstrong{\sphinxupquote{fname}} (\sphinxstyleliteralemphasis{\sphinxupquote{str}}) \textendash{} the file name for the data file for activity duration

\item {} 
\sphinxstyleliteralstrong{\sphinxupquote{s\_params}} ({\hyperref[\detokenize{chad_params:chad_params.CHAD_params}]{\sphinxcrossref{\sphinxstyleliteralemphasis{\sphinxupquote{chad\_params.CHAD\_params}}}}}) \textendash{} the parameters that limit the sampling of         respective statistical data for a given activity

\end{itemize}

\item[{Returns}] \leavevmode
the CHAD data for activity end time suitable for ABMHAP simulation

\item[{Return type}] \leavevmode
pandas.core.frame.DataFrame

\end{description}\end{quote}

\end{fulllineitems}

\index{get\_chad\_stats\_data\_start() (trial.Trial method)}

\begin{fulllineitems}
\phantomsection\label{\detokenize{trial:trial.Trial.get_chad_stats_data_start}}\pysiglinewithargsret{\sphinxbfcode{\sphinxupquote{get\_chad\_stats\_data\_start}}}{\emph{z}, \emph{fname}, \emph{s\_params}}{}
This function obtains the CHAD data for activity start time data that are         suitable for ABMHAP simulation.
\begin{quote}\begin{description}
\item[{Parameters}] \leavevmode\begin{itemize}
\item {} 
\sphinxstyleliteralstrong{\sphinxupquote{z}} (\sphinxstyleliteralemphasis{\sphinxupquote{zifpile.Zipfile}}) \textendash{} the zipfile of the activity data

\item {} 
\sphinxstyleliteralstrong{\sphinxupquote{fname}} (\sphinxstyleliteralemphasis{\sphinxupquote{str}}) \textendash{} the file name for the data file for activity duration

\item {} 
\sphinxstyleliteralstrong{\sphinxupquote{s\_params}} ({\hyperref[\detokenize{chad_params:chad_params.CHAD_params}]{\sphinxcrossref{\sphinxstyleliteralemphasis{\sphinxupquote{chad\_params.CHAD\_params}}}}}) \textendash{} the parameters that limit the sampling of         respective statistical data for a given activity

\end{itemize}

\item[{Returns}] \leavevmode
the CHAD data for activity duration suitable for ABMHAP simulation

\item[{Return type}] \leavevmode
pandas.core.frame.DataFrame

\end{description}\end{quote}

\end{fulllineitems}

\index{get\_diary() (trial.Trial method)}

\begin{fulllineitems}
\phantomsection\label{\detokenize{trial:trial.Trial.get_diary}}\pysiglinewithargsret{\sphinxbfcode{\sphinxupquote{get\_diary}}}{\emph{u}}{}
This function takes the simulation data in terms of a list of         {\hyperref[\detokenize{universe:universe.Universe}]{\sphinxcrossref{\sphinxcode{\sphinxupquote{universe.Universe}}}}} and creates a list         of {\hyperref[\detokenize{diary:diary.Diary}]{\sphinxcrossref{\sphinxcode{\sphinxupquote{diary.Diary}}}}} that contain the activity diaries.         One per each household in the simulation.
\begin{quote}\begin{description}
\item[{Parameters}] \leavevmode
\sphinxstyleliteralstrong{\sphinxupquote{u}} ({\hyperref[\detokenize{universe:universe.Universe}]{\sphinxcrossref{\sphinxstyleliteralemphasis{\sphinxupquote{universe.Universe}}}}}) \textendash{} contains all of the simulation data

\item[{Returns}] \leavevmode
the activity diaries (1 entry per person)

\item[{Return type}] \leavevmode
list of {\hyperref[\detokenize{diary:diary.Diary}]{\sphinxcrossref{\sphinxcode{\sphinxupquote{diary.Diary}}}}}

\end{description}\end{quote}

\end{fulllineitems}

\index{get\_diary\_help() (trial.Trial method)}

\begin{fulllineitems}
\phantomsection\label{\detokenize{trial:trial.Trial.get_diary_help}}\pysiglinewithargsret{\sphinxbfcode{\sphinxupquote{get\_diary\_help}}}{\emph{t}, \emph{hist\_act}, \emph{hist\_loc}}{}
This function takes data on the activity start times, activity codes, and location codes         from an activity diary and fills out the activity, minute-by-minute in between two adjacent         activities.
\begin{quote}\begin{description}
\item[{Parameters}] \leavevmode\begin{itemize}
\item {} 
\sphinxstyleliteralstrong{\sphinxupquote{t}} (\sphinxstyleliteralemphasis{\sphinxupquote{numpy.ndarray}}) \textendash{} the start time from an activity diary

\item {} 
\sphinxstyleliteralstrong{\sphinxupquote{hist\_act}} (\sphinxstyleliteralemphasis{\sphinxupquote{numpy.ndarray}}) \textendash{} the activity codes from an activity diary

\item {} 
\sphinxstyleliteralstrong{\sphinxupquote{hist\_loc}} (\sphinxstyleliteralemphasis{\sphinxupquote{numpy.ndarray}}) \textendash{} the location codes from an activity diary

\end{itemize}

\item[{Returns}] \leavevmode
the minute by minute information from an ABMHAP simulation for the         following: time information, activity codes, and location codes

\item[{Return type}] \leavevmode
numpy.ndarray, numpy.ndarray, numpy.ndarray

\end{description}\end{quote}

\end{fulllineitems}

\index{get\_stats\_data() (trial.Trial method)}

\begin{fulllineitems}
\phantomsection\label{\detokenize{trial:trial.Trial.get_stats_data}}\pysiglinewithargsret{\sphinxbfcode{\sphinxupquote{get\_stats\_data}}}{\emph{z}, \emph{f\_stats}, \emph{s\_params}}{}
Assign the CHAD statistical parameters for a given activity to the agent.
\begin{quote}\begin{description}
\item[{Parameters}] \leavevmode\begin{itemize}
\item {} 
\sphinxstyleliteralstrong{\sphinxupquote{z}} (\sphinxstyleliteralemphasis{\sphinxupquote{zipfile.ZipFile}}) \textendash{} the file name (.zip) for the demographic data

\item {} 
\sphinxstyleliteralstrong{\sphinxupquote{f\_stats}} (\sphinxstyleliteralemphasis{\sphinxupquote{a dictionary of int - str}}) \textendash{} the file names of the statistical data relevant to the start time,         end time, duration, and CHAD records for a given activity

\item {} 
\sphinxstyleliteralstrong{\sphinxupquote{s\_params}} ({\hyperref[\detokenize{chad_params:chad_params.CHAD_params}]{\sphinxcrossref{\sphinxstyleliteralemphasis{\sphinxupquote{chad\_params.CHAD\_params}}}}}) \textendash{} the parameters that limit the sampling of         respective statistical data

\end{itemize}

\item[{Returns}] \leavevmode
relevant parameters for each person in the household for         a given activity. The tuple contains the following {[}in hours{]}: mean start time, standard         deviation of start time, mean end time, standard deviation of end time, mean duration,         and standard deviation of duration for each person in the household.

\item[{Rtype data}] \leavevmode
tuple of numpy.ndarray, numpy.ndarray, numpy.ndarray,         numpy.ndarray, numpy.ndarray, numpy.ndarray

\end{description}\end{quote}

\end{fulllineitems}

\index{get\_stats\_data\_dt() (trial.Trial method)}

\begin{fulllineitems}
\phantomsection\label{\detokenize{trial:trial.Trial.get_stats_data_dt}}\pysiglinewithargsret{\sphinxbfcode{\sphinxupquote{get\_stats\_data\_dt}}}{\emph{df}, \emph{num\_people}, \emph{n\_data}}{}
This function samples the duration data from CHAD from a particular activity and         gets the mean and standard deviation of duration for the respective activity         for each person in the household.
\begin{quote}\begin{description}
\item[{Parameters}] \leavevmode\begin{itemize}
\item {} 
\sphinxstyleliteralstrong{\sphinxupquote{df}} (\sphinxstyleliteralemphasis{\sphinxupquote{pandas.core.frame.DataFrame}}) \textendash{} duration CHAD data

\item {} 
\sphinxstyleliteralstrong{\sphinxupquote{num\_people}} (\sphinxstyleliteralemphasis{\sphinxupquote{int}}) \textendash{} the number of people in the household

\item {} 
\sphinxstyleliteralstrong{\sphinxupquote{n\_data}} (\sphinxstyleliteralemphasis{\sphinxupquote{int}}) \textendash{} the minimum number of data points per         CHAD-person record used in sampling the CHAD data

\end{itemize}

\item[{Returns}] \leavevmode
the mean and standard deviation {[}in hours{]} for a given activity for         each person in the household

\item[{Return type}] \leavevmode
numpy.ndarray, numpy.ndarray

\end{description}\end{quote}

\end{fulllineitems}

\index{get\_stats\_data\_help() (trial.Trial method)}

\begin{fulllineitems}
\phantomsection\label{\detokenize{trial:trial.Trial.get_stats_data_help}}\pysiglinewithargsret{\sphinxbfcode{\sphinxupquote{get\_stats\_data\_help}}}{\emph{df}, \emph{num\_people}, \emph{n\_data}}{}
This function samples the CHAD data to obtain information         on the mean and standard deviation data. This is done by doing         the following
\begin{enumerate}
\item {} 
creating an empirical distribution for the mean and standard deviation of the data

\item {} 
randomly choosing a value out of the distribution for each agent in the household

\end{enumerate}
\begin{quote}\begin{description}
\item[{Parameters}] \leavevmode\begin{itemize}
\item {} 
\sphinxstyleliteralstrong{\sphinxupquote{df}} (\sphinxstyleliteralemphasis{\sphinxupquote{pandas.core.frame.DataFrame}}) \textendash{} the CHAD statistical data

\item {} 
\sphinxstyleliteralstrong{\sphinxupquote{num\_people}} (\sphinxstyleliteralemphasis{\sphinxupquote{int}}) \textendash{} number of people in the household

\item {} 
\sphinxstyleliteralstrong{\sphinxupquote{n\_data}} (\sphinxstyleliteralemphasis{\sphinxupquote{int}}) \textendash{} the minimum number of data points per CHAD-person record used in sampling the CHAD data

\end{itemize}

\item[{Returns}] \leavevmode
the mean and standard deviation {[}in hours{]} for a given activity for         each person in the household

\item[{Return type}] \leavevmode
numpy.ndarray, numpy.ndarray

\end{description}\end{quote}

\end{fulllineitems}

\index{get\_stats\_data\_start\_end() (trial.Trial method)}

\begin{fulllineitems}
\phantomsection\label{\detokenize{trial:trial.Trial.get_stats_data_start_end}}\pysiglinewithargsret{\sphinxbfcode{\sphinxupquote{get\_stats\_data\_start\_end}}}{\emph{df\_start}, \emph{df\_end}, \emph{num\_people}, \emph{n\_data}}{}
This function samples data for activities that are parametrized         by both start time and end time activity-parameters.
\begin{quote}\begin{description}
\item[{Parameters}] \leavevmode\begin{itemize}
\item {} 
\sphinxstyleliteralstrong{\sphinxupquote{df\_start}} (\sphinxstyleliteralemphasis{\sphinxupquote{pandas.core.frame.DataFrame}}) \textendash{} the CHAD data for start time {[}hours{]}

\item {} 
\sphinxstyleliteralstrong{\sphinxupquote{df\_end}} (\sphinxstyleliteralemphasis{\sphinxupquote{pandas.core.frame.DataFrame}}) \textendash{} the CHAD data for end time {[}hours{]}

\item {} 
\sphinxstyleliteralstrong{\sphinxupquote{num\_people}} (\sphinxstyleliteralemphasis{\sphinxupquote{int}}) \textendash{} the number of people in the household

\item {} 
\sphinxstyleliteralstrong{\sphinxupquote{n\_data}} (\sphinxstyleliteralemphasis{\sphinxupquote{int}}) \textendash{} the number of data points to be considered “longitudinal”

\end{itemize}

\item[{Returns}] \leavevmode
the mean and standard deviation for the start time and end time respectively

\item[{Return type}] \leavevmode
numpy.ndarray, numpy.ndarray, numpy.ndarray, numpy.ndarray

\end{description}\end{quote}

\end{fulllineitems}

\index{get\_stats\_data\_start\_end\_help() (trial.Trial method)}

\begin{fulllineitems}
\phantomsection\label{\detokenize{trial:trial.Trial.get_stats_data_start_end_help}}\pysiglinewithargsret{\sphinxbfcode{\sphinxupquote{get\_stats\_data\_start\_end\_help}}}{\emph{df\_start}, \emph{df\_end}, \emph{num\_people}, \emph{n\_data}}{}
This function samples data for activities that are parametrized         by both start time and end time activity-parameters.
\begin{quote}\begin{description}
\item[{Parameters}] \leavevmode\begin{itemize}
\item {} 
\sphinxstyleliteralstrong{\sphinxupquote{df\_start}} (\sphinxstyleliteralemphasis{\sphinxupquote{pandas.core.frame.DataFrame}}) \textendash{} the CHAD data for start time {[}hours{]}

\item {} 
\sphinxstyleliteralstrong{\sphinxupquote{df\_end}} (\sphinxstyleliteralemphasis{\sphinxupquote{pandas.core.frame.DataFrame}}) \textendash{} the CHAD data for end time {[}hours{]}

\item {} 
\sphinxstyleliteralstrong{\sphinxupquote{num\_people}} (\sphinxstyleliteralemphasis{\sphinxupquote{int}}) \textendash{} the number of people in the household

\item {} 
\sphinxstyleliteralstrong{\sphinxupquote{n\_data}} (\sphinxstyleliteralemphasis{\sphinxupquote{int}}) \textendash{} the number of data points to be or not be considered “longitudinal”

\end{itemize}

\item[{Returns}] \leavevmode
the mean and standard deviation for the start time and end time respectively

\item[{Return type}] \leavevmode
numpy.ndarray, numpy.ndarray, numpy.ndarray, numpy.ndarray

\end{description}\end{quote}

\end{fulllineitems}

\index{initialize() (trial.Trial method)}

\begin{fulllineitems}
\phantomsection\label{\detokenize{trial:trial.Trial.initialize}}\pysiglinewithargsret{\sphinxbfcode{\sphinxupquote{initialize}}}{\emph{demo}}{}
This function initializes each activity in the trial for a given demographic by         using CHAD data to parametrize the activity-parameters (i.e., the mean         and standard deviation of star time, end time, and duration).
\begin{quote}\begin{description}
\item[{Parameters}] \leavevmode
\sphinxstyleliteralstrong{\sphinxupquote{demo}} ({\hyperref[\detokenize{chad_demography:chad_demography.CHAD_demography}]{\sphinxcrossref{\sphinxstyleliteralemphasis{\sphinxupquote{chad\_demography.CHAD\_demography}}}}}) \textendash{} contains much information about the demographic

\item[{Returns}] \leavevmode
a dictionary containing a tuple of the mean duration, standard deviation of duration,         mean start time, standard deviation of start time (in hours, float)

\item[{Return type}] \leavevmode
a dictionary of int to numpy.ndarray, numpy.ndarray, numpy.ndarray, numpy.

\end{description}\end{quote}

\end{fulllineitems}

\index{initialize\_person() (trial.Trial method)}

\begin{fulllineitems}
\phantomsection\label{\detokenize{trial:trial.Trial.initialize_person}}\pysiglinewithargsret{\sphinxbfcode{\sphinxupquote{initialize\_person}}}{\emph{u}, \emph{idx}}{}
This function creates and initializes an agent with the proper parameters for         simulation.

More specifically, the function does
\begin{enumerate}
\item {} 
creates the agent

\item {} 
initializes the agent’s parameters to the respective values in {\hyperref[\detokenize{params:module-params}]{\sphinxcrossref{\sphinxcode{\sphinxupquote{params}}}}}

\end{enumerate}
\begin{quote}\begin{description}
\item[{Parameters}] \leavevmode\begin{itemize}
\item {} 
\sphinxstyleliteralstrong{\sphinxupquote{u}} ({\hyperref[\detokenize{universe:universe.Universe}]{\sphinxcrossref{\sphinxstyleliteralemphasis{\sphinxupquote{universe.Universe}}}}}) \textendash{} the universe the agent will reside in

\item {} 
\sphinxstyleliteralstrong{\sphinxupquote{idx}} (\sphinxstyleliteralemphasis{\sphinxupquote{int}}) \textendash{} the index of the agent within the household

\end{itemize}

\item[{Return p}] \leavevmode
the agent

\item[{Return type}] \leavevmode
{\hyperref[\detokenize{singleton:singleton.Singleton}]{\sphinxcrossref{singleton.Singleton}}}

\end{description}\end{quote}

\end{fulllineitems}

\index{pseudo\_intraindividual\_variation() (trial.Trial method)}

\begin{fulllineitems}
\phantomsection\label{\detokenize{trial:trial.Trial.pseudo_intraindividual_variation}}\pysiglinewithargsret{\sphinxbfcode{\sphinxupquote{pseudo\_intraindividual\_variation}}}{\emph{start\_mean}, \emph{end\_mean}}{}
This function assigns intraindividual variation for start time and end time based data where
there is \sphinxstylestrong{no} longitudinal data (hence the name “pseudo”). The variation is assigned by
having the following assumptions:
\begin{enumerate}
\item {} 
Given that the mean start time and end time are assigned

\item {} 
Calculate the mean duration based on the mean start time and mean end time

\item {} \begin{description}
\item[{Calculate the variance of the start time and end time with the following assumptions}] \leavevmode\begin{itemize}
\item {} 
assume that start time and end time are independent

\item {} 
variance of start time is equal to the variance of the end time

\item {} 
standard deviation of the duration is set to be the coefficient of variation times             the previously calculated mean duration

\end{itemize}

\end{description}

\end{enumerate}

These assumptions are expressed mathematically below where
\begin{itemize}
\item {} 
\(X_{start}, X_{end}, X_{\Delta{t}}\) are random variables for the start time, end time,         and duration, respectively

\item {} 
\(\sigma^2, \sigma, c_v\) are the variance, standard deviation, and coefficient of variation

\item {} 
\(E[\cdot], Cov(\cdot, \cdot)\) are the expected value operator and covariance  operator

\end{itemize}

Given \(X_{start}\) and \(X_{end}\),

Let,
\begin{equation*}
\begin{split}X_{\Delta{t}} = X_{end} - X_{start}\end{split}
\end{equation*}
Then,
\begin{equation*}
\begin{split}\sigma^2_{\Delta{t}} = \sigma^2_{start}  + \sigma^2_{end} - 2*Cov(X_{start}, X_{end})\end{split}
\end{equation*}
Assuming \(X_{start}\) and \(X_{end}\) are independent, then,
\begin{equation*}
\begin{split}\sigma^2_{\Delta{t}} = \sigma^2_{start}  + \sigma^2_{end}\end{split}
\end{equation*}
Assuming \(\sigma^2_{start}  = \sigma^2_{end}\), then,
\begin{equation*}
\begin{split}\sigma^2_{\Delta{t}} = 2\sigma^2_{start}\end{split}
\end{equation*}
Finally,
\begin{equation*}
\begin{split}\sigma_{start} &= \frac{ \sigma_{\Delta{t}} }{ \sqrt{2} } \\
\sigma_{start} = \sigma_{end} &= \frac{ c_v E[ X_{\Delta{t}} ] }{ \sqrt{2} }\end{split}
\end{equation*}\begin{quote}\begin{description}
\item[{Parameters}] \leavevmode\begin{itemize}
\item {} 
\sphinxstyleliteralstrong{\sphinxupquote{start\_mean}} (\sphinxstyleliteralemphasis{\sphinxupquote{numpy.ndarray}}) \textendash{} the mean start time {[}in hours{]} for each person bing parametrized

\item {} 
\sphinxstyleliteralstrong{\sphinxupquote{end\_mean}} (\sphinxstyleliteralemphasis{\sphinxupquote{numpy.ndarray}}) \textendash{} the mean end time {[}in hours{]} for each person being parametrized

\end{itemize}

\item[{Returns}] \leavevmode
standard deviation for start time and end time, respectively for each person being parametrized

\item[{Rytpe}] \leavevmode
numpy.ndarray, numpy.ndarray

\end{description}\end{quote}

\end{fulllineitems}

\index{run() (trial.Trial method)}

\begin{fulllineitems}
\phantomsection\label{\detokenize{trial:trial.Trial.run}}\pysiglinewithargsret{\sphinxbfcode{\sphinxupquote{run}}}{}{}
This function runs 1 simulations of the ABMHAP using data from         CHAD. The function can handle having more than 1 person in the household.

More specifically the function does the following for each simulation:
\begin{enumerate}
\item {} 
creates the universe

\item {} 
create / initialize the person

\item {} 
run the ABMHAP simulation

\item {} 
store the results / data from the simulation

\end{enumerate}
\begin{quote}\begin{description}
\item[{Return u}] \leavevmode
the results of the simulation

\item[{Return type}] \leavevmode
{\hyperref[\detokenize{universe:universe.Universe}]{\sphinxcrossref{universe.Universe}}}

\end{description}\end{quote}

\end{fulllineitems}

\index{sample() (trial.Trial method)}

\begin{fulllineitems}
\phantomsection\label{\detokenize{trial:trial.Trial.sample}}\pysiglinewithargsret{\sphinxbfcode{\sphinxupquote{sample}}}{\emph{df}}{}
This function samples the statistical data (of activity moments) from the CHAD diaries.

The function samples the \sphinxstylestrong{distributions} of both the means and the         the standard deviations independently of each other.
\begin{quote}\begin{description}
\item[{Parameters}] \leavevmode
\sphinxstyleliteralstrong{\sphinxupquote{df}} (\sphinxstyleliteralemphasis{\sphinxupquote{pandas.core.frame.DataFrame}}) \textendash{} a list of statistical data (mean, standard deviation,         coefficient of variation) for activity information (duration, start, or end)

\item[{Returns}] \leavevmode
values for the mean, standard deviation, and coefficient of variation, respectively

\item[{Return type}] \leavevmode
numpy.ndarray, numpy.ndarray, numpy.ndarray

\end{description}\end{quote}

\end{fulllineitems}


\end{fulllineitems}



\subsection{variation module}
\label{\detokenize{variation::doc}}\label{\detokenize{variation:module-variation}}\label{\detokenize{variation:variation-module}}\index{variation (module)}
\begin{sphinxadmonition}{warning}{Warning:}
This file as antiquated and needs to be \sphinxstylestrong{REMOVED}.
\end{sphinxadmonition}
\index{f() (in module variation)}

\begin{fulllineitems}
\phantomsection\label{\detokenize{variation:variation.f}}\pysiglinewithargsret{\sphinxcode{\sphinxupquote{variation.}}\sphinxbfcode{\sphinxupquote{f}}}{\emph{x}}{}~\begin{quote}\begin{description}
\item[{Parameters}] \leavevmode
\sphinxstyleliteralstrong{\sphinxupquote{x}} (\sphinxstyleliteralemphasis{\sphinxupquote{numpy.ndarray}}) \textendash{} the standard deviation a

\item[{Returns}] \leavevmode


\end{description}\end{quote}

\end{fulllineitems}

\index{integrate\_residual() (in module variation)}

\begin{fulllineitems}
\phantomsection\label{\detokenize{variation:variation.integrate_residual}}\pysiglinewithargsret{\sphinxcode{\sphinxupquote{variation.}}\sphinxbfcode{\sphinxupquote{integrate\_residual}}}{\emph{result}, \emph{df\_obs}, \emph{act\_code}, \emph{do\_periodic}, \emph{do\_weekday}, \emph{do\_duration}, \emph{N=10001}}{}
\end{fulllineitems}

\index{run\_initial() (in module variation)}

\begin{fulllineitems}
\phantomsection\label{\detokenize{variation:variation.run_initial}}\pysiglinewithargsret{\sphinxcode{\sphinxupquote{variation.}}\sphinxbfcode{\sphinxupquote{run\_initial}}}{\emph{trials}, \emph{chad\_param\_list}, \emph{do\_print=True}, \emph{num\_cpu=1}, \emph{pool=None}}{}
\end{fulllineitems}

\index{run\_simulation() (in module variation)}

\begin{fulllineitems}
\phantomsection\label{\detokenize{variation:variation.run_simulation}}\pysiglinewithargsret{\sphinxcode{\sphinxupquote{variation.}}\sphinxbfcode{\sphinxupquote{run\_simulation}}}{\emph{t\_0}, \emph{u\_list}, \emph{chad\_param\_list}, \emph{num\_cpu=1}, \emph{pool=None}, \emph{do\_print=True}}{}
\end{fulllineitems}

\index{run\_trial\_parallel() (in module variation)}

\begin{fulllineitems}
\phantomsection\label{\detokenize{variation:variation.run_trial_parallel}}\pysiglinewithargsret{\sphinxcode{\sphinxupquote{variation.}}\sphinxbfcode{\sphinxupquote{run\_trial\_parallel}}}{\emph{t}}{}
\end{fulllineitems}

\index{run\_uni\_parallel() (in module variation)}

\begin{fulllineitems}
\phantomsection\label{\detokenize{variation:variation.run_uni_parallel}}\pysiglinewithargsret{\sphinxcode{\sphinxupquote{variation.}}\sphinxbfcode{\sphinxupquote{run\_uni\_parallel}}}{\emph{t}}{}
\end{fulllineitems}

\index{run\_universe\_parallel() (in module variation)}

\begin{fulllineitems}
\phantomsection\label{\detokenize{variation:variation.run_universe_parallel}}\pysiglinewithargsret{\sphinxcode{\sphinxupquote{variation.}}\sphinxbfcode{\sphinxupquote{run\_universe\_parallel}}}{\emph{u}}{}
\end{fulllineitems}

\index{sweep() (in module variation)}

\begin{fulllineitems}
\phantomsection\label{\detokenize{variation:variation.sweep}}\pysiglinewithargsret{\sphinxcode{\sphinxupquote{variation.}}\sphinxbfcode{\sphinxupquote{sweep}}}{\emph{x}, \emph{chad\_param\_list}, \emph{u\_list}}{}
\end{fulllineitems}

\index{sweep\_parallel() (in module variation)}

\begin{fulllineitems}
\phantomsection\label{\detokenize{variation:variation.sweep_parallel}}\pysiglinewithargsret{\sphinxcode{\sphinxupquote{variation.}}\sphinxbfcode{\sphinxupquote{sweep\_parallel}}}{\emph{x}}{}
\end{fulllineitems}



\subsection{work\_trial module}
\label{\detokenize{work_trial::doc}}\label{\detokenize{work_trial:module-work_trial}}\label{\detokenize{work_trial:work-trial-module}}\index{work\_trial (module)}
This module contains code in order to run Monte-Carlo simulations to comparing the Agent-Based Model of Human Activity Patterns (ABMHAP) with the data from the Consolidated Human Activity Database (CHAD) for the \sphinxstylestrong{work} activity.

This module contains class {\hyperref[\detokenize{work_trial:work_trial.Work_Trial}]{\sphinxcrossref{\sphinxcode{\sphinxupquote{work\_trial.Work\_Trial}}}}}.
\index{Work\_Trial (class in work\_trial)}

\begin{fulllineitems}
\phantomsection\label{\detokenize{work_trial:work_trial.Work_Trial}}\pysiglinewithargsret{\sphinxbfcode{\sphinxupquote{class }}\sphinxcode{\sphinxupquote{work\_trial.}}\sphinxbfcode{\sphinxupquote{Work\_Trial}}}{\emph{parameters}, \emph{sampling\_params}, \emph{demographic}}{}
Bases: {\hyperref[\detokenize{trial:trial.Trial}]{\sphinxcrossref{\sphinxcode{\sphinxupquote{trial.Trial}}}}}

This class runs the ABM simulations initialized with work data from CHAD.
\begin{quote}\begin{description}
\item[{Parameters}] \leavevmode\begin{itemize}
\item {} 
\sphinxstyleliteralstrong{\sphinxupquote{paramters}} ({\hyperref[\detokenize{params:params.Params}]{\sphinxcrossref{\sphinxstyleliteralemphasis{\sphinxupquote{params.Params}}}}}) \textendash{} the parameters describing each person in the household

\item {} 
\sphinxstyleliteralstrong{\sphinxupquote{sampling\_params}} ({\hyperref[\detokenize{chad_params:chad_params.CHAD_params}]{\sphinxcrossref{\sphinxstyleliteralemphasis{\sphinxupquote{chad\_params.CHAD\_params}}}}}) \textendash{} the sampling parameters used to filter “good” CHAD     work data

\item {} 
\sphinxstyleliteralstrong{\sphinxupquote{demographic}} (\sphinxstyleliteralemphasis{\sphinxupquote{int}}) \textendash{} the demographic identifier

\end{itemize}

\end{description}\end{quote}
\index{adjust\_params() (work\_trial.Work\_Trial method)}

\begin{fulllineitems}
\phantomsection\label{\detokenize{work_trial:work_trial.Work_Trial.adjust_params}}\pysiglinewithargsret{\sphinxbfcode{\sphinxupquote{adjust\_params}}}{\emph{start\_mean}, \emph{start\_std}, \emph{end\_mean}, \emph{end\_std}}{}
This function adjusts the values for the mean and standard deviation of both work         duration and work start time in the key-word arguments based on the CHAD data         that was sampled. These new values will be used in the runs.
\begin{quote}\begin{description}
\item[{Parameters}] \leavevmode\begin{itemize}
\item {} 
\sphinxstyleliteralstrong{\sphinxupquote{dt\_mean}} (\sphinxstyleliteralemphasis{\sphinxupquote{numpy.ndarray}}) \textendash{} the work duration mean {[}minutes{]} for each person

\item {} 
\sphinxstyleliteralstrong{\sphinxupquote{dt\_std}} (\sphinxstyleliteralemphasis{\sphinxupquote{numpy.ndarray}}) \textendash{} the work duration standard deviation {[}minutes{]} for each person

\item {} 
\sphinxstyleliteralstrong{\sphinxupquote{start\_mean}} (\sphinxstyleliteralemphasis{\sphinxupquote{numpy.ndarray}}) \textendash{} the mean work start time {[}minutes{]} for each person

\item {} 
\sphinxstyleliteralstrong{\sphinxupquote{start\_std}} (\sphinxstyleliteralemphasis{\sphinxupquote{numpy.ndarray}}) \textendash{} the standard deviation of start time {[}minutes{]} for each person

\end{itemize}

\item[{Returns}] \leavevmode


\end{description}\end{quote}

\end{fulllineitems}

\index{create\_universe() (work\_trial.Work\_Trial method)}

\begin{fulllineitems}
\phantomsection\label{\detokenize{work_trial:work_trial.Work_Trial.create_universe}}\pysiglinewithargsret{\sphinxbfcode{\sphinxupquote{create\_universe}}}{}{}
This function creates a universe object that simulations will run in. The assets that this simulation
uses in {\hyperref[\detokenize{workplace:workplace.Workplace}]{\sphinxcrossref{\sphinxcode{\sphinxupquote{workplace.Workplace}}}}} and {\hyperref[\detokenize{transport:transport.Transport}]{\sphinxcrossref{\sphinxcode{\sphinxupquote{transport.Transport()}}}}}.
\begin{quote}\begin{description}
\item[{Returns}] \leavevmode
the universe

\item[{Return type}] \leavevmode
{\hyperref[\detokenize{universe:universe.Universe}]{\sphinxcrossref{universe.Universe}}}

\end{description}\end{quote}

\end{fulllineitems}

\index{initialize() (work\_trial.Work\_Trial method)}

\begin{fulllineitems}
\phantomsection\label{\detokenize{work_trial:work_trial.Work_Trial.initialize}}\pysiglinewithargsret{\sphinxbfcode{\sphinxupquote{initialize}}}{}{}
This function sets up the trial.
\begin{enumerate}
\item {} 
gets the CHAD data for work under the appropriate conditions for means and standard deviations         for both work duration and sleep start time

\item {} 
gets N samples the CHAD data for work duration and work start time for the N trials

\item {} 
updates the {\hyperref[\detokenize{params:module-params}]{\sphinxcrossref{\sphinxcode{\sphinxupquote{params}}}}} to reflect the newly assigned sleep parameters for the simulation

\end{enumerate}
\begin{quote}\begin{description}
\item[{Returns}] \leavevmode


\end{description}\end{quote}

\end{fulllineitems}


\end{fulllineitems}



\section{Plotting Directory}
\label{\detokenize{index:plotting-directory}}
These functions handle plotting capabilities.

Contents:


\subsection{plot\_diary notebook}
\label{\detokenize{plot_diary::doc}}\label{\detokenize{plot_diary:plot-diary-notebook}}
\fvset{hllines={, ,}}%
\begin{sphinxVerbatim}[commandchars=\\\{\}]
\PYG{c+c1}{\PYGZsh{} The United States Environmental Protection Agency through its Office of}
\PYG{c+c1}{\PYGZsh{} Research and Development has developed this software. The code is made}
\PYG{c+c1}{\PYGZsh{} publicly available to better communicate the research. All input data}
\PYG{c+c1}{\PYGZsh{} used fora given application should be reviewed by the researcher so}
\PYG{c+c1}{\PYGZsh{} that the model results are based on appropriate data for any given}
\PYG{c+c1}{\PYGZsh{} application. This model is under continued development. The model and}
\PYG{c+c1}{\PYGZsh{} data included herein do not represent and should not be construed to}
\PYG{c+c1}{\PYGZsh{} represent any Agency determination or policy.}
\PYG{c+c1}{\PYGZsh{}}
\PYG{c+c1}{\PYGZsh{} This file was written by Dr. Namdi Brandon}
\PYG{c+c1}{\PYGZsh{} ORCID: 0000\PYGZhy{}0001\PYGZhy{}7050\PYGZhy{}1538}
\PYG{c+c1}{\PYGZsh{} March 20, 2018}
\end{sphinxVerbatim}

This file contains the functions necessary to visualize the activity
diaries.

Import

\fvset{hllines={, ,}}%
\begin{sphinxVerbatim}[commandchars=\\\{\}]
\PYG{k+kn}{import} \PYG{n+nn}{sys}
\PYG{n}{sys}\PYG{o}{.}\PYG{n}{path}\PYG{o}{.}\PYG{n}{append}\PYG{p}{(}\PYG{l+s+s1}{\PYGZsq{}}\PYG{l+s+s1}{..}\PYG{l+s+se}{\PYGZbs{}\PYGZbs{}}\PYG{l+s+s1}{source}\PYG{l+s+s1}{\PYGZsq{}}\PYG{p}{)}

\PYG{c+c1}{\PYGZsh{} plotting functions}
\PYG{k+kn}{import} \PYG{n+nn}{matplotlib}\PYG{n+nn}{.}\PYG{n+nn}{pylab} \PYG{k}{as} \PYG{n+nn}{plt}

\PYG{c+c1}{\PYGZsh{} mathematical capability}
\PYG{k+kn}{import} \PYG{n+nn}{numpy} \PYG{k}{as} \PYG{n+nn}{np}

\PYG{c+c1}{\PYGZsh{} dataframe capability}
\PYG{k+kn}{import} \PYG{n+nn}{pandas} \PYG{k}{as} \PYG{n+nn}{pd}

\PYG{c+c1}{\PYGZsh{} agent\PYGZhy{}based model modules}
\PYG{k+kn}{import} \PYG{n+nn}{my\PYGZus{}globals} \PYG{k}{as} \PYG{n+nn}{mg}
\PYG{k+kn}{import} \PYG{n+nn}{activity}\PYG{o}{,} \PYG{n+nn}{temporal}
\end{sphinxVerbatim}

\fvset{hllines={, ,}}%
\begin{sphinxVerbatim}[commandchars=\\\{\}]
\PYG{c+c1}{\PYGZsh{} plotting scheme}
\PYG{o}{\PYGZpc{}}\PYG{k}{matplotlib} auto
\end{sphinxVerbatim}

\fvset{hllines={, ,}}%
\begin{sphinxVerbatim}[commandchars=\\\{\}]
\PYG{n}{Using} \PYG{n}{matplotlib} \PYG{n}{backend}\PYG{p}{:} \PYG{n}{Qt5Agg}
\end{sphinxVerbatim}

Functions

\fvset{hllines={, ,}}%
\begin{sphinxVerbatim}[commandchars=\\\{\}]
\PYG{k}{def} \PYG{n+nf}{plot\PYGZus{}activity\PYGZus{}diary}\PYG{p}{(}\PYG{n}{df}\PYG{p}{,} \PYG{n}{show\PYGZus{}legend}\PYG{o}{=}\PYG{k+kc}{False}\PYG{p}{,} \PYG{n}{fontsize}\PYG{o}{=}\PYG{l+m+mi}{8}\PYG{p}{,} \PYG{n}{dpi}\PYG{o}{=}\PYG{l+m+mi}{300}\PYG{p}{)}\PYG{p}{:}

    \PYG{l+s+sd}{\PYGZdq{}\PYGZdq{}\PYGZdq{}}
\PYG{l+s+sd}{    This function plots the activity diary for a given agent. The information is represented}
\PYG{l+s+sd}{    in terms of horizontal barcharts in which the agent is performing an activity and where}
\PYG{l+s+sd}{    the x\PYGZhy{}axis is the time of day (in hours).}

\PYG{l+s+sd}{    :param pandas.core.frame.DataFrame df: the activity diary of a given agent}
\PYG{l+s+sd}{    :param bool show\PYGZus{}legend: a flag indicating whether (if True) or not (if False) to show \PYGZbs{}}
\PYG{l+s+sd}{    the legend in the plot}
\PYG{l+s+sd}{    :param int fontsize: the font size of the text within the plot}
\PYG{l+s+sd}{    :param int dpi: the resolution of the plot in dots per inch}

\PYG{l+s+sd}{    :return: a tuple of a list of the lines that were plotted AND a list of the labels. This \PYGZbs{}}
\PYG{l+s+sd}{    information is used in plotting the legend seperately}
\PYG{l+s+sd}{    \PYGZdq{}\PYGZdq{}\PYGZdq{}}

    \PYG{c+c1}{\PYGZsh{} set the font size for ticks, labels, titles, and legend}
    \PYG{n}{fontsize\PYGZus{}ticks} \PYG{o}{=} \PYG{n}{fontsize}
    \PYG{n}{fontsize\PYGZus{}title} \PYG{o}{=} \PYG{n}{fontsize}
    \PYG{n}{fontsize\PYGZus{}label} \PYG{o}{=} \PYG{n}{fontsize}
    \PYG{n}{fontsize\PYGZus{}title} \PYG{o}{=} \PYG{n}{fontsize}
    \PYG{n}{fontsize\PYGZus{}legend} \PYG{o}{=} \PYG{n}{fontsize}

    \PYG{c+c1}{\PYGZsh{} set font axis parameters}
    \PYG{n}{font\PYGZus{}axis} \PYG{o}{=} \PYG{p}{\PYGZob{}}\PYG{l+s+s1}{\PYGZsq{}}\PYG{l+s+s1}{family}\PYG{l+s+s1}{\PYGZsq{}}\PYG{p}{:} \PYG{l+s+s1}{\PYGZsq{}}\PYG{l+s+s1}{serif}\PYG{l+s+s1}{\PYGZsq{}}\PYG{p}{,}
        \PYG{l+s+s1}{\PYGZsq{}}\PYG{l+s+s1}{color}\PYG{l+s+s1}{\PYGZsq{}}\PYG{p}{:}  \PYG{l+s+s1}{\PYGZsq{}}\PYG{l+s+s1}{black}\PYG{l+s+s1}{\PYGZsq{}}\PYG{p}{,}
        \PYG{l+s+s1}{\PYGZsq{}}\PYG{l+s+s1}{weight}\PYG{l+s+s1}{\PYGZsq{}}\PYG{p}{:} \PYG{l+s+s1}{\PYGZsq{}}\PYG{l+s+s1}{normal}\PYG{l+s+s1}{\PYGZsq{}}\PYG{p}{,}
        \PYG{l+s+s1}{\PYGZsq{}}\PYG{l+s+s1}{size}\PYG{l+s+s1}{\PYGZsq{}}\PYG{p}{:} \PYG{n}{fontsize\PYGZus{}ticks}\PYG{p}{,}\PYG{p}{\PYGZcb{}}

    \PYG{c+c1}{\PYGZsh{}}
    \PYG{c+c1}{\PYGZsh{} plot horizontal bars using matplotlib}
    \PYG{c+c1}{\PYGZsh{}}

    \PYG{c+c1}{\PYGZsh{} create the plot}
    \PYG{n}{f}\PYG{p}{,} \PYG{n}{ax} \PYG{o}{=} \PYG{n}{plt}\PYG{o}{.}\PYG{n}{subplots}\PYG{p}{(}\PYG{n}{dpi}\PYG{o}{=}\PYG{n}{dpi}\PYG{p}{)}

    \PYG{c+c1}{\PYGZsh{} a list of the lines plotted}
    \PYG{n}{lines} \PYG{o}{=} \PYG{n+nb}{list}\PYG{p}{(}\PYG{p}{)}
    \PYG{n}{align} \PYG{o}{=} \PYG{l+s+s1}{\PYGZsq{}}\PYG{l+s+s1}{center}\PYG{l+s+s1}{\PYGZsq{}}

    \PYG{c+c1}{\PYGZsh{} the labels in chornological order}
    \PYG{n}{labels} \PYG{o}{=} \PYG{p}{[} \PYG{n}{activity}\PYG{o}{.}\PYG{n}{INT\PYGZus{}2\PYGZus{}STR}\PYG{p}{[}\PYG{n}{x}\PYG{p}{]} \PYG{k}{for} \PYG{n}{x} \PYG{o+ow}{in} \PYG{n}{df}\PYG{o}{.}\PYG{n}{act}\PYG{o}{.}\PYG{n}{unique}\PYG{p}{(}\PYG{p}{)} \PYG{p}{]}

    \PYG{c+c1}{\PYGZsh{} set the label for \PYGZdq{}no actviity\PYGZdq{} to \PYGZdq{}Idle\PYGZdq{}}
    \PYG{k}{for} \PYG{n}{i}\PYG{p}{,} \PYG{n}{x} \PYG{o+ow}{in} \PYG{n+nb}{enumerate}\PYG{p}{(}\PYG{n}{labels}\PYG{p}{)}\PYG{p}{:}
        \PYG{k}{if} \PYG{n}{x} \PYG{o}{==} \PYG{n}{activity}\PYG{o}{.}\PYG{n}{INT\PYGZus{}2\PYGZus{}STR}\PYG{p}{[}\PYG{n}{activity}\PYG{o}{.}\PYG{n}{NO\PYGZus{}ACTIVITY}\PYG{p}{]}\PYG{p}{:}
            \PYG{n}{labels}\PYG{p}{[}\PYG{n}{i}\PYG{p}{]} \PYG{o}{=} \PYG{l+s+s1}{\PYGZsq{}}\PYG{l+s+s1}{Idle}\PYG{l+s+s1}{\PYGZsq{}}

    \PYG{c+c1}{\PYGZsh{} the flag to indicate whether the figure lines will be used for the legend}
    \PYG{n}{do\PYGZus{}legend} \PYG{o}{=} \PYG{p}{[} \PYG{p}{(}\PYG{n}{x}\PYG{p}{,} \PYG{k+kc}{True}\PYG{p}{)} \PYG{k}{for} \PYG{n}{x} \PYG{o+ow}{in} \PYG{n}{df}\PYG{o}{.}\PYG{n}{act}\PYG{o}{.}\PYG{n}{unique}\PYG{p}{(}\PYG{p}{)}\PYG{p}{]}
    \PYG{n}{do\PYGZus{}legend} \PYG{o}{=} \PYG{n+nb}{dict}\PYG{p}{(}\PYG{n}{do\PYGZus{}legend}\PYG{p}{)}

    \PYG{c+c1}{\PYGZsh{} plot the diaries}
    \PYG{k}{for} \PYG{n}{i} \PYG{o+ow}{in} \PYG{n+nb}{range}\PYG{p}{(} \PYG{n+nb}{len}\PYG{p}{(}\PYG{n}{df}\PYG{p}{)} \PYG{p}{)}\PYG{p}{:}

        \PYG{c+c1}{\PYGZsh{} get the activity entry}
        \PYG{n}{x} \PYG{o}{=} \PYG{n}{df}\PYG{o}{.}\PYG{n}{iloc}\PYG{p}{[}\PYG{n}{i}\PYG{p}{]}

        \PYG{c+c1}{\PYGZsh{} get the corresponding color and label}
        \PYG{n}{color} \PYG{o}{=} \PYG{n}{activity}\PYG{o}{.}\PYG{n}{INT\PYGZus{}2\PYGZus{}COLOR}\PYG{p}{[}\PYG{n}{x}\PYG{o}{.}\PYG{n}{act}\PYG{p}{]}
        \PYG{n}{label} \PYG{o}{=} \PYG{n}{activity}\PYG{o}{.}\PYG{n}{INT\PYGZus{}2\PYGZus{}STR}\PYG{p}{[}\PYG{n}{x}\PYG{o}{.}\PYG{n}{act}\PYG{p}{]}

        \PYG{c+c1}{\PYGZsh{} for the first entry}
        \PYG{k}{if} \PYG{n}{i} \PYG{o}{==} \PYG{l+m+mi}{0}\PYG{p}{:}
            \PYG{c+c1}{\PYGZsh{} plot the entry in the beginning of the bar chart}
            \PYG{n}{p} \PYG{o}{=} \PYG{n}{ax}\PYG{o}{.}\PYG{n}{barh}\PYG{p}{(}\PYG{n}{x}\PYG{o}{.}\PYG{n}{day}\PYG{p}{,} \PYG{n}{x}\PYG{o}{.}\PYG{n}{start}\PYG{p}{,} \PYG{n}{color}\PYG{o}{=}\PYG{n}{color}\PYG{p}{,} \PYG{n}{label}\PYG{o}{=}\PYG{n}{label}\PYG{p}{,} \PYG{n}{left}\PYG{o}{=}\PYG{n}{x}\PYG{o}{.}\PYG{n}{start}\PYG{p}{,} \PYG{n}{align}\PYG{o}{=}\PYG{n}{align}\PYG{p}{)}

        \PYG{k}{else}\PYG{p}{:}
            \PYG{c+c1}{\PYGZsh{} if the activity starts on one day and ends on the next,}
            \PYG{k}{if} \PYG{n}{x}\PYG{o}{.}\PYG{n}{start} \PYG{o}{\PYGZgt{}} \PYG{n}{x}\PYG{o}{.}\PYG{n}{end}\PYG{p}{:}
                \PYG{c+c1}{\PYGZsh{} plot the activity entry until midnight on the first days bar chart and}
                \PYG{c+c1}{\PYGZsh{} and starting at midnight on the next day\PYGZsq{}s bar chart}
                \PYG{n}{p} \PYG{o}{=} \PYG{n}{ax}\PYG{o}{.}\PYG{n}{barh}\PYG{p}{(}\PYG{n}{x}\PYG{o}{.}\PYG{n}{day}\PYG{p}{,} \PYG{n}{x}\PYG{o}{.}\PYG{n}{start}\PYG{p}{,} \PYG{n}{left}\PYG{o}{=}\PYG{n}{df}\PYG{o}{.}\PYG{n}{iloc}\PYG{p}{[}\PYG{n}{i}\PYG{o}{\PYGZhy{}}\PYG{l+m+mi}{1}\PYG{p}{]}\PYG{o}{.}\PYG{n}{end}\PYG{p}{,} \PYG{n}{color}\PYG{o}{=}\PYG{n}{color}\PYG{p}{,} \PYG{n}{label}\PYG{o}{=}\PYG{n}{label}\PYG{p}{,} \PYG{n}{align}\PYG{o}{=}\PYG{n}{align}\PYG{p}{)}
                \PYG{n}{ax}\PYG{o}{.}\PYG{n}{barh}\PYG{p}{(}\PYG{n}{x}\PYG{o}{.}\PYG{n}{day}\PYG{o}{+}\PYG{l+m+mi}{1}\PYG{p}{,} \PYG{n}{x}\PYG{o}{.}\PYG{n}{end}\PYG{p}{,} \PYG{n}{left}\PYG{o}{=}\PYG{l+m+mi}{0}\PYG{p}{,} \PYG{n}{color}\PYG{o}{=}\PYG{n}{color}\PYG{p}{,} \PYG{n}{label}\PYG{o}{=}\PYG{n}{label}\PYG{p}{,} \PYG{n}{align}\PYG{o}{=}\PYG{n}{align}\PYG{p}{)}

            \PYG{k}{else}\PYG{p}{:}
                \PYG{c+c1}{\PYGZsh{} add the activity entry to the current day\PYGZsq{}s bar chart}
                \PYG{n}{p} \PYG{o}{=} \PYG{n}{ax}\PYG{o}{.}\PYG{n}{barh}\PYG{p}{(}\PYG{n}{x}\PYG{o}{.}\PYG{n}{day}\PYG{p}{,} \PYG{n}{x}\PYG{o}{.}\PYG{n}{start}\PYG{p}{,} \PYG{n}{left}\PYG{o}{=}\PYG{n}{df}\PYG{o}{.}\PYG{n}{iloc}\PYG{p}{[}\PYG{n}{i}\PYG{o}{\PYGZhy{}}\PYG{l+m+mi}{1}\PYG{p}{]}\PYG{o}{.}\PYG{n}{end}\PYG{p}{,} \PYG{n}{color}\PYG{o}{=}\PYG{n}{color}\PYG{p}{,} \PYG{n}{label}\PYG{o}{=}\PYG{n}{label}\PYG{p}{,} \PYG{n}{align}\PYG{o}{=}\PYG{n}{align}\PYG{p}{)}

        \PYG{c+c1}{\PYGZsh{} if it\PYGZsq{}s the first time an activity is plotted, add it to the legend.}
        \PYG{k}{if} \PYG{n}{do\PYGZus{}legend}\PYG{p}{[}\PYG{n}{x}\PYG{o}{.}\PYG{n}{act}\PYG{p}{]}\PYG{p}{:}
            \PYG{n}{lines}\PYG{o}{.}\PYG{n}{append}\PYG{p}{(}\PYG{n}{p}\PYG{p}{)}
            \PYG{n}{do\PYGZus{}legend}\PYG{p}{[}\PYG{n}{x}\PYG{o}{.}\PYG{n}{act}\PYG{p}{]} \PYG{o}{=} \PYG{k+kc}{False}

    \PYG{c+c1}{\PYGZsh{}}
    \PYG{c+c1}{\PYGZsh{} handle the text related to plotting}
    \PYG{c+c1}{\PYGZsh{}}

    \PYG{c+c1}{\PYGZsh{} set the title}
    \PYG{n}{f}\PYG{o}{.}\PYG{n}{suptitle}\PYG{p}{(}\PYG{l+s+s1}{\PYGZsq{}}\PYG{l+s+s1}{Daily Activity Diary}\PYG{l+s+s1}{\PYGZsq{}}\PYG{p}{,} \PYG{n}{fontsize}\PYG{o}{=}\PYG{n}{fontsize\PYGZus{}title}\PYG{p}{)}

    \PYG{c+c1}{\PYGZsh{} create the legend}
    \PYG{k}{if} \PYG{n}{show\PYGZus{}legend}\PYG{p}{:}
        \PYG{n}{f}\PYG{o}{.}\PYG{n}{legend}\PYG{p}{(}\PYG{n}{lines}\PYG{p}{,} \PYG{n}{labels}\PYG{p}{,} \PYG{l+s+s1}{\PYGZsq{}}\PYG{l+s+s1}{best}\PYG{l+s+s1}{\PYGZsq{}}\PYG{p}{,} \PYG{n}{fontsize}\PYG{o}{=}\PYG{n}{fontsize\PYGZus{}legend}\PYG{p}{)}

    \PYG{c+c1}{\PYGZsh{} set the x limits}
    \PYG{n}{ax}\PYG{o}{.}\PYG{n}{set\PYGZus{}xlim}\PYG{p}{(} \PYG{p}{[}\PYG{l+m+mi}{0}\PYG{p}{,} \PYG{l+m+mi}{24}\PYG{p}{]}\PYG{p}{)}

    \PYG{c+c1}{\PYGZsh{} set the x tick\PYGZhy{}marks}
    \PYG{n}{xticks} \PYG{o}{=} \PYG{n}{np}\PYG{o}{.}\PYG{n}{linspace}\PYG{p}{(}\PYG{l+m+mi}{0}\PYG{p}{,} \PYG{l+m+mi}{24}\PYG{p}{,} \PYG{l+m+mi}{9}\PYG{p}{)}
    \PYG{n}{ax}\PYG{o}{.}\PYG{n}{set\PYGZus{}xticks}\PYG{p}{(}\PYG{n}{xticks}\PYG{p}{)}

    \PYG{c+c1}{\PYGZsh{} set the font size of the x ticks}
    \PYG{n}{ax}\PYG{o}{.}\PYG{n}{tick\PYGZus{}params}\PYG{p}{(}\PYG{n}{axis}\PYG{o}{=}\PYG{l+s+s1}{\PYGZsq{}}\PYG{l+s+s1}{both}\PYG{l+s+s1}{\PYGZsq{}}\PYG{p}{,} \PYG{n}{labelsize}\PYG{o}{=}\PYG{n}{fontsize\PYGZus{}ticks}\PYG{p}{)}

    \PYG{c+c1}{\PYGZsh{} label axes}
    \PYG{n}{ax}\PYG{o}{.}\PYG{n}{set\PYGZus{}xlabel}\PYG{p}{(}\PYG{l+s+s1}{\PYGZsq{}}\PYG{l+s+s1}{Time [h]}\PYG{l+s+s1}{\PYGZsq{}}\PYG{p}{,} \PYG{n}{fontdict}\PYG{o}{=}\PYG{n}{font\PYGZus{}axis}\PYG{p}{)}
    \PYG{n}{ax}\PYG{o}{.}\PYG{n}{set\PYGZus{}ylabel}\PYG{p}{(}\PYG{l+s+s1}{\PYGZsq{}}\PYG{l+s+s1}{Day}\PYG{l+s+s1}{\PYGZsq{}}\PYG{p}{,} \PYG{n}{fontdict}\PYG{o}{=}\PYG{n}{font\PYGZus{}axis}\PYG{p}{)}

    \PYG{c+c1}{\PYGZsh{} invert yaxis}
    \PYG{n}{ax}\PYG{o}{.}\PYG{n}{invert\PYGZus{}yaxis}\PYG{p}{(}\PYG{p}{)}

    \PYG{k}{return} \PYG{n}{lines}\PYG{p}{,} \PYG{n}{labels}

\PYG{k}{def} \PYG{n+nf}{plot\PYGZus{}longitude}\PYG{p}{(}\PYG{n}{data}\PYG{p}{,} \PYG{n}{titles}\PYG{p}{,} \PYG{n}{linewidth}\PYG{o}{=}\PYG{l+m+mi}{1}\PYG{p}{)}\PYG{p}{:}

    \PYG{l+s+sd}{\PYGZdq{}\PYGZdq{}\PYGZdq{}}
\PYG{l+s+sd}{    This function plots a chart showing the amount of time spent during each activity. The x\PYGZhy{}axis is the}
\PYG{l+s+sd}{    time in hours and the y\PYGZhy{}axis is the duration (in minutes) represented in a log10 scale.}

\PYG{l+s+sd}{    :param list data: a list of dataframes where each dataframe represents an activity diary of an agent.}
\PYG{l+s+sd}{    :param list titles: a list of titles for each plot}
\PYG{l+s+sd}{    :param int linewidth: the linewidth of the lines within the plot}
\PYG{l+s+sd}{    \PYGZdq{}\PYGZdq{}\PYGZdq{}}

    \PYG{c+c1}{\PYGZsh{} the number of rows and columns (the dimensions) for the subplots}
    \PYG{n}{nrows}\PYG{p}{,} \PYG{n}{ncols} \PYG{o}{=} \PYG{l+m+mi}{1}\PYG{p}{,} \PYG{l+m+mi}{1}

    \PYG{c+c1}{\PYGZsh{}}
    \PYG{c+c1}{\PYGZsh{} create axes}
    \PYG{c+c1}{\PYGZsh{}}
    \PYG{n}{f}\PYG{p}{,} \PYG{n}{ax} \PYG{o}{=} \PYG{n}{plt}\PYG{o}{.}\PYG{n}{subplots}\PYG{p}{(}\PYG{n}{nrows}\PYG{p}{,} \PYG{n}{ncols}\PYG{p}{,} \PYG{n}{sharex}\PYG{o}{=}\PYG{k+kc}{True}\PYG{p}{,} \PYG{n}{sharey}\PYG{o}{=}\PYG{k+kc}{True}\PYG{p}{)}


    \PYG{c+c1}{\PYGZsh{} plot the graphs}
    \PYG{n}{K} \PYG{o}{=} \PYG{p}{[} \PYG{n}{plot\PYGZus{}longitude\PYGZus{}help}\PYG{p}{(}\PYG{n}{ax}\PYG{p}{,} \PYG{n}{data}\PYG{p}{[}\PYG{n}{i}\PYG{p}{]}\PYG{p}{,} \PYG{n}{linewidth}\PYG{p}{)} \PYG{k}{for} \PYG{n}{i}\PYG{p}{,} \PYG{n}{ax} \PYG{o+ow}{in} \PYG{n+nb}{enumerate}\PYG{p}{(}\PYG{n}{f}\PYG{o}{.}\PYG{n}{axes}\PYG{p}{)}\PYG{p}{]}

    \PYG{c+c1}{\PYGZsh{} the number of unique activities, including idle time}
    \PYG{n}{K0} \PYG{o}{=} \PYG{n}{data}\PYG{p}{[}\PYG{l+m+mi}{0}\PYG{p}{]}\PYG{o}{.}\PYG{n}{act}\PYG{o}{.}\PYG{n}{unique}\PYG{p}{(}\PYG{p}{)}

    \PYG{c+c1}{\PYGZsh{} a list of each activity expressed as a string}
    \PYG{n}{keys} \PYG{o}{=} \PYG{p}{[} \PYG{n}{activity}\PYG{o}{.}\PYG{n}{INT\PYGZus{}2\PYGZus{}STR}\PYG{p}{[}\PYG{n}{k}\PYG{p}{]} \PYG{k}{for} \PYG{n}{k} \PYG{o+ow}{in} \PYG{n}{K0}\PYG{p}{]}
    \PYG{n+nb}{print}\PYG{p}{(}\PYG{n}{keys}\PYG{p}{)}

    \PYG{c+c1}{\PYGZsh{} show the legend}
    \PYG{n}{f}\PYG{o}{.}\PYG{n}{legend}\PYG{p}{(} \PYG{n}{f}\PYG{o}{.}\PYG{n}{axes}\PYG{p}{[}\PYG{l+m+mi}{0}\PYG{p}{]}\PYG{o}{.}\PYG{n}{lines}\PYG{p}{,} \PYG{n}{keys}\PYG{p}{,} \PYG{l+s+s1}{\PYGZsq{}}\PYG{l+s+s1}{best}\PYG{l+s+s1}{\PYGZsq{}} \PYG{p}{)}

    \PYG{c+c1}{\PYGZsh{} the subplot title size}
    \PYG{n}{fontsize\PYGZus{}title}\PYG{o}{=}\PYG{l+m+mi}{18}

    \PYG{c+c1}{\PYGZsh{} the tick size}
    \PYG{n}{ticksize}\PYG{o}{=}\PYG{l+m+mi}{14}

    \PYG{c+c1}{\PYGZsh{} for each plot, set the font size and the tick size}
    \PYG{k}{for} \PYG{n}{i}\PYG{p}{,} \PYG{n}{ax} \PYG{o+ow}{in} \PYG{n+nb}{enumerate}\PYG{p}{(}\PYG{n}{f}\PYG{o}{.}\PYG{n}{axes}\PYG{p}{)}\PYG{p}{:}
        \PYG{n}{ax}\PYG{o}{.}\PYG{n}{set\PYGZus{}title}\PYG{p}{(}\PYG{n}{titles}\PYG{p}{[}\PYG{n}{i}\PYG{p}{]}\PYG{p}{,} \PYG{n}{fontsize}\PYG{o}{=}\PYG{n}{fontsize\PYGZus{}title}\PYG{p}{)}
        \PYG{n}{ax}\PYG{o}{.}\PYG{n}{tick\PYGZus{}params}\PYG{p}{(}\PYG{n}{axis}\PYG{o}{=}\PYG{l+s+s1}{\PYGZsq{}}\PYG{l+s+s1}{both}\PYG{l+s+s1}{\PYGZsq{}}\PYG{p}{,} \PYG{n}{labelsize}\PYG{o}{=}\PYG{n}{ticksize}\PYG{p}{)}

    \PYG{c+c1}{\PYGZsh{} set the main title}
    \PYG{n}{f}\PYG{o}{.}\PYG{n}{suptitle}\PYG{p}{(}\PYG{l+s+s1}{\PYGZsq{}}\PYG{l+s+s1}{Daily Activity Duration}\PYG{l+s+s1}{\PYGZsq{}}\PYG{p}{,} \PYG{n}{fontsize}\PYG{o}{=}\PYG{n}{fontsize\PYGZus{}title}\PYG{p}{)}

    \PYG{c+c1}{\PYGZsh{} write axes for x and y}
    \PYG{n}{df} \PYG{o}{=} \PYG{n}{data}\PYG{p}{[}\PYG{l+m+mi}{0}\PYG{p}{]}
    \PYG{n}{xlabel}\PYG{p}{,} \PYG{n}{ylabel} \PYG{o}{=} \PYG{l+s+s1}{\PYGZsq{}}\PYG{l+s+s1}{Day}\PYG{l+s+s1}{\PYGZsq{}}\PYG{p}{,} \PYG{l+s+s1}{\PYGZsq{}}\PYG{l+s+s1}{Duration [minutes]}\PYG{l+s+s1}{\PYGZsq{}}
    \PYG{n}{x\PYGZus{}min}\PYG{p}{,} \PYG{n}{x\PYGZus{}max} \PYG{o}{=} \PYG{n}{df}\PYG{o}{.}\PYG{n}{day}\PYG{o}{.}\PYG{n}{values}\PYG{p}{[}\PYG{l+m+mi}{0}\PYG{p}{]}\PYG{p}{,} \PYG{n}{df}\PYG{o}{.}\PYG{n}{day}\PYG{o}{.}\PYG{n}{values}\PYG{p}{[}\PYG{o}{\PYGZhy{}}\PYG{l+m+mi}{1}\PYG{p}{]}

    \PYG{c+c1}{\PYGZsh{}}
    \PYG{c+c1}{\PYGZsh{} set the x and y axes}
    \PYG{c+c1}{\PYGZsh{}}

    \PYG{c+c1}{\PYGZsh{} the y\PYGZhy{}label size}
    \PYG{n}{fontsize\PYGZus{}label} \PYG{o}{=} \PYG{l+m+mi}{18}

    \PYG{c+c1}{\PYGZsh{} set the ylabel}
    \PYG{n}{ax}\PYG{o}{.}\PYG{n}{set\PYGZus{}ylabel}\PYG{p}{(}\PYG{n}{ylabel}\PYG{p}{,} \PYG{n}{fontsize}\PYG{o}{=}\PYG{n}{fontsize\PYGZus{}label}\PYG{p}{)}

    \PYG{k}{return}

\PYG{k}{def} \PYG{n+nf}{plot\PYGZus{}longitude\PYGZus{}help}\PYG{p}{(}\PYG{n}{ax}\PYG{p}{,} \PYG{n}{df}\PYG{p}{,} \PYG{n}{linewidth}\PYG{o}{=}\PYG{l+m+mi}{1}\PYG{p}{)}\PYG{p}{:}

    \PYG{l+s+sd}{\PYGZdq{}\PYGZdq{}\PYGZdq{}}
\PYG{l+s+sd}{    This function actually handles plotting the longitude plot. This is to be used in}
\PYG{l+s+sd}{    plot\PYGZus{}longitude(). For each activity, the function plots the respectivie activity\PYGZhy{}duration}
\PYG{l+s+sd}{    on a long10 scale on each day.}

\PYG{l+s+sd}{    :param matplotlib.axes.\PYGZus{}subplots.AxesSubplot ax: for plotting object}
\PYG{l+s+sd}{    :param pandas.core.frame.DataFrame df: the activity diary of a given agent}
\PYG{l+s+sd}{    :param int linewidth: the linewidth of the lines within the plot}

\PYG{l+s+sd}{    :return: a list of the unique activity codes in the activity diary}
\PYG{l+s+sd}{    \PYGZdq{}\PYGZdq{}\PYGZdq{}}

    \PYG{n}{colors} \PYG{o}{=} \PYG{n}{activity}\PYG{o}{.}\PYG{n}{INT\PYGZus{}2\PYGZus{}COLOR}

    \PYG{c+c1}{\PYGZsh{} the days in the simulation}
    \PYG{n}{days} \PYG{o}{=} \PYG{n}{df}\PYG{o}{.}\PYG{n}{day}\PYG{o}{.}\PYG{n}{unique}\PYG{p}{(}\PYG{p}{)}

    \PYG{c+c1}{\PYGZsh{} the activities that were done by the person in the simulation}
    \PYG{n}{keys} \PYG{o}{=} \PYG{n}{df}\PYG{o}{.}\PYG{n}{act}\PYG{o}{.}\PYG{n}{unique}\PYG{p}{(}\PYG{p}{)}

    \PYG{c+c1}{\PYGZsh{} group activities by day}
    \PYG{n}{gb} \PYG{o}{=} \PYG{n}{df}\PYG{o}{.}\PYG{n}{groupby}\PYG{p}{(}\PYG{l+s+s1}{\PYGZsq{}}\PYG{l+s+s1}{day}\PYG{l+s+s1}{\PYGZsq{}}\PYG{p}{)}

    \PYG{c+c1}{\PYGZsh{} for each activity, plot the duration}
    \PYG{k}{for} \PYG{n}{k} \PYG{o+ow}{in} \PYG{n}{keys}\PYG{p}{:}

        \PYG{c+c1}{\PYGZsh{} the duration data}
        \PYG{n}{y} \PYG{o}{=} \PYG{n}{np}\PYG{o}{.}\PYG{n}{zeros}\PYG{p}{(}\PYG{n}{days}\PYG{o}{.}\PYG{n}{shape}\PYG{p}{)}

        \PYG{c+c1}{\PYGZsh{} for each day}
        \PYG{k}{for} \PYG{n}{i}\PYG{p}{,} \PYG{n}{d} \PYG{o+ow}{in} \PYG{n+nb}{enumerate}\PYG{p}{(}\PYG{n}{days}\PYG{p}{)}\PYG{p}{:}

            \PYG{c+c1}{\PYGZsh{} get the activity data for the given day}
            \PYG{n}{temp} \PYG{o}{=} \PYG{n}{gb}\PYG{o}{.}\PYG{n}{get\PYGZus{}group}\PYG{p}{(}\PYG{n}{d}\PYG{p}{)}
            \PYG{n}{temp} \PYG{o}{=} \PYG{n}{temp}\PYG{p}{[}\PYG{n}{temp}\PYG{o}{.}\PYG{n}{act} \PYG{o}{==} \PYG{n}{k}\PYG{p}{]}

            \PYG{c+c1}{\PYGZsh{} if there the respectivie activity does not happen that day, return NaN}
            \PYG{c+c1}{\PYGZsh{} this allows python to avoid plotting the activity on that specific day}
            \PYG{k}{if} \PYG{n}{temp}\PYG{o}{.}\PYG{n}{size} \PYG{o}{==} \PYG{l+m+mi}{0}\PYG{p}{:}
                \PYG{n}{dt} \PYG{o}{=} \PYG{n}{np}\PYG{o}{.}\PYG{n}{nan}
            \PYG{k}{else}\PYG{p}{:}
                \PYG{n}{dt} \PYG{o}{=} \PYG{n}{temp}\PYG{o}{.}\PYG{n}{dt}\PYG{o}{.}\PYG{n}{values}\PYG{o}{.}\PYG{n}{sum}\PYG{p}{(}\PYG{p}{)}

            \PYG{c+c1}{\PYGZsh{} convert the duration from hours to minutes}
            \PYG{n}{y}\PYG{p}{[}\PYG{n}{i}\PYG{p}{]} \PYG{o}{=} \PYG{n}{temporal}\PYG{o}{.}\PYG{n}{HOUR\PYGZus{}2\PYGZus{}MIN} \PYG{o}{*} \PYG{n}{dt}

        \PYG{c+c1}{\PYGZsh{} plot the data for the kth activity on a log10 scale}
        \PYG{n}{ax}\PYG{o}{.}\PYG{n}{plot}\PYG{p}{(}\PYG{n}{days}\PYG{p}{,} \PYG{n}{np}\PYG{o}{.}\PYG{n}{log10}\PYG{p}{(}\PYG{n}{y}\PYG{p}{)}\PYG{p}{,} \PYG{l+s+s1}{\PYGZsq{}}\PYG{l+s+s1}{\PYGZhy{}*}\PYG{l+s+s1}{\PYGZsq{}}\PYG{p}{,} \PYG{n}{label}\PYG{o}{=}\PYG{n}{activity}\PYG{o}{.}\PYG{n}{INT\PYGZus{}2\PYGZus{}STR}\PYG{p}{[}\PYG{n}{k}\PYG{p}{]}\PYG{p}{,} \PYG{n}{color}\PYG{o}{=}\PYG{n}{colors}\PYG{p}{[}\PYG{n}{k}\PYG{p}{]}\PYG{p}{,} \PYG{n}{linewidth}\PYG{o}{=}\PYG{n}{linewidth}\PYG{p}{)}

    \PYG{k}{return} \PYG{n}{keys}
\end{sphinxVerbatim}

Run

Load Activity Diary

\fvset{hllines={, ,}}%
\begin{sphinxVerbatim}[commandchars=\\\{\}]
\PYG{c+c1}{\PYGZsh{} the file name of the activity diary}
\PYG{n}{fname} \PYG{o}{=} \PYG{n}{mg}\PYG{o}{.}\PYG{n}{FDIR\PYGZus{}MY\PYGZus{}DATA} \PYG{o}{+} \PYG{l+s+s1}{\PYGZsq{}}\PYG{l+s+se}{\PYGZbs{}\PYGZbs{}}\PYG{l+s+s1}{main\PYGZus{}result.csv}\PYG{l+s+s1}{\PYGZsq{}}

\PYG{c+c1}{\PYGZsh{} load the activity diary as a dataframe}
\PYG{n}{df} \PYG{o}{=} \PYG{n}{pd}\PYG{o}{.}\PYG{n}{read\PYGZus{}csv}\PYG{p}{(}\PYG{n}{fname}\PYG{p}{)}
\end{sphinxVerbatim}

Plot the activity diary

\fvset{hllines={, ,}}%
\begin{sphinxVerbatim}[commandchars=\\\{\}]
\PYG{c+c1}{\PYGZsh{} figure resolution [ dots per inch (dpi) ]}
\PYG{c+c1}{\PYGZsh{} dpi needs to be at least 300 for submission to some journals}
\PYG{n}{dpi}\PYG{o}{=}\PYG{l+m+mi}{300}

\PYG{c+c1}{\PYGZsh{} font size of text within the figure}
\PYG{n}{fontsize} \PYG{o}{=} \PYG{l+m+mi}{8}

\PYG{c+c1}{\PYGZsh{} plot the activity diary}
\PYG{n}{lines}\PYG{p}{,} \PYG{n}{labels} \PYG{o}{=} \PYG{n}{plot\PYGZus{}activity\PYGZus{}diary}\PYG{p}{(}\PYG{n}{df}\PYG{p}{,} \PYG{n}{dpi}\PYG{o}{=}\PYG{n}{dpi}\PYG{p}{,} \PYG{n}{fontsize}\PYG{o}{=}\PYG{l+m+mi}{8}\PYG{p}{)}

\PYG{c+c1}{\PYGZsh{} show the plot}
\PYG{n}{plt}\PYG{o}{.}\PYG{n}{show}\PYG{p}{(}\PYG{p}{)}
\end{sphinxVerbatim}

Isolate the legend

\fvset{hllines={, ,}}%
\begin{sphinxVerbatim}[commandchars=\\\{\}]
\PYG{c+c1}{\PYGZsh{} create the plot}
\PYG{n}{fig}\PYG{p}{,} \PYG{n}{ax} \PYG{o}{=} \PYG{n}{plt}\PYG{o}{.}\PYG{n}{subplots}\PYG{p}{(}\PYG{n}{dpi}\PYG{o}{=}\PYG{n}{dpi}\PYG{p}{)}

\PYG{c+c1}{\PYGZsh{} plot the legend}
\PYG{n}{fig}\PYG{o}{.}\PYG{n}{legend}\PYG{p}{(}\PYG{n}{lines}\PYG{p}{,} \PYG{n}{labels}\PYG{p}{,} \PYG{l+s+s1}{\PYGZsq{}}\PYG{l+s+s1}{best}\PYG{l+s+s1}{\PYGZsq{}}\PYG{p}{,} \PYG{n}{fontsize}\PYG{o}{=}\PYG{n}{fontsize}\PYG{p}{)}

\PYG{c+c1}{\PYGZsh{} do not plot anything else}
\PYG{n}{ax}\PYG{o}{.}\PYG{n}{set\PYGZus{}xticks}\PYG{p}{(}\PYG{p}{[}\PYG{p}{]}\PYG{p}{)}
\PYG{n}{ax}\PYG{o}{.}\PYG{n}{set\PYGZus{}yticks}\PYG{p}{(}\PYG{p}{[}\PYG{p}{]}\PYG{p}{)}
\PYG{n}{ax}\PYG{o}{.}\PYG{n}{axis}\PYG{p}{(}\PYG{l+s+s1}{\PYGZsq{}}\PYG{l+s+s1}{off}\PYG{l+s+s1}{\PYGZsq{}}\PYG{p}{)}

\PYG{c+c1}{\PYGZsh{} show the plot}
\PYG{n}{plt}\PYG{o}{.}\PYG{n}{show}\PYG{p}{(}\PYG{p}{)}
\end{sphinxVerbatim}
\begin{sphinxalltt}
C:UsersnbrandonAppDataLocalContinuumAnaconda3libsite-packagesmatplotliblegend.py:338: UserWarning: Automatic legend placement (loc="best") not implemented for figure legend. Falling back on "upper right".
  warnings.warn('Automatic legend placement (loc="best") not '
\end{sphinxalltt}

Longitudinal Activity-Duration Plots

\fvset{hllines={, ,}}%
\begin{sphinxVerbatim}[commandchars=\\\{\}]
\PYG{c+c1}{\PYGZsh{}}
\PYG{c+c1}{\PYGZsh{} plot longitudinal plots of the daily activities}
\PYG{c+c1}{\PYGZsh{}}

\PYG{c+c1}{\PYGZsh{} the title}
\PYG{n}{titles} \PYG{o}{=} \PYG{p}{(}\PYG{l+s+s1}{\PYGZsq{}}\PYG{l+s+s1}{Working Adult}\PYG{l+s+s1}{\PYGZsq{}}\PYG{p}{,}\PYG{p}{)}

\PYG{c+c1}{\PYGZsh{} the activity data}
\PYG{n}{data} \PYG{o}{=} \PYG{p}{(}\PYG{n}{df}\PYG{p}{,}\PYG{p}{)}

\PYG{c+c1}{\PYGZsh{} the width of the lines in the plots}
\PYG{n}{linewidth} \PYG{o}{=} \PYG{l+m+mi}{1}

\PYG{c+c1}{\PYGZsh{} plot the activity durations}
\PYG{n}{plot\PYGZus{}longitude}\PYG{p}{(}\PYG{n}{data}\PYG{o}{=}\PYG{n}{data}\PYG{p}{,} \PYG{n}{titles}\PYG{o}{=}\PYG{n}{titles}\PYG{p}{,} \PYG{n}{linewidth}\PYG{o}{=}\PYG{n}{linewidth}\PYG{p}{)}

\PYG{c+c1}{\PYGZsh{} show the plot}
\PYG{n}{plt}\PYG{o}{.}\PYG{n}{show}\PYG{p}{(}\PYG{p}{)}
\end{sphinxVerbatim}

\fvset{hllines={, ,}}%
\begin{sphinxVerbatim}[commandchars=\\\{\}]
\PYG{p}{[}\PYG{l+s+s1}{\PYGZsq{}}\PYG{l+s+s1}{No Activity}\PYG{l+s+s1}{\PYGZsq{}}\PYG{p}{,} \PYG{l+s+s1}{\PYGZsq{}}\PYG{l+s+s1}{Eat Dinner}\PYG{l+s+s1}{\PYGZsq{}}\PYG{p}{,} \PYG{l+s+s1}{\PYGZsq{}}\PYG{l+s+s1}{Sleep}\PYG{l+s+s1}{\PYGZsq{}}\PYG{p}{,} \PYG{l+s+s1}{\PYGZsq{}}\PYG{l+s+s1}{Commute to Work}\PYG{l+s+s1}{\PYGZsq{}}\PYG{p}{,} \PYG{l+s+s1}{\PYGZsq{}}\PYG{l+s+s1}{Work}\PYG{l+s+s1}{\PYGZsq{}}\PYG{p}{,} \PYG{l+s+s1}{\PYGZsq{}}\PYG{l+s+s1}{Eat Lunch}\PYG{l+s+s1}{\PYGZsq{}}\PYG{p}{,} \PYG{l+s+s1}{\PYGZsq{}}\PYG{l+s+s1}{Commute from Work}\PYG{l+s+s1}{\PYGZsq{}}\PYG{p}{,} \PYG{l+s+s1}{\PYGZsq{}}\PYG{l+s+s1}{Eat Breakfast}\PYG{l+s+s1}{\PYGZsq{}}\PYG{p}{]}
\end{sphinxVerbatim}
\begin{sphinxalltt}
C:UsersnbrandonAppDataLocalContinuumAnaconda3libsite-packagesmatplotliblegend.py:338: UserWarning: Automatic legend placement (loc="best") not implemented for figure legend. Falling back on "upper right".
  warnings.warn('Automatic legend placement (loc="best") not '
\end{sphinxalltt}


\subsection{plotter module}
\label{\detokenize{plotter::doc}}\label{\detokenize{plotter:plotter-module}}\label{\detokenize{plotter:module-plotter}}\index{plotter (module)}
This module contains information and functions for plotting various data related to the algorithm. In short,
this module is a plotting library for the algorithm.
\index{calc\_log\_weight() (in module plotter)}

\begin{fulllineitems}
\phantomsection\label{\detokenize{plotter:plotter.calc_log_weight}}\pysiglinewithargsret{\sphinxcode{\sphinxupquote{plotter.}}\sphinxbfcode{\sphinxupquote{calc\_log\_weight}}}{\emph{w}}{}
This function calculates the log10 of the weights. To avoid the possibility of getting an     error due to taking log10(w=0), we zero-valued weight values to None.
\begin{quote}\begin{description}
\item[{Parameters}] \leavevmode
\sphinxstyleliteralstrong{\sphinxupquote{w}} (\sphinxstyleliteralemphasis{\sphinxupquote{numpy.ndarray}}) \textendash{} the values of the weights of a corresponding need

\item[{Returns}] \leavevmode
the log10 for the non-zero values of the weights

\end{description}\end{quote}

\end{fulllineitems}

\index{calc\_weight() (in module plotter)}

\begin{fulllineitems}
\phantomsection\label{\detokenize{plotter:plotter.calc_weight}}\pysiglinewithargsret{\sphinxcode{\sphinxupquote{plotter.}}\sphinxbfcode{\sphinxupquote{calc\_weight}}}{\emph{x}, \emph{threshold=0.2}}{}
This function calculates the weight value corresponding to a given value     of satiation and threshold value.
\begin{quote}\begin{description}
\item[{Parameters}] \leavevmode\begin{itemize}
\item {} 
\sphinxstyleliteralstrong{\sphinxupquote{x}} (\sphinxstyleliteralemphasis{\sphinxupquote{numpy.ndarray}}) \textendash{} the satiation values from an agent

\item {} 
\sphinxstyleliteralstrong{\sphinxupquote{threshold}} (\sphinxstyleliteralemphasis{\sphinxupquote{float}}) \textendash{} the threshold value for a need

\end{itemize}

\item[{Returns}] \leavevmode
an array of the weight values

\end{description}\end{quote}

\end{fulllineitems}

\index{get\_figure\_data() (in module plotter)}

\begin{fulllineitems}
\phantomsection\label{\detokenize{plotter:plotter.get_figure_data}}\pysiglinewithargsret{\sphinxcode{\sphinxupquote{plotter.}}\sphinxbfcode{\sphinxupquote{get\_figure\_data}}}{\emph{fpaths}, \emph{fpath\_figure\_save}, \emph{fname}, \emph{fnames\_load=None}, \emph{do\_single\_day=False}}{}
This function gets figure data from the subplots of cumulative distribution functions (CDFs)
of activity-parameters (start time, end time, and duration).
\begin{quote}\begin{description}
\item[{Parameters}] \leavevmode\begin{itemize}
\item {} 
\sphinxstyleliteralstrong{\sphinxupquote{fpaths}} (\sphinxstyleliteralemphasis{\sphinxupquote{list of str}}) \textendash{} a list of file paths of the figure data for each activity to load

\item {} 
\sphinxstyleliteralstrong{\sphinxupquote{fpath\_figure\_save}} (\sphinxstyleliteralemphasis{\sphinxupquote{str}}) \textendash{} the file path to save the figure

\item {} 
\sphinxstyleliteralstrong{\sphinxupquote{fname}} (\sphinxstyleliteralemphasis{\sphinxupquote{str}}) \textendash{} the file name (no file path) to save the data

\item {} 
\sphinxstyleliteralstrong{\sphinxupquote{fnames\_load}} (\sphinxstyleliteralemphasis{\sphinxupquote{list of str}}) \textendash{} the ending of the file names of the figure files to load (start time, end time, duration)

\item {} 
\sphinxstyleliteralstrong{\sphinxupquote{do\_single\_day}} (\sphinxstyleliteralemphasis{\sphinxupquote{bool}}) \textendash{} a flag indicating whether to load single-day (if True) or     longitudinal(if False) figure data

\end{itemize}

\item[{Returns}] \leavevmode
the x and y values of the lines in the figure for start time, end time, and duration plots

\item[{Return type}] \leavevmode
list, str

\end{description}\end{quote}

\end{fulllineitems}

\index{get\_satiation\_and\_weight() (in module plotter)}

\begin{fulllineitems}
\phantomsection\label{\detokenize{plotter:plotter.get_satiation_and_weight}}\pysiglinewithargsret{\sphinxcode{\sphinxupquote{plotter.}}\sphinxbfcode{\sphinxupquote{get\_satiation\_and\_weight}}}{\emph{p}, \emph{start\_day}, \emph{end\_day}}{}
This function obtains the satiation values and weight values for the agent during the simulation over the
range of the selected days.
\begin{quote}\begin{description}
\item[{Parameters}] \leavevmode\begin{itemize}
\item {} 
\sphinxstyleliteralstrong{\sphinxupquote{p}} ({\hyperref[\detokenize{person:person.Person}]{\sphinxcrossref{\sphinxstyleliteralemphasis{\sphinxupquote{person.Person}}}}}) \textendash{} the agent whose satiation and weight values are to be plotted

\item {} 
\sphinxstyleliteralstrong{\sphinxupquote{start\_day}} (\sphinxstyleliteralemphasis{\sphinxupquote{int}}) \textendash{} the day to start plotting

\item {} 
\sphinxstyleliteralstrong{\sphinxupquote{end\_day}} (\sphinxstyleliteralemphasis{\sphinxupquote{int}}) \textendash{} the day to end plotting

\end{itemize}

\item[{Returns}] \leavevmode
a tuple of an array of the selected time (in hours), a list of the satiation values, and     a list of the weights for the respective times

\end{description}\end{quote}

\end{fulllineitems}

\index{load\_fig\_data() (in module plotter)}

\begin{fulllineitems}
\phantomsection\label{\detokenize{plotter:plotter.load_fig_data}}\pysiglinewithargsret{\sphinxcode{\sphinxupquote{plotter.}}\sphinxbfcode{\sphinxupquote{load\_fig\_data}}}{\emph{fname}}{}
Load figure data.
:param str fname: the file name of the figure to load. The file must be a .pkl file.
\begin{quote}\begin{description}
\item[{Returns}] \leavevmode
the x and y values of the lines in the figure

\item[{Return type}] \leavevmode
list

\end{description}\end{quote}

\end{fulllineitems}

\index{plot\_activity\_cdfs() (in module plotter)}

\begin{fulllineitems}
\phantomsection\label{\detokenize{plotter:plotter.plot_activity_cdfs}}\pysiglinewithargsret{\sphinxcode{\sphinxupquote{plotter.}}\sphinxbfcode{\sphinxupquote{plot\_activity\_cdfs}}}{\emph{d}, \emph{keys}}{}
This function plots the cumulative distribution function of start time, end time, and duration for     each activity in the the simulation.
\begin{quote}\begin{description}
\item[{Parameters}] \leavevmode\begin{itemize}
\item {} 
\sphinxstyleliteralstrong{\sphinxupquote{d}} ({\hyperref[\detokenize{diary:diary.Diary}]{\sphinxcrossref{\sphinxstyleliteralemphasis{\sphinxupquote{diary.Diary}}}}}) \textendash{} the results of the simulation

\item {} 
\sphinxstyleliteralstrong{\sphinxupquote{keys}} (\sphinxstyleliteralemphasis{\sphinxupquote{list}}) \textendash{} list of activities to graph

\end{itemize}

\item[{Returns}] \leavevmode


\end{description}\end{quote}

\end{fulllineitems}

\index{plot\_activity\_histograms() (in module plotter)}

\begin{fulllineitems}
\phantomsection\label{\detokenize{plotter:plotter.plot_activity_histograms}}\pysiglinewithargsret{\sphinxcode{\sphinxupquote{plotter.}}\sphinxbfcode{\sphinxupquote{plot\_activity\_histograms}}}{\emph{d}, \emph{keys}}{}
This function plots the histograms of start time, end time, and duration for each activity in     the the simulation.
\begin{quote}\begin{description}
\item[{Parameters}] \leavevmode\begin{itemize}
\item {} 
\sphinxstyleliteralstrong{\sphinxupquote{d}} ({\hyperref[\detokenize{diary:diary.Diary}]{\sphinxcrossref{\sphinxstyleliteralemphasis{\sphinxupquote{diary.Diary}}}}}) \textendash{} the results of the simulation

\item {} 
\sphinxstyleliteralstrong{\sphinxupquote{keys}} (\sphinxstyleliteralemphasis{\sphinxupquote{list}}) \textendash{} list of activities to graph

\end{itemize}

\item[{Returns}] \leavevmode


\end{description}\end{quote}

\end{fulllineitems}

\index{plot\_count() (in module plotter)}

\begin{fulllineitems}
\phantomsection\label{\detokenize{plotter:plotter.plot_count}}\pysiglinewithargsret{\sphinxcode{\sphinxupquote{plotter.}}\sphinxbfcode{\sphinxupquote{plot\_count}}}{\emph{data}, \emph{keys}, \emph{do\_abs=True}, \emph{title=None}}{}
This function plots a histogram showing the amount of times each activity was done     in an ABMHAP simulation.
\begin{quote}\begin{description}
\item[{Parameters}] \leavevmode\begin{itemize}
\item {} 
\sphinxstyleliteralstrong{\sphinxupquote{data}} (\sphinxstyleliteralemphasis{\sphinxupquote{pandas.core.frame.DataFrame}}) \textendash{} the activity diary

\item {} 
\sphinxstyleliteralstrong{\sphinxupquote{keys}} (\sphinxstyleliteralemphasis{\sphinxupquote{list}}) \textendash{} the activity codes

\item {} 
\sphinxstyleliteralstrong{\sphinxupquote{do\_abs}} (\sphinxstyleliteralemphasis{\sphinxupquote{bool}}) \textendash{} whether (if True) to plot a histogram of the number     of agents or (if False) to plot a histogram of percentage of agents

\item {} 
\sphinxstyleliteralstrong{\sphinxupquote{title}} (\sphinxstyleliteralemphasis{\sphinxupquote{str}}) \textendash{} the title of the plot

\end{itemize}

\item[{Returns}] \leavevmode


\end{description}\end{quote}

\end{fulllineitems}

\index{plot\_history() (in module plotter)}

\begin{fulllineitems}
\phantomsection\label{\detokenize{plotter:plotter.plot_history}}\pysiglinewithargsret{\sphinxcode{\sphinxupquote{plotter.}}\sphinxbfcode{\sphinxupquote{plot\_history}}}{\emph{t}, \emph{y\_list}, \emph{labels}, \emph{colors}, \emph{linestyles}, \emph{ylabel}, \emph{linewidth=None}}{}
This function plots information related to data related to needs (such as satiation and     weight function values) over time.
\begin{quote}\begin{description}
\item[{Parameters}] \leavevmode\begin{itemize}
\item {} 
\sphinxstyleliteralstrong{\sphinxupquote{t}} (\sphinxstyleliteralemphasis{\sphinxupquote{numpy.ndarray}}) \textendash{} the time values {[}hours{]} of interest

\item {} 
\sphinxstyleliteralstrong{\sphinxupquote{y\_list}} (\sphinxstyleliteralemphasis{\sphinxupquote{list}}) \textendash{} the satiation values for each need over time

\item {} 
\sphinxstyleliteralstrong{\sphinxupquote{labels}} (\sphinxstyleliteralemphasis{\sphinxupquote{list}}) \textendash{} the labels that corresponds to the respective need

\item {} 
\sphinxstyleliteralstrong{\sphinxupquote{colors}} (\sphinxstyleliteralemphasis{\sphinxupquote{list}}) \textendash{} the colors that corresponds to the respective need

\item {} 
\sphinxstyleliteralstrong{\sphinxupquote{linestyles}} (\sphinxstyleliteralemphasis{\sphinxupquote{list}}) \textendash{} the line styles that corresponds to the respective need

\item {} 
\sphinxstyleliteralstrong{\sphinxupquote{ylabel}} (\sphinxstyleliteralemphasis{\sphinxupquote{str}}) \textendash{} the y-axis label

\item {} 
\sphinxstyleliteralstrong{\sphinxupquote{linewidth}} (\sphinxstyleliteralemphasis{\sphinxupquote{int}}) \textendash{} the line width for each line

\end{itemize}

\item[{Returns}] \leavevmode


\end{description}\end{quote}

\end{fulllineitems}

\index{plot\_longitude() (in module plotter)}

\begin{fulllineitems}
\phantomsection\label{\detokenize{plotter:plotter.plot_longitude}}\pysiglinewithargsret{\sphinxcode{\sphinxupquote{plotter.}}\sphinxbfcode{\sphinxupquote{plot\_longitude}}}{\emph{data}, \emph{titles}, \emph{linewidth=1}}{}
This function plots the day-to-day variation of activity duration for each activity     over time from an ABMHAP simulation. This is done for each demographic in order     to compare their differences and daily behavior. Within each subplot, an agent     representing a respective demographic has its activity behavior is plotted
in a log10 scale over time.
\begin{quote}\begin{description}
\item[{Parameters}] \leavevmode\begin{itemize}
\item {} 
\sphinxstyleliteralstrong{\sphinxupquote{data}} (\sphinxstyleliteralemphasis{\sphinxupquote{list of pandas.core.frame.DataFrame}}) \textendash{} the activity diaries of the agents to plot. Each agent represents     a different demograhic.

\item {} 
\sphinxstyleliteralstrong{\sphinxupquote{titles}} (\sphinxstyleliteralemphasis{\sphinxupquote{list of str}}) \textendash{} the names of the demographics that are being plot

\item {} 
\sphinxstyleliteralstrong{\sphinxupquote{linewidth}} (\sphinxstyleliteralemphasis{\sphinxupquote{float}}) \textendash{} the line width of the plot lines

\end{itemize}

\item[{Returns}] \leavevmode


\end{description}\end{quote}

\end{fulllineitems}

\index{plot\_longitude\_help() (in module plotter)}

\begin{fulllineitems}
\phantomsection\label{\detokenize{plotter:plotter.plot_longitude_help}}\pysiglinewithargsret{\sphinxcode{\sphinxupquote{plotter.}}\sphinxbfcode{\sphinxupquote{plot\_longitude\_help}}}{\emph{ax}, \emph{df}, \emph{linewidth=1}}{}
This function plots the day-to-day variation of activity duration for each activity     over time from an ABMHAP simulation. Within each subplot, an agent has its activity     behavior is plotted in a log10 scale over time.
\begin{quote}\begin{description}
\item[{Parameters}] \leavevmode\begin{itemize}
\item {} 
\sphinxstyleliteralstrong{\sphinxupquote{ax}} (\sphinxstyleliteralemphasis{\sphinxupquote{matplotlib.figure.Figure}}) \textendash{} the subplot object

\item {} 
\sphinxstyleliteralstrong{\sphinxupquote{df}} (\sphinxstyleliteralemphasis{\sphinxupquote{pandas.core.frame.DataFrame}}) \textendash{} the activity diary of an agent

\item {} 
\sphinxstyleliteralstrong{\sphinxupquote{linewidth}} (\sphinxstyleliteralemphasis{\sphinxupquote{float}}) \textendash{} the line width of the plot lines

\end{itemize}

\item[{Returns}] \leavevmode


\item[{Return type}] \leavevmode
list of int

\end{description}\end{quote}

\end{fulllineitems}

\index{plot\_satiation\_and\_weight() (in module plotter)}

\begin{fulllineitems}
\phantomsection\label{\detokenize{plotter:plotter.plot_satiation_and_weight}}\pysiglinewithargsret{\sphinxcode{\sphinxupquote{plotter.}}\sphinxbfcode{\sphinxupquote{plot\_satiation\_and\_weight}}}{\emph{p}, \emph{start\_day}, \emph{end\_day}, \emph{fid\_satiation=100}, \emph{fid\_weight=101}}{}
This function plots the satiation values and weight values for the agent during the simulation.

\begin{sphinxadmonition}{warning}{Warning:}
This function is best used when the simulation moves through time minute by minute. If not,
the slopes in both the satiation and weight plots will \sphinxstylestrong{not} be accurate.
\end{sphinxadmonition}
\begin{quote}\begin{description}
\item[{Parameters}] \leavevmode\begin{itemize}
\item {} 
\sphinxstyleliteralstrong{\sphinxupquote{p}} ({\hyperref[\detokenize{person:person.Person}]{\sphinxcrossref{\sphinxstyleliteralemphasis{\sphinxupquote{person.Person}}}}}) \textendash{} the agent whose satiation and weight values are to be plotted

\item {} 
\sphinxstyleliteralstrong{\sphinxupquote{start\_day}} (\sphinxstyleliteralemphasis{\sphinxupquote{int}}) \textendash{} the day to start plotting

\item {} 
\sphinxstyleliteralstrong{\sphinxupquote{end\_day}} (\sphinxstyleliteralemphasis{\sphinxupquote{int}}) \textendash{} the day to end plotting

\item {} 
\sphinxstyleliteralstrong{\sphinxupquote{fid\_satiation}} (\sphinxstyleliteralemphasis{\sphinxupquote{int}}) \textendash{} the figure identifier for the satiation plot

\item {} 
\sphinxstyleliteralstrong{\sphinxupquote{fid\_weight}} (\sphinxstyleliteralemphasis{\sphinxupquote{int}}) \textendash{} the figure identifier for the weights plot

\end{itemize}

\item[{Returns}] \leavevmode


\end{description}\end{quote}

\end{fulllineitems}

\index{separate\_activities\_into\_days() (in module plotter)}

\begin{fulllineitems}
\phantomsection\label{\detokenize{plotter:plotter.separate_activities_into_days}}\pysiglinewithargsret{\sphinxcode{\sphinxupquote{plotter.}}\sphinxbfcode{\sphinxupquote{separate\_activities\_into\_days}}}{\emph{data}}{}
This function finds the activities tha occur over midnight and breaks down     creates a new activity diary in which an activity occurring over midnight     is split into two activities: one activity entry ending at midnight, and     one activity entry starting at midnight.
\begin{quote}\begin{description}
\item[{Parameters}] \leavevmode
\sphinxstyleliteralstrong{\sphinxupquote{data}} (\sphinxstyleliteralemphasis{\sphinxupquote{pandas.core.frame.DataFrame}}) \textendash{} the activity diary of an agent

\item[{Returns}] \leavevmode
the new activity diary

\item[{Return type}] \leavevmode
pandas.core.frame.DataFrame

\end{description}\end{quote}

\end{fulllineitems}



\section{Processing Directory}
\label{\detokenize{index:processing-directory}}
These functions handle the logistics in dealing with CHAD.

Contents:


\subsection{commute\_school notebook}
\label{\detokenize{commute_school::doc}}\label{\detokenize{commute_school:commute-school-notebook}}
\fvset{hllines={, ,}}%
\begin{sphinxVerbatim}[commandchars=\\\{\}]
\PYG{c+c1}{\PYGZsh{} The United States Environmental Protection Agency through its Office of}
\PYG{c+c1}{\PYGZsh{} Research and Development has developed this software. The code is made}
\PYG{c+c1}{\PYGZsh{} publicly available to better communicate the research. All input data}
\PYG{c+c1}{\PYGZsh{} used fora given application should be reviewed by the researcher so}
\PYG{c+c1}{\PYGZsh{} that the model results are based on appropriate data for any given}
\PYG{c+c1}{\PYGZsh{} application. This model is under continued development. The model and}
\PYG{c+c1}{\PYGZsh{} data included herein do not represent and should not be construed to}
\PYG{c+c1}{\PYGZsh{} represent any Agency determination or policy.}
\PYG{c+c1}{\PYGZsh{}}
\PYG{c+c1}{\PYGZsh{} This file was written by Dr. Namdi Brandon}
\PYG{c+c1}{\PYGZsh{} ORCID: 0000\PYGZhy{}0001\PYGZhy{}7050\PYGZhy{}1538}
\PYG{c+c1}{\PYGZsh{} March 22, 2018}
\end{sphinxVerbatim}

This file goes through the data from the Consoldiated Human Activity
Database (CHAD) and gets information relevent to \sphinxstylestrong{commuting to school}
and \sphinxstylestrong{commuting from school} and processes the data for use in the
Agent-Based Model of Human Activity Patterns (ABMHAP) for the school-age
children demographic. More specficially, this file does the following:

For a given demographic,
\begin{enumerate}
\item {} 
This function goes through the CHAD data and finds the commute
activity data

\item {} 
The CHAD activity data are seperated into start time, end time,
duration, and CHAD record data

\item {} 
The CHAD activity data is saved into longitudinal data and
single-activity data

\end{enumerate}

Import

\fvset{hllines={, ,}}%
\begin{sphinxVerbatim}[commandchars=\\\{\}]
\PYG{k+kn}{import} \PYG{n+nn}{sys}
\PYG{n}{sys}\PYG{o}{.}\PYG{n}{path}\PYG{o}{.}\PYG{n}{append}\PYG{p}{(}\PYG{l+s+s1}{\PYGZsq{}}\PYG{l+s+s1}{..}\PYG{l+s+se}{\PYGZbs{}\PYGZbs{}}\PYG{l+s+s1}{source}\PYG{l+s+s1}{\PYGZsq{}}\PYG{p}{)}

\PYG{c+c1}{\PYGZsh{} ABMHAP capability}
\PYG{k+kn}{import} \PYG{n+nn}{demography} \PYG{k}{as} \PYG{n+nn}{dmg}
\PYG{k+kn}{import} \PYG{n+nn}{datum}
\end{sphinxVerbatim}

\fvset{hllines={, ,}}%
\begin{sphinxVerbatim}[commandchars=\\\{\}]
\PYG{o}{\PYGZpc{}}\PYG{k}{matplotlib} notebook
\end{sphinxVerbatim}

Load

\fvset{hllines={, ,}}%
\begin{sphinxVerbatim}[commandchars=\\\{\}]
\PYG{c+c1}{\PYGZsh{}}
\PYG{c+c1}{\PYGZsh{} demographic}
\PYG{c+c1}{\PYGZsh{}}
\PYG{c+c1}{\PYGZsh{} the input file and output file directory}
\PYG{n}{key} \PYG{o}{=} \PYG{n}{dmg}\PYG{o}{.}\PYG{n}{CHILD\PYGZus{}SCHOOL}

\PYG{c+c1}{\PYGZsh{} the input file and output file directory}
\PYG{n}{fname\PYGZus{}input}\PYG{p}{,} \PYG{n}{fpath\PYGZus{}output} \PYG{o}{=} \PYG{n}{dmg}\PYG{o}{.}\PYG{n}{INT\PYGZus{}2\PYGZus{}FIN\PYGZus{}FOUT\PYGZus{}LARGE}\PYG{p}{[}\PYG{n}{key}\PYG{p}{]}

\PYG{c+c1}{\PYGZsh{} load the data}
\PYG{n}{data} \PYG{o}{=} \PYG{n}{dmg}\PYG{o}{.}\PYG{n}{load}\PYG{p}{(}\PYG{n}{fname\PYGZus{}input}\PYG{p}{)}
\end{sphinxVerbatim}

Processing data

\fvset{hllines={, ,}}%
\begin{sphinxVerbatim}[commandchars=\\\{\}]
\PYG{c+c1}{\PYGZsh{} get the raw commute data}
\PYG{n}{d}\PYG{p}{,} \PYG{n}{d\PYGZus{}to\PYGZus{}school}\PYG{p}{,} \PYG{n}{d\PYGZus{}from\PYGZus{}school} \PYG{o}{=} \PYG{n}{datum}\PYG{o}{.}\PYG{n}{analyze\PYGZus{}commute\PYGZus{}school}\PYG{p}{(}\PYG{n}{data}\PYG{p}{)}
\end{sphinxVerbatim}

Plotting

\fvset{hllines={, ,}}%
\begin{sphinxVerbatim}[commandchars=\\\{\}]
\PYG{c+c1}{\PYGZsh{}}
\PYG{c+c1}{\PYGZsh{} choose to save longitudinal data or single\PYGZhy{}day data}
\PYG{c+c1}{\PYGZsh{}}
\PYG{c+c1}{\PYGZsh{} note that N for the LONGITUDINAL DATA is 1}
\PYG{c+c1}{\PYGZsh{} this was done becaause there is NOT ENOUGH LONGITUDINAL DATA for adults and working}
\PYG{c+c1}{\PYGZsh{}}
\PYG{n}{chooser} \PYG{o}{=} \PYG{p}{\PYGZob{}}\PYG{k+kc}{True}\PYG{p}{:} \PYG{p}{(}\PYG{l+m+mi}{1}\PYG{p}{,} \PYG{n}{fpath\PYGZus{}output} \PYG{o}{+} \PYG{l+s+s1}{\PYGZsq{}}\PYG{l+s+se}{\PYGZbs{}\PYGZbs{}}\PYG{l+s+s1}{longitude}\PYG{l+s+s1}{\PYGZsq{}}\PYG{p}{)}\PYG{p}{,}
           \PYG{k+kc}{False}\PYG{p}{:} \PYG{p}{(}\PYG{l+m+mi}{1}\PYG{p}{,} \PYG{n}{fpath\PYGZus{}output} \PYG{o}{+} \PYG{l+s+s1}{\PYGZsq{}}\PYG{l+s+se}{\PYGZbs{}\PYGZbs{}}\PYG{l+s+s1}{solo}\PYG{l+s+s1}{\PYGZsq{}}\PYG{p}{)}\PYG{p}{,} \PYG{p}{\PYGZcb{}}

\PYG{c+c1}{\PYGZsh{} whether to save the longitudinal data (if True) or the single\PYGZhy{}day data (if False)}
\PYG{c+c1}{\PYGZsh{} there is not enough longitudinal data to have a longitudinal model}
\PYG{n}{do\PYGZus{}long} \PYG{o}{=} \PYG{k+kc}{False}
\end{sphinxVerbatim}

\fvset{hllines={, ,}}%
\begin{sphinxVerbatim}[commandchars=\\\{\}]
\PYG{c+c1}{\PYGZsh{} save the longitude data}
\PYG{n}{do\PYGZus{}save} \PYG{o}{=} \PYG{k+kc}{False}

\PYG{k}{if} \PYG{n}{do\PYGZus{}save}\PYG{p}{:}

    \PYG{n}{N}\PYG{p}{,} \PYG{n}{fpath} \PYG{o}{=} \PYG{n}{chooser}\PYG{p}{[}\PYG{n}{do\PYGZus{}long}\PYG{p}{]}

    \PYG{c+c1}{\PYGZsh{} the directories the data should be saved in}
    \PYG{n}{fpaths} \PYG{o}{=} \PYG{p}{[}\PYG{n}{fpath} \PYG{o}{+} \PYG{l+s+s1}{\PYGZsq{}}\PYG{l+s+se}{\PYGZbs{}\PYGZbs{}}\PYG{l+s+s1}{commute\PYGZus{}to\PYGZus{}work}\PYG{l+s+s1}{\PYGZsq{}}\PYG{p}{,} \PYG{n}{fpath} \PYG{o}{+} \PYG{l+s+s1}{\PYGZsq{}}\PYG{l+s+se}{\PYGZbs{}\PYGZbs{}}\PYG{l+s+s1}{commute\PYGZus{}from\PYGZus{}work}\PYG{l+s+s1}{\PYGZsq{}}\PYG{p}{]}

    \PYG{c+c1}{\PYGZsh{} the dictionaries holding the data}
    \PYG{n}{data\PYGZus{}dict} \PYG{o}{=} \PYG{p}{[}\PYG{n}{d\PYGZus{}to\PYGZus{}school}\PYG{p}{,} \PYG{n}{d\PYGZus{}from\PYGZus{}school}\PYG{p}{]}

    \PYG{c+c1}{\PYGZsh{} save the data}
    \PYG{k}{for} \PYG{n}{fpath}\PYG{p}{,} \PYG{n}{d} \PYG{o+ow}{in} \PYG{n+nb}{zip}\PYG{p}{(}\PYG{n}{fpaths}\PYG{p}{,} \PYG{n}{data\PYGZus{}dict}\PYG{p}{)}\PYG{p}{:}

        \PYG{n}{stats\PYGZus{}dt}\PYG{p}{,} \PYG{n}{stats\PYGZus{}start}\PYG{p}{,} \PYG{n}{stats\PYGZus{}end}\PYG{p}{,} \PYG{n}{record} \PYG{o}{=} \PYG{n}{d}\PYG{p}{[}\PYG{l+s+s1}{\PYGZsq{}}\PYG{l+s+s1}{stats\PYGZus{}dt}\PYG{l+s+s1}{\PYGZsq{}}\PYG{p}{]}\PYG{p}{,} \PYG{n}{d}\PYG{p}{[}\PYG{l+s+s1}{\PYGZsq{}}\PYG{l+s+s1}{stats\PYGZus{}start}\PYG{l+s+s1}{\PYGZsq{}}\PYG{p}{]}\PYG{p}{,} \PYG{n}{d}\PYG{p}{[}\PYG{l+s+s1}{\PYGZsq{}}\PYG{l+s+s1}{stats\PYGZus{}end}\PYG{l+s+s1}{\PYGZsq{}}\PYG{p}{]}\PYG{p}{,} \PYG{n}{d}\PYG{p}{[}\PYG{l+s+s1}{\PYGZsq{}}\PYG{l+s+s1}{data}\PYG{l+s+s1}{\PYGZsq{}}\PYG{p}{]}

        \PYG{k}{if} \PYG{n}{do\PYGZus{}long}\PYG{p}{:}
            \PYG{n}{dt}\PYG{p}{,} \PYG{n}{start}\PYG{p}{,} \PYG{n}{end}\PYG{p}{,} \PYG{n}{rec} \PYG{o}{=} \PYG{n}{datum}\PYG{o}{.}\PYG{n}{get\PYGZus{}longitude}\PYG{p}{(}\PYG{n}{stats\PYGZus{}dt}\PYG{p}{,} \PYG{n}{stats\PYGZus{}start}\PYG{p}{,} \PYG{n}{stats\PYGZus{}end}\PYG{p}{,} \PYG{n}{record}\PYG{p}{,} \PYG{n}{N}\PYG{o}{=}\PYG{n}{N}\PYG{p}{)}
        \PYG{k}{else}\PYG{p}{:}
            \PYG{n}{dt}\PYG{p}{,} \PYG{n}{start}\PYG{p}{,} \PYG{n}{end}\PYG{p}{,} \PYG{n}{rec} \PYG{o}{=} \PYG{n}{datum}\PYG{o}{.}\PYG{n}{get\PYGZus{}solo}\PYG{p}{(}\PYG{n}{stats\PYGZus{}dt}\PYG{p}{,} \PYG{n}{stats\PYGZus{}start}\PYG{p}{,} \PYG{n}{stats\PYGZus{}end}\PYG{p}{,} \PYG{n}{record}\PYG{p}{)}

        \PYG{n}{datum}\PYG{o}{.}\PYG{n}{save}\PYG{p}{(}\PYG{n}{fpath}\PYG{p}{,} \PYG{n}{record}\PYG{o}{=}\PYG{n}{rec}\PYG{p}{,} \PYG{n}{stats\PYGZus{}dt}\PYG{o}{=}\PYG{n}{dt}\PYG{p}{,} \PYG{n}{stats\PYGZus{}start}\PYG{o}{=}\PYG{n}{start}\PYG{p}{,} \PYG{n}{stats\PYGZus{}end}\PYG{o}{=}\PYG{n}{end}\PYG{p}{)}
\end{sphinxVerbatim}


\subsection{commute\_work notebook}
\label{\detokenize{commute_work::doc}}\label{\detokenize{commute_work:commute-work-notebook}}
\fvset{hllines={, ,}}%
\begin{sphinxVerbatim}[commandchars=\\\{\}]
\PYG{c+c1}{\PYGZsh{} The United States Environmental Protection Agency through its Office of}
\PYG{c+c1}{\PYGZsh{} Research and Development has developed this software. The code is made}
\PYG{c+c1}{\PYGZsh{} publicly available to better communicate the research. All input data}
\PYG{c+c1}{\PYGZsh{} used fora given application should be reviewed by the researcher so}
\PYG{c+c1}{\PYGZsh{} that the model results are based on appropriate data for any given}
\PYG{c+c1}{\PYGZsh{} application. This model is under continued development. The model and}
\PYG{c+c1}{\PYGZsh{} data included herein do not represent and should not be construed to}
\PYG{c+c1}{\PYGZsh{} represent any Agency determination or policy.}
\PYG{c+c1}{\PYGZsh{}}
\PYG{c+c1}{\PYGZsh{} This file was written by Dr. Namdi Brandon}
\PYG{c+c1}{\PYGZsh{} ORCID: 0000\PYGZhy{}0001\PYGZhy{}7050\PYGZhy{}1538}
\PYG{c+c1}{\PYGZsh{} March 22, 2018}
\end{sphinxVerbatim}

This file goes through the data from the Consoldiated Human Activity
Database (CHAD) and gets information relevent to \sphinxstylestrong{commuting to work},
\sphinxstylestrong{commuting from work}, and \sphinxstylestrong{working} and processes the data for use
in the Agent-Based Model of Human Activity Patterns (ABMHAP) for the
working adult demographic. More specficially, this file does the
following:
\begin{enumerate}
\item {} 
This function goes through the CHAD data and finds the commute and
work-activity data

\item {} 
The data is chosen such that events are chosen such that the work
events are sandwiched between the commute to work and commtue from
work event

\item {} 
The CHAD activity data are seperated into start time, end time,
duration, and CHAD record data

\item {} 
The CHAD activity data is saved into longitudinal data and
single-activity data

\end{enumerate}

Import

\fvset{hllines={, ,}}%
\begin{sphinxVerbatim}[commandchars=\\\{\}]
\PYG{k+kn}{import} \PYG{n+nn}{sys}
\PYG{n}{sys}\PYG{o}{.}\PYG{n}{path}\PYG{o}{.}\PYG{n}{append}\PYG{p}{(}\PYG{l+s+s1}{\PYGZsq{}}\PYG{l+s+s1}{..}\PYG{l+s+se}{\PYGZbs{}\PYGZbs{}}\PYG{l+s+s1}{source}\PYG{l+s+s1}{\PYGZsq{}}\PYG{p}{)}

\PYG{c+c1}{\PYGZsh{} plotting capability}
\PYG{k+kn}{import} \PYG{n+nn}{matplotlib}\PYG{n+nn}{.}\PYG{n+nn}{pylab} \PYG{k}{as} \PYG{n+nn}{plt}

\PYG{c+c1}{\PYGZsh{} ABMHAP modules}
\PYG{k+kn}{import} \PYG{n+nn}{demography} \PYG{k}{as} \PYG{n+nn}{dmg}
\PYG{k+kn}{import} \PYG{n+nn}{datum}
\end{sphinxVerbatim}

\fvset{hllines={, ,}}%
\begin{sphinxVerbatim}[commandchars=\\\{\}]
\PYG{o}{\PYGZpc{}}\PYG{k}{matplotlib} notebook
\end{sphinxVerbatim}

Load

\fvset{hllines={, ,}}%
\begin{sphinxVerbatim}[commandchars=\\\{\}]
\PYG{c+c1}{\PYGZsh{}}
\PYG{c+c1}{\PYGZsh{} demographic}
\PYG{c+c1}{\PYGZsh{}}
\PYG{c+c1}{\PYGZsh{} the input file and output file directory}
\PYG{n}{key} \PYG{o}{=} \PYG{n}{dmg}\PYG{o}{.}\PYG{n}{ADULT\PYGZus{}WORK}

\PYG{c+c1}{\PYGZsh{} the input file and output file directory}
\PYG{n}{fname\PYGZus{}input}\PYG{p}{,} \PYG{n}{fpath\PYGZus{}output} \PYG{o}{=} \PYG{n}{dmg}\PYG{o}{.}\PYG{n}{INT\PYGZus{}2\PYGZus{}FIN\PYGZus{}FOUT\PYGZus{}LARGE}\PYG{p}{[}\PYG{n}{key}\PYG{p}{]}

\PYG{c+c1}{\PYGZsh{} load the data}
\PYG{n}{data} \PYG{o}{=} \PYG{n}{dmg}\PYG{o}{.}\PYG{n}{load}\PYG{p}{(}\PYG{n}{fname\PYGZus{}input}\PYG{p}{)}
\end{sphinxVerbatim}

Processing data

\fvset{hllines={, ,}}%
\begin{sphinxVerbatim}[commandchars=\\\{\}]
\PYG{c+c1}{\PYGZsh{} analyze the commuting data}
\PYG{n}{d}\PYG{p}{,} \PYG{n}{d\PYGZus{}to\PYGZus{}work}\PYG{p}{,} \PYG{n}{d\PYGZus{}from\PYGZus{}work}\PYG{p}{,} \PYG{n}{d\PYGZus{}at\PYGZus{}work} \PYG{o}{=} \PYG{n}{datum}\PYG{o}{.}\PYG{n}{analyze\PYGZus{}commute}\PYG{p}{(}\PYG{n}{data}\PYG{p}{)}
\end{sphinxVerbatim}

Saving Data

\fvset{hllines={, ,}}%
\begin{sphinxVerbatim}[commandchars=\\\{\}]
\PYG{c+c1}{\PYGZsh{} choose to save longitudinal data or single\PYGZhy{}day data}
\PYG{c+c1}{\PYGZsh{}}
\PYG{c+c1}{\PYGZsh{} note that N for the LONGITUDINAL DATA is 1}
\PYG{c+c1}{\PYGZsh{} this was done becaause there is NOT ENOUGH LONGITUDINAL DATA for adults and working}
\PYG{c+c1}{\PYGZsh{}}
\PYG{n}{chooser} \PYG{o}{=} \PYG{p}{\PYGZob{}}\PYG{k+kc}{True}\PYG{p}{:} \PYG{p}{(}\PYG{l+m+mi}{1}\PYG{p}{,} \PYG{n}{fpath\PYGZus{}output} \PYG{o}{+} \PYG{l+s+s1}{\PYGZsq{}}\PYG{l+s+se}{\PYGZbs{}\PYGZbs{}}\PYG{l+s+s1}{longitude}\PYG{l+s+s1}{\PYGZsq{}}\PYG{p}{)}\PYG{p}{,}
           \PYG{k+kc}{False}\PYG{p}{:} \PYG{p}{(}\PYG{l+m+mi}{1}\PYG{p}{,} \PYG{n}{fpath\PYGZus{}output} \PYG{o}{+} \PYG{l+s+s1}{\PYGZsq{}}\PYG{l+s+se}{\PYGZbs{}\PYGZbs{}}\PYG{l+s+s1}{solo}\PYG{l+s+s1}{\PYGZsq{}}\PYG{p}{)}\PYG{p}{,} \PYG{p}{\PYGZcb{}}

\PYG{c+c1}{\PYGZsh{} whether to save the longitudinal data (if True) or the single\PYGZhy{}day data (if False)}
\PYG{c+c1}{\PYGZsh{} there is not enough longitudinal data to have a longitudinal model}
\PYG{n}{do\PYGZus{}long} \PYG{o}{=} \PYG{k+kc}{True}
\end{sphinxVerbatim}

\fvset{hllines={, ,}}%
\begin{sphinxVerbatim}[commandchars=\\\{\}]
\PYG{c+c1}{\PYGZsh{} save the longitude data}
\PYG{n}{do\PYGZus{}save} \PYG{o}{=} \PYG{k+kc}{False}

\PYG{k}{if} \PYG{n}{do\PYGZus{}save}\PYG{p}{:}

    \PYG{n}{N}\PYG{p}{,} \PYG{n}{fpath} \PYG{o}{=} \PYG{n}{chooser}\PYG{p}{[}\PYG{n}{do\PYGZus{}long}\PYG{p}{]}

    \PYG{c+c1}{\PYGZsh{} the directories the data should be saved in}
    \PYG{n}{fpaths} \PYG{o}{=} \PYG{p}{[}\PYG{n}{fpath} \PYG{o}{+} \PYG{l+s+s1}{\PYGZsq{}}\PYG{l+s+se}{\PYGZbs{}\PYGZbs{}}\PYG{l+s+s1}{commute\PYGZus{}to\PYGZus{}work}\PYG{l+s+s1}{\PYGZsq{}}\PYG{p}{,} \PYG{n}{fpath} \PYG{o}{+} \PYG{l+s+s1}{\PYGZsq{}}\PYG{l+s+se}{\PYGZbs{}\PYGZbs{}}\PYG{l+s+s1}{commute\PYGZus{}from\PYGZus{}work}\PYG{l+s+s1}{\PYGZsq{}}\PYG{p}{,} \PYG{n}{fpath} \PYG{o}{+} \PYG{l+s+s1}{\PYGZsq{}}\PYG{l+s+se}{\PYGZbs{}\PYGZbs{}}\PYG{l+s+s1}{work}\PYG{l+s+s1}{\PYGZsq{}}\PYG{p}{]}

    \PYG{c+c1}{\PYGZsh{} the dictionaries holding the data}
    \PYG{n}{data\PYGZus{}dict} \PYG{o}{=} \PYG{p}{[}\PYG{n}{d\PYGZus{}to\PYGZus{}work}\PYG{p}{,} \PYG{n}{d\PYGZus{}from\PYGZus{}work}\PYG{p}{,} \PYG{n}{d\PYGZus{}at\PYGZus{}work}\PYG{p}{]}

    \PYG{c+c1}{\PYGZsh{} save the data}
    \PYG{k}{for} \PYG{n}{fpath}\PYG{p}{,} \PYG{n}{d} \PYG{o+ow}{in} \PYG{n+nb}{zip}\PYG{p}{(}\PYG{n}{fpaths}\PYG{p}{,} \PYG{n}{data\PYGZus{}dict}\PYG{p}{)}\PYG{p}{:}

        \PYG{n}{stats\PYGZus{}dt}\PYG{p}{,} \PYG{n}{stats\PYGZus{}start}\PYG{p}{,} \PYG{n}{stats\PYGZus{}end}\PYG{p}{,} \PYG{n}{record} \PYG{o}{=} \PYG{n}{d}\PYG{p}{[}\PYG{l+s+s1}{\PYGZsq{}}\PYG{l+s+s1}{stats\PYGZus{}dt}\PYG{l+s+s1}{\PYGZsq{}}\PYG{p}{]}\PYG{p}{,} \PYG{n}{d}\PYG{p}{[}\PYG{l+s+s1}{\PYGZsq{}}\PYG{l+s+s1}{stats\PYGZus{}start}\PYG{l+s+s1}{\PYGZsq{}}\PYG{p}{]}\PYG{p}{,} \PYG{n}{d}\PYG{p}{[}\PYG{l+s+s1}{\PYGZsq{}}\PYG{l+s+s1}{stats\PYGZus{}end}\PYG{l+s+s1}{\PYGZsq{}}\PYG{p}{]}\PYG{p}{,} \PYG{n}{d}\PYG{p}{[}\PYG{l+s+s1}{\PYGZsq{}}\PYG{l+s+s1}{data}\PYG{l+s+s1}{\PYGZsq{}}\PYG{p}{]}

        \PYG{k}{if} \PYG{n}{do\PYGZus{}long}\PYG{p}{:}
            \PYG{n}{dt}\PYG{p}{,} \PYG{n}{start}\PYG{p}{,} \PYG{n}{end}\PYG{p}{,} \PYG{n}{rec} \PYG{o}{=} \PYG{n}{datum}\PYG{o}{.}\PYG{n}{get\PYGZus{}longitude}\PYG{p}{(}\PYG{n}{stats\PYGZus{}dt}\PYG{p}{,} \PYG{n}{stats\PYGZus{}start}\PYG{p}{,} \PYG{n}{stats\PYGZus{}end}\PYG{p}{,} \PYG{n}{record}\PYG{p}{,} \PYG{n}{N}\PYG{o}{=}\PYG{n}{N}\PYG{p}{)}
        \PYG{k}{else}\PYG{p}{:}
            \PYG{n}{dt}\PYG{p}{,} \PYG{n}{start}\PYG{p}{,} \PYG{n}{end}\PYG{p}{,} \PYG{n}{rec} \PYG{o}{=} \PYG{n}{datum}\PYG{o}{.}\PYG{n}{get\PYGZus{}solo}\PYG{p}{(}\PYG{n}{stats\PYGZus{}dt}\PYG{p}{,} \PYG{n}{stats\PYGZus{}start}\PYG{p}{,} \PYG{n}{stats\PYGZus{}end}\PYG{p}{,} \PYG{n}{record}\PYG{p}{)}

        \PYG{n}{datum}\PYG{o}{.}\PYG{n}{save}\PYG{p}{(}\PYG{n}{fpath}\PYG{p}{,} \PYG{n}{record}\PYG{o}{=}\PYG{n}{rec}\PYG{p}{,} \PYG{n}{stats\PYGZus{}dt}\PYG{o}{=}\PYG{n}{dt}\PYG{p}{,} \PYG{n}{stats\PYGZus{}start}\PYG{o}{=}\PYG{n}{start}\PYG{p}{,} \PYG{n}{stats\PYGZus{}end}\PYG{o}{=}\PYG{n}{end}\PYG{p}{)}
\end{sphinxVerbatim}


\subsection{count\_records notebook}
\label{\detokenize{count_records::doc}}\label{\detokenize{count_records:count-records-notebook}}
\fvset{hllines={, ,}}%
\begin{sphinxVerbatim}[commandchars=\\\{\}]
\PYG{c+c1}{\PYGZsh{} The United States Environmental Protection Agency through its Office of}
\PYG{c+c1}{\PYGZsh{} Research and Development has developed this software. The code is made}
\PYG{c+c1}{\PYGZsh{} publicly available to better communicate the research. All input data}
\PYG{c+c1}{\PYGZsh{} used fora given application should be reviewed by the researcher so}
\PYG{c+c1}{\PYGZsh{} that the model results are based on appropriate data for any given}
\PYG{c+c1}{\PYGZsh{} application. This model is under continued development. The model and}
\PYG{c+c1}{\PYGZsh{} data included herein do not represent and should not be construed to}
\PYG{c+c1}{\PYGZsh{} represent any Agency determination or policy.}
\PYG{c+c1}{\PYGZsh{}}
\PYG{c+c1}{\PYGZsh{} This file was written by Dr. Namdi Brandon}
\PYG{c+c1}{\PYGZsh{} ORCID: 0000\PYGZhy{}0001\PYGZhy{}7050\PYGZhy{}1538}
\PYG{c+c1}{\PYGZsh{} March 22, 2018}
\end{sphinxVerbatim}

This function reports the amount of records from the Consolidated Human
Activity Database (CHAD) records for each activity for each demographic
that are suitable for use within the Agent-Based Model of Human Activity
Patterns (ABMHAP) code.

import

\fvset{hllines={, ,}}%
\begin{sphinxVerbatim}[commandchars=\\\{\}]
\PYG{c+c1}{\PYGZsh{}}
\PYG{c+c1}{\PYGZsh{} import}
\PYG{c+c1}{\PYGZsh{}}
\PYG{k+kn}{import} \PYG{n+nn}{sys}
\PYG{n}{sys}\PYG{o}{.}\PYG{n}{path}\PYG{o}{.}\PYG{n}{append}\PYG{p}{(}\PYG{l+s+s1}{\PYGZsq{}}\PYG{l+s+s1}{..}\PYG{l+s+se}{\PYGZbs{}\PYGZbs{}}\PYG{l+s+s1}{source}\PYG{l+s+s1}{\PYGZsq{}}\PYG{p}{)}
\PYG{n}{sys}\PYG{o}{.}\PYG{n}{path}\PYG{o}{.}\PYG{n}{append}\PYG{p}{(}\PYG{l+s+s1}{\PYGZsq{}}\PYG{l+s+s1}{..}\PYG{l+s+se}{\PYGZbs{}\PYGZbs{}}\PYG{l+s+s1}{run\PYGZus{}chad}\PYG{l+s+s1}{\PYGZsq{}}\PYG{p}{)}

\PYG{c+c1}{\PYGZsh{} math capability}
\PYG{k+kn}{import} \PYG{n+nn}{numpy} \PYG{k}{as} \PYG{n+nn}{np}

\PYG{c+c1}{\PYGZsh{} data frame capability}
\PYG{k+kn}{import} \PYG{n+nn}{pandas} \PYG{k}{as} \PYG{n+nn}{pd}

\PYG{c+c1}{\PYGZsh{} zipfile capability}
\PYG{k+kn}{import} \PYG{n+nn}{zipfile}

\PYG{c+c1}{\PYGZsh{} ABMHAP modules}
\PYG{k+kn}{import} \PYG{n+nn}{my\PYGZus{}globals} \PYG{k}{as} \PYG{n+nn}{mg}
\PYG{k+kn}{import} \PYG{n+nn}{chad\PYGZus{}demography\PYGZus{}adult\PYGZus{}work} \PYG{k}{as} \PYG{n+nn}{cdaw}
\PYG{k+kn}{import} \PYG{n+nn}{chad\PYGZus{}demography\PYGZus{}adult\PYGZus{}non\PYGZus{}work} \PYG{k}{as} \PYG{n+nn}{cdanw}
\PYG{k+kn}{import} \PYG{n+nn}{chad\PYGZus{}demography\PYGZus{}child\PYGZus{}school} \PYG{k}{as} \PYG{n+nn}{cdcs}
\PYG{k+kn}{import} \PYG{n+nn}{chad\PYGZus{}demography\PYGZus{}child\PYGZus{}young} \PYG{k}{as} \PYG{n+nn}{cdcy}

\PYG{k+kn}{import} \PYG{n+nn}{chad}
\end{sphinxVerbatim}

define functions

\fvset{hllines={, ,}}%
\begin{sphinxVerbatim}[commandchars=\\\{\}]
\PYG{k}{def} \PYG{n+nf}{counter}\PYG{p}{(}\PYG{n}{demos}\PYG{p}{,} \PYG{n}{names}\PYG{p}{,} \PYG{n}{key}\PYG{p}{)}\PYG{p}{:}

    \PYG{l+s+sd}{\PYGZdq{}\PYGZdq{}\PYGZdq{}}
\PYG{l+s+sd}{    This create a dataframe that contains the amount of CHAD records for the single\PYGZhy{}entry \PYGZbs{}}
\PYG{l+s+sd}{    and longitdinal data.}

\PYG{l+s+sd}{    :param demos: the demographics to compare the results to}
\PYG{l+s+sd}{    :type demoos: list of demography.Demography}
\PYG{l+s+sd}{    :param names: the names of the demographcs, respectively}
\PYG{l+s+sd}{    :type names: list of str}
\PYG{l+s+sd}{    :param int key: the ABMHAP activity code}

\PYG{l+s+sd}{    :return: a table the shows how many individuals have single\PYGZhy{}entry and longitudinal data \PYGZbs{}}
\PYG{l+s+sd}{    within each demographic}
\PYG{l+s+sd}{    :retype: pandas.core.frame.DataFrame}
\PYG{l+s+sd}{    \PYGZdq{}\PYGZdq{}\PYGZdq{}}

    \PYG{n}{do\PYGZus{}periodic} \PYG{o}{=} \PYG{k+kc}{False}

    \PYG{k}{if} \PYG{n}{key} \PYG{o}{==} \PYG{n}{mg}\PYG{o}{.}\PYG{n}{KEY\PYGZus{}SLEEP}\PYG{p}{:}
        \PYG{n}{do\PYGZus{}periodic} \PYG{o}{=} \PYG{k+kc}{True}

    \PYG{n}{solo\PYGZus{}count} \PYG{o}{=} \PYG{n}{np}\PYG{o}{.}\PYG{n}{zeros}\PYG{p}{(} \PYG{p}{(}\PYG{n+nb}{len}\PYG{p}{(}\PYG{n}{demos}\PYG{p}{)}\PYG{p}{,} \PYG{p}{)} \PYG{p}{)}
    \PYG{n}{long\PYGZus{}count} \PYG{o}{=} \PYG{n}{np}\PYG{o}{.}\PYG{n}{zeros}\PYG{p}{(} \PYG{n}{solo\PYGZus{}count}\PYG{o}{.}\PYG{n}{shape}\PYG{p}{)}

    \PYG{k}{for} \PYG{n}{i}\PYG{p}{,} \PYG{n}{demo} \PYG{o+ow}{in} \PYG{n+nb}{enumerate}\PYG{p}{(}\PYG{n}{demos}\PYG{p}{)}\PYG{p}{:}
        \PYG{n}{solo}\PYG{p}{,} \PYG{n}{long} \PYG{o}{=} \PYG{n}{f}\PYG{p}{(}\PYG{n}{demo}\PYG{o}{.}\PYG{n}{fname\PYGZus{}zip}\PYG{p}{,} \PYG{n}{demo}\PYG{o}{.}\PYG{n}{fname\PYGZus{}stats}\PYG{p}{[}\PYG{n}{key}\PYG{p}{]}\PYG{p}{[}\PYG{n}{chad}\PYG{o}{.}\PYG{n}{RECORD}\PYG{p}{]}\PYG{p}{,} \PYG{n}{demo}\PYG{o}{.}\PYG{n}{int\PYGZus{}2\PYGZus{}param}\PYG{p}{[}\PYG{n}{key}\PYG{p}{]}\PYG{p}{,}
                       \PYG{n}{do\PYGZus{}periodic}\PYG{p}{)}

        \PYG{n}{solo\PYGZus{}count}\PYG{p}{[}\PYG{n}{i}\PYG{p}{]} \PYG{o}{=} \PYG{n+nb}{sum}\PYG{p}{(} \PYG{n}{solo} \PYG{o}{==} \PYG{l+m+mi}{1} \PYG{p}{)}
        \PYG{n}{long\PYGZus{}count}\PYG{p}{[}\PYG{n}{i}\PYG{p}{]} \PYG{o}{=} \PYG{n+nb}{sum}\PYG{p}{(} \PYG{n}{long} \PYG{o}{\PYGZgt{}}\PYG{o}{=} \PYG{l+m+mi}{2}\PYG{p}{)}

    \PYG{n}{df} \PYG{o}{=} \PYG{n}{pd}\PYG{o}{.}\PYG{n}{DataFrame}\PYG{p}{(} \PYG{n}{np}\PYG{o}{.}\PYG{n}{vstack}\PYG{p}{(} \PYG{p}{(}\PYG{n}{solo\PYGZus{}count}\PYG{p}{,} \PYG{n}{long\PYGZus{}count}\PYG{p}{)} \PYG{p}{)}\PYG{o}{.}\PYG{n}{T} \PYG{p}{)}
    \PYG{n}{df}\PYG{o}{.}\PYG{n}{columns} \PYG{o}{=} \PYG{p}{(}\PYG{l+s+s1}{\PYGZsq{}}\PYG{l+s+s1}{single}\PYG{l+s+s1}{\PYGZsq{}}\PYG{p}{,} \PYG{l+s+s1}{\PYGZsq{}}\PYG{l+s+s1}{long}\PYG{l+s+s1}{\PYGZsq{}}\PYG{p}{)}
    \PYG{n}{df}\PYG{o}{.}\PYG{n}{index} \PYG{o}{=} \PYG{n}{names}

    \PYG{k}{return} \PYG{n}{df}

\PYG{k}{def} \PYG{n+nf}{f}\PYG{p}{(}\PYG{n}{fname\PYGZus{}zip}\PYG{p}{,} \PYG{n}{fname\PYGZus{}record}\PYG{p}{,} \PYG{n}{s\PYGZus{}param}\PYG{p}{,} \PYG{n}{do\PYGZus{}periodic}\PYG{p}{)}\PYG{p}{:}

    \PYG{l+s+sd}{\PYGZdq{}\PYGZdq{}\PYGZdq{}}
\PYG{l+s+sd}{    This function opens the demographic data and counts the number of both the single\PYGZhy{}entry \PYGZbs{}}
\PYG{l+s+sd}{    (solo) records and the longitudinal (multiple\PYGZhy{}entry) records that can be used within \PYGZbs{}}
\PYG{l+s+sd}{    ABMHAP according to the sepcific activity\PYGZsq{}s requirements for filtering CHAD data}

\PYG{l+s+sd}{    :param str fname\PYGZus{}zip: the file name of the .zip file of the CHAD data for a specific \PYGZbs{}}
\PYG{l+s+sd}{    demographic}
\PYG{l+s+sd}{    :param str fname\PYGZus{}record: the file name of the CHAD record data for a given activity \PYGZbs{}}
\PYG{l+s+sd}{    within the specific demographic}
\PYG{l+s+sd}{    :param chad\PYGZus{}params.CHAD\PYGZus{}params: the CHAD sampling parameters for the specific activity}
\PYG{l+s+sd}{    :param bool do\PYGZus{}periodic: a flag to inicate whether (if True) or not (if False) \PYGZbs{}}
\PYG{l+s+sd}{    to express time of day in hours [\PYGZhy{}12, 12)}

\PYG{l+s+sd}{    :return: for each person within the deographic in the CHAD data, the number of activity \PYGZbs{}}
\PYG{l+s+sd}{    instances from the single\PYGZhy{}entry record data, multiple\PYGZhy{}entry record data}
\PYG{l+s+sd}{    :rtype: numpy.ndarray, numpy.ndarray}
\PYG{l+s+sd}{    \PYGZdq{}\PYGZdq{}\PYGZdq{}}

    \PYG{c+c1}{\PYGZsh{} the zipfile of the data for the given demographic}
    \PYG{n}{z} \PYG{o}{=} \PYG{n}{zipfile}\PYG{o}{.}\PYG{n}{ZipFile}\PYG{p}{(}\PYG{n}{fname\PYGZus{}zip}\PYG{p}{)}

    \PYG{c+c1}{\PYGZsh{} count the number of activity instances per PID for the multiple\PYGZhy{}entry records}
    \PYG{n}{long} \PYG{o}{=} \PYG{n}{f\PYGZus{}temp}\PYG{p}{(}\PYG{n}{z}\PYG{p}{,} \PYG{n}{fname\PYGZus{}record}\PYG{p}{,} \PYG{n}{s\PYGZus{}param}\PYG{p}{,} \PYG{n}{do\PYGZus{}periodic}\PYG{p}{)}

    \PYG{c+c1}{\PYGZsh{} count the number of activity instances per PID for the single\PYGZhy{}entry records}
    \PYG{n}{solo} \PYG{o}{=} \PYG{n}{f\PYGZus{}temp}\PYG{p}{(}\PYG{n}{z}\PYG{p}{,} \PYG{n}{fname\PYGZus{}record}\PYG{o}{.}\PYG{n}{replace}\PYG{p}{(}\PYG{l+s+s1}{\PYGZsq{}}\PYG{l+s+s1}{longitude}\PYG{l+s+s1}{\PYGZsq{}}\PYG{p}{,} \PYG{l+s+s1}{\PYGZsq{}}\PYG{l+s+s1}{solo}\PYG{l+s+s1}{\PYGZsq{}}\PYG{p}{)}\PYG{p}{,} \PYG{n}{s\PYGZus{}param}\PYG{p}{,} \PYG{n}{do\PYGZus{}periodic}\PYG{p}{)}

    \PYG{k}{return} \PYG{n}{solo}\PYG{p}{,} \PYG{n}{long}

\PYG{k}{def} \PYG{n+nf}{f\PYGZus{}temp}\PYG{p}{(}\PYG{n}{z}\PYG{p}{,} \PYG{n}{fname\PYGZus{}record}\PYG{p}{,} \PYG{n}{s\PYGZus{}param}\PYG{p}{,} \PYG{n}{do\PYGZus{}periodic}\PYG{p}{)}\PYG{p}{:}

    \PYG{l+s+sd}{\PYGZdq{}\PYGZdq{}\PYGZdq{}}
\PYG{l+s+sd}{    This function reads the record file and counts the number of entries of a person in \PYGZbs{}}
\PYG{l+s+sd}{    CHAD for a given activity with single\PYGZhy{}entry or multiple\PYGZhy{}entry data.}

\PYG{l+s+sd}{    :param zipfile.Zipfile:}
\PYG{l+s+sd}{    :param str fname\PYGZus{}record: the file name of the CHAD record data for a given activity \PYGZbs{}}
\PYG{l+s+sd}{    within the specific demographic}
\PYG{l+s+sd}{    :param chad\PYGZus{}params.CHAD\PYGZus{}params: the CHAD sampling parameters for the specific activity}
\PYG{l+s+sd}{    :param bool do\PYGZus{}periodic: a flag to inicate whether (if True) or not (if False) \PYGZbs{}}
\PYG{l+s+sd}{    to express time of day in hours [\PYGZhy{}12, 12)}

\PYG{l+s+sd}{    :return: the number of activity instances per PID}
\PYG{l+s+sd}{    :rtype: numpy.ndarray}
\PYG{l+s+sd}{    \PYGZdq{}\PYGZdq{}\PYGZdq{}}

    \PYG{c+c1}{\PYGZsh{} read the record file}
    \PYG{n}{df}      \PYG{o}{=} \PYG{n}{pd}\PYG{o}{.}\PYG{n}{read\PYGZus{}csv}\PYG{p}{(} \PYG{n}{z}\PYG{o}{.}\PYG{n}{open}\PYG{p}{(}\PYG{n}{fname\PYGZus{}record}\PYG{p}{,} \PYG{n}{mode}\PYG{o}{=}\PYG{l+s+s1}{\PYGZsq{}}\PYG{l+s+s1}{r}\PYG{l+s+s1}{\PYGZsq{}}\PYG{p}{)} \PYG{p}{)}

    \PYG{c+c1}{\PYGZsh{} filter the dataframe for valid values for the reocrds}
    \PYG{n}{df}      \PYG{o}{=} \PYG{n}{s\PYGZus{}param}\PYG{o}{.}\PYG{n}{get\PYGZus{}record}\PYG{p}{(}\PYG{n}{df}\PYG{p}{,} \PYG{n}{do\PYGZus{}periodic}\PYG{p}{)}

    \PYG{c+c1}{\PYGZsh{} group the records by PID}
    \PYG{n}{gb}      \PYG{o}{=} \PYG{n}{df}\PYG{o}{.}\PYG{n}{groupby}\PYG{p}{(}\PYG{l+s+s1}{\PYGZsq{}}\PYG{l+s+s1}{PID}\PYG{l+s+s1}{\PYGZsq{}}\PYG{p}{)}

    \PYG{c+c1}{\PYGZsh{} count the number of records per PID}
    \PYG{n}{counts}  \PYG{o}{=} \PYG{n}{np}\PYG{o}{.}\PYG{n}{array}\PYG{p}{(} \PYG{p}{[} \PYG{n+nb}{len}\PYG{p}{(}\PYG{n}{gb}\PYG{o}{.}\PYG{n}{get\PYGZus{}group}\PYG{p}{(}\PYG{n}{u}\PYG{p}{)}\PYG{p}{)} \PYG{k}{for} \PYG{n}{u} \PYG{o+ow}{in} \PYG{n}{df}\PYG{o}{.}\PYG{n}{PID}\PYG{o}{.}\PYG{n}{unique}\PYG{p}{(}\PYG{p}{)} \PYG{p}{]} \PYG{p}{)}

    \PYG{k}{return} \PYG{n}{counts}


\PYG{k}{def} \PYG{n+nf}{print\PYGZus{}count}\PYG{p}{(}\PYG{n}{demo}\PYG{p}{,} \PYG{n}{key}\PYG{p}{,} \PYG{n}{do\PYGZus{}periodic}\PYG{o}{=}\PYG{k+kc}{False}\PYG{p}{)}\PYG{p}{:}

    \PYG{l+s+sd}{\PYGZdq{}\PYGZdq{}\PYGZdq{}}
\PYG{l+s+sd}{    This function prints the counts of single\PYGZhy{}entry data and longitudinal data.}

\PYG{l+s+sd}{    :param demography.Demography: the demographic of interest}
\PYG{l+s+sd}{    :int key: activity code}
\PYG{l+s+sd}{    :param bool do\PYGZus{}periodic: a flag to inicate whether (if True) or not (if False) \PYGZbs{}}
\PYG{l+s+sd}{    to express time of day in hours [\PYGZhy{}12, 12)}

\PYG{l+s+sd}{    :return:}
\PYG{l+s+sd}{    \PYGZdq{}\PYGZdq{}\PYGZdq{}}

    \PYG{c+c1}{\PYGZsh{} count the number of activity instances per PID for the given activity within}
    \PYG{c+c1}{\PYGZsh{} both the single\PYGZhy{}entry data and longitudinal data}
    \PYG{n}{solo}\PYG{p}{,} \PYG{n}{long} \PYG{o}{=} \PYG{n}{f}\PYG{p}{(}\PYG{n}{demo}\PYG{o}{.}\PYG{n}{fname\PYGZus{}zip}\PYG{p}{,} \PYG{n}{demo}\PYG{o}{.}\PYG{n}{fname\PYGZus{}stats}\PYG{p}{[}\PYG{n}{key}\PYG{p}{]}\PYG{p}{[}\PYG{n}{chad}\PYG{o}{.}\PYG{n}{RECORD}\PYG{p}{]}\PYG{p}{,} \PYG{n}{demo}\PYG{o}{.}\PYG{n}{int\PYGZus{}2\PYGZus{}param}\PYG{p}{[}\PYG{n}{key}\PYG{p}{]}\PYG{p}{,} \PYGZbs{}
                  \PYG{n}{do\PYGZus{}periodic}\PYG{p}{)}

    \PYG{c+c1}{\PYGZsh{} print the results}
    \PYG{n+nb}{print}\PYG{p}{(} \PYG{l+s+s1}{\PYGZsq{}}\PYG{l+s+s1}{solo: }\PYG{l+s+si}{\PYGZpc{}d}\PYG{l+s+se}{\PYGZbs{}t}\PYG{l+s+s1}{long: }\PYG{l+s+si}{\PYGZpc{}d}\PYG{l+s+s1}{\PYGZsq{}} \PYG{o}{\PYGZpc{}} \PYG{p}{(}\PYG{n+nb}{sum}\PYG{p}{(}\PYG{n}{solo} \PYG{o}{==} \PYG{l+m+mi}{1}\PYG{p}{)}\PYG{p}{,} \PYG{n+nb}{sum}\PYG{p}{(}\PYG{n}{long} \PYG{o}{\PYGZgt{}}\PYG{o}{=} \PYG{l+m+mi}{2}\PYG{p}{)} \PYG{p}{)} \PYG{p}{)}

    \PYG{k}{return}
\end{sphinxVerbatim}

load the demographics information

\fvset{hllines={, ,}}%
\begin{sphinxVerbatim}[commandchars=\\\{\}]
\PYG{c+c1}{\PYGZsh{}}
\PYG{c+c1}{\PYGZsh{} load demographics}
\PYG{c+c1}{\PYGZsh{}}
\PYG{n}{adult\PYGZus{}work} \PYG{o}{=} \PYG{n}{cdaw}\PYG{o}{.}\PYG{n}{CHAD\PYGZus{}demography\PYGZus{}adult\PYGZus{}work}\PYG{p}{(}\PYG{p}{)}
\PYG{n}{adult\PYGZus{}non\PYGZus{}work} \PYG{o}{=} \PYG{n}{cdanw}\PYG{o}{.}\PYG{n}{CHAD\PYGZus{}demography\PYGZus{}adult\PYGZus{}non\PYGZus{}work}\PYG{p}{(}\PYG{p}{)}
\PYG{n}{child\PYGZus{}school} \PYG{o}{=} \PYG{n}{cdcs}\PYG{o}{.}\PYG{n}{CHAD\PYGZus{}demography\PYGZus{}child\PYGZus{}school}\PYG{p}{(}\PYG{p}{)}
\PYG{n}{child\PYGZus{}young} \PYG{o}{=} \PYG{n}{cdcy}\PYG{o}{.}\PYG{n}{CHAD\PYGZus{}demography\PYGZus{}child\PYGZus{}young}\PYG{p}{(}\PYG{p}{)}
\end{sphinxVerbatim}

\fvset{hllines={, ,}}%
\begin{sphinxVerbatim}[commandchars=\\\{\}]
\PYG{c+c1}{\PYGZsh{} set the demographics and names for the data frame rows}
\PYG{n}{demos} \PYG{o}{=} \PYG{p}{[}\PYG{n}{adult\PYGZus{}work}\PYG{p}{,} \PYG{n}{adult\PYGZus{}non\PYGZus{}work}\PYG{p}{,} \PYG{n}{child\PYGZus{}school}\PYG{p}{,} \PYG{n}{child\PYGZus{}young}\PYG{p}{]}
\PYG{n}{names} \PYG{o}{=} \PYG{p}{[}\PYG{l+s+s1}{\PYGZsq{}}\PYG{l+s+s1}{adult\PYGZus{}work}\PYG{l+s+s1}{\PYGZsq{}}\PYG{p}{,} \PYG{l+s+s1}{\PYGZsq{}}\PYG{l+s+s1}{adult\PYGZus{}non\PYGZus{}work}\PYG{l+s+s1}{\PYGZsq{}}\PYG{p}{,} \PYG{l+s+s1}{\PYGZsq{}}\PYG{l+s+s1}{child\PYGZus{}school}\PYG{l+s+s1}{\PYGZsq{}}\PYG{p}{,} \PYG{l+s+s1}{\PYGZsq{}}\PYG{l+s+s1}{child\PYGZus{}young}\PYG{l+s+s1}{\PYGZsq{}}\PYG{p}{]}

\PYG{n}{demos\PYGZus{}work} \PYG{o}{=} \PYG{p}{[}\PYG{n}{adult\PYGZus{}work}\PYG{p}{,} \PYG{n}{child\PYGZus{}school}\PYG{p}{]}
\PYG{n}{names\PYGZus{}work} \PYG{o}{=} \PYG{p}{[}\PYG{l+s+s1}{\PYGZsq{}}\PYG{l+s+s1}{adult\PYGZus{}work}\PYG{l+s+s1}{\PYGZsq{}}\PYG{p}{,} \PYG{l+s+s1}{\PYGZsq{}}\PYG{l+s+s1}{child\PYGZus{}school}\PYG{l+s+s1}{\PYGZsq{}}\PYG{p}{]}
\end{sphinxVerbatim}

meals and sleep

\fvset{hllines={, ,}}%
\begin{sphinxVerbatim}[commandchars=\\\{\}]
\PYG{c+c1}{\PYGZsh{} breakfast}
\PYG{n}{bf} \PYG{o}{=} \PYG{n}{counter}\PYG{p}{(}\PYG{n}{demos}\PYG{p}{,} \PYG{n}{names}\PYG{p}{,} \PYG{n}{mg}\PYG{o}{.}\PYG{n}{KEY\PYGZus{}EAT\PYGZus{}BREAKFAST}\PYG{p}{)}

\PYG{c+c1}{\PYGZsh{} lunch}
\PYG{n}{lunch} \PYG{o}{=} \PYG{n}{counter}\PYG{p}{(}\PYG{n}{demos}\PYG{p}{,} \PYG{n}{names}\PYG{p}{,} \PYG{n}{mg}\PYG{o}{.}\PYG{n}{KEY\PYGZus{}EAT\PYGZus{}LUNCH}\PYG{p}{)}

\PYG{c+c1}{\PYGZsh{} dinner}
\PYG{n}{dinner} \PYG{o}{=} \PYG{n}{counter}\PYG{p}{(}\PYG{n}{demos}\PYG{p}{,} \PYG{n}{names}\PYG{p}{,} \PYG{n}{mg}\PYG{o}{.}\PYG{n}{KEY\PYGZus{}EAT\PYGZus{}DINNER}\PYG{p}{)}

\PYG{c+c1}{\PYGZsh{} sleep}
\PYG{n}{sleep} \PYG{o}{=} \PYG{n}{counter}\PYG{p}{(}\PYG{n}{demos}\PYG{p}{,} \PYG{n}{names}\PYG{p}{,} \PYG{n}{mg}\PYG{o}{.}\PYG{n}{KEY\PYGZus{}SLEEP}\PYG{p}{)}
\end{sphinxVerbatim}

commuting, working

\fvset{hllines={, ,}}%
\begin{sphinxVerbatim}[commandchars=\\\{\}]
\PYG{n}{work} \PYG{o}{=} \PYG{n}{counter}\PYG{p}{(}\PYG{n}{demos\PYGZus{}work}\PYG{p}{,} \PYG{n}{names\PYGZus{}work}\PYG{p}{,} \PYG{n}{mg}\PYG{o}{.}\PYG{n}{KEY\PYGZus{}WORK}\PYG{p}{)}
\PYG{n}{commute\PYGZus{}to\PYGZus{}work} \PYG{o}{=} \PYG{n}{counter}\PYG{p}{(}\PYG{n}{demos\PYGZus{}work}\PYG{p}{,} \PYG{n}{names\PYGZus{}work}\PYG{p}{,} \PYG{n}{mg}\PYG{o}{.}\PYG{n}{KEY\PYGZus{}COMMUTE\PYGZus{}TO\PYGZus{}WORK}\PYG{p}{)}
\PYG{n}{commute\PYGZus{}from\PYGZus{}work} \PYG{o}{=} \PYG{n}{counter}\PYG{p}{(}\PYG{n}{demos\PYGZus{}work}\PYG{p}{,} \PYG{n}{names\PYGZus{}work}\PYG{p}{,} \PYG{n}{mg}\PYG{o}{.}\PYG{n}{KEY\PYGZus{}COMMUTE\PYGZus{}FROM\PYGZus{}WORK}\PYG{p}{)}
\end{sphinxVerbatim}

View

\fvset{hllines={, ,}}%
\begin{sphinxVerbatim}[commandchars=\\\{\}]
\PYG{n}{sleep}
\end{sphinxVerbatim}




\subsection{datum module}
\label{\detokenize{datum::doc}}\label{\detokenize{datum:datum-module}}\label{\detokenize{datum:module-datum}}\index{datum (module)}
This module contains functions that analyze the raw data from the Consolidated Human Activity Database (CHAD) to be processed/ filtered for use by the Agent-Based Model of Human Activity Patterns (ABMHAP).

This function primarily encapsulates functions to analyze data to be used as an imported module. However, it may also be run as a main file.
\index{analyze\_commute() (in module datum)}

\begin{fulllineitems}
\phantomsection\label{\detokenize{datum:datum.analyze_commute}}\pysiglinewithargsret{\sphinxcode{\sphinxupquote{datum.}}\sphinxbfcode{\sphinxupquote{analyze\_commute}}}{\emph{data}}{}
This function analyzes the commuting data to get information about BOTH     commuting to work, commuting from work, \sphinxstylestrong{AND} working. The data are chosen from     entries where a work event is sandwiched between a commuting to work event and a     commuting from work event. The commuting data and working data are processed and     filtered for use for ABMHAP.
\begin{quote}\begin{description}
\item[{Parameters}] \leavevmode
\sphinxstyleliteralstrong{\sphinxupquote{data}} ({\hyperref[\detokenize{chad:chad.CHAD_RAW}]{\sphinxcrossref{\sphinxstyleliteralemphasis{\sphinxupquote{chad.CHAD\_RAW}}}}}) \textendash{} the raw CHAD data

\item[{Returns}] \leavevmode
the raw CHAD commuting data also the data of people with both commute and work data,    statistical data of commuting to work, statistical data of commuting from work,     statistical data of working.

\item[{Return type}] \leavevmode
dictionary, dictionary, dictionary, dictionary

\end{description}\end{quote}

\end{fulllineitems}

\index{analyze\_commute\_school() (in module datum)}

\begin{fulllineitems}
\phantomsection\label{\detokenize{datum:datum.analyze_commute_school}}\pysiglinewithargsret{\sphinxcode{\sphinxupquote{datum.}}\sphinxbfcode{\sphinxupquote{analyze\_commute\_school}}}{\emph{data}}{}
This function analyzes the commuting to school data to get information     to get data about commuting to school and commuting from school. The     commuting to school data are processed and filtered for use for ABMHAP.
\begin{quote}\begin{description}
\item[{Parameters}] \leavevmode
\sphinxstyleliteralstrong{\sphinxupquote{data}} ({\hyperref[\detokenize{chad:chad.CHAD_RAW}]{\sphinxcrossref{\sphinxstyleliteralemphasis{\sphinxupquote{chad.CHAD\_RAW}}}}}) \textendash{} the raw CHAD data

\item[{Returns}] \leavevmode
the raw CHAD commuting data also the CHAD commuting data modified to handle     over night events, statistical data of commuting to school, statistical data of     commuting from school

\item[{Return type}] \leavevmode
dictionary, dictionary, dictionary

\end{description}\end{quote}

\end{fulllineitems}

\index{analyze\_eat() (in module datum)}

\begin{fulllineitems}
\phantomsection\label{\detokenize{datum:datum.analyze_eat}}\pysiglinewithargsret{\sphinxcode{\sphinxupquote{datum.}}\sphinxbfcode{\sphinxupquote{analyze\_eat}}}{\emph{data}}{}
This function analyzes the CHAD data for eating in order to get information     on eating breakfast, eating lunch, and eating dinner data. The data     are processed and filtered for use for ABMHAP for the respective activities.
\begin{quote}\begin{description}
\item[{Parameters}] \leavevmode
\sphinxstyleliteralstrong{\sphinxupquote{data}} ({\hyperref[\detokenize{chad:chad.CHAD_RAW}]{\sphinxcrossref{\sphinxstyleliteralemphasis{\sphinxupquote{chad.CHAD\_RAW}}}}}) \textendash{} the raw CHAD data

\item[{Returns}] \leavevmode
statistical data of eating breakfast, statistical data of eating lunch,     statistical data of eating dinner

\item[{Return type}] \leavevmode
dictionary, dictionary, dictionary

\end{description}\end{quote}

\end{fulllineitems}

\index{analyze\_education() (in module datum)}

\begin{fulllineitems}
\phantomsection\label{\detokenize{datum:datum.analyze_education}}\pysiglinewithargsret{\sphinxcode{\sphinxupquote{datum.}}\sphinxbfcode{\sphinxupquote{analyze\_education}}}{\emph{data}}{}
This function analyzes the CHAD data for schooling in order to get information on     going to school.     The data are processed and filtered for use for ABMHAP for the school activity,     namely school data are only taken if the event is considered “fulltime”,     (i.e., having a long enough duration) in order to avoid part-time school events.
\begin{quote}\begin{description}
\item[{Parameters}] \leavevmode
\sphinxstyleliteralstrong{\sphinxupquote{data}} ({\hyperref[\detokenize{chad:chad.CHAD_RAW}]{\sphinxcrossref{\sphinxstyleliteralemphasis{\sphinxupquote{chad.CHAD\_RAW}}}}}) \textendash{} the raw CHAD data

\item[{Returns}] \leavevmode
the CHAD schooling data for “fulltime” educational data.

\item[{Return type}] \leavevmode
dictionary

\end{description}\end{quote}

\end{fulllineitems}

\index{analyze\_moments() (in module datum)}

\begin{fulllineitems}
\phantomsection\label{\detokenize{datum:datum.analyze_moments}}\pysiglinewithargsret{\sphinxcode{\sphinxupquote{datum.}}\sphinxbfcode{\sphinxupquote{analyze\_moments}}}{\emph{df}, \emph{start\_periodic=False}}{}
This function analyzes the data for each person by calculating the moments for     duration, start time, and end time for the following three cases.
\begin{enumerate}
\item {} 
General (weekday and weekend)

\item {} 
Weekday

\item {} 
Weekend

\end{enumerate}
\begin{quote}\begin{description}
\item[{Parameters}] \leavevmode
\sphinxstyleliteralstrong{\sphinxupquote{df}} (\sphinxstyleliteralemphasis{\sphinxupquote{pandas.core.frame}}) \textendash{} the data in the form of CHAD records to analyze

\item[{Returns}] \leavevmode
the statistical moments data for the following:     general duration, general start time, general end time,     weekday duration, weekday start time, weekday end time,     weekend duration, weekend start time, weekend end time

\item[{Return type}] \leavevmode
pandas.core.frame.DataFrame, pandas.core.frame.DataFrame, pandas.core.frame.DataFrame,     pandas.core.frame.DataFrame, pandas.core.frame.DataFrame, pandas.core.frame.DataFrame,     pandas.core.frame.DataFrame, pandas.core.frame.DatFrame, pandas.core.frame.DataFrame

\end{description}\end{quote}

\end{fulllineitems}

\index{analyze\_sleep() (in module datum)}

\begin{fulllineitems}
\phantomsection\label{\detokenize{datum:datum.analyze_sleep}}\pysiglinewithargsret{\sphinxcode{\sphinxupquote{datum.}}\sphinxbfcode{\sphinxupquote{analyze\_sleep}}}{\emph{data}}{}
This function analyzes the CHAD data for sleeping in order to get information     on sleeping. The data are processed and filtered for use for ABMHAP for the     sleep activity.
\begin{quote}\begin{description}
\item[{Parameters}] \leavevmode
\sphinxstyleliteralstrong{\sphinxupquote{data}} ({\hyperref[\detokenize{chad:chad.CHAD_RAW}]{\sphinxcrossref{\sphinxstyleliteralemphasis{\sphinxupquote{chad.CHAD\_RAW}}}}}) \textendash{} the raw CHAD data

\item[{Returns}] \leavevmode
the statistical data on CHAD sleep data

\item[{Return type}] \leavevmode
dictionary

\end{description}\end{quote}

\end{fulllineitems}

\index{analyze\_work() (in module datum)}

\begin{fulllineitems}
\phantomsection\label{\detokenize{datum:datum.analyze_work}}\pysiglinewithargsret{\sphinxcode{\sphinxupquote{datum.}}\sphinxbfcode{\sphinxupquote{analyze\_work}}}{\emph{data}}{}
This function analyzes the CHAD data for working. The data are processed and     filtered for use for ABMHAP for the work activity. Data in only chosen if the     person surveyed in CHAD is marked as fulltime employed. This function does a statistical     analysis of the following:
\begin{enumerate}
\item {} 
raw work data

\item {} 
longitudinal data

\item {} 
fulltime work data

\end{enumerate}

\begin{sphinxadmonition}{warning}{Warning:}
This function may be antiquated and not currently used. Instead see {\hyperref[\detokenize{datum:datum.analyze_commute}]{\sphinxcrossref{\sphinxcode{\sphinxupquote{analyze\_commute()}}}}}         for obtaining work information.
\end{sphinxadmonition}
\begin{quote}\begin{description}
\item[{Parameters}] \leavevmode
\sphinxstyleliteralstrong{\sphinxupquote{data}} ({\hyperref[\detokenize{chad:chad.CHAD_RAW}]{\sphinxcrossref{\sphinxstyleliteralemphasis{\sphinxupquote{chad.CHAD\_RAW}}}}}) \textendash{} the raw CHAD data

\item[{Returns}] \leavevmode
statistical data on CHAD work data on the following: raw CHAD data,     raw CHAD data after being processed for overnight activities, raw CHAD data     after being processed for data from people employed fulltime

\item[{Return type}] \leavevmode
dictionary, dictionary, dictionary

\end{description}\end{quote}

\end{fulllineitems}

\index{filter\_commute() (in module datum)}

\begin{fulllineitems}
\phantomsection\label{\detokenize{datum:datum.filter_commute}}\pysiglinewithargsret{\sphinxcode{\sphinxupquote{datum.}}\sphinxbfcode{\sphinxupquote{filter\_commute}}}{\emph{df}, \emph{start\_min}, \emph{start\_max}, \emph{end\_max}}{}
This function finds indices of the data that satisfy the filters     placed on the commuting data by limiting the data to be within the start     time range and end time range.
\begin{quote}\begin{description}
\item[{Parameters}] \leavevmode\begin{itemize}
\item {} 
\sphinxstyleliteralstrong{\sphinxupquote{df}} (\sphinxstyleliteralemphasis{\sphinxupquote{pandas.core.frame.DataFrame}}) \textendash{} the commuting data

\item {} 
\sphinxstyleliteralstrong{\sphinxupquote{start\_min}} (\sphinxstyleliteralemphasis{\sphinxupquote{float}}) \textendash{} the minimum start time {[}hours{]}

\item {} 
\sphinxstyleliteralstrong{\sphinxupquote{start\_max}} (\sphinxstyleliteralemphasis{\sphinxupquote{float}}) \textendash{} the maximum start time {[}hours{]}

\item {} 
\sphinxstyleliteralstrong{\sphinxupquote{end\_max}} (\sphinxstyleliteralemphasis{\sphinxupquote{float}}) \textendash{} the maximum end time {[}hours{]}

\end{itemize}

\item[{Returns}] \leavevmode
indices of the commuting data that satisfy the filtering

\item[{Return type}] \leavevmode
numpy.ndarray

\end{description}\end{quote}

\end{fulllineitems}

\index{get\_commute\_data() (in module datum)}

\begin{fulllineitems}
\phantomsection\label{\detokenize{datum:datum.get_commute_data}}\pysiglinewithargsret{\sphinxcode{\sphinxupquote{datum.}}\sphinxbfcode{\sphinxupquote{get\_commute\_data}}}{\emph{df\_all}}{}
This function finds the following commuting data for BOTH commuting to work AND commuting     from work.
\begin{quote}\begin{description}
\item[{Parameters}] \leavevmode
\sphinxstyleliteralstrong{\sphinxupquote{df\_all}} (\sphinxstyleliteralemphasis{\sphinxupquote{pandas.core.frame.DataFrame}}) \textendash{} the dataframe containing commuting and work data

\item[{Returns}] \leavevmode
the commute to work data, the commute from work data, the work activity data

\end{description}\end{quote}

\end{fulllineitems}

\index{get\_data\_help() (in module datum)}

\begin{fulllineitems}
\phantomsection\label{\detokenize{datum:datum.get_data_help}}\pysiglinewithargsret{\sphinxcode{\sphinxupquote{datum.}}\sphinxbfcode{\sphinxupquote{get\_data\_help}}}{\emph{idx}, \emph{stats\_dt}, \emph{stats\_start}, \emph{stats\_end}, \emph{record}}{}
This function returns statistical information from the activity duration,     start time, end time, and the CHAD records from the given indices.
\begin{quote}\begin{description}
\item[{Parameters}] \leavevmode\begin{itemize}
\item {} 
\sphinxstyleliteralstrong{\sphinxupquote{idx}} (\sphinxstyleliteralemphasis{\sphinxupquote{numpy.ndarray}}) \textendash{} the indices of the CHAD individuals to     keep in the statistical data

\item {} 
\sphinxstyleliteralstrong{\sphinxupquote{stats\_dt}} (\sphinxstyleliteralemphasis{\sphinxupquote{pandas.core.frame.DataFrame}}) \textendash{} the statistical moments for the     activity duration

\item {} 
\sphinxstyleliteralstrong{\sphinxupquote{stats\_start}} (\sphinxstyleliteralemphasis{\sphinxupquote{pandas.core.frame.DataFrame}}) \textendash{} the statistical moments for     the start time activity duration

\item {} 
\sphinxstyleliteralstrong{\sphinxupquote{stats\_end}} (\sphinxstyleliteralemphasis{\sphinxupquote{pandas.core.frame.DataFrame}}) \textendash{} the statistical moments for     the end time activity duration

\item {} 
\sphinxstyleliteralstrong{\sphinxupquote{record}} (\sphinxstyleliteralemphasis{\sphinxupquote{pandas.core.frame.DataFrame}}) \textendash{} the CHAD records for a given     activity

\end{itemize}

\item[{Returns}] \leavevmode
the statistical data on duration, start time, and end time;     the CHAD record data from the chosen individuals given by the indices.

\item[{Return type}] \leavevmode
pandas.core.frame.DataFrame, pandas.core.frame.DataFrame,     pandas.core.frame.DataFrame

\end{description}\end{quote}

\end{fulllineitems}

\index{get\_end\_date() (in module datum)}

\begin{fulllineitems}
\phantomsection\label{\detokenize{datum:datum.get_end_date}}\pysiglinewithargsret{\sphinxcode{\sphinxupquote{datum.}}\sphinxbfcode{\sphinxupquote{get\_end\_date}}}{\emph{date}, \emph{start}, \emph{end}}{}
This function finds the date that an activity ends.
\begin{quote}\begin{description}
\item[{Parameters}] \leavevmode\begin{itemize}
\item {} 
\sphinxstyleliteralstrong{\sphinxupquote{date}} \textendash{} the date the activities start

\item {} 
\sphinxstyleliteralstrong{\sphinxupquote{start}} (\sphinxstyleliteralemphasis{\sphinxupquote{numpy.ndarray}}) \textendash{} the start time of the activities

\item {} 
\sphinxstyleliteralstrong{\sphinxupquote{end}} (\sphinxstyleliteralemphasis{\sphinxupquote{numpy.ndarray}}) \textendash{} the end time of activities

\end{itemize}

\item[{Type}] \leavevmode
numpy.ndarray of datetime.timedelta

\item[{Returns}] \leavevmode
the end date for an activity

\item[{Return type}] \leavevmode
numpy.ndarray of datetime.timedelta

\end{description}\end{quote}

\end{fulllineitems}

\index{get\_fulltime\_data() (in module datum)}

\begin{fulllineitems}
\phantomsection\label{\detokenize{datum:datum.get_fulltime_data}}\pysiglinewithargsret{\sphinxcode{\sphinxupquote{datum.}}\sphinxbfcode{\sphinxupquote{get\_fulltime\_data}}}{\emph{df}, \emph{start\_min=4}}{}
This function finds the data from CHAD that pertain to individuals that     are working fulltime. That is, activities starting with with a minimum     given mean start time.
\begin{quote}\begin{description}
\item[{Parameters}] \leavevmode\begin{itemize}
\item {} 
\sphinxstyleliteralstrong{\sphinxupquote{df}} (\sphinxstyleliteralemphasis{\sphinxupquote{pandas.core.frame.DataFrame}}) \textendash{} the CHAD work data

\item {} 
\sphinxstyleliteralstrong{\sphinxupquote{start\_min}} (\sphinxstyleliteralemphasis{\sphinxupquote{float}}) \textendash{} the minimum start time to be accepted {[}0, 24)

\end{itemize}

\item[{Returns}] \leavevmode
the data frame of the workers

\item[{Return type}] \leavevmode
pandas.core.frame.DataFrame

\end{description}\end{quote}

\end{fulllineitems}

\index{get\_longitude() (in module datum)}

\begin{fulllineitems}
\phantomsection\label{\detokenize{datum:datum.get_longitude}}\pysiglinewithargsret{\sphinxcode{\sphinxupquote{datum.}}\sphinxbfcode{\sphinxupquote{get\_longitude}}}{\emph{stats\_dt}, \emph{stats\_start}, \emph{stats\_end}, \emph{record}, \emph{N=2}}{}
This function gets the longitudinal CHAD statistical data for     duration, start time, and end time. This function also gets     the CHAD record data from the respective statistical data.
\begin{quote}\begin{description}
\item[{Parameters}] \leavevmode\begin{itemize}
\item {} 
\sphinxstyleliteralstrong{\sphinxupquote{stats\_dt}} (\sphinxstyleliteralemphasis{\sphinxupquote{pandas.core.frame.DataFrame}}) \textendash{} the statistical moments for the     activity duration

\item {} 
\sphinxstyleliteralstrong{\sphinxupquote{stats\_start}} (\sphinxstyleliteralemphasis{\sphinxupquote{pandas.core.frame.DataFrame}}) \textendash{} the statistical moments for     the start time activity duration

\item {} 
\sphinxstyleliteralstrong{\sphinxupquote{stats\_end}} (\sphinxstyleliteralemphasis{\sphinxupquote{pandas.core.frame.DataFrame}}) \textendash{} the statistical moments for     the end time activity duration

\item {} 
\sphinxstyleliteralstrong{\sphinxupquote{record}} (\sphinxstyleliteralemphasis{\sphinxupquote{pandas.core.frame.DataFrame}}) \textendash{} the CHAD records for a given     activity

\item {} 
\sphinxstyleliteralstrong{\sphinxupquote{N}} (\sphinxstyleliteralemphasis{\sphinxupquote{int}}) \textendash{} the minimum number of activities to be considered     longitudinal

\end{itemize}

\item[{Returns}] \leavevmode
longitudinal data for statistical moments for activity duration,     start time, and end time also longitudinal CHAD records

\item[{Return type}] \leavevmode
pandas.core.frame.DataFrame, pandas.core.frame.DataFrame,     pandas.core.frame.DataFrame, pandas.core.frame.DataFrame

\end{description}\end{quote}

\end{fulllineitems}

\index{get\_meals() (in module datum)}

\begin{fulllineitems}
\phantomsection\label{\detokenize{datum:datum.get_meals}}\pysiglinewithargsret{\sphinxcode{\sphinxupquote{datum.}}\sphinxbfcode{\sphinxupquote{get\_meals}}}{\emph{df}}{}
This function takes in eating data and separates that data into meals: breakfast,     lunch, and dinner by filtering the data by minimum and maximum start time,     end time, and duration.
\begin{quote}\begin{description}
\item[{Parameters}] \leavevmode
\sphinxstyleliteralstrong{\sphinxupquote{df}} (\sphinxstyleliteralemphasis{\sphinxupquote{pandas.core.frame.DataFrame}}) \textendash{} CHAD data on the eating data

\item[{Returns}] \leavevmode
breakfast data, lunch data, and dinner data

\item[{Return type}] \leavevmode
pandas.core.frame.DataFrame, pandas.core.frame.DataFrame,     pandas.core.frame.DataFrame

\end{description}\end{quote}

\end{fulllineitems}

\index{get\_moments() (in module datum)}

\begin{fulllineitems}
\phantomsection\label{\detokenize{datum:datum.get_moments}}\pysiglinewithargsret{\sphinxcode{\sphinxupquote{datum.}}\sphinxbfcode{\sphinxupquote{get\_moments}}}{\emph{x}, \emph{start\_periodic}}{}
This function calculates data about the moments of start time, end time, and duration     weekday + weekend data, weekday data, weekend data. Also there are the CHAD records for     the following situations: daily data, weekday data, and weekend data.
\begin{quote}\begin{description}
\item[{Parameters}] \leavevmode\begin{itemize}
\item {} 
\sphinxstyleliteralstrong{\sphinxupquote{x}} (\sphinxstyleliteralemphasis{\sphinxupquote{pandas.core.frame.DataFrame}}) \textendash{} the CHAD data to be analyzed

\item {} 
\sphinxstyleliteralstrong{\sphinxupquote{start\_periodic}} (\sphinxstyleliteralemphasis{\sphinxupquote{bool}}) \textendash{} a flag indicating whether start times should be analyzed     in {[}-12, 12) if true or {[}0, 24) if false

\end{itemize}

\item[{Returns}] \leavevmode
a dictionary of statistical moments for the following data: duration,     start time, end time, weekday duration, weekday start time, weekday end time,     weekend duration, weekend start time, weekend end time. Also there are the following     CHAD records: daily records, weekend records, weekday records.

\item[{Return type}] \leavevmode
dictionary of pandas.core.frame.DataFrame, pandas.core.frame.DataFrame, pandas.core.frame.DataFrame,     pandas.core.frame.DataFrame, pandas.core.frame.DataFrame, pandas.core.frame.DataFrame,     pandas.core.frame.DataFrame, pandas.core.frame.DataFrame, pandas.core.frame.DataFrame,     pandas.core.frame.DataFrame, pandas.core.frame.DataFrame, pandas.core.frame.DataFrame

\end{description}\end{quote}

\end{fulllineitems}

\index{get\_skipped\_meals() (in module datum)}

\begin{fulllineitems}
\phantomsection\label{\detokenize{datum:datum.get_skipped_meals}}\pysiglinewithargsret{\sphinxcode{\sphinxupquote{datum.}}\sphinxbfcode{\sphinxupquote{get\_skipped\_meals}}}{\emph{df}}{}
For each person identified within CHAD, this function goes through activity     data and finds, on a workday, and finds whether or not the individual     skipped a meal (i.e., skipped breakfast, lunch, and/ or dinner).

\begin{sphinxadmonition}{warning}{Warning:}
This function is antiquated and not used.
\end{sphinxadmonition}
\begin{quote}\begin{description}
\item[{Parameters}] \leavevmode
\sphinxstyleliteralstrong{\sphinxupquote{df}} (\sphinxstyleliteralemphasis{\sphinxupquote{pandas.core.frame.DataFrame}}) \textendash{} CHAD activity data

\item[{Returns}] \leavevmode
the activity data of people within CHAD where a meal was skipped

\item[{Return type}] \leavevmode
pandas.core.frame.DataFrame

\end{description}\end{quote}

\end{fulllineitems}

\index{get\_solo() (in module datum)}

\begin{fulllineitems}
\phantomsection\label{\detokenize{datum:datum.get_solo}}\pysiglinewithargsret{\sphinxcode{\sphinxupquote{datum.}}\sphinxbfcode{\sphinxupquote{get\_solo}}}{\emph{stats\_dt}, \emph{stats\_start}, \emph{stats\_end}, \emph{record}}{}
This function gets the single-day (i.e. from individuals with only 1 entry)     CHAD statistical data for     duration, start time, and end time. This function also gets     the CHAD record data from the respective statistical data.
\begin{quote}\begin{description}
\item[{Parameters}] \leavevmode\begin{itemize}
\item {} 
\sphinxstyleliteralstrong{\sphinxupquote{stats\_dt}} (\sphinxstyleliteralemphasis{\sphinxupquote{pandas.core.frame.DataFrame}}) \textendash{} the statistical moments for the     activity duration

\item {} 
\sphinxstyleliteralstrong{\sphinxupquote{stats\_start}} (\sphinxstyleliteralemphasis{\sphinxupquote{pandas.core.frame.DataFrame}}) \textendash{} the statistical moments for     the start time activity duration

\item {} 
\sphinxstyleliteralstrong{\sphinxupquote{stats\_end}} (\sphinxstyleliteralemphasis{\sphinxupquote{pandas.core.frame.DataFrame}}) \textendash{} the statistical moments for     the end time activity duration

\item {} 
\sphinxstyleliteralstrong{\sphinxupquote{record}} (\sphinxstyleliteralemphasis{\sphinxupquote{pandas.core.frame.DataFrame}}) \textendash{} the CHAD records for a given     activity

\end{itemize}

\item[{Returns}] \leavevmode
single-day data for statistical moments for activity duration,     start time, and end time also longitudinal CHAD records

\item[{Return type}] \leavevmode
pandas.core.frame.DataFrame, pandas.core.frame.DataFrame,     pandas.core.frame.DataFrame, pandas.core.frame.DataFrame

\end{description}\end{quote}

\end{fulllineitems}

\index{get\_stats() (in module datum)}

\begin{fulllineitems}
\phantomsection\label{\detokenize{datum:datum.get_stats}}\pysiglinewithargsret{\sphinxcode{\sphinxupquote{datum.}}\sphinxbfcode{\sphinxupquote{get\_stats}}}{\emph{pid}, \emph{data}, \emph{do\_periodic=False}}{}
This function gets the statistics about an activity-parameter (start time, end time,     or duration) and stores the following data within a dataframe:
\begin{enumerate}
\item {} 
person identifier (PID)

\item {} 
the number of events (N)

\item {} 
the mean (mu)

\item {} 
the standard deviation (std)

\item {} 
the coefficient of variation (cv)

\end{enumerate}
\begin{quote}\begin{description}
\item[{Parameters}] \leavevmode\begin{itemize}
\item {} 
\sphinxstyleliteralstrong{\sphinxupquote{pid}} (\sphinxstyleliteralemphasis{\sphinxupquote{numpy.ndarray of str}}) \textendash{} the identifiers for the individuals within CHAD for a given activity

\item {} 
\sphinxstyleliteralstrong{\sphinxupquote{data}} (\sphinxstyleliteralemphasis{\sphinxupquote{numpy.ndarray}}) \textendash{} the CHAD records for a given activity

\item {} 
\sphinxstyleliteralstrong{\sphinxupquote{do\_periodic}} (\sphinxstyleliteralemphasis{\sphinxupquote{bool}}) \textendash{} a flag whether (if True) or not (if False) time of day     should be expressed in {[}-12, 12)

\end{itemize}

\item[{Returns}] \leavevmode
the statistical results from an activity-parameter (start time, end time,     or duration)

\item[{Return type}] \leavevmode
pandas.core.frame.DataFrame

\end{description}\end{quote}

\end{fulllineitems}

\index{get\_stats\_individual() (in module datum)}

\begin{fulllineitems}
\phantomsection\label{\detokenize{datum:datum.get_stats_individual}}\pysiglinewithargsret{\sphinxcode{\sphinxupquote{datum.}}\sphinxbfcode{\sphinxupquote{get\_stats\_individual}}}{\emph{x}}{}
This function gets the data from the records and returns the following.
\begin{enumerate}
\item {} 
the mean (mu)

\item {} 
the standard deviation (std)

\item {} 
the coefficient of variation (cv)

\item {} 
the number of events (N)

\end{enumerate}
\begin{quote}\begin{description}
\item[{Parameters}] \leavevmode
\sphinxstyleliteralstrong{\sphinxupquote{x}} (\sphinxstyleliteralemphasis{\sphinxupquote{numpy.ndarray}}) \textendash{} the individual records data

\item[{Returns}] \leavevmode
the mean, standard deviation, coefficient of variation,     and  number of entries

\item[{Return type}] \leavevmode
numpy.ndarray, numpy.ndarray, numpy.ndarray, int

\end{description}\end{quote}

\end{fulllineitems}

\index{get\_stats\_weekend() (in module datum)}

\begin{fulllineitems}
\phantomsection\label{\detokenize{datum:datum.get_stats_weekend}}\pysiglinewithargsret{\sphinxcode{\sphinxupquote{datum.}}\sphinxbfcode{\sphinxupquote{get\_stats\_weekend}}}{\emph{pid}, \emph{data}, \emph{date}, \emph{start}, \emph{end}, \emph{do\_weekend=True}, \emph{do\_periodic=False}}{}
This function calculates the stats about the moments of the activity     that occur on a weekends OR weekedays.
\begin{quote}\begin{description}
\item[{Parameters}] \leavevmode\begin{itemize}
\item {} 
\sphinxstyleliteralstrong{\sphinxupquote{pid}} (\sphinxstyleliteralemphasis{\sphinxupquote{numpy.ndarray of str}}) \textendash{} the personal identifiers in the CHAD data

\item {} 
\sphinxstyleliteralstrong{\sphinxupquote{data}} \textendash{} the CHAD records of the activity data

\item {} 
\sphinxstyleliteralstrong{\sphinxupquote{date}} (\sphinxstyleliteralemphasis{\sphinxupquote{numpy.ndarray of datetime.timedelta}}) \textendash{} the dates of the activity data

\item {} 
\sphinxstyleliteralstrong{\sphinxupquote{start}} (\sphinxstyleliteralemphasis{\sphinxupquote{numpy.ndarray}}) \textendash{} the start time of the activity data

\item {} 
\sphinxstyleliteralstrong{\sphinxupquote{end}} (\sphinxstyleliteralemphasis{\sphinxupquote{numpy.ndarray}}) \textendash{} the end time of the activity data

\item {} 
\sphinxstyleliteralstrong{\sphinxupquote{do\_weekend}} (\sphinxstyleliteralemphasis{\sphinxupquote{bool}}) \textendash{} a flag whether (if True) to use data that occurs on the     weekend or (if False) and the weekday

\item {} 
\sphinxstyleliteralstrong{\sphinxupquote{do\_periodic}} (\sphinxstyleliteralemphasis{\sphinxupquote{bool}}) \textendash{} a flag whether (if True) or not (if False) time of day     should be expressed in {[}-12, 12)

\end{itemize}

\item[{Returns}] \leavevmode
the statistical data for an activity-parameter (i.e. start time,     end time, and duration) that occurs on the weekend or weekday

\item[{Return type}] \leavevmode
pandas.core.frame.DataFrame

\end{description}\end{quote}

\end{fulllineitems}

\index{get\_weekend\_index() (in module datum)}

\begin{fulllineitems}
\phantomsection\label{\detokenize{datum:datum.get_weekend_index}}\pysiglinewithargsret{\sphinxcode{\sphinxupquote{datum.}}\sphinxbfcode{\sphinxupquote{get\_weekend\_index}}}{\emph{date}, \emph{start}, \emph{end}}{}
This function gets the indices of activity information     of the weekend data.
\begin{quote}\begin{description}
\item[{Parameters}] \leavevmode\begin{itemize}
\item {} 
\sphinxstyleliteralstrong{\sphinxupquote{date}} (\sphinxstyleliteralemphasis{\sphinxupquote{numpy.ndarray of datetime.timedelta}}) \textendash{} the date of the activity information

\item {} 
\sphinxstyleliteralstrong{\sphinxupquote{start}} (\sphinxstyleliteralemphasis{\sphinxupquote{numpy.ndarray}}) \textendash{} the start time of the activity information

\item {} 
\sphinxstyleliteralstrong{\sphinxupquote{end}} (\sphinxstyleliteralemphasis{\sphinxupquote{numpy.ndarray}}) \textendash{} the end time of the activity information

\end{itemize}

\item[{Returns}] \leavevmode
this function gets the indices of activities that occur     during the weekend

\item[{Return type}] \leavevmode
numpy.ndarray of bool

\end{description}\end{quote}

\end{fulllineitems}

\index{get\_weekend\_index\_df() (in module datum)}

\begin{fulllineitems}
\phantomsection\label{\detokenize{datum:datum.get_weekend_index_df}}\pysiglinewithargsret{\sphinxcode{\sphinxupquote{datum.}}\sphinxbfcode{\sphinxupquote{get\_weekend\_index\_df}}}{\emph{df}}{}
This function gets the boolean indices of weekend data from a dataframe.
\begin{quote}\begin{description}
\item[{Parameters}] \leavevmode
\sphinxstyleliteralstrong{\sphinxupquote{df}} (\sphinxstyleliteralemphasis{\sphinxupquote{pandas.core.frame.DataFrame}}) \textendash{} CHAD activity record data

\item[{Returns}] \leavevmode
the boolean indices of weekend data

\item[{Return type}] \leavevmode
numpy.ndarray of bool

\end{description}\end{quote}

\end{fulllineitems}

\index{histogram() (in module datum)}

\begin{fulllineitems}
\phantomsection\label{\detokenize{datum:datum.histogram}}\pysiglinewithargsret{\sphinxcode{\sphinxupquote{datum.}}\sphinxbfcode{\sphinxupquote{histogram}}}{\emph{ax}, \emph{x}, \emph{bins=None}, \emph{color='b'}, \emph{label=''}, \emph{alpha=1.0}}{}
This function plots a histogram of the data where the y axis corresponds     to the relative frequency.
\begin{quote}\begin{description}
\item[{Parameters}] \leavevmode\begin{itemize}
\item {} 
\sphinxstyleliteralstrong{\sphinxupquote{ax}} (\sphinxstyleliteralemphasis{\sphinxupquote{matplotlib.figure.Figure}}) \textendash{} the plotting axis (plt or from axes)

\item {} 
\sphinxstyleliteralstrong{\sphinxupquote{x}} (\sphinxstyleliteralemphasis{\sphinxupquote{numpy.ndarray}}) \textendash{} the data to be plotted

\item {} 
\sphinxstyleliteralstrong{\sphinxupquote{bins}} (\sphinxstyleliteralemphasis{\sphinxupquote{numpy.ndarray}}) \textendash{} the bins for the histogram

\item {} 
\sphinxstyleliteralstrong{\sphinxupquote{color}} (\sphinxstyleliteralemphasis{\sphinxupquote{str}}) \textendash{} the color for the histogram

\item {} 
\sphinxstyleliteralstrong{\sphinxupquote{label}} (\sphinxstyleliteralemphasis{\sphinxupquote{str}}) \textendash{} the label of the data

\item {} 
\sphinxstyleliteralstrong{\sphinxupquote{alpha}} (\sphinxstyleliteralemphasis{\sphinxupquote{float}}) \textendash{} the alpha for plotting

\end{itemize}

\item[{Returns}] \leavevmode


\end{description}\end{quote}

\end{fulllineitems}

\index{merge() (in module datum)}

\begin{fulllineitems}
\phantomsection\label{\detokenize{datum:datum.merge}}\pysiglinewithargsret{\sphinxcode{\sphinxupquote{datum.}}\sphinxbfcode{\sphinxupquote{merge}}}{\emph{df\_full}}{}
For each person in the activity data, the function does the following:
\begin{enumerate}
\item {} 
groups the contiguous daily activity data

\item {} 
merges data that occur over midnight into one event

\end{enumerate}
\begin{quote}\begin{description}
\item[{Parameters}] \leavevmode
\sphinxstyleliteralstrong{\sphinxupquote{df\_full}} (\sphinxstyleliteralemphasis{\sphinxupquote{pandas.core.frame.DataFrame}}) \textendash{} the full set of the activity data

\item[{Returns}] \leavevmode
a data frame that merges activities that occur over midnight

\item[{Return type}] \leavevmode
pandas.core.frame.DataFrame

\end{description}\end{quote}

\end{fulllineitems}

\index{merge\_end\_of\_day() (in module datum)}

\begin{fulllineitems}
\phantomsection\label{\detokenize{datum:datum.merge_end_of_day}}\pysiglinewithargsret{\sphinxcode{\sphinxupquote{datum.}}\sphinxbfcode{\sphinxupquote{merge\_end\_of\_day}}}{\emph{df}}{}
This function takes longitudinal data and merges the data if the data     starts before midnight and ends after midnight.
\begin{quote}\begin{description}
\item[{Parameters}] \leavevmode
\sphinxstyleliteralstrong{\sphinxupquote{df}} (\sphinxstyleliteralemphasis{\sphinxupquote{pandas.core.frame.DataFrame}}) \textendash{} the activity records data

\item[{Returns}] \leavevmode
activity events that start before midnight and end after midnight

\item[{Return type}] \leavevmode
pandas.core.frame.DataFrame

\end{description}\end{quote}

\end{fulllineitems}

\index{periodicity\_CHADID() (in module datum)}

\begin{fulllineitems}
\phantomsection\label{\detokenize{datum:datum.periodicity_CHADID}}\pysiglinewithargsret{\sphinxcode{\sphinxupquote{datum.}}\sphinxbfcode{\sphinxupquote{periodicity\_CHADID}}}{\emph{df}}{}
This function combines entries for sleep with the periodicity assumption for a     given day (CHADID).

If there are two events starting at 0:00 and ending in the morning AND     another event starting in the evening and ending at 0:00 on the SAME DAY,     we combine the two events into one event. We assume that the person goes to sleep     on the same start time and wakes up at the same time (periodicity assumption).
\begin{quote}\begin{description}
\item[{Parameters}] \leavevmode
\sphinxstyleliteralstrong{\sphinxupquote{df}} (\sphinxstyleliteralemphasis{\sphinxupquote{pandas.core.frame.DataFrame}}) \textendash{} sleep events for 1 CHADID

\item[{Returns}] \leavevmode
return sleep data with the periodicity assumption for 1 CHADID

\item[{Return type}] \leavevmode
pandas.core.frame.DataFrame

\end{description}\end{quote}

\end{fulllineitems}

\index{periodicity\_PID() (in module datum)}

\begin{fulllineitems}
\phantomsection\label{\detokenize{datum:datum.periodicity_PID}}\pysiglinewithargsret{\sphinxcode{\sphinxupquote{datum.}}\sphinxbfcode{\sphinxupquote{periodicity\_PID}}}{\emph{df}}{}
Perform the periodicity assumption for a given person by its     person identifier (PID).
\begin{quote}\begin{description}
\item[{Parameters}] \leavevmode
\sphinxstyleliteralstrong{\sphinxupquote{df}} (\sphinxstyleliteralemphasis{\sphinxupquote{pandas.core.frame.DataFrame}}) \textendash{} the sleep data of a person with 1 PID

\item[{Returns}] \leavevmode
sleep data with the periodicity assumption

\item[{Return type}] \leavevmode
list of pandas.core.frame.DataFrame

\end{description}\end{quote}

\end{fulllineitems}

\index{periodicity\_sleep() (in module datum)}

\begin{fulllineitems}
\phantomsection\label{\detokenize{datum:datum.periodicity_sleep}}\pysiglinewithargsret{\sphinxcode{\sphinxupquote{datum.}}\sphinxbfcode{\sphinxupquote{periodicity\_sleep}}}{\emph{data}}{}
Perform the periodicity assumption (i.e., expressing time as {[}-12, 12)) for an     entire dataset of multiple entries.
\begin{quote}\begin{description}
\item[{Parameters}] \leavevmode
\sphinxstyleliteralstrong{\sphinxupquote{data}} (\sphinxstyleliteralemphasis{\sphinxupquote{pandas.core.frame.DataFrame}}) \textendash{} the sleep data over many individuals

\item[{Returns}] \leavevmode
sleep data with the periodicity assumption

\item[{Return type}] \leavevmode
pandas.core.frame.DataFrame

\end{description}\end{quote}

\end{fulllineitems}

\index{save() (in module datum)}

\begin{fulllineitems}
\phantomsection\label{\detokenize{datum:datum.save}}\pysiglinewithargsret{\sphinxcode{\sphinxupquote{datum.}}\sphinxbfcode{\sphinxupquote{save}}}{\emph{fpath}, \emph{record}, \emph{stats\_dt}, \emph{stats\_start}, \emph{stats\_end}}{}
This function saves the following information as a .csv file:
\begin{enumerate}
\item {} 
the statistical moments data for the activity duration (‘stats\_dt.csv’)

\item {} 
the statistical moments data for the activity start time (‘stats\_start.csv’)

\item {} 
the statistical moments data for the activity end time (‘stats\_end.csv’)

\item {} 
the statistical moments data for the activity records (‘record.csv’)

\end{enumerate}
\begin{quote}\begin{description}
\item[{Parameters}] \leavevmode\begin{itemize}
\item {} 
\sphinxstyleliteralstrong{\sphinxupquote{fpath}} (\sphinxstyleliteralemphasis{\sphinxupquote{str}}) \textendash{} the file directory in which to save the data

\item {} 
\sphinxstyleliteralstrong{\sphinxupquote{record}} (\sphinxstyleliteralemphasis{\sphinxupquote{pandas.core.frame.DataFrame}}) \textendash{} the CHAD records for a given     activity

\item {} 
\sphinxstyleliteralstrong{\sphinxupquote{stats\_dt}} (\sphinxstyleliteralemphasis{\sphinxupquote{pandas.core.frame.DataFrame}}) \textendash{} the statistical moments for the     activity duration

\item {} 
\sphinxstyleliteralstrong{\sphinxupquote{stats\_start}} (\sphinxstyleliteralemphasis{\sphinxupquote{pandas.core.frame.DataFrame}}) \textendash{} the statistical moments for     the start time activity duration

\item {} 
\sphinxstyleliteralstrong{\sphinxupquote{stats\_end}} (\sphinxstyleliteralemphasis{\sphinxupquote{pandas.core.frame.DataFrame}}) \textendash{} the statistical moments for     the end time activity duration

\end{itemize}

\item[{Returns}] \leavevmode


\end{description}\end{quote}

\end{fulllineitems}

\index{sequential\_data() (in module datum)}

\begin{fulllineitems}
\phantomsection\label{\detokenize{datum:datum.sequential_data}}\pysiglinewithargsret{\sphinxcode{\sphinxupquote{datum.}}\sphinxbfcode{\sphinxupquote{sequential\_data}}}{\emph{df}}{}
For a given PID, this function groups the data in terms of sets of data     for consecutive days. This function     assumes that all the data given is for a given (generalized) activity.

\begin{sphinxadmonition}{note}{Note:}
In the data, it is not necessarily the case that if there are multiple days of consecutive activity,         that all of them form 1 contiguous period. Ex. It is possible to have entries Jan 1, Jan 2, Jan 3, Feb 10,         Feb 11. This function will group the data into 2 groups when this occurs.
\end{sphinxadmonition}
\begin{quote}\begin{description}
\item[{Parameters}] \leavevmode
\sphinxstyleliteralstrong{\sphinxupquote{df}} (\sphinxstyleliteralemphasis{\sphinxupquote{pandas.core.frame.DataFrame}}) \textendash{} the data of a specific PID for an activity

\item[{Returns}] \leavevmode
a list of dataframes for sequential longitudinal-data

\item[{Return type}] \leavevmode
list of pandas.core.frame.DataFrame

\end{description}\end{quote}

\end{fulllineitems}

\index{sequential\_days() (in module datum)}

\begin{fulllineitems}
\phantomsection\label{\detokenize{datum:datum.sequential_days}}\pysiglinewithargsret{\sphinxcode{\sphinxupquote{datum.}}\sphinxbfcode{\sphinxupquote{sequential\_days}}}{\emph{date}, \emph{start=None}, \emph{end=None}}{}
This creates label indicating sequential days. This is done by writing a sequence     where each group of consecutive dates have a label starting at 0.

\begin{sphinxadmonition}{note}{Note:}
the following sequence of dates {[}0, 0, 1,1, 3, 4, 5, 10{]}, would have the         following sequence {[}0, 0, 0, 0, 1, 1, 1, 2{]}
\end{sphinxadmonition}
\begin{quote}\begin{description}
\item[{Parameters}] \leavevmode\begin{itemize}
\item {} 
\sphinxstyleliteralstrong{\sphinxupquote{date}} (\sphinxstyleliteralemphasis{\sphinxupquote{numpy.ndarray datetime.timedelta}}) \textendash{} the date of the activity data

\item {} 
\sphinxstyleliteralstrong{\sphinxupquote{start}} (\sphinxstyleliteralemphasis{\sphinxupquote{numpy.ndarray}}) \textendash{} the start time of the activity data

\item {} 
\sphinxstyleliteralstrong{\sphinxupquote{end}} (\sphinxstyleliteralemphasis{\sphinxupquote{numpy.ndarray}}) \textendash{} the end time of the activity data

\end{itemize}

\item[{Returns}] \leavevmode
a sequence whose indices indicates sequential dates for an activity

\item[{Return type}] \leavevmode
numpy.ndarray

\end{description}\end{quote}

\end{fulllineitems}



\subsection{demographics notebook}
\label{\detokenize{demographics::doc}}\label{\detokenize{demographics:demographics-notebook}}
\fvset{hllines={, ,}}%
\begin{sphinxVerbatim}[commandchars=\\\{\}]
\PYG{c+c1}{\PYGZsh{} The United States Environmental Protection Agency through its Office of}
\PYG{c+c1}{\PYGZsh{} Research and Development has developed this software. The code is made}
\PYG{c+c1}{\PYGZsh{} publicly available to better communicate the research. All input data}
\PYG{c+c1}{\PYGZsh{} used fora given application should be reviewed by the researcher so}
\PYG{c+c1}{\PYGZsh{} that the model results are based on appropriate data for any given}
\PYG{c+c1}{\PYGZsh{} application. This model is under continued development. The model and}
\PYG{c+c1}{\PYGZsh{} data included herein do not represent and should not be construed to}
\PYG{c+c1}{\PYGZsh{} represent any Agency determination or policy.}
\PYG{c+c1}{\PYGZsh{}}
\PYG{c+c1}{\PYGZsh{} This file was written by Dr. Namdi Brandon}
\PYG{c+c1}{\PYGZsh{} ORCID: 0000\PYGZhy{}0001\PYGZhy{}7050\PYGZhy{}1538}
\PYG{c+c1}{\PYGZsh{} March 22, 2018}
\end{sphinxVerbatim}

This file does the following
\begin{enumerate}
\item {} 
Goes through the Consolidated Human Activity Database (CHAD) data and
seprates CHAD into datasets of different demographic groups

\item {} 
Or loads saved datasets representing different demographic groups for
CHAD

\item {} 
Saves data for each demographic group:
\begin{itemize}
\item {} 
Saves the demographic data into the ‘data\_large’ directory

\item {} 
Saves the demographic in a compressed form in the ‘data’ directory
as zip files

\end{itemize}

\item {} 
For a given demographic group and a given collection of activities
\begin{itemize}
\item {} 
prints the amount of individuals found doing each activity given
by a unique CHAD code

\item {} 
plots the histogram and/or CDF of distributions of start time, end
time, and duration for each specific activity given by a CHAD code

\item {} 
Saves the plots

\end{itemize}

\end{enumerate}

import

\fvset{hllines={, ,}}%
\begin{sphinxVerbatim}[commandchars=\\\{\}]
\PYG{c+c1}{\PYGZsh{}}
\PYG{c+c1}{\PYGZsh{} import}
\PYG{c+c1}{\PYGZsh{}}
\PYG{k+kn}{import} \PYG{n+nn}{sys}
\PYG{n}{sys}\PYG{o}{.}\PYG{n}{path}\PYG{o}{.}\PYG{n}{append}\PYG{p}{(}\PYG{l+s+s1}{\PYGZsq{}}\PYG{l+s+s1}{..}\PYG{l+s+se}{\PYGZbs{}\PYGZbs{}}\PYG{l+s+s1}{source}\PYG{l+s+s1}{\PYGZsq{}}\PYG{p}{)}
\PYG{n}{sys}\PYG{o}{.}\PYG{n}{path}\PYG{o}{.}\PYG{n}{append}\PYG{p}{(}\PYG{l+s+s1}{\PYGZsq{}}\PYG{l+s+s1}{..}\PYG{l+s+se}{\PYGZbs{}\PYGZbs{}}\PYG{l+s+s1}{run\PYGZus{}chad}\PYG{l+s+s1}{\PYGZsq{}}\PYG{p}{)}
\PYG{k+kn}{import} \PYG{n+nn}{os}

\PYG{c+c1}{\PYGZsh{} plotting capabilities}
\PYG{k+kn}{import} \PYG{n+nn}{matplotlib}\PYG{n+nn}{.}\PYG{n+nn}{pylab} \PYG{k}{as} \PYG{n+nn}{plt}

\PYG{c+c1}{\PYGZsh{} math capability}
\PYG{k+kn}{import} \PYG{n+nn}{numpy} \PYG{k}{as} \PYG{n+nn}{np}

\PYG{c+c1}{\PYGZsh{} ABMHAP modules}
\PYG{k+kn}{import} \PYG{n+nn}{my\PYGZus{}globals} \PYG{k}{as} \PYG{n+nn}{mg}
\PYG{k+kn}{import} \PYG{n+nn}{demography} \PYG{k}{as} \PYG{n+nn}{dmg}

\PYG{k+kn}{import} \PYG{n+nn}{chad}\PYG{o}{,} \PYG{n+nn}{chad\PYGZus{}code}
\end{sphinxVerbatim}

functions

\fvset{hllines={, ,}}%
\begin{sphinxVerbatim}[commandchars=\\\{\}]
\PYG{k}{def} \PYG{n+nf}{plot\PYGZus{}cdfs}\PYG{p}{(}\PYG{n}{df}\PYG{p}{,} \PYG{n}{codes}\PYG{p}{,} \PYG{n}{N}\PYG{o}{=}\PYG{l+m+mi}{1000}\PYG{p}{,} \PYG{n}{linewidth}\PYG{o}{=}\PYG{l+m+mi}{1}\PYG{p}{,} \PYG{n}{do\PYGZus{}save}\PYG{o}{=}\PYG{k+kc}{False}\PYG{p}{,} \PYG{n}{fpath}\PYG{o}{=}\PYG{l+s+s1}{\PYGZsq{}}\PYG{l+s+s1}{\PYGZsq{}}\PYG{p}{)}\PYG{p}{:}

    \PYG{l+s+sd}{\PYGZdq{}\PYGZdq{}\PYGZdq{}}
\PYG{l+s+sd}{    This function plots the distribution of activity distrbution of \PYGZbs{}}
\PYG{l+s+sd}{    start time, end time, and duration as cumulative distribution \PYGZbs{}}
\PYG{l+s+sd}{    functions (CDFs) from the CHAD data of the given activity.}

\PYG{l+s+sd}{    :param pandas.core.frame.DataFrame df:}
\PYG{l+s+sd}{    :param codes: the CHAD activity codes}
\PYG{l+s+sd}{    :type codes: list of list of int}
\PYG{l+s+sd}{    :param int N: the number of points sampled within the empirical CDF}
\PYG{l+s+sd}{    :param int linewidth: the width of the plotted lines}
\PYG{l+s+sd}{    :param bool do\PYGZus{}save: a flag indicating whether (if True) to save the \PYGZbs{}}
\PYG{l+s+sd}{    figures or not(if False)}
\PYG{l+s+sd}{    :param str fpath: the file directory to save the files in}

\PYG{l+s+sd}{    :return:}
\PYG{l+s+sd}{    \PYGZdq{}\PYGZdq{}\PYGZdq{}}

    \PYG{c+c1}{\PYGZsh{} codes: chad\PYGZus{}codes for each activity}

    \PYG{n}{figs}\PYG{p}{,} \PYG{n}{fnames} \PYG{o}{=} \PYG{p}{[}\PYG{p}{]}\PYG{p}{,} \PYG{p}{[}\PYG{p}{]}

    \PYG{c+c1}{\PYGZsh{} for each activity category within the CHAD codes}
    \PYG{k}{for} \PYG{n}{act} \PYG{o+ow}{in} \PYG{n}{codes}\PYG{p}{:}

        \PYG{c+c1}{\PYGZsh{} get the data w}
        \PYG{n}{temp} \PYG{o}{=} \PYG{n}{df}\PYG{p}{[}\PYG{n}{df}\PYG{o}{.}\PYG{n}{act} \PYG{o}{==} \PYG{n}{act}\PYG{p}{]}
        \PYG{n}{gb} \PYG{o}{=} \PYG{n}{temp}\PYG{o}{.}\PYG{n}{groupby}\PYG{p}{(}\PYG{l+s+s1}{\PYGZsq{}}\PYG{l+s+s1}{PID}\PYG{l+s+s1}{\PYGZsq{}}\PYG{p}{)}

        \PYG{c+c1}{\PYGZsh{} get the mean duration data}
        \PYG{n}{y\PYGZus{}dt} \PYG{o}{=} \PYG{n}{np}\PYG{o}{.}\PYG{n}{array}\PYG{p}{(} \PYG{p}{[} \PYG{n}{gb}\PYG{o}{.}\PYG{n}{get\PYGZus{}group}\PYG{p}{(}\PYG{n}{p}\PYG{p}{)}\PYG{o}{.}\PYG{n}{dt}\PYG{o}{.}\PYG{n}{mean}\PYG{p}{(}\PYG{p}{)} \PYG{k}{for} \PYG{n}{p} \PYG{o+ow}{in} \PYG{n}{temp}\PYG{o}{.}\PYG{n}{PID}\PYG{o}{.}\PYG{n}{unique}\PYG{p}{(}\PYG{p}{)} \PYG{p}{]} \PYG{p}{)}

        \PYG{c+c1}{\PYGZsh{} get the mean start time data}
        \PYG{n}{y\PYGZus{}start} \PYG{o}{=} \PYG{n}{np}\PYG{o}{.}\PYG{n}{array}\PYG{p}{(} \PYG{p}{[} \PYG{n}{gb}\PYG{o}{.}\PYG{n}{get\PYGZus{}group}\PYG{p}{(}\PYG{n}{p}\PYG{p}{)}\PYG{o}{.}\PYG{n}{start}\PYG{o}{.}\PYG{n}{mean}\PYG{p}{(}\PYG{p}{)} \PYG{k}{for} \PYG{n}{p} \PYG{o+ow}{in} \PYG{n}{temp}\PYG{o}{.}\PYG{n}{PID}\PYG{o}{.}\PYG{n}{unique}\PYG{p}{(}\PYG{p}{)} \PYG{p}{]} \PYG{p}{)}

        \PYG{c+c1}{\PYGZsh{} get the mean end time data}
        \PYG{n}{y\PYGZus{}end} \PYG{o}{=} \PYG{n}{np}\PYG{o}{.}\PYG{n}{array}\PYG{p}{(} \PYG{p}{[} \PYG{n}{gb}\PYG{o}{.}\PYG{n}{get\PYGZus{}group}\PYG{p}{(}\PYG{n}{p}\PYG{p}{)}\PYG{o}{.}\PYG{n}{end}\PYG{o}{.}\PYG{n}{mean}\PYG{p}{(}\PYG{p}{)} \PYG{k}{for} \PYG{n}{p} \PYG{o+ow}{in} \PYG{n}{temp}\PYG{o}{.}\PYG{n}{PID}\PYG{o}{.}\PYG{n}{unique}\PYG{p}{(}\PYG{p}{)} \PYG{p}{]} \PYG{p}{)}

        \PYG{k}{if} \PYG{n+nb}{len}\PYG{p}{(}\PYG{n}{y\PYGZus{}dt}\PYG{p}{)} \PYG{o}{!=} \PYG{l+m+mi}{0}\PYG{p}{:}

            \PYG{c+c1}{\PYGZsh{} create subplots}
            \PYG{n}{fig}\PYG{p}{,} \PYG{n}{axes} \PYG{o}{=} \PYG{n}{plt}\PYG{o}{.}\PYG{n}{subplots}\PYG{p}{(}\PYG{l+m+mi}{2}\PYG{p}{,}\PYG{l+m+mi}{2}\PYG{p}{)}

            \PYG{c+c1}{\PYGZsh{} create title}
            \PYG{n}{fig}\PYG{o}{.}\PYG{n}{suptitle}\PYG{p}{(}\PYG{n}{chad\PYGZus{}code}\PYG{o}{.}\PYG{n}{INT\PYGZus{}2\PYGZus{}STR}\PYG{p}{[}\PYG{n}{act}\PYG{p}{]}\PYG{p}{)}

            \PYG{c+c1}{\PYGZsh{} plot the start time}
            \PYG{n}{ax} \PYG{o}{=} \PYG{n}{axes}\PYG{p}{[}\PYG{l+m+mi}{0}\PYG{p}{,} \PYG{l+m+mi}{0}\PYG{p}{]}
            \PYG{n}{x}\PYG{p}{,} \PYG{n}{y} \PYG{o}{=} \PYG{n}{mg}\PYG{o}{.}\PYG{n}{get\PYGZus{}ecdf}\PYG{p}{(}\PYG{n}{y\PYGZus{}start}\PYG{p}{,} \PYG{n}{N}\PYG{p}{)}
            \PYG{n}{ax}\PYG{o}{.}\PYG{n}{plot}\PYG{p}{(}\PYG{n}{x}\PYG{p}{,} \PYG{n}{y}\PYG{p}{,} \PYG{n}{color}\PYG{o}{=}\PYG{l+s+s1}{\PYGZsq{}}\PYG{l+s+s1}{blue}\PYG{l+s+s1}{\PYGZsq{}}\PYG{p}{,} \PYG{n}{label}\PYG{o}{=}\PYG{l+s+s1}{\PYGZsq{}}\PYG{l+s+s1}{start}\PYG{l+s+s1}{\PYGZsq{}}\PYG{p}{,} \PYG{n}{lw}\PYG{o}{=}\PYG{n}{linewidth}\PYG{p}{)}

            \PYG{c+c1}{\PYGZsh{} plot the end time}
            \PYG{n}{ax} \PYG{o}{=} \PYG{n}{axes}\PYG{p}{[}\PYG{l+m+mi}{0}\PYG{p}{,} \PYG{l+m+mi}{1}\PYG{p}{]}
            \PYG{n}{x}\PYG{p}{,} \PYG{n}{y} \PYG{o}{=} \PYG{n}{mg}\PYG{o}{.}\PYG{n}{get\PYGZus{}ecdf}\PYG{p}{(}\PYG{n}{y\PYGZus{}end}\PYG{p}{,} \PYG{n}{N}\PYG{p}{)}
            \PYG{n}{ax}\PYG{o}{.}\PYG{n}{plot}\PYG{p}{(}\PYG{n}{x}\PYG{p}{,} \PYG{n}{y}\PYG{p}{,} \PYG{n}{color}\PYG{o}{=}\PYG{l+s+s1}{\PYGZsq{}}\PYG{l+s+s1}{purple}\PYG{l+s+s1}{\PYGZsq{}}\PYG{p}{,} \PYG{n}{label}\PYG{o}{=}\PYG{l+s+s1}{\PYGZsq{}}\PYG{l+s+s1}{end}\PYG{l+s+s1}{\PYGZsq{}}\PYG{p}{,} \PYG{n}{lw}\PYG{o}{=}\PYG{n}{linewidth}\PYG{p}{)}

            \PYG{c+c1}{\PYGZsh{} plot the duration}
            \PYG{n}{ax} \PYG{o}{=} \PYG{n}{axes}\PYG{p}{[}\PYG{l+m+mi}{1}\PYG{p}{,} \PYG{l+m+mi}{0}\PYG{p}{]}
            \PYG{n}{x}\PYG{p}{,} \PYG{n}{y} \PYG{o}{=} \PYG{n}{mg}\PYG{o}{.}\PYG{n}{get\PYGZus{}ecdf}\PYG{p}{(}\PYG{n}{y\PYGZus{}dt}\PYG{p}{,} \PYG{n}{N}\PYG{p}{)}
            \PYG{n}{ax}\PYG{o}{.}\PYG{n}{plot}\PYG{p}{(}\PYG{n}{x}\PYG{p}{,} \PYG{n}{y}\PYG{p}{,} \PYG{n}{color}\PYG{o}{=}\PYG{l+s+s1}{\PYGZsq{}}\PYG{l+s+s1}{red}\PYG{l+s+s1}{\PYGZsq{}}\PYG{p}{,} \PYG{n}{label}\PYG{o}{=}\PYG{l+s+s1}{\PYGZsq{}}\PYG{l+s+s1}{duration}\PYG{l+s+s1}{\PYGZsq{}}\PYG{p}{,} \PYG{n}{lw}\PYG{o}{=}\PYG{n}{linewidth}\PYG{p}{)}

            \PYG{c+c1}{\PYGZsh{} plot axis label and legend}
            \PYG{k}{for} \PYG{n}{ax} \PYG{o+ow}{in} \PYG{n}{axes}\PYG{o}{.}\PYG{n}{flatten}\PYG{p}{(}\PYG{p}{)}\PYG{p}{:}
                \PYG{n}{ax}\PYG{o}{.}\PYG{n}{set\PYGZus{}xlabel}\PYG{p}{(}\PYG{l+s+s1}{\PYGZsq{}}\PYG{l+s+s1}{Hours}\PYG{l+s+s1}{\PYGZsq{}}\PYG{p}{)}
                \PYG{n}{ax}\PYG{o}{.}\PYG{n}{legend}\PYG{p}{(}\PYG{n}{loc}\PYG{o}{=}\PYG{l+s+s1}{\PYGZsq{}}\PYG{l+s+s1}{best}\PYG{l+s+s1}{\PYGZsq{}}\PYG{p}{)}

            \PYG{c+c1}{\PYGZsh{}}
            \PYG{c+c1}{\PYGZsh{} save}
            \PYG{c+c1}{\PYGZsh{}}
            \PYG{k}{if} \PYG{n}{do\PYGZus{}save}\PYG{p}{:}
                \PYG{c+c1}{\PYGZsh{} figure name}
                \PYG{n}{fname} \PYG{o}{=} \PYG{n}{fpath} \PYG{o}{+} \PYG{n}{chad\PYGZus{}code}\PYG{o}{.}\PYG{n}{INT\PYGZus{}2\PYGZus{}SAVE\PYGZus{}FIG\PYGZus{}FNAME}\PYG{p}{[}\PYG{n}{act}\PYG{p}{]}

                \PYG{c+c1}{\PYGZsh{} split the file name into 2 parts from the back}
                \PYG{n}{x} \PYG{o}{=} \PYG{n}{fname}\PYG{o}{.}\PYG{n}{rsplit}\PYG{p}{(}\PYG{l+s+s1}{\PYGZsq{}}\PYG{l+s+se}{\PYGZbs{}\PYGZbs{}}\PYG{l+s+s1}{\PYGZsq{}}\PYG{p}{,} \PYG{n}{maxsplit}\PYG{o}{=}\PYG{l+m+mi}{1}\PYG{p}{)}

                \PYG{c+c1}{\PYGZsh{} create the filename}
                \PYG{n}{fname} \PYG{o}{=} \PYG{n}{x}\PYG{p}{[}\PYG{l+m+mi}{0}\PYG{p}{]} \PYG{o}{+} \PYG{l+s+s1}{\PYGZsq{}}\PYG{l+s+se}{\PYGZbs{}\PYGZbs{}}\PYG{l+s+s1}{cdf}\PYG{l+s+se}{\PYGZbs{}\PYGZbs{}}\PYG{l+s+s1}{\PYGZsq{}} \PYG{o}{+} \PYG{n}{x}\PYG{p}{[}\PYG{l+m+mi}{1}\PYG{p}{]}

                \PYG{n+nb}{print}\PYG{p}{(}\PYG{n}{fname}\PYG{p}{)}

                \PYG{c+c1}{\PYGZsh{} add list of figures and finle names}
                \PYG{n}{figs}\PYG{o}{.}\PYG{n}{append}\PYG{p}{(}\PYG{n}{fig}\PYG{p}{)}
                \PYG{n}{fnames}\PYG{o}{.}\PYG{n}{append}\PYG{p}{(}\PYG{n}{fname}\PYG{p}{)}

    \PYG{c+c1}{\PYGZsh{} save the figures}
    \PYG{k}{if} \PYG{n}{do\PYGZus{}save}\PYG{p}{:}
        \PYG{k}{for} \PYG{n}{fig}\PYG{p}{,} \PYG{n}{fname} \PYG{o+ow}{in} \PYG{n+nb}{zip}\PYG{p}{(}\PYG{n}{figs}\PYG{p}{,} \PYG{n}{fnames}\PYG{p}{)}\PYG{p}{:}
            \PYG{n}{os}\PYG{o}{.}\PYG{n}{makedirs}\PYG{p}{(}\PYG{n}{os}\PYG{o}{.}\PYG{n}{path}\PYG{o}{.}\PYG{n}{dirname}\PYG{p}{(}\PYG{n}{fname}\PYG{p}{)}\PYG{p}{,} \PYG{n}{exist\PYGZus{}ok}\PYG{o}{=}\PYG{k+kc}{True}\PYG{p}{)}
            \PYG{n}{fig}\PYG{o}{.}\PYG{n}{savefig}\PYG{p}{(}\PYG{n}{fname}\PYG{p}{,} \PYG{n}{dpi}\PYG{o}{=}\PYG{l+m+mi}{800}\PYG{p}{)}
            \PYG{n}{plt}\PYG{o}{.}\PYG{n}{close}\PYG{p}{(}\PYG{n}{fig}\PYG{p}{)}


    \PYG{k}{return}

\PYG{k}{def} \PYG{n+nf}{plot\PYGZus{}histograms}\PYG{p}{(}\PYG{n}{df}\PYG{p}{,} \PYG{n}{codes}\PYG{p}{,} \PYG{n}{num\PYGZus{}bins}\PYG{o}{=}\PYG{l+m+mi}{12}\PYG{p}{,} \PYG{n}{fpath}\PYG{o}{=}\PYG{l+s+s1}{\PYGZsq{}}\PYG{l+s+s1}{\PYGZsq{}}\PYG{p}{,} \PYG{n}{do\PYGZus{}save}\PYG{o}{=}\PYG{k+kc}{False}\PYG{p}{)}\PYG{p}{:}

    \PYG{l+s+sd}{\PYGZdq{}\PYGZdq{}\PYGZdq{}}
\PYG{l+s+sd}{    This function plots the distribution of activity distrbution of \PYGZbs{}}
\PYG{l+s+sd}{    start time, end time, and duration as histograms from the CHAD \PYGZbs{}}
\PYG{l+s+sd}{    data of the given activity.}

\PYG{l+s+sd}{    :param pandas.core.frame.DataFrame df:}
\PYG{l+s+sd}{    :param codes: the CHAD activity codes}
\PYG{l+s+sd}{    :type codes: list of list of int}
\PYG{l+s+sd}{    :param int num\PYGZus{}bins: the number of bins within the histogram}
\PYG{l+s+sd}{    :param bool do\PYGZus{}save: a flag indicating whether (if True) to save the \PYGZbs{}}
\PYG{l+s+sd}{    figures or not(if False)}
\PYG{l+s+sd}{    :param str fpath: the file directory to save the files in}

\PYG{l+s+sd}{    :return:}
\PYG{l+s+sd}{    \PYGZdq{}\PYGZdq{}\PYGZdq{}}

    \PYG{n}{figs}\PYG{p}{,} \PYG{n}{fnames} \PYG{o}{=} \PYG{p}{[}\PYG{p}{]}\PYG{p}{,} \PYG{p}{[}\PYG{p}{]}

    \PYG{c+c1}{\PYGZsh{} for each activitiy within the CHAD activity codes}
    \PYG{k}{for} \PYG{n}{act} \PYG{o+ow}{in} \PYG{n}{codes}\PYG{p}{:}

        \PYG{c+c1}{\PYGZsh{} get the data w}
        \PYG{n}{temp} \PYG{o}{=} \PYG{n}{df}\PYG{p}{[}\PYG{n}{df}\PYG{o}{.}\PYG{n}{act} \PYG{o}{==} \PYG{n}{act}\PYG{p}{]}
        \PYG{n}{gb} \PYG{o}{=} \PYG{n}{temp}\PYG{o}{.}\PYG{n}{groupby}\PYG{p}{(}\PYG{l+s+s1}{\PYGZsq{}}\PYG{l+s+s1}{PID}\PYG{l+s+s1}{\PYGZsq{}}\PYG{p}{)}


        \PYG{c+c1}{\PYGZsh{} get the mean duration data}
        \PYG{n}{y\PYGZus{}dt} \PYG{o}{=} \PYG{n}{np}\PYG{o}{.}\PYG{n}{array}\PYG{p}{(} \PYG{p}{[} \PYG{n}{gb}\PYG{o}{.}\PYG{n}{get\PYGZus{}group}\PYG{p}{(}\PYG{n}{p}\PYG{p}{)}\PYG{o}{.}\PYG{n}{dt}\PYG{o}{.}\PYG{n}{mean}\PYG{p}{(}\PYG{p}{)} \PYG{k}{for} \PYG{n}{p} \PYG{o+ow}{in} \PYG{n}{temp}\PYG{o}{.}\PYG{n}{PID}\PYG{o}{.}\PYG{n}{unique}\PYG{p}{(}\PYG{p}{)} \PYG{p}{]} \PYG{p}{)}

        \PYG{c+c1}{\PYGZsh{} get the mean start time data}
        \PYG{n}{y\PYGZus{}start} \PYG{o}{=} \PYG{n}{np}\PYG{o}{.}\PYG{n}{array}\PYG{p}{(} \PYG{p}{[} \PYG{n}{gb}\PYG{o}{.}\PYG{n}{get\PYGZus{}group}\PYG{p}{(}\PYG{n}{p}\PYG{p}{)}\PYG{o}{.}\PYG{n}{start}\PYG{o}{.}\PYG{n}{mean}\PYG{p}{(}\PYG{p}{)} \PYG{k}{for} \PYG{n}{p} \PYG{o+ow}{in} \PYG{n}{temp}\PYG{o}{.}\PYG{n}{PID}\PYG{o}{.}\PYG{n}{unique}\PYG{p}{(}\PYG{p}{)} \PYG{p}{]} \PYG{p}{)}

        \PYG{c+c1}{\PYGZsh{} get the mean end time data}
        \PYG{n}{y\PYGZus{}end} \PYG{o}{=} \PYG{n}{np}\PYG{o}{.}\PYG{n}{array}\PYG{p}{(} \PYG{p}{[} \PYG{n}{gb}\PYG{o}{.}\PYG{n}{get\PYGZus{}group}\PYG{p}{(}\PYG{n}{p}\PYG{p}{)}\PYG{o}{.}\PYG{n}{end}\PYG{o}{.}\PYG{n}{mean}\PYG{p}{(}\PYG{p}{)} \PYG{k}{for} \PYG{n}{p} \PYG{o+ow}{in} \PYG{n}{temp}\PYG{o}{.}\PYG{n}{PID}\PYG{o}{.}\PYG{n}{unique}\PYG{p}{(}\PYG{p}{)} \PYG{p}{]} \PYG{p}{)}

        \PYG{k}{if} \PYG{n+nb}{len}\PYG{p}{(}\PYG{n}{y\PYGZus{}dt}\PYG{p}{)} \PYG{o}{!=} \PYG{l+m+mi}{0}\PYG{p}{:}
            \PYG{c+c1}{\PYGZsh{} create subplots}
            \PYG{n}{fig}\PYG{p}{,} \PYG{n}{axes} \PYG{o}{=} \PYG{n}{plt}\PYG{o}{.}\PYG{n}{subplots}\PYG{p}{(}\PYG{l+m+mi}{2}\PYG{p}{,}\PYG{l+m+mi}{2}\PYG{p}{)}

            \PYG{c+c1}{\PYGZsh{} create title}
            \PYG{n}{fig}\PYG{o}{.}\PYG{n}{suptitle}\PYG{p}{(}\PYG{n}{chad\PYGZus{}code}\PYG{o}{.}\PYG{n}{INT\PYGZus{}2\PYGZus{}STR}\PYG{p}{[}\PYG{n}{act}\PYG{p}{]}\PYG{p}{)}

            \PYG{c+c1}{\PYGZsh{} plot the start time}
            \PYG{n}{ax} \PYG{o}{=} \PYG{n}{axes}\PYG{p}{[}\PYG{l+m+mi}{0}\PYG{p}{,} \PYG{l+m+mi}{0}\PYG{p}{]}
            \PYG{n}{ax}\PYG{o}{.}\PYG{n}{hist}\PYG{p}{(}\PYG{n}{y\PYGZus{}start}\PYG{p}{,} \PYG{n}{bins}\PYG{o}{=}\PYG{n}{num\PYGZus{}bins}\PYG{p}{,} \PYG{n}{color}\PYG{o}{=}\PYG{l+s+s1}{\PYGZsq{}}\PYG{l+s+s1}{blue}\PYG{l+s+s1}{\PYGZsq{}}\PYG{p}{,} \PYG{n}{label}\PYG{o}{=}\PYG{l+s+s1}{\PYGZsq{}}\PYG{l+s+s1}{start}\PYG{l+s+s1}{\PYGZsq{}}\PYG{p}{)}

            \PYG{c+c1}{\PYGZsh{} plot the end time}
            \PYG{n}{ax} \PYG{o}{=} \PYG{n}{axes}\PYG{p}{[}\PYG{l+m+mi}{0}\PYG{p}{,} \PYG{l+m+mi}{1}\PYG{p}{]}
            \PYG{n}{ax}\PYG{o}{.}\PYG{n}{hist}\PYG{p}{(}\PYG{n}{y\PYGZus{}end}\PYG{p}{,} \PYG{n}{bins}\PYG{o}{=}\PYG{n}{num\PYGZus{}bins}\PYG{p}{,} \PYG{n}{color}\PYG{o}{=}\PYG{l+s+s1}{\PYGZsq{}}\PYG{l+s+s1}{purple}\PYG{l+s+s1}{\PYGZsq{}}\PYG{p}{,} \PYG{n}{label}\PYG{o}{=}\PYG{l+s+s1}{\PYGZsq{}}\PYG{l+s+s1}{end}\PYG{l+s+s1}{\PYGZsq{}}\PYG{p}{)}

            \PYG{c+c1}{\PYGZsh{} plot the duration}
            \PYG{n}{ax} \PYG{o}{=} \PYG{n}{axes}\PYG{p}{[}\PYG{l+m+mi}{1}\PYG{p}{,} \PYG{l+m+mi}{0}\PYG{p}{]}
            \PYG{n}{ax}\PYG{o}{.}\PYG{n}{hist}\PYG{p}{(}\PYG{n}{y\PYGZus{}dt}\PYG{p}{,} \PYG{n}{bins}\PYG{o}{=}\PYG{n}{num\PYGZus{}bins}\PYG{p}{,} \PYG{n}{color}\PYG{o}{=}\PYG{l+s+s1}{\PYGZsq{}}\PYG{l+s+s1}{red}\PYG{l+s+s1}{\PYGZsq{}}\PYG{p}{,} \PYG{n}{label}\PYG{o}{=}\PYG{l+s+s1}{\PYGZsq{}}\PYG{l+s+s1}{duration}\PYG{l+s+s1}{\PYGZsq{}}\PYG{p}{)}

            \PYG{c+c1}{\PYGZsh{} plot axis label and legend}
            \PYG{k}{for} \PYG{n}{ax} \PYG{o+ow}{in} \PYG{n}{axes}\PYG{o}{.}\PYG{n}{flatten}\PYG{p}{(}\PYG{p}{)}\PYG{p}{:}
                \PYG{n}{ax}\PYG{o}{.}\PYG{n}{set\PYGZus{}xlabel}\PYG{p}{(}\PYG{l+s+s1}{\PYGZsq{}}\PYG{l+s+s1}{Hours}\PYG{l+s+s1}{\PYGZsq{}}\PYG{p}{)}
                \PYG{n}{ax}\PYG{o}{.}\PYG{n}{legend}\PYG{p}{(}\PYG{n}{loc}\PYG{o}{=}\PYG{l+s+s1}{\PYGZsq{}}\PYG{l+s+s1}{best}\PYG{l+s+s1}{\PYGZsq{}}\PYG{p}{)}

            \PYG{c+c1}{\PYGZsh{}}
            \PYG{c+c1}{\PYGZsh{} save}
            \PYG{c+c1}{\PYGZsh{}}
            \PYG{k}{if} \PYG{n}{do\PYGZus{}save}\PYG{p}{:}

                \PYG{c+c1}{\PYGZsh{} figure name}
                \PYG{n}{fname} \PYG{o}{=} \PYG{n}{fpath} \PYG{o}{+} \PYG{n}{chad\PYGZus{}code}\PYG{o}{.}\PYG{n}{INT\PYGZus{}2\PYGZus{}SAVE\PYGZus{}FIG\PYGZus{}FNAME}\PYG{p}{[}\PYG{n}{act}\PYG{p}{]}

                \PYG{c+c1}{\PYGZsh{} split the file name into 2 parts from the back}
                \PYG{n}{x} \PYG{o}{=} \PYG{n}{fname}\PYG{o}{.}\PYG{n}{rsplit}\PYG{p}{(}\PYG{l+s+s1}{\PYGZsq{}}\PYG{l+s+se}{\PYGZbs{}\PYGZbs{}}\PYG{l+s+s1}{\PYGZsq{}}\PYG{p}{,} \PYG{n}{maxsplit}\PYG{o}{=}\PYG{l+m+mi}{1}\PYG{p}{)}

                \PYG{n}{fname} \PYG{o}{=} \PYG{n}{x}\PYG{p}{[}\PYG{l+m+mi}{0}\PYG{p}{]} \PYG{o}{+} \PYG{l+s+s1}{\PYGZsq{}}\PYG{l+s+se}{\PYGZbs{}\PYGZbs{}}\PYG{l+s+s1}{histo}\PYG{l+s+se}{\PYGZbs{}\PYGZbs{}}\PYG{l+s+s1}{\PYGZsq{}} \PYG{o}{+} \PYG{n}{x}\PYG{p}{[}\PYG{l+m+mi}{1}\PYG{p}{]}

                \PYG{n+nb}{print}\PYG{p}{(}\PYG{n}{fname}\PYG{p}{)}
                \PYG{c+c1}{\PYGZsh{} add list of figures and finle names}
                \PYG{n}{figs}\PYG{o}{.}\PYG{n}{append}\PYG{p}{(}\PYG{n}{fig}\PYG{p}{)}
                \PYG{n}{fnames}\PYG{o}{.}\PYG{n}{append}\PYG{p}{(}\PYG{n}{fname}\PYG{p}{)}

    \PYG{c+c1}{\PYGZsh{} save the figures}
    \PYG{k}{if} \PYG{n}{do\PYGZus{}save}\PYG{p}{:}
        \PYG{k}{for} \PYG{n}{fig}\PYG{p}{,} \PYG{n}{fname} \PYG{o+ow}{in} \PYG{n+nb}{zip}\PYG{p}{(}\PYG{n}{figs}\PYG{p}{,} \PYG{n}{fnames}\PYG{p}{)}\PYG{p}{:}

            \PYG{n}{os}\PYG{o}{.}\PYG{n}{makedirs}\PYG{p}{(}\PYG{n}{os}\PYG{o}{.}\PYG{n}{path}\PYG{o}{.}\PYG{n}{dirname}\PYG{p}{(}\PYG{n}{fname}\PYG{p}{)}\PYG{p}{,} \PYG{n}{exist\PYGZus{}ok}\PYG{o}{=}\PYG{k+kc}{True}\PYG{p}{)}
            \PYG{n}{fig}\PYG{o}{.}\PYG{n}{savefig}\PYG{p}{(}\PYG{n}{fname}\PYG{p}{,} \PYG{n}{dpi}\PYG{o}{=}\PYG{l+m+mi}{800}\PYG{p}{)}
            \PYG{n}{plt}\PYG{o}{.}\PYG{n}{close}\PYG{p}{(}\PYG{n}{fig}\PYG{p}{)}
    \PYG{k}{return}

\PYG{k}{def} \PYG{n+nf}{save}\PYG{p}{(}\PYG{n}{x}\PYG{p}{,} \PYG{n}{fname}\PYG{p}{)}\PYG{p}{:}

    \PYG{l+s+sd}{\PYGZdq{}\PYGZdq{}\PYGZdq{}}
\PYG{l+s+sd}{    This function saves the data for a given demographic.}

\PYG{l+s+sd}{    :param chad.CHAD\PYGZus{}RAW x: the data to be pickled}
\PYG{l+s+sd}{    :param str fname: the name of the file}
\PYG{l+s+sd}{    \PYGZdq{}\PYGZdq{}\PYGZdq{}}

    \PYG{c+c1}{\PYGZsh{} first, close the zip file. This is necessary to avoid an pickling error}
    \PYG{n}{x}\PYG{o}{.}\PYG{n}{z}\PYG{o}{.}\PYG{n}{close}\PYG{p}{(}\PYG{p}{)}

    \PYG{c+c1}{\PYGZsh{} pickle the data}
    \PYG{n}{mg}\PYG{o}{.}\PYG{n}{save}\PYG{p}{(}\PYG{n}{x}\PYG{p}{,} \PYG{n}{fname}\PYG{p}{)}

    \PYG{k}{return}
\end{sphinxVerbatim}

Load data

\fvset{hllines={, ,}}%
\begin{sphinxVerbatim}[commandchars=\\\{\}]
\PYG{c+c1}{\PYGZsh{} set flags}

\PYG{c+c1}{\PYGZsh{} flag to load pre\PYGZhy{}saved CHAD data(if True) or (if False) to process the CHAD data, \PYGZbs{}}
\PYG{c+c1}{\PYGZsh{} which takes substantially more time}
\PYG{n}{do\PYGZus{}load} \PYG{o}{=} \PYG{k+kc}{True}

\PYG{c+c1}{\PYGZsh{} flag to show messages}
\PYG{n}{do\PYGZus{}print} \PYG{o}{=} \PYG{k+kc}{True}
\end{sphinxVerbatim}

\fvset{hllines={, ,}}%
\begin{sphinxVerbatim}[commandchars=\\\{\}]
\PYG{c+c1}{\PYGZsh{}}
\PYG{c+c1}{\PYGZsh{} load all of the data}
\PYG{c+c1}{\PYGZsh{}}
\PYG{k}{if} \PYG{n}{do\PYGZus{}load}\PYG{p}{:}
    \PYG{n}{all\PYGZus{}data}  \PYG{o}{=} \PYG{n}{mg}\PYG{o}{.}\PYG{n}{load}\PYG{p}{(}\PYG{n}{dmg}\PYG{o}{.}\PYG{n}{FNAME\PYGZus{}ALL}\PYG{p}{)}
\PYG{k}{else}\PYG{p}{:}
    \PYG{n}{all\PYGZus{}data} \PYG{o}{=} \PYG{n}{dmg}\PYG{o}{.}\PYG{n}{get\PYGZus{}all}\PYG{p}{(}\PYG{p}{)}
\end{sphinxVerbatim}

\fvset{hllines={, ,}}%
\begin{sphinxVerbatim}[commandchars=\\\{\}]
\PYG{c+c1}{\PYGZsh{}}
\PYG{c+c1}{\PYGZsh{} get all of the data for working age adults}
\PYG{c+c1}{\PYGZsh{}}
\PYG{k}{if} \PYG{n}{do\PYGZus{}load}\PYG{p}{:}
    \PYG{n}{adult} \PYG{o}{=} \PYG{n}{mg}\PYG{o}{.}\PYG{n}{load}\PYG{p}{(}\PYG{n}{dmg}\PYG{o}{.}\PYG{n}{FNAME\PYGZus{}ADULT}\PYG{p}{)}
\PYG{k}{else}\PYG{p}{:}
    \PYG{n}{adult} \PYG{o}{=} \PYG{n}{dmg}\PYG{o}{.}\PYG{n}{get\PYGZus{}adult}\PYG{p}{(}\PYG{p}{)}
\end{sphinxVerbatim}

\fvset{hllines={, ,}}%
\begin{sphinxVerbatim}[commandchars=\\\{\}]
\PYG{c+c1}{\PYGZsh{}}
\PYG{c+c1}{\PYGZsh{} get data for working adults}
\PYG{c+c1}{\PYGZsh{}}
\PYG{k}{if} \PYG{n}{do\PYGZus{}load}\PYG{p}{:}
    \PYG{n}{adult\PYGZus{}work} \PYG{o}{=} \PYG{n}{mg}\PYG{o}{.}\PYG{n}{load}\PYG{p}{(}\PYG{n}{dmg}\PYG{o}{.}\PYG{n}{FNAME\PYGZus{}ADULT\PYGZus{}WORK}\PYG{p}{)}
\PYG{k}{else}\PYG{p}{:}
    \PYG{n}{adult\PYGZus{}work} \PYG{o}{=} \PYG{n}{dmg}\PYG{o}{.}\PYG{n}{get\PYGZus{}adult\PYGZus{}work}\PYG{p}{(}\PYG{n}{adult}\PYG{p}{)}
\end{sphinxVerbatim}

\fvset{hllines={, ,}}%
\begin{sphinxVerbatim}[commandchars=\\\{\}]
\PYG{c+c1}{\PYGZsh{}}
\PYG{c+c1}{\PYGZsh{} get data for non\PYGZhy{}working adults}
\PYG{c+c1}{\PYGZsh{}}
\PYG{k}{if} \PYG{n}{do\PYGZus{}load}\PYG{p}{:}
    \PYG{n}{adult\PYGZus{}non\PYGZus{}work} \PYG{o}{=} \PYG{n}{mg}\PYG{o}{.}\PYG{n}{load}\PYG{p}{(}\PYG{n}{dmg}\PYG{o}{.}\PYG{n}{FNAME\PYGZus{}ADULT\PYGZus{}NON\PYGZus{}WORK}\PYG{p}{)}
\PYG{k}{else}\PYG{p}{:}
    \PYG{n}{adult\PYGZus{}non\PYGZus{}work} \PYG{o}{=} \PYG{n}{dmg}\PYG{o}{.}\PYG{n}{get\PYGZus{}adult\PYGZus{}non\PYGZus{}work}\PYG{p}{(}\PYG{n}{adult}\PYG{p}{)}
\end{sphinxVerbatim}

\fvset{hllines={, ,}}%
\begin{sphinxVerbatim}[commandchars=\\\{\}]
\PYG{c+c1}{\PYGZsh{}}
\PYG{c+c1}{\PYGZsh{} children school}
\PYG{c+c1}{\PYGZsh{}}
\PYG{k}{if} \PYG{n}{do\PYGZus{}load}\PYG{p}{:}
    \PYG{n}{child\PYGZus{}school} \PYG{o}{=} \PYG{n}{mg}\PYG{o}{.}\PYG{n}{load}\PYG{p}{(}\PYG{n}{dmg}\PYG{o}{.}\PYG{n}{FNAME\PYGZus{}CHILD\PYGZus{}SCHOOL}\PYG{p}{)}
\PYG{k}{else}\PYG{p}{:}
    \PYG{n}{child\PYGZus{}school} \PYG{o}{=} \PYG{n}{dmg}\PYG{o}{.}\PYG{n}{get\PYGZus{}child\PYGZus{}school}\PYG{p}{(}\PYG{p}{)}
\end{sphinxVerbatim}

\fvset{hllines={, ,}}%
\begin{sphinxVerbatim}[commandchars=\\\{\}]
\PYG{c+c1}{\PYGZsh{}}
\PYG{c+c1}{\PYGZsh{} pre\PYGZhy{}school children}
\PYG{c+c1}{\PYGZsh{}}
\PYG{k}{if} \PYG{n}{do\PYGZus{}load}\PYG{p}{:}
    \PYG{n}{child\PYGZus{}young} \PYG{o}{=} \PYG{n}{mg}\PYG{o}{.}\PYG{n}{load}\PYG{p}{(}\PYG{n}{dmg}\PYG{o}{.}\PYG{n}{FNAME\PYGZus{}CHILD\PYGZus{}YOUNG}\PYG{p}{)}
\PYG{k}{else}\PYG{p}{:}
    \PYG{n}{child\PYGZus{}young} \PYG{o}{=} \PYG{n}{dmg}\PYG{o}{.}\PYG{n}{get\PYGZus{}child\PYGZus{}young}\PYG{p}{(}\PYG{p}{)}
\end{sphinxVerbatim}

save data

save all the information for the demographics in data\_large directory

\fvset{hllines={, ,}}%
\begin{sphinxVerbatim}[commandchars=\\\{\}]
\PYG{c+c1}{\PYGZsh{} save all of the information for the following demographics}

\PYG{n}{do\PYGZus{}save} \PYG{o}{=} \PYG{k+kc}{False}

\PYG{k}{if} \PYG{n}{do\PYGZus{}save}\PYG{p}{:}
    \PYG{n}{x} \PYG{o}{=} \PYG{p}{[}\PYG{n}{all\PYGZus{}data}\PYG{p}{,} \PYG{n}{adult}\PYG{p}{,} \PYG{n}{adult\PYGZus{}non\PYGZus{}work}\PYG{p}{,} \PYG{n}{adult\PYGZus{}work}\PYG{p}{,} \PYG{n}{child\PYGZus{}school}\PYG{p}{,} \PYG{n}{child\PYGZus{}young}\PYG{p}{]}
    \PYG{n}{fnames} \PYG{o}{=} \PYG{p}{[} \PYG{n}{dmg}\PYG{o}{.}\PYG{n}{FNAME\PYGZus{}ALL}\PYG{p}{,} \PYG{n}{dmg}\PYG{o}{.}\PYG{n}{FNAME\PYGZus{}ADULT}\PYG{p}{,} \PYG{n}{dmg}\PYG{o}{.}\PYG{n}{FNAME\PYGZus{}ADULT\PYGZus{}NON\PYGZus{}WORK}\PYG{p}{,} \PYG{n}{dmg}\PYG{o}{.}\PYG{n}{FNAME\PYGZus{}ADULT\PYGZus{}WORK}\PYG{p}{,} \PYGZbs{}
               \PYG{n}{dmg}\PYG{o}{.}\PYG{n}{FNAME\PYGZus{}CHILD\PYGZus{}SCHOOL}\PYG{p}{,} \PYG{n}{dmg}\PYG{o}{.}\PYG{n}{FNAME\PYGZus{}CHILD\PYGZus{}YOUNG} \PYG{p}{]}

    \PYG{c+c1}{\PYGZsh{} save all of the data}
    \PYG{k}{for} \PYG{n}{y}\PYG{p}{,} \PYG{n}{fname} \PYG{o+ow}{in} \PYG{n+nb}{zip}\PYG{p}{(}\PYG{n}{x}\PYG{p}{,} \PYG{n}{fnames}\PYG{p}{)}\PYG{p}{:}
        \PYG{n}{save}\PYG{p}{(}\PYG{n}{y}\PYG{p}{,} \PYG{n}{fname}\PYG{p}{)}
\end{sphinxVerbatim}

Compress the demographics direcotory information

\fvset{hllines={, ,}}%
\begin{sphinxVerbatim}[commandchars=\\\{\}]
\PYG{c+c1}{\PYGZsh{}}
\PYG{c+c1}{\PYGZsh{} The demographic}
\PYG{c+c1}{\PYGZsh{}}
\PYG{n}{demos} \PYG{o}{=} \PYG{p}{[}\PYG{n}{dmg}\PYG{o}{.}\PYG{n}{ADULT\PYGZus{}WORK}\PYG{p}{,} \PYG{n}{dmg}\PYG{o}{.}\PYG{n}{ADULT\PYGZus{}NON\PYGZus{}WORK}\PYG{p}{,} \PYG{n}{dmg}\PYG{o}{.}\PYG{n}{CHILD\PYGZus{}SCHOOL}\PYG{p}{,} \PYG{n}{dmg}\PYG{o}{.}\PYG{n}{CHILD\PYGZus{}YOUNG}\PYG{p}{]}
\end{sphinxVerbatim}

\fvset{hllines={, ,}}%
\begin{sphinxVerbatim}[commandchars=\\\{\}]
\PYG{c+c1}{\PYGZsh{}}
\PYG{c+c1}{\PYGZsh{} compress the directory in the non\PYGZhy{}large data directory}
\PYG{c+c1}{\PYGZsh{}}
\PYG{n}{do\PYGZus{}compression} \PYG{o}{=} \PYG{k+kc}{False}

\PYG{n}{chooser\PYGZus{}temp} \PYG{o}{=} \PYG{p}{\PYGZob{}}\PYG{n}{dmg}\PYG{o}{.}\PYG{n}{ADULT}\PYG{p}{:} \PYG{p}{(}\PYG{n}{chad}\PYG{o}{.}\PYG{n}{FNAME\PYGZus{}ADULT}\PYG{p}{[}\PYG{p}{:}\PYG{o}{\PYGZhy{}}\PYG{l+m+mi}{4}\PYG{p}{]}\PYG{p}{,} \PYG{n}{chad}\PYG{o}{.}\PYG{n}{FDIR\PYGZus{}ADULT\PYGZus{}LARGE}\PYG{p}{)}\PYG{p}{,}
           \PYG{n}{dmg}\PYG{o}{.}\PYG{n}{ADULT\PYGZus{}WORK}\PYG{p}{:} \PYG{p}{(}\PYG{n}{chad}\PYG{o}{.}\PYG{n}{FNAME\PYGZus{}ADULT\PYGZus{}WORK}\PYG{p}{[}\PYG{p}{:}\PYG{o}{\PYGZhy{}}\PYG{l+m+mi}{4}\PYG{p}{]}\PYG{p}{,} \PYG{n}{chad}\PYG{o}{.}\PYG{n}{FDIR\PYGZus{}ADULT\PYGZus{}WORK\PYGZus{}LARGE}\PYG{p}{)}\PYG{p}{,}
           \PYG{n}{dmg}\PYG{o}{.}\PYG{n}{ADULT\PYGZus{}NON\PYGZus{}WORK}\PYG{p}{:} \PYG{p}{(}\PYG{n}{chad}\PYG{o}{.}\PYG{n}{FNAME\PYGZus{}ADULT\PYGZus{}NON\PYGZus{}WORK}\PYG{p}{[}\PYG{p}{:}\PYG{o}{\PYGZhy{}}\PYG{l+m+mi}{4}\PYG{p}{]}\PYG{p}{,} \PYG{n}{chad}\PYG{o}{.}\PYG{n}{FDIR\PYGZus{}ADULT\PYGZus{}NON\PYGZus{}WORK\PYGZus{}LARGE}\PYG{p}{)}\PYG{p}{,}
           \PYG{n}{dmg}\PYG{o}{.}\PYG{n}{CHILD\PYGZus{}SCHOOL}\PYG{p}{:} \PYG{p}{(}\PYG{n}{chad}\PYG{o}{.}\PYG{n}{FNAME\PYGZus{}CHILD\PYGZus{}SCHOOL}\PYG{p}{[}\PYG{p}{:}\PYG{o}{\PYGZhy{}}\PYG{l+m+mi}{4}\PYG{p}{]}\PYG{p}{,} \PYG{n}{chad}\PYG{o}{.}\PYG{n}{FDIR\PYGZus{}CHILD\PYGZus{}SCHOOL\PYGZus{}LARGE}\PYG{p}{)}\PYG{p}{,}
           \PYG{n}{dmg}\PYG{o}{.}\PYG{n}{CHILD\PYGZus{}YOUNG}\PYG{p}{:} \PYG{p}{(}\PYG{n}{chad}\PYG{o}{.}\PYG{n}{FNAME\PYGZus{}CHILD\PYGZus{}YOUNG}\PYG{p}{[}\PYG{p}{:}\PYG{o}{\PYGZhy{}}\PYG{l+m+mi}{4}\PYG{p}{]}\PYG{p}{,} \PYG{n}{chad}\PYG{o}{.}\PYG{n}{FDIR\PYGZus{}CHILD\PYGZus{}YOUNG\PYGZus{}LARGE}\PYG{p}{)}\PYG{p}{,}
          \PYG{p}{\PYGZcb{}}

\PYG{k}{if} \PYG{n}{do\PYGZus{}compression}\PYG{p}{:}
    \PYG{k}{for} \PYG{n}{d} \PYG{o+ow}{in} \PYG{n}{demos}\PYG{p}{:}
        \PYG{n}{fname\PYGZus{}out}\PYG{p}{,} \PYG{n}{fdir\PYGZus{}src} \PYG{o}{=} \PYG{n}{chooser\PYGZus{}temp}\PYG{p}{[}\PYG{n}{d}\PYG{p}{]}
        \PYG{n}{mg}\PYG{o}{.}\PYG{n}{save\PYGZus{}zip}\PYG{p}{(}\PYG{n}{out\PYGZus{}file}\PYG{o}{=}\PYG{n}{fname\PYGZus{}out}\PYG{p}{,} \PYG{n}{source\PYGZus{}dir}\PYG{o}{=}\PYG{n}{fdir\PYGZus{}src}\PYG{p}{)}
\end{sphinxVerbatim}

printing information about the data

\fvset{hllines={, ,}}%
\begin{sphinxVerbatim}[commandchars=\\\{\}]
\PYG{c+c1}{\PYGZsh{}}
\PYG{c+c1}{\PYGZsh{} get the data}
\PYG{c+c1}{\PYGZsh{}}
\PYG{n}{code\PYGZus{}groups} \PYG{o}{=} \PYG{p}{[} \PYG{n}{chad\PYGZus{}code}\PYG{o}{.}\PYG{n}{SLEEP}\PYG{p}{,} \PYG{n}{chad\PYGZus{}code}\PYG{o}{.}\PYG{n}{EAT}\PYG{p}{,} \PYG{n}{chad\PYGZus{}code}\PYG{o}{.}\PYG{n}{EDUCATION}\PYG{p}{,} \PYG{n}{chad\PYGZus{}code}\PYG{o}{.}\PYG{n}{WORK}\PYG{p}{,} \PYG{n}{chad\PYGZus{}code}\PYG{o}{.}\PYG{n}{COMMUTE}\PYG{p}{,} \PYGZbs{}
               \PYG{n}{chad\PYGZus{}code}\PYG{o}{.}\PYG{n}{COMMUTE\PYGZus{}EDU} \PYG{p}{]}

\PYG{c+c1}{\PYGZsh{} code\PYGZus{}groups = [chad\PYGZus{}code.SLEEP]}

\PYG{n}{df\PYGZus{}list} \PYG{o}{=} \PYG{p}{[} \PYG{n}{data}\PYG{o}{.}\PYG{n}{activity\PYGZus{}times}\PYG{p}{(}\PYG{n}{data}\PYG{o}{.}\PYG{n}{events}\PYG{p}{,} \PYG{n}{codes}\PYG{p}{)} \PYG{k}{for} \PYG{n}{codes} \PYG{o+ow}{in} \PYG{n}{code\PYGZus{}groups} \PYG{p}{]}
\end{sphinxVerbatim}

\fvset{hllines={, ,}}%
\begin{sphinxVerbatim}[commandchars=\\\{\}]
\PYG{c+c1}{\PYGZsh{}}
\PYG{c+c1}{\PYGZsh{} for each CHAD code, print information about the amount of data that is in the respective demographic group}
\PYG{c+c1}{\PYGZsh{}}
\PYG{k}{for} \PYG{n}{df}\PYG{p}{,} \PYG{n}{codes} \PYG{o+ow}{in} \PYG{n+nb}{zip}\PYG{p}{(}\PYG{n}{df\PYGZus{}list}\PYG{p}{,} \PYG{n}{code\PYGZus{}groups}\PYG{p}{)}\PYG{p}{:}

    \PYG{k}{if} \PYG{n}{do\PYGZus{}print}\PYG{p}{:}
        \PYG{n+nb}{print}\PYG{p}{(}\PYG{l+s+s1}{\PYGZsq{}}\PYG{l+s+s1}{data shape}\PYG{l+s+s1}{\PYGZsq{}}\PYG{p}{)}
        \PYG{n+nb}{print}\PYG{p}{(}\PYG{n}{df}\PYG{o}{.}\PYG{n}{shape}\PYG{p}{)}


        \PYG{n+nb}{print}\PYG{p}{(}\PYG{l+s+s1}{\PYGZsq{}}\PYG{l+s+s1}{number of individuals: }\PYG{l+s+si}{\PYGZpc{}d}\PYG{l+s+s1}{\PYGZsq{}} \PYG{o}{\PYGZpc{}} \PYG{n+nb}{len}\PYG{p}{(} \PYG{n}{df}\PYG{o}{.}\PYG{n}{PID}\PYG{o}{.}\PYG{n}{unique}\PYG{p}{(}\PYG{p}{)} \PYG{p}{)} \PYG{p}{)}

        \PYG{k}{for} \PYG{n}{act} \PYG{o+ow}{in} \PYG{n}{codes}\PYG{p}{:}
            \PYG{n}{temp} \PYG{o}{=} \PYG{n}{df}\PYG{p}{[}\PYG{n}{df}\PYG{o}{.}\PYG{n}{act} \PYG{o}{==} \PYG{n}{act}\PYG{p}{]}
            \PYG{n+nb}{print}\PYG{p}{(}\PYG{l+s+s1}{\PYGZsq{}}\PYG{l+s+si}{\PYGZpc{}s}\PYG{l+s+s1}{:}\PYG{l+s+se}{\PYGZbs{}t}\PYG{l+s+s1}{Individuals:}\PYG{l+s+se}{\PYGZbs{}t}\PYG{l+s+si}{\PYGZpc{}d}\PYG{l+s+se}{\PYGZbs{}t}\PYG{l+s+s1}{Count:}\PYG{l+s+se}{\PYGZbs{}t}\PYG{l+s+si}{\PYGZpc{}d}\PYG{l+s+s1}{\PYGZsq{}} \PYG{o}{\PYGZpc{}} \PYG{p}{(}\PYG{n}{chad\PYGZus{}code}\PYG{o}{.}\PYG{n}{INT\PYGZus{}2\PYGZus{}STR}\PYG{p}{[}\PYG{n}{act}\PYG{p}{]}\PYG{p}{,} \PYG{n+nb}{len}\PYG{p}{(}\PYG{n}{temp}\PYG{o}{.}\PYG{n}{PID}\PYG{o}{.}\PYG{n}{unique}\PYG{p}{(}\PYG{p}{)}\PYG{p}{)}\PYG{p}{,} \PYGZbs{}
                                                         \PYG{n+nb}{len}\PYG{p}{(}\PYG{n}{temp}\PYG{p}{)} \PYG{p}{)} \PYG{p}{)}

        \PYG{n+nb}{print}\PYG{p}{(}\PYG{l+s+s1}{\PYGZsq{}}\PYG{l+s+se}{\PYGZbs{}n}\PYG{l+s+s1}{\PYGZsq{}}\PYG{p}{)}
\end{sphinxVerbatim}

plotting

\fvset{hllines={, ,}}%
\begin{sphinxVerbatim}[commandchars=\\\{\}]
\PYG{n}{chooser\PYGZus{}fpath} \PYG{o}{=}\PYG{p}{\PYGZob{}}\PYG{n}{dmg}\PYG{o}{.}\PYG{n}{ALL}\PYG{p}{:} \PYG{n}{mg}\PYG{o}{.}\PYG{n}{FDIR\PYGZus{}SAVE\PYGZus{}FIG\PYGZus{}ALL}\PYG{p}{,}
                \PYG{n}{dmg}\PYG{o}{.}\PYG{n}{ADULT}\PYG{p}{:} \PYG{n}{mg}\PYG{o}{.}\PYG{n}{FDIR\PYGZus{}SAVE\PYGZus{}FIG\PYGZus{}ADULT}\PYG{p}{,}
                \PYG{n}{dmg}\PYG{o}{.}\PYG{n}{ADULT\PYGZus{}WORK}\PYG{p}{:} \PYG{n}{mg}\PYG{o}{.}\PYG{n}{FDIR\PYGZus{}SAVE\PYGZus{}FIG\PYGZus{}ADULT\PYGZus{}WORK}\PYG{p}{,}
                \PYG{n}{dmg}\PYG{o}{.}\PYG{n}{ADULT\PYGZus{}NON\PYGZus{}WORK}\PYG{p}{:} \PYG{n}{mg}\PYG{o}{.}\PYG{n}{FDIR\PYGZus{}SAVE\PYGZus{}FIG\PYGZus{}ADULT\PYGZus{}NON\PYGZus{}WORK}\PYG{p}{,}
                \PYG{n}{dmg}\PYG{o}{.}\PYG{n}{CHILD\PYGZus{}SCHOOL}\PYG{p}{:} \PYG{n}{mg}\PYG{o}{.}\PYG{n}{FDIR\PYGZus{}SAVE\PYGZus{}FIG\PYGZus{}CHILD\PYGZus{}SCHOOL}\PYG{p}{,}
                \PYG{n}{dmg}\PYG{o}{.}\PYG{n}{CHILD\PYGZus{}YOUNG}\PYG{p}{:} \PYG{n}{mg}\PYG{o}{.}\PYG{n}{FDIR\PYGZus{}SAVE\PYGZus{}FIG\PYGZus{}CHILD\PYGZus{}YOUNG}\PYG{p}{,}
               \PYG{p}{\PYGZcb{}}

\PYG{n}{chooser\PYGZus{}data} \PYG{o}{=} \PYG{p}{\PYGZob{}}\PYG{n}{dmg}\PYG{o}{.}\PYG{n}{ALL}\PYG{p}{:} \PYG{n}{all\PYGZus{}data}\PYG{p}{,}
                \PYG{n}{dmg}\PYG{o}{.}\PYG{n}{ADULT}\PYG{p}{:} \PYG{n}{adult}\PYG{p}{,}
                \PYG{n}{dmg}\PYG{o}{.}\PYG{n}{ADULT\PYGZus{}WORK}\PYG{p}{:} \PYG{n}{adult\PYGZus{}work}\PYG{p}{,}
                \PYG{n}{dmg}\PYG{o}{.}\PYG{n}{ADULT\PYGZus{}NON\PYGZus{}WORK}\PYG{p}{:} \PYG{n}{adult\PYGZus{}non\PYGZus{}work}\PYG{p}{,}
                \PYG{n}{dmg}\PYG{o}{.}\PYG{n}{CHILD\PYGZus{}SCHOOL}\PYG{p}{:} \PYG{n}{child\PYGZus{}school}\PYG{p}{,}
                \PYG{n}{dmg}\PYG{o}{.}\PYG{n}{CHILD\PYGZus{}YOUNG}\PYG{p}{:} \PYG{n}{child\PYGZus{}young}\PYG{p}{,}
               \PYG{p}{\PYGZcb{}}
\end{sphinxVerbatim}

\fvset{hllines={, ,}}%
\begin{sphinxVerbatim}[commandchars=\\\{\}]
\PYG{c+c1}{\PYGZsh{}}
\PYG{c+c1}{\PYGZsh{} get data and fpath for saving}
\PYG{c+c1}{\PYGZsh{}}
\PYG{n}{data} \PYG{o}{=} \PYG{n}{chooser\PYGZus{}data}\PYG{p}{[}\PYG{n}{demo}\PYG{p}{]}
\PYG{n}{fpath} \PYG{o}{=} \PYG{n}{chooser\PYGZus{}fpath}\PYG{p}{[}\PYG{n}{demo}\PYG{p}{]} \PYG{o}{+} \PYG{l+s+s1}{\PYGZsq{}}\PYG{l+s+se}{\PYGZbs{}\PYGZbs{}}\PYG{l+s+s1}{chad}\PYG{l+s+s1}{\PYGZsq{}}

\PYG{n+nb}{print}\PYG{p}{(}\PYG{n}{fpath}\PYG{p}{)}
\end{sphinxVerbatim}
\begin{sphinxalltt}
..my\_datafigdemographicadult\_workchad
\end{sphinxalltt}

\fvset{hllines={, ,}}%
\begin{sphinxVerbatim}[commandchars=\\\{\}]
\PYG{c+c1}{\PYGZsh{} flags for figures}

\PYG{c+c1}{\PYGZsh{} plot the figures}
\PYG{n}{do\PYGZus{}plot} \PYG{o}{=} \PYG{k+kc}{False}

\PYG{c+c1}{\PYGZsh{} save the figure plots}
\PYG{n}{do\PYGZus{}save\PYGZus{}fig}\PYG{o}{=} \PYG{k+kc}{False}
\end{sphinxVerbatim}

\fvset{hllines={, ,}}%
\begin{sphinxVerbatim}[commandchars=\\\{\}]
\PYG{c+c1}{\PYGZsh{}}
\PYG{c+c1}{\PYGZsh{} plot the histograms}
\PYG{c+c1}{\PYGZsh{}}

\PYG{k}{if} \PYG{n}{do\PYGZus{}plot}\PYG{p}{:}

    \PYG{k}{for} \PYG{n}{df}\PYG{p}{,} \PYG{n}{codes} \PYG{o+ow}{in} \PYG{n+nb}{zip}\PYG{p}{(}\PYG{n}{df\PYGZus{}list}\PYG{p}{,} \PYG{n}{code\PYGZus{}groups}\PYG{p}{)}\PYG{p}{:}

        \PYG{n}{plot\PYGZus{}histograms}\PYG{p}{(}\PYG{n}{df}\PYG{p}{,} \PYG{n}{codes}\PYG{p}{,} \PYG{n}{num\PYGZus{}bins}\PYG{o}{=}\PYG{l+m+mi}{24}\PYG{p}{,} \PYG{n}{do\PYGZus{}save}\PYG{o}{=}\PYG{n}{do\PYGZus{}save\PYGZus{}fig}\PYG{p}{,} \PYG{n}{fpath}\PYG{o}{=}\PYG{n}{fpath}\PYG{p}{)}

    \PYG{n}{plt}\PYG{o}{.}\PYG{n}{show}\PYG{p}{(}\PYG{p}{)}
\end{sphinxVerbatim}

\fvset{hllines={, ,}}%
\begin{sphinxVerbatim}[commandchars=\\\{\}]
\PYG{c+c1}{\PYGZsh{}}
\PYG{c+c1}{\PYGZsh{} plot the CDFs}
\PYG{c+c1}{\PYGZsh{}}
\PYG{k}{if} \PYG{n}{do\PYGZus{}plot}\PYG{p}{:}

    \PYG{k}{for} \PYG{n}{df}\PYG{p}{,} \PYG{n}{codes} \PYG{o+ow}{in} \PYG{n+nb}{zip}\PYG{p}{(}\PYG{n}{df\PYGZus{}list}\PYG{p}{,} \PYG{n}{code\PYGZus{}groups}\PYG{p}{)}\PYG{p}{:}

        \PYG{n}{plot\PYGZus{}cdfs}\PYG{p}{(}\PYG{n}{df}\PYG{p}{,} \PYG{n}{codes}\PYG{p}{,} \PYG{n}{linewidth}\PYG{o}{=}\PYG{l+m+mi}{2}\PYG{p}{,} \PYG{n}{do\PYGZus{}save}\PYG{o}{=}\PYG{n}{do\PYGZus{}save\PYGZus{}fig}\PYG{p}{,} \PYG{n}{fpath}\PYG{o}{=}\PYG{n}{fpath}\PYG{p}{)}

    \PYG{n}{plt}\PYG{o}{.}\PYG{n}{show}\PYG{p}{(}\PYG{p}{)}
\end{sphinxVerbatim}


\subsection{demography module}
\label{\detokenize{demography::doc}}\label{\detokenize{demography:module-demography}}\label{\detokenize{demography:demography-module}}\index{demography (module)}
This module handles the logistics of data dealing with demographics from the raw data from the Consolidated Human Activity Database (CHAD) data in order to be used in Agent-Based Model of Human Activity Patterns (ABMHAP).
\index{filter\_adult() (in module demography)}

\begin{fulllineitems}
\phantomsection\label{\detokenize{demography:demography.filter_adult}}\pysiglinewithargsret{\sphinxcode{\sphinxupquote{demography.}}\sphinxbfcode{\sphinxupquote{filter\_adult}}}{\emph{x}, \emph{do\_work}}{}
This function goes through the adult CHAD data and filters the results if     the data is supposed to be for working adult or non-working adults.
\begin{quote}\begin{description}
\item[{Parameters}] \leavevmode\begin{itemize}
\item {} 
\sphinxstyleliteralstrong{\sphinxupquote{x}} ({\hyperref[\detokenize{chad:chad.CHAD_RAW}]{\sphinxcrossref{\sphinxstyleliteralemphasis{\sphinxupquote{chad.CHAD\_RAW}}}}}) \textendash{} CHAD data for adults

\item {} 
\sphinxstyleliteralstrong{\sphinxupquote{do\_work}} (\sphinxstyleliteralemphasis{\sphinxupquote{bool}}) \textendash{} a flag indicating whether to get data from working     adults (if True) or non-working adults (if False)

\end{itemize}

\item[{Returns}] \leavevmode


\end{description}\end{quote}

\end{fulllineitems}

\index{get\_adult() (in module demography)}

\begin{fulllineitems}
\phantomsection\label{\detokenize{demography:demography.get_adult}}\pysiglinewithargsret{\sphinxcode{\sphinxupquote{demography.}}\sphinxbfcode{\sphinxupquote{get\_adult}}}{}{}
This function gets the CHAD data for adults.
\begin{quote}\begin{description}
\item[{Returns}] \leavevmode
the raw CHAD data from individuals that correspond to adult age

\item[{Return type}] \leavevmode
{\hyperref[\detokenize{chad:chad.CHAD_RAW}]{\sphinxcrossref{chad.CHAD\_RAW}}}

\end{description}\end{quote}

\end{fulllineitems}

\index{get\_adult\_non\_work() (in module demography)}

\begin{fulllineitems}
\phantomsection\label{\detokenize{demography:demography.get_adult_non_work}}\pysiglinewithargsret{\sphinxcode{\sphinxupquote{demography.}}\sphinxbfcode{\sphinxupquote{get\_adult\_non\_work}}}{\emph{adult}}{}
This function gets raw CHAD data from non-working adults.
\begin{quote}\begin{description}
\item[{Parameters}] \leavevmode
\sphinxstyleliteralstrong{\sphinxupquote{adult}} ({\hyperref[\detokenize{chad:chad.CHAD_RAW}]{\sphinxcrossref{\sphinxstyleliteralemphasis{\sphinxupquote{chad.CHAD\_RAW}}}}}) \textendash{} the raw adult data from CHAD

\item[{Returns}] \leavevmode
raw CHAD data from non-working adults

\item[{Return type}] \leavevmode
{\hyperref[\detokenize{chad:chad.CHAD_RAW}]{\sphinxcrossref{chad.CHAD\_RAW}}}

\end{description}\end{quote}

\end{fulllineitems}

\index{get\_adult\_work() (in module demography)}

\begin{fulllineitems}
\phantomsection\label{\detokenize{demography:demography.get_adult_work}}\pysiglinewithargsret{\sphinxcode{\sphinxupquote{demography.}}\sphinxbfcode{\sphinxupquote{get\_adult\_work}}}{\emph{adult}}{}
This function gets raw CHAD data from working adults.
\begin{quote}\begin{description}
\item[{Parameters}] \leavevmode
\sphinxstyleliteralstrong{\sphinxupquote{adult}} ({\hyperref[\detokenize{chad:chad.CHAD_RAW}]{\sphinxcrossref{\sphinxstyleliteralemphasis{\sphinxupquote{chad.CHAD\_RAW}}}}}) \textendash{} the raw adult data from CHAD

\item[{Returns}] \leavevmode
raw CHAD data from working adults

\item[{Return type}] \leavevmode
{\hyperref[\detokenize{chad:chad.CHAD_RAW}]{\sphinxcrossref{chad.CHAD\_RAW}}}

\end{description}\end{quote}

\end{fulllineitems}

\index{get\_all() (in module demography)}

\begin{fulllineitems}
\phantomsection\label{\detokenize{demography:demography.get_all}}\pysiglinewithargsret{\sphinxcode{\sphinxupquote{demography.}}\sphinxbfcode{\sphinxupquote{get\_all}}}{}{}
This function gets all of the raw CHAD data.
\begin{quote}\begin{description}
\item[{Returns}] \leavevmode
all of the raw CHAD data

\item[{Return type}] \leavevmode
{\hyperref[\detokenize{chad:chad.CHAD_RAW}]{\sphinxcrossref{chad.CHAD\_RAW}}}

\end{description}\end{quote}

\end{fulllineitems}

\index{get\_child\_school() (in module demography)}

\begin{fulllineitems}
\phantomsection\label{\detokenize{demography:demography.get_child_school}}\pysiglinewithargsret{\sphinxcode{\sphinxupquote{demography.}}\sphinxbfcode{\sphinxupquote{get\_child\_school}}}{}{}
This function gets the CHAD data for school-age children.
\begin{quote}\begin{description}
\item[{Returns}] \leavevmode
the raw CHAD data from individuals that correspond to school-age children

\item[{Return type}] \leavevmode
{\hyperref[\detokenize{chad:chad.CHAD_RAW}]{\sphinxcrossref{chad.CHAD\_RAW}}}

\end{description}\end{quote}

\end{fulllineitems}

\index{get\_child\_young() (in module demography)}

\begin{fulllineitems}
\phantomsection\label{\detokenize{demography:demography.get_child_young}}\pysiglinewithargsret{\sphinxcode{\sphinxupquote{demography.}}\sphinxbfcode{\sphinxupquote{get\_child\_young}}}{}{}
This function gets the CHAD data for preschool children.
\begin{quote}\begin{description}
\item[{Returns}] \leavevmode
the raw CHAD data from individuals that correspond to preschool children

\item[{Return type}] \leavevmode
{\hyperref[\detokenize{chad:chad.CHAD_RAW}]{\sphinxcrossref{chad.CHAD\_RAW}}}

\end{description}\end{quote}

\end{fulllineitems}

\index{load() (in module demography)}

\begin{fulllineitems}
\phantomsection\label{\detokenize{demography:demography.load}}\pysiglinewithargsret{\sphinxcode{\sphinxupquote{demography.}}\sphinxbfcode{\sphinxupquote{load}}}{\emph{fname}}{}
This function loads data given by the file name
\begin{quote}\begin{description}
\item[{Parameters}] \leavevmode
\sphinxstyleliteralstrong{\sphinxupquote{fname}} (\sphinxstyleliteralemphasis{\sphinxupquote{str}}) \textendash{} the file name of the data to load

\item[{Returns}] \leavevmode
the data

\end{description}\end{quote}

\end{fulllineitems}

\index{save() (in module demography)}

\begin{fulllineitems}
\phantomsection\label{\detokenize{demography:demography.save}}\pysiglinewithargsret{\sphinxcode{\sphinxupquote{demography.}}\sphinxbfcode{\sphinxupquote{save}}}{\emph{x}, \emph{fname}}{}
This function saves the raw CHAD data for the given demographic as a .pkl file.
\begin{quote}\begin{description}
\item[{Parameters}] \leavevmode\begin{itemize}
\item {} 
\sphinxstyleliteralstrong{\sphinxupquote{x}} ({\hyperref[\detokenize{chad:chad.CHAD_RAW}]{\sphinxcrossref{\sphinxstyleliteralemphasis{\sphinxupquote{chad.CHAD\_RAW}}}}}) \textendash{} the raw CHAD data to save for a given demographic

\item {} 
\sphinxstyleliteralstrong{\sphinxupquote{fname}} (\sphinxstyleliteralemphasis{\sphinxupquote{str}}) \textendash{} the file name to save raw CHAD data for a given demographic

\end{itemize}

\item[{Returns}] \leavevmode


\end{description}\end{quote}

\end{fulllineitems}



\subsection{eat\_new notebook}
\label{\detokenize{eat_new::doc}}\label{\detokenize{eat_new:eat-new-notebook}}
\fvset{hllines={, ,}}%
\begin{sphinxVerbatim}[commandchars=\\\{\}]
\PYG{c+c1}{\PYGZsh{} The United States Environmental Protection Agency through its Office of}
\PYG{c+c1}{\PYGZsh{} Research and Development has developed this software. The code is made}
\PYG{c+c1}{\PYGZsh{} publicly available to better communicate the research. All input data}
\PYG{c+c1}{\PYGZsh{} used fora given application should be reviewed by the researcher so}
\PYG{c+c1}{\PYGZsh{} that the model results are based on appropriate data for any given}
\PYG{c+c1}{\PYGZsh{} application. This model is under continued development. The model and}
\PYG{c+c1}{\PYGZsh{} data included herein do not represent and should not be construed to}
\PYG{c+c1}{\PYGZsh{} represent any Agency determination or policy.}
\PYG{c+c1}{\PYGZsh{}}
\PYG{c+c1}{\PYGZsh{} This file was written by Dr. Namdi Brandon}
\PYG{c+c1}{\PYGZsh{} ORCID: 0000\PYGZhy{}0001\PYGZhy{}7050\PYGZhy{}1538}
\PYG{c+c1}{\PYGZsh{} March 22, 2018}
\end{sphinxVerbatim}

This file goes through the data from the Consoldiated Human Activity
Database (CHAD) and gets information relevent to \sphinxstylestrong{eating breakfast},
\sphinxstylestrong{eating lunch}, and \sphinxstylestrong{eating dinner} and processes the data for use
in the Agent-Based Model of Human Activity Patterns (ABMHAP) for each
demographic. More specficially, this file does the following:

For a given demographic,
\begin{enumerate}
\item {} 
This function goes through the CHAD data and finds the eat-activity
data

\item {} 
The CHAD activity data are seperated into start time, end time,
duration, and CHAD record data for the meals: breakfast, lunch, and
dinner

\item {} 
The CHAD activity data is saved into longitudinal data and
single-activity data

\end{enumerate}

Import

\fvset{hllines={, ,}}%
\begin{sphinxVerbatim}[commandchars=\\\{\}]
\PYG{k+kn}{import} \PYG{n+nn}{sys}
\PYG{n}{sys}\PYG{o}{.}\PYG{n}{path}\PYG{o}{.}\PYG{n}{append}\PYG{p}{(}\PYG{l+s+s1}{\PYGZsq{}}\PYG{l+s+s1}{..}\PYG{l+s+se}{\PYGZbs{}\PYGZbs{}}\PYG{l+s+s1}{source}\PYG{l+s+s1}{\PYGZsq{}}\PYG{p}{)}

\PYG{c+c1}{\PYGZsh{} plotting capability}
\PYG{k+kn}{import} \PYG{n+nn}{matplotlib}\PYG{n+nn}{.}\PYG{n+nn}{pylab} \PYG{k}{as} \PYG{n+nn}{plt}

\PYG{c+c1}{\PYGZsh{} ABMHAP modules}
\PYG{k+kn}{import} \PYG{n+nn}{demography} \PYG{k}{as} \PYG{n+nn}{dmg}
\PYG{k+kn}{import} \PYG{n+nn}{datum}
\end{sphinxVerbatim}

\fvset{hllines={, ,}}%
\begin{sphinxVerbatim}[commandchars=\\\{\}]
\PYG{o}{\PYGZpc{}}\PYG{k}{matplotlib} notebook
\end{sphinxVerbatim}

Load data

\fvset{hllines={, ,}}%
\begin{sphinxVerbatim}[commandchars=\\\{\}]
\PYG{c+c1}{\PYGZsh{}}
\PYG{c+c1}{\PYGZsh{} the demographic}
\PYG{c+c1}{\PYGZsh{}}
\PYG{n}{key} \PYG{o}{=} \PYG{n}{dmg}\PYG{o}{.}\PYG{n}{CHILD\PYGZus{}YOUNG}

\PYG{c+c1}{\PYGZsh{} the input file and output file directory}
\PYG{n}{fname\PYGZus{}input}\PYG{p}{,} \PYG{n}{fpath\PYGZus{}output} \PYG{o}{=} \PYG{n}{dmg}\PYG{o}{.}\PYG{n}{INT\PYGZus{}2\PYGZus{}FIN\PYGZus{}FOUT\PYGZus{}LARGE}\PYG{p}{[}\PYG{n}{key}\PYG{p}{]}

\PYG{c+c1}{\PYGZsh{} load the data}
\PYG{n}{data} \PYG{o}{=} \PYG{n}{dmg}\PYG{o}{.}\PYG{n}{load}\PYG{p}{(}\PYG{n}{fname\PYGZus{}input}\PYG{p}{)}
\end{sphinxVerbatim}

Process the data

\fvset{hllines={, ,}}%
\begin{sphinxVerbatim}[commandchars=\\\{\}]
\PYG{c+c1}{\PYGZsh{} analyze the eat\PYGZhy{}activity data}
\PYG{n}{d\PYGZus{}breakfast}\PYG{p}{,} \PYG{n}{d\PYGZus{}lunch}\PYG{p}{,} \PYG{n}{d\PYGZus{}dinner} \PYG{o}{=} \PYG{n}{datum}\PYG{o}{.}\PYG{n}{analyze\PYGZus{}eat}\PYG{p}{(}\PYG{n}{data}\PYG{p}{)}
\end{sphinxVerbatim}

Plot the distribution

\fvset{hllines={, ,}}%
\begin{sphinxVerbatim}[commandchars=\\\{\}]
\PYG{c+c1}{\PYGZsh{}}
\PYG{c+c1}{\PYGZsh{} plot the distribution}
\PYG{c+c1}{\PYGZsh{}}
\PYG{n}{d} \PYG{o}{=} \PYG{n}{d\PYGZus{}dinner}

\PYG{n}{temp} \PYG{o}{=} \PYG{n}{d}\PYG{p}{[}\PYG{l+s+s1}{\PYGZsq{}}\PYG{l+s+s1}{data}\PYG{l+s+s1}{\PYGZsq{}}\PYG{p}{]}

\PYG{n}{ylabel} \PYG{o}{=} \PYG{l+s+s1}{\PYGZsq{}}\PYG{l+s+s1}{Relative Frequency}\PYG{l+s+s1}{\PYGZsq{}}
\PYG{n}{xlabel} \PYG{o}{=} \PYG{l+s+s1}{\PYGZsq{}}\PYG{l+s+s1}{Time [h]}\PYG{l+s+s1}{\PYGZsq{}}

\PYG{n}{fig}\PYG{p}{,} \PYG{n}{axes} \PYG{o}{=} \PYG{n}{plt}\PYG{o}{.}\PYG{n}{subplots}\PYG{p}{(}\PYG{l+m+mi}{2}\PYG{p}{,}\PYG{l+m+mi}{2}\PYG{p}{)}

\PYG{c+c1}{\PYGZsh{} start time}
\PYG{n}{ax} \PYG{o}{=} \PYG{n}{axes}\PYG{p}{[}\PYG{l+m+mi}{0}\PYG{p}{,}\PYG{l+m+mi}{0}\PYG{p}{]}

\PYG{n}{datum}\PYG{o}{.}\PYG{n}{histogram}\PYG{p}{(}\PYG{n}{ax}\PYG{p}{,} \PYG{n}{temp}\PYG{o}{.}\PYG{n}{start}\PYG{o}{.}\PYG{n}{values}\PYG{p}{,} \PYG{n}{color}\PYG{o}{=}\PYG{l+s+s1}{\PYGZsq{}}\PYG{l+s+s1}{b}\PYG{l+s+s1}{\PYGZsq{}}\PYG{p}{,} \PYG{n}{label}\PYG{o}{=}\PYG{l+s+s1}{\PYGZsq{}}\PYG{l+s+s1}{start}\PYG{l+s+s1}{\PYGZsq{}}\PYG{p}{)}
\PYG{n}{ax}\PYG{o}{.}\PYG{n}{set\PYGZus{}ylabel}\PYG{p}{(}\PYG{n}{ylabel}\PYG{p}{)}
\PYG{n}{ax}\PYG{o}{.}\PYG{n}{set\PYGZus{}xlabel}\PYG{p}{(}\PYG{n}{xlabel}\PYG{p}{)}
\PYG{n}{ax}\PYG{o}{.}\PYG{n}{legend}\PYG{p}{(}\PYG{n}{loc}\PYG{o}{=}\PYG{l+s+s1}{\PYGZsq{}}\PYG{l+s+s1}{best}\PYG{l+s+s1}{\PYGZsq{}}\PYG{p}{)}

\PYG{c+c1}{\PYGZsh{} end time}
\PYG{n}{ax} \PYG{o}{=} \PYG{n}{axes}\PYG{p}{[}\PYG{l+m+mi}{0}\PYG{p}{,} \PYG{l+m+mi}{1}\PYG{p}{]}
\PYG{n}{datum}\PYG{o}{.}\PYG{n}{histogram}\PYG{p}{(}\PYG{n}{ax}\PYG{p}{,} \PYG{n}{temp}\PYG{o}{.}\PYG{n}{end}\PYG{o}{.}\PYG{n}{values}\PYG{p}{,} \PYG{n}{color}\PYG{o}{=}\PYG{l+s+s1}{\PYGZsq{}}\PYG{l+s+s1}{g}\PYG{l+s+s1}{\PYGZsq{}}\PYG{p}{,} \PYG{n}{label}\PYG{o}{=}\PYG{l+s+s1}{\PYGZsq{}}\PYG{l+s+s1}{end}\PYG{l+s+s1}{\PYGZsq{}}\PYG{p}{)}
\PYG{n}{ax}\PYG{o}{.}\PYG{n}{set\PYGZus{}ylabel}\PYG{p}{(}\PYG{n}{ylabel}\PYG{p}{)}
\PYG{n}{ax}\PYG{o}{.}\PYG{n}{set\PYGZus{}xlabel}\PYG{p}{(}\PYG{n}{xlabel}\PYG{p}{)}
\PYG{n}{ax}\PYG{o}{.}\PYG{n}{legend}\PYG{p}{(}\PYG{n}{loc}\PYG{o}{=}\PYG{l+s+s1}{\PYGZsq{}}\PYG{l+s+s1}{best}\PYG{l+s+s1}{\PYGZsq{}}\PYG{p}{)}


\PYG{c+c1}{\PYGZsh{} duration}
\PYG{n}{ax} \PYG{o}{=} \PYG{n}{axes}\PYG{p}{[}\PYG{l+m+mi}{1}\PYG{p}{,} \PYG{l+m+mi}{0}\PYG{p}{]}
\PYG{n}{datum}\PYG{o}{.}\PYG{n}{histogram}\PYG{p}{(}\PYG{n}{ax}\PYG{p}{,} \PYG{n}{temp}\PYG{o}{.}\PYG{n}{dt}\PYG{o}{.}\PYG{n}{values}\PYG{p}{,} \PYG{n}{color}\PYG{o}{=}\PYG{l+s+s1}{\PYGZsq{}}\PYG{l+s+s1}{r}\PYG{l+s+s1}{\PYGZsq{}}\PYG{p}{,} \PYG{n}{label}\PYG{o}{=}\PYG{l+s+s1}{\PYGZsq{}}\PYG{l+s+s1}{duration}\PYG{l+s+s1}{\PYGZsq{}}\PYG{p}{)}
\PYG{n}{ax}\PYG{o}{.}\PYG{n}{set\PYGZus{}ylabel}\PYG{p}{(}\PYG{n}{ylabel}\PYG{p}{)}
\PYG{n}{ax}\PYG{o}{.}\PYG{n}{set\PYGZus{}xlabel}\PYG{p}{(}\PYG{n}{xlabel}\PYG{p}{)}
\PYG{n}{ax}\PYG{o}{.}\PYG{n}{legend}\PYG{p}{(}\PYG{n}{loc}\PYG{o}{=}\PYG{l+s+s1}{\PYGZsq{}}\PYG{l+s+s1}{best}\PYG{l+s+s1}{\PYGZsq{}}\PYG{p}{)}

\PYG{n}{plt}\PYG{o}{.}\PYG{n}{show}\PYG{p}{(}\PYG{p}{)}
\end{sphinxVerbatim}

Save the data

\fvset{hllines={, ,}}%
\begin{sphinxVerbatim}[commandchars=\\\{\}]
\PYG{c+c1}{\PYGZsh{} choose to save longitudinal data or single\PYGZhy{}day data}
\PYG{n}{chooser} \PYG{o}{=} \PYG{p}{\PYGZob{}}\PYG{k+kc}{True}\PYG{p}{:} \PYG{p}{(}\PYG{l+m+mi}{2}\PYG{p}{,} \PYG{n}{fpath\PYGZus{}output} \PYG{o}{+} \PYG{l+s+s1}{\PYGZsq{}}\PYG{l+s+se}{\PYGZbs{}\PYGZbs{}}\PYG{l+s+s1}{longitude}\PYG{l+s+s1}{\PYGZsq{}}\PYG{p}{)}\PYG{p}{,}
           \PYG{k+kc}{False}\PYG{p}{:} \PYG{p}{(}\PYG{l+m+mi}{1}\PYG{p}{,} \PYG{n}{fpath\PYGZus{}output} \PYG{o}{+} \PYG{l+s+s1}{\PYGZsq{}}\PYG{l+s+se}{\PYGZbs{}\PYGZbs{}}\PYG{l+s+s1}{solo}\PYG{l+s+s1}{\PYGZsq{}}\PYG{p}{)}\PYG{p}{,} \PYG{p}{\PYGZcb{}}

\PYG{c+c1}{\PYGZsh{} whether to save the longitudinal data (if True) or the single\PYGZhy{}day data (if False)}
\PYG{n}{do\PYGZus{}long} \PYG{o}{=} \PYG{k+kc}{False}
\end{sphinxVerbatim}

\fvset{hllines={, ,}}%
\begin{sphinxVerbatim}[commandchars=\\\{\}]
\PYG{c+c1}{\PYGZsh{}}
\PYG{c+c1}{\PYGZsh{} save the data}

\PYG{n}{do\PYGZus{}save} \PYG{o}{=} \PYG{k+kc}{False}

\PYG{k}{if} \PYG{n}{do\PYGZus{}save}\PYG{p}{:}

    \PYG{n}{N}\PYG{p}{,} \PYG{n}{fpath} \PYG{o}{=} \PYG{n}{chooser}\PYG{p}{[}\PYG{n}{do\PYGZus{}long}\PYG{p}{]}

    \PYG{c+c1}{\PYGZsh{} the directories the data should be saved in}
    \PYG{n}{fpaths} \PYG{o}{=} \PYG{p}{[}\PYG{n}{fpath} \PYG{o}{+} \PYG{l+s+s1}{\PYGZsq{}}\PYG{l+s+se}{\PYGZbs{}\PYGZbs{}}\PYG{l+s+s1}{eat\PYGZus{}breakfast}\PYG{l+s+s1}{\PYGZsq{}}\PYG{p}{,} \PYG{n}{fpath} \PYG{o}{+} \PYG{l+s+s1}{\PYGZsq{}}\PYG{l+s+se}{\PYGZbs{}\PYGZbs{}}\PYG{l+s+s1}{eat\PYGZus{}lunch}\PYG{l+s+s1}{\PYGZsq{}}\PYG{p}{,} \PYG{n}{fpath} \PYG{o}{+} \PYG{l+s+s1}{\PYGZsq{}}\PYG{l+s+se}{\PYGZbs{}\PYGZbs{}}\PYG{l+s+s1}{eat\PYGZus{}dinner}\PYG{l+s+s1}{\PYGZsq{}}\PYG{p}{]}

    \PYG{c+c1}{\PYGZsh{} the dictionaries holding the data}
    \PYG{n}{data\PYGZus{}dict} \PYG{o}{=} \PYG{p}{[}\PYG{n}{d\PYGZus{}breakfast}\PYG{p}{,} \PYG{n}{d\PYGZus{}lunch}\PYG{p}{,} \PYG{n}{d\PYGZus{}dinner}\PYG{p}{]}


    \PYG{c+c1}{\PYGZsh{} save the data}
    \PYG{k}{for} \PYG{n}{fpath}\PYG{p}{,} \PYG{n}{d} \PYG{o+ow}{in} \PYG{n+nb}{zip}\PYG{p}{(}\PYG{n}{fpaths}\PYG{p}{,} \PYG{n}{data\PYGZus{}dict}\PYG{p}{)}\PYG{p}{:}

        \PYG{n}{stats\PYGZus{}dt}\PYG{p}{,} \PYG{n}{stats\PYGZus{}start}\PYG{p}{,} \PYG{n}{stats\PYGZus{}end}\PYG{p}{,} \PYG{n}{record} \PYG{o}{=} \PYG{n}{d}\PYG{p}{[}\PYG{l+s+s1}{\PYGZsq{}}\PYG{l+s+s1}{stats\PYGZus{}dt}\PYG{l+s+s1}{\PYGZsq{}}\PYG{p}{]}\PYG{p}{,} \PYG{n}{d}\PYG{p}{[}\PYG{l+s+s1}{\PYGZsq{}}\PYG{l+s+s1}{stats\PYGZus{}start}\PYG{l+s+s1}{\PYGZsq{}}\PYG{p}{]}\PYG{p}{,} \PYG{n}{d}\PYG{p}{[}\PYG{l+s+s1}{\PYGZsq{}}\PYG{l+s+s1}{stats\PYGZus{}end}\PYG{l+s+s1}{\PYGZsq{}}\PYG{p}{]}\PYG{p}{,} \PYG{n}{d}\PYG{p}{[}\PYG{l+s+s1}{\PYGZsq{}}\PYG{l+s+s1}{data}\PYG{l+s+s1}{\PYGZsq{}}\PYG{p}{]}

        \PYG{k}{if} \PYG{n}{do\PYGZus{}long}\PYG{p}{:}
            \PYG{n}{dt}\PYG{p}{,} \PYG{n}{start}\PYG{p}{,} \PYG{n}{end}\PYG{p}{,} \PYG{n}{rec} \PYG{o}{=} \PYG{n}{datum}\PYG{o}{.}\PYG{n}{get\PYGZus{}longitude}\PYG{p}{(}\PYG{n}{stats\PYGZus{}dt}\PYG{p}{,} \PYG{n}{stats\PYGZus{}start}\PYG{p}{,} \PYG{n}{stats\PYGZus{}end}\PYG{p}{,} \PYG{n}{record}\PYG{p}{,} \PYG{n}{N}\PYG{o}{=}\PYG{n}{N}\PYG{p}{)}
        \PYG{k}{else}\PYG{p}{:}
            \PYG{n}{dt}\PYG{p}{,} \PYG{n}{start}\PYG{p}{,} \PYG{n}{end}\PYG{p}{,} \PYG{n}{rec} \PYG{o}{=} \PYG{n}{datum}\PYG{o}{.}\PYG{n}{get\PYGZus{}solo}\PYG{p}{(}\PYG{n}{stats\PYGZus{}dt}\PYG{p}{,} \PYG{n}{stats\PYGZus{}start}\PYG{p}{,} \PYG{n}{stats\PYGZus{}end}\PYG{p}{,} \PYG{n}{record}\PYG{p}{)}

        \PYG{n}{datum}\PYG{o}{.}\PYG{n}{save}\PYG{p}{(}\PYG{n}{fpath}\PYG{p}{,} \PYG{n}{record}\PYG{o}{=}\PYG{n}{rec}\PYG{p}{,} \PYG{n}{stats\PYGZus{}dt}\PYG{o}{=}\PYG{n}{dt}\PYG{p}{,} \PYG{n}{stats\PYGZus{}start}\PYG{o}{=}\PYG{n}{start}\PYG{p}{,} \PYG{n}{stats\PYGZus{}end}\PYG{o}{=}\PYG{n}{end}\PYG{p}{)}
\end{sphinxVerbatim}


\subsection{school\_new notebook}
\label{\detokenize{school_new::doc}}\label{\detokenize{school_new:school-new-notebook}}
\fvset{hllines={, ,}}%
\begin{sphinxVerbatim}[commandchars=\\\{\}]
\PYG{c+c1}{\PYGZsh{} The United States Environmental Protection Agency through its Office of}
\PYG{c+c1}{\PYGZsh{} Research and Development has developed this software. The code is made}
\PYG{c+c1}{\PYGZsh{} publicly available to better communicate the research. All input data}
\PYG{c+c1}{\PYGZsh{} used fora given application should be reviewed by the researcher so}
\PYG{c+c1}{\PYGZsh{} that the model results are based on appropriate data for any given}
\PYG{c+c1}{\PYGZsh{} application. This model is under continued development. The model and}
\PYG{c+c1}{\PYGZsh{} data included herein do not represent and should not be construed to}
\PYG{c+c1}{\PYGZsh{} represent any Agency determination or policy.}
\PYG{c+c1}{\PYGZsh{}}
\PYG{c+c1}{\PYGZsh{} This file was written by Dr. Namdi Brandon}
\PYG{c+c1}{\PYGZsh{} ORCID: 0000\PYGZhy{}0001\PYGZhy{}7050\PYGZhy{}1538}
\PYG{c+c1}{\PYGZsh{} March 22, 2018}
\end{sphinxVerbatim}

This file goes through the data from the Consoldiated Human Activity
Database (CHAD) and gets information relevent to ** school** and
processes the data for use in the Agent-Based Model of Human Activity
Patterns (ABMHAP) for the school-age children demographic. More
specficially, this file does the following:

For school-age children demographic,
\begin{enumerate}
\item {} 
This function goes through the CHAD data and finds the school
activity data

\item {} 
The CHAD activity data are seperated into start time, end time,
duration, and CHAD record data

\item {} 
The CHAD activity data is saved into longitudinal data and
single-activity data

\end{enumerate}

import

\fvset{hllines={, ,}}%
\begin{sphinxVerbatim}[commandchars=\\\{\}]
\PYG{k+kn}{import} \PYG{n+nn}{sys}
\PYG{n}{sys}\PYG{o}{.}\PYG{n}{path}\PYG{o}{.}\PYG{n}{append}\PYG{p}{(}\PYG{l+s+s1}{\PYGZsq{}}\PYG{l+s+s1}{..}\PYG{l+s+se}{\PYGZbs{}\PYGZbs{}}\PYG{l+s+s1}{source}\PYG{l+s+s1}{\PYGZsq{}}\PYG{p}{)}

\PYG{c+c1}{\PYGZsh{} ABMHAP modules}
\PYG{k+kn}{import} \PYG{n+nn}{demography} \PYG{k}{as} \PYG{n+nn}{dmg}
\PYG{k+kn}{import} \PYG{n+nn}{datum}
\end{sphinxVerbatim}

load data

\fvset{hllines={, ,}}%
\begin{sphinxVerbatim}[commandchars=\\\{\}]
\PYG{c+c1}{\PYGZsh{}}
\PYG{c+c1}{\PYGZsh{} demographic}
\PYG{c+c1}{\PYGZsh{}}
\PYG{n}{key} \PYG{o}{=} \PYG{n}{dmg}\PYG{o}{.}\PYG{n}{CHILD\PYGZus{}SCHOOL}

\PYG{n}{fname\PYGZus{}input}\PYG{p}{,} \PYG{n}{fpath\PYGZus{}output} \PYG{o}{=} \PYG{n}{dmg}\PYG{o}{.}\PYG{n}{INT\PYGZus{}2\PYGZus{}FIN\PYGZus{}FOUT\PYGZus{}LARGE}\PYG{p}{[}\PYG{n}{key}\PYG{p}{]}

\PYG{c+c1}{\PYGZsh{} load the data}
\PYG{n}{data} \PYG{o}{=} \PYG{n}{dmg}\PYG{o}{.}\PYG{n}{load}\PYG{p}{(}\PYG{n}{fname\PYGZus{}input}\PYG{p}{)}
\end{sphinxVerbatim}

process the data

\fvset{hllines={, ,}}%
\begin{sphinxVerbatim}[commandchars=\\\{\}]
\PYG{c+c1}{\PYGZsh{} dictionaries about the moments}
\PYG{n}{d} \PYG{o}{=} \PYG{n}{datum}\PYG{o}{.}\PYG{n}{analyze\PYGZus{}education}\PYG{p}{(}\PYG{n}{data}\PYG{p}{)}
\end{sphinxVerbatim}

save the data

\fvset{hllines={, ,}}%
\begin{sphinxVerbatim}[commandchars=\\\{\}]
\PYG{c+c1}{\PYGZsh{} choose to save longitudinal data or single\PYGZhy{}day data}
\PYG{n}{chooser} \PYG{o}{=} \PYG{p}{\PYGZob{}}\PYG{k+kc}{True}\PYG{p}{:} \PYG{p}{(}\PYG{l+m+mi}{2}\PYG{p}{,} \PYG{n}{fpath\PYGZus{}output} \PYG{o}{+} \PYG{l+s+s1}{\PYGZsq{}}\PYG{l+s+se}{\PYGZbs{}\PYGZbs{}}\PYG{l+s+s1}{longitude}\PYG{l+s+s1}{\PYGZsq{}}\PYG{p}{)}\PYG{p}{,}
           \PYG{k+kc}{False}\PYG{p}{:} \PYG{p}{(}\PYG{l+m+mi}{1}\PYG{p}{,} \PYG{n}{fpath\PYGZus{}output} \PYG{o}{+} \PYG{l+s+s1}{\PYGZsq{}}\PYG{l+s+se}{\PYGZbs{}\PYGZbs{}}\PYG{l+s+s1}{solo}\PYG{l+s+s1}{\PYGZsq{}}\PYG{p}{)}\PYG{p}{,} \PYG{p}{\PYGZcb{}}

\PYG{c+c1}{\PYGZsh{} whether to save the longitudinal data (if True) or the single\PYGZhy{}day data (if False)}
\PYG{n}{do\PYGZus{}long} \PYG{o}{=} \PYG{k+kc}{True}
\end{sphinxVerbatim}

\fvset{hllines={, ,}}%
\begin{sphinxVerbatim}[commandchars=\\\{\}]
\PYG{c+c1}{\PYGZsh{}}
\PYG{c+c1}{\PYGZsh{} save the data}
\PYG{c+c1}{\PYGZsh{}}
\PYG{n}{do\PYGZus{}save} \PYG{o}{=} \PYG{k+kc}{False}

\PYG{k}{if} \PYG{n}{do\PYGZus{}save}\PYG{p}{:}

    \PYG{n}{N}\PYG{p}{,} \PYG{n}{fpath} \PYG{o}{=} \PYG{n}{chooser}\PYG{p}{[}\PYG{n}{do\PYGZus{}long}\PYG{p}{]}

    \PYG{c+c1}{\PYGZsh{} the directory the data should be saved in}
    \PYG{n}{fpath} \PYG{o}{=} \PYG{n}{fpath} \PYG{o}{+} \PYG{l+s+s1}{\PYGZsq{}}\PYG{l+s+se}{\PYGZbs{}\PYGZbs{}}\PYG{l+s+s1}{education}\PYG{l+s+s1}{\PYGZsq{}}

    \PYG{c+c1}{\PYGZsh{} save the data}
    \PYG{n}{stats\PYGZus{}dt}\PYG{p}{,} \PYG{n}{stats\PYGZus{}start}\PYG{p}{,} \PYG{n}{stats\PYGZus{}end}\PYG{p}{,} \PYG{n}{record} \PYG{o}{=} \PYG{n}{d}\PYG{p}{[}\PYG{l+s+s1}{\PYGZsq{}}\PYG{l+s+s1}{stats\PYGZus{}dt}\PYG{l+s+s1}{\PYGZsq{}}\PYG{p}{]}\PYG{p}{,} \PYG{n}{d}\PYG{p}{[}\PYG{l+s+s1}{\PYGZsq{}}\PYG{l+s+s1}{stats\PYGZus{}start}\PYG{l+s+s1}{\PYGZsq{}}\PYG{p}{]}\PYG{p}{,} \PYG{n}{d}\PYG{p}{[}\PYG{l+s+s1}{\PYGZsq{}}\PYG{l+s+s1}{stats\PYGZus{}end}\PYG{l+s+s1}{\PYGZsq{}}\PYG{p}{]}\PYG{p}{,} \PYG{n}{d}\PYG{p}{[}\PYG{l+s+s1}{\PYGZsq{}}\PYG{l+s+s1}{data}\PYG{l+s+s1}{\PYGZsq{}}\PYG{p}{]}

    \PYG{k}{if} \PYG{n}{do\PYGZus{}long}\PYG{p}{:}
        \PYG{n}{dt}\PYG{p}{,} \PYG{n}{start}\PYG{p}{,} \PYG{n}{end}\PYG{p}{,} \PYG{n}{rec} \PYG{o}{=} \PYG{n}{datum}\PYG{o}{.}\PYG{n}{get\PYGZus{}longitude}\PYG{p}{(}\PYG{n}{stats\PYGZus{}dt}\PYG{p}{,} \PYG{n}{stats\PYGZus{}start}\PYG{p}{,} \PYG{n}{stats\PYGZus{}end}\PYG{p}{,} \PYG{n}{record}\PYG{p}{,} \PYG{n}{N}\PYG{o}{=}\PYG{n}{N}\PYG{p}{)}
    \PYG{k}{else}\PYG{p}{:}
        \PYG{n}{dt}\PYG{p}{,} \PYG{n}{start}\PYG{p}{,} \PYG{n}{end}\PYG{p}{,} \PYG{n}{rec} \PYG{o}{=} \PYG{n}{datum}\PYG{o}{.}\PYG{n}{get\PYGZus{}solo}\PYG{p}{(}\PYG{n}{stats\PYGZus{}dt}\PYG{p}{,} \PYG{n}{stats\PYGZus{}start}\PYG{p}{,} \PYG{n}{stats\PYGZus{}end}\PYG{p}{,} \PYG{n}{record}\PYG{p}{)}

    \PYG{n}{datum}\PYG{o}{.}\PYG{n}{save}\PYG{p}{(}\PYG{n}{fpath}\PYG{p}{,} \PYG{n}{record}\PYG{o}{=}\PYG{n}{rec}\PYG{p}{,} \PYG{n}{stats\PYGZus{}dt}\PYG{o}{=}\PYG{n}{dt}\PYG{p}{,} \PYG{n}{stats\PYGZus{}start}\PYG{o}{=}\PYG{n}{start}\PYG{p}{,} \PYG{n}{stats\PYGZus{}end}\PYG{o}{=}\PYG{n}{end}\PYG{p}{)}
\end{sphinxVerbatim}


\subsection{sleep\_new notebook}
\label{\detokenize{sleep_new::doc}}\label{\detokenize{sleep_new:sleep-new-notebook}}
\fvset{hllines={, ,}}%
\begin{sphinxVerbatim}[commandchars=\\\{\}]
\PYG{c+c1}{\PYGZsh{} The United States Environmental Protection Agency through its Office of}
\PYG{c+c1}{\PYGZsh{} Research and Development has developed this software. The code is made}
\PYG{c+c1}{\PYGZsh{} publicly available to better communicate the research. All input data}
\PYG{c+c1}{\PYGZsh{} used fora given application should be reviewed by the researcher so}
\PYG{c+c1}{\PYGZsh{} that the model results are based on appropriate data for any given}
\PYG{c+c1}{\PYGZsh{} application. This model is under continued development. The model and}
\PYG{c+c1}{\PYGZsh{} data included herein do not represent and should not be construed to}
\PYG{c+c1}{\PYGZsh{} represent any Agency determination or policy.}
\PYG{c+c1}{\PYGZsh{}}
\PYG{c+c1}{\PYGZsh{} This file was written by Dr. Namdi Brandon}
\PYG{c+c1}{\PYGZsh{} ORCID: 0000\PYGZhy{}0001\PYGZhy{}7050\PYGZhy{}1538}
\PYG{c+c1}{\PYGZsh{} March 22, 2018}
\end{sphinxVerbatim}

This file goes through the data from the Consoldiated Human Activity
Database (CHAD) and gets information relevent to \sphinxstylestrong{sleeping} and
processes the data for use in the Agent-Based Model of Human Activity
Patterns (ABMHAP) for each demographic. More specficially, this file
does the following:

For a given demographic,
\begin{enumerate}
\item {} 
This function goes through the CHAD data and finds the sleep-activity
data

\item {} 
The CHAD activity data are seperated into start time, end time,
duration, and CHAD record data

\item {} 
The CHAD activity data is saved into longitudinal data and
single-activity data

\end{enumerate}

Import

\fvset{hllines={, ,}}%
\begin{sphinxVerbatim}[commandchars=\\\{\}]
\PYG{k+kn}{import} \PYG{n+nn}{sys}
\PYG{n}{sys}\PYG{o}{.}\PYG{n}{path}\PYG{o}{.}\PYG{n}{append}\PYG{p}{(}\PYG{l+s+s1}{\PYGZsq{}}\PYG{l+s+s1}{..}\PYG{l+s+se}{\PYGZbs{}\PYGZbs{}}\PYG{l+s+s1}{source}\PYG{l+s+s1}{\PYGZsq{}}\PYG{p}{)}

\PYG{c+c1}{\PYGZsh{} plotting capability}
\PYG{k+kn}{import} \PYG{n+nn}{matplotlib}\PYG{n+nn}{.}\PYG{n+nn}{pylab} \PYG{k}{as} \PYG{n+nn}{plt}

\PYG{c+c1}{\PYGZsh{} ABMHAP modules}
\PYG{k+kn}{import} \PYG{n+nn}{demography} \PYG{k}{as} \PYG{n+nn}{dmg}
\PYG{k+kn}{import} \PYG{n+nn}{my\PYGZus{}globals} \PYG{k}{as} \PYG{n+nn}{mg}
\PYG{k+kn}{import} \PYG{n+nn}{datum}
\end{sphinxVerbatim}

\fvset{hllines={, ,}}%
\begin{sphinxVerbatim}[commandchars=\\\{\}]
\PYG{o}{\PYGZpc{}}\PYG{k}{matplotlib} notebook
\end{sphinxVerbatim}

Load

\fvset{hllines={, ,}}%
\begin{sphinxVerbatim}[commandchars=\\\{\}]
\PYG{c+c1}{\PYGZsh{}}
\PYG{c+c1}{\PYGZsh{} demographic}
\PYG{c+c1}{\PYGZsh{}}
\PYG{n}{demo} \PYG{o}{=} \PYG{n}{dmg}\PYG{o}{.}\PYG{n}{CHILD\PYGZus{}YOUNG}

\PYG{c+c1}{\PYGZsh{} the input file and output file directory}
\PYG{n}{fname\PYGZus{}input}\PYG{p}{,} \PYG{n}{fpath\PYGZus{}output} \PYG{o}{=} \PYG{n}{dmg}\PYG{o}{.}\PYG{n}{INT\PYGZus{}2\PYGZus{}FIN\PYGZus{}FOUT\PYGZus{}LARGE}\PYG{p}{[}\PYG{n}{key}\PYG{p}{]}

\PYG{c+c1}{\PYGZsh{} load the data}
\PYG{n}{data} \PYG{o}{=} \PYG{n}{dmg}\PYG{o}{.}\PYG{n}{load}\PYG{p}{(}\PYG{n}{fname\PYGZus{}input}\PYG{p}{)}
\end{sphinxVerbatim}

Process data

\fvset{hllines={, ,}}%
\begin{sphinxVerbatim}[commandchars=\\\{\}]
\PYG{c+c1}{\PYGZsh{} analyze the data}
\PYG{n}{d\PYGZus{}slumber} \PYG{o}{=} \PYG{n}{datum}\PYG{o}{.}\PYG{n}{analyze\PYGZus{}sleep}\PYG{p}{(}\PYG{n}{data}\PYG{p}{)}
\end{sphinxVerbatim}

\fvset{hllines={, ,}}%
\begin{sphinxVerbatim}[commandchars=\\\{\}]
\PYG{c+c1}{\PYGZsh{} get the statistical data}
\PYG{n}{d} \PYG{o}{=} \PYG{n}{d\PYGZus{}slumber}

\PYG{n}{slumber}\PYG{p}{,} \PYG{n}{stats\PYGZus{}dt}\PYG{p}{,} \PYG{n}{stats\PYGZus{}start}\PYG{p}{,} \PYG{n}{stats\PYGZus{}end} \PYG{o}{=} \PYG{n}{d}\PYG{p}{[}\PYG{l+s+s1}{\PYGZsq{}}\PYG{l+s+s1}{data}\PYG{l+s+s1}{\PYGZsq{}}\PYG{p}{]}\PYG{p}{,} \PYG{n}{d}\PYG{p}{[}\PYG{l+s+s1}{\PYGZsq{}}\PYG{l+s+s1}{stats\PYGZus{}dt}\PYG{l+s+s1}{\PYGZsq{}}\PYG{p}{]}\PYG{p}{,} \PYG{n}{d}\PYG{p}{[}\PYG{l+s+s1}{\PYGZsq{}}\PYG{l+s+s1}{stats\PYGZus{}start}\PYG{l+s+s1}{\PYGZsq{}}\PYG{p}{]}\PYG{p}{,} \PYG{n}{d}\PYG{p}{[}\PYG{l+s+s1}{\PYGZsq{}}\PYG{l+s+s1}{stats\PYGZus{}end}\PYG{l+s+s1}{\PYGZsq{}}\PYG{p}{]}

\PYG{n}{slumber\PYGZus{}we}\PYG{p}{,} \PYG{n}{stats\PYGZus{}we\PYGZus{}dt}\PYG{p}{,} \PYG{n}{stats\PYGZus{}we\PYGZus{}start}\PYG{p}{,} \PYG{n}{stats\PYGZus{}we\PYGZus{}end} \PYG{o}{=} \PYGZbs{}
\PYG{n}{d}\PYG{p}{[}\PYG{l+s+s1}{\PYGZsq{}}\PYG{l+s+s1}{data\PYGZus{}weekend}\PYG{l+s+s1}{\PYGZsq{}}\PYG{p}{]}\PYG{p}{,} \PYG{n}{d}\PYG{p}{[}\PYG{l+s+s1}{\PYGZsq{}}\PYG{l+s+s1}{stats\PYGZus{}we\PYGZus{}dt}\PYG{l+s+s1}{\PYGZsq{}}\PYG{p}{]}\PYG{p}{,} \PYG{n}{d}\PYG{p}{[}\PYG{l+s+s1}{\PYGZsq{}}\PYG{l+s+s1}{stats\PYGZus{}we\PYGZus{}start}\PYG{l+s+s1}{\PYGZsq{}}\PYG{p}{]}\PYG{p}{,} \PYG{n}{d}\PYG{p}{[}\PYG{l+s+s1}{\PYGZsq{}}\PYG{l+s+s1}{stats\PYGZus{}we\PYGZus{}end}\PYG{l+s+s1}{\PYGZsq{}}\PYG{p}{]}

\PYG{n}{slumber\PYGZus{}wd}\PYG{p}{,} \PYG{n}{stats\PYGZus{}wd\PYGZus{}dt}\PYG{p}{,} \PYG{n}{stats\PYGZus{}wd\PYGZus{}start}\PYG{p}{,} \PYG{n}{stats\PYGZus{}wd\PYGZus{}end} \PYG{o}{=} \PYGZbs{}
\PYG{n}{d}\PYG{p}{[}\PYG{l+s+s1}{\PYGZsq{}}\PYG{l+s+s1}{data\PYGZus{}weekday}\PYG{l+s+s1}{\PYGZsq{}}\PYG{p}{]}\PYG{p}{,} \PYG{n}{d}\PYG{p}{[}\PYG{l+s+s1}{\PYGZsq{}}\PYG{l+s+s1}{stats\PYGZus{}wd\PYGZus{}dt}\PYG{l+s+s1}{\PYGZsq{}}\PYG{p}{]}\PYG{p}{,} \PYG{n}{d}\PYG{p}{[}\PYG{l+s+s1}{\PYGZsq{}}\PYG{l+s+s1}{stats\PYGZus{}wd\PYGZus{}start}\PYG{l+s+s1}{\PYGZsq{}}\PYG{p}{]}\PYG{p}{,} \PYG{n}{d}\PYG{p}{[}\PYG{l+s+s1}{\PYGZsq{}}\PYG{l+s+s1}{stats\PYGZus{}wd\PYGZus{}end}\PYG{l+s+s1}{\PYGZsq{}}\PYG{p}{]}
\end{sphinxVerbatim}

save the data

\fvset{hllines={, ,}}%
\begin{sphinxVerbatim}[commandchars=\\\{\}]
\PYG{c+c1}{\PYGZsh{} the minimum number of activity entries per individual to be considered longitudinal}
\PYG{n}{N\PYGZus{}long} \PYG{o}{=} \PYG{l+m+mi}{2}

\PYG{c+c1}{\PYGZsh{} there is not much longitudinal information of pre\PYGZhy{}school children}
\PYG{k}{if} \PYG{n}{demo} \PYG{o+ow}{in} \PYG{p}{[}\PYG{n}{dmg}\PYG{o}{.}\PYG{n}{CHILD\PYGZus{}YOUNG}\PYG{p}{]}\PYG{p}{:}
    \PYG{n}{N\PYGZus{}long} \PYG{o}{=} \PYG{l+m+mi}{1}

\PYG{c+c1}{\PYGZsh{} choose to save longitudinal data or single\PYGZhy{}day data}
\PYG{n}{chooser} \PYG{o}{=} \PYG{p}{\PYGZob{}}\PYG{k+kc}{True}\PYG{p}{:} \PYG{p}{(}\PYG{n}{N\PYGZus{}long}\PYG{p}{,} \PYG{n}{fpath\PYGZus{}output} \PYG{o}{+} \PYG{l+s+s1}{\PYGZsq{}}\PYG{l+s+se}{\PYGZbs{}\PYGZbs{}}\PYG{l+s+s1}{longitude}\PYG{l+s+s1}{\PYGZsq{}}\PYG{p}{)}\PYG{p}{,}
           \PYG{k+kc}{False}\PYG{p}{:} \PYG{p}{(}\PYG{l+m+mi}{1}\PYG{p}{,} \PYG{n}{fpath\PYGZus{}output} \PYG{o}{+} \PYG{l+s+s1}{\PYGZsq{}}\PYG{l+s+se}{\PYGZbs{}\PYGZbs{}}\PYG{l+s+s1}{solo}\PYG{l+s+s1}{\PYGZsq{}}\PYG{p}{)}\PYG{p}{,} \PYG{p}{\PYGZcb{}}

\PYG{c+c1}{\PYGZsh{} whether to save the longitudinal data (if True) or the single\PYGZhy{}day data (if False)}
\PYG{n}{do\PYGZus{}long} \PYG{o}{=} \PYG{k+kc}{True}
\end{sphinxVerbatim}

\fvset{hllines={, ,}}%
\begin{sphinxVerbatim}[commandchars=\\\{\}]
\PYG{c+c1}{\PYGZsh{} save the and solo data}
\PYG{n}{do\PYGZus{}save} \PYG{o}{=} \PYG{k+kc}{False}

\PYG{k}{if} \PYG{n}{do\PYGZus{}save}\PYG{p}{:}

    \PYG{n}{N}\PYG{p}{,} \PYG{n}{fpath} \PYG{o}{=} \PYG{n}{chooser}\PYG{p}{[}\PYG{n}{do\PYGZus{}long}\PYG{p}{]}

    \PYG{k}{if} \PYG{n}{do\PYGZus{}long}\PYG{p}{:}
        \PYG{n}{data\PYGZus{}all} \PYG{o}{=} \PYG{n}{datum}\PYG{o}{.}\PYG{n}{get\PYGZus{}longitude}\PYG{p}{(}\PYG{n}{stats\PYGZus{}dt}\PYG{p}{,} \PYG{n}{stats\PYGZus{}start}\PYG{p}{,} \PYG{n}{stats\PYGZus{}end}\PYG{p}{,} \PYG{n}{slumber}\PYG{p}{,} \PYG{n}{N}\PYG{o}{=}\PYG{n}{N}\PYG{p}{)}
        \PYG{n}{data\PYGZus{}weekend} \PYG{o}{=} \PYG{n}{datum}\PYG{o}{.}\PYG{n}{get\PYGZus{}longitude}\PYG{p}{(}\PYG{n}{stats\PYGZus{}we\PYGZus{}dt}\PYG{p}{,} \PYG{n}{stats\PYGZus{}we\PYGZus{}start}\PYG{p}{,} \PYG{n}{stats\PYGZus{}we\PYGZus{}end}\PYG{p}{,} \PYG{n}{slumber\PYGZus{}we}\PYG{p}{,} \PYG{n}{N}\PYG{o}{=}\PYG{n}{N}\PYG{p}{)}
        \PYG{n}{data\PYGZus{}weekday} \PYG{o}{=} \PYG{n}{datum}\PYG{o}{.}\PYG{n}{get\PYGZus{}longitude}\PYG{p}{(}\PYG{n}{stats\PYGZus{}wd\PYGZus{}dt}\PYG{p}{,} \PYG{n}{stats\PYGZus{}wd\PYGZus{}start}\PYG{p}{,} \PYG{n}{stats\PYGZus{}wd\PYGZus{}end}\PYG{p}{,} \PYG{n}{slumber\PYGZus{}wd}\PYG{p}{,} \PYG{n}{N}\PYG{o}{=}\PYG{n}{N}\PYG{p}{)}
    \PYG{k}{else}\PYG{p}{:}
        \PYG{n}{data\PYGZus{}all} \PYG{o}{=} \PYG{n}{datum}\PYG{o}{.}\PYG{n}{get\PYGZus{}solo}\PYG{p}{(}\PYG{n}{stats\PYGZus{}dt}\PYG{p}{,} \PYG{n}{stats\PYGZus{}start}\PYG{p}{,} \PYG{n}{stats\PYGZus{}end}\PYG{p}{,} \PYG{n}{slumber}\PYG{p}{)}
        \PYG{n}{data\PYGZus{}weekend} \PYG{o}{=} \PYG{n}{datum}\PYG{o}{.}\PYG{n}{get\PYGZus{}solo}\PYG{p}{(}\PYG{n}{stats\PYGZus{}we\PYGZus{}dt}\PYG{p}{,} \PYG{n}{stats\PYGZus{}we\PYGZus{}start}\PYG{p}{,} \PYG{n}{stats\PYGZus{}we\PYGZus{}end}\PYG{p}{,} \PYG{n}{slumber\PYGZus{}we}\PYG{p}{)}
        \PYG{n}{data\PYGZus{}weekday} \PYG{o}{=} \PYG{n}{datum}\PYG{o}{.}\PYG{n}{get\PYGZus{}solo}\PYG{p}{(}\PYG{n}{stats\PYGZus{}wd\PYGZus{}dt}\PYG{p}{,} \PYG{n}{stats\PYGZus{}wd\PYGZus{}start}\PYG{p}{,} \PYG{n}{stats\PYGZus{}wd\PYGZus{}end}\PYG{p}{,} \PYG{n}{slumber\PYGZus{}wd}\PYG{p}{)}

    \PYG{c+c1}{\PYGZsh{} the directories the data should be saved in}
    \PYG{n}{fpath} \PYG{o}{=} \PYG{n}{fpath} \PYG{o}{+} \PYG{l+s+s1}{\PYGZsq{}}\PYG{l+s+se}{\PYGZbs{}\PYGZbs{}}\PYG{l+s+s1}{sleep}\PYG{l+s+s1}{\PYGZsq{}}
    \PYG{n}{fpaths} \PYG{o}{=} \PYG{p}{[} \PYG{n}{fpath} \PYG{o}{+} \PYG{l+s+s1}{\PYGZsq{}}\PYG{l+s+se}{\PYGZbs{}\PYGZbs{}}\PYG{l+s+s1}{all}\PYG{l+s+s1}{\PYGZsq{}}\PYG{p}{,} \PYG{n}{fpath} \PYG{o}{+} \PYG{l+s+s1}{\PYGZsq{}}\PYG{l+s+se}{\PYGZbs{}\PYGZbs{}}\PYG{l+s+s1}{non\PYGZus{}workday}\PYG{l+s+s1}{\PYGZsq{}}\PYG{p}{,} \PYG{n}{fpath} \PYG{o}{+} \PYG{l+s+s1}{\PYGZsq{}}\PYG{l+s+se}{\PYGZbs{}\PYGZbs{}}\PYG{l+s+s1}{workday}\PYG{l+s+s1}{\PYGZsq{}} \PYG{p}{]}

    \PYG{c+c1}{\PYGZsh{} the dictionaries holding the data}
    \PYG{n}{data\PYGZus{}list} \PYG{o}{=} \PYG{p}{[}\PYG{n}{data\PYGZus{}all}\PYG{p}{,} \PYG{n}{data\PYGZus{}weekend}\PYG{p}{,} \PYG{n}{data\PYGZus{}weekday}\PYG{p}{]}

    \PYG{c+c1}{\PYGZsh{} save the data}
    \PYG{k}{for} \PYG{n}{fpath}\PYG{p}{,} \PYG{n}{d} \PYG{o+ow}{in} \PYG{n+nb}{zip}\PYG{p}{(}\PYG{n}{fpaths}\PYG{p}{,} \PYG{n}{data\PYGZus{}list}\PYG{p}{)}\PYG{p}{:}

        \PYG{n}{stats\PYGZus{}dt}\PYG{p}{,} \PYG{n}{stats\PYGZus{}start}\PYG{p}{,} \PYG{n}{stats\PYGZus{}end}\PYG{p}{,} \PYG{n}{record} \PYG{o}{=} \PYG{n}{d}
        \PYG{n}{datum}\PYG{o}{.}\PYG{n}{save}\PYG{p}{(}\PYG{n}{fpath}\PYG{p}{,} \PYG{n}{record}\PYG{o}{=}\PYG{n}{record}\PYG{p}{,} \PYG{n}{stats\PYGZus{}dt}\PYG{o}{=}\PYG{n}{stats\PYGZus{}dt}\PYG{p}{,} \PYG{n}{stats\PYGZus{}start}\PYG{o}{=}\PYG{n}{stats\PYGZus{}start}\PYG{p}{,} \PYG{n}{stats\PYGZus{}end}\PYG{o}{=}\PYG{n}{stats\PYGZus{}end}\PYG{p}{)}
\end{sphinxVerbatim}


\chapter{Indices and tables}
\label{\detokenize{index:indices-and-tables}}\begin{itemize}
\item {} 
\DUrole{xref,std,std-ref}{genindex}

\item {} 
\DUrole{xref,std,std-ref}{modindex}

\item {} 
\DUrole{xref,std,std-ref}{search}

\end{itemize}


\renewcommand{\indexname}{Python Module Index}
\begin{sphinxtheindex}
\def\bigletter#1{{\Large\sffamily#1}\nopagebreak\vspace{1mm}}
\bigletter{a}
\item {\sphinxstyleindexentry{activity}}\sphinxstyleindexpageref{activity:\detokenize{module-activity}}
\item {\sphinxstyleindexentry{analysis}}\sphinxstyleindexpageref{analysis:\detokenize{module-analysis}}
\item {\sphinxstyleindexentry{analyzer}}\sphinxstyleindexpageref{analyzer:\detokenize{module-analyzer}}
\item {\sphinxstyleindexentry{asset}}\sphinxstyleindexpageref{asset:\detokenize{module-asset}}
\indexspace
\bigletter{b}
\item {\sphinxstyleindexentry{bed}}\sphinxstyleindexpageref{bed:\detokenize{module-bed}}
\item {\sphinxstyleindexentry{bio}}\sphinxstyleindexpageref{bio:\detokenize{module-bio}}
\indexspace
\bigletter{c}
\item {\sphinxstyleindexentry{chad}}\sphinxstyleindexpageref{chad:\detokenize{module-chad}}
\item {\sphinxstyleindexentry{chad\_code}}\sphinxstyleindexpageref{chad_code:\detokenize{module-chad_code}}
\item {\sphinxstyleindexentry{chad\_demography}}\sphinxstyleindexpageref{chad_demography:\detokenize{module-chad_demography}}
\item {\sphinxstyleindexentry{chad\_demography\_adult\_non\_work}}\sphinxstyleindexpageref{chad_demography_adult_non_work:\detokenize{module-chad_demography_adult_non_work}}
\item {\sphinxstyleindexentry{chad\_demography\_adult\_work}}\sphinxstyleindexpageref{chad_demography_adult_work:\detokenize{module-chad_demography_adult_work}}
\item {\sphinxstyleindexentry{chad\_demography\_child\_school}}\sphinxstyleindexpageref{chad_demography_child_school:\detokenize{module-chad_demography_child_school}}
\item {\sphinxstyleindexentry{chad\_demography\_child\_young}}\sphinxstyleindexpageref{chad_demography_child_young:\detokenize{module-chad_demography_child_young}}
\item {\sphinxstyleindexentry{chad\_params}}\sphinxstyleindexpageref{chad_params:\detokenize{module-chad_params}}
\item {\sphinxstyleindexentry{commute}}\sphinxstyleindexpageref{commute:\detokenize{module-commute}}
\item {\sphinxstyleindexentry{commute\_from\_work\_trial}}\sphinxstyleindexpageref{commute_from_work_trial:\detokenize{module-commute_from_work_trial}}
\item {\sphinxstyleindexentry{commute\_to\_work\_trial}}\sphinxstyleindexpageref{commute_to_work_trial:\detokenize{module-commute_to_work_trial}}
\indexspace
\bigletter{d}
\item {\sphinxstyleindexentry{datum}}\sphinxstyleindexpageref{datum:\detokenize{module-datum}}
\item {\sphinxstyleindexentry{demography}}\sphinxstyleindexpageref{demography:\detokenize{module-demography}}
\item {\sphinxstyleindexentry{diary}}\sphinxstyleindexpageref{diary:\detokenize{module-diary}}
\item {\sphinxstyleindexentry{driver}}\sphinxstyleindexpageref{driver:\detokenize{module-driver}}
\item {\sphinxstyleindexentry{driver\_params}}\sphinxstyleindexpageref{driver_params:\detokenize{module-driver_params}}
\item {\sphinxstyleindexentry{driver\_result}}\sphinxstyleindexpageref{driver_result:\detokenize{module-driver_result}}
\indexspace
\bigletter{e}
\item {\sphinxstyleindexentry{eat}}\sphinxstyleindexpageref{eat:\detokenize{module-eat}}
\item {\sphinxstyleindexentry{eat\_breakfast\_trial}}\sphinxstyleindexpageref{eat_breakfast_trial:\detokenize{module-eat_breakfast_trial}}
\item {\sphinxstyleindexentry{eat\_dinner\_trial}}\sphinxstyleindexpageref{eat_dinner_trial:\detokenize{module-eat_dinner_trial}}
\item {\sphinxstyleindexentry{eat\_lunch\_trial}}\sphinxstyleindexpageref{eat_lunch_trial:\detokenize{module-eat_lunch_trial}}
\item {\sphinxstyleindexentry{evaluation}}\sphinxstyleindexpageref{evaluation:\detokenize{module-evaluation}}
\indexspace
\bigletter{f}
\item {\sphinxstyleindexentry{fig\_driver}}\sphinxstyleindexpageref{fig_driver:\detokenize{module-fig_driver}}
\item {\sphinxstyleindexentry{food}}\sphinxstyleindexpageref{food:\detokenize{module-food}}
\indexspace
\bigletter{h}
\item {\sphinxstyleindexentry{home}}\sphinxstyleindexpageref{home:\detokenize{module-home}}
\item {\sphinxstyleindexentry{hunger}}\sphinxstyleindexpageref{hunger:\detokenize{module-hunger}}
\indexspace
\bigletter{i}
\item {\sphinxstyleindexentry{income}}\sphinxstyleindexpageref{income:\detokenize{module-income}}
\item {\sphinxstyleindexentry{interrupt}}\sphinxstyleindexpageref{interrupt:\detokenize{module-interrupt}}
\item {\sphinxstyleindexentry{interruption}}\sphinxstyleindexpageref{interruption:\detokenize{module-interruption}}
\indexspace
\bigletter{l}
\item {\sphinxstyleindexentry{location}}\sphinxstyleindexpageref{location:\detokenize{module-location}}
\indexspace
\bigletter{m}
\item {\sphinxstyleindexentry{main}}\sphinxstyleindexpageref{main:\detokenize{module-main}}
\item {\sphinxstyleindexentry{main\_params}}\sphinxstyleindexpageref{main_params:\detokenize{module-main_params}}
\item {\sphinxstyleindexentry{meal}}\sphinxstyleindexpageref{meal:\detokenize{module-meal}}
\item {\sphinxstyleindexentry{my\_debug}}\sphinxstyleindexpageref{my_debug:\detokenize{module-my_debug}}
\item {\sphinxstyleindexentry{my\_globals}}\sphinxstyleindexpageref{my_globals:\detokenize{module-my_globals}}
\indexspace
\bigletter{n}
\item {\sphinxstyleindexentry{need}}\sphinxstyleindexpageref{need:\detokenize{module-need}}
\indexspace
\bigletter{o}
\item {\sphinxstyleindexentry{occupation}}\sphinxstyleindexpageref{occupation:\detokenize{module-occupation}}
\item {\sphinxstyleindexentry{omni\_trial}}\sphinxstyleindexpageref{omni_trial:\detokenize{module-omni_trial}}
\indexspace
\bigletter{p}
\item {\sphinxstyleindexentry{params}}\sphinxstyleindexpageref{params:\detokenize{module-params}}
\item {\sphinxstyleindexentry{person}}\sphinxstyleindexpageref{person:\detokenize{module-person}}
\item {\sphinxstyleindexentry{plotter}}\sphinxstyleindexpageref{plotter:\detokenize{module-plotter}}
\indexspace
\bigletter{r}
\item {\sphinxstyleindexentry{rest}}\sphinxstyleindexpageref{rest:\detokenize{module-rest}}
\indexspace
\bigletter{s}
\item {\sphinxstyleindexentry{scenario}}\sphinxstyleindexpageref{scenario:\detokenize{module-scenario}}
\item {\sphinxstyleindexentry{scheduler}}\sphinxstyleindexpageref{scheduler:\detokenize{module-scheduler}}
\item {\sphinxstyleindexentry{singleton}}\sphinxstyleindexpageref{singleton:\detokenize{module-singleton}}
\item {\sphinxstyleindexentry{sleep}}\sphinxstyleindexpageref{sleep:\detokenize{module-sleep}}
\item {\sphinxstyleindexentry{sleep\_trial}}\sphinxstyleindexpageref{sleep_trial:\detokenize{module-sleep_trial}}
\item {\sphinxstyleindexentry{social}}\sphinxstyleindexpageref{social:\detokenize{module-social}}
\item {\sphinxstyleindexentry{state}}\sphinxstyleindexpageref{state:\detokenize{module-state}}
\indexspace
\bigletter{t}
\item {\sphinxstyleindexentry{temporal}}\sphinxstyleindexpageref{temporal:\detokenize{module-temporal}}
\item {\sphinxstyleindexentry{transport}}\sphinxstyleindexpageref{transport:\detokenize{module-transport}}
\item {\sphinxstyleindexentry{travel}}\sphinxstyleindexpageref{travel:\detokenize{module-travel}}
\item {\sphinxstyleindexentry{trial}}\sphinxstyleindexpageref{trial:\detokenize{module-trial}}
\indexspace
\bigletter{u}
\item {\sphinxstyleindexentry{universe}}\sphinxstyleindexpageref{universe:\detokenize{module-universe}}
\indexspace
\bigletter{v}
\item {\sphinxstyleindexentry{variation}}\sphinxstyleindexpageref{variation:\detokenize{module-variation}}
\indexspace
\bigletter{w}
\item {\sphinxstyleindexentry{work}}\sphinxstyleindexpageref{work:\detokenize{module-work}}
\item {\sphinxstyleindexentry{work\_trial}}\sphinxstyleindexpageref{work_trial:\detokenize{module-work_trial}}
\item {\sphinxstyleindexentry{workplace}}\sphinxstyleindexpageref{workplace:\detokenize{module-workplace}}
\end{sphinxtheindex}

\renewcommand{\indexname}{Index}
\printindex
\end{document}