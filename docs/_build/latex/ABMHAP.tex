% Generated by Sphinx.
\def\sphinxdocclass{report}
\documentclass[letterpaper,10pt,english]{sphinxmanual}
\usepackage[utf8]{inputenc}
\DeclareUnicodeCharacter{00A0}{\nobreakspace}
\usepackage{cmap}
\usepackage[T1]{fontenc}
\usepackage{babel}
\usepackage{times}
\usepackage[Sonny]{fncychap}
\usepackage{longtable}
\usepackage{sphinx}
\usepackage{multirow}

\addto\captionsenglish{\renewcommand{\figurename}{Fig. }}
\addto\captionsenglish{\renewcommand{\tablename}{Table }}
\floatname{literal-block}{Listing }



\title{The Agent-Based Model of Human Activity Patterns   (ABMHAP): Documentation and Users Guide}
\date{August 14, 2017}
\release{2017.08}
\author{Namdi Brandon}
\newcommand{\sphinxlogo}{}
\renewcommand{\releasename}{Release}
\makeindex

\makeatletter
\def\PYG@reset{\let\PYG@it=\relax \let\PYG@bf=\relax%
    \let\PYG@ul=\relax \let\PYG@tc=\relax%
    \let\PYG@bc=\relax \let\PYG@ff=\relax}
\def\PYG@tok#1{\csname PYG@tok@#1\endcsname}
\def\PYG@toks#1+{\ifx\relax#1\empty\else%
    \PYG@tok{#1}\expandafter\PYG@toks\fi}
\def\PYG@do#1{\PYG@bc{\PYG@tc{\PYG@ul{%
    \PYG@it{\PYG@bf{\PYG@ff{#1}}}}}}}
\def\PYG#1#2{\PYG@reset\PYG@toks#1+\relax+\PYG@do{#2}}

\expandafter\def\csname PYG@tok@s\endcsname{\def\PYG@tc##1{\textcolor[rgb]{0.25,0.44,0.63}{##1}}}
\expandafter\def\csname PYG@tok@gd\endcsname{\def\PYG@tc##1{\textcolor[rgb]{0.63,0.00,0.00}{##1}}}
\expandafter\def\csname PYG@tok@ge\endcsname{\let\PYG@it=\textit}
\expandafter\def\csname PYG@tok@nn\endcsname{\let\PYG@bf=\textbf\def\PYG@tc##1{\textcolor[rgb]{0.05,0.52,0.71}{##1}}}
\expandafter\def\csname PYG@tok@cpf\endcsname{\let\PYG@it=\textit\def\PYG@tc##1{\textcolor[rgb]{0.25,0.50,0.56}{##1}}}
\expandafter\def\csname PYG@tok@sr\endcsname{\def\PYG@tc##1{\textcolor[rgb]{0.14,0.33,0.53}{##1}}}
\expandafter\def\csname PYG@tok@si\endcsname{\let\PYG@it=\textit\def\PYG@tc##1{\textcolor[rgb]{0.44,0.63,0.82}{##1}}}
\expandafter\def\csname PYG@tok@cm\endcsname{\let\PYG@it=\textit\def\PYG@tc##1{\textcolor[rgb]{0.25,0.50,0.56}{##1}}}
\expandafter\def\csname PYG@tok@c1\endcsname{\let\PYG@it=\textit\def\PYG@tc##1{\textcolor[rgb]{0.25,0.50,0.56}{##1}}}
\expandafter\def\csname PYG@tok@ne\endcsname{\def\PYG@tc##1{\textcolor[rgb]{0.00,0.44,0.13}{##1}}}
\expandafter\def\csname PYG@tok@s2\endcsname{\def\PYG@tc##1{\textcolor[rgb]{0.25,0.44,0.63}{##1}}}
\expandafter\def\csname PYG@tok@kd\endcsname{\let\PYG@bf=\textbf\def\PYG@tc##1{\textcolor[rgb]{0.00,0.44,0.13}{##1}}}
\expandafter\def\csname PYG@tok@go\endcsname{\def\PYG@tc##1{\textcolor[rgb]{0.20,0.20,0.20}{##1}}}
\expandafter\def\csname PYG@tok@dl\endcsname{\def\PYG@tc##1{\textcolor[rgb]{0.25,0.44,0.63}{##1}}}
\expandafter\def\csname PYG@tok@err\endcsname{\def\PYG@bc##1{\setlength{\fboxsep}{0pt}\fcolorbox[rgb]{1.00,0.00,0.00}{1,1,1}{\strut ##1}}}
\expandafter\def\csname PYG@tok@mi\endcsname{\def\PYG@tc##1{\textcolor[rgb]{0.13,0.50,0.31}{##1}}}
\expandafter\def\csname PYG@tok@gi\endcsname{\def\PYG@tc##1{\textcolor[rgb]{0.00,0.63,0.00}{##1}}}
\expandafter\def\csname PYG@tok@m\endcsname{\def\PYG@tc##1{\textcolor[rgb]{0.13,0.50,0.31}{##1}}}
\expandafter\def\csname PYG@tok@sd\endcsname{\let\PYG@it=\textit\def\PYG@tc##1{\textcolor[rgb]{0.25,0.44,0.63}{##1}}}
\expandafter\def\csname PYG@tok@bp\endcsname{\def\PYG@tc##1{\textcolor[rgb]{0.00,0.44,0.13}{##1}}}
\expandafter\def\csname PYG@tok@ni\endcsname{\let\PYG@bf=\textbf\def\PYG@tc##1{\textcolor[rgb]{0.84,0.33,0.22}{##1}}}
\expandafter\def\csname PYG@tok@se\endcsname{\let\PYG@bf=\textbf\def\PYG@tc##1{\textcolor[rgb]{0.25,0.44,0.63}{##1}}}
\expandafter\def\csname PYG@tok@no\endcsname{\def\PYG@tc##1{\textcolor[rgb]{0.38,0.68,0.84}{##1}}}
\expandafter\def\csname PYG@tok@mb\endcsname{\def\PYG@tc##1{\textcolor[rgb]{0.13,0.50,0.31}{##1}}}
\expandafter\def\csname PYG@tok@mh\endcsname{\def\PYG@tc##1{\textcolor[rgb]{0.13,0.50,0.31}{##1}}}
\expandafter\def\csname PYG@tok@nv\endcsname{\def\PYG@tc##1{\textcolor[rgb]{0.73,0.38,0.84}{##1}}}
\expandafter\def\csname PYG@tok@sx\endcsname{\def\PYG@tc##1{\textcolor[rgb]{0.78,0.36,0.04}{##1}}}
\expandafter\def\csname PYG@tok@kt\endcsname{\def\PYG@tc##1{\textcolor[rgb]{0.56,0.13,0.00}{##1}}}
\expandafter\def\csname PYG@tok@mf\endcsname{\def\PYG@tc##1{\textcolor[rgb]{0.13,0.50,0.31}{##1}}}
\expandafter\def\csname PYG@tok@nt\endcsname{\let\PYG@bf=\textbf\def\PYG@tc##1{\textcolor[rgb]{0.02,0.16,0.45}{##1}}}
\expandafter\def\csname PYG@tok@sa\endcsname{\def\PYG@tc##1{\textcolor[rgb]{0.25,0.44,0.63}{##1}}}
\expandafter\def\csname PYG@tok@vg\endcsname{\def\PYG@tc##1{\textcolor[rgb]{0.73,0.38,0.84}{##1}}}
\expandafter\def\csname PYG@tok@vm\endcsname{\def\PYG@tc##1{\textcolor[rgb]{0.73,0.38,0.84}{##1}}}
\expandafter\def\csname PYG@tok@gp\endcsname{\let\PYG@bf=\textbf\def\PYG@tc##1{\textcolor[rgb]{0.78,0.36,0.04}{##1}}}
\expandafter\def\csname PYG@tok@sh\endcsname{\def\PYG@tc##1{\textcolor[rgb]{0.25,0.44,0.63}{##1}}}
\expandafter\def\csname PYG@tok@sb\endcsname{\def\PYG@tc##1{\textcolor[rgb]{0.25,0.44,0.63}{##1}}}
\expandafter\def\csname PYG@tok@il\endcsname{\def\PYG@tc##1{\textcolor[rgb]{0.13,0.50,0.31}{##1}}}
\expandafter\def\csname PYG@tok@ow\endcsname{\let\PYG@bf=\textbf\def\PYG@tc##1{\textcolor[rgb]{0.00,0.44,0.13}{##1}}}
\expandafter\def\csname PYG@tok@nf\endcsname{\def\PYG@tc##1{\textcolor[rgb]{0.02,0.16,0.49}{##1}}}
\expandafter\def\csname PYG@tok@s1\endcsname{\def\PYG@tc##1{\textcolor[rgb]{0.25,0.44,0.63}{##1}}}
\expandafter\def\csname PYG@tok@nd\endcsname{\let\PYG@bf=\textbf\def\PYG@tc##1{\textcolor[rgb]{0.33,0.33,0.33}{##1}}}
\expandafter\def\csname PYG@tok@mo\endcsname{\def\PYG@tc##1{\textcolor[rgb]{0.13,0.50,0.31}{##1}}}
\expandafter\def\csname PYG@tok@kp\endcsname{\def\PYG@tc##1{\textcolor[rgb]{0.00,0.44,0.13}{##1}}}
\expandafter\def\csname PYG@tok@gs\endcsname{\let\PYG@bf=\textbf}
\expandafter\def\csname PYG@tok@kn\endcsname{\let\PYG@bf=\textbf\def\PYG@tc##1{\textcolor[rgb]{0.00,0.44,0.13}{##1}}}
\expandafter\def\csname PYG@tok@vi\endcsname{\def\PYG@tc##1{\textcolor[rgb]{0.73,0.38,0.84}{##1}}}
\expandafter\def\csname PYG@tok@vc\endcsname{\def\PYG@tc##1{\textcolor[rgb]{0.73,0.38,0.84}{##1}}}
\expandafter\def\csname PYG@tok@gt\endcsname{\def\PYG@tc##1{\textcolor[rgb]{0.00,0.27,0.87}{##1}}}
\expandafter\def\csname PYG@tok@k\endcsname{\let\PYG@bf=\textbf\def\PYG@tc##1{\textcolor[rgb]{0.00,0.44,0.13}{##1}}}
\expandafter\def\csname PYG@tok@cs\endcsname{\def\PYG@tc##1{\textcolor[rgb]{0.25,0.50,0.56}{##1}}\def\PYG@bc##1{\setlength{\fboxsep}{0pt}\colorbox[rgb]{1.00,0.94,0.94}{\strut ##1}}}
\expandafter\def\csname PYG@tok@cp\endcsname{\def\PYG@tc##1{\textcolor[rgb]{0.00,0.44,0.13}{##1}}}
\expandafter\def\csname PYG@tok@kc\endcsname{\let\PYG@bf=\textbf\def\PYG@tc##1{\textcolor[rgb]{0.00,0.44,0.13}{##1}}}
\expandafter\def\csname PYG@tok@nl\endcsname{\let\PYG@bf=\textbf\def\PYG@tc##1{\textcolor[rgb]{0.00,0.13,0.44}{##1}}}
\expandafter\def\csname PYG@tok@fm\endcsname{\def\PYG@tc##1{\textcolor[rgb]{0.02,0.16,0.49}{##1}}}
\expandafter\def\csname PYG@tok@gh\endcsname{\let\PYG@bf=\textbf\def\PYG@tc##1{\textcolor[rgb]{0.00,0.00,0.50}{##1}}}
\expandafter\def\csname PYG@tok@nc\endcsname{\let\PYG@bf=\textbf\def\PYG@tc##1{\textcolor[rgb]{0.05,0.52,0.71}{##1}}}
\expandafter\def\csname PYG@tok@ss\endcsname{\def\PYG@tc##1{\textcolor[rgb]{0.32,0.47,0.09}{##1}}}
\expandafter\def\csname PYG@tok@o\endcsname{\def\PYG@tc##1{\textcolor[rgb]{0.40,0.40,0.40}{##1}}}
\expandafter\def\csname PYG@tok@gu\endcsname{\let\PYG@bf=\textbf\def\PYG@tc##1{\textcolor[rgb]{0.50,0.00,0.50}{##1}}}
\expandafter\def\csname PYG@tok@kr\endcsname{\let\PYG@bf=\textbf\def\PYG@tc##1{\textcolor[rgb]{0.00,0.44,0.13}{##1}}}
\expandafter\def\csname PYG@tok@gr\endcsname{\def\PYG@tc##1{\textcolor[rgb]{1.00,0.00,0.00}{##1}}}
\expandafter\def\csname PYG@tok@na\endcsname{\def\PYG@tc##1{\textcolor[rgb]{0.25,0.44,0.63}{##1}}}
\expandafter\def\csname PYG@tok@nb\endcsname{\def\PYG@tc##1{\textcolor[rgb]{0.00,0.44,0.13}{##1}}}
\expandafter\def\csname PYG@tok@c\endcsname{\let\PYG@it=\textit\def\PYG@tc##1{\textcolor[rgb]{0.25,0.50,0.56}{##1}}}
\expandafter\def\csname PYG@tok@sc\endcsname{\def\PYG@tc##1{\textcolor[rgb]{0.25,0.44,0.63}{##1}}}
\expandafter\def\csname PYG@tok@w\endcsname{\def\PYG@tc##1{\textcolor[rgb]{0.73,0.73,0.73}{##1}}}
\expandafter\def\csname PYG@tok@ch\endcsname{\let\PYG@it=\textit\def\PYG@tc##1{\textcolor[rgb]{0.25,0.50,0.56}{##1}}}

\def\PYGZbs{\char`\\}
\def\PYGZus{\char`\_}
\def\PYGZob{\char`\{}
\def\PYGZcb{\char`\}}
\def\PYGZca{\char`\^}
\def\PYGZam{\char`\&}
\def\PYGZlt{\char`\<}
\def\PYGZgt{\char`\>}
\def\PYGZsh{\char`\#}
\def\PYGZpc{\char`\%}
\def\PYGZdl{\char`\$}
\def\PYGZhy{\char`\-}
\def\PYGZsq{\char`\'}
\def\PYGZdq{\char`\"}
\def\PYGZti{\char`\~}
% for compatibility with earlier versions
\def\PYGZat{@}
\def\PYGZlb{[}
\def\PYGZrb{]}
\makeatother

\renewcommand\PYGZsq{\textquotesingle}

\begin{document}

\maketitle
\tableofcontents
\phantomsection\label{index::doc}


The Agent-Based Model of Human Activity Patterns (ABMHAP, pronounced ``ab-map'') is one of the
modules for the Life Cycle Human Exposure Model (LC-HEM) project as described in the United States Environmental
Protection Agency (U.S. EPA) research plan, which may be found
\href{https://19january2017snapshot.epa.gov/sites/production/files/2016-11/documents/css\_fy16-19\_strap.pdf}{here}.
ABMHAP is a model capable of creating agents that simulate longitudinal human behavior. The current version of
ABMHAP is able to simulate daily routines for the following behaviors:
\begin{enumerate}
\item {} 
Sleeping

\item {} 
Eating Breakfast

\item {} 
Eating Lunch

\item {} 
Eating Dinner

\item {} 
Working

\item {} 
Commuting to Work

\item {} 
Commuting from Work

\item {} 
Being idle (i.e., time spent not doing the above activities)

\end{enumerate}

The current version of ABMHAP requires the user to input parameters that describe the longitudinal variation in behavior of a single individual.

The current version of ABMHAP is written in Python version 3.5.3. More information on the Python programming
language may be found \href{https://www.python.org/}{here}. The Python libraries that must be installed in
order for ABMHAP to run are listed below.
\begin{itemize}
\item {} 
matplotlib

\item {} 
multiprocessing

\item {} 
numpy

\item {} 
pandas

\item {} 
scipy

\item {} 
sphinx

\item {} 
statsmodels

\end{itemize}

ABMHAP is written by Dr. Namdi Brandon (ORCID: 0000-0001-7050-1538).

\begin{DUlineblock}{0em}
\item[] 
\end{DUlineblock}
\begin{description}
\item[{Disclaimer}] \leavevmode
The United States Environmental Protection Agency through its Office of Research and Development has
developed this software. The code is made publicly available to better communicate the research. All
input data used for a given application should be reviewed by the researcher so that the model results
are based on appropriate data for any given application. This model is under continued development. The
model and data included herein do not represent and should not be construed to represent any Agency
determination or policy.

\end{description}


\chapter{How to Run the Code}
\label{index:welcome-to-the-documentation-for-the-agent-based-model-of-human-activity-patterns-abmhap}\label{index:how-to-run-the-code}
The following describes how to run an ABMHAP simulation of one agent. In order to do so, the user must
do the following:
\begin{enumerate}
\item {} 
set the input parameters of the simulation in the file \code{\textbackslash{}run\textbackslash{}main\_params.py}

\item {} 
run the executable using \code{\textbackslash{}run\textbackslash{}main.py}

\end{enumerate}


\section{Setting the input parameters}
\label{index:setting-the-input-parameters}
In order to run ABMHAP, the user must set the following types of input parameters in
\code{\textbackslash{}run\textbackslash{}main\_params.py}:
\begin{enumerate}
\item {} 
input parameters that govern the general logistics of the simulation

\item {} 
input parameters that govern the the length of simulation duration

\item {} 
input parameters that define the behavior of the agent

\end{enumerate}

For illustrative purposes, what follows is a demonstration of how to parametrize a run for ABMHAP.

The below code does the following:
\begin{itemize}
\item {} 
informs the algorithm to not print the output to the screen

\item {} 
informs the algorithm to not plot the output

\item {} 
informs the algorithm to not save the output to a file

\item {} 
should the output file be saved, sets the output file to \code{\textbackslash{}some\_directory\textbackslash{}outputfile.csv}

\end{itemize}

The user must set the input parameters that govern the general logistics of the simulation:

\begin{Verbatim}[commandchars=\\\{\}]
\PYG{c+c1}{\PYGZsh{} whether (if True) or not (if False) the output of the simulation should}
\PYG{c+c1}{\PYGZsh{} print a message to screen}
\PYG{n}{do\PYGZus{}print}    \PYG{o}{=} \PYG{n+nb+bp}{False}

\PYG{c+c1}{\PYGZsh{} whether (if True) or not (if False) the output of the simulation should}
\PYG{c+c1}{\PYGZsh{} be plotted a message to screen}
\PYG{n}{do\PYGZus{}plot}     \PYG{o}{=} \PYG{n+nb+bp}{False}

\PYG{c+c1}{\PYGZsh{} whether (if True) or not (if False) the output of the simulation should}
\PYG{c+c1}{\PYGZsh{} be saved in a file}
\PYG{n}{do\PYGZus{}save}     \PYG{o}{=} \PYG{n+nb+bp}{False}

\PYG{c+c1}{\PYGZsh{} the name of the output file should the output be saved. The filename}
\PYG{c+c1}{\PYGZsh{} should have a \PYGZdq{}.csv\PYGZdq{} extension}
\PYG{n}{fname}       \PYG{o}{=} \PYG{l+s+s1}{\PYGZsq{}}\PYG{l+s+s1}{some\PYGZus{}directory}\PYG{l+s+se}{\PYGZbs{}\PYGZbs{}}\PYG{l+s+s1}{outputfile.csv}\PYG{l+s+s1}{\PYGZsq{}}
\end{Verbatim}

The following code shows how to set ABMHAP to run starting on Sunday, Day 0 starting from 16:00
and ending on Monday, Day 7 at 0:00. It's recommended that the user start running the code on a Sunday or Saturday
at 16:00 in order to minimize potential activity conflicts at initiation.

The user must set the input parameters dealing with the duration of the simulation:

\begin{Verbatim}[commandchars=\\\{\}]
\PYG{c+c1}{\PYGZsh{} the number of days for the simulation}
\PYG{n}{num\PYGZus{}days}    \PYG{o}{=} \PYG{l+m+mi}{7}

\PYG{c+c1}{\PYGZsh{} the number of additional hours}
\PYG{n}{num\PYGZus{}hours}   \PYG{o}{=} \PYG{l+m+mi}{8}

\PYG{c+c1}{\PYGZsh{} the number of additional minutes}
\PYG{n}{num\PYGZus{}min}     \PYG{o}{=} \PYG{l+m+mi}{0}
\end{Verbatim}

The user must set the input parameters dealing with when in the simulation year the simulation should start:

\begin{Verbatim}[commandchars=\\\{\}]
\PYG{c+c1}{\PYGZsh{} start the simulation on Sunday, Day 0 at 16:00}
\PYG{n}{t\PYGZus{}start}     \PYG{o}{=} \PYG{n}{WINTER} \PYG{o}{*} \PYG{n}{SEASON\PYGZus{}2\PYGZus{}MIN} \PYG{o}{+} \PYG{l+m+mi}{0} \PYG{o}{*} \PYG{n}{WEEK\PYGZus{}2\PYGZus{}MIN} \PYGZbs{}
            \PYG{o}{+} \PYG{n}{SUNDAY} \PYG{o}{*} \PYG{n}{DAY\PYGZus{}2\PYGZus{}MIN} \PYG{o}{+} \PYG{l+m+mi}{16} \PYG{o}{*} \PYG{n}{HOUR\PYGZus{}2\PYGZus{}MIN}
\end{Verbatim}

where the following constants are useful in assigning input parameters that define
the start time of the simulation:

\begin{Verbatim}[commandchars=\\\{\}]
\PYG{c+c1}{\PYGZsh{} an agent\PYGZhy{}based model module with capabilities concerning time}
\PYG{k+kn}{import} \PYG{n+nn}{temporal}

\PYG{c+c1}{\PYGZsh{} the value of Sunday}
\PYG{n}{SUNDAY}         \PYG{o}{=} \PYG{n}{temporal}\PYG{o}{.}\PYG{n}{SUNDAY}

\PYG{c+c1}{\PYGZsh{} convert one day into minutes}
\PYG{n}{DAY\PYGZus{}2\PYGZus{}MIN}      \PYG{o}{=} \PYG{n}{temporal}\PYG{o}{.}\PYG{n}{DAY\PYGZus{}2\PYGZus{}MIN}

\PYG{c+c1}{\PYGZsh{} convert one hour into minutes}
\PYG{n}{HOUR\PYGZus{}2\PYGZus{}MIN}     \PYG{o}{=} \PYG{n}{temporal}\PYG{o}{.}\PYG{n}{HOUR\PYGZus{}2\PYGZus{}MIN}

\PYG{c+c1}{\PYGZsh{} the number of minutes in one season (13 weeks)}
\PYG{n}{SEASON\PYGZus{}2\PYGZus{}MIN}   \PYG{o}{=} \PYG{n}{temporal}\PYG{o}{.}\PYG{n}{SEASON\PYGZus{}2\PYGZus{}MIN}

\PYG{c+c1}{\PYGZsh{} the number of minutes in one week}
\PYG{n}{WEEK\PYGZus{}2\PYGZus{}MIN}     \PYG{o}{=} \PYG{n}{WEEK\PYGZus{}2\PYGZus{}MIN}

\PYG{c+c1}{\PYGZsh{} the winter season (has the value 0)}
\PYG{n}{WINTER}         \PYG{o}{=} \PYG{n}{temporal}\PYG{o}{.}\PYG{n}{WINTER}
\end{Verbatim}

The user must set the input parameters that govern the behavior of the agent. The input parameters will govern
the agent's behavior for the following activities.
\begin{enumerate}
\item {} 
sleeping

\item {} 
eating breakfast

\item {} 
eating lunch

\item {} 
eating dinner

\item {} 
working

\item {} 
commuting to work

\item {} 
commuting from work

\end{enumerate}

In order to set the sleeping behavior, the user must set the the mean and standard deviation for the start time
and end time for the sleep activity. The agent's behavior for sleeping is set as follows:

\begin{Verbatim}[commandchars=\\\{\}]
\PYG{c+c1}{\PYGZsh{} set the mean start time to be 22:00}
\PYG{n}{sleep\PYGZus{}start\PYGZus{}mean}     \PYG{o}{=} \PYG{n}{np}\PYG{o}{.}\PYG{n}{array}\PYG{p}{(} \PYG{p}{[}\PYG{l+m+mi}{22} \PYG{o}{*} \PYG{n}{HOUR\PYGZus{}2\PYGZus{}MIN}\PYG{p}{]} \PYG{p}{)}

\PYG{c+c1}{\PYGZsh{} set the standard deviation of the start time to be 30 minutes}
\PYG{n}{sleep\PYGZus{}start\PYGZus{}std}      \PYG{o}{=} \PYG{n}{np}\PYG{o}{.}\PYG{n}{array}\PYG{p}{(} \PYG{p}{[}\PYG{l+m+mi}{30}\PYG{p}{]} \PYG{p}{)}

\PYG{c+c1}{\PYGZsh{} set the mean end time to be 8:00}
\PYG{n}{sleep\PYGZus{}end\PYGZus{}mean}       \PYG{o}{=} \PYG{n}{np}\PYG{o}{.}\PYG{n}{array}\PYG{p}{(} \PYG{p}{[}\PYG{l+m+mi}{8} \PYG{o}{*} \PYG{n}{HOUR\PYGZus{}2\PYGZus{}MIN}\PYG{p}{]} \PYG{p}{)}

\PYG{c+c1}{\PYGZsh{} set the standard deviation of the end time to be 15 minutes}
\PYG{n}{sleep\PYGZus{}end\PYGZus{}std}        \PYG{o}{=} \PYG{n}{np}\PYG{o}{.}\PYG{n}{array}\PYG{p}{(} \PYG{p}{[}\PYG{l+m+mi}{15}\PYG{p}{]} \PYG{p}{)}
\end{Verbatim}

In order to set the eat breakfast behavior, the user must set the mean and standard deviation for the start time
and duration for the eat breakfast activity. The agent's behavior for eating breakfast is set as follows:

\begin{Verbatim}[commandchars=\\\{\}]
\PYG{c+c1}{\PYGZsh{} set the mean start time to be 8:00}
\PYG{n}{bf\PYGZus{}start\PYGZus{}mean}       \PYG{o}{=} \PYG{n}{np}\PYG{o}{.}\PYG{n}{array}\PYG{p}{(} \PYG{p}{[}\PYG{l+m+mi}{8} \PYG{o}{*} \PYG{n}{HOUR\PYGZus{}2\PYGZus{}MIN}\PYG{p}{]} \PYG{p}{)}

\PYG{c+c1}{\PYGZsh{} set the standard deviation of the start time to be 10 minutes}
\PYG{n}{bf\PYGZus{}start\PYGZus{}std}        \PYG{o}{=} \PYG{n}{np}\PYG{o}{.}\PYG{n}{array}\PYG{p}{(} \PYG{p}{[}\PYG{l+m+mi}{10}\PYG{p}{]} \PYG{p}{)}

\PYG{c+c1}{\PYGZsh{} set the mean duration to be 15 minutes}
\PYG{n}{bf\PYGZus{}dt\PYGZus{}mean}          \PYG{o}{=} \PYG{n}{np}\PYG{o}{.}\PYG{n}{array}\PYG{p}{(} \PYG{p}{[}\PYG{l+m+mi}{15}\PYG{p}{]} \PYG{p}{)}

\PYG{c+c1}{\PYGZsh{} set the standard deviation of the duration to be 10 minutes}
\PYG{n}{bf\PYGZus{}dt\PYGZus{}std}           \PYG{o}{=} \PYG{n}{np}\PYG{o}{.}\PYG{n}{array}\PYG{p}{(} \PYG{p}{[}\PYG{l+m+mi}{10}\PYG{p}{]} \PYG{p}{)}
\end{Verbatim}

In order to set the eat lunch behavior, the user must set the mean and standard deviation for the start time
and duration for the eat lunch activity. The agent's behavior for eating lunch is set as follows:

\begin{Verbatim}[commandchars=\\\{\}]
\PYG{c+c1}{\PYGZsh{} set the mean start time to be 12:000}
\PYG{n}{lunch\PYGZus{}start\PYGZus{}mean}       \PYG{o}{=} \PYG{n}{np}\PYG{o}{.}\PYG{n}{array}\PYG{p}{(} \PYG{p}{[}\PYG{l+m+mi}{12} \PYG{o}{*} \PYG{n}{HOUR\PYGZus{}2\PYGZus{}MIN}\PYG{p}{]} \PYG{p}{)}

\PYG{c+c1}{\PYGZsh{} set the standard deviation of start time to be 15 minutes}
\PYG{n}{lunch\PYGZus{}start\PYGZus{}std}        \PYG{o}{=} \PYG{n}{np}\PYG{o}{.}\PYG{n}{array}\PYG{p}{(} \PYG{p}{[}\PYG{l+m+mi}{15}\PYG{p}{]} \PYG{p}{)}

\PYG{c+c1}{\PYGZsh{} set the mean duration to be 30 minutes}
\PYG{n}{lunch\PYGZus{}dt\PYGZus{}mean}          \PYG{o}{=} \PYG{n}{np}\PYG{o}{.}\PYG{n}{array}\PYG{p}{(} \PYG{p}{[}\PYG{l+m+mi}{30}\PYG{p}{]} \PYG{p}{)}

\PYG{c+c1}{\PYGZsh{} set the standard deviation of duration to be 10 minutes}
\PYG{n}{lunch\PYGZus{}dt\PYGZus{}std}           \PYG{o}{=} \PYG{n}{np}\PYG{o}{.}\PYG{n}{array}\PYG{p}{(} \PYG{p}{[}\PYG{l+m+mi}{10}\PYG{p}{]} \PYG{p}{)}
\end{Verbatim}

In order to set the eat dinner behavior, the user must set the mean and standard deviation for the start time
and duration for the eat dinner activity. The agent's behavior for eating dinner is set as follows:

\begin{Verbatim}[commandchars=\\\{\}]
\PYG{c+c1}{\PYGZsh{} set the mean start time to be 19:00}
\PYG{n}{dinner\PYGZus{}start\PYGZus{}mean}       \PYG{o}{=} \PYG{n}{np}\PYG{o}{.}\PYG{n}{array}\PYG{p}{(} \PYG{p}{[}\PYG{l+m+mi}{19} \PYG{o}{*} \PYG{n}{HOUR\PYGZus{}2\PYGZus{}MIN}\PYG{p}{]} \PYG{p}{)}

\PYG{c+c1}{\PYGZsh{} set the standard deviation of start time to be 10 minutes}
\PYG{n}{dinner\PYGZus{}start\PYGZus{}std}        \PYG{o}{=} \PYG{n}{np}\PYG{o}{.}\PYG{n}{array}\PYG{p}{(} \PYG{p}{[}\PYG{l+m+mi}{10}\PYG{p}{]} \PYG{p}{)}

\PYG{c+c1}{\PYGZsh{} set the mean of duration to be 45 minutes}
\PYG{n}{dinner\PYGZus{}dt\PYGZus{}mean}          \PYG{o}{=} \PYG{n}{np}\PYG{o}{.}\PYG{n}{array}\PYG{p}{(} \PYG{p}{[}\PYG{l+m+mi}{45}\PYG{p}{]} \PYG{p}{)}

\PYG{c+c1}{\PYGZsh{} set the standard deviation of duration to be 5 minutes}
\PYG{n}{dinner\PYGZus{}dt\PYGZus{}std}           \PYG{o}{=} \PYG{n}{np}\PYG{o}{.}\PYG{n}{array}\PYG{p}{(} \PYG{p}{[}\PYG{l+m+mi}{5}\PYG{p}{]} \PYG{p}{)}
\end{Verbatim}

In order to set the work behavior, the user must set the mean and standard deviation for the start time and
end time for the work activity. The agent's behavior for working is set as follows:

\begin{Verbatim}[commandchars=\\\{\}]
\PYG{c+c1}{\PYGZsh{} set the mean start time to be 9:00}
\PYG{n}{work\PYGZus{}start\PYGZus{}mean}     \PYG{o}{=} \PYG{n}{np}\PYG{o}{.}\PYG{n}{array}\PYG{p}{(} \PYG{p}{[}\PYG{l+m+mi}{9} \PYG{o}{*} \PYG{n}{HOUR\PYGZus{}2\PYGZus{}MIN}\PYG{p}{]} \PYG{p}{)}

\PYG{c+c1}{\PYGZsh{} set the standard deviation of start time to be 15 minutes}
\PYG{n}{work\PYGZus{}start\PYGZus{}std}      \PYG{o}{=} \PYG{n}{np}\PYG{o}{.}\PYG{n}{array}\PYG{p}{(} \PYG{p}{[}\PYG{l+m+mi}{15}\PYG{p}{]} \PYG{p}{)}

\PYG{c+c1}{\PYGZsh{} set the mean end time to be 17:00}
\PYG{n}{work\PYGZus{}end\PYGZus{}mean}       \PYG{o}{=} \PYG{n}{np}\PYG{o}{.}\PYG{n}{array}\PYG{p}{(} \PYG{p}{[}\PYG{l+m+mi}{17} \PYG{o}{*} \PYG{n}{HOUR\PYGZus{}2\PYGZus{}MIN}\PYG{p}{]} \PYG{p}{)}

\PYG{c+c1}{\PYGZsh{} set the standard deviation of end time to be 5 minutes}
\PYG{n}{work\PYGZus{}end\PYGZus{}std}        \PYG{o}{=} \PYG{n}{np}\PYG{o}{.}\PYG{n}{array}\PYG{p}{(} \PYG{p}{[}\PYG{l+m+mi}{5}\PYG{p}{]} \PYG{p}{)}
\end{Verbatim}

The user must set the agent's employment status. The agent's employment status is set as follows:

\begin{Verbatim}[commandchars=\\\{\}]
\PYG{c+c1}{\PYGZsh{} an agent\PYGZhy{}based model module for functionality dealing with occupation}
\PYG{k+kn}{import} \PYG{n+nn}{occupation}

\PYG{c+c1}{\PYGZsh{} set the job identifier (job id) as standard job if the agent}
\PYG{c+c1}{\PYGZsh{} is supposed to work}
\PYG{n}{job\PYGZus{}id}   \PYG{o}{=} \PYG{n}{occupation}\PYG{o}{.}\PYG{n}{STANDARD\PYGZus{}JOB}

\PYG{c+c1}{\PYGZsh{} OR set the job identifier (job id) as not having a job if the agent}
\PYG{c+c1}{\PYGZsh{} is NOT supposed to work}
\PYG{n}{job\PYGZus{}id}   \PYG{o}{=} \PYG{n}{occupation}\PYG{o}{.}\PYG{n}{NO\PYGZus{}JOB}
\end{Verbatim}

In order to set the commute to work behavior, the user must set the mean and standard deviation for the duration
of the commute to work activity. The agent's behavior for commuting to work is set as follows:

\begin{Verbatim}[commandchars=\\\{\}]
\PYG{c+c1}{\PYGZsh{} set the mean duration to be 30 minutes}
\PYG{n}{commute\PYGZus{}to\PYGZus{}work\PYGZus{}dt\PYGZus{}mean}     \PYG{o}{=} \PYG{n}{np}\PYG{o}{.}\PYG{n}{array}\PYG{p}{(} \PYG{p}{[}\PYG{l+m+mi}{30}\PYG{p}{]} \PYG{p}{)}

\PYG{c+c1}{\PYGZsh{} set the standard deviation to be 10 minutes}
\PYG{n}{commute\PYGZus{}to\PYGZus{}work\PYGZus{}dt\PYGZus{}std}      \PYG{o}{=} \PYG{n}{np}\PYG{o}{.}\PYG{n}{array}\PYG{p}{(} \PYG{p}{[}\PYG{l+m+mi}{10}\PYG{p}{]} \PYG{p}{)}
\end{Verbatim}

In order to set the commute from work behavior, the user must set the mean and standard deviation for the duration
of the commute from work activity. The agent's behavior for commuting from work is set as follows:

\begin{Verbatim}[commandchars=\\\{\}]
\PYG{c+c1}{\PYGZsh{} set the mean duration to be 30 minutes}
\PYG{n}{commute\PYGZus{}from\PYGZus{}work\PYGZus{}dt\PYGZus{}mean}     \PYG{o}{=} \PYG{n}{np}\PYG{o}{.}\PYG{n}{array}\PYG{p}{(} \PYG{p}{[}\PYG{l+m+mi}{30}\PYG{p}{]} \PYG{p}{)}

\PYG{c+c1}{\PYGZsh{} set the standard deviation to be 10 minutes}
\PYG{n}{commute\PYGZus{}from\PYGZus{}work\PYGZus{}dt\PYGZus{}std}      \PYG{o}{=} \PYG{n}{np}\PYG{o}{.}\PYG{n}{array}\PYG{p}{(} \PYG{p}{[}\PYG{l+m+mi}{10}\PYG{p}{]} \PYG{p}{)}
\end{Verbatim}


\section{Running the simulation}
\label{index:running-the-simulation}
After defining all of the parameters in the file \code{\textbackslash{}run\textbackslash{}main\_params.py}, the code is run by doing
the following:
\begin{enumerate}
\item {} 
go to the \code{\textbackslash{}run} directory.

\item {} 
enter \code{python main.py} into the terminal (or command line)

\item {} 
press enter (or return)

\end{enumerate}


\section{Interpreting the output}
\label{index:interpreting-the-output}
ABMHAP outputs the record of the activities that the agent did during the simulation. This record is called an
\textbf{activity diary}. An activity diary is a chronological record contains the following information about each
activity: day, start time, end time, duration, and location.

Below is an example of the output of ABMHAP. Recall that ABMHAP saves the activity diary in terms of a .csv file

\begin{tabulary}{\linewidth}{|L|L|L|L|L|L|}
\hline
\textsf{\relax 
day
} & \textsf{\relax 
start
} & \textsf{\relax 
end
} & \textsf{\relax 
dt
} & \textsf{\relax 
act
} & \textsf{\relax 
loc
}\\
\hline
0
 & 
16
 & 
19
 & 
3
 & 
-1
 & 
0
\\
\hline
0
 & 
19
 & 
19.75
 & 
0.75
 & 
4
 & 
0
\\
\hline
0
 & 
19.75
 & 
22
 & 
2.25
 & 
-1
 & 
0
\\
\hline
0
 & 
22
 & 
8
 & 
10
 & 
6
 & 
0
\\
\hline
1
 & 
8
 & 
8.25
 & 
0.25
 & 
3
 & 
0
\\
\hline
1
 & 
8.25
 & 
8.5
 & 
0.25
 & 
-1
 & 
0
\\
\hline
1
 & 
8.5
 & 
9
 & 
0.5
 & 
2
 & 
1
\\
\hline
1
 & 
9
 & 
12
 & 
3
 & 
7
 & 
3
\\
\hline
1
 & 
12
 & 
12.5
 & 
0.5
 & 
5
 & 
3
\\
\hline
1
 & 
12.5
 & 
17
 & 
4.5
 & 
7
 & 
3
\\
\hline
1
 & 
17
 & 
17.5
 & 
0.5
 & 
1
 & 
1
\\
\hline
1
 & 
17.5
 & 
19
 & 
1.5
 & 
-1
 & 
0
\\
\hline
1
 & 
19
 & 
19.75
 & 
0.75
 & 
4
 & 
0
\\
\hline
1
 & 
19.75
 & 
22
 & 
2.25
 & 
-1
 & 
0
\\
\hline
1
 & 
22
 & 
8
 & 
10
 & 
6
 & 
0
\\
\hline\end{tabulary}


where day, start, end, dt, act, and loc represent the day the activity starts, the start time of the
activity (in hours), the end time of the activity (in hours), the duration of the activity (in hours), the
activity identifier, and the location identifier, respectively. In the results, the time of day 16:30 is
represented as 16.5.

The following table is an interpretation of the example output shown above. In the table, the duration is
expressed in minutes.

\begin{tabulary}{\linewidth}{|L|L|L|L|L|L|}
\hline
\textsf{\relax 
Day
} & \textsf{\relax 
Start
} & \textsf{\relax 
End
} & \textsf{\relax 
Duration
} & \textsf{\relax 
Activity Code
} & \textsf{\relax 
Location Code
}\\
\hline
0
 & 
16:00
 & 
19:00
 & 
180
 & 
Idle
 & 
Home
\\
\hline
0
 & 
19:00
 & 
19:45
 & 
45
 & 
Eat dinner
 & 
Home
\\
\hline
0
 & 
19:45
 & 
22:00
 & 
135
 & 
Idle
 & 
Home
\\
\hline
0
 & 
22:00
 & 
8:00
 & 
600
 & 
Sleep
 & 
Home
\\
\hline
1
 & 
8:00
 & 
8:15
 & 
15
 & 
Eat breakfast
 & 
Home
\\
\hline
1
 & 
8:15
 & 
8:30
 & 
15
 & 
Idle
 & 
Home
\\
\hline
1
 & 
8:30
 & 
9:00
 & 
30
 & 
Commute to work
 & 
Out of doors
\\
\hline
1
 & 
9:00
 & 
12:00
 & 
180
 & 
Work
 & 
Workplace
\\
\hline
1
 & 
12:00
 & 
12:30
 & 
30
 & 
Eat lunch
 & 
Workplace
\\
\hline
1
 & 
12:30
 & 
17:00
 & 
270
 & 
Work
 & 
Workplace
\\
\hline
1
 & 
17:00
 & 
17:30
 & 
30
 & 
Commute from work
 & 
Out of doors
\\
\hline
1
 & 
17:30
 & 
19:00
 & 
90
 & 
Idle
 & 
Home
\\
\hline
1
 & 
19:00
 & 
19:45
 & 
45
 & 
Eat dinner
 & 
Home
\\
\hline
1
 & 
19.45
 & 
22:00
 & 
135
 & 
Idle
 & 
Home
\\
\hline
1
 & 
22:00
 & 
8:00
 & 
600
 & 
Sleep
 & 
Home
\\
\hline\end{tabulary}


The activity codes map as the following:

\begin{tabulary}{\linewidth}{|L|L|}
\hline
\textsf{\relax 
Activity
} & \textsf{\relax 
Activity Code
}\\
\hline
Idle
 & 
-1
\\
\hline
Commute from work
 & 
1
\\
\hline
Commute to work
 & 
2
\\
\hline
Eat breakfast
 & 
3
\\
\hline
Eat dinner
 & 
4
\\
\hline
Eat lunch
 & 
5
\\
\hline
Sleep
 & 
6
\\
\hline
Work
 & 
7
\\
\hline\end{tabulary}


The location codes map as the following:

\begin{tabulary}{\linewidth}{|L|L|}
\hline
\textsf{\relax 
Location
} & \textsf{\relax 
Location Code
}\\
\hline
Home
 & 
0
\\
\hline
Out of doors
 & 
1
\\
\hline
Workplace
 & 
3
\\
\hline\end{tabulary}



\chapter{Source Directory}
\label{index:source-directory}
These files are the key modules that are used to create the ABMHAP algorithm.

Contents:


\section{activity module}
\label{activity::doc}\label{activity:module-activity}\label{activity:activity-module}\index{activity (module)}
This module contains code that governs the activities that the agent performs
in order to satisfy its needs.

This module contains the following class: {\hyperref[activity:activity.Activity]{\emph{\code{activity.Activity}}}}.

\emph{Module author: Dr. Namdi Brandon}
\index{Activity (class in activity)}

\begin{fulllineitems}
\phantomsection\label{activity:activity.Activity}\pysigline{\strong{class }\code{activity.}\bfcode{Activity}}
Bases: \code{object}

An activity enables a {\hyperref[person:person.Person]{\emph{\code{person.Person}}}} to address its satiation
\(n(t)\). This class's purpose is to encapsulate general
capabilities that specific instances of activity will derive from.
\begin{quote}\begin{description}
\item[{Variables}] \leavevmode\begin{itemize}
\item {} 
\textbf{\texttt{category}} (\emph{int}) -- an unique identifier naming the type of activity.

\item {} 
\textbf{\texttt{t\_end}} (\emph{int}) -- the end time of the activity {[}universal time, seconds{]}

\item {} 
\textbf{\texttt{t\_start}} (\emph{int}) -- the start time of the activity {[}universal time, seconds{]}

\item {} 
\textbf{\texttt{dt}} (\emph{int}) -- the duration of the activity {[}seconds{]}

\end{itemize}

\end{description}\end{quote}
\index{advertise() (activity.Activity method)}

\begin{fulllineitems}
\phantomsection\label{activity:activity.Activity.advertise}\pysiglinewithargsret{\bfcode{advertise}}{\emph{the\_need}, \emph{dt}}{}
Calculates the advertised score of doing an activity. Let
\(\Omega\) be the set of all needs addressed by the activity.
The score is calculated by doing the following
\begin{gather}
\begin{split}S = \begin{cases}
    0  & n(t) > \lambda \\
    \sum_{j \in \Omega} W_j( n_j(t) ) - W_j( n_j(t + \Delta{t} )) & n(t) \le \lambda
\end{cases}\end{split}\notag
\end{gather}
where \(W(n)\) is the weight function for the particular need.
\begin{quote}\begin{description}
\item[{Parameters}] \leavevmode\begin{itemize}
\item {} 
\textbf{\texttt{the\_need}} ({\hyperref[need:need.Need]{\emph{\emph{need.Need}}}}) -- the primary need associated with the respective activity

\item {} 
\textbf{\texttt{dt}} (\emph{int}) -- the duration \(\Delta{t}\) of doing the activity {[}seconds{]}

\end{itemize}

\item[{Returns score}] \leavevmode
the score of the advertisement

\item[{Return type}] \leavevmode
float

\end{description}\end{quote}

\end{fulllineitems}

\index{advertise\_interruption() (activity.Activity method)}

\begin{fulllineitems}
\phantomsection\label{activity:activity.Activity.advertise_interruption}\pysiglinewithargsret{\bfcode{advertise\_interruption}}{}{}
Advertise the score if this activity interrupts another activity.

\begin{notice}{note}{Note:}
This function should be overloaded in derived classes.
\end{notice}
\begin{quote}\begin{description}
\item[{Returns score}] \leavevmode
the advertised score

\item[{Return type}] \leavevmode
float

\end{description}\end{quote}

\end{fulllineitems}

\index{end() (activity.Activity method)}

\begin{fulllineitems}
\phantomsection\label{activity:activity.Activity.end}\pysiglinewithargsret{\bfcode{end}}{\emph{p}}{}
This function handles some of the common logistics in ending a specific activity assuming         the activity ends without interruption.

Currently the function does the following:
\begin{enumerate}
\item {} 
Reset the {\hyperref[state:state.State]{\emph{\code{state.State}}}} variable's start time to the current time

\item {} 
Reset the {\hyperref[state:state.State]{\emph{\code{state.State}}}} variable's end time to the current time

\end{enumerate}
\begin{quote}\begin{description}
\item[{Parameters}] \leavevmode
\textbf{\texttt{p}} ({\hyperref[person:person.Person]{\emph{\emph{person.Person}}}}) -- the person whose activity is ending.

\item[{Returns}] \leavevmode
None

\end{description}\end{quote}

\end{fulllineitems}

\index{halt() (activity.Activity method)}

\begin{fulllineitems}
\phantomsection\label{activity:activity.Activity.halt}\pysiglinewithargsret{\bfcode{halt}}{\emph{p}}{}
This function handles some of the common logistics in ending a specific activity due to an         interruption.

Currently the function does the following:
\begin{enumerate}
\item {} 
Reset the {\hyperref[state:state.State]{\emph{\code{state.State}}}} variable's start time to the current time

\item {} 
Reset the {\hyperref[state:state.State]{\emph{\code{state.State}}}} variable's end time to the current time

\end{enumerate}
\begin{quote}\begin{description}
\item[{Parameters}] \leavevmode
\textbf{\texttt{p}} ({\hyperref[person:person.Person]{\emph{\emph{person.Person}}}}) -- The person whose activity is being interrupted.

\item[{Returns}] \leavevmode
None

\end{description}\end{quote}

\end{fulllineitems}

\index{print\_id() (activity.Activity method)}

\begin{fulllineitems}
\phantomsection\label{activity:activity.Activity.print_id}\pysiglinewithargsret{\bfcode{print\_id}}{}{}
This function represents the activity category as a string.
\begin{quote}\begin{description}
\item[{Return msg}] \leavevmode
The string representation of the category

\item[{Return type}] \leavevmode
str

\end{description}\end{quote}

\end{fulllineitems}

\index{start() (activity.Activity method)}

\begin{fulllineitems}
\phantomsection\label{activity:activity.Activity.start}\pysiglinewithargsret{\bfcode{start}}{}{}
This function starts a specific activity.

\begin{notice}{note}{Note:}
This function is meant to be overloaded by derived activity classes.
\end{notice}
\begin{quote}\begin{description}
\item[{Returns}] \leavevmode
None

\end{description}\end{quote}

\end{fulllineitems}

\index{toString() (activity.Activity method)}

\begin{fulllineitems}
\phantomsection\label{activity:activity.Activity.toString}\pysiglinewithargsret{\bfcode{toString}}{}{}
This function represents the activity object as a string.
\begin{quote}\begin{description}
\item[{Return msg}] \leavevmode
The string representation of the activity object

\item[{Return type}] \leavevmode
str

\end{description}\end{quote}

\end{fulllineitems}


\end{fulllineitems}



\section{asset module}
\label{asset::doc}\label{asset:module-asset}\label{asset:asset-module}\index{asset (module)}
This module contains code that governs objects that enable access to activities ({\hyperref[activity:activity.Activity]{\emph{\code{activity.Activity}}}}) that an agent may use in order to address a need.

This module contains the following class: {\hyperref[asset:asset.Asset]{\emph{\code{asset.Asset}}}}.

\emph{Module author: Dr. Namdi Brandon}
\index{Asset (class in asset)}

\begin{fulllineitems}
\phantomsection\label{asset:asset.Asset}\pysigline{\strong{class }\code{asset.}\bfcode{Asset}}
Bases: \code{object}

An asset is an object that allows the agent to perform an activity. Each asset     contains a list Activities that an agent can use to perform actions.
\begin{quote}\begin{description}
\item[{Variables}] \leavevmode\begin{itemize}
\item {} 
\textbf{\texttt{activities}} (\emph{dict}) -- This is a dictionary of all the activities associated with this asset.

\item {} 
\textbf{\texttt{category}} (\emph{int}) -- This indicates the type of asset.

\item {} 
\textbf{\texttt{id}} (\emph{int}) -- This is an identifier number for the asset.

\item {} 
\textbf{\texttt{'location'}} ({\hyperref[location:location.Location]{\emph{\emph{location.Location}}}}) -- This is the location of the asset.

\item {} 
\textbf{\texttt{max\_users}} (\emph{int}) -- This indicates the maximum number of users that can simultaneously access the asset.

\item {} 
\textbf{\texttt{num\_users}} (\emph{int}) -- This is the current number of users for the asset.

\item {} 
\textbf{\texttt{status}} (\emph{int}) -- This is the state of the asset.

\end{itemize}

\end{description}\end{quote}
\index{free() (asset.Asset method)}

\begin{fulllineitems}
\phantomsection\label{asset:asset.Asset.free}\pysiglinewithargsret{\bfcode{free}}{}{}
This function changes the state of an asset once it is freed by a Person by doing the following:
\begin{enumerate}
\item {} 
Decreases the number of users of the asset by 1

\item {} 
If the number of users is zero, the status of the asset is set to idle (\code{state.IDLE})

\end{enumerate}
\begin{quote}\begin{description}
\item[{Returns}] \leavevmode
None

\end{description}\end{quote}

\end{fulllineitems}

\index{initialize() (asset.Asset method)}

\begin{fulllineitems}
\phantomsection\label{asset:asset.Asset.initialize}\pysiglinewithargsret{\bfcode{initialize}}{\emph{people}}{}
This function initializes the asset at the beginning of the simulation.

\begin{notice}{note}{Note:}
This function is meant to be overridden
\end{notice}
\begin{quote}\begin{description}
\item[{Parameters}] \leavevmode
\textbf{\texttt{people}} (\emph{list{[} person.Person {]}}) -- the Person objects who could be using the asset.

\item[{Returns}] \leavevmode
None

\end{description}\end{quote}

\end{fulllineitems}

\index{print\_category() (asset.Asset method)}

\begin{fulllineitems}
\phantomsection\label{asset:asset.Asset.print_category}\pysiglinewithargsret{\bfcode{print\_category}}{}{}~\begin{quote}

This function represents the category as a string.
\end{quote}
\begin{quote}\begin{description}
\item[{Returns}] \leavevmode
the string representation of the category

\item[{Return type}] \leavevmode
str

\end{description}\end{quote}

\end{fulllineitems}

\index{reset() (asset.Asset method)}

\begin{fulllineitems}
\phantomsection\label{asset:asset.Asset.reset}\pysiglinewithargsret{\bfcode{reset}}{}{}
This function does the following.
\begin{enumerate}
\item {} 
Sets the number of users to zero

\item {} 
Sets the asset's status to idle

\end{enumerate}
\begin{quote}\begin{description}
\item[{Returns}] \leavevmode
None

\end{description}\end{quote}

\end{fulllineitems}

\index{toString() (asset.Asset method)}

\begin{fulllineitems}
\phantomsection\label{asset:asset.Asset.toString}\pysiglinewithargsret{\bfcode{toString}}{}{}~\begin{quote}

This function represents the asset as a string.
\end{quote}
\begin{quote}\begin{description}
\item[{Return msg}] \leavevmode
The string representation of the asset object.

\item[{Return type}] \leavevmode
str

\end{description}\end{quote}

\end{fulllineitems}

\index{update() (asset.Asset method)}

\begin{fulllineitems}
\phantomsection\label{asset:asset.Asset.update}\pysiglinewithargsret{\bfcode{update}}{}{}
This function changes the state of the asset once it is used by a Person.
\begin{enumerate}
\item {} 
Increases the number of people by 1

\item {} 
If the number of users is at the maximum number, set the asset's status to busy

\item {} 
If the number of users is less than the maximum number, set the assets' status to busy but         able to be used by another agent

\end{enumerate}
\begin{quote}\begin{description}
\item[{Returns}] \leavevmode
None

\end{description}\end{quote}

\end{fulllineitems}


\end{fulllineitems}



\section{bed module}
\label{bed::doc}\label{bed:bed-module}\label{bed:module-bed}\index{bed (module)}
This module contains code that enables the agent to use a bed. This class allows access to the sleep ({\hyperref[sleep:sleep.Sleep]{\emph{\code{sleep.Sleep}}}}) activity.

This module contains the following class: {\hyperref[bed:bed.Bed]{\emph{\code{bed.Bed}}}}.

\emph{Module author: Dr. Namdi Brandon}
\index{Bed (class in bed)}

\begin{fulllineitems}
\phantomsection\label{bed:bed.Bed}\pysigline{\strong{class }\code{bed.}\bfcode{Bed}}
Bases: {\hyperref[asset:asset.Asset]{\emph{\code{asset.Asset}}}}

This asset models a bed. It allows the agent to address the Rest ({\hyperref[rest:rest.Rest]{\emph{\code{rest.Rest}}}}) need by doing the     sleep ({\hyperref[sleep:sleep.Sleep]{\emph{\code{sleep.Sleep}}}}) action .

\end{fulllineitems}



\section{bio module}
\label{bio::doc}\label{bio:module-bio}\label{bio:bio-module}\index{bio (module)}
This module contains information about a {\hyperref[person:person.Person]{\emph{\code{person.Person}}}}`s biology.

This module contains the following class: {\hyperref[bio:bio.Bio]{\emph{\code{bio.Bio}}}}.

\emph{Module author: Dr. Namdi Brandon}
\index{Bio (class in bio)}

\begin{fulllineitems}
\phantomsection\label{bio:bio.Bio}\pysigline{\strong{class }\code{bio.}\bfcode{Bio}}
Bases: \code{object}

This class holds the biologically relevant information for a person. This information is:
\begin{itemize}
\item {} 
Age

\item {} 
Gender

\item {} 
Mean / standard deviation of start time for sleeping

\item {} 
Mean / standard deviation of end time for sleeping

\item {} 
Probability distribution function sleep start time / end time

\end{itemize}
\begin{quote}\begin{description}
\item[{Variables}] \leavevmode\begin{itemize}
\item {} 
\textbf{\texttt{age}} (\emph{int}) -- the age {[}years{]}

\item {} 
\textbf{\texttt{gender}} (\emph{int}) -- the gender

\item {} 
\textbf{\texttt{sleep\_dt}} (\emph{int}) -- the duration of time for a sleep event {[}minutes{]}

\item {} 
\textbf{\texttt{sleep\_start\_mean}} (\emph{int}) -- the mean start time for a sleep event {[}minutes{]}

\item {} 
\textbf{\texttt{sleep\_start\_std}} (\emph{int}) -- the standard deviation for a starting a sleep event {[}minutes{]}

\item {} 
\textbf{\texttt{sleep\_start}} (\emph{int}) -- the start time for sleep {[}minutes, time of day{]}

\item {} 
\textbf{\texttt{sleep\_start\_univ}} (\emph{int}) -- the start time for sleep{[}minutes, universal time{]}

\item {} 
\textbf{\texttt{sleep\_end\_mean}} (\emph{int}) -- the mean end time for a sleep event {[}minutes{]}

\item {} 
\textbf{\texttt{sleep\_end\_std}} (\emph{int}) -- the standard deviation for end time for a sleep event {[}minutes{]}

\item {} 
\textbf{\texttt{sleep\_end}} (\emph{int}) -- the end time for sleep{[}minutes, time of day{]}

\item {} 
\textbf{\texttt{sleep\_end\_univ}} (\emph{int}) -- the end time for sleep {[}minutes, universal time{]}

\item {} 
\textbf{\texttt{start\_trunc}} (\emph{int}) -- the number of standard deviations to allow when sampling sleep start time truncated     distribution

\item {} 
\textbf{\texttt{end\_trunc}} (\emph{int}) -- the number of standard deviations to allow when sampling sleep end time truncated     distribution

\item {} 
\textbf{\texttt{f\_sleep\_start}} (\emph{func}) -- the distribution data for start time for sleep

\item {} 
\textbf{\texttt{f\_sleep\_end}} (\emph{func}) -- the distribution data for end time for sleep

\end{itemize}

\end{description}\end{quote}
\index{calc\_awake\_duration() (bio.Bio method)}

\begin{fulllineitems}
\phantomsection\label{bio:bio.Bio.calc_awake_duration}\pysiglinewithargsret{\bfcode{calc\_awake\_duration}}{\emph{t}}{}
This function calculates the amount of time the person is expected to be awake.
\begin{quote}\begin{description}
\item[{Parameters}] \leavevmode
\textbf{\texttt{t}} (\emph{int}) -- time of day {[}minutes{]}

\item[{Returns}] \leavevmode
the duration {[}minutes{]} until the agent is expected to awaken

\end{description}\end{quote}

\end{fulllineitems}

\index{print\_gender() (bio.Bio method)}

\begin{fulllineitems}
\phantomsection\label{bio:bio.Bio.print_gender}\pysiglinewithargsret{\bfcode{print\_gender}}{}{}
This function returns a string representation of gender
\begin{quote}\begin{description}
\item[{Returns}] \leavevmode
the string representation of gender

\item[{Return type}] \leavevmode
str

\end{description}\end{quote}

\end{fulllineitems}

\index{set\_sleep\_params() (bio.Bio method)}

\begin{fulllineitems}
\phantomsection\label{bio:bio.Bio.set_sleep_params}\pysiglinewithargsret{\bfcode{set\_sleep\_params}}{\emph{start\_mean}, \emph{start\_std}, \emph{end\_mean}, \emph{end\_std}}{}
This function sets the biological sleep parameters themselves and the sleep parameter distribution functions.
\begin{quote}\begin{description}
\item[{Parameters}] \leavevmode\begin{itemize}
\item {} 
\textbf{\texttt{start\_mean}} (\emph{int}) -- the mean sleep start time {[}minutes{]}

\item {} 
\textbf{\texttt{start\_std}} (\emph{int}) -- the standard deviation of start time {[}minutes{]}

\item {} 
\textbf{\texttt{end\_mean}} (\emph{int}) -- the mean sleep end time {[}minutes{]}

\item {} 
\textbf{\texttt{end\_std}} (\emph{int}) -- the standard deviation of end time {[}minutes{]}

\end{itemize}

\item[{Returns}] \leavevmode
None

\end{description}\end{quote}

\end{fulllineitems}

\index{toString() (bio.Bio method)}

\begin{fulllineitems}
\phantomsection\label{bio:bio.Bio.toString}\pysiglinewithargsret{\bfcode{toString}}{\emph{do\_decimal=False}}{}
This function represents the {\hyperref[bio:bio.Bio]{\emph{\code{bio.Bio}}}} object as a string.
\begin{quote}\begin{description}
\item[{Parameters}] \leavevmode
\textbf{\texttt{do\_decimal}} (\emph{bool}) -- This controls whether or not to represent the values in time in a                                 decimal (hours) format where {[}1:30pm is 13.5{]} if True or as the minutes                                 in the day if False {[}1:30pm is 13 * 60 + 30{]}.

\item[{Return msg}] \leavevmode
the string representation of the Bio object

\item[{Return type}] \leavevmode
string

\end{description}\end{quote}

\end{fulllineitems}

\index{update\_sleep\_dt() (bio.Bio method)}

\begin{fulllineitems}
\phantomsection\label{bio:bio.Bio.update_sleep_dt}\pysiglinewithargsret{\bfcode{update\_sleep\_dt}}{}{}
This function sets the duration of sleep.
\begin{quote}\begin{description}
\item[{Returns}] \leavevmode
None

\end{description}\end{quote}

\end{fulllineitems}

\index{update\_sleep\_end() (bio.Bio method)}

\begin{fulllineitems}
\phantomsection\label{bio:bio.Bio.update_sleep_end}\pysiglinewithargsret{\bfcode{update\_sleep\_end}}{}{}
This function samples the sleep end time distribution and sets the end time.
\begin{quote}\begin{description}
\item[{Returns}] \leavevmode
None

\end{description}\end{quote}

\end{fulllineitems}

\index{update\_sleep\_end\_univ() (bio.Bio method)}

\begin{fulllineitems}
\phantomsection\label{bio:bio.Bio.update_sleep_end_univ}\pysiglinewithargsret{\bfcode{update\_sleep\_end\_univ}}{\emph{time\_of\_day}, \emph{t\_univ}}{}
This function sets the end time for sleep in terms of universal time.
\begin{quote}\begin{description}
\item[{Parameters}] \leavevmode\begin{itemize}
\item {} 
\textbf{\texttt{time\_of\_day}} (\emph{int}) -- the current time of day {[}minutes{]}

\item {} 
\textbf{\texttt{t\_univ}} (\emph{int}) -- the universal time {[}minutes{]}

\end{itemize}

\item[{Returns}] \leavevmode
None

\end{description}\end{quote}

\end{fulllineitems}

\index{update\_sleep\_start() (bio.Bio method)}

\begin{fulllineitems}
\phantomsection\label{bio:bio.Bio.update_sleep_start}\pysiglinewithargsret{\bfcode{update\_sleep\_start}}{}{}
This function samples the sleep start time distribution and sets the start time.
\begin{quote}\begin{description}
\item[{Returns}] \leavevmode
None

\end{description}\end{quote}

\end{fulllineitems}

\index{update\_sleep\_start\_univ() (bio.Bio method)}

\begin{fulllineitems}
\phantomsection\label{bio:bio.Bio.update_sleep_start_univ}\pysiglinewithargsret{\bfcode{update\_sleep\_start\_univ}}{\emph{time\_of\_day}, \emph{t\_univ}}{}
This function sets the start time for sleep in terms of universal time.
\begin{quote}\begin{description}
\item[{Parameters}] \leavevmode\begin{itemize}
\item {} 
\textbf{\texttt{time\_of\_day}} (\emph{int}) -- the current time of day {[}minutes{]}

\item {} 
\textbf{\texttt{t\_univ}} (\emph{int}) -- the universal time {[}minutes{]}

\end{itemize}

\item[{Returns}] \leavevmode
None

\end{description}\end{quote}

\end{fulllineitems}

\index{update\_time\_univ() (bio.Bio method)}

\begin{fulllineitems}
\phantomsection\label{bio:bio.Bio.update_time_univ}\pysiglinewithargsret{\bfcode{update\_time\_univ}}{\emph{time\_of\_day}, \emph{t\_univ}, \emph{t}}{}
This function updates a time \(t\), which represents sleep start time or end time, to be in the         next occurrence
\begin{quote}\begin{description}
\item[{Parameters}] \leavevmode\begin{itemize}
\item {} 
\textbf{\texttt{time\_of\_day}} (\emph{int}) -- the current time of day {[}minutes{]}

\item {} 
\textbf{\texttt{t\_univ}} (\emph{int}) -- the universal time {[}minutes{]}

\item {} 
\textbf{\texttt{t}} (\emph{int}) -- the time to be set{[}minutes, time of day{]}

\end{itemize}

\item[{Return out}] \leavevmode
the time of the next event in universal time

\item[{Return type}] \leavevmode
int

\end{description}\end{quote}

\end{fulllineitems}


\end{fulllineitems}



\section{commute module}
\label{commute::doc}\label{commute:module-commute}\label{commute:commute-module}\index{commute (module)}
This module contains about activities associated with commuting to and from work. This class is an {\hyperref[activity:activity.Activity]{\emph{\code{activity.Activity}}}} that gives a {\hyperref[person:person.Person]{\emph{\code{person.Person}}}} the ability to commute to/ from work/ school and satisfy the need Travel {\hyperref[travel:travel.Travel]{\emph{\code{travel.Travel}}}}.

This module contains the following classes:
\begin{enumerate}
\item {} 
{\hyperref[commute:commute.Commute]{\emph{\code{commute.Commute}}}} (general commuting capability)

\item {} 
{\hyperref[commute:commute.Commute_To_Work]{\emph{\code{commute.Commute\_To\_Work}}}} (commute to work/ school)

\item {} 
{\hyperref[commute:commute.Commute_From_Work]{\emph{\code{commute.Commute\_From\_Work}}}} (commute from work/ school)

\end{enumerate}

\emph{Module author: Dr. Namdi Brandon}
\index{Commute (class in commute)}

\begin{fulllineitems}
\phantomsection\label{commute:commute.Commute}\pysigline{\strong{class }\code{commute.}\bfcode{Commute}}
Bases: {\hyperref[activity:activity.Activity]{\emph{\code{activity.Activity}}}}

This class allows for commuting. This class is to be derived from.
\index{end() (commute.Commute method)}

\begin{fulllineitems}
\phantomsection\label{commute:commute.Commute.end}\pysiglinewithargsret{\bfcode{end}}{\emph{p}, \emph{local}}{}
This function handles the end of an Activity.
\begin{quote}\begin{description}
\item[{Parameters}] \leavevmode\begin{itemize}
\item {} 
\textbf{\texttt{p}} ({\hyperref[person:person.Person]{\emph{\emph{person.Person}}}}) -- the person of interest

\item {} 
\textbf{\texttt{local}} (\emph{int}) -- the local location (work or home)

\end{itemize}

\item[{Returns}] \leavevmode
None

\end{description}\end{quote}

\end{fulllineitems}

\index{end\_commute() (commute.Commute method)}

\begin{fulllineitems}
\phantomsection\label{commute:commute.Commute.end_commute}\pysiglinewithargsret{\bfcode{end\_commute}}{\emph{p}}{}
This function ends the commuting activity.

\begin{notice}{note}{Note:}
This function is to be overridden
\end{notice}
\begin{quote}\begin{description}
\item[{Parameters}] \leavevmode
\textbf{\texttt{p}} ({\hyperref[person:person.Person]{\emph{\emph{person.Person}}}}) -- the person of interest

\item[{Returns}] \leavevmode
None

\end{description}\end{quote}

\end{fulllineitems}

\index{start() (commute.Commute method)}

\begin{fulllineitems}
\phantomsection\label{commute:commute.Commute.start}\pysiglinewithargsret{\bfcode{start}}{\emph{p}}{}
This handles the start of the commute activity.
\begin{itemize}
\item {} 
If the current location of person is at home, the person is going to work, so set the         location to \code{location.OFF\_SITE}

\item {} 
If the current location of the person is off site, the person is going back home, so         set the location to \code{location.HOME}

\end{itemize}
\begin{quote}\begin{description}
\item[{Parameters}] \leavevmode
\textbf{\texttt{p}} ({\hyperref[person:person.Person]{\emph{\emph{person.Person}}}}) -- the person of interest

\item[{Returns}] \leavevmode
None

\end{description}\end{quote}

\end{fulllineitems}

\index{start\_commute() (commute.Commute method)}

\begin{fulllineitems}
\phantomsection\label{commute:commute.Commute.start_commute}\pysiglinewithargsret{\bfcode{start\_commute}}{\emph{p}}{}
This function sets the variables pertaining to starting the commute activity.
\begin{enumerate}
\item {} 
set the status of the person to \code{location.TRANSIT}

\item {} 
set the location of the asset to \code{location.TRANSIT}

\item {} 
set the person's state start time of the commute

\item {} 
set the person's state end time for the commute

\item {} 
update the asset

\item {} 
update the scheduler for the travel need for the end of the commute

\end{enumerate}
\begin{quote}\begin{description}
\item[{Parameters}] \leavevmode
\textbf{\texttt{p}} ({\hyperref[person:person.Person]{\emph{\emph{person.Person}}}}) -- the person of interest

\item[{Returns}] \leavevmode
None

\end{description}\end{quote}

\end{fulllineitems}


\end{fulllineitems}

\index{Commute\_From\_Work (class in commute)}

\begin{fulllineitems}
\phantomsection\label{commute:commute.Commute_From_Work}\pysigline{\strong{class }\code{commute.}\bfcode{Commute\_From\_Work}}
Bases: {\hyperref[commute:commute.Commute]{\emph{\code{commute.Commute}}}}

This class allows for the activity: commuting from work.
\index{advertise() (commute.Commute\_From\_Work method)}

\begin{fulllineitems}
\phantomsection\label{commute:commute.Commute_From_Work.advertise}\pysiglinewithargsret{\bfcode{advertise}}{\emph{p}}{}
This function calculates the score of an activities advertisement         advertise the score to commute.
\begin{enumerate}
\item {} 
calculate advertisement only if the person is located at work (off-site)

\item {} 
calculate the score
\begin{quote}
\begin{gather}
\begin{split}S = \begin{cases}
0  & n_{travel}(t) > \lambda \\
W( n_{travel}(t) ) - W( n_{travel}(t + \Delta{t} )) & n_{travel}(t) \le \lambda
\end{cases}\end{split}\notag
\end{gather}\end{quote}

\end{enumerate}
\begin{quote}\begin{description}
\item[{Parameters}] \leavevmode
\textbf{\texttt{p}} ({\hyperref[person:person.Person]{\emph{\emph{person.Person}}}}) -- the person of interest

\item[{Returns}] \leavevmode
the advertised score

\item[{Return type}] \leavevmode
float

\end{description}\end{quote}

\end{fulllineitems}

\index{calc\_end\_time() (commute.Commute\_From\_Work method)}

\begin{fulllineitems}
\phantomsection\label{commute:commute.Commute_From_Work.calc_end_time}\pysiglinewithargsret{\bfcode{calc\_end\_time}}{\emph{p}}{}~\begin{enumerate}
\item {} 
calculate the end time (minutes, universal time) of the commute

\item {} 
set the the end time in the person's state

\end{enumerate}
\begin{quote}\begin{description}
\item[{Parameters}] \leavevmode
\textbf{\texttt{p}} ({\hyperref[person:person.Person]{\emph{\emph{person.Person}}}}) -- the person of interest

\item[{Returns}] \leavevmode
None

\end{description}\end{quote}

\end{fulllineitems}

\index{end() (commute.Commute\_From\_Work method)}

\begin{fulllineitems}
\phantomsection\label{commute:commute.Commute_From_Work.end}\pysiglinewithargsret{\bfcode{end}}{\emph{p}}{}
This function handles the end of an Activity.
\begin{quote}\begin{description}
\item[{Parameters}] \leavevmode
\textbf{\texttt{p}} ({\hyperref[person:person.Person]{\emph{\emph{person.Person}}}}) -- the person of interest

\item[{Returns}] \leavevmode
None

\end{description}\end{quote}

\end{fulllineitems}

\index{end\_commute() (commute.Commute\_From\_Work method)}

\begin{fulllineitems}
\phantomsection\label{commute:commute.Commute_From_Work.end_commute}\pysiglinewithargsret{\bfcode{end\_commute}}{\emph{p}}{}
This function sets the variables pertaining to ending the commute activity.
\begin{enumerate}
\item {} 
Sets the person's state to idle(\code{state.IDLE})

\item {} 
Updates the asset's state and number of users

\item {} 
Sets the travel magnitude

\item {} 
Sets the work magnitude to \code{need.MAG\_WORK}, to allow for work         to be the next activity, even if commute ends begin the work-start time

\item {} 
Sets the person's state's end time

\end{enumerate}
\begin{quote}\begin{description}
\item[{Parameters}] \leavevmode\begin{itemize}
\item {} 
\textbf{\texttt{p}} ({\hyperref[person:person.Person]{\emph{\emph{person.Person}}}}) -- person of interest

\item {} 
\textbf{\texttt{destination}} (\emph{int}) -- a local location where the commute ends (home or workplace)

\end{itemize}

\item[{Returns}] \leavevmode
None

\end{description}\end{quote}

\end{fulllineitems}

\index{start() (commute.Commute\_From\_Work method)}

\begin{fulllineitems}
\phantomsection\label{commute:commute.Commute_From_Work.start}\pysiglinewithargsret{\bfcode{start}}{\emph{p}}{}
This handles the start of the commute activity.

If the current location of person is at home, the person is going to work, so set the         location to \code{location.OFF\_SITE}

If the current location of the person is off site, the person is going back home, so         set the location to \code{location.HOME}
\begin{quote}\begin{description}
\item[{Parameters}] \leavevmode
\textbf{\texttt{p}} ({\hyperref[person:person.Person]{\emph{\emph{person.Person}}}}) -- the person of interest

\item[{Returns}] \leavevmode
None

\end{description}\end{quote}

\end{fulllineitems}


\end{fulllineitems}

\index{Commute\_To\_Work (class in commute)}

\begin{fulllineitems}
\phantomsection\label{commute:commute.Commute_To_Work}\pysigline{\strong{class }\code{commute.}\bfcode{Commute\_To\_Work}}
Bases: {\hyperref[commute:commute.Commute]{\emph{\code{commute.Commute}}}}

This class allows for the activity: commute to work
\index{advertise() (commute.Commute\_To\_Work method)}

\begin{fulllineitems}
\phantomsection\label{commute:commute.Commute_To_Work.advertise}\pysiglinewithargsret{\bfcode{advertise}}{\emph{p}}{}
This function calculates the score of an activities advertisement         advertise the score to commute.
\begin{enumerate}
\item {} 
calculate advertisement only if the person is located at work (off-site)

\item {} 
calculate the score
\begin{quote}
\begin{gather}
\begin{split}S = \begin{cases}
0  & n_{travel}(t) > \lambda \\
W( n_{travel}(t) ) - W( n_{travel}(t + \Delta{t} )) & n_{travel}(t) \le \lambda
\end{cases}\end{split}\notag
\end{gather}\end{quote}

\end{enumerate}
\begin{quote}\begin{description}
\item[{Parameters}] \leavevmode
\textbf{\texttt{p}} ({\hyperref[person:person.Person]{\emph{\emph{person.Person}}}}) -- the person of interest

\item[{Return score}] \leavevmode
the advertisement score

\item[{Return type}] \leavevmode
float

\end{description}\end{quote}

\end{fulllineitems}

\index{calc\_end\_time() (commute.Commute\_To\_Work method)}

\begin{fulllineitems}
\phantomsection\label{commute:commute.Commute_To_Work.calc_end_time}\pysiglinewithargsret{\bfcode{calc\_end\_time}}{\emph{p}}{}
Given the commute duration, store the end time.
\begin{enumerate}
\item {} 
calculate the end time {[}universal time{]} of the commute.

\item {} 
store the end time in the person.state

\end{enumerate}
\begin{quote}\begin{description}
\item[{Parameters}] \leavevmode
\textbf{\texttt{p}} ({\hyperref[person:person.Person]{\emph{\emph{person.Person}}}}) -- the person of interest

\item[{Returns}] \leavevmode
None

\end{description}\end{quote}

\end{fulllineitems}

\index{end() (commute.Commute\_To\_Work method)}

\begin{fulllineitems}
\phantomsection\label{commute:commute.Commute_To_Work.end}\pysiglinewithargsret{\bfcode{end}}{\emph{p}}{}
This function handles the logistics of ending the commute to work activity.
\begin{quote}\begin{description}
\item[{Parameters}] \leavevmode
\textbf{\texttt{p}} ({\hyperref[person:person.Person]{\emph{\emph{person.Person}}}}) -- the person of interest

\item[{Returns}] \leavevmode
None

\end{description}\end{quote}

\end{fulllineitems}

\index{end\_commute() (commute.Commute\_To\_Work method)}

\begin{fulllineitems}
\phantomsection\label{commute:commute.Commute_To_Work.end_commute}\pysiglinewithargsret{\bfcode{end\_commute}}{\emph{p}}{}
This function handles the logistics concerning ending the commute.
\begin{enumerate}
\item {} 
the asset is freed up from use

\item {} 
the magnitude of the travel need is set \(n_{travel}=1\)

\item {} 
the person's state is set to idle (\code{state.IDLE})

\item {} 
the person's location is set to the location of the job

\item {} 
the asset's location is set to the location of the job

\item {} 
the person's income need is set to \(n_{income}=\eta_{work}\)

\item {} 
update the commute to work duration

\item {} 
calculate the time until the next leave work event

\item {} 
update the schedule for the travel need

\end{enumerate}
\begin{quote}\begin{description}
\item[{Parameters}] \leavevmode
\textbf{\texttt{p}} ({\hyperref[person:person.Person]{\emph{\emph{person.Person}}}}) -- the person of interest

\item[{Returns}] \leavevmode


\end{description}\end{quote}

\end{fulllineitems}

\index{start() (commute.Commute\_To\_Work method)}

\begin{fulllineitems}
\phantomsection\label{commute:commute.Commute_To_Work.start}\pysiglinewithargsret{\bfcode{start}}{\emph{p}}{}
This function handles the start of the commute to work activity. If the current location of person is         at home, the person is going to work, so set the location to workplace location (\code{location.OFF\_SITE})
\begin{quote}\begin{description}
\item[{Parameters}] \leavevmode
\textbf{\texttt{p}} ({\hyperref[person:person.Person]{\emph{\emph{person.Person}}}}) -- the person of interest

\item[{Returns}] \leavevmode
None

\end{description}\end{quote}

\end{fulllineitems}

\index{start\_commute() (commute.Commute\_To\_Work method)}

\begin{fulllineitems}
\phantomsection\label{commute:commute.Commute_To_Work.start_commute}\pysiglinewithargsret{\bfcode{start\_commute}}{\emph{p}}{}
This function sets the variables pertaining to starting the commute to work activity.
\begin{enumerate}
\item {} 
set the person's status to \code{state.TRANSIT}

\item {} 
set the asset's location to \code{location.TRANSIT}

\item {} 
set the person's state start time to the current time

\item {} 
calculate the end time of commute to work

\item {} 
update the asset's update

\item {} 
update the scheduler for the travel need to take into account the end of the commute

\item {} 
update the scheduler for the income need to take into account the end of the commute

\end{enumerate}
\begin{quote}\begin{description}
\item[{Parameters}] \leavevmode
\textbf{\texttt{p}} ({\hyperref[person:person.Person]{\emph{\emph{person.Person}}}}) -- the person of interest

\item[{Returns}] \leavevmode
None

\end{description}\end{quote}

\end{fulllineitems}


\end{fulllineitems}



\section{diary module}
\label{diary::doc}\label{diary:module-diary}\label{diary:diary-module}\index{diary (module)}
This module contains code that governs the activity-diaries. Each activity diary contains
dataframes that store the activity-diaries for each person. The activity-diaries are the
output of the Agent-Based Model of Human Activity Patterns (ABMHAP) simulation.

This module contains class {\hyperref[diary:diary.Diary]{\emph{\code{diary.Diary}}}}.

\emph{Module author: Dr. Namdi Brandon}
\index{Diary (class in diary)}

\begin{fulllineitems}
\phantomsection\label{diary:diary.Diary}\pysiglinewithargsret{\strong{class }\code{diary.}\bfcode{Diary}}{\emph{t}, \emph{act}, \emph{local}}{}
Bases: \code{object}

This class represents the activity-diaries for a person.
\begin{quote}\begin{description}
\item[{Parameters}] \leavevmode\begin{itemize}
\item {} 
\textbf{\texttt{t}} (\emph{numpy.ndarray}) -- the start times for each activity {[}universal time, minutes{]}

\item {} 
\textbf{\texttt{act}} (\emph{numpy.ndarray}) -- the activity code done at each time step {[}integer{]} (flattened array)

\item {} 
\textbf{\texttt{local}} (\emph{numpy.ndarray}) -- the history of location codes done by a person

\end{itemize}

\item[{Variables}] \leavevmode\begin{itemize}
\item {} 
\textbf{\texttt{colnames}} (\emph{list}) -- the column names for the activity diary in order

\item {} 
\textbf{\texttt{df}} (\emph{pandas.core.frame.DataFrame}) -- the activity-diary

\end{itemize}

\end{description}\end{quote}
\index{create\_activity\_diary() (diary.Diary method)}

\begin{fulllineitems}
\phantomsection\label{diary:diary.Diary.create_activity_diary}\pysiglinewithargsret{\bfcode{create\_activity\_diary}}{\emph{t}, \emph{act}, \emph{local}}{}
This function creates the activity diary for a given agent in the simulation.

The activity diary contains:
\begin{enumerate}
\item {} 
The start-time and end-time for each activity

\item {} 
The activity code

\end{enumerate}
\begin{quote}\begin{description}
\item[{Parameters}] \leavevmode\begin{itemize}
\item {} 
\textbf{\texttt{t}} (\emph{numpy.ndarray}) -- the simulation times {[}universal time, minutes{]}

\item {} 
\textbf{\texttt{act}} (\emph{numpy.ndarray}) -- the activity code done at each time step {[}integer{]} (flattened array)

\end{itemize}

\item[{Returns}] \leavevmode
a tuple containing the following: the array indices for each activity grouping, the activity diaries         in a numerical format, the activity diary in a string format, and the column names for each data type

\end{description}\end{quote}

Each diary is a tuple that contains the following:
\begin{enumerate}
\item {} 
the day number of the start of the activity

\item {} 
the (start-time, end-time) for the activity event

\item {} 
the activity code for the activity event

\item {} 
the location of the event

\end{enumerate}

\end{fulllineitems}

\index{get\_day\_end() (diary.Diary method)}

\begin{fulllineitems}
\phantomsection\label{diary:diary.Diary.get_day_end}\pysiglinewithargsret{\bfcode{get\_day\_end}}{\emph{day\_start}, \emph{start}, \emph{dt}}{}
This function gets the day that an activity ends.
\begin{quote}\begin{description}
\item[{Parameters}] \leavevmode\begin{itemize}
\item {} 
\textbf{\texttt{day\_start}} (\emph{numpy.ndarray}) -- the day an activity starts

\item {} 
\textbf{\texttt{start}} (\emph{numpy.ndarray}) -- the time an activity starts {[}hours{]}

\item {} 
\textbf{\texttt{dt}} (\emph{numpy.ndarray}) -- the duration for an activity {[}hours{]}

\end{itemize}

\item[{Returns}] \leavevmode
the day an activity ends

\item[{Return type}] \leavevmode
numpy.ndarray

\end{description}\end{quote}

\end{fulllineitems}

\index{get\_weekday\_data() (diary.Diary method)}

\begin{fulllineitems}
\phantomsection\label{diary:diary.Diary.get_weekday_data}\pysiglinewithargsret{\bfcode{get\_weekday\_data}}{\emph{df=None}}{}
This function pulls out data that only corresponds to the weekday data.
\begin{quote}\begin{description}
\item[{Parameters}] \leavevmode
\textbf{\texttt{df}} (\emph{pandas.core.frame.DataFrame}) -- the activity-diary of interest. If df is None, then use the dataframe         associated with the diary object

\item[{Returns}] \leavevmode
the activity-diary of data that occur on weekdays

\end{description}\end{quote}

\end{fulllineitems}

\index{get\_weekday\_idx() (diary.Diary method)}

\begin{fulllineitems}
\phantomsection\label{diary:diary.Diary.get_weekday_idx}\pysiglinewithargsret{\bfcode{get\_weekday\_idx}}{\emph{df=None}}{}
Get the indices of the data that occurs on weekdays. An activity is considered to be on the weekday if         the activity \textbf{ends} on Monday - Friday.
\begin{quote}\begin{description}
\item[{Parameters}] \leavevmode
\textbf{\texttt{df}} (\emph{pandas.core.frame.DataFrame}) -- the activity-diary of interest. If df is None, then use the dataframe         associated with the diary object

\item[{Returns}] \leavevmode
boolean indices of which activities end during the weekend

\item[{Return type}] \leavevmode
numpy.ndarray

\end{description}\end{quote}

\end{fulllineitems}

\index{get\_weekend\_data() (diary.Diary method)}

\begin{fulllineitems}
\phantomsection\label{diary:diary.Diary.get_weekend_data}\pysiglinewithargsret{\bfcode{get\_weekend\_data}}{\emph{df=None}}{}
This function pulls out data that only corresponds to the weekend data.
\begin{quote}\begin{description}
\item[{Parameters}] \leavevmode
\textbf{\texttt{df}} (\emph{pandas.core.frame.DataFrame}) -- the activity-diary of interest. If df is None, the use  the dataframe         associated with the current diary object

\item[{Returns}] \leavevmode
an activity-diary of data that occurs on weekends

\end{description}\end{quote}

\end{fulllineitems}

\index{get\_weekend\_idx() (diary.Diary method)}

\begin{fulllineitems}
\phantomsection\label{diary:diary.Diary.get_weekend_idx}\pysiglinewithargsret{\bfcode{get\_weekend\_idx}}{\emph{df=None}}{}
Get the indices of the data that occurs on weekend. An activity is considered to be on the weekend if         the activity \textbf{ends} on Saturday or Sunday.
\begin{quote}\begin{description}
\item[{Parameters}] \leavevmode
\textbf{\texttt{df}} (\emph{pandas.core.frame.DataFrame}) -- the activity-diary of interest. If df is None, then use the dataframe         associated with the diary object

\item[{Returns}] \leavevmode
boolean indices of which activities end during the weekend

\item[{Return type}] \leavevmode
numpy.ndarray

\end{description}\end{quote}

\end{fulllineitems}

\index{group\_activity() (diary.Diary method)}

\begin{fulllineitems}
\phantomsection\label{diary:diary.Diary.group_activity}\pysiglinewithargsret{\bfcode{group\_activity}}{\emph{t}, \emph{y}}{}
This function groups activities in chronological order.
\begin{quote}\begin{description}
\item[{Parameters}] \leavevmode\begin{itemize}
\item {} 
\textbf{\texttt{t}} (\emph{numpy.ndarray}) -- the start time for activities

\item {} 
\textbf{\texttt{y}} (\emph{numpy.ndarray}) -- the activity code that corresponds with the respective time

\end{itemize}

\item[{Returns}] \leavevmode
a list of each unique group-lists. Each group-list contains a tuple         for (time step, activity code)

\end{description}\end{quote}

\end{fulllineitems}

\index{group\_activity\_indices() (diary.Diary method)}

\begin{fulllineitems}
\phantomsection\label{diary:diary.Diary.group_activity_indices}\pysiglinewithargsret{\bfcode{group\_activity\_indices}}{\emph{groups}}{}
This function returns the indices for each continuous activity in chronological order.

\begin{notice}{note}{Note:}
The output is the time step number \textbf{not} the value of time
\end{notice}
\begin{quote}\begin{description}
\item[{Parameters}] \leavevmode
\textbf{\texttt{groups}} (\emph{list}) -- a list of tuples of (time step, activity code)

\item[{Returns}] \leavevmode


\end{description}\end{quote}

\end{fulllineitems}

\index{group\_activity\_key() (diary.Diary method)}

\begin{fulllineitems}
\phantomsection\label{diary:diary.Diary.group_activity_key}\pysiglinewithargsret{\bfcode{group\_activity\_key}}{\emph{x}}{}
This is the key function used in groupby in order to group consecutive time-step-activity pairs. This is         necessary for creating an activity diary.
\begin{quote}\begin{description}
\item[{Parameters}] \leavevmode
\textbf{\texttt{x}} (\emph{tuple}) -- the data in the form of ( index, (time step, activity code) )

\end{description}\end{quote}

:return the key for sorting ( , activity code)
:rtype: tuple

\end{fulllineitems}

\index{is\_weekend() (diary.Diary method)}

\begin{fulllineitems}
\phantomsection\label{diary:diary.Diary.is_weekend}\pysiglinewithargsret{\bfcode{is\_weekend}}{\emph{day}}{}
This function returns true if a day is in the weekend and false if it's in a weekday.
\begin{quote}\begin{description}
\item[{Parameters}] \leavevmode
\textbf{\texttt{day}} (\emph{numpy.ndarray}) -- the day of the weekend

\item[{Returns}] \leavevmode
boolean index of whether or not a day is in the weekend (True) or not (False)

\item[{Return type}] \leavevmode
numpy.ndarray

\end{description}\end{quote}

\end{fulllineitems}

\index{same\_day() (diary.Diary method)}

\begin{fulllineitems}
\phantomsection\label{diary:diary.Diary.same_day}\pysiglinewithargsret{\bfcode{same\_day}}{\emph{start}, \emph{dt}}{}
This function returns true if the activity starts and ends on the same day.
\begin{quote}\begin{description}
\item[{Parameters}] \leavevmode\begin{itemize}
\item {} 
\textbf{\texttt{start}} (\emph{numpy.ndarray}) -- the time an activity starts {[}hours{]}

\item {} 
\textbf{\texttt{dt}} (\emph{numpy.ndarray}) -- the time an activity ends {[}hours{]}

\end{itemize}

\item[{Returns}] \leavevmode
a boolean index of whether or not an activity started and ended on the same day

\item[{Return type}] \leavevmode
numpy.ndarray

\end{description}\end{quote}

\end{fulllineitems}

\index{toString() (diary.Diary method)}

\begin{fulllineitems}
\phantomsection\label{diary:diary.Diary.toString}\pysiglinewithargsret{\bfcode{toString}}{}{}
This expresses the diary as a string
\begin{quote}\begin{description}
\item[{Returns}] \leavevmode
an expression of the diary as a string

\item[{Return type}] \leavevmode
string

\end{description}\end{quote}

\end{fulllineitems}


\end{fulllineitems}



\section{eat module}
\label{eat::doc}\label{eat:eat-module}\label{eat:module-eat}\index{eat (module)}
This module contains information about the activities associated with eating. This class is an {\hyperref[activity:activity.Activity]{\emph{\code{activity.Activity}}}} that gives a {\hyperref[person:person.Person]{\emph{\code{person.Person}}}} the ability to eat and satisfy the need {\hyperref[hunger:hunger.Hunger]{\emph{\code{hunger.Hunger}}}}.

This module contains the following classes:
\begin{enumerate}
\item {} 
{\hyperref[eat:eat.Eat]{\emph{\code{eat.Eat}}}} (general eating capabilities)

\item {} 
{\hyperref[eat:eat.Eat_Breakfast]{\emph{\code{eat.Eat\_Breakfast}}}} (eating breakfast)

\item {} 
{\hyperref[eat:eat.Eat_Lunch]{\emph{\code{eat.Eat\_Lunch}}}} (eating lunch)

\item {} 
{\hyperref[eat:eat.Eat_Dinner]{\emph{\code{eat.Eat\_Dinner}}}} (eating dinner)

\end{enumerate}

\emph{Module author: Dr. Namdi Brandon}
\index{Eat (class in eat)}

\begin{fulllineitems}
\phantomsection\label{eat:eat.Eat}\pysigline{\strong{class }\code{eat.}\bfcode{Eat}}
Bases: {\hyperref[activity:activity.Activity]{\emph{\code{activity.Activity}}}}

This class has general capabilities that allow the person to eat in order to satisfy {\hyperref[hunger:hunger.Hunger]{\emph{\code{hunger.Hunger}}}}.     This class acts as a parent class and is expected to inherited.
\index{advertise() (eat.Eat method)}

\begin{fulllineitems}
\phantomsection\label{eat:eat.Eat.advertise}\pysiglinewithargsret{\bfcode{advertise}}{\emph{p}}{}
This function handles advertising the score to an agent. This function returns 0.

\begin{notice}{note}{Note:}
This function should be overloaded when inherited.
\end{notice}
\begin{quote}\begin{description}
\item[{Parameters}] \leavevmode
\textbf{\texttt{p}} ({\hyperref[person:person.Person]{\emph{\emph{person.Person}}}}) -- the person of interest

\item[{Returns}] \leavevmode
the score (0)

\item[{Return type}] \leavevmode
float

\end{description}\end{quote}

\end{fulllineitems}

\index{advertise\_help() (eat.Eat method)}

\begin{fulllineitems}
\phantomsection\label{eat:eat.Eat.advertise_help}\pysiglinewithargsret{\bfcode{advertise\_help}}{\emph{p}, \emph{dt}}{}
This function does some of the logistics needed for {\hyperref[eat:eat.Eat.advertise]{\emph{\code{advertise()}}}}.

This function does the following:
\begin{enumerate}
\item {} 
sets the suggested recharge rate for hunger

\item {} 
calculates the score

\end{enumerate}
\begin{quote}\begin{description}
\item[{Parameters}] \leavevmode\begin{itemize}
\item {} 
\textbf{\texttt{p}} ({\hyperref[person:person.Person]{\emph{\emph{person.Person}}}}) -- the person who is being advertised to

\item {} 
\textbf{\texttt{dt}} (\emph{float}) -- the duration of the activity

\end{itemize}

\item[{Returns}] \leavevmode
the score

\item[{Return type}] \leavevmode
float

\end{description}\end{quote}

\end{fulllineitems}

\index{advertise\_interruption() (eat.Eat method)}

\begin{fulllineitems}
\phantomsection\label{eat:eat.Eat.advertise_interruption}\pysiglinewithargsret{\bfcode{advertise\_interruption}}{\emph{p}}{}
This function calculates the score of an activity advertisement when a person is going to interrupt an         ongoing activity in order to do an eating activity.

This function does the following:
\begin{enumerate}
\item {} 
temporarily sets the satiation of hunger  \(n_{hunger}(t) = \eta_{interruption}\)

\item {} 
calculate the score advertised for the potential eating activity that will interrupt a current activity

\item {} 
restores the the satiation for hunger to the original value

\end{enumerate}
\begin{quote}\begin{description}
\item[{Parameters}] \leavevmode
\textbf{\texttt{p}} ({\hyperref[person:person.Person]{\emph{\emph{person.Person}}}}) -- the person of interest

\item[{Return score}] \leavevmode
the value of the advertisement

\item[{Return type}] \leavevmode
float

\end{description}\end{quote}

\end{fulllineitems}

\index{end() (eat.Eat method)}

\begin{fulllineitems}
\phantomsection\label{eat:eat.Eat.end}\pysiglinewithargsret{\bfcode{end}}{\emph{p}}{}
This function ends the eat activity.
\begin{quote}\begin{description}
\item[{Parameters}] \leavevmode
\textbf{\texttt{p}} ({\hyperref[person:person.Person]{\emph{\emph{person.Person}}}}) -- the person whose activity is ending

\item[{Returns}] \leavevmode
None

\end{description}\end{quote}

\end{fulllineitems}

\index{end\_meal() (eat.Eat method)}

\begin{fulllineitems}
\phantomsection\label{eat:eat.Eat.end_meal}\pysiglinewithargsret{\bfcode{end\_meal}}{\emph{p}}{}
This function ends the eat activity by doing the following.
\begin{enumerate}
\item {} 
frees the person's use of the asset

\item {} 
sets the state to idle (\code{state.IDLE})

\item {} 
sets the satiation of hunger

\item {} 
set the current meal for the next day

\item {} 
set any skipped meals to be on the next day

\item {} 
find the the next meal

\item {} 
sets the decay rate of hunger

\item {} 
update the scheduler so that hunger will trigger the schedule to stop at the next meal

\item {} 
set the next meal to the current meal

\end{enumerate}
\begin{quote}\begin{description}
\item[{Parameters}] \leavevmode
\textbf{\texttt{p}} ({\hyperref[person:person.Person]{\emph{\emph{person.Person}}}}) -- The person whose meal is ending.

\item[{Returns}] \leavevmode
None

\end{description}\end{quote}

\end{fulllineitems}

\index{set\_end\_time() (eat.Eat method)}

\begin{fulllineitems}
\phantomsection\label{eat:eat.Eat.set_end_time}\pysiglinewithargsret{\bfcode{set\_end\_time}}{\emph{p}}{}
This function returns the end time of eating (universal time).
\begin{quote}\begin{description}
\item[{Parameters}] \leavevmode
\textbf{\texttt{p}} ({\hyperref[person:person.Person]{\emph{\emph{person.Person}}}}) -- the person of interest.

\item[{Return t\_end}] \leavevmode
the end time of eating {[}minutes, universal time{]}

\item[{Return type}] \leavevmode
int

\end{description}\end{quote}

\end{fulllineitems}

\index{start() (eat.Eat method)}

\begin{fulllineitems}
\phantomsection\label{eat:eat.Eat.start}\pysiglinewithargsret{\bfcode{start}}{\emph{p}}{}
This function starts the eating activity.
\begin{quote}\begin{description}
\item[{Parameters}] \leavevmode
\textbf{\texttt{p}} ({\hyperref[person:person.Person]{\emph{\emph{person.Person}}}}) -- The person whose activity is starting.

\item[{Returns}] \leavevmode
None

\end{description}\end{quote}

\end{fulllineitems}

\index{start\_meal() (eat.Eat method)}

\begin{fulllineitems}
\phantomsection\label{eat:eat.Eat.start_meal}\pysiglinewithargsret{\bfcode{start\_meal}}{\emph{p}}{}
This function starts the eat activity by doing the following.
\begin{enumerate}
\item {} 
sets the person's state to busy (\code{state.BUSY})

\item {} 
set the decay rate of hunger to 0

\item {} 
store the start time to the state

\item {} 
sets the end time

\item {} 
sets the hunger recharge rate

\item {} 
updates the asset's state and number of users

\item {} 
update the schedule for the hunger need to trigger when the eat activity is scheduled to end

\end{enumerate}
\begin{quote}\begin{description}
\item[{Parameters}] \leavevmode
\textbf{\texttt{p}} ({\hyperref[person:person.Person]{\emph{\emph{person.Person}}}}) -- the person who is starting the meal

\item[{Returns}] \leavevmode
None

\end{description}\end{quote}

\end{fulllineitems}

\index{test\_func() (eat.Eat method)}

\begin{fulllineitems}
\phantomsection\label{eat:eat.Eat.test_func}\pysiglinewithargsret{\bfcode{test\_func}}{\emph{p}}{}
\end{fulllineitems}


\end{fulllineitems}

\index{Eat\_Breakfast (class in eat)}

\begin{fulllineitems}
\phantomsection\label{eat:eat.Eat_Breakfast}\pysigline{\strong{class }\code{eat.}\bfcode{Eat\_Breakfast}}
Bases: {\hyperref[eat:eat.Eat]{\emph{\code{eat.Eat}}}}

This class is used to handle the logistics for eating breakfast.
\index{advertise() (eat.Eat\_Breakfast method)}

\begin{fulllineitems}
\phantomsection\label{eat:eat.Eat_Breakfast.advertise}\pysiglinewithargsret{\bfcode{advertise}}{\emph{p}}{}
This function calculates the score of an activities advertisement to a person.

Advertise to the agent if the following conditions are met
\begin{enumerate}
\item {} 
the current meal is breakfast

\item {} 
the agent's location is at home (\code{location.HOME})

\item {} 
calculate the score

\end{enumerate}
\begin{gather}
\begin{split}S = \begin{cases}
0  & n_{hunger}(t) > \lambda \\
W( n_{hunger}(t) ) - W( n_{hunger}(t + \Delta{t} )) & n_{hunger}(t) \le \lambda
\end{cases}\end{split}\notag
\end{gather}\begin{quote}\begin{description}
\item[{Parameters}] \leavevmode
\textbf{\texttt{p}} ({\hyperref[person:person.Person]{\emph{\emph{person.Person}}}}) -- the person of interest

\item[{Return score}] \leavevmode
the advertised score of doing the eat breakfast activity

\item[{Return type}] \leavevmode
float

\end{description}\end{quote}

\end{fulllineitems}

\index{end\_meal() (eat.Eat\_Breakfast method)}

\begin{fulllineitems}
\phantomsection\label{eat:eat.Eat_Breakfast.end_meal}\pysiglinewithargsret{\bfcode{end\_meal}}{\emph{p}}{}
This function handles the logistics for ending the eat activity by doing the following.
\begin{enumerate}
\item {} 
call \code{eat.end\_meal()}

\item {} 
If planning to skip lunch, update the lunch event to be the next day

\end{enumerate}
\begin{quote}\begin{description}
\item[{Parameters}] \leavevmode
\textbf{\texttt{p}} ({\hyperref[person:person.Person]{\emph{\emph{person.Person}}}}) -- the person who's meal is ending

\item[{Returns}] \leavevmode


\end{description}\end{quote}

\end{fulllineitems}

\index{start\_meal() (eat.Eat\_Breakfast method)}

\begin{fulllineitems}
\phantomsection\label{eat:eat.Eat_Breakfast.start_meal}\pysiglinewithargsret{\bfcode{start\_meal}}{\emph{p}}{}
This function handles the logistics for starting the eat activity by doing the following
\begin{enumerate}
\item {} 
set the current meal to breakfast

\item {} 
call \code{eat.start\_meal()}

\end{enumerate}
\begin{quote}\begin{description}
\item[{Parameters}] \leavevmode
\textbf{\texttt{p}} ({\hyperref[person:person.Person]{\emph{\emph{person.Person}}}}) -- the person who is starting the eat activity

\item[{Returns}] \leavevmode


\end{description}\end{quote}

\end{fulllineitems}


\end{fulllineitems}

\index{Eat\_Dinner (class in eat)}

\begin{fulllineitems}
\phantomsection\label{eat:eat.Eat_Dinner}\pysigline{\strong{class }\code{eat.}\bfcode{Eat\_Dinner}}
Bases: {\hyperref[eat:eat.Eat]{\emph{\code{eat.Eat}}}}

This class is used to handle the logistics for eating dinner.
\index{advertise() (eat.Eat\_Dinner method)}

\begin{fulllineitems}
\phantomsection\label{eat:eat.Eat_Dinner.advertise}\pysiglinewithargsret{\bfcode{advertise}}{\emph{p}}{}
This function calculates the score of an activities advertisement to a Person.

Advertise to the agent if the following conditions are met
\begin{enumerate}
\item {} 
the current meal is lunch

\item {} 
the agent's location is at home (\code{location.HOME})

\item {} 
calculate the score

\end{enumerate}
\begin{gather}
\begin{split}S = \begin{cases}
0  & n_{hunger}(t) > \lambda \\
W( n_{hunger}(t) ) - W( n_{hunger}(t + \Delta{t} )) & n_{hunger}(t) \le \lambda
\end{cases}\end{split}\notag
\end{gather}
If the threshold is not met, score is 0. The advertisements assume that the duration         of the activity is the mean duration.
\begin{quote}\begin{description}
\item[{Parameters}] \leavevmode
\textbf{\texttt{p}} ({\hyperref[person:person.Person]{\emph{\emph{person.Person}}}}) -- The person of interest

\item[{Return score}] \leavevmode
the advertised score of doing the Eat activity

\item[{Return type}] \leavevmode
float

\end{description}\end{quote}

\end{fulllineitems}

\index{end\_meal() (eat.Eat\_Dinner method)}

\begin{fulllineitems}
\phantomsection\label{eat:eat.Eat_Dinner.end_meal}\pysiglinewithargsret{\bfcode{end\_meal}}{\emph{p}}{}
This function goes through the logistics of ending the dinner meal by doing the following:
\begin{enumerate}
\item {} 
calls \code{end.end\_meal()}

\item {} 
If breakfast will be skipped, update the lunch event to be 2 days from the current day

\end{enumerate}
\begin{quote}\begin{description}
\item[{Parameters}] \leavevmode
\textbf{\texttt{p}} ({\hyperref[person:person.Person]{\emph{\emph{person.Person}}}}) -- the person who is finishing the eating dinner event

\item[{Returns}] \leavevmode
None

\end{description}\end{quote}

\end{fulllineitems}

\index{start\_meal() (eat.Eat\_Dinner method)}

\begin{fulllineitems}
\phantomsection\label{eat:eat.Eat_Dinner.start_meal}\pysiglinewithargsret{\bfcode{start\_meal}}{\emph{p}}{}
This function goes through the logistics of starting the dinner meal by doing the following:
\begin{enumerate}
\item {} 
set the current meal to dinner

\item {} 
call \code{eat.start\_meal()}

\end{enumerate}
\begin{quote}\begin{description}
\item[{Parameters}] \leavevmode
\textbf{\texttt{p}} ({\hyperref[person:person.Person]{\emph{\emph{person.Person}}}}) -- the person who is starting the eat dinner event

\item[{Returns}] \leavevmode
None

\end{description}\end{quote}

\end{fulllineitems}


\end{fulllineitems}

\index{Eat\_Lunch (class in eat)}

\begin{fulllineitems}
\phantomsection\label{eat:eat.Eat_Lunch}\pysigline{\strong{class }\code{eat.}\bfcode{Eat\_Lunch}}
Bases: {\hyperref[eat:eat.Eat]{\emph{\code{eat.Eat}}}}

This class is used to handle the logistics for eating lunch.
\index{advertise() (eat.Eat\_Lunch method)}

\begin{fulllineitems}
\phantomsection\label{eat:eat.Eat_Lunch.advertise}\pysiglinewithargsret{\bfcode{advertise}}{\emph{p}}{}
This function calculates the score of an activities advertisement to a person.

Advertise to the agent if the following conditions are met
\begin{enumerate}
\item {} 
the current meal is lunch

\item {} 
the agent's location is at home (\code{location.HOME}) or the agent's location is at the         workplace (\code{location.OFF\_SITE})

\item {} 
calculate the score

\end{enumerate}
\begin{gather}
\begin{split}S = \begin{cases}
0  & n_{hunger}(t) > \lambda \\
W( n_{hunger}(t) ) - W( n_{hunger}(t + \Delta{t} )) & n_{hunger}(t) \le \lambda
\end{cases}\end{split}\notag
\end{gather}
If the threshold is not met, score is 0. The advertisements assume that the duration         of the activity is the mean duration.
\begin{quote}\begin{description}
\item[{Parameters}] \leavevmode
\textbf{\texttt{p}} ({\hyperref[person:person.Person]{\emph{\emph{person.Person}}}}) -- The person of interest

\item[{Return score}] \leavevmode
the advertised score of doing the Eat activity

\item[{Return type}] \leavevmode
float

\end{description}\end{quote}

\end{fulllineitems}

\index{end\_meal() (eat.Eat\_Lunch method)}

\begin{fulllineitems}
\phantomsection\label{eat:eat.Eat_Lunch.end_meal}\pysiglinewithargsret{\bfcode{end\_meal}}{\emph{p}}{}
This function ends the eat activity by doing the following:
\begin{enumerate}
\item {} 
calls \code{eat.end\_meal()}

\item {} \begin{description}
\item[{if dinner is to be skipped, update the dinner event by doing the following:}] \leavevmode\begin{itemize}
\item {} \begin{description}
\item[{if the lunch is an interrupting activity}] \leavevmode\begin{itemize}
\item {} 
set the time until the next lunch activity

\item {} 
update the schedule for the interruption to the next lunch activity

\item {} 
set the interruption state to False

\end{itemize}

\end{description}

\end{itemize}

\end{description}

\end{enumerate}
\begin{quote}\begin{description}
\item[{Parameters}] \leavevmode
\textbf{\texttt{p}} ({\hyperref[person:person.Person]{\emph{\emph{person.Person}}}}) -- The person whose meal is ending.

\item[{Returns}] \leavevmode
None

\end{description}\end{quote}

\end{fulllineitems}

\index{start\_meal() (eat.Eat\_Lunch method)}

\begin{fulllineitems}
\phantomsection\label{eat:eat.Eat_Lunch.start_meal}\pysiglinewithargsret{\bfcode{start\_meal}}{\emph{p}}{}
This function handles the logistics for starting the eat activity by doing the following:
\begin{enumerate}
\item {} 
sets the current meal to lunch

\item {} 
call \code{eat.start\_meal()}

\end{enumerate}
\begin{quote}\begin{description}
\item[{Parameters}] \leavevmode
\textbf{\texttt{p}} ({\hyperref[person:person.Person]{\emph{\emph{person.Person}}}}) -- the person starting the eat lunch event

\item[{Returns}] \leavevmode
None

\end{description}\end{quote}

\end{fulllineitems}


\end{fulllineitems}



\section{food module}
\label{food::doc}\label{food:module-food}\label{food:food-module}\index{food (module)}
This module contains information about the asset that allows for the eating activity.

This module contains the following class: {\hyperref[food:food.Food]{\emph{\code{food.Food}}}}.

\emph{Module author: Dr. Namdi Brandon.}
\index{Food (class in food)}

\begin{fulllineitems}
\phantomsection\label{food:food.Food}\pysigline{\strong{class }\code{food.}\bfcode{Food}}
Bases: {\hyperref[asset:asset.Asset]{\emph{\code{asset.Asset}}}}

This class represents an asset that allows the agent to eat breakfast, eat lunch, and eat dinner.

Activities in this asset:
\begin{enumerate}
\item {} 
\code{eat\_breakfast.Eat\_Breakfast}

\item {} 
\code{eat\_lunch.Eat\_Lunch}

\item {} 
\code{eat\_dinner.Eat\_Dinner}

\end{enumerate}

\end{fulllineitems}



\section{home module}
\label{home::doc}\label{home:module-home}\label{home:home-module}\index{home (module)}
This module governs the control of assets used in the simulation. Mainly, the home contains all of the assets used in the simulation for the current version of the code.

This module contains the following class: {\hyperref[home:home.Home]{\emph{\code{home.Home}}}}

\emph{Module author: Dr. Namdi Brandon.}
\index{Home (class in home)}

\begin{fulllineitems}
\phantomsection\label{home:home.Home}\pysiglinewithargsret{\strong{class }\code{home.}\bfcode{Home}}{\emph{clock}}{}
Bases: \code{object}

Contains all of the physical characteristics of a home/ residence. Currently, the home     contains all of the assets within the simulation.
\begin{quote}\begin{description}
\item[{Parameters}] \leavevmode
\textbf{\texttt{clock}} ({\hyperref[temporal:temporal.Temporal]{\emph{\emph{temporal.Temporal}}}}) -- the time

\item[{Variables}] \leavevmode\begin{itemize}
\item {} 
\textbf{\texttt{assets}} (\emph{dict}) -- contains a list of all of the assets available in the home.

\item {} 
\textbf{\texttt{category}} (\emph{int}) -- the type of home

\item {} 
\textbf{\texttt{clock}} ({\hyperref[temporal:temporal.Temporal]{\emph{\emph{temporal.Temporal}}}}) -- the time

\item {} 
\textbf{\texttt{id}} (\emph{int}) -- a unique home identification number

\item {} 
{\hyperref[location:module-location]{\emph{\textbf{\texttt{location}}}}} ({\hyperref[location:location.Location]{\emph{\emph{location.Location}}}}) -- the location of the home

\item {} 
\textbf{\texttt{population}} (\emph{int}) -- the number of people who reside in a home

\item {} 
\textbf{\texttt{revenue}} (\emph{float}) -- the household revenue

\end{itemize}

\end{description}\end{quote}
\index{advertise() (home.Home method)}

\begin{fulllineitems}
\phantomsection\label{home:home.Home.advertise}\pysiglinewithargsret{\bfcode{advertise}}{\emph{p}, \emph{do\_interruption=False}, \emph{locale=None}}{}
This function handles all of the Activities' advertisements to a Person. This occurs by looping         through each asset in the home and collecting a list of advertisements for each activity in each asset.
\begin{enumerate}
\item {} 
loop through each asset

\item {} \begin{description}
\item[{if the asset is busy \emph{and} is in the same location of the person}] \leavevmode\begin{itemize}
\item {} \begin{description}
\item[{for each activity in the given asset}] \leavevmode\begin{enumerate}
\item {} 
advertise for interrupting activities

\item {} 
advertise for non interrupting activities

\item {} 
collect the advertisements

\end{enumerate}

\end{description}

\end{itemize}

\end{description}

\end{enumerate}
\begin{quote}\begin{description}
\item[{Parameters}] \leavevmode\begin{itemize}
\item {} 
\textbf{\texttt{p}} ({\hyperref[person:person.Person]{\emph{\emph{person.Person}}}}) -- a person to whom the assets are advertising

\item {} 
\textbf{\texttt{do\_interruption}} (\emph{bool}) -- a flag that indicates whether or not we should advertise for interruptions

\item {} 
\textbf{\texttt{locale}} (\emph{int}) -- a local location identifier

\end{itemize}

\item[{Returns}] \leavevmode
the advertisements (score, asset, activity, person) containing the various data for                         each advertisement: (``score'', ``asset'', ``activity'', ``person'') coupling

\item[{Return type}] \leavevmode
dict

\end{description}\end{quote}

\end{fulllineitems}

\index{initialize() (home.Home method)}

\begin{fulllineitems}
\phantomsection\label{home:home.Home.initialize}\pysiglinewithargsret{\bfcode{initialize}}{\emph{people}}{}
Initialize the assets in the home.
\begin{quote}\begin{description}
\item[{Parameters}] \leavevmode
\textbf{\texttt{people}} (\emph{list}) -- a list of people who reside in the home

\item[{Returns}] \leavevmode
None

\end{description}\end{quote}

\end{fulllineitems}

\index{print\_category() (home.Home method)}

\begin{fulllineitems}
\phantomsection\label{home:home.Home.print_category}\pysiglinewithargsret{\bfcode{print\_category}}{}{}
This function expresses the category variable as a string.
\begin{quote}\begin{description}
\item[{Returns}] \leavevmode
string representation of the category

\item[{Return type}] \leavevmode
str

\end{description}\end{quote}

\end{fulllineitems}

\index{reset() (home.Home method)}

\begin{fulllineitems}
\phantomsection\label{home:home.Home.reset}\pysiglinewithargsret{\bfcode{reset}}{}{}
This function resets the each asset in the home.
\begin{quote}\begin{description}
\item[{Returns}] \leavevmode
None

\end{description}\end{quote}

\end{fulllineitems}

\index{set\_population() (home.Home method)}

\begin{fulllineitems}
\phantomsection\label{home:home.Home.set_population}\pysiglinewithargsret{\bfcode{set\_population}}{\emph{people}}{}
Set the population of the house.
\begin{quote}\begin{description}
\item[{Parameters}] \leavevmode
\textbf{\texttt{people}} (\emph{list}) -- the list of people living in the home

\item[{Returns}] \leavevmode
None

\end{description}\end{quote}

\end{fulllineitems}

\index{set\_revenue() (home.Home method)}

\begin{fulllineitems}
\phantomsection\label{home:home.Home.set_revenue}\pysiglinewithargsret{\bfcode{set\_revenue}}{\emph{people}}{}
Sets the revenue of the home by adding the revenue of each person in the home.
\begin{quote}\begin{description}
\item[{Parameters}] \leavevmode
\textbf{\texttt{people}} (\emph{list}) -- the list of people living in the home

\item[{Returns}] \leavevmode
None

\end{description}\end{quote}

\end{fulllineitems}

\index{toString() (home.Home method)}

\begin{fulllineitems}
\phantomsection\label{home:home.Home.toString}\pysiglinewithargsret{\bfcode{toString}}{}{}
This function expresses the Home object as a string
\begin{quote}\begin{description}
\item[{Return msg}] \leavevmode
the string representation of the home object

\item[{Return type}] \leavevmode
str

\end{description}\end{quote}

\end{fulllineitems}


\end{fulllineitems}



\section{hunger module}
\label{hunger::doc}\label{hunger:module-hunger}\label{hunger:hunger-module}\index{hunger (module)}
This module contains information about governing the need Hunger.

This module contains the class hunger ({\hyperref[hunger:hunger.Hunger]{\emph{\code{hunger.Hunger}}}}).

\emph{Module author: Dr. Namdi Brandon}
\index{Hunger (class in hunger)}

\begin{fulllineitems}
\phantomsection\label{hunger:hunger.Hunger}\pysiglinewithargsret{\strong{class }\code{hunger.}\bfcode{Hunger}}{\emph{clock}, \emph{num\_sample\_points}}{}
Bases: {\hyperref[need:need.Need]{\emph{\code{need.Need}}}}

This class governs the behavior of the need Hunger need. When Hunger is unstatisfied,
the agent feels compelled to eat a meal in order to satisfy the need. Mathematically     speaking, Hunger is modeled as linearly-behaving need.
\begin{quote}\begin{description}
\item[{Parameters}] \leavevmode\begin{itemize}
\item {} 
\textbf{\texttt{clock}} ({\hyperref[temporal:temporal.Temporal]{\emph{\emph{temporal.Temporal}}}}) -- the time

\item {} 
\textbf{\texttt{num\_sample\_points}} (\emph{int}) -- the number of temporal nodes in the simulation

\end{itemize}

\item[{Variables}] \leavevmode\begin{itemize}
\item {} 
\textbf{\texttt{category}} (\emph{int}) -- the category of the need

\item {} 
\textbf{\texttt{decay\_rate}} (\emph{float}) -- the decay rate of the Hunger need {[}need/minute{]}

\item {} 
\textbf{\texttt{recharge\_rate}} (\emph{float}) -- the recharge rate of the Hunger need {[}need/min{]}

\item {} 
\textbf{\texttt{suggested\_recharge\_rate}} (\emph{float}) -- an approximate recharge rate used to calculate the end time of an     event before rounding

\end{itemize}

\end{description}\end{quote}
\index{decay() (hunger.Hunger method)}

\begin{fulllineitems}
\phantomsection\label{hunger:hunger.Hunger.decay}\pysiglinewithargsret{\bfcode{decay}}{\emph{status}}{}
This function decreases the satiation in Hunger by doing the following:
\begin{gather}
\begin{split}n(t + 1) = n(t) + m_{decay}                        \end{split}\notag
\end{gather}
\begin{notice}{warning}{Warning:}
This function may be antiquated and \textbf{not used}
\end{notice}
\begin{quote}\begin{description}
\item[{Parameters}] \leavevmode
\textbf{\texttt{status}} (\emph{int}) -- indicates the current status of the person's state (not-used)

\item[{Returns}] \leavevmode
None

\end{description}\end{quote}

\end{fulllineitems}

\index{decay\_new() (hunger.Hunger method)}

\begin{fulllineitems}
\phantomsection\label{hunger:hunger.Hunger.decay_new}\pysiglinewithargsret{\bfcode{decay\_new}}{\emph{dt}}{}
This function sets the default decrease in the Hunger need.
\begin{gather}
\begin{split}n(t + \Delta{t}) = n(t) + m_{decay}\,\Delta{t}\end{split}\notag
\end{gather}\begin{quote}\begin{description}
\item[{Parameters}] \leavevmode
\textbf{\texttt{dt}} (\emph{int}) -- the duration of time {[}minutes{]} used to decay the need

\item[{Returns}] \leavevmode
None

\end{description}\end{quote}

\end{fulllineitems}

\index{initialize() (hunger.Hunger method)}

\begin{fulllineitems}
\phantomsection\label{hunger:hunger.Hunger.initialize}\pysiglinewithargsret{\bfcode{initialize}}{\emph{p}}{}
This function initializes the the hunger need at the first step of the simulation. The function         checks to see whether or not the current time implies that there should be an eating event. The         hunger object is set to the respective state.

This function does the following exactly
\begin{enumerate}
\item {} 
initialize all of the meals

\item {} 
check to see if a meal should be occurring at the current time

\item {} \begin{description}
\item[{if no meals should be occurring}] \leavevmode\begin{itemize}
\item {} 
figure out the next meal \(m\)

\item {} 
calculate the decay rate for hunger until the next meal

\item {} 
calculate the amount of time until the next meal \(\Delta{t}\)

\item {} 
set the current meal \(m\)

\item {} 
update the schedule for the hunger need to be the time the next meal starts

\end{itemize}

\end{description}

\item {} \begin{description}
\item[{if a meal should be occurring}] \leavevmode\begin{itemize}
\item {} 
get the index of the meal that should be occurring \(m\)

\item {} 
set the current meal \(m\)

\item {} 
calculate the final time of the meal

\item {} 
calculate the duration until the end of the next meal \(\Delta{t}\)

\item {} 
set the recharge rate

\item {} 
update the scheduler for the hunger need to be the time the current meal should end

\end{itemize}

\end{description}

\item {} 
initialize the start time for each meal

\end{enumerate}
\begin{quote}\begin{description}
\item[{Parameters}] \leavevmode
\textbf{\texttt{p}} ({\hyperref[person:person.Person]{\emph{\emph{person.Person}}}}) -- the person whose hunger need is being initialized

\item[{Returns}] \leavevmode
None

\end{description}\end{quote}

\end{fulllineitems}

\index{is\_meal\_time() (hunger.Hunger method)}

\begin{fulllineitems}
\phantomsection\label{hunger:hunger.Hunger.is_meal_time}\pysiglinewithargsret{\bfcode{is\_meal\_time}}{\emph{t}, \emph{the\_meal}}{}
This checks whether or not it is time for a meal.
\begin{quote}\begin{description}
\item[{Parameters}] \leavevmode\begin{itemize}
\item {} 
\textbf{\texttt{t}} (\emph{int}) -- time of day {[}minutes{]}

\item {} 
\textbf{\texttt{the\_meal}} ({\hyperref[meal:meal.Meal]{\emph{\emph{meal.Meal}}}}) -- the respective meal to see whether the current time implies                             that an eating event should happen

\end{itemize}

\item[{Returns}] \leavevmode
True if the current time is within the time to eat. False, otherwise

\item[{Return type}] \leavevmode
bool

\end{description}\end{quote}

\end{fulllineitems}

\index{is\_meal\_time\_all() (hunger.Hunger method)}

\begin{fulllineitems}
\phantomsection\label{hunger:hunger.Hunger.is_meal_time_all}\pysiglinewithargsret{\bfcode{is\_meal\_time\_all}}{\emph{t}, \emph{meals}}{}
This function checks every meal and sees whether or not the current time         implies that there should be an eventing event for a respective meal
\begin{quote}\begin{description}
\item[{Parameters}] \leavevmode\begin{itemize}
\item {} 
\textbf{\texttt{t}} (\emph{int}) -- The current time of day {[}minutes{]}

\item {} 
\textbf{\texttt{meals}} (\emph{list}) -- A list of Meals

\end{itemize}

\item[{Returns}] \leavevmode
A list of boolean flags indicating True or False, indicating whether or not an         eating event should occur for the respective meal

\item[{Return type}] \leavevmode
list

\end{description}\end{quote}

\end{fulllineitems}

\index{perceive() (hunger.Hunger method)}

\begin{fulllineitems}
\phantomsection\label{hunger:hunger.Hunger.perceive}\pysiglinewithargsret{\bfcode{perceive}}{\emph{future\_clock}}{}
This gives the result if eat is done now until a later time corresponding to clock.
\begin{quote}\begin{description}
\item[{Parameters}] \leavevmode
\textbf{\texttt{future\_clock}} ({\hyperref[temporal:temporal.Temporal]{\emph{\emph{temporal.Temporal}}}}) -- a clock at a future time

\item[{Return out}] \leavevmode
the perceived hunger need association level

\item[{Return type}] \leavevmode
float

\end{description}\end{quote}

\end{fulllineitems}

\index{reset() (hunger.Hunger method)}

\begin{fulllineitems}
\phantomsection\label{hunger:hunger.Hunger.reset}\pysiglinewithargsret{\bfcode{reset}}{}{}
This function resets the values in order for the need to be used in the next simulation
:return:

\end{fulllineitems}

\index{set\_decay\_rate() (hunger.Hunger method)}

\begin{fulllineitems}
\phantomsection\label{hunger:hunger.Hunger.set_decay_rate}\pysiglinewithargsret{\bfcode{set\_decay\_rate}}{\emph{t\_start}}{}
This function calculates the decay rate of hunger to the next meal.
\begin{quote}\begin{description}
\item[{Parameters}] \leavevmode\begin{itemize}
\item {} 
\textbf{\texttt{dt}} (\emph{int}) -- the amount of time to the next meal {[}minutes{]}

\item {} 
\textbf{\texttt{t\_start}} (\emph{int}) -- the start time {[}in minutes{]} of the next meal

\end{itemize}

\item[{Returns}] \leavevmode
None

\end{description}\end{quote}

\end{fulllineitems}

\index{set\_decay\_rate\_new() (hunger.Hunger method)}

\begin{fulllineitems}
\phantomsection\label{hunger:hunger.Hunger.set_decay_rate_new}\pysiglinewithargsret{\bfcode{set\_decay\_rate\_new}}{\emph{dt}}{}
This function calculates the decay rate of hunger to the next meal.
\begin{quote}\begin{description}
\item[{Parameters}] \leavevmode
\textbf{\texttt{dt}} (\emph{int}) -- the amount of time to the next meal {[}minutes{]}

\item[{Returns}] \leavevmode
None

\end{description}\end{quote}

\end{fulllineitems}

\index{set\_recharge\_rate() (hunger.Hunger method)}

\begin{fulllineitems}
\phantomsection\label{hunger:hunger.Hunger.set_recharge_rate}\pysiglinewithargsret{\bfcode{set\_recharge\_rate}}{\emph{dt}}{}
This function calculates the recharge rate of hunger due to eating the current meal.
\begin{quote}\begin{description}
\item[{Parameters}] \leavevmode
\textbf{\texttt{dt}} (\emph{int}) -- the amount of time it takes to finish a meal {[}minutes{]}

\item[{Returns}] \leavevmode
None

\end{description}\end{quote}

\end{fulllineitems}

\index{set\_suggested\_recharge\_rate() (hunger.Hunger method)}

\begin{fulllineitems}
\phantomsection\label{hunger:hunger.Hunger.set_suggested_recharge_rate}\pysiglinewithargsret{\bfcode{set\_suggested\_recharge\_rate}}{\emph{dt}}{}
This function sets the suggested recharge rate assuming a \textbf{linear function} behavior

The suggested recharge rate is based on the duration of the sleeping event         and the threshold. The sleep duration is based on the biological data (no rounding).
\begin{quote}\begin{description}
\item[{Parameters}] \leavevmode
\textbf{\texttt{dt}} (\emph{int}) -- The duration of time of the eating event {[}minutes{]}

\item[{Returns}] \leavevmode
None

\end{description}\end{quote}

\end{fulllineitems}

\index{toString() (hunger.Hunger method)}

\begin{fulllineitems}
\phantomsection\label{hunger:hunger.Hunger.toString}\pysiglinewithargsret{\bfcode{toString}}{}{}~\begin{quote}

Represents the Hunger object as a string.
\end{quote}
\begin{quote}\begin{description}
\item[{Return msg}] \leavevmode
the string representation of the huger object

\item[{Return type}] \leavevmode
str

\end{description}\end{quote}

\end{fulllineitems}


\end{fulllineitems}



\section{income module}
\label{income::doc}\label{income:module-income}\label{income:income-module}\index{income (module)}
This is module contains code for governing the need to work/ be schooled.

This module contains the class {\hyperref[income:income.Income]{\emph{\code{income.Income}}}}.

\emph{Module author: Dr. Namdi Brandon}
\index{Income (class in income)}

\begin{fulllineitems}
\phantomsection\label{income:income.Income}\pysiglinewithargsret{\strong{class }\code{income.}\bfcode{Income}}{\emph{clock}, \emph{num\_sample\_points}}{}
Bases: {\hyperref[need:need.Need]{\emph{\code{need.Need}}}}

This class governs the need association dealing with work / school. Recall that income mathematically     resembles a step function.
\begin{quote}\begin{description}
\item[{Parameters}] \leavevmode\begin{itemize}
\item {} 
\textbf{\texttt{clock}} ({\hyperref[temporal:temporal.Temporal]{\emph{\emph{temporal.Temporal}}}}) -- the time

\item {} 
\textbf{\texttt{num\_sample\_points}} (\emph{int}) -- the number of temporal node points in the simulation

\end{itemize}

\end{description}\end{quote}
\index{decay() (income.Income method)}

\begin{fulllineitems}
\phantomsection\label{income:income.Income.decay}\pysiglinewithargsret{\bfcode{decay}}{\emph{p}}{}
This function decays the magnitude of the need. Income only decays after the job start time.
\begin{enumerate}
\item {} 
Find out if it is time to work

\item {} 
If it's time to work, set the satiation \(n_{income} = \eta_{work}\)

\end{enumerate}
\begin{quote}\begin{description}
\item[{Parameters}] \leavevmode
\textbf{\texttt{p}} ({\hyperref[person:person.Person]{\emph{\emph{person.Person}}}}) -- the person of interest

\item[{Returns}] \leavevmode
None

\end{description}\end{quote}

\end{fulllineitems}

\index{initialize() (income.Income method)}

\begin{fulllineitems}
\phantomsection\label{income:income.Income.initialize}\pysiglinewithargsret{\bfcode{initialize}}{\emph{p}}{}
This function is used to initialize the agent's income need at the beginning of the simulation.         This function initializes the Person to be at the workplace (\code{location.OFF\_SITE}) if it is work time.         This function does the following:
\begin{enumerate}
\item {} 
decay the income satiation

\item {} \begin{description}
\item[{if the person is supposed to be at work}] \leavevmode\begin{itemize}
\item {} 
set the person to the workplace location

\item {} 
else, set the amount of time until the next work event

\end{itemize}

\end{description}

\item {} 
update the scheduler for the income need

\end{enumerate}
\begin{quote}\begin{description}
\item[{Parameters}] \leavevmode
\textbf{\texttt{p}} ({\hyperref[person:person.Person]{\emph{\emph{person.Person}}}}) -- the person of interest

\item[{Returns}] \leavevmode
None

\end{description}\end{quote}

\end{fulllineitems}

\index{perceive() (income.Income method)}

\begin{fulllineitems}
\phantomsection\label{income:income.Income.perceive}\pysiglinewithargsret{\bfcode{perceive}}{\emph{clock}, \emph{job}}{}
This gives the satiation of income \textbf{if} the income need is addressed now.
\begin{enumerate}
\item {} 
find out if the time associated with clock implies a work time for the person

\item {} \begin{description}
\item[{If it should be work time}] \leavevmode\begin{itemize}
\item {} 
the perceived satiation is \(\eta_{work} \le \lambda\)

\item {} 
else, the perceived satiation is \(1.0\)

\end{itemize}

\end{description}

\end{enumerate}
\begin{quote}\begin{description}
\item[{Parameters}] \leavevmode\begin{itemize}
\item {} 
\textbf{\texttt{clock}} ({\hyperref[temporal:temporal.Temporal]{\emph{\emph{temporal.Temporal}}}}) -- the future time the activity the should be perceived to be done

\item {} 
\textbf{\texttt{job}} ({\hyperref[occupation:occupation.Occupation]{\emph{\emph{occupation.Occupation}}}}) -- the job

\end{itemize}

\item[{Returns}] \leavevmode
the satiation at the perceived time

\item[{Return type}] \leavevmode
float

\end{description}\end{quote}

\end{fulllineitems}


\end{fulllineitems}



\section{interrupt module}
\label{interrupt::doc}\label{interrupt:module-interrupt}\label{interrupt:interrupt-module}\index{interrupt (module)}
This module contains code for interrupting a current activity.

This module contains class {\hyperref[interrupt:interrupt.Interrupt]{\emph{\code{interrupt.Interrupt}}}}.

\emph{Module author: Dr. Namdi Brandon}
\index{Interrupt (class in interrupt)}

\begin{fulllineitems}
\phantomsection\label{interrupt:interrupt.Interrupt}\pysigline{\strong{class }\code{interrupt.}\bfcode{Interrupt}}
Bases: {\hyperref[activity:activity.Activity]{\emph{\code{activity.Activity}}}}

This class allows for the current activity to be interrupted by another activity.
\index{advertise() (interrupt.Interrupt method)}

\begin{fulllineitems}
\phantomsection\label{interrupt:interrupt.Interrupt.advertise}\pysiglinewithargsret{\bfcode{advertise}}{\emph{p}, \emph{str\_need}, \emph{act}}{}
This function calculates the score of an activities advertisement to a Person. This function does the         the following:
\begin{enumerate}
\item {} 
temporarily sets the value of the Need that must be immediately addressed to a low level.

\item {} 
send an advertisement is is made from the potentially interrupting activity

\item {} 
calculate the score from the potentially interrupting activity

\end{enumerate}
\begin{quote}\begin{description}
\item[{Parameters}] \leavevmode\begin{itemize}
\item {} 
\textbf{\texttt{p}} ({\hyperref[person:person.Person]{\emph{\emph{person.Person}}}}) -- the person who is being advertised to

\item {} 
\textbf{\texttt{str\_need}} (\emph{int}) -- the id of the Need that needs to be addressed, which                                 could potentially cause an interrupting event

\item {} 
\textbf{\texttt{act}} ({\hyperref[activity:activity.Activity]{\emph{\emph{activity.Activity}}}}) -- the activity of interest that could immediately                             interrupt a current activity

\end{itemize}

\item[{Returns}] \leavevmode
the value of the advertisement

\item[{Return type}] \leavevmode
float

\end{description}\end{quote}

\end{fulllineitems}

\index{start() (interrupt.Interrupt method)}

\begin{fulllineitems}
\phantomsection\label{interrupt:interrupt.Interrupt.start}\pysiglinewithargsret{\bfcode{start}}{\emph{p}}{}
This handles the start of an activity.
\begin{quote}\begin{description}
\item[{Parameters}] \leavevmode
\textbf{\texttt{p}} ({\hyperref[person:person.Person]{\emph{\emph{person.Person}}}}) -- the person of interest

\item[{Returns}] \leavevmode
None

\end{description}\end{quote}

\end{fulllineitems}


\end{fulllineitems}



\section{interruption module}
\label{interruption::doc}\label{interruption:module-interruption}\label{interruption:interruption-module}\index{interruption (module)}
This class gives an agent the ability to interrupt a current activity.

This module contains class {\hyperref[interruption:interruption.Interruption]{\emph{\code{interruption.Interruption}}}}.

\emph{Module author: Dr. Namdi Brandon}
\index{Interruption (class in interruption)}

\begin{fulllineitems}
\phantomsection\label{interruption:interruption.Interruption}\pysiglinewithargsret{\strong{class }\code{interruption.}\bfcode{Interruption}}{\emph{clock}, \emph{num\_sample\_points}}{}
Bases: {\hyperref[need:need.Need]{\emph{\code{need.Need}}}}

This class enables a Person to interrupt a current activity.

Interruption is a minor Need (need-association). That is, this need-association is NOT     independent of other needs.
\begin{quote}\begin{description}
\item[{Parameters}] \leavevmode\begin{itemize}
\item {} 
\textbf{\texttt{clock}} ({\hyperref[temporal:temporal.Temporal]{\emph{\emph{temporal.Temporal}}}}) -- The clock governing time in the simulation

\item {} 
\textbf{\texttt{num\_sample\_points}} (\emph{int}) -- The number of time nodes in the simulation

\end{itemize}

\item[{Variables}] \leavevmode\begin{itemize}
\item {} 
\textbf{\texttt{category}} (\emph{int}) -- the category of the interruption Need

\item {} 
\textbf{\texttt{activity\_start}} (\emph{int}) -- the category of the (interrupting) activity to start

\item {} 
\textbf{\texttt{activity\_stop}} (\emph{int}) -- the category of the (interrupted) activity to stop

\end{itemize}

\end{description}\end{quote}
\index{decay() (interruption.Interruption method)}

\begin{fulllineitems}
\phantomsection\label{interruption:interruption.Interruption.decay}\pysiglinewithargsret{\bfcode{decay}}{\emph{p}}{}~\begin{quote}

This function sets the default decrease in the Interruption need
\end{quote}
\begin{quote}\begin{description}
\item[{Parameters}] \leavevmode
\textbf{\texttt{p}} ({\hyperref[person:person.Person]{\emph{\emph{person.Person}}}}) -- the person of interest

\item[{Returns}] \leavevmode
None

\end{description}\end{quote}

\end{fulllineitems}

\index{get\_time\_to\_next\_work\_lunch() (interruption.Interruption method)}

\begin{fulllineitems}
\phantomsection\label{interruption:interruption.Interruption.get_time_to_next_work_lunch}\pysiglinewithargsret{\bfcode{get\_time\_to\_next\_work\_lunch}}{\emph{p}}{}
\end{fulllineitems}

\index{initialize() (interruption.Interruption method)}

\begin{fulllineitems}
\phantomsection\label{interruption:interruption.Interruption.initialize}\pysiglinewithargsret{\bfcode{initialize}}{\emph{p}}{}
Initializes the need during the first step
\begin{quote}\begin{description}
\item[{Parameters}] \leavevmode
\textbf{\texttt{p}} ({\hyperref[person:person.Person]{\emph{\emph{person.Person}}}}) -- Initialize the state of the interruption need-association at the                          beginning of the simulation

\item[{Returns}] \leavevmode
None

\end{description}\end{quote}

\end{fulllineitems}

\index{is\_lunch\_time() (interruption.Interruption method)}

\begin{fulllineitems}
\phantomsection\label{interruption:interruption.Interruption.is_lunch_time}\pysiglinewithargsret{\bfcode{is\_lunch\_time}}{\emph{time\_of\_day}, \emph{meals}}{}
This function indicates whether it is lunch time or not. This is used in the interruption to stop the work         activity and begin the eat lunch activity.
\begin{quote}\begin{description}
\item[{Parameters}] \leavevmode\begin{itemize}
\item {} 
\textbf{\texttt{time\_of\_day}} (\emph{int}) -- the time of day {[}minutes{]}

\item {} 
\textbf{\texttt{meals}} (\emph{list}) -- a list of the meals ({\hyperref[meal:meal.Meal]{\emph{\code{meal.Meal}}}}) for the agents. Some of the entries in the list         may be None.

\end{itemize}

\item[{Return is\_lunch}] \leavevmode
a flag indicating whether it is lunch time

\end{description}\end{quote}

\end{fulllineitems}

\index{perceive() (interruption.Interruption method)}

\begin{fulllineitems}
\phantomsection\label{interruption:interruption.Interruption.perceive}\pysiglinewithargsret{\bfcode{perceive}}{\emph{clock}}{}
This gives the result if sleep is done now until a later time corresponding to clock.
\begin{quote}\begin{description}
\item[{Parameters}] \leavevmode
\textbf{\texttt{clock}} ({\hyperref[temporal:temporal.Temporal]{\emph{\emph{temporal.Temporal}}}}) -- a clock at a future time

\item[{Return out}] \leavevmode
the perceived interruption magnitude

\end{description}\end{quote}

\end{fulllineitems}

\index{reset() (interruption.Interruption method)}

\begin{fulllineitems}
\phantomsection\label{interruption:interruption.Interruption.reset}\pysiglinewithargsret{\bfcode{reset}}{}{}
This function resets the interruption need completely in order to re run the simulation. In this reset,         the history is also reset.
\begin{quote}\begin{description}
\item[{Returns}] \leavevmode


\end{description}\end{quote}

\end{fulllineitems}

\index{reset\_minor() (interruption.Interruption method)}

\begin{fulllineitems}
\phantomsection\label{interruption:interruption.Interruption.reset_minor}\pysiglinewithargsret{\bfcode{reset\_minor}}{}{}
This function resets the interruption need
\begin{quote}\begin{description}
\item[{Returns}] \leavevmode
None

\end{description}\end{quote}

\end{fulllineitems}

\index{stop\_work\_to\_eat() (interruption.Interruption method)}

\begin{fulllineitems}
\phantomsection\label{interruption:interruption.Interruption.stop_work_to_eat}\pysiglinewithargsret{\bfcode{stop\_work\_to\_eat}}{\emph{p}}{}
This function checks to see if an interruption should occur to allow a Person to         start the eating activity while doing the work activity

An agent may stop the work activity to eat lunch if the following conditions are met:
\begin{enumerate}
\item {} 
The agent is hungry

\item {} 
The current activity is work

\item {} 
It is lunch time

\end{enumerate}
\begin{quote}\begin{description}
\item[{Parameters}] \leavevmode
\textbf{\texttt{p}} ({\hyperref[person:person.Person]{\emph{\emph{person.Person}}}}) -- the person of interest

\item[{Returns}] \leavevmode
None

\end{description}\end{quote}

\end{fulllineitems}


\end{fulllineitems}



\section{location module}
\label{location::doc}\label{location:module-location}\label{location:location-module}\index{location (module)}
This module is responsible for containing information about the concept of location.

This module contains class {\hyperref[location:location.Location]{\emph{\code{location.Location}}}}.

\emph{Module author: Dr. Namdi Brandon}
\index{Location (class in location)}

\begin{fulllineitems}
\phantomsection\label{location:location.Location}\pysiglinewithargsret{\strong{class }\code{location.}\bfcode{Location}}{\emph{geography=1}, \emph{local=0}}{}
Bases: \code{object}

This class holds information relevant to the location of persons and assets.
\begin{quote}\begin{description}
\item[{Parameters}] \leavevmode\begin{itemize}
\item {} 
\textbf{\texttt{geography}} (\emph{int}) -- the geographical location code

\item {} 
\textbf{\texttt{local}} (\emph{int}) -- the local location code

\end{itemize}

\item[{Variables}] \leavevmode\begin{itemize}
\item {} 
\textbf{\texttt{geo}} (\emph{int}) -- the geographical location code within the U.S (e.g. north, south, eats, or west)

\item {} 
\textbf{\texttt{local}} (\emph{int}) -- the local location code (e.g. home, off site, etc)

\end{itemize}

\end{description}\end{quote}
\index{print\_geo() (location.Location method)}

\begin{fulllineitems}
\phantomsection\label{location:location.Location.print_geo}\pysiglinewithargsret{\bfcode{print\_geo}}{}{}
Returns the geographical location in a string format
\begin{quote}\begin{description}
\item[{Returns}] \leavevmode
the string representation of the geographical location

\item[{Return type}] \leavevmode
str

\end{description}\end{quote}

\end{fulllineitems}

\index{print\_local() (location.Location method)}

\begin{fulllineitems}
\phantomsection\label{location:location.Location.print_local}\pysiglinewithargsret{\bfcode{print\_local}}{}{}
Returns the local location in a string format
\begin{quote}\begin{description}
\item[{Returns}] \leavevmode
the string representation of the local location

\item[{Return type}] \leavevmode
str

\end{description}\end{quote}

\end{fulllineitems}

\index{reset() (location.Location method)}

\begin{fulllineitems}
\phantomsection\label{location:location.Location.reset}\pysiglinewithargsret{\bfcode{reset}}{}{}
This function resets the location to the default value, (\code{location.HOME}).
\begin{quote}\begin{description}
\item[{Returns}] \leavevmode
None

\end{description}\end{quote}

\end{fulllineitems}

\index{toString() (location.Location method)}

\begin{fulllineitems}
\phantomsection\label{location:location.Location.toString}\pysiglinewithargsret{\bfcode{toString}}{}{}
This function represents the Location object as a string.
\begin{quote}\begin{description}
\item[{Return msg}] \leavevmode
the string representation of the Location object

\item[{Return type}] \leavevmode
str

\end{description}\end{quote}

\end{fulllineitems}


\end{fulllineitems}



\section{meal module}
\label{meal::doc}\label{meal:meal-module}\label{meal:module-meal}\index{meal (module)}
This module contains contains information about various meals that an agent  would eat in.

This module contains code for class {\hyperref[meal:meal.Meal]{\emph{\code{meal.Meal}}}}.

\emph{Module author: Dr. Namdi Brandon}
\index{Meal (class in meal)}

\begin{fulllineitems}
\phantomsection\label{meal:meal.Meal}\pysiglinewithargsret{\strong{class }\code{meal.}\bfcode{Meal}}{\emph{id=0}, \emph{start\_mean=390}, \emph{start\_std=10}, \emph{start\_trunc=1}, \emph{dt\_mean=15}, \emph{dt\_std=5}, \emph{dt\_trunc=1}}{}
Bases: \code{object}

This is class contains information about meals (breakfast, dinner, and lunch)
\begin{quote}\begin{description}
\item[{Variables}] \leavevmode\begin{itemize}
\item {} 
\textbf{\texttt{id}} (\emph{int}) -- the meal type (breakfast, lunch, or dinner)

\item {} 
\textbf{\texttt{dt}} (\emph{int}) -- the duration of a meal {[}minutes{]}

\item {} 
\textbf{\texttt{dt\_mean}} (\emph{int}) -- the mean duration of a meal {[}minutes{]}

\item {} 
\textbf{\texttt{dt\_std}} (\emph{int}) -- the standard deviation of meal duration {[}minutes{]}

\item {} 
\textbf{\texttt{dt\_trunc}} (\emph{int}) -- the number of standard deviation in the duration distribution

\item {} 
\textbf{\texttt{t\_start}} (\emph{int}) -- the start time of a meal {[}minutes, time of day{]}

\item {} 
\textbf{\texttt{t\_start\_univ}} (\emph{int}) -- the start time of a meals {[}minutes, universal time{]}

\item {} 
\textbf{\texttt{start\_mean}} (\emph{int}) -- the mean start time of a meal {[}minutes, time of day{]}

\item {} 
\textbf{\texttt{start\_std}} (\emph{int}) -- the standard deviation of start time of a meal {[}minutes{]}

\item {} 
\textbf{\texttt{start\_trunc}} (\emph{int}) -- the number of standard deviation of in the start time distribution

\item {} 
\textbf{\texttt{f\_start}} -- the start time distribution function

\item {} 
\textbf{\texttt{f\_dt}} -- the duration distribution function

\item {} 
\textbf{\texttt{day}} (\emph{int}) -- the day the meal should occur

\end{itemize}

\end{description}\end{quote}
\index{initialize() (meal.Meal method)}

\begin{fulllineitems}
\phantomsection\label{meal:meal.Meal.initialize}\pysiglinewithargsret{\bfcode{initialize}}{\emph{t\_univ}}{}
At the beginning of the simulation, make sure that the meals are initialized to the proper times         (\code{t\_start\_univ}) so that they relate to the simulation time (t\_univ)
\begin{quote}\begin{description}
\item[{Parameters}] \leavevmode
\textbf{\texttt{t\_univ}} (\emph{int}) -- the simulation time {[}minutes, universal time{]}

\item[{Returns}] \leavevmode
None

\end{description}\end{quote}

\end{fulllineitems}

\index{print\_id() (meal.Meal method)}

\begin{fulllineitems}
\phantomsection\label{meal:meal.Meal.print_id}\pysiglinewithargsret{\bfcode{print\_id}}{}{}
This function returns a string representation of the meal id
\begin{quote}\begin{description}
\item[{Returns}] \leavevmode
a string representation of the meal id

\item[{Return type}] \leavevmode
str

\end{description}\end{quote}

\end{fulllineitems}

\index{set\_meal() (meal.Meal method)}

\begin{fulllineitems}
\phantomsection\label{meal:meal.Meal.set_meal}\pysiglinewithargsret{\bfcode{set\_meal}}{\emph{id}, \emph{start\_mean}, \emph{start\_std}, \emph{start\_trunc}, \emph{dt\_mean}, \emph{dt\_std}, \emph{dt\_trunc}}{}
This function sets the values associated with the Meal object.
\begin{quote}\begin{description}
\item[{Parameters}] \leavevmode\begin{itemize}
\item {} 
\textbf{\texttt{id}} (\emph{int}) -- the meal type (breakfast, lunch, or dinner)

\item {} 
\textbf{\texttt{start\_mean}} (\emph{int}) -- the mean start time of the meal {[}minutes, time of day{]}

\item {} 
\textbf{\texttt{start\_std}} (\emph{int}) -- the standard deviation of start time {[}minutes{]}

\item {} 
\textbf{\texttt{start\_turnc}} (\emph{int}) -- the number of standard deviations in the start time distribution

\item {} 
\textbf{\texttt{dt\_mean}} (\emph{int}) -- the mean duration of a meal {[}minutes{]}

\item {} 
\textbf{\texttt{dt\_std}} (\emph{int}) -- the standard deviation of meal duration {[}minutes{]}

\item {} 
\textbf{\texttt{dt\_trunc}} (\emph{int}) -- the number of standard deviations in the duration distribution

\end{itemize}

\item[{Returns}] \leavevmode
None

\end{description}\end{quote}

\end{fulllineitems}

\index{toString() (meal.Meal method)}

\begin{fulllineitems}
\phantomsection\label{meal:meal.Meal.toString}\pysiglinewithargsret{\bfcode{toString}}{}{}
This function returns a string representation of the Meal object.
\begin{quote}\begin{description}
\item[{Return msg}] \leavevmode
a string representation of the Meal object

\item[{Return type}] \leavevmode
str

\end{description}\end{quote}

\end{fulllineitems}

\index{update() (meal.Meal method)}

\begin{fulllineitems}
\phantomsection\label{meal:meal.Meal.update}\pysiglinewithargsret{\bfcode{update}}{\emph{day}}{}
Given the day for the meal, update the meal. The following does the following:
\begin{enumerate}
\item {} 
Update the start time for the meal

\item {} 
Update the duration for the meal

\item {} 
Update the universal start time for the meal

\end{enumerate}
\begin{quote}\begin{description}
\item[{Parameters}] \leavevmode
\textbf{\texttt{day}} (\emph{int}) -- the day for the meal to occur

\item[{Returns}] \leavevmode
None

\end{description}\end{quote}

\end{fulllineitems}

\index{update\_day() (meal.Meal method)}

\begin{fulllineitems}
\phantomsection\label{meal:meal.Meal.update_day}\pysiglinewithargsret{\bfcode{update\_day}}{}{}
Update the day for the next meal, given the universal start time for the meal (\code{t\_start\_univ}).
\begin{quote}\begin{description}
\item[{Returns}] \leavevmode
None

\end{description}\end{quote}

\end{fulllineitems}

\index{update\_dt() (meal.Meal method)}

\begin{fulllineitems}
\phantomsection\label{meal:meal.Meal.update_dt}\pysiglinewithargsret{\bfcode{update\_dt}}{}{}
Sample the duration distribution to get a duration.
\begin{quote}\begin{description}
\item[{Returns}] \leavevmode
None

\end{description}\end{quote}

\end{fulllineitems}

\index{update\_start() (meal.Meal method)}

\begin{fulllineitems}
\phantomsection\label{meal:meal.Meal.update_start}\pysiglinewithargsret{\bfcode{update\_start}}{}{}
Sample the start time distribution to get a start time.
\begin{quote}\begin{description}
\item[{Returns}] \leavevmode
None

\end{description}\end{quote}

\end{fulllineitems}

\index{update\_start\_univ() (meal.Meal method)}

\begin{fulllineitems}
\phantomsection\label{meal:meal.Meal.update_start_univ}\pysiglinewithargsret{\bfcode{update\_start\_univ}}{\emph{day}}{}
Given the day for the next meal, update the universal start time for the meal.
\begin{quote}\begin{description}
\item[{Parameters}] \leavevmode
\textbf{\texttt{day}} (\emph{int}) -- the day for the meal

\item[{Returns}] \leavevmode
None

\end{description}\end{quote}

\end{fulllineitems}


\end{fulllineitems}



\section{my\_globals module}
\label{my_globals::doc}\label{my_globals:module-my_globals}\label{my_globals:my-globals-module}\index{my\_globals (module)}
This module contains constants and functions that are used for general use.

This module contains information about the following constants:
\begin{enumerate}
\item {} 
Identifiers for activity codes

\item {} 
File names file paths for saving figures for the different demographics

\item {} 
File names file paths for saving figures for the different activities

\end{enumerate}

\emph{Module author: Dr. Namdi Brandon}
\index{fill\_out\_data() (in module my\_globals)}

\begin{fulllineitems}
\phantomsection\label{my_globals:my_globals.fill_out_data}\pysiglinewithargsret{\code{my\_globals.}\bfcode{fill\_out\_data}}{\emph{t}, \emph{y}}{}
This function takes an array of activity start times and activity codes from an activity diary and     fills out the activity, minute-by-minute in between two adjacent activities.
\begin{quote}\begin{description}
\item[{Parameters}] \leavevmode\begin{itemize}
\item {} 
\textbf{\texttt{t}} (\emph{numpy.ndarray}) -- the start times in an activity diary

\item {} 
\textbf{\texttt{y}} (\emph{numpy.ndarray}) -- the activity codes in an activity diary

\end{itemize}

\item[{Returns}] \leavevmode


\end{description}\end{quote}

\end{fulllineitems}

\index{fill\_out\_time() (in module my\_globals)}

\begin{fulllineitems}
\phantomsection\label{my_globals:my_globals.fill_out_time}\pysiglinewithargsret{\code{my\_globals.}\bfcode{fill\_out\_time}}{\emph{t}}{}
This function takes an array of activity start times from an activity diary and fills out the time,     minute-by-minute in between two adjacent activities

Example, if t = (0, 4, 7) (and \(t_{final}=10\)) we get the following:
\begin{itemize}
\item {} 
(0, 1, 2, 3)

\item {} 
(4, 5, 6)

\item {} 
(7, 8, 9, 10)

\end{itemize}
\begin{quote}\begin{description}
\item[{Parameters}] \leavevmode
\textbf{\texttt{t}} (\emph{numpy.ndarray}) -- the start times in the activity diary {[}minutes, universal time{]}

\item[{Returns}] \leavevmode
None

\end{description}\end{quote}

\end{fulllineitems}

\index{from\_periodic() (in module my\_globals)}

\begin{fulllineitems}
\phantomsection\label{my_globals:my_globals.from_periodic}\pysiglinewithargsret{\code{my\_globals.}\bfcode{from\_periodic}}{\emph{t}, \emph{do\_hours=True}}{}
This function returns the time of day in a 24 hour format. It takes the time \(t \in [-12, 12)\) and     expresses it at time \(x \in [0, 24)\) where 0 represents midnight. The same calculation can be     done to represent the time in minutes
\begin{quote}\begin{description}
\item[{Parameters}] \leavevmode\begin{itemize}
\item {} 
\textbf{\texttt{t}} (\emph{float}) -- the time in hours {[}-12, 12), or the respective minutes {[}-12 * 60, 12 * 60)

\item {} 
\textbf{\texttt{do\_hours}} (\emph{bool}) -- a flag to do the calculations in hours (if True)

\end{itemize}

\item[{Returns}] \leavevmode
the time in {[}0, 24) or in minutes {[}0, 24 * 60)

\end{description}\end{quote}

\end{fulllineitems}

\index{get\_ecdf() (in module my\_globals)}

\begin{fulllineitems}
\phantomsection\label{my_globals:my_globals.get_ecdf}\pysiglinewithargsret{\code{my\_globals.}\bfcode{get\_ecdf}}{\emph{data}, \emph{N=100}}{}
Given data, this function calculates the probability data from the empirical cumulative     distribution function (ECDF).
\begin{quote}\begin{description}
\item[{Parameters}] \leavevmode\begin{itemize}
\item {} 
\textbf{\texttt{data}} (\emph{float}) -- an array containing the relevant data to get the ECDF of

\item {} 
\textbf{\texttt{N}} (\emph{int}) -- the amount of samples in calculating the ECDF results

\end{itemize}

\item[{Return y}] \leavevmode
the ECDF

\item[{Return type}] \leavevmode
float array

\item[{Return x}] \leavevmode
the values sampled for the ECDF

\item[{Return type}] \leavevmode
float array

\end{description}\end{quote}

\end{fulllineitems}

\index{group\_time() (in module my\_globals)}

\begin{fulllineitems}
\phantomsection\label{my_globals:my_globals.group_time}\pysiglinewithargsret{\code{my\_globals.}\bfcode{group\_time}}{\emph{t}}{}
This function takes data from an activity diary and groups that activity diary into ,
minute by minute arrays from start to end for each activity (start, start + 1, ... end-1, end)
\begin{quote}\begin{description}
\item[{Parameters}] \leavevmode
\textbf{\texttt{t}} (\emph{numpy.ndarray}) -- the start times from an activity diary {[}minutes, universal time{]}

\item[{Returns}] \leavevmode
the grouped start/end pairs for ech activitiy

\item[{Return type}] \leavevmode
list

\end{description}\end{quote}

\end{fulllineitems}

\index{hours\_to\_minutes() (in module my\_globals)}

\begin{fulllineitems}
\phantomsection\label{my_globals:my_globals.hours_to_minutes}\pysiglinewithargsret{\code{my\_globals.}\bfcode{hours\_to\_minutes}}{\emph{t}}{}
This function takes a duration of time in hours and converts the time rounded to the nearest minutes.
\begin{quote}\begin{description}
\item[{Parameters}] \leavevmode
\textbf{\texttt{t}} (\emph{float}) -- a duration of time {[}hours{]}

\item[{Returns}] \leavevmode
the time in minutes

\end{description}\end{quote}

\end{fulllineitems}

\index{load() (in module my\_globals)}

\begin{fulllineitems}
\phantomsection\label{my_globals:my_globals.load}\pysiglinewithargsret{\code{my\_globals.}\bfcode{load}}{\emph{fname}}{}
This function loads data from a .pkl file.
\begin{quote}\begin{description}
\item[{Parameters}] \leavevmode
\textbf{\texttt{fname}} (\emph{str}) -- the file name to be loaded from

\item[{Returns}] \leavevmode
the data unpickled

\end{description}\end{quote}

\end{fulllineitems}

\index{sample\_normal() (in module my\_globals)}

\begin{fulllineitems}
\phantomsection\label{my_globals:my_globals.sample_normal}\pysiglinewithargsret{\code{my\_globals.}\bfcode{sample\_normal}}{\emph{std}, \emph{dx}}{}
This function samples a normal distribution centered at zero assuming a max and minimum acceptable value {[}dx, -dx{]}.
\begin{quote}\begin{description}
\item[{Parameters}] \leavevmode\begin{itemize}
\item {} 
\textbf{\texttt{std}} (\emph{float}) -- the standard deviation

\item {} 
\textbf{\texttt{dx}} (\emph{float}) -- the amount of total deviation from the mean allowd

\end{itemize}

\item[{Returns}] \leavevmode


\end{description}\end{quote}

\end{fulllineitems}

\index{save() (in module my\_globals)}

\begin{fulllineitems}
\phantomsection\label{my_globals:my_globals.save}\pysiglinewithargsret{\code{my\_globals.}\bfcode{save}}{\emph{x}, \emph{fname}}{}
This function saves a python variable by pickling it.
\begin{quote}\begin{description}
\item[{Parameters}] \leavevmode\begin{itemize}
\item {} 
\textbf{\texttt{x}} -- the data to be saved

\item {} 
\textbf{\texttt{fname}} (\emph{str}) -- the file name of the saved file. It must end with .pkl

\end{itemize}

\end{description}\end{quote}

\end{fulllineitems}

\index{save\_zip() (in module my\_globals)}

\begin{fulllineitems}
\phantomsection\label{my_globals:my_globals.save_zip}\pysiglinewithargsret{\code{my\_globals.}\bfcode{save\_zip}}{\emph{out\_file}, \emph{source\_dir}}{}
This function compresses an entire directory as a zip file.
\begin{quote}\begin{description}
\item[{Parameters}] \leavevmode\begin{itemize}
\item {} 
\textbf{\texttt{out\_file}} (\emph{str}) -- the filename of the save zip file with out the .zip extension

\item {} 
\textbf{\texttt{source\_dir}} (\emph{str}) -- the directory to be compressed

\end{itemize}

\item[{Returns}] \leavevmode
the name of the compressed directory

\end{description}\end{quote}

\end{fulllineitems}

\index{set\_distribution() (in module my\_globals)}

\begin{fulllineitems}
\phantomsection\label{my_globals:my_globals.set_distribution}\pysiglinewithargsret{\code{my\_globals.}\bfcode{set\_distribution}}{\emph{lower}, \emph{upper}, \emph{mu}, \emph{std}}{}
This function sets the truncated normal probability distribution.
\begin{quote}\begin{description}
\item[{Parameters}] \leavevmode\begin{itemize}
\item {} 
\textbf{\texttt{lower}} (\emph{int}) -- the lower bound in number of standard deviation from the mean

\item {} 
\textbf{\texttt{upper}} (\emph{int}) -- the upper bound in number of standard deviation from the mean

\item {} 
\textbf{\texttt{mu}} (\emph{int}) -- the mean

\item {} 
\textbf{\texttt{std}} (\emph{int}) -- the standard deviation

\end{itemize}

\item[{Returns}] \leavevmode
the function for the truncated normal distribution

\end{description}\end{quote}

\end{fulllineitems}

\index{set\_distribution\_dt() (in module my\_globals)}

\begin{fulllineitems}
\phantomsection\label{my_globals:my_globals.set_distribution_dt}\pysiglinewithargsret{\code{my\_globals.}\bfcode{set\_distribution\_dt}}{\emph{lower}, \emph{upper}, \emph{mu}, \emph{std}, \emph{x\_min}}{}
This function set the truncated normal probability distribution subject to the fact that there     is an assigned lowest value.

If the lowest value of the normal distribution is lower than the lowest     allowed value, change the distribution so that the standard deviation allows the distribution to not be     lower than the lowest allowed value.
\begin{quote}\begin{description}
\item[{Parameters}] \leavevmode\begin{itemize}
\item {} 
\textbf{\texttt{lower}} (\emph{int}) -- the lower bound in number of standard deviation from the mean

\item {} 
\textbf{\texttt{upper}} (\emph{int}) -- the upper bound in number of standard deviation from the mean

\item {} 
\textbf{\texttt{mu}} (\emph{int}) -- the mean

\item {} 
\textbf{\texttt{std}} (\emph{int}) -- the standard deviation

\item {} 
\textbf{\texttt{x\_min}} (\emph{int}) -- the lowest allowed value

\end{itemize}

\item[{Returns}] \leavevmode
the function for the truncated normal distribution, the standard deviation of the distribution

\item[{Return type}] \leavevmode
tuple

\end{description}\end{quote}

\end{fulllineitems}

\index{to\_periodic() (in module my\_globals)}

\begin{fulllineitems}
\phantomsection\label{my_globals:my_globals.to_periodic}\pysiglinewithargsret{\code{my\_globals.}\bfcode{to\_periodic}}{\emph{t}, \emph{do\_hours=True}}{}
This function returns the time of day in a periodic format. It takes the time \(t \in [0, 24)\) and     expresses it at time \(x \in [-12, 12)\) where 0 represents midnight.
\begin{quote}\begin{description}
\item[{Parameters}] \leavevmode\begin{itemize}
\item {} 
\textbf{\texttt{t}} (\emph{float}) -- the time in hours {[}0, 24)

\item {} 
\textbf{\texttt{do\_hours}} (\emph{bool}) -- a flag to do the calculations in hours (if True) or minutes if False

\end{itemize}

\item[{Returns}] \leavevmode
the time in {[}-12, 12) or minutes {[}-12 * 60, 12 * 60)

\item[{Return type}] \leavevmode
float

\end{description}\end{quote}

\end{fulllineitems}



\section{need module}
\label{need::doc}\label{need:module-need}\label{need:need-module}\index{need (module)}
This module contains information about governing the various needs that agents have in the simulation.

This module contains the class {\hyperref[need:need.Need]{\emph{\code{need.Need}}}}.

\emph{Module author: Dr. Namdi Brandon}
\index{Need (class in need)}

\begin{fulllineitems}
\phantomsection\label{need:need.Need}\pysiglinewithargsret{\strong{class }\code{need.}\bfcode{Need}}{\emph{clock}, \emph{num\_sample\_points}}{}
Bases: \code{object}

This class holds information about need-associations.
\begin{quote}\begin{description}
\item[{Parameters}] \leavevmode\begin{itemize}
\item {} 
\textbf{\texttt{clock}} ({\hyperref[temporal:temporal.Temporal]{\emph{\emph{temporal.Temporal}}}}) -- the clock governing time in the simulation

\item {} 
\textbf{\texttt{num\_sample\_points}} (\emph{int}) -- the number of time nodes in the simulation

\end{itemize}

\item[{Variables}] \leavevmode\begin{itemize}
\item {} 
\textbf{\texttt{category}} (\emph{int}) -- the need- identifier

\item {} 
\textbf{\texttt{clock}} ({\hyperref[temporal:temporal.Temporal]{\emph{\emph{temporal.Temporal}}}}) -- keeps track of the time

\item {} 
\textbf{\texttt{history}} (\emph{float}) -- an array containing the magnitude level \([0,1]\) of the need at all                         sample times.

\item {} 
\textbf{\texttt{magnitude}} (\emph{float}) -- the magnitude of the need (the satiation)

\item {} 
\textbf{\texttt{t0}} (\emph{int}) -- this keeps track of the last time the need was addressed

\item {} 
\textbf{\texttt{threshold}} (\emph{float}) -- the threshold of the need

\end{itemize}

\end{description}\end{quote}
\index{decay() (need.Need method)}

\begin{fulllineitems}
\phantomsection\label{need:need.Need.decay}\pysiglinewithargsret{\bfcode{decay}}{}{}
This calculates the amount of decay over a time step dt.

\begin{notice}{note}{Note:}
This function is meant to be overridden.
\end{notice}
\begin{quote}\begin{description}
\item[{Returns}] \leavevmode
None

\end{description}\end{quote}

\end{fulllineitems}

\index{initialize() (need.Need method)}

\begin{fulllineitems}
\phantomsection\label{need:need.Need.initialize}\pysiglinewithargsret{\bfcode{initialize}}{}{}
This function initializes the state of the Need at the very beginning of simulation.

\begin{notice}{note}{Note:}
This function is meant to be overridden.
\end{notice}
\begin{quote}\begin{description}
\item[{Returns}] \leavevmode
None

\end{description}\end{quote}

\end{fulllineitems}

\index{print\_category() (need.Need method)}

\begin{fulllineitems}
\phantomsection\label{need:need.Need.print_category}\pysiglinewithargsret{\bfcode{print\_category}}{}{}
This function represents the category as a string.
\begin{quote}\begin{description}
\item[{Returns}] \leavevmode
the string representation of the category

\item[{Return type}] \leavevmode
str

\end{description}\end{quote}

\end{fulllineitems}

\index{reset() (need.Need method)}

\begin{fulllineitems}
\phantomsection\label{need:need.Need.reset}\pysiglinewithargsret{\bfcode{reset}}{}{}
This function resets the values in order for the need to be used in the next simulation. This function         does the following:
\begin{enumerate}
\item {} 
sets the satiation to 1.0

\item {} 
sets the history to zero

\end{enumerate}
\begin{quote}\begin{description}
\item[{Returns}] \leavevmode
None

\end{description}\end{quote}

\end{fulllineitems}

\index{toString() (need.Need method)}

\begin{fulllineitems}
\phantomsection\label{need:need.Need.toString}\pysiglinewithargsret{\bfcode{toString}}{}{}
This function represents the Need as a string.
\begin{quote}\begin{description}
\item[{Return msg}] \leavevmode
The string representation of the Need object.

\item[{Return type}] \leavevmode
str

\end{description}\end{quote}

\end{fulllineitems}

\index{under\_threshold() (need.Need method)}

\begin{fulllineitems}
\phantomsection\label{need:need.Need.under_threshold}\pysiglinewithargsret{\bfcode{under\_threshold}}{\emph{n}}{}
Compares the value of a Need to the threshold.
\begin{quote}\begin{description}
\item[{Parameters}] \leavevmode
\textbf{\texttt{n}} (\emph{float}) -- the satiation

\item[{Returns}] \leavevmode
True if the satiation is less than or equal to the threshold, False otherwise

\item[{Return type}] \leavevmode
bool

\end{description}\end{quote}

\end{fulllineitems}

\index{weight() (need.Need method)}

\begin{fulllineitems}
\phantomsection\label{need:need.Need.weight}\pysiglinewithargsret{\bfcode{weight}}{\emph{n}}{}
This function calculates the weight function of a need.
\begin{quote}\begin{description}
\item[{Parameters}] \leavevmode
\textbf{\texttt{n}} (\emph{float}) -- the satiation

\item[{Returns}] \leavevmode
the weight due to the  need

\item[{Return type}] \leavevmode
float

\end{description}\end{quote}

\end{fulllineitems}


\end{fulllineitems}



\section{occupation module}
\label{occupation::doc}\label{occupation:module-occupation}\label{occupation:occupation-module}\index{occupation (module)}
This module contains info needed for the occupation of a Person. In addition, this file also contains functions useful for the module itself.

This module contains class {\hyperref[occupation:occupation.Occupation]{\emph{\code{occupation.Occupation}}}}.

This module contains constants relevant to the occupational information:
\begin{itemize}
\item {} 
job identifiers

\item {} 
job categories

\item {} 
default start time information

\item {} 
default end time information

\item {} 
default commuting to work information

\item {} 
default commuting from work information

\item {} 
default summer vacation (from school) information

\end{itemize}

\emph{Module author: Dr. Namdi Brandon}
\index{Occupation (class in occupation)}

\begin{fulllineitems}
\phantomsection\label{occupation:occupation.Occupation}\pysigline{\strong{class }\code{occupation.}\bfcode{Occupation}}
Bases: \code{object}

This class contains information relevant to an occupation of a Person.
\begin{quote}\begin{description}
\item[{Variables}] \leavevmode\begin{itemize}
\item {} 
\textbf{\texttt{category}} (\emph{int}) -- the category of the job

\item {} 
\textbf{\texttt{id}} (\emph{int}) -- the identifier for the job

\item {} 
\textbf{\texttt{commute\_to\_work\_dt\_mean}} (\emph{int}) -- the mean duration to commute to work {[}minutes{]}

\item {} 
\textbf{\texttt{commute\_to\_work\_dt\_std}} (\emph{int}) -- the standard deviation to commute to work {[}minutes{]}

\item {} 
\textbf{\texttt{commute\_to\_work\_dt}} (\emph{int}) -- the duration to commute to work {[}minutes{]}

\item {} 
\textbf{\texttt{commute\_to\_work\_dt\_trunc}} (\emph{int}) -- the number of standard deviation in the commute to work     duration distribution

\item {} 
\textbf{\texttt{commute\_to\_work\_start}} (\emph{int}) -- the start time for the commute to work activity {[}minutes{]}

\item {} 
\textbf{\texttt{dt\_commute}} (\emph{int}) -- the duration of the commute {[}minutes{]}

\item {} 
\textbf{\texttt{dt}} (\emph{int}) -- the duration of the work activity {[}minutes{]}

\item {} 
\textbf{\texttt{commute\_from\_work\_dt\_mean}} (\emph{int}) -- the mean duration to commute from work {[}minutes{]}

\item {} 
\textbf{\texttt{commute\_from\_work\_dt\_std}} (\emph{int}) -- the standard deviation to commute from work {[}minutes{]}

\item {} 
\textbf{\texttt{commute\_from\_work\_dt}} (\emph{int}) -- the duration to commute from work {[}minutes{]}

\item {} 
\textbf{\texttt{commute\_from\_work\_dt\_trunc}} (\emph{int}) -- the number of standard deviations in the commute from work     duration distribution

\item {} 
\textbf{\texttt{t\_start\_mean}} (\emph{int}) -- the mean start time for the job {[}minutes, time of day{]}

\item {} 
\textbf{\texttt{t\_start\_std}} (\emph{int}) -- the standard deviation of the start time for the job

\item {} 
\textbf{\texttt{t\_start}} (\emph{int}) -- the start time for the job {[}minutes, time of day{]}

\item {} 
\textbf{\texttt{t\_start\_univ}} (\emph{int}) -- the start time for the job {[}minutes, universal time{]}

\item {} 
\textbf{\texttt{work\_start\_trunc}} (\emph{int}) -- the number of standard deviations in the work start time distribution

\item {} 
\textbf{\texttt{day\_start}} (\emph{int}) -- the day the work activity start {[}minutes{]}

\item {} 
\textbf{\texttt{t\_end\_mean}} (\emph{int}) -- the mean end time for the job {[}minutes, time of day{]}

\item {} 
\textbf{\texttt{t\_end\_std}} (\emph{int}) -- the standard deviation of the end time for the job

\item {} 
\textbf{\texttt{t\_end}} (\emph{int}) -- the end time for the job {[}minutes, time of day{]}

\item {} 
\textbf{\texttt{t\_end\_univ}} (\emph{int}) -- the end time for the job {[}minutes, universal time{]}

\item {} 
\textbf{\texttt{work\_end\_trunc}} (\emph{int}) -- the number of standard deviations in the work end time distribution

\item {} 
\textbf{\texttt{is\_employed}} (\emph{bool}) -- this is a flag saying whether this is a job or not

\item {} 
\textbf{\texttt{is\_same\_day}} (\emph{bool}) -- This is a flag to see whether the start time and end time of a job are                             on the same day. If so, True. Else, False. If a person has \code{NO\_JOB}, the flag                             is set to True

\item {} 
\textbf{\texttt{'location'}} ({\hyperref[location:location.Location]{\emph{\emph{location.Location}}}}) -- the location of the Occupation

\item {} 
\textbf{\texttt{wage}} (\emph{float}) -- the yearly wage for that job {[}U.S. dollars{]}

\item {} 
\textbf{\texttt{work\_days}} (\emph{list}) -- a list of ints, giving the days the job starts

\item {} 
\textbf{\texttt{f\_commute\_to\_work\_dt}} -- the commute to work duration distribution

\item {} 
\textbf{\texttt{f\_commute\_from\_work\_dt}} -- the commute from work duration distribution

\item {} 
\textbf{\texttt{f\_work\_start}} -- the work start time distribution

\item {} 
\textbf{\texttt{f\_work\_end}} -- the work end time distribution

\end{itemize}

\end{description}\end{quote}
\index{is\_summer\_vacation() (occupation.Occupation method)}

\begin{fulllineitems}
\phantomsection\label{occupation:occupation.Occupation.is_summer_vacation}\pysiglinewithargsret{\bfcode{is\_summer\_vacation}}{\emph{week\_of\_year}}{}
This function returns True if the agent should not go to school due to summer vacation. False, otherwise.
\begin{quote}\begin{description}
\item[{Parameters}] \leavevmode
\textbf{\texttt{week\_of\_year}} (\emph{int}) -- the week of the year

\item[{Returns}] \leavevmode


\end{description}\end{quote}

\end{fulllineitems}

\index{print\_category() (occupation.Occupation method)}

\begin{fulllineitems}
\phantomsection\label{occupation:occupation.Occupation.print_category}\pysiglinewithargsret{\bfcode{print\_category}}{}{}
This function represents the Occupation category as a string
\begin{quote}\begin{description}
\item[{Returns}] \leavevmode
the string representation of a Occupation category

\item[{Return type}] \leavevmode
str

\end{description}\end{quote}

\end{fulllineitems}

\index{print\_id() (occupation.Occupation method)}

\begin{fulllineitems}
\phantomsection\label{occupation:occupation.Occupation.print_id}\pysiglinewithargsret{\bfcode{print\_id}}{}{}
This function writes the Occupation id as a string
\begin{quote}\begin{description}
\item[{Returns}] \leavevmode
a string representation of the job ID

\item[{Return type}] \leavevmode
str

\end{description}\end{quote}

\end{fulllineitems}

\index{set\_commute\_distribution() (occupation.Occupation method)}

\begin{fulllineitems}
\phantomsection\label{occupation:occupation.Occupation.set_commute_distribution}\pysiglinewithargsret{\bfcode{set\_commute\_distribution}}{}{}
This function sets the following:
\begin{itemize}
\item {} 
commute to work duration distribution

\item {} 
commute from work duration distribution.

\end{itemize}
\begin{quote}\begin{description}
\item[{Returns}] \leavevmode
None

\end{description}\end{quote}

\end{fulllineitems}

\index{set\_is\_job() (occupation.Occupation method)}

\begin{fulllineitems}
\phantomsection\label{occupation:occupation.Occupation.set_is_job}\pysiglinewithargsret{\bfcode{set\_is\_job}}{}{}
This function checks to see if the current job is actually a job (eg. that it is not         \code{NO\_JOB}).

Sets self.is\_job to True if the Occupation is \code{NO\_JOB}, returns False otherwise
\begin{quote}\begin{description}
\item[{Returns}] \leavevmode
None

\end{description}\end{quote}

\end{fulllineitems}

\index{set\_is\_same\_day() (occupation.Occupation method)}

\begin{fulllineitems}
\phantomsection\label{occupation:occupation.Occupation.set_is_same_day}\pysiglinewithargsret{\bfcode{set\_is\_same\_day}}{}{}
This function sets a flag indicating whether or not a job starts and ends         on the same day. The function sets \code{is\_same\_day} to True if the Occupation start time and end time         are within the same day. False, otherwise.
\begin{quote}\begin{description}
\item[{Returns}] \leavevmode
None

\end{description}\end{quote}

\end{fulllineitems}

\index{set\_job\_params() (occupation.Occupation method)}

\begin{fulllineitems}
\phantomsection\label{occupation:occupation.Occupation.set_job_params}\pysiglinewithargsret{\bfcode{set\_job\_params}}{\emph{id\_job}, \emph{start\_mean}, \emph{start\_std}, \emph{end\_mean}, \emph{end\_std}, \emph{commute\_to\_work\_dt\_mean}, \emph{commute\_to\_work\_dt\_std}, \emph{commute\_from\_work\_dt\_mean}, \emph{commute\_from\_work\_dt\_std}}{}
This function sets the Occupation parameters.
\begin{quote}\begin{description}
\item[{Parameters}] \leavevmode\begin{itemize}
\item {} 
\textbf{\texttt{id\_job}} (\emph{int}) -- the job identifier

\item {} 
\textbf{\texttt{start\_mean}} (\emph{int}) -- the mean start time for the occupation

\item {} 
\textbf{\texttt{start\_std}} (\emph{int}) -- the standard deviation of the start time for the occupation

\item {} 
\textbf{\texttt{end\_mean}} (\emph{int}) -- the mean end time for the occupation

\item {} 
\textbf{\texttt{end\_std}} (\emph{int}) -- the standard deviation for the end time

\item {} 
\textbf{\texttt{commute\_to\_work\_dt\_mean}} (\emph{int}) -- the mean commute to work duration

\item {} 
\textbf{\texttt{commute\_to\_work\_dt\_std}} (\emph{int}) -- the standard deviation of the commute to work duration

\item {} 
\textbf{\texttt{commute\_from\_work\_dt\_mean}} (\emph{int}) -- the mean commute from work duration

\item {} 
\textbf{\texttt{commute\_from\_work\_dt\_std}} (\emph{int}) -- the standard deviation to commute from work duration

\end{itemize}

\item[{Returns}] \leavevmode
None

\end{description}\end{quote}

\end{fulllineitems}

\index{set\_job\_preset() (occupation.Occupation method)}

\begin{fulllineitems}
\phantomsection\label{occupation:occupation.Occupation.set_job_preset}\pysiglinewithargsret{\bfcode{set\_job\_preset}}{}{}
Sets Occupation to one of the following preset jobs:
\begin{itemize}
\item {} 
\code{NO\_JOB}

\item {} 
\code{STANDARD\_JOB}

\item {} 
\code{STUDENT}

\end{itemize}
\begin{quote}\begin{description}
\item[{Returns}] \leavevmode
None

\end{description}\end{quote}

\end{fulllineitems}

\index{set\_no\_job() (occupation.Occupation method)}

\begin{fulllineitems}
\phantomsection\label{occupation:occupation.Occupation.set_no_job}\pysiglinewithargsret{\bfcode{set\_no\_job}}{}{}
Set the Occupation to having no job.
\begin{quote}\begin{description}
\item[{Parameters}] \leavevmode
\textbf{\texttt{job}} ({\hyperref[occupation:occupation.Occupation]{\emph{\emph{occupation.Occupation}}}}) -- the job of which to set the attributes

\item[{Returns}] \leavevmode
None

\end{description}\end{quote}

\end{fulllineitems}

\index{set\_standard\_job() (occupation.Occupation method)}

\begin{fulllineitems}
\phantomsection\label{occupation:occupation.Occupation.set_standard_job}\pysiglinewithargsret{\bfcode{set\_standard\_job}}{}{}
This function sets the Occupation to the standard job.
\begin{itemize}
\item {} 
has fixed shift of 9:00 - 17:00

\item {} 
Monday through Friday

\item {} 
wage of \$40,000

\item {} 
30 minute commute to work

\item {} 
60 minute commute from work

\end{itemize}
\begin{quote}\begin{description}
\item[{Parameters}] \leavevmode
\textbf{\texttt{job}} ({\hyperref[occupation:occupation.Occupation]{\emph{\emph{occupation.Occupation}}}}) -- the job of which to set the attributes

\item[{Returns}] \leavevmode
None

\end{description}\end{quote}

\end{fulllineitems}

\index{set\_student() (occupation.Occupation method)}

\begin{fulllineitems}
\phantomsection\label{occupation:occupation.Occupation.set_student}\pysiglinewithargsret{\bfcode{set\_student}}{}{}
This function sets the Occupation to the standard job.
\begin{itemize}
\item {} 
Fixed shift of 8:00 - 15:00

\item {} 
Monday through Friday

\item {} 
wage of \$0

\item {} 
30 minute commute to school

\item {} 
60 minute commute from school

\end{itemize}
\begin{quote}\begin{description}
\item[{Parameters}] \leavevmode
\textbf{\texttt{job}} ({\hyperref[occupation:occupation.Occupation]{\emph{\emph{occupation.Occupation}}}}) -- the job of which to set the attributes

\item[{Returns}] \leavevmode
None

\end{description}\end{quote}

\end{fulllineitems}

\index{set\_work\_distribution() (occupation.Occupation method)}

\begin{fulllineitems}
\phantomsection\label{occupation:occupation.Occupation.set_work_distribution}\pysiglinewithargsret{\bfcode{set\_work\_distribution}}{}{}
This function sets the following distributions for work:
\begin{itemize}
\item {} 
work start time distribution

\item {} 
work end time distribution

\end{itemize}
\begin{quote}\begin{description}
\item[{Returns}] \leavevmode
None

\end{description}\end{quote}

\end{fulllineitems}

\index{toString() (occupation.Occupation method)}

\begin{fulllineitems}
\phantomsection\label{occupation:occupation.Occupation.toString}\pysiglinewithargsret{\bfcode{toString}}{}{}
Represents the Occupation object as a string
\begin{quote}\begin{description}
\item[{Return msg}] \leavevmode
The representation of the Occupation object as a string

\item[{Return type}] \leavevmode
str

\end{description}\end{quote}

\end{fulllineitems}

\index{update\_commute\_from\_work\_dt() (occupation.Occupation method)}

\begin{fulllineitems}
\phantomsection\label{occupation:occupation.Occupation.update_commute_from_work_dt}\pysiglinewithargsret{\bfcode{update\_commute\_from\_work\_dt}}{}{}
Pull a commute from work duration from the respective distribution.
\begin{quote}\begin{description}
\item[{Returns}] \leavevmode
None

\end{description}\end{quote}

\end{fulllineitems}

\index{update\_commute\_to\_work\_dt() (occupation.Occupation method)}

\begin{fulllineitems}
\phantomsection\label{occupation:occupation.Occupation.update_commute_to_work_dt}\pysiglinewithargsret{\bfcode{update\_commute\_to\_work\_dt}}{}{}
Pull a commute to work duration from the respective distribution. Also, update the commute to work         start time place holder.
\begin{quote}\begin{description}
\item[{Returns}] \leavevmode
None

\end{description}\end{quote}

\end{fulllineitems}

\index{update\_commute\_to\_work\_start() (occupation.Occupation method)}

\begin{fulllineitems}
\phantomsection\label{occupation:occupation.Occupation.update_commute_to_work_start}\pysiglinewithargsret{\bfcode{update\_commute\_to\_work\_start}}{}{}
Update the commute to work start time.
\begin{quote}\begin{description}
\item[{Returns}] \leavevmode
None

\end{description}\end{quote}

\end{fulllineitems}

\index{update\_work\_dt() (occupation.Occupation method)}

\begin{fulllineitems}
\phantomsection\label{occupation:occupation.Occupation.update_work_dt}\pysiglinewithargsret{\bfcode{update\_work\_dt}}{}{}
Update the work duration
\begin{quote}\begin{description}
\item[{Returns}] \leavevmode
None

\end{description}\end{quote}

\end{fulllineitems}

\index{update\_work\_end() (occupation.Occupation method)}

\begin{fulllineitems}
\phantomsection\label{occupation:occupation.Occupation.update_work_end}\pysiglinewithargsret{\bfcode{update\_work\_end}}{}{}
Update the work end time.
\begin{quote}\begin{description}
\item[{Returns}] \leavevmode
None

\end{description}\end{quote}

\end{fulllineitems}

\index{update\_work\_start() (occupation.Occupation method)}

\begin{fulllineitems}
\phantomsection\label{occupation:occupation.Occupation.update_work_start}\pysiglinewithargsret{\bfcode{update\_work\_start}}{}{}
Update the work start time.
\begin{quote}\begin{description}
\item[{Returns}] \leavevmode
None

\end{description}\end{quote}

\end{fulllineitems}


\end{fulllineitems}

\index{is\_work\_time() (in module occupation)}

\begin{fulllineitems}
\phantomsection\label{occupation:occupation.is_work_time}\pysiglinewithargsret{\code{occupation.}\bfcode{is\_work\_time}}{\emph{clock}, \emph{job}, \emph{is\_commute\_to\_work=False}}{}
Given a clock and a job, this function says whether the clock's time corresponds     to a time to be at work \textbf{or} a time to commute to work.

If \(\Delta{t} > 0\), it indicates when it's time to commute to work.
\begin{quote}\begin{description}
\item[{Parameters}] \leavevmode\begin{itemize}
\item {} 
\textbf{\texttt{clock}} ({\hyperref[temporal:temporal.Temporal]{\emph{\emph{temporal.Temporal}}}}) -- the time

\item {} 
\textbf{\texttt{job}} ({\hyperref[occupation:occupation.Occupation]{\emph{\emph{occupation.Occupation}}}}) -- the job to inquiry

\item {} 
\textbf{\texttt{is\_commute\_to\_work}} (\emph{bool}) -- a flag indicating whether we are interested in calculating if it is                             time to commute to work

\end{itemize}

\item[{Returns}] \leavevmode
a flag indicating if it is / is not work time (or commute time if is\_commute\_to\_work is True)

\item[{Return type}] \leavevmode
bool

\end{description}\end{quote}

\end{fulllineitems}

\index{is\_work\_time\_help() (in module occupation)}

\begin{fulllineitems}
\phantomsection\label{occupation:occupation.is_work_time_help}\pysiglinewithargsret{\code{occupation.}\bfcode{is\_work\_time\_help}}{\emph{clock}, \emph{job}}{}
Given a clock and a job, this function says whether the clock's time corresponds to
a time at work.
\begin{quote}\begin{description}
\item[{Parameters}] \leavevmode\begin{itemize}
\item {} 
\textbf{\texttt{clock}} ({\hyperref[temporal:temporal.Temporal]{\emph{\emph{temporal.Temporal}}}}) -- the time

\item {} 
\textbf{\texttt{job}} ({\hyperref[occupation:occupation.Occupation]{\emph{\emph{occupation.Occupation}}}}) -- the job to inquiry

\end{itemize}

\item[{Returns}] \leavevmode
is\_work\_time: a flag indicating if the time (clock) corresponds to a work time

\item[{Return type}] \leavevmode
bool

\end{description}\end{quote}

\end{fulllineitems}

\index{set\_grave\_shift() (in module occupation)}

\begin{fulllineitems}
\phantomsection\label{occupation:occupation.set_grave_shift}\pysiglinewithargsret{\code{occupation.}\bfcode{set\_grave\_shift}}{\emph{job}}{}
This function sets the Occupation to a grave shift.
\begin{itemize}
\item {} 
shift job from  22:00 to 6:00

\item {} 
Monday through Friday

\item {} 
30 minute commute to work

\item {} 
60 minute commute from work

\item {} 
wage of \$40,0000.

\end{itemize}
\begin{quote}\begin{description}
\item[{Parameters}] \leavevmode
\textbf{\texttt{job}} ({\hyperref[occupation:occupation.Occupation]{\emph{\emph{occupation.Occupation}}}}) -- the job of which to set the attributes

\item[{Returns}] \leavevmode
None

\end{description}\end{quote}

\end{fulllineitems}

\index{set\_no\_job() (in module occupation)}

\begin{fulllineitems}
\phantomsection\label{occupation:occupation.set_no_job}\pysiglinewithargsret{\code{occupation.}\bfcode{set\_no\_job}}{\emph{job}}{}
Set the Occupation to having no job.
\begin{quote}\begin{description}
\item[{Parameters}] \leavevmode
\textbf{\texttt{job}} ({\hyperref[occupation:occupation.Occupation]{\emph{\emph{occupation.Occupation}}}}) -- the job of which to set the attributes

\item[{Returns}] \leavevmode
None

\end{description}\end{quote}

\end{fulllineitems}

\index{set\_standard\_job() (in module occupation)}

\begin{fulllineitems}
\phantomsection\label{occupation:occupation.set_standard_job}\pysiglinewithargsret{\code{occupation.}\bfcode{set\_standard\_job}}{\emph{job}}{}
This function sets the Occupation to the standard job.
\begin{itemize}
\item {} 
fixed shift of 9:00 - 17:00

\item {} 
Monday through Friday

\item {} 
wage \$40,000

\item {} 
30 minute commute to work

\item {} 
60 minute commute from work

\end{itemize}
\begin{quote}\begin{description}
\item[{Parameters}] \leavevmode
\textbf{\texttt{job}} ({\hyperref[occupation:occupation.Occupation]{\emph{\emph{occupation.Occupation}}}}) -- the job of which to set the attributes

\item[{Returns}] \leavevmode
None

\end{description}\end{quote}

\end{fulllineitems}

\index{set\_student() (in module occupation)}

\begin{fulllineitems}
\phantomsection\label{occupation:occupation.set_student}\pysiglinewithargsret{\code{occupation.}\bfcode{set\_student}}{\emph{job}}{}
This function sets a job to the preset values of student occupation.
\begin{itemize}
\item {} 
fixed shift of 08:00 - 15:00

\item {} 
Monday through Friday

\item {} 
wage of \$0

\item {} 
30 minute commute to school

\item {} 
60 minute commute from school

\end{itemize}
\begin{quote}\begin{description}
\item[{Parameters}] \leavevmode
\textbf{\texttt{job}} ({\hyperref[occupation:occupation.Occupation]{\emph{\emph{occupation.Occupation}}}}) -- the job to set

\item[{Returns}] \leavevmode
None

\end{description}\end{quote}

\end{fulllineitems}



\section{params module}
\label{params::doc}\label{params:params-module}\label{params:module-params}\index{params (module)}
The purpose of this module is to assign parameters necessary to run the Agent-Based Model of Human Activity Patterns (ABMHAP) initialized with data from the Consolidated Human Activity Database (CHAD). This module also have constants used in default runs.

This module contains class {\hyperref[params:params.Params]{\emph{\code{params.Params}}}}.
\begin{quote}

\emph{Module author: Dr. Namdi Brandon}
\end{quote}
\index{Params (class in params)}

\begin{fulllineitems}
\phantomsection\label{params:params.Params}\pysiglinewithargsret{\strong{class }\code{params.}\bfcode{Params}}{\emph{dt=1}, \emph{num\_people=1}, \emph{num\_days=1}, \emph{num\_hours=0}, \emph{num\_min=0}, \emph{t\_start=1865}, \emph{gender=None}, \emph{sleep\_start\_mean=None}, \emph{sleep\_start\_std=None}, \emph{sleep\_end\_mean=None}, \emph{sleep\_end\_std=None}, \emph{job\_id=None}, \emph{do\_alarm=None}, \emph{dt\_alarm=None}, \emph{bf\_start\_mean=None}, \emph{bf\_start\_std=None}, \emph{bf\_start\_trunc=None}, \emph{bf\_dt\_mean=None}, \emph{bf\_dt\_std=None}, \emph{bf\_dt\_trunc=None}, \emph{lunch\_start\_mean=None}, \emph{lunch\_start\_std=None}, \emph{lunch\_start\_trunc=None}, \emph{lunch\_dt\_mean=None}, \emph{lunch\_dt\_std=None}, \emph{lunch\_dt\_trunc=None}, \emph{dinner\_start\_mean=None}, \emph{dinner\_start\_std=None}, \emph{dinner\_start\_trunc=None}, \emph{dinner\_dt\_mean=None}, \emph{dinner\_dt\_std=None}, \emph{dinner\_dt\_trunc=None}, \emph{work\_start\_mean=None}, \emph{work\_start\_std=None}, \emph{work\_end\_mean=None}, \emph{work\_end\_std=None}, \emph{commute\_to\_work\_dt\_mean=None}, \emph{commute\_to\_work\_dt\_std=None}, \emph{commute\_from\_work\_dt\_mean=None}, \emph{commute\_from\_work\_dt\_std=None}}{}
Bases: \code{object}

This class contains the parameters that are needed to parametrize a household.

\begin{notice}{note}{Note:}
Some of the class attributes are \textbf{not} really used and need to be phased out in future versions of the         model. Some of these attribtues are:
\begin{itemize}
\item {} 
\code{dt}

\item {} 
\code{do\_alarm}

\item {} 
\code{dt\_alarm}

\end{itemize}
\end{notice}
\begin{quote}\begin{description}
\item[{Parameters}] \leavevmode\begin{itemize}
\item {} 
\textbf{\texttt{dt}} (\emph{int}) -- the step size {[}in minutes{]} in the simulation

\item {} 
\textbf{\texttt{num\_people}} (\emph{int}) -- the number of people in the household

\item {} 
\textbf{\texttt{num\_days}} (\emph{int}) -- the number of days in the simulation

\item {} 
\textbf{\texttt{num\_hours}} (\emph{int}) -- the number of additional hours in the simulation

\item {} 
\textbf{\texttt{num\_min}} (\emph{int}) -- the number of additional minutes in the simulation

\item {} 
\textbf{\texttt{t\_start}} (\emph{int}) -- the start time {[}in minutes{]} in the simulation

\item {} 
\textbf{\texttt{gender}} (\emph{list}) -- the gender of each person in the household

\item {} 
\textbf{\texttt{sleep\_start\_mean}} (\emph{list}) -- the mean sleep start time {[}in minutes, time of day{]} for each person in     the household

\item {} 
\textbf{\texttt{sleep\_start\_std}} (\emph{list}) -- the standard deviation of sleep start time {[}in minutes{]} for each person in the     household

\item {} 
\textbf{\texttt{sleep\_end\_mean}} (\emph{list}) -- the mean sleep end time {[}in minutes, time of day{]} for each person in the household

\item {} 
\textbf{\texttt{sleep\_end\_std}} (\emph{list}) -- the standard deviation of the sleep end time {[}in minutes{]} for each person in     the household

\item {} 
\textbf{\texttt{job\_id}} (\emph{list}) -- the occupation identifier for each person in the household

\item {} 
\textbf{\texttt{do\_alarm}} (\emph{list}) -- a flag indicating whether or not a person uses an alarm for each person in the household

\item {} 
\textbf{\texttt{dt\_alarm}} (\emph{list}) -- the duration of time {[}in minutes{]} before an alarm goes off before its respective event

\item {} 
\textbf{\texttt{numpy.ndarray}} -- bf\_start\_mean: the mean breakfast start time for each person in the household     {[}minutes, time of day{]}

\item {} 
\textbf{\texttt{bf\_start\_std}} (\emph{numpy.ndarray}) -- the standard deviation for breakfast start time for each person in the     household {[}minutes{]}

\item {} 
\textbf{\texttt{bf\_start\_trunc}} (\emph{numpy.ndarray}) -- the number of standard deviations used in the breakfast start time     distribution for each person

\item {} 
\textbf{\texttt{bf\_dt\_mean}} (\emph{numpy.ndarray}) -- the mean breakfast duration for each person in the household {[}minutes{]}

\item {} 
\textbf{\texttt{bf\_dt\_std}} (\emph{numpy.ndarray}) -- the standard deviation for breakfast duration for each person in the     household {[}minutes{]}

\item {} 
\textbf{\texttt{bf\_dt\_trunc}} (\emph{numpy.ndarray}) -- the number of standard deviations used in the breakfast duration     distribution for each person

\item {} 
\textbf{\texttt{lunch\_dt\_mean}} (\emph{numpy.ndarray}) -- the mean lunch duration for each person in the household {[}minutes{]}

\item {} 
\textbf{\texttt{lunch\_dt\_std}} (\emph{numpy.ndarray}) -- the standard deviation for lunch duration for each person in the     household {[}minutes{]}

\item {} 
\textbf{\texttt{lunch\_dt\_trunc}} (\emph{numpy.ndarray}) -- the number of standard deviations used in the lunch duration     distribution for each person

\item {} 
\textbf{\texttt{lunch\_start\_mean}} (\emph{numpy.ndarray}) -- the mean lunch start time for each person in the household     {[}minutes, time of day{]}

\item {} 
\textbf{\texttt{lunch\_start\_std}} (\emph{numpy.ndarray}) -- the standard deviation for lunch start time for each person in the     household {[}minutes{]}

\item {} 
\textbf{\texttt{lunch\_start\_trunc}} (\emph{numpy.ndarray}) -- the number of standard deviations used in the lunch start time     distribution for each person

\item {} 
\textbf{\texttt{dinner\_start\_mean}} (\emph{numpy.ndarray}) -- the mean dinner start time for each person in the household     {[}minutes, time of day{]}

\item {} 
\textbf{\texttt{dinner\_start\_std}} (\emph{numpy.ndarray}) -- the standard deviation for dinner start time for each person in the     household {[}minutes{]}

\item {} 
\textbf{\texttt{dinner\_start\_trunc}} (\emph{numpy.ndarray}) -- the number of standard deviations used in the dinner start time     distribution for each person

\item {} 
\textbf{\texttt{dinner\_dt\_mean}} (\emph{numpy.ndarray}) -- the mean dinner duration for each person in the household {[}minutes{]}

\item {} 
\textbf{\texttt{dinner\_dt\_std}} (\emph{numpy.ndarray}) -- the standard deviation for dinner duration for each person in the     household {[}minutes{]}

\item {} 
\textbf{\texttt{dinner\_dt\_trunc}} (\emph{numpy.ndarray}) -- the number of standard deviations used in the dinner duration     distribution for each person

\item {} 
\textbf{\texttt{work\_start\_mean}} (\emph{numpy.ndarray}) -- the mean work start time for each person in the household     {[}minutes, time of day{]}

\item {} 
\textbf{\texttt{work\_start\_std}} (\emph{numpy.ndarray}) -- the standard deviation of work start time for each person in the     household {[}minutes{]}

\item {} 
\textbf{\texttt{work\_end\_mean}} (\emph{numpy.ndarray}) -- the work end time for each person in the household {[}minutes, time of day{]}

\item {} 
\textbf{\texttt{work\_end\_std}} (\emph{numpy.ndarray}) -- the work standard deviation for each person in the household     {[}minutes, time of day{]}

\item {} 
\textbf{\texttt{commute\_to\_work\_dt\_mean}} (\emph{numpy.ndarray}) -- the mean duration for commuting to work {[}minutes{]} for each     person in the household

\item {} 
\textbf{\texttt{commute\_to\_work\_dt\_std}} (\emph{numpy.ndarray}) -- the standard deviation for commuting to work {[}minutes{]} for each     person in the household

\item {} 
\textbf{\texttt{commute\_from\_work\_dt\_mean}} (\emph{numpy.ndarray}) -- the mean duration for commuting from work {[}minutes{]} for each     person in the household

\item {} 
\textbf{\texttt{commute\_from\_work\_dt\_std}} (\emph{numpy.ndarray}) -- the standard deviation for commuting from work {[}minutes{]} for     each person in the household

\end{itemize}

\item[{Variables}] \leavevmode\begin{itemize}
\item {} 
\textbf{\texttt{dt}} (\emph{int}) -- the step size {[}in minutes{]} in the simulation

\item {} 
\textbf{\texttt{num\_people}} (\emph{int}) -- the number of people in the household

\item {} 
\textbf{\texttt{num\_days}} (\emph{int}) -- the number of days in the simulation

\item {} 
\textbf{\texttt{num\_hours}} (\emph{int}) -- the number of additional hours in the simulation

\item {} 
\textbf{\texttt{num\_min}} (\emph{int}) -- the number of additional minutes in the simulation

\item {} 
\textbf{\texttt{t\_start}} (\emph{int}) -- the start time {[}in minutes{]} in the simulation

\item {} 
\textbf{\texttt{gender}} (\emph{list}) -- the gender of each person in the household

\item {} 
\textbf{\texttt{sleep\_start\_mean}} (\emph{list}) -- the mean sleep start time {[}in minutes, time of day{]} for each person in     the household

\item {} 
\textbf{\texttt{sleep\_start\_std}} (\emph{list}) -- the standard deviation of sleep start time {[}in minutes{]} for each person in the     household

\item {} 
\textbf{\texttt{sleep\_end\_mean}} (\emph{list}) -- the mean sleep end time {[}in minutes, time of day{]} for each person in the household

\item {} 
\textbf{\texttt{sleep\_end\_std}} (\emph{list}) -- the standard deviation of the sleep end time {[}in minutes{]} for each person in     the household

\item {} 
\textbf{\texttt{job\_id}} (\emph{list}) -- the occupation identifier for each person in the household

\item {} 
\textbf{\texttt{do\_alarm}} (\emph{list}) -- a flag indicating whether or not a person uses an alarm for each person in the household

\item {} 
\textbf{\texttt{dt\_alarm}} (\emph{list}) -- the duration of time {[}in minutes{]} before an alarm goes off before its respective event

\item {} 
\textbf{\texttt{breakfasts}} (\emph{list}) -- the breakfast meal objects for each person in the household

\item {} 
\textbf{\texttt{lunches}} (\emph{list}) -- the lunch meal objects for each person in the household

\item {} 
\textbf{\texttt{dinners}} (\emph{list}) -- the dinner meal objects for each person in the household

\item {} 
\textbf{\texttt{work\_start\_mean}} (\emph{numpy.ndarray}) -- the mean work start time for each person in the household     {[}minutes, time of day{]}

\item {} 
\textbf{\texttt{work\_start\_std}} (\emph{numpy.ndarray}) -- the standard deviation of work start time for each person in the     household {[}minutes{]}

\item {} 
\textbf{\texttt{work\_end\_mean}} (\emph{numpy.ndarray}) -- the work end time for each person in the household {[}minutes, time of day{]}

\item {} 
\textbf{\texttt{work\_end\_std}} (\emph{numpy.ndarray}) -- the work standard deviation for each person in the household     {[}minutes, time of day{]}

\item {} 
\textbf{\texttt{commute\_to\_work\_dt\_mean}} (\emph{numpy.ndarray}) -- the mean duration for commuting to work {[}minutes{]} for each person     in the household

\item {} 
\textbf{\texttt{commute\_to\_work\_dt\_std}} (\emph{numpy.ndarray}) -- the standard deviation for commuting to work {[}minutes{]} for     each person in the household

\item {} 
\textbf{\texttt{commute\_from\_work\_dt\_mean}} (\emph{numpy.ndarray}) -- the mean duration for commuting from work {[}minutes{]} for     each person in the household

\item {} 
\textbf{\texttt{commute\_from\_work\_dt\_std}} (\emph{numpy.ndarray}) -- the standard deviation for commuting from work {[}minutes{]} for each     person in the household

\end{itemize}

\end{description}\end{quote}
\index{init\_help() (params.Params method)}

\begin{fulllineitems}
\phantomsection\label{params:params.Params.init_help}\pysiglinewithargsret{\bfcode{init\_help}}{\emph{val}, \emph{default\_val}}{}
This function assigns a default value to an attribute in case it was not assigned already. This is,         function is particularly useful if the value to be assigned is an array depending on \code{num\_people}

More specifically,
\begin{itemize}
\item {} 
If val is not None, return val

\item {} 
If val is None, return the default value (default\_val)

\end{itemize}
\begin{quote}\begin{description}
\item[{Parameters}] \leavevmode\begin{itemize}
\item {} 
\textbf{\texttt{val}} -- the value to be assigned

\item {} 
\textbf{\texttt{default\_val}} -- the default value to assign in case val is None

\end{itemize}

\item[{Returns}] \leavevmode
the non-None value

\end{description}\end{quote}

\end{fulllineitems}

\index{init\_meal() (params.Params method)}

\begin{fulllineitems}
\phantomsection\label{params:params.Params.init_meal}\pysiglinewithargsret{\bfcode{init\_meal}}{\emph{m\_id}, \emph{start\_mean=None}, \emph{start\_std=None}, \emph{start\_trunc=None}, \emph{dt\_mean=None}, \emph{dt\_std=None}, \emph{dt\_trunc=None}}{}
This function returns the data for each person in the household for the respective meal given by ``m\_id''
\begin{itemize}
\item {} 
if specific parameters have been assigned, create meals with the respective parameters

\item {} 
if specific parameters have not been assigned, create meals with the default meal parameters for         each meal

\end{itemize}
\begin{quote}\begin{description}
\item[{Parameters}] \leavevmode\begin{itemize}
\item {} 
\textbf{\texttt{m\_id}} (\emph{int}) -- the identifier of meal type

\item {} 
\textbf{\texttt{start\_mean}} (\emph{numpy.ndarray}) -- the mean start time for the meal for each person in the household

\item {} 
\textbf{\texttt{start\_std}} (\emph{numpy.ndarray}) -- the standard deviation of start time for the meal for each person in         the household

\item {} 
\textbf{\texttt{start\_trunc}} (\emph{numpy.ndarray}) -- the amount of standard deviations allowed before truncating the         start time distribution for each person in the household

\item {} 
\textbf{\texttt{dt\_mean}} (\emph{numpy.ndarray}) -- the mean duration for the meal for each person in the household

\item {} 
\textbf{\texttt{dt\_std}} (\emph{numpy.ndarray}) -- the standard deviation for the meal for each person in the household

\item {} 
\textbf{\texttt{dt\_trunc}} (\emph{numpy.ndarray}) -- the amount of standard deviations allowed before truncating the         duration distribution for each person in the household

\end{itemize}

\item[{Returns}] \leavevmode
the meals for each person in the household

\item[{Return type}] \leavevmode
list

\end{description}\end{quote}

\end{fulllineitems}

\index{init\_meal\_old() (params.Params method)}

\begin{fulllineitems}
\phantomsection\label{params:params.Params.init_meal_old}\pysiglinewithargsret{\bfcode{init\_meal\_old}}{\emph{id}, \emph{start\_mean=None}, \emph{start\_std=None}, \emph{dt\_mean=None}, \emph{dt\_std=None}}{}
This function returns the data for each person in the household for the respective meal given by ``id''

\begin{notice}{warning}{Warning:}
This function may be \textbf{not} used because it is antiquated.
\end{notice}
\begin{quote}\begin{description}
\item[{Parameters}] \leavevmode\begin{itemize}
\item {} 
\textbf{\texttt{id}} (\emph{int}) -- the id of meal type

\item {} 
\textbf{\texttt{start\_mean}} (\emph{numpy.ndarray}) -- the mean start time for the meal for each person in the household

\item {} 
\textbf{\texttt{dt\_mean}} (\emph{numpy.ndarray}) -- the mean duration for the meal for each person in the household

\item {} 
\textbf{\texttt{dt\_std}} (\emph{numpy.ndarray}) -- the mean standard deviation for the meal for each person in the household

\end{itemize}

\item[{Returns}] \leavevmode
the meals for each person in the household

\item[{Return type}] \leavevmode
list

\end{description}\end{quote}

\end{fulllineitems}

\index{set\_num\_steps() (params.Params method)}

\begin{fulllineitems}
\phantomsection\label{params:params.Params.set_num_steps}\pysiglinewithargsret{\bfcode{set\_num\_steps}}{}{}
This function calculates and sets the number of time steps the ABM will run.
\begin{quote}\begin{description}
\item[{Return type}] \leavevmode
None

\end{description}\end{quote}

\end{fulllineitems}

\index{tester() (params.Params method)}

\begin{fulllineitems}
\phantomsection\label{params:params.Params.tester}\pysiglinewithargsret{\bfcode{tester}}{}{}~
\begin{notice}{warning}{Warning:}
This function is just for testing. It checks to see whether the expected dinner time is before             the expected end time for work.
\end{notice}
\begin{quote}\begin{description}
\item[{Returns}] \leavevmode


\end{description}\end{quote}

\end{fulllineitems}

\index{toString() (params.Params method)}

\begin{fulllineitems}
\phantomsection\label{params:params.Params.toString}\pysiglinewithargsret{\bfcode{toString}}{}{}
This function represents the {\hyperref[params:params.Params]{\emph{\code{params.Params}}}} object as a string. For now, it prints         the tuple (start time, duration, end time) in hours{[}0, 24{]} for the following activities:
\begin{enumerate}
\item {} 
eat breakfast

\item {} 
commute to work

\item {} 
work

\item {} 
eat lunch

\item {} 
commute from work

\item {} 
eat dinner

\item {} 
sleep

\end{enumerate}

in order of start time. The commute activities only have duration information.
\begin{quote}\begin{description}
\item[{Returns}] \leavevmode
the parameter information

\end{description}\end{quote}

\end{fulllineitems}


\end{fulllineitems}



\section{person module}
\label{person::doc}\label{person:module-person}\label{person:person-module}\index{person (module)}
This module has code that governs information about the agent.

This module contains information about class {\hyperref[person:person.Person]{\emph{\code{person.Person}}}}.

\emph{Module author: Dr. Namdi Brandon}
\index{Person (class in person)}

\begin{fulllineitems}
\phantomsection\label{person:person.Person}\pysiglinewithargsret{\strong{class }\code{person.}\bfcode{Person}}{\emph{house}, \emph{clock}, \emph{schedule}}{}
Bases: \code{object}

This class contains all of the information relevant for a Person.

A person is parametrized by the following
\begin{itemize}
\item {} 
a place of residence

\item {} 
a biology

\item {} 
social behavior

\item {} 
a location

\item {} 
a history of activities and states

\item {} \begin{description}
\item[{Needs}] \leavevmode\begin{enumerate}
\item {} 
Hunger

\item {} 
Rest

\item {} 
Income

\item {} 
Travel

\item {} 
Interruption

\end{enumerate}

\end{description}

\end{itemize}
\begin{quote}\begin{description}
\item[{Parameters}] \leavevmode\begin{itemize}
\item {} 
\textbf{\texttt{house}} ({\hyperref[home:home.Home]{\emph{\emph{home.Home}}}}) -- the Home object the Person resides in. (will need to remove this)

\item {} 
\textbf{\texttt{clock}} ({\hyperref[temporal:temporal.Temporal]{\emph{\emph{temporal.Temporal}}}}) -- the time

\item {} 
\textbf{\texttt{schedule}} ({\hyperref[scheduler:scheduler.Scheduler]{\emph{\emph{scheduler.Scheduler}}}}) -- the schedule

\end{itemize}

\item[{Variables}] \leavevmode\begin{itemize}
\item {} 
\textbf{\texttt{'bio'}} ({\hyperref[bio:bio.Bio]{\emph{\emph{bio.Bio}}}}) -- the biological characteristics

\item {} 
\textbf{\texttt{clock}} ({\hyperref[temporal:temporal.Temporal]{\emph{\emph{temporal.Temporal}}}}) -- keeps track of the current time. It is linked to the Universe clock

\item {} 
\textbf{\texttt{hist\_state}} (\emph{numpy.ndarray}) -- the state history {[}int{]} for each time step

\item {} 
\textbf{\texttt{hist\_activity}} (\emph{numpy.ndarray}) -- the activity history {[}int{]} for each time step

\item {} 
\textbf{\texttt{'home'}} ({\hyperref[home:home.Home]{\emph{\emph{home.Home}}}}) -- this contains the place where the Person resides

\item {} 
\textbf{\texttt{id}} (\emph{int}) -- unique person identifier

\item {} 
\textbf{\texttt{'income'}} ({\hyperref[income:income.Income]{\emph{\emph{income.Income}}}}) -- the need that concerns itself with working/school

\item {} 
\textbf{\texttt{'interruption'}} ({\hyperref[interruption:interruption.Interruption]{\emph{\emph{interruption.Interruption}}}}) -- the need that concerns itself with interrupting an ongoing activity

\item {} 
\textbf{\texttt{'location'}} ({\hyperref[location:location.Location]{\emph{\emph{location.Location}}}}) -- the location data of a Person

\item {} 
\textbf{\texttt{needs}} (\emph{dict}) -- a dictionary of all of the  needs

\item {} 
\textbf{\texttt{'rest'}} ({\hyperref[rest:rest.Rest]{\emph{\emph{rest.Rest}}}}) -- the need that concerns itself with sleeping

\item {} 
\textbf{\texttt{socio}} ({\hyperref[social:social.Social]{\emph{\emph{social.Social}}}}) -- the social characteristics of a Person

\item {} 
\textbf{\texttt{'state'}} ({\hyperref[state:state.State]{\emph{\emph{state.State}}}}) -- information about a Person's state

\item {} 
\textbf{\texttt{'travel'}} ({\hyperref[travel:travel.Travel]{\emph{\emph{travel.Travel}}}}) -- the need that concerns itself with moving from one area to another

\item {} 
\textbf{\texttt{hist\_state}} -- the state of the person at each time step

\item {} 
\textbf{\texttt{hist\_activity}} -- the activity code of the person at each time step

\item {} 
\textbf{\texttt{hist\_local}} (\emph{numpy.ndarray}) -- the location code of the person at each time step

\item {} 
\textbf{\texttt{H}} (\emph{numpy.ndarray}) -- the satiation level for each need at each time step

\item {} 
\textbf{\texttt{need\_vector}} (\emph{numpy.ndarray}) -- the satiation level for each need at a given time step

\end{itemize}

\end{description}\end{quote}
\index{print\_basic\_info() (person.Person method)}

\begin{fulllineitems}
\phantomsection\label{person:person.Person.print_basic_info}\pysiglinewithargsret{\bfcode{print\_basic\_info}}{}{}
This function expresses basic information about the Person as a string.

This prints the following:
\begin{itemize}
\item {} 
person identifier

\item {} 
home identifier

\item {} 
age

\item {} 
gender

\end{itemize}
\begin{quote}\begin{description}
\item[{Returns}] \leavevmode
basic information about the Person

\item[{Return type}] \leavevmode
str

\end{description}\end{quote}

\end{fulllineitems}

\index{reset() (person.Person method)}

\begin{fulllineitems}
\phantomsection\label{person:person.Person.reset}\pysiglinewithargsret{\bfcode{reset}}{}{}
This function rests the person at the beginning of a simulation by doing the following:
\begin{enumerate}
\item {} 
reset the history

\item {} 
reset the state

\item {} 
reset the location

\item {} 
reset the needs

\end{enumerate}

\begin{notice}{note}{Note:}
the clock needs to be set to the beginning of simulation
\end{notice}
\begin{quote}\begin{description}
\item[{Returns}] \leavevmode
None

\end{description}\end{quote}

\end{fulllineitems}

\index{reset\_history() (person.Person method)}

\begin{fulllineitems}
\phantomsection\label{person:person.Person.reset_history}\pysiglinewithargsret{\bfcode{reset\_history}}{}{}
This function resets the variables:
\begin{enumerate}
\item {} 
history of the state

\item {} 
history of the activity

\item {} 
history of the location

\end{enumerate}
\begin{quote}\begin{description}
\item[{Returns}] \leavevmode
None

\end{description}\end{quote}

\end{fulllineitems}

\index{reset\_needs() (person.Person method)}

\begin{fulllineitems}
\phantomsection\label{person:person.Person.reset_needs}\pysiglinewithargsret{\bfcode{reset\_needs}}{}{}
This function resets the needs.
\begin{quote}\begin{description}
\item[{Returns}] \leavevmode
None

\end{description}\end{quote}

\end{fulllineitems}

\index{toString() (person.Person method)}

\begin{fulllineitems}
\phantomsection\label{person:person.Person.toString}\pysiglinewithargsret{\bfcode{toString}}{}{}
This function represents the Person object as a string.
\begin{quote}\begin{description}
\item[{Returns}] \leavevmode
information about the Person

\item[{Return type}] \leavevmode
str

\end{description}\end{quote}

\end{fulllineitems}

\index{update\_history() (person.Person method)}

\begin{fulllineitems}
\phantomsection\label{person:person.Person.update_history}\pysiglinewithargsret{\bfcode{update\_history}}{}{}
This function updates the history of the following values with their current values
\begin{itemize}
\item {} 
state history

\item {} 
location history

\item {} 
activity history

\item {} 
need (satiation) history

\end{itemize}
\begin{quote}\begin{description}
\item[{Returns}] \leavevmode


\end{description}\end{quote}

\end{fulllineitems}

\index{update\_history\_activity() (person.Person method)}

\begin{fulllineitems}
\phantomsection\label{person:person.Person.update_history_activity}\pysiglinewithargsret{\bfcode{update\_history\_activity}}{}{}
This function updates the activity history with the current values.
\begin{quote}\begin{description}
\item[{Returns}] \leavevmode
None

\end{description}\end{quote}

\end{fulllineitems}

\index{update\_history\_needs() (person.Person method)}

\begin{fulllineitems}
\phantomsection\label{person:person.Person.update_history_needs}\pysiglinewithargsret{\bfcode{update\_history\_needs}}{}{}
This function updates the needs (satiation) history with the current values.
\begin{quote}\begin{description}
\item[{Returns}] \leavevmode
None

\end{description}\end{quote}

\end{fulllineitems}


\end{fulllineitems}



\section{rest module}
\label{rest::doc}\label{rest:module-rest}\label{rest:rest-module}\index{rest (module)}
This file contains information about the need dealing with resting.

This module contains class {\hyperref[rest:rest.Rest]{\emph{\code{rest.Rest}}}}.

\emph{Module author: Dr. Namdi Brandon}
\index{Rest (class in rest)}

\begin{fulllineitems}
\phantomsection\label{rest:rest.Rest}\pysiglinewithargsret{\strong{class }\code{rest.}\bfcode{Rest}}{\emph{clock}, \emph{num\_sample\_points}}{}
Bases: {\hyperref[need:need.Need]{\emph{\code{need.Need}}}}

This class contains relevant information about the rest need.
\begin{quote}\begin{description}
\item[{Parameters}] \leavevmode\begin{itemize}
\item {} 
\textbf{\texttt{clock}} ({\hyperref[temporal:temporal.Temporal]{\emph{\emph{temporal.Temporal}}}}) -- this keeps track of the current time. It is linked to the Universe clock.

\item {} 
\textbf{\texttt{num\_sample\_points}} (\emph{int}) -- the number of temporal nodes in the simulation

\end{itemize}

\end{description}\end{quote}
\index{decay() (rest.Rest method)}

\begin{fulllineitems}
\phantomsection\label{rest:rest.Rest.decay}\pysiglinewithargsret{\bfcode{decay}}{\emph{status}}{}~
\begin{notice}{warning}{Warning:}
This function is old and antiquated.
\end{notice}

This function decays the Rest magnitude

Rest only decays if the person is \textbf{not} asleep.  The decay in sleep
\begin{gather}
\begin{split}\delta &= m_{decay} \Delta{t} \\
n(t + \Delta{t}) &= n(t) + \delta\end{split}\notag
\end{gather}\begin{description}
\item[{where}] \leavevmode\begin{itemize}
\item {} 
\(m_{decay}\) is the decay rate

\item {} 
\(\Delta{t}\) is the duration of time in 1 time step of simulation {[}minutes{]}

\item {} 
\(\delta\) is the amount of decay of rest

\item {} 
\(n(t)\) is the satiation at time t

\end{itemize}

\end{description}
\begin{quote}\begin{description}
\item[{Parameters}] \leavevmode
\textbf{\texttt{status}} (\emph{int}) -- the current state of a person

\item[{Returns}] \leavevmode
None

\end{description}\end{quote}

\end{fulllineitems}

\index{decay\_new() (rest.Rest method)}

\begin{fulllineitems}
\phantomsection\label{rest:rest.Rest.decay_new}\pysiglinewithargsret{\bfcode{decay\_new}}{\emph{status}, \emph{dt}}{}
This function decays the Rest magnitude

Rest only decays if the person is NOT asleep.  The decay in sleep
\begin{gather}
\begin{split}\delta &= m_{decay} \Delta{t} \\
n(t + \Delta{t}) &= n(t) + \delta\end{split}\notag
\end{gather}\begin{description}
\item[{where}] \leavevmode\begin{itemize}
\item {} 
\(m_{decay}\) is the decay rate

\item {} 
\(\Delta{t}\) is the duration of time in 1 time step of simulation {[}minutes{]}

\item {} 
\(\delta\) is the amount of decay of Rest

\item {} 
\(n(t)\) is the satiation at time t

\end{itemize}

\end{description}
\begin{quote}\begin{description}
\item[{Parameters}] \leavevmode\begin{itemize}
\item {} 
\textbf{\texttt{status}} (\emph{int}) -- the current state of a person

\item {} 
\textbf{\texttt{dt}} (\emph{int}) -- the duration of time {[}minutes{]} used to decay the need

\end{itemize}

\item[{Returns}] \leavevmode
None

\end{description}\end{quote}

\end{fulllineitems}

\index{initialize() (rest.Rest method)}

\begin{fulllineitems}
\phantomsection\label{rest:rest.Rest.initialize}\pysiglinewithargsret{\bfcode{initialize}}{\emph{p}}{}
The purpose of this code is to help initialize the Rest need and whatever activity that goes with         it, depending on any time the simulation begins

\begin{notice}{note}{Note:}
This code is a work in progress
\end{notice}
\begin{enumerate}
\item {} 
update the sleep start and end time

\item {} 
find out if the person should be asleep

\item {} \begin{description}
\item[{if the Person is asleep, }] \leavevmode\begin{itemize}
\item {} 
sets the appropriate duration of sleep left to do

\item {} 
sets the rest magnitude to threshold

\item {} 
sets the rest recharge rate

\item {} 
sets the schedule to trigger when when the person is scheduled to wake up

\end{itemize}

\end{description}

\item {} \begin{description}
\item[{if the Person is not asleep, }] \leavevmode\begin{itemize}
\item {} 
sets the decay rate

\item {} 
set the magnitude

\item {} 
sets the schedule to trigger when when the person is scheduled to start sleeping

\end{itemize}

\end{description}

\item {} 
update the schedule for the rest need

\end{enumerate}
\begin{quote}\begin{description}
\item[{Parameters}] \leavevmode
\textbf{\texttt{p}} ({\hyperref[person:person.Person]{\emph{\emph{person.Person}}}}) -- the person of interest

\item[{Returns}] \leavevmode
None

\end{description}\end{quote}

\end{fulllineitems}

\index{is\_workday() (rest.Rest method)}

\begin{fulllineitems}
\phantomsection\label{rest:rest.Rest.is_workday}\pysiglinewithargsret{\bfcode{is\_workday}}{\emph{p}}{}
This function indicates whether or not the sleep event resembles that from a person sleeping for         a workday.
\begin{quote}\begin{description}
\item[{Parameters}] \leavevmode
\textbf{\texttt{socio}} ({\hyperref[social:social.Social]{\emph{\emph{social.Social}}}}) -- the social characteristics of the person of interest

\item[{Returns}] \leavevmode
True, if the sleep event resembles a workday. False, otherwise.

\end{description}\end{quote}

\end{fulllineitems}

\index{perceive() (rest.Rest method)}

\begin{fulllineitems}
\phantomsection\label{rest:rest.Rest.perceive}\pysiglinewithargsret{\bfcode{perceive}}{\emph{future\_clock}}{}
This functions gives the updated rest magnitude if sleep is done from now until a later time         corresponding to clock.
\begin{gather}
\begin{split}\delta = m_{suggested}\Delta{t}\end{split}\notag
\end{gather}\begin{description}
\item[{where}] \leavevmode\begin{itemize}
\item {} 
\(m_{suggested}\) is the suggested recharge rate

\item {} \begin{description}
\item[{\(\Delta{t}\) is the duration of time from now until the future time}] \leavevmode
given by future\_clock

\end{description}

\end{itemize}

\end{description}
\begin{quote}\begin{description}
\item[{Parameters}] \leavevmode
\textbf{\texttt{future\_clock}} ({\hyperref[temporal:temporal.Temporal]{\emph{\emph{temporal.Temporal}}}}) -- a clock corresponding to a future time

\item[{Returns}] \leavevmode
the perceived rest level

\item[{Return type}] \leavevmode
float

\end{description}\end{quote}

\end{fulllineitems}

\index{reset() (rest.Rest method)}

\begin{fulllineitems}
\phantomsection\label{rest:rest.Rest.reset}\pysiglinewithargsret{\bfcode{reset}}{}{}
This function resets the values in order for the need to be used in the next simulation
\begin{quote}\begin{description}
\item[{Returns}] \leavevmode
None

\end{description}\end{quote}

\end{fulllineitems}

\index{set\_decay\_rate() (rest.Rest method)}

\begin{fulllineitems}
\phantomsection\label{rest:rest.Rest.set_decay_rate}\pysiglinewithargsret{\bfcode{set\_decay\_rate}}{\emph{dt}}{}
This function sets the decay rate.

The decay rate (\(m_{decay}\)) is assumed to be the slope of a linear function
\begin{description}
\item[{where}] \leavevmode\begin{itemize}
\item {} 
\(\Delta{t}\) is the duration of time expected to be awake

\item {} 
\(\lambda\) is the rest threshold

\end{itemize}

\end{description}
\begin{gather}
\begin{split}m_{decay} = \frac{-1 + \lambda}{\Delta{t}}\end{split}\notag
\end{gather}\begin{quote}\begin{description}
\item[{Parameters}] \leavevmode
\textbf{\texttt{dt}} (\emph{int}) -- the duration of sleep {[}minutes{]}

\item[{Returns}] \leavevmode
None

\end{description}\end{quote}

\end{fulllineitems}

\index{set\_recharge\_rate() (rest.Rest method)}

\begin{fulllineitems}
\phantomsection\label{rest:rest.Rest.set_recharge_rate}\pysiglinewithargsret{\bfcode{set\_recharge\_rate}}{\emph{dt}}{}
This function sets the recharge rate.

The recharge rate (\(m_{recharge}\)) is assumed to be the slope of a linear function
\begin{description}
\item[{where}] \leavevmode\begin{itemize}
\item {} 
\(\Delta{t}\) is the duration of sleep

\item {} 
\(\lambda\) is the rest need threshold

\end{itemize}

\end{description}
\begin{gather}
\begin{split}m_{recharge} = \frac{1 - \lambda}{ \Delta{t} }\end{split}\notag
\end{gather}\begin{quote}\begin{description}
\item[{Parameters}] \leavevmode
\textbf{\texttt{dt}} (\emph{int}) -- the duration of sleep after rounding {[}minutes{]}

\item[{Returns}] \leavevmode
None

\end{description}\end{quote}

\end{fulllineitems}

\index{set\_suggested\_recharge\_rate() (rest.Rest method)}

\begin{fulllineitems}
\phantomsection\label{rest:rest.Rest.set_suggested_recharge_rate}\pysiglinewithargsret{\bfcode{set\_suggested\_recharge\_rate}}{\emph{dt}}{}
This function sets the ``suggested'' recharge rate. That is, the rate of recharge assuming exact         arithmetic (there is no rounding in time, say to the nearest minute)
\begin{description}
\item[{where}] \leavevmode\begin{itemize}
\item {} 
\(\Delta{t}\) is the duration of sleep

\item {} 
\(\lambda\) is the rest need threshold

\end{itemize}

\end{description}
\begin{gather}
\begin{split}m_{suggested} = \frac{ 1 - \lambda }{ \Delta{t} }\end{split}\notag
\end{gather}\begin{quote}\begin{description}
\item[{Parameters}] \leavevmode
\textbf{\texttt{dt}} (\emph{int}) -- the duration of sleep {[}minutes{]}

\item[{Returns}] \leavevmode
None

\end{description}\end{quote}

\end{fulllineitems}

\index{should\_be\_asleep() (rest.Rest method)}

\begin{fulllineitems}
\phantomsection\label{rest:rest.Rest.should_be_asleep}\pysiglinewithargsret{\bfcode{should\_be\_asleep}}{\emph{t\_start}, \emph{t\_end}}{}
This function finds out if the person should be asleep for the initialization of the ABM module
\begin{quote}\begin{description}
\item[{Parameters}] \leavevmode\begin{itemize}
\item {} 
\textbf{\texttt{t\_start}} (\emph{int}) -- start time of sleep {[}minutes, time of day{]}

\item {} 
\textbf{\texttt{t\_end}} (\emph{int}) -- end time of sleep {[}minutes, time of day{]}

\end{itemize}

\item[{Returns}] \leavevmode
a flag indicating whether a Person should be asleep (if True) or awake (if False)

\item[{Return type}] \leavevmode
bool

\end{description}\end{quote}

\end{fulllineitems}

\index{toString() (rest.Rest method)}

\begin{fulllineitems}
\phantomsection\label{rest:rest.Rest.toString}\pysiglinewithargsret{\bfcode{toString}}{}{}
Represent the Rest object as a string
\begin{quote}\begin{description}
\item[{Returns}] \leavevmode
the representation of the Rest object

\item[{Return type}] \leavevmode
str

\end{description}\end{quote}

\end{fulllineitems}


\end{fulllineitems}



\section{scheduler module}
\label{scheduler::doc}\label{scheduler:module-scheduler}\label{scheduler:scheduler-module}\index{scheduler (module)}
This module contains code that is is responsible for controlling the scheduler for the simulation. Note that the simulation does \textbf{not} run continuously in from one adjacent time step to the next. Instead the simulation jumps forward in time (i.e. move across multiple time steps in time), stopping only at time steps in which an action could occur. The ability to jump forward in time is controlled by the scheduler.

The scheduler will trigger the simulation to stop skipping time steps for the following reasons:
\begin{enumerate}
\item {} 
an activity should start

\item {} 
an activity should end

\item {} 
a need is under threshold

\end{enumerate}

This module contains class {\hyperref[scheduler:scheduler.Scheduler]{\emph{\code{scheduler.Scheduler}}}}.

\emph{Module author: Dr. Namdi Brandon}
\index{Scheduler (class in scheduler)}

\begin{fulllineitems}
\phantomsection\label{scheduler:scheduler.Scheduler}\pysiglinewithargsret{\strong{class }\code{scheduler.}\bfcode{Scheduler}}{\emph{clock}, \emph{num\_people}}{}
Bases: \code{object}

This class contains the code for the scheduler. The scheduler is in charge of jumping forward in time and     stopping at only potentially relevant time steps. The scheduler keeps track of the needs for every person in     in the household and stops at time steps where any person should have an action / need that needs to be     addressed.
\begin{quote}\begin{description}
\item[{Parameters}] \leavevmode\begin{itemize}
\item {} 
\textbf{\texttt{clock}} ({\hyperref[temporal:temporal.Temporal]{\emph{\emph{temporal.Temporal}}}}) -- the time

\item {} 
\textbf{\texttt{num\_people}} (\emph{int}) -- the number of people in the household

\end{itemize}

\item[{Variables}] \leavevmode\begin{itemize}
\item {} 
\textbf{\texttt{clock}} ({\hyperref[temporal:temporal.Temporal]{\emph{\emph{temporal.Temporal}}}}) -- the time

\item {} 
\textbf{\texttt{A}} (\emph{numpy.ndarray}) -- the schedule matrix of dimension (number of people x number of needs). This matrix     contains the times {[}minutes, universal time{]} that the simulation should not skip over

\item {} 
\textbf{\texttt{dt}} (\emph{int}) -- the duration of time between events

\item {} 
\textbf{\texttt{t\_old}} (\emph{int}) -- the time {[}minutes, universal time{]} of the prior event

\end{itemize}

\end{description}\end{quote}
\index{get\_next\_event\_time() (scheduler.Scheduler method)}

\begin{fulllineitems}
\phantomsection\label{scheduler:scheduler.Scheduler.get_next_event_time}\pysiglinewithargsret{\bfcode{get\_next\_event\_time}}{}{}
This function searches the schedule matrix and finds the next time that that model should handle.

\begin{notice}{note}{Note:}
This function is only capable of handling \textbf{single-occupancy} households.
\end{notice}
\begin{quote}\begin{description}
\item[{Returns}] \leavevmode
the next time {[}minutes, time of day{]} that the model should address

\item[{Return type}] \leavevmode
int

\end{description}\end{quote}

\end{fulllineitems}

\index{toString() (scheduler.Scheduler method)}

\begin{fulllineitems}
\phantomsection\label{scheduler:scheduler.Scheduler.toString}\pysiglinewithargsret{\bfcode{toString}}{}{}
This function presents the Scheduler object as a string.
\begin{quote}\begin{description}
\item[{Returns}] \leavevmode
a string representation of the object

\end{description}\end{quote}

\end{fulllineitems}

\index{update() (scheduler.Scheduler method)}

\begin{fulllineitems}
\phantomsection\label{scheduler:scheduler.Scheduler.update}\pysiglinewithargsret{\bfcode{update}}{\emph{id\_person}, \emph{id\_need}, \emph{dt}}{}
This function updates the schedule matrix for a given person and need with the duration for the next event, 
for the respective person-need combination.
\begin{quote}\begin{description}
\item[{Parameters}] \leavevmode\begin{itemize}
\item {} 
\textbf{\texttt{id\_person}} (\emph{int}) -- the person identifier

\item {} 
\textbf{\texttt{id\_need}} (\emph{int}) -- the need identifier

\item {} 
\textbf{\texttt{dt}} (\emph{int}) -- the duration to the next event

\end{itemize}

\item[{Returns}] \leavevmode
None

\end{description}\end{quote}

\end{fulllineitems}


\end{fulllineitems}



\section{sleep module}
\label{sleep::doc}\label{sleep:module-sleep}\label{sleep:sleep-module}\index{sleep (module)}
This module contains information about the activity dealing with sleeping. This class is {\hyperref[activity:activity.Activity]{\emph{\code{activity.Activity}}}} that gives a {\hyperref[person:person.Person]{\emph{\code{person.Person}}}} the ability to eat and satisfy the need {\hyperref[rest:rest.Rest]{\emph{\code{rest.Rest}}}}.

This file contains class {\hyperref[sleep:sleep.Sleep]{\emph{\code{sleep.Sleep}}}}.

\emph{Module author: Dr. Namdi Brandon}
\index{Sleep (class in sleep)}

\begin{fulllineitems}
\phantomsection\label{sleep:sleep.Sleep}\pysigline{\strong{class }\code{sleep.}\bfcode{Sleep}}
Bases: {\hyperref[activity:activity.Activity]{\emph{\code{activity.Activity}}}}

This class is responsible for the act of sleeping, which satisfies the need {\hyperref[rest:rest.Rest]{\emph{\code{rest.Rest}}}}.
\index{advertise() (sleep.Sleep method)}

\begin{fulllineitems}
\phantomsection\label{sleep:sleep.Sleep.advertise}\pysiglinewithargsret{\bfcode{advertise}}{\emph{p}}{}
This function calculates the score of an activity advertisement to a Person
\begin{quote}\begin{description}
\item[{Parameters}] \leavevmode
\textbf{\texttt{p}} ({\hyperref[person:person.Person]{\emph{\emph{person.Person}}}}) -- the person being advertised to

\item[{Returns}] \leavevmode
the value of the advertisement

\item[{Return type}] \leavevmode
float

\end{description}\end{quote}

\end{fulllineitems}

\index{end() (sleep.Sleep method)}

\begin{fulllineitems}
\phantomsection\label{sleep:sleep.Sleep.end}\pysiglinewithargsret{\bfcode{end}}{\emph{p}}{}
This handles the end of the sleep activity.
\begin{quote}\begin{description}
\item[{Parameters}] \leavevmode
\textbf{\texttt{p}} ({\hyperref[person:person.Person]{\emph{\emph{person.Person}}}}) -- the person of interest

\item[{Returns}] \leavevmode
None

\end{description}\end{quote}

\end{fulllineitems}

\index{end\_sleep() (sleep.Sleep method)}

\begin{fulllineitems}
\phantomsection\label{sleep:sleep.Sleep.end_sleep}\pysiglinewithargsret{\bfcode{end\_sleep}}{\emph{p}}{}
This function addresses logistics with a person waking up from sleep
\begin{enumerate}
\item {} 
free the asset from use

\item {} 
set the state of the person to idle (\code{state.IDLE})

\item {} 
update the satiation

\item {} 
update the start time and end time

\item {} 
set the decay rate

\item {} 
update the schedule for the rest need

\end{enumerate}
\begin{quote}\begin{description}
\item[{Parameters}] \leavevmode
\textbf{\texttt{p}} ({\hyperref[person:person.Person]{\emph{\emph{person.Person}}}}) -- the person of interest

\item[{Returns}] \leavevmode
None

\end{description}\end{quote}

\end{fulllineitems}

\index{is\_workday() (sleep.Sleep method)}

\begin{fulllineitems}
\phantomsection\label{sleep:sleep.Sleep.is_workday}\pysiglinewithargsret{\bfcode{is\_workday}}{\emph{p}}{}
This function indicates whether or not the sleep event resembles that from a person sleeping for         a workday.
\begin{quote}\begin{description}
\item[{Parameters}] \leavevmode
\textbf{\texttt{p}} ({\hyperref[person:person.Person]{\emph{\emph{person.Person}}}}) -- the person of interest

\item[{Returns}] \leavevmode
True, if the sleep event resembles a workday. False, otherwise.

\end{description}\end{quote}

\end{fulllineitems}

\index{set\_end\_time() (sleep.Sleep method)}

\begin{fulllineitems}
\phantomsection\label{sleep:sleep.Sleep.set_end_time}\pysiglinewithargsret{\bfcode{set\_end\_time}}{\emph{p}}{}
This function returns the end time of sleeping

The end time \(t_{end}\) is set as follows
\[
\begin{cases}
    \Delta{t} &= \frac{ 1 - n(t) }{ m_{suggested} } \\
    t_{end} &= t + \Delta{t}
\end{cases}
\]\begin{description}
\item[{where}] \leavevmode\begin{itemize}
\item {} 
\(\Delta{t}\) is the duration of sleep

\item {} 
\(m_{suggested}\) is the suggested recharge rate

\item {} 
\(n(t)\) is the magnitude of sleep at time t

\end{itemize}

\end{description}
\begin{quote}\begin{description}
\item[{Parameters}] \leavevmode
\textbf{\texttt{p}} ({\hyperref[person:person.Person]{\emph{\emph{person.Person}}}}) -- the person of interest

\item[{Return t\_end}] \leavevmode
the end time of the sleep event {[}minutes, universal time{]}

\item[{Return type}] \leavevmode
int

\end{description}\end{quote}

\end{fulllineitems}

\index{start() (sleep.Sleep method)}

\begin{fulllineitems}
\phantomsection\label{sleep:sleep.Sleep.start}\pysiglinewithargsret{\bfcode{start}}{\emph{p}}{}
This handles the start of the sleep activity.
\begin{quote}\begin{description}
\item[{Parameters}] \leavevmode
\textbf{\texttt{p}} ({\hyperref[person:person.Person]{\emph{\emph{person.Person}}}}) -- the person of interest

\item[{Returns}] \leavevmode
None

\end{description}\end{quote}

\end{fulllineitems}

\index{start\_sleep() (sleep.Sleep method)}

\begin{fulllineitems}
\phantomsection\label{sleep:sleep.Sleep.start_sleep}\pysiglinewithargsret{\bfcode{start\_sleep}}{\emph{p}}{}
This handles what happens when a Person goes to sleep.
\begin{enumerate}
\item {} 
The asset's status is updated.

\item {} 
The Person's state is set to the sleep state (\code{state.SLEEP})

\item {} 
The end time is calculated

\item {} 
The recharge rate is set (according to whether or not it is a workday / non-workday)

\end{enumerate}
\begin{quote}\begin{description}
\item[{Parameters}] \leavevmode
\textbf{\texttt{p}} ({\hyperref[person:person.Person]{\emph{\emph{person.Person}}}}) -- the person of interest

\item[{Returns}] \leavevmode
None

\end{description}\end{quote}

\end{fulllineitems}

\index{toString() (sleep.Sleep method)}

\begin{fulllineitems}
\phantomsection\label{sleep:sleep.Sleep.toString}\pysiglinewithargsret{\bfcode{toString}}{}{}
This function represents the Sleep object as a string
\begin{quote}\begin{description}
\item[{Return msg}] \leavevmode
the representation of the Sleep object

\item[{Return type}] \leavevmode
str

\end{description}\end{quote}

\end{fulllineitems}


\end{fulllineitems}



\section{social module}
\label{social::doc}\label{social:module-social}\label{social:social-module}\index{social (module)}
This module contains code that governs the social behavior/ characteristics relevant to a Person.

This module contains class {\hyperref[social:social.Social]{\emph{\code{social.Social}}}}.

\emph{Module author: Dr. Namdi Brandon}
\index{Social (class in social)}

\begin{fulllineitems}
\phantomsection\label{social:social.Social}\pysiglinewithargsret{\strong{class }\code{social.}\bfcode{Social}}{\emph{age}, \emph{num\_meals=3}}{}
Bases: \code{object}

This class contains all of the relevant information governing the Person's     social behavior.

\begin{notice}{note}{Note:}
The current version of the ABM does not have any ``alarm'' functionality / capability. The remnants of any         code that governs the use of an alarm  will be removed in future updates.
\end{notice}
\begin{quote}\begin{description}
\item[{Parameters}] \leavevmode\begin{itemize}
\item {} 
\textbf{\texttt{age}} (\emph{int}) -- the age of the Person {[}years{]}

\item {} 
\textbf{\texttt{num\_meals}} (\emph{int}) -- the number of meals per day

\end{itemize}

\item[{Variables}] \leavevmode\begin{itemize}
\item {} 
\textbf{\texttt{is\_child}} (\emph{bool}) -- this flag is True if the Person is a child, False otherwise

\item {} 
\textbf{\texttt{job}} ({\hyperref[occupation:occupation.Occupation]{\emph{\emph{occupation.Occupation}}}}) -- the information pertaining the the job

\item {} 
\textbf{\texttt{num\_meals}} (\emph{int}) -- the number of meals per day a person will eat

\item {} 
\textbf{\texttt{meals}} (\emph{list}) -- a list of the meals that a person eats ({\hyperref[meal:meal.Meal]{\emph{\code{meal.Meal}}}})

\item {} 
\textbf{\texttt{current\_meal}} ({\hyperref[meal:meal.Meal]{\emph{\emph{meal.Meal}}}}) -- the meal that is currently being eaten \textbf{or} if the Person is not eating a meal,     it is the upcoming meal

\item {} 
\textbf{\texttt{next\_meal}} ({\hyperref[meal:meal.Meal]{\emph{\emph{meal.Meal}}}}) -- the meal that is after the meal indicated by \code{current\_meal}

\item {} 
\textbf{\texttt{uses\_alarm}} (\emph{bool}) -- indicates whether or not a person uses an alarm to wake up

\item {} 
\textbf{\texttt{is\_alarm\_set}} (\emph{bool}) -- indicates whether or not an alarm is set for the current day

\item {} 
\textbf{\texttt{t\_alarm}} (\emph{int}) -- the time an alarm is supposed to go off {[}minutes, time of day{]}

\end{itemize}

\end{description}\end{quote}
\index{duration\_to\_next\_commute\_event() (social.Social method)}

\begin{fulllineitems}
\phantomsection\label{social:social.Social.duration_to_next_commute_event}\pysiglinewithargsret{\bfcode{duration\_to\_next\_commute\_event}}{\emph{clock}}{}
This function is called in in order to calculate the amount of time until the next commute event by         doing the following.
\begin{enumerate}
\item {} 
If the agent is unemployed, return infinity

\item {} 
If the time indicates that the agent should be currently working, set the duration to be the         length of time remaining at work

\item {} 
If the time indicates that the agent should be currently commuting to work, set the duration to be         the duration until the commute to work should start

\item {} 
If the time indicates that the agent should be currently commuting from work, set the duration to be         the amount of time until the commute from work should end

\item {} 
Else, calculate the amount of time until the next commute to work event

\end{enumerate}

\begin{notice}{note}{Note:}
The only reason this code is place here is because the work activity and the commute activity use it.
\end{notice}
\begin{quote}\begin{description}
\item[{Parameters}] \leavevmode
\textbf{\texttt{clock}} ({\hyperref[temporal:temporal.Temporal]{\emph{\emph{temporal.Temporal}}}}) -- the current time

\item[{Returns}] \leavevmode
the duration in time {[}minutes{]} until the next commute event

\item[{Return type}] \leavevmode
int

\end{description}\end{quote}

\end{fulllineitems}

\index{duration\_to\_next\_meal() (social.Social method)}

\begin{fulllineitems}
\phantomsection\label{social:social.Social.duration_to_next_meal}\pysiglinewithargsret{\bfcode{duration\_to\_next\_meal}}{\emph{t\_univ}}{}
This function calculates the amount of time until the next meal.
\begin{quote}\begin{description}
\item[{Parameters}] \leavevmode
\textbf{\texttt{t\_univ}} (\emph{int}) -- the current time {[}minutes, universal time{]}

\item[{Returns}] \leavevmode
the duration to the next meal {[}minutes{]}

\item[{Return type}] \leavevmode
int

\item[{Returns}] \leavevmode
the scheduled next meal

\item[{Return type}] \leavevmode
{\hyperref[meal:meal.Meal]{\emph{meal.Meal}}}

\end{description}\end{quote}

\end{fulllineitems}

\index{duration\_to\_work\_event() (social.Social method)}

\begin{fulllineitems}
\phantomsection\label{social:social.Social.duration_to_work_event}\pysiglinewithargsret{\bfcode{duration\_to\_work\_event}}{\emph{clock}}{}
This function is called in in order to calculate the amount of time until the next work event.
\begin{enumerate}
\item {} 
If the person is employed, the duration to the next meal is set to infinity

\item {} \begin{description}
\item[{If the current time is a workday before the time work starts,}] \leavevmode\begin{itemize}
\item {} 
set the duration to the amount of time until the start of work

\end{itemize}

\end{description}

\item {} \begin{description}
\item[{Else,}] \leavevmode\begin{itemize}
\item {} 
set the duration until the next work event

\end{itemize}

\end{description}

\end{enumerate}

\begin{notice}{note}{Note:}
The only reason this code is place here is because the work activity and the commute activity use it.
\end{notice}
\begin{quote}\begin{description}
\item[{Parameters}] \leavevmode
\textbf{\texttt{clock}} ({\hyperref[temporal:temporal.Temporal]{\emph{\emph{temporal.Temporal}}}}) -- the current time

\item[{Returns}] \leavevmode
the duration {[}minutes{]} until the next minutes

\item[{Return type}] \leavevmode
int

\end{description}\end{quote}

\end{fulllineitems}

\index{get\_current\_meal() (social.Social method)}

\begin{fulllineitems}
\phantomsection\label{social:social.Social.get_current_meal}\pysiglinewithargsret{\bfcode{get\_current\_meal}}{\emph{time\_of\_day}}{}
This function gets the closest meal to the time of day.
\begin{quote}\begin{description}
\item[{Parameters}] \leavevmode
\textbf{\texttt{time\_of\_day}} (\emph{int}) -- the time of day

\item[{Returns}] \leavevmode
return the meal

\item[{Return type}] \leavevmode
{\hyperref[meal:meal.Meal]{\emph{meal.Meal}}}

\end{description}\end{quote}

\end{fulllineitems}

\index{get\_meal() (social.Social method)}

\begin{fulllineitems}
\phantomsection\label{social:social.Social.get_meal}\pysiglinewithargsret{\bfcode{get\_meal}}{\emph{id\_meal}}{}
Get the specific meal given by a meal identifier.
\begin{quote}\begin{description}
\item[{Parameters}] \leavevmode
\textbf{\texttt{id\_meal}} (\emph{int}) -- the meal identifier

\item[{Returns}] \leavevmode
the meal given by the id

\item[{Return type}] \leavevmode
{\hyperref[meal:meal.Meal]{\emph{meal.Meal}}}

\end{description}\end{quote}

\end{fulllineitems}

\index{get\_next\_meal() (social.Social method)}

\begin{fulllineitems}
\phantomsection\label{social:social.Social.get_next_meal}\pysiglinewithargsret{\bfcode{get\_next\_meal}}{\emph{clock}}{}
This function gets the next meal. The meal must occur after the current time.
\begin{quote}\begin{description}
\item[{Parameters}] \leavevmode
\textbf{\texttt{clock}} ({\hyperref[temporal:temporal.Temporal]{\emph{\emph{temporal.Temporal}}}}) -- the current time

\item[{Return the\_meal}] \leavevmode
the next meal

\item[{Return type}] \leavevmode
{\hyperref[meal:meal.Meal]{\emph{meal.Meal}}}

\end{description}\end{quote}

\end{fulllineitems}

\index{print\_child\_status() (social.Social method)}

\begin{fulllineitems}
\phantomsection\label{social:social.Social.print_child_status}\pysiglinewithargsret{\bfcode{print\_child\_status}}{}{}
This function represents the child status as a string.
\begin{quote}\begin{description}
\item[{Return msg}] \leavevmode
the child/ adult status

\item[{Return type}] \leavevmode
str

\end{description}\end{quote}

\end{fulllineitems}

\index{set\_child\_flag() (social.Social method)}

\begin{fulllineitems}
\phantomsection\label{social:social.Social.set_child_flag}\pysiglinewithargsret{\bfcode{set\_child\_flag}}{\emph{age}}{}
Sets the flag indicating whether a person is a child.
\begin{quote}\begin{description}
\item[{Parameters}] \leavevmode
\textbf{\texttt{age}} (\emph{int}) -- the age of the person {[}years{]}

\item[{Returns}] \leavevmode
None

\end{description}\end{quote}

\end{fulllineitems}

\index{set\_job() (social.Social method)}

\begin{fulllineitems}
\phantomsection\label{social:social.Social.set_job}\pysiglinewithargsret{\bfcode{set\_job}}{\emph{job\_id}, \emph{dt=0}}{}
This function sets the job and the alarm time (if used) that corresponds to the job. The alarm  is set,         if a person is using the alarm.
\begin{quote}\begin{description}
\item[{Parameters}] \leavevmode\begin{itemize}
\item {} 
\textbf{\texttt{job\_id}} (\emph{int}) -- job identifier

\item {} 
\textbf{\texttt{dt}} (\emph{int}) -- the amount of time before the job start.

\end{itemize}

\item[{Returns}] \leavevmode
None

\end{description}\end{quote}

\end{fulllineitems}

\index{set\_work\_alarm() (social.Social method)}

\begin{fulllineitems}
\phantomsection\label{social:social.Social.set_work_alarm}\pysiglinewithargsret{\bfcode{set\_work\_alarm}}{\emph{dt=0}}{}
This sets the alarm time due to work

If a person uses an alarm, the alarm is set to be ``dt'' minutes before work time
\begin{quote}\begin{description}
\item[{Parameters}] \leavevmode
\textbf{\texttt{dt}} (\emph{int}) -- the amount of time to wake up before the work event {[}minutes{]}

\item[{Returns}] \leavevmode
None

\end{description}\end{quote}

\end{fulllineitems}

\index{test\_func() (social.Social method)}

\begin{fulllineitems}
\phantomsection\label{social:social.Social.test_func}\pysiglinewithargsret{\bfcode{test\_func}}{\emph{time\_of\_day}, \emph{the\_meal}}{}
This is used for testing.

\begin{notice}{note}{Note:}
This function has no real purpose and will be deleted in future versions.
\end{notice}
\begin{quote}\begin{description}
\item[{Parameters}] \leavevmode\begin{itemize}
\item {} 
\textbf{\texttt{time\_of\_day}} (\emph{int}) -- the time of day in minutes

\item {} 
\textbf{\texttt{the\_meal}} ({\hyperref[meal:meal.Meal]{\emph{\emph{meal.Meal}}}}) -- a meal object

\end{itemize}

\item[{Returns}] \leavevmode
None

\end{description}\end{quote}

\end{fulllineitems}

\index{toString() (social.Social method)}

\begin{fulllineitems}
\phantomsection\label{social:social.Social.toString}\pysiglinewithargsret{\bfcode{toString}}{}{}
Represents the Social object as a string.
\begin{quote}\begin{description}
\item[{Returns}] \leavevmode
the representation of the Social object

\item[{Return type}] \leavevmode
str

\end{description}\end{quote}

\end{fulllineitems}


\end{fulllineitems}



\section{state module}
\label{state::doc}\label{state:module-state}\label{state:state-module}\index{state (module)}
This module contains code that governs information relevant to a Person's state.

This module contains class {\hyperref[state:state.State]{\emph{\code{state.State}}}}.
\index{State (class in state)}

\begin{fulllineitems}
\phantomsection\label{state:state.State}\pysiglinewithargsret{\strong{class }\code{state.}\bfcode{State}}{\emph{status=0}}{}
Bases: \code{object}

This class contains information relevant to a Person state
\begin{quote}\begin{description}
\item[{Parameters}] \leavevmode
\textbf{\texttt{status}} (\emph{int}) -- the status of the Person

\item[{Variables}] \leavevmode\begin{itemize}
\item {} 
\textbf{\texttt{'activity'}} ({\hyperref[activity:activity.Activity]{\emph{\emph{activity.Activity}}}}) -- the particular Activity of the Asset

\item {} 
\textbf{\texttt{arg\_start}} (\emph{list}) -- the list of arguments for the start() function

\item {} 
\textbf{\texttt{arg\_end}} (\emph{list}) -- the list of arguments for the end() function

\item {} 
\textbf{\texttt{'asset'}} ({\hyperref[asset:asset.Asset]{\emph{\emph{asset.Asset}}}}) -- the Asset that is being used

\item {} 
\textbf{\texttt{asset\_list}} (\emph{list}) -- 

\item {} 
\textbf{\texttt{is\_init}} (\emph{bool}) -- this is a flag indicating whether or not the Agent is in the initialization state.     This state only occurs during the first step of the simulation.

\item {} 
\textbf{\texttt{status}} (\emph{int}) -- the status of a Person

\item {} 
\textbf{\texttt{t\_end}} (\emph{int}) -- the end time of a state {[}minutes, universal time{]}

\item {} 
\textbf{\texttt{t\_start}} (\emph{int}) -- the start time of the current state {[}minutes, universal time{]}

\item {} 
\textbf{\texttt{round\_dt}} (\emph{int}) -- the amount of minutes {[}-1, 0, 1{]} to round an activity duration

\item {} 
\textbf{\texttt{dt\_frac}} (\emph{float}) -- the fraction of a minutes subtracted from rounding down from the true projected     activity duration

\item {} 
\textbf{\texttt{do\_interruption}} (\emph{bool}) -- a flag indicating whether the Person is interrupting an ongoing activity

\end{itemize}

\end{description}\end{quote}
\index{end\_activity() (state.State method)}

\begin{fulllineitems}
\phantomsection\label{state:state.State.end_activity}\pysiglinewithargsret{\bfcode{end\_activity}}{}{}
This function ends an activity.
\begin{quote}\begin{description}
\item[{Returns}] \leavevmode
None

\end{description}\end{quote}

\end{fulllineitems}

\index{halt\_activity() (state.State method)}

\begin{fulllineitems}
\phantomsection\label{state:state.State.halt_activity}\pysiglinewithargsret{\bfcode{halt\_activity}}{\emph{p}}{}
This function runs the halt activity. The function is used by interruptions         to stop an activity \textbf{immediately} without giving benefits to the need that the         halted activity addressed.
\begin{quote}\begin{description}
\item[{Parameters}] \leavevmode
\textbf{\texttt{p}} ({\hyperref[person:person.Person]{\emph{\emph{person.Person}}}}) -- the person of interest

\item[{Returns}] \leavevmode
None

\end{description}\end{quote}

\end{fulllineitems}

\index{print\_activity() (state.State method)}

\begin{fulllineitems}
\phantomsection\label{state:state.State.print_activity}\pysiglinewithargsret{\bfcode{print\_activity}}{}{}
The string representation of the activity. This function handles the         possibility of the activity being None.
\begin{quote}\begin{description}
\item[{Returns}] \leavevmode
the representation of the activity

\item[{Return type}] \leavevmode
str

\end{description}\end{quote}

\end{fulllineitems}

\index{print\_asset() (state.State method)}

\begin{fulllineitems}
\phantomsection\label{state:state.State.print_asset}\pysiglinewithargsret{\bfcode{print\_asset}}{}{}
This function represents the asset as a string. This function handles         the possibility of the asset being None.
\begin{quote}\begin{description}
\item[{Returns}] \leavevmode
the representation of the asset

\item[{Return type}] \leavevmode
str

\end{description}\end{quote}

\end{fulllineitems}

\index{print\_status() (state.State method)}

\begin{fulllineitems}
\phantomsection\label{state:state.State.print_status}\pysiglinewithargsret{\bfcode{print\_status}}{}{}
This function represents the status as a string.
\begin{quote}\begin{description}
\item[{Returns}] \leavevmode
the representation of the status

\item[{Return type}] \leavevmode
str

\end{description}\end{quote}

\end{fulllineitems}

\index{reset() (state.State method)}

\begin{fulllineitems}
\phantomsection\label{state:state.State.reset}\pysiglinewithargsret{\bfcode{reset}}{\emph{t\_univ}}{}
Reset the state object to the default behavior at the beginning of the simulation.
\begin{quote}\begin{description}
\item[{Parameters}] \leavevmode
\textbf{\texttt{t\_univ}} (\emph{int}) -- the time of the beginning of the simulation in universal time {[}seconds{]}

\item[{Returns}] \leavevmode
None

\end{description}\end{quote}

\end{fulllineitems}

\index{reset\_rounding\_parameters() (state.State method)}

\begin{fulllineitems}
\phantomsection\label{state:state.State.reset_rounding_parameters}\pysiglinewithargsret{\bfcode{reset\_rounding\_parameters}}{}{}
This function resets the rounding parameters to zero.
\begin{quote}\begin{description}
\item[{Returns}] \leavevmode
None

\end{description}\end{quote}

\end{fulllineitems}

\index{reset\_time\_status() (state.State method)}

\begin{fulllineitems}
\phantomsection\label{state:state.State.reset_time_status}\pysiglinewithargsret{\bfcode{reset\_time\_status}}{\emph{t\_start}, \emph{status=0}}{}
This function resets the time information to the current time and         sets the status. This function is usually used at the end of an activity.
\begin{quote}\begin{description}
\item[{Parameters}] \leavevmode\begin{itemize}
\item {} 
\textbf{\texttt{t\_start}} (\emph{int}) -- the start time {[}minutes, universal time{]}

\item {} 
\textbf{\texttt{status}} (\emph{int}) -- the status of the person

\end{itemize}

\item[{Returns}] \leavevmode
None

\end{description}\end{quote}

\end{fulllineitems}

\index{run\_activity() (state.State method)}

\begin{fulllineitems}
\phantomsection\label{state:state.State.run_activity}\pysiglinewithargsret{\bfcode{run\_activity}}{\emph{arg}, \emph{func}}{}
This function allows an activity to start, end, or halt
\begin{quote}\begin{description}
\item[{Parameters}] \leavevmode\begin{itemize}
\item {} 
\textbf{\texttt{arg}} (\emph{list}) -- arguments for the func() function

\item {} 
\textbf{\texttt{func}} (\emph{function}) -- arguments for the func() function

\end{itemize}

\item[{Returns}] \leavevmode
None

\end{description}\end{quote}

\end{fulllineitems}

\index{start\_activity() (state.State method)}

\begin{fulllineitems}
\phantomsection\label{state:state.State.start_activity}\pysiglinewithargsret{\bfcode{start\_activity}}{}{}
This function starts an activity
\begin{quote}\begin{description}
\item[{Returns}] \leavevmode
None

\end{description}\end{quote}

\end{fulllineitems}

\index{toString() (state.State method)}

\begin{fulllineitems}
\phantomsection\label{state:state.State.toString}\pysiglinewithargsret{\bfcode{toString}}{}{}
This function represents the State object as a string.
\begin{quote}\begin{description}
\item[{Returns}] \leavevmode
the representation of the State object

\item[{Return type}] \leavevmode
str

\end{description}\end{quote}

\end{fulllineitems}


\end{fulllineitems}



\section{temporal module}
\label{temporal::doc}\label{temporal:temporal-module}\label{temporal:module-temporal}\index{temporal (module)}
This file contains code that handles the time related aspects of this code.

This file contains code for class {\hyperref[temporal:temporal.Temporal]{\emph{\code{temporal.Temporal}}}}. This file also includes other functions that are accessed outside of the Temporal class.

\emph{Module author: Dr. Namdi Brandon}
\index{Temporal (class in temporal)}

\begin{fulllineitems}
\phantomsection\label{temporal:temporal.Temporal}\pysiglinewithargsret{\strong{class }\code{temporal.}\bfcode{Temporal}}{\emph{t\_univ=0}}{}
Bases: \code{object}

This class handles all the time keeping responsibilities.

Universal time is the total amount of time in minutes elapsed from the start of the
calendar year.

Day 0 at 0:00 corresponds to a universal time of 0

Day 1 at 0:00 corresponds to a universal time of 1 * 24 * 60

Day 359 at 0:00 corresponds to a universal time of 359 * 24 * 60
\begin{quote}\begin{description}
\item[{Parameters}] \leavevmode
\textbf{\texttt{t\_univ}} (\emph{int}) -- the time in universal time {[}minutes{]}

\item[{Variables}] \leavevmode\begin{itemize}
\item {} 
\textbf{\texttt{day}} (\emph{int}) -- the day number in the simulation

\item {} 
\textbf{\texttt{day\_of\_week}} (\emph{int}) -- a number 0, 1, 2, ... 6 corresponding to days of the week where 0 is Sunday, 1 is     Monday,  ... 6 is Saturday

\item {} 
\textbf{\texttt{dt}} (\emph{int}) -- the step size in the simulation {[}minutes{]} (\textbf{antiquated})

\item {} 
\textbf{\texttt{hour\_of\_day}} (\emph{int}) -- the hour of the day {[}0, 23{]}

\item {} 
\textbf{\texttt{is\_weekday}} (\emph{bool}) -- a flag indicating if it's a weekday (Monday-Friday) if True. False, otherwise.

\item {} 
\textbf{\texttt{is\_night}} (\emph{bool}) -- a flag indicating if the time of day is after DUSK and before DAWN if True.     False, otherwise.

\item {} 
\textbf{\texttt{min\_of\_day}} (\emph{int}) -- the minute of the day {[}0, 60 - 1{]}

\item {} 
\textbf{\texttt{t\_univ}} (\emph{int}) -- the universal time {[}minutes{]}

\item {} 
\textbf{\texttt{time\_of\_day}} (\emph{int}) -- the time of the day {[}minutes{]}, {[}0, 1, ... 24 * 60 -1{]}

\item {} 
\textbf{\texttt{season}} (\emph{int}) -- the season

\item {} 
\textbf{\texttt{tic}} (\emph{int}) -- indicates that current tick (each tick corresponds to a step of size dt)

\item {} 
\textbf{\texttt{step}} (\emph{int}) -- indicates the current step in the simulation {[}0, ... num\_steps-1{]}

\end{itemize}

\end{description}\end{quote}
\index{print\_day\_night() (temporal.Temporal method)}

\begin{fulllineitems}
\phantomsection\label{temporal:temporal.Temporal.print_day_night}\pysiglinewithargsret{\bfcode{print\_day\_night}}{}{}
Represents whether it's day or night as a string
\begin{quote}\begin{description}
\item[{Return msg}] \leavevmode
daytime / nighttime status (or an error message, if there is an error)

\item[{Return type}] \leavevmode
str

\end{description}\end{quote}

\end{fulllineitems}

\index{print\_day\_of\_week() (temporal.Temporal method)}

\begin{fulllineitems}
\phantomsection\label{temporal:temporal.Temporal.print_day_of_week}\pysiglinewithargsret{\bfcode{print\_day\_of\_week}}{}{}
Represents the day of the week as a string
\begin{quote}\begin{description}
\item[{Return msg}] \leavevmode
the day of the week (or an error message, if there is an error)

\item[{Return type}] \leavevmode
str

\end{description}\end{quote}

\end{fulllineitems}

\index{print\_season() (temporal.Temporal method)}

\begin{fulllineitems}
\phantomsection\label{temporal:temporal.Temporal.print_season}\pysiglinewithargsret{\bfcode{print\_season}}{}{}
Represents the seasons as a string
\begin{quote}\begin{description}
\item[{Returns}] \leavevmode
the season (or an error message, if there is an error)

\item[{Return type}] \leavevmode
str

\end{description}\end{quote}

\end{fulllineitems}

\index{print\_time\_of\_day\_to\_military() (temporal.Temporal method)}

\begin{fulllineitems}
\phantomsection\label{temporal:temporal.Temporal.print_time_of_day_to_military}\pysiglinewithargsret{\bfcode{print\_time\_of\_day\_to\_military}}{}{}
Represents the time of day as military time.
\begin{quote}\begin{description}
\item[{Returns}] \leavevmode
the time of day in military time

\item[{Return type}] \leavevmode
str

\end{description}\end{quote}

\end{fulllineitems}

\index{reset() (temporal.Temporal method)}

\begin{fulllineitems}
\phantomsection\label{temporal:temporal.Temporal.reset}\pysiglinewithargsret{\bfcode{reset}}{\emph{t\_univ}}{}
Reset the temporal object to the initial state.
\begin{quote}\begin{description}
\item[{Parameters}] \leavevmode
\textbf{\texttt{t\_univ}} (\emph{int}) -- The time {[}seconds, universal time{]} that the time should be reset to

\item[{Returns}] \leavevmode


\end{description}\end{quote}

\end{fulllineitems}

\index{set\_day\_of\_week() (temporal.Temporal method)}

\begin{fulllineitems}
\phantomsection\label{temporal:temporal.Temporal.set_day_of_week}\pysiglinewithargsret{\bfcode{set\_day\_of\_week}}{}{}
This function sets the day of the week.

In addition, this function sets the day count, the day of the week, and a flag indicating whether it         is a weekday or not.
\begin{quote}\begin{description}
\item[{Returns}] \leavevmode
None

\end{description}\end{quote}

\end{fulllineitems}

\index{set\_season() (temporal.Temporal method)}

\begin{fulllineitems}
\phantomsection\label{temporal:temporal.Temporal.set_season}\pysiglinewithargsret{\bfcode{set\_season}}{}{}
This function sets the season.

Day 0 is the beginning of winter
\begin{quote}\begin{description}
\item[{Returns}] \leavevmode
None

\end{description}\end{quote}

\end{fulllineitems}

\index{set\_time() (temporal.Temporal method)}

\begin{fulllineitems}
\phantomsection\label{temporal:temporal.Temporal.set_time}\pysiglinewithargsret{\bfcode{set\_time}}{}{}
This function sets all the time variable due to the universal time.

This function sets
\begin{enumerate}
\item {} 
the time of day

\item {} 
the day of the week

\item {} 
the season

\item {} 
the tic.

\end{enumerate}
\begin{quote}\begin{description}
\item[{Returns}] \leavevmode
None

\end{description}\end{quote}

\end{fulllineitems}

\index{set\_time\_of\_day() (temporal.Temporal method)}

\begin{fulllineitems}
\phantomsection\label{temporal:temporal.Temporal.set_time_of_day}\pysiglinewithargsret{\bfcode{set\_time\_of\_day}}{}{}
Given the universal time, this function sets the time of day in minutes.
\begin{quote}\begin{description}
\item[{Returns}] \leavevmode
None

\end{description}\end{quote}

\end{fulllineitems}

\index{toString() (temporal.Temporal method)}

\begin{fulllineitems}
\phantomsection\label{temporal:temporal.Temporal.toString}\pysiglinewithargsret{\bfcode{toString}}{}{}
This function represents the Temporal object as a string.
\begin{quote}\begin{description}
\item[{Return msg}] \leavevmode
the representation of the temporal object

\item[{Return type}] \leavevmode
str

\end{description}\end{quote}

\end{fulllineitems}

\index{update\_time() (temporal.Temporal method)}

\begin{fulllineitems}
\phantomsection\label{temporal:temporal.Temporal.update_time}\pysiglinewithargsret{\bfcode{update\_time}}{}{}
Increments the time by 1 time step.

\begin{notice}{warning}{Warning:}
This function is outdated!
\end{notice}
\begin{quote}\begin{description}
\item[{Returns}] \leavevmode
None

\end{description}\end{quote}

\end{fulllineitems}


\end{fulllineitems}

\index{convert\_cyclical\_to\_decimal() (in module temporal)}

\begin{fulllineitems}
\phantomsection\label{temporal:temporal.convert_cyclical_to_decimal}\pysiglinewithargsret{\code{temporal.}\bfcode{convert\_cyclical\_to\_decimal}}{\emph{t}}{}
This function converts cyclical time to decimal time
\begin{quote}\begin{description}
\item[{Parameters}] \leavevmode
\textbf{\texttt{t}} (\emph{int}) -- the time of day {[}minutes{]}

\item[{Return out}] \leavevmode
the time of day in {[}hours{]}

\item[{Return type}] \leavevmode
float

\end{description}\end{quote}

\end{fulllineitems}

\index{convert\_cylical\_to\_universal() (in module temporal)}

\begin{fulllineitems}
\phantomsection\label{temporal:temporal.convert_cylical_to_universal}\pysiglinewithargsret{\code{temporal.}\bfcode{convert\_cylical\_to\_universal}}{\emph{day}, \emph{time\_of\_day}}{}
This function converts a cyclical time to the universal time.
\begin{quote}\begin{description}
\item[{Parameters}] \leavevmode\begin{itemize}
\item {} 
\textbf{\texttt{day}} (\emph{int}) -- the day of the year

\item {} 
\textbf{\texttt{time\_of\_day}} (\emph{int}) -- the time of day {[}minutes{]}

\end{itemize}

\item[{Return t}] \leavevmode
the time in universal time

\item[{Return type}] \leavevmode
int

\end{description}\end{quote}

\end{fulllineitems}

\index{convert\_decimal\_to\_min() (in module temporal)}

\begin{fulllineitems}
\phantomsection\label{temporal:temporal.convert_decimal_to_min}\pysiglinewithargsret{\code{temporal.}\bfcode{convert\_decimal\_to\_min}}{\emph{t}}{}
This function takes in the time of day as a decimal and outputs the time in minutes
\begin{quote}\begin{description}
\item[{Parameters}] \leavevmode
\textbf{\texttt{t}} (\emph{float}) -- the time of day {[}0, 24) {[}hours{]}

\item[{Return out}] \leavevmode
the time of day {[}minutes{]}

\item[{Return type}] \leavevmode
int

\end{description}\end{quote}

\end{fulllineitems}

\index{convert\_universal\_to\_decimal() (in module temporal)}

\begin{fulllineitems}
\phantomsection\label{temporal:temporal.convert_universal_to_decimal}\pysiglinewithargsret{\code{temporal.}\bfcode{convert\_universal\_to\_decimal}}{\emph{t\_univ}}{}
This function takes in the universal time and converts it to the time of day in decimal format {[}0, 24)
\begin{quote}\begin{description}
\item[{Parameters}] \leavevmode
\textbf{\texttt{t\_univ}} (\emph{int}) -- the universal time {[}minutes{]}

\item[{Return out}] \leavevmode
the universal time {[}hours{]}

\item[{Return type}] \leavevmode
float

\end{description}\end{quote}

\end{fulllineitems}

\index{print\_military\_time() (in module temporal)}

\begin{fulllineitems}
\phantomsection\label{temporal:temporal.print_military_time}\pysiglinewithargsret{\code{temporal.}\bfcode{print\_military\_time}}{\emph{t}}{}
Represents the time of day in military time  assume that time is in minutes format.
\begin{quote}\begin{description}
\item[{Parameters}] \leavevmode
\textbf{\texttt{t}} (\emph{int}) -- the time of day {[}minutes{]}

\item[{Return msg}] \leavevmode
the time of day in military time 00:00

\item[{Return type}] \leavevmode
str

\end{description}\end{quote}

\end{fulllineitems}



\section{transport module}
\label{transport::doc}\label{transport:module-transport}\label{transport:transport-module}\index{transport (module)}
This module contains information about the {\hyperref[asset:asset.Asset]{\emph{\code{asset.Asset}}}} that allows a {\hyperref[person:person.Person]{\emph{\code{person.Person}}}} to do the following:
\begin{enumerate}
\item {} 
{\hyperref[commute:commute.Commute_To_Work]{\emph{\code{commute.Commute\_To\_Work}}}}

\item {} 
{\hyperref[commute:commute.Commute_From_Work]{\emph{\code{commute.Commute\_From\_Work}}}}

\end{enumerate}

activities. This module contains code for {\hyperref[transport:transport.Transport]{\emph{\code{transport.Transport}}}}.

\emph{Module author: Dr. Namdi Brandon}
\index{Transport (class in transport)}

\begin{fulllineitems}
\phantomsection\label{transport:transport.Transport}\pysigline{\strong{class }\code{transport.}\bfcode{Transport}}
Bases: {\hyperref[asset:asset.Asset]{\emph{\code{asset.Asset}}}}

This class is an asset that allows for commuting.

Activities in this asset:
\begin{enumerate}
\item {} 
{\hyperref[commute:commute.Commute_To_Work]{\emph{\code{commute.Commute\_To\_Work}}}}

\item {} 
{\hyperref[commute:commute.Commute_From_Work]{\emph{\code{commute.Commute\_From\_Work}}}}

\end{enumerate}
\index{initialize() (transport.Transport method)}

\begin{fulllineitems}
\phantomsection\label{transport:transport.Transport.initialize}\pysiglinewithargsret{\bfcode{initialize}}{\emph{people}}{}
This function sets the transport location according to whether or not the Person is commuting to or         from work.

\begin{notice}{note}{Note:}
This function just sets the transport object to be at the home
\end{notice}
\begin{quote}\begin{description}
\item[{Parameters}] \leavevmode
\textbf{\texttt{people}} (\emph{list}) -- a list of people in the simulation

\item[{Returns}] \leavevmode
None

\end{description}\end{quote}

\end{fulllineitems}


\end{fulllineitems}



\section{travel module}
\label{travel::doc}\label{travel:travel-module}\label{travel:module-travel}\index{travel (module)}
This module contains code for the {\hyperref[need:need.Need]{\emph{\code{need.Need}}}} associated with the desire to move from one environment to another.

This file contains code for {\hyperref[travel:travel.Travel]{\emph{\code{travel.Travel}}}}.

\emph{Module author: Dr. Namdi Brandon}
\index{Travel (class in travel)}

\begin{fulllineitems}
\phantomsection\label{travel:travel.Travel}\pysiglinewithargsret{\strong{class }\code{travel.}\bfcode{Travel}}{\emph{clock}, \emph{num\_sample\_points}}{}
Bases: {\hyperref[need:need.Need]{\emph{\code{need.Need}}}}

This class governs the need for traveling.
\begin{quote}\begin{description}
\item[{Parameters}] \leavevmode\begin{itemize}
\item {} 
\textbf{\texttt{clock}} ({\hyperref[temporal:temporal.Temporal]{\emph{\emph{temporal.Temporal}}}}) -- the time

\item {} 
\textbf{\texttt{num\_sample\_points}} (\emph{int}) -- the number of temporal nodes in the simulation

\end{itemize}

\end{description}\end{quote}
\index{decay() (travel.Travel method)}

\begin{fulllineitems}
\phantomsection\label{travel:travel.Travel.decay}\pysiglinewithargsret{\bfcode{decay}}{\emph{p}}{}
This function decays the satiation. Travel for commuting only decays when the work need is low
\begin{quote}\begin{description}
\item[{Parameters}] \leavevmode
\textbf{\texttt{p}} ({\hyperref[person:person.Person]{\emph{\emph{person.Person}}}}) -- the person whose satiation is decaying

\item[{Returns}] \leavevmode
None

\end{description}\end{quote}

\end{fulllineitems}

\index{decay\_work\_commute() (travel.Travel method)}

\begin{fulllineitems}
\phantomsection\label{travel:travel.Travel.decay_work_commute}\pysiglinewithargsret{\bfcode{decay\_work\_commute}}{\emph{p}}{}
This decays the satiation level in order to commute to work. For the satiation to decay the         person needs the following
\begin{enumerate}
\item {} 
The agent should leave the home to go to work

\item {} 
The agent should leave work to go home

\end{enumerate}
\begin{quote}\begin{description}
\item[{Parameters}] \leavevmode
\textbf{\texttt{p}} ({\hyperref[person:person.Person]{\emph{\emph{person.Person}}}}) -- the person of interest

\item[{Returns}] \leavevmode
None

\end{description}\end{quote}

\end{fulllineitems}

\index{initialize() (travel.Travel method)}

\begin{fulllineitems}
\phantomsection\label{travel:travel.Travel.initialize}\pysiglinewithargsret{\bfcode{initialize}}{\emph{p}}{}
This function initializes the Travel by updating the {\hyperref[scheduler:scheduler.Scheduler]{\emph{\code{scheduler.Scheduler}}}} for Travel
\begin{quote}\begin{description}
\item[{Parameters}] \leavevmode
\textbf{\texttt{p}} ({\hyperref[person:person.Person]{\emph{\emph{person.Person}}}}) -- the person of interest

\item[{Returns}] \leavevmode
None

\end{description}\end{quote}

\end{fulllineitems}

\index{perceive() (travel.Travel method)}

\begin{fulllineitems}
\phantomsection\label{travel:travel.Travel.perceive}\pysiglinewithargsret{\bfcode{perceive}}{\emph{clock}, \emph{job}}{}
This function gives the satiation for Travel if the Travel need is addressed now.
\begin{quote}\begin{description}
\item[{Note}] \leavevmode
going to work can only happen according to work hours of the job.

\item[{Parameters}] \leavevmode\begin{itemize}
\item {} 
\textbf{\texttt{clock}} ({\hyperref[temporal:temporal.Temporal]{\emph{\emph{temporal.Temporal}}}}) -- the time the need to travel is perceived

\item {} 
\textbf{\texttt{job}} ({\hyperref[occupation:occupation.Occupation]{\emph{\emph{occupation.Occupation}}}}) -- the job of the person

\end{itemize}

\item[{Return mag}] \leavevmode
the perceived magnitude of the need

\item[{Return type}] \leavevmode
float

\end{description}\end{quote}

\end{fulllineitems}


\end{fulllineitems}



\section{universe module}
\label{universe::doc}\label{universe:universe-module}\label{universe:module-universe}\index{universe (module)}
This module contains code that is responsible for running the simulation. This file contains {\hyperref[universe:universe.Universe]{\emph{\code{universe.Universe}}}}. The Universe contains all agents and objects. The Universe is responsible for running the simulation itself.

\emph{Module author: Dr. Namdi Brandon}
\index{Universe (class in universe)}

\begin{fulllineitems}
\phantomsection\label{universe:universe.Universe}\pysiglinewithargsret{\strong{class }\code{universe.}\bfcode{Universe}}{\emph{num\_steps}, \emph{dt}, \emph{t\_start}, \emph{num\_people}}{}
Bases: \code{object}

The Universe is the governing engine of the simulation.
\begin{quote}\begin{description}
\item[{Parameters}] \leavevmode\begin{itemize}
\item {} 
\textbf{\texttt{num\_steps}} (\emph{int}) -- the number of time steps in the simulation

\item {} 
\textbf{\texttt{dt}} (\emph{int}) -- the step size in the simulation {[}minutes{]}

\item {} 
\textbf{\texttt{t\_start}} (\emph{int}) -- the start time for the simulation {[}minutes, universal time{]}

\item {} 
\textbf{\texttt{num\_people}} (\emph{int}) -- the number of people in the household

\end{itemize}

\item[{Variables}] \leavevmode\begin{itemize}
\item {} 
\textbf{\texttt{clock}} ({\hyperref[temporal:temporal.Temporal]{\emph{\emph{temporal.Temporal}}}}) -- does the timekeeping in the simulation

\item {} 
\textbf{\texttt{"home"}} ({\hyperref[home:home.Home]{\emph{\emph{home.Home}}}}) -- the home the Persons live in

\item {} 
\textbf{\texttt{people}} (\emph{list}) -- a list of all Person objects created in the Universe object

\item {} 
\textbf{\texttt{t\_start}} (\emph{int}) -- the start time for the simulation {[}minutes, universal time{]}

\item {} 
\textbf{\texttt{t\_end}} (\emph{int}) -- the last time for the simulation {[}minutes, universal time{]}

\item {} 
\textbf{\texttt{schedule}} ({\hyperref[scheduler:scheduler.Scheduler]{\emph{\emph{scheduler.Scheduler}}}}) -- the schedule governing each agent's needs

\end{itemize}

\end{description}\end{quote}
\index{address\_needs() (universe.Universe method)}

\begin{fulllineitems}
\phantomsection\label{universe:universe.Universe.address_needs}\pysiglinewithargsret{\bfcode{address\_needs}}{\emph{do\_interruption=False}}{}
This function checks the needs of the agents

The function uses a recursion loop to choose activities.

The Recursion:
\begin{enumerate}
\item {} 
Gather all of the advertisements (object-person pairings)

\item {} 
Assigns 1 activity to the Person with the highest score.

\item {} 
That Person starts the activity, thereby updating the state of available activities in the home.

\item {} 
The recursion starts again, where the Home advertises to all remaining Person(s).

\end{enumerate}
\begin{quote}\begin{description}
\item[{Note}] \leavevmode
If no activity will be done this time step to a Person, a Person is set to             the temporary status \code{state.IDLE\_TEMP}, so that the Home knows not to advertise to that Person.

\item[{Parameters}] \leavevmode
\textbf{\texttt{do\_interruption}} (\emph{bool}) -- this flag indicates whether or not advertisements should be made             for activities that will interrupt the current activity (if True). If False, the advertisements             are made for non-interrupting activities.

\item[{Returns}] \leavevmode
None

\end{description}\end{quote}

\end{fulllineitems}

\index{advertise() (universe.Universe method)}

\begin{fulllineitems}
\phantomsection\label{universe:universe.Universe.advertise}\pysiglinewithargsret{\bfcode{advertise}}{\emph{do\_interruption=False}}{}
This function obtains a list of all of the possible activities each person could potentially start in         this time step.
\begin{quote}\begin{description}
\item[{Parameters}] \leavevmode
\textbf{\texttt{do\_interruption}} (\emph{bool}) -- this flag indicates whether to make advertisements due to an         interrupting activity (if True) or not (if False).

\item[{Return ads}] \leavevmode
ads is a list of dictionaries for advertisements:

Dictionary  (score, asset, activity, person) containing the various data for
each advertisement: (score, asset, activity, person) coupling

\item[{Return type}] \leavevmode
list

\end{description}\end{quote}

\end{fulllineitems}

\index{check\_expired\_activities() (universe.Universe method)}

\begin{fulllineitems}
\phantomsection\label{universe:universe.Universe.check_expired_activities}\pysiglinewithargsret{\bfcode{check\_expired\_activities}}{}{}
This function checks for expired activities. If found, end the activities.
\begin{quote}\begin{description}
\item[{Returns}] \leavevmode
None

\end{description}\end{quote}

\end{fulllineitems}

\index{decay\_needs() (universe.Universe method)}

\begin{fulllineitems}
\phantomsection\label{universe:universe.Universe.decay_needs}\pysiglinewithargsret{\bfcode{decay\_needs}}{\emph{dt=None}}{}
This function decays the needs according to the default behavior. That is, assume the needs are not         addressed earlier.
\begin{quote}\begin{description}
\item[{Parameters}] \leavevmode
\textbf{\texttt{dt}} (\emph{int}) -- the number of minutes to decay the needs by. The default behavior is to use the scheduler's         time. If a number is specified, then it should be the number of minutes until the end of the simulation.

\item[{Returns}] \leavevmode
None

\end{description}\end{quote}

\end{fulllineitems}

\index{initial\_step() (universe.Universe method)}

\begin{fulllineitems}
\phantomsection\label{universe:universe.Universe.initial_step}\pysiglinewithargsret{\bfcode{initial\_step}}{}{}
This function is supposed to run the first time step of the run() loop
\begin{enumerate}
\item {} 
store the current time

\item {} 
address the needs assuming interruption

\item {} 
address the needs assuming NO interruption

\item {} 
update the history

\item {} 
update the clock

\item {} 
decay the needs

\end{enumerate}

\begin{notice}{note}{Note:}
this function is \textbf{NOT} called on in the current implementation yet
\end{notice}
\begin{quote}\begin{description}
\item[{Returns}] \leavevmode
None

\end{description}\end{quote}

\end{fulllineitems}

\index{initialize\_needs() (universe.Universe method)}

\begin{fulllineitems}
\phantomsection\label{universe:universe.Universe.initialize_needs}\pysiglinewithargsret{\bfcode{initialize\_needs}}{}{}
This function initializes the need state of each Person at the beginning of simulation based on         the current time.

The needs are initialized in this order (the order matters)
\begin{enumerate}
\item {} 
Rest

\item {} 
Hunger

\item {} 
Income

\item {} 
Travel

\item {} 
Interruption

\end{enumerate}
\begin{quote}\begin{description}
\item[{Returns}] \leavevmode
None

\end{description}\end{quote}

\end{fulllineitems}

\index{print\_activity\_info() (universe.Universe method)}

\begin{fulllineitems}
\phantomsection\label{universe:universe.Universe.print_activity_info}\pysiglinewithargsret{\bfcode{print\_activity\_info}}{\emph{p}}{}
This function stores activity info used for testing / developing/ debugging as a string.
\begin{quote}\begin{description}
\item[{Parameters}] \leavevmode
\textbf{\texttt{p}} ({\hyperref[person:person.Person]{\emph{\emph{person.Person}}}}) -- the person of interest

\item[{Returns}] \leavevmode
None

\end{description}\end{quote}

\end{fulllineitems}

\index{reset() (universe.Universe method)}

\begin{fulllineitems}
\phantomsection\label{universe:universe.Universe.reset}\pysiglinewithargsret{\bfcode{reset}}{\emph{t\_univ}}{}
This code resets the simulation by initializing the agents, home, and clock to the beginning status         of the simulation.

This code does the following:
\begin{enumerate}
\item {} 
reset the clock

\item {} 
reset the home

\item {} 
reset each person

\item {} 
initialize each person

\item {} 
initialize the home

\end{enumerate}
\begin{quote}\begin{description}
\item[{Parameters}] \leavevmode\begin{itemize}
\item {} 
\textbf{\texttt{p}} ({\hyperref[params:params.Params]{\emph{\emph{params.Params}}}}) -- the parameters

\item {} 
\textbf{\texttt{t\_univ}} (\emph{int}) -- the time of the beginning of the simulation {[}seconds{]}

\end{itemize}

\item[{Returns}] \leavevmode


\end{description}\end{quote}

\end{fulllineitems}

\index{run() (universe.Universe method)}

\begin{fulllineitems}
\phantomsection\label{universe:universe.Universe.run}\pysiglinewithargsret{\bfcode{run}}{}{}
This function is responsible for running the simulation. Instead of running the simulation minute-by-minute,         in an effort to reduce run-time, the simulation skips time steps and addresses the agent at times that         actions should occur. These times are dictated by the scheduler.

The function proceeds as following:

While the current time is less than the final time
\begin{enumerate}
\item {} 
check for expired activities for all agents. If activities should have expired, tell the agent to end them

\item {} 
start new activities by addressing the needs for all agents (assuming no interruption)

\item {} 
decay the satiation for Interruption for all agents

\item {} 
start new activities by addressing the needs for all agents (assuming interruptions only)

\item {} 
update the history of the status of each agent

\item {} 
find the next time to jump to in the simulation according to the scheduler

\item {} 
update the clock to the new time

\item {} 
decay the needs for all agents

\item {} 
Repeat

\end{enumerate}

For the last time step
\begin{enumerate}
\item {} 
update the clock

\item {} 
decay the needs for each agent

\item {} 
update the history of the status of each agent

\end{enumerate}

\begin{notice}{note}{Note:}
I must change N\_MAX to N\_MAX = DAY\_2\_MIN * 365
\end{notice}
\begin{quote}\begin{description}
\item[{Returns}] \leavevmode


\end{description}\end{quote}

\end{fulllineitems}

\index{select\_activity() (universe.Universe method)}

\begin{fulllineitems}
\phantomsection\label{universe:universe.Universe.select_activity}\pysiglinewithargsret{\bfcode{select\_activity}}{\emph{ads}}{}
Given a list of activity advertisements, this function selects the Person
with the largest activity score and outputs the score, asset, activity, and person.
\begin{quote}\begin{description}
\item[{Parameters}] \leavevmode
\textbf{\texttt{ads}} (\emph{list}) -- a list of advertisements for this time step

\item[{Return chosen}] \leavevmode
the selected activity advertisement (score, asset, activity, person)

\item[{Return type}] \leavevmode
dict

\end{description}\end{quote}

\end{fulllineitems}

\index{set\_alarm() (universe.Universe method)}

\begin{fulllineitems}
\phantomsection\label{universe:universe.Universe.set_alarm}\pysiglinewithargsret{\bfcode{set\_alarm}}{}{}
This function sets the alarm for those Person(s) who use an alarm

\begin{notice}{note}{Note:}
This function is \textbf{NOT} used. There is currently no alarm capability.
\end{notice}
\begin{quote}\begin{description}
\item[{Returns}] \leavevmode
None

\end{description}\end{quote}

\end{fulllineitems}

\index{test\_func() (universe.Universe method)}

\begin{fulllineitems}
\phantomsection\label{universe:universe.Universe.test_func}\pysiglinewithargsret{\bfcode{test\_func}}{}{}~
\begin{notice}{note}{Note:}
This function is just for debugging.
\end{notice}
\begin{quote}\begin{description}
\item[{Returns}] \leavevmode


\end{description}\end{quote}

\end{fulllineitems}

\index{toString() (universe.Universe method)}

\begin{fulllineitems}
\phantomsection\label{universe:universe.Universe.toString}\pysiglinewithargsret{\bfcode{toString}}{}{}
Represent the Universe object as a string.

This function outputs the representation of:
\begin{enumerate}
\item {} 
the clock

\item {} 
the home

\item {} 
agent person residing in the home

\end{enumerate}
\begin{quote}\begin{description}
\item[{Return msg}] \leavevmode
a representation of the Universe object

\item[{Return type}] \leavevmode
str

\end{description}\end{quote}

\end{fulllineitems}

\index{update\_clock() (universe.Universe method)}

\begin{fulllineitems}
\phantomsection\label{universe:universe.Universe.update_clock}\pysiglinewithargsret{\bfcode{update\_clock}}{\emph{t}}{}
This function updates the clock by
\begin{enumerate}
\item {} 
setting the clock to the given time

\item {} 
updating the step of the simulation

\item {} 
storing the history of the time nodes used in the simulation

\end{enumerate}
\begin{quote}\begin{description}
\item[{Parameters}] \leavevmode
\textbf{\texttt{t}} (\emph{int}) -- the time the clock should be set to

\item[{Returns}] \leavevmode


\end{description}\end{quote}

\end{fulllineitems}

\index{update\_history() (universe.Universe method)}

\begin{fulllineitems}
\phantomsection\label{universe:universe.Universe.update_history}\pysiglinewithargsret{\bfcode{update\_history}}{\emph{step}}{}
Update the histories for each Person by storing the following:
\begin{enumerate}
\item {} 
the current state's status

\item {} 
the current activity

\item {} 
the current satiation value for each needs

\item {} 
the current location

\end{enumerate}
\begin{quote}\begin{description}
\item[{Parameters}] \leavevmode
\textbf{\texttt{step}} (\emph{int}) -- the time step

\item[{Returns}] \leavevmode
None

\end{description}\end{quote}

\end{fulllineitems}

\index{update\_history\_new() (universe.Universe method)}

\begin{fulllineitems}
\phantomsection\label{universe:universe.Universe.update_history_new}\pysiglinewithargsret{\bfcode{update\_history\_new}}{}{}
Update the histories of each person.
\begin{quote}\begin{description}
\item[{Returns}] \leavevmode
None

\end{description}\end{quote}

\end{fulllineitems}


\end{fulllineitems}



\section{work module}
\label{work::doc}\label{work:module-work}\label{work:work-module}\index{work (module)}
This module contains code that governs the {\hyperref[activity:activity.Activity]{\emph{\code{activity.Activity}}}} that gives a Person the ability to go to work/ school.

This file contains {\hyperref[work:work.Work]{\emph{\code{work.Work}}}}.

\emph{Module author: Dr. Namdi Brandon}
\index{Work (class in work)}

\begin{fulllineitems}
\phantomsection\label{work:work.Work}\pysigline{\strong{class }\code{work.}\bfcode{Work}}
Bases: {\hyperref[activity:activity.Activity]{\emph{\code{activity.Activity}}}}

This class allows a Person to work / go to school in order to satisfy the need     {\hyperref[income:income.Income]{\emph{\code{income.Income}}}}.
\index{advertise() (work.Work method)}

\begin{fulllineitems}
\phantomsection\label{work:work.Work.advertise}\pysiglinewithargsret{\bfcode{advertise}}{\emph{p}}{}
This function calculates the score of the advertised work activity to a Person
\begin{quote}\begin{description}
\item[{Parameters}] \leavevmode
\textbf{\texttt{p}} ({\hyperref[person:person.Person]{\emph{\emph{person.Person}}}}) -- the person of interest

\item[{Return score}] \leavevmode
\item[{Return type}] \leavevmode
float

\end{description}\end{quote}

\end{fulllineitems}

\index{end() (work.Work method)}

\begin{fulllineitems}
\phantomsection\label{work:work.Work.end}\pysiglinewithargsret{\bfcode{end}}{\emph{p}}{}
This function handles the end of an Activity
\begin{quote}\begin{description}
\item[{Parameters}] \leavevmode
\textbf{\texttt{p}} ({\hyperref[person:person.Person]{\emph{\emph{person.Person}}}}) -- the person of interest

\item[{Returns}] \leavevmode
None

\end{description}\end{quote}

\end{fulllineitems}

\index{end\_work() (work.Work method)}

\begin{fulllineitems}
\phantomsection\label{work:work.Work.end_work}\pysiglinewithargsret{\bfcode{end\_work}}{\emph{p}}{}
This function sets the variables pertaining to coming back from work by doing the following:
\begin{enumerate}
\item {} 
free the asset from use

\item {} 
set the asset's state to \code{state.IDLE}

\item {} 
set the Income satiation to 1

\item {} 
decay the need Travel

\item {} 
sample the new work start time

\item {} 
sample the new work end time

\item {} 
update the scheduler to take into account the next work event

\end{enumerate}
\begin{quote}\begin{description}
\item[{Parameters}] \leavevmode
\textbf{\texttt{p}} ({\hyperref[person:person.Person]{\emph{\emph{person.Person}}}}) -- the person of interest

\item[{Returns}] \leavevmode
None

\end{description}\end{quote}

\end{fulllineitems}

\index{halt() (work.Work method)}

\begin{fulllineitems}
\phantomsection\label{work:work.Work.halt}\pysiglinewithargsret{\bfcode{halt}}{\emph{p}}{}
This function handles an interruption of an Activity.
\begin{quote}\begin{description}
\item[{Parameters}] \leavevmode
\textbf{\texttt{p}} ({\hyperref[person:person.Person]{\emph{\emph{person.Person}}}}) -- the person of interest

\item[{Returns}] \leavevmode
None

\end{description}\end{quote}

\end{fulllineitems}

\index{halt\_work() (work.Work method)}

\begin{fulllineitems}
\phantomsection\label{work:work.Work.halt_work}\pysiglinewithargsret{\bfcode{halt\_work}}{\emph{p}}{}
This function interrupts the work behavior by doing the following:
\begin{enumerate}
\item {} 
frees the current asset

\item {} 
the asset's state is set to \code{state.IDLE}

\item {} 
the Interruption satiation is set to 1.0

\item {} 
the Interruption's activity start/ stop

\end{enumerate}
\begin{quote}\begin{description}
\item[{Note}] \leavevmode
No benefits of working are given while being interrupted

\item[{Parameters}] \leavevmode
\textbf{\texttt{p}} ({\hyperref[person:person.Person]{\emph{\emph{person.Person}}}}) -- the person of interest

\item[{Returns}] \leavevmode
None

\end{description}\end{quote}

\end{fulllineitems}

\index{set\_end\_time() (work.Work method)}

\begin{fulllineitems}
\phantomsection\label{work:work.Work.set_end_time}\pysiglinewithargsret{\bfcode{set\_end\_time}}{\emph{p}}{}
Calculates the end time of work.
\begin{quote}\begin{description}
\item[{Parameters}] \leavevmode
\textbf{\texttt{p}} ({\hyperref[person:person.Person]{\emph{\emph{person.Person}}}}) -- the person of interest

\item[{Return t\_end}] \leavevmode
the end time {[}minutes, universal time{]}

\item[{Return type}] \leavevmode
int

\end{description}\end{quote}

\end{fulllineitems}

\index{start() (work.Work method)}

\begin{fulllineitems}
\phantomsection\label{work:work.Work.start}\pysiglinewithargsret{\bfcode{start}}{\emph{p}}{}
This handles the start of an Activity
\begin{quote}\begin{description}
\item[{Parameters}] \leavevmode
\textbf{\texttt{p}} ({\hyperref[person:person.Person]{\emph{\emph{person.Person}}}}) -- the person of interest

\item[{Returns}] \leavevmode
None

\end{description}\end{quote}

\end{fulllineitems}

\index{start\_work() (work.Work method)}

\begin{fulllineitems}
\phantomsection\label{work:work.Work.start_work}\pysiglinewithargsret{\bfcode{start\_work}}{\emph{p}}{}
This function starts the work activity
\begin{itemize}
\item {} 
updates that asset's status and number of users

\item {} 
changes the location of the Person

\item {} 
updates that person's status

\item {} 
calculates the end time of the work activity

\item {} 
update the scheduler for the Income satiation

\item {} 
update the scheduler for the Travel satiation

\item {} 
set the day for the work period

\end{itemize}
\begin{quote}\begin{description}
\item[{Parameters}] \leavevmode
\textbf{\texttt{p}} ({\hyperref[person:person.Person]{\emph{\emph{person.Person}}}}) -- the person of interest

\item[{Returns}] \leavevmode
None

\end{description}\end{quote}

\end{fulllineitems}

\index{test\_func() (work.Work method)}

\begin{fulllineitems}
\phantomsection\label{work:work.Work.test_func}\pysiglinewithargsret{\bfcode{test\_func}}{\emph{p}}{}~
\begin{notice}{note}{Note:}
This function is \textbf{NOT} used.
\end{notice}
\begin{quote}\begin{description}
\item[{Parameters}] \leavevmode
\textbf{\texttt{p}} ({\hyperref[person:person.Person]{\emph{\emph{person.Person}}}}) -- the person of interest

\item[{Returns}] \leavevmode


\end{description}\end{quote}

\end{fulllineitems}


\end{fulllineitems}



\section{workplace module}
\label{workplace::doc}\label{workplace:module-workplace}\label{workplace:workplace-module}\index{workplace (module)}
This module contains code for the {\hyperref[asset:asset.Asset]{\emph{\code{asset.Asset}}}} that allows a Person to go to work / school.

This file contains {\hyperref[workplace:workplace.Workplace]{\emph{\code{workplace.Workplace}}}}.

\emph{Module author: Dr. Namdi Brandon}
\index{Workplace (class in workplace)}

\begin{fulllineitems}
\phantomsection\label{workplace:workplace.Workplace}\pysigline{\strong{class }\code{workplace.}\bfcode{Workplace}}
Bases: {\hyperref[asset:asset.Asset]{\emph{\code{asset.Asset}}}}

This class allows a Person to go to work / school.

Activities in this asset: {\hyperref[work:work.Work]{\emph{\code{work.Work}}}}

\end{fulllineitems}



\chapter{Run Directory}
\label{index:run-directory}
These are the files needed to run an instance of ABMHAP with one agent parametrized by user-defined parameters.

The driver for these type of runs is main.py.

Contents:


\section{main module}
\label{main::doc}\label{main:main-module}\label{main:module-main}\index{main (module)}
This is code is runs the simulation for the Agent-Based Model of Human Activity Patterns (ABMHAP) module of the Life Cycle Human Exposure Model (LC-HEM) project.

In order to run the code, do the following:
\begin{enumerate}
\item {} 
set the user-defined parameters of the simulation in main\_params.py

\item {} \begin{description}
\item[{run the code via}] \leavevmode
\textgreater{} python main.py

\end{description}

\end{enumerate}

\begin{notice}{note}{Note:}
In order to run the debugger do the following in windows:

\textgreater{} python -m pdb main.py
\end{notice}

\emph{Module author: Dr. Namdi Brandon}
\index{get\_diary() (in module main)}

\begin{fulllineitems}
\phantomsection\label{main:main.get_diary}\pysiglinewithargsret{\code{main.}\bfcode{get\_diary}}{\emph{u}}{}
This function output the result of the simulation in terms of an activity diary.
\begin{quote}\begin{description}
\item[{Parameters}] \leavevmode
\textbf{\texttt{u}} ({\hyperref[universe:universe.Universe]{\emph{\emph{universe.Universe}}}}) -- the governing engine of the simulation

\item[{Returns}] \leavevmode
the activity diary describing the behavior of the agent

\item[{Return type}] \leavevmode
{\hyperref[diary:diary.Diary]{\emph{diary.Diary}}}

\end{description}\end{quote}

\end{fulllineitems}

\index{plot\_cdfs() (in module main)}

\begin{fulllineitems}
\phantomsection\label{main:main.plot_cdfs}\pysiglinewithargsret{\code{main.}\bfcode{plot\_cdfs}}{\emph{d}, \emph{keys}}{}
This function plots the cumulative distribution function of start time, end time, and duration for     each activity in the the simulation.
\begin{quote}\begin{description}
\item[{Parameters}] \leavevmode\begin{itemize}
\item {} 
\textbf{\texttt{d}} ({\hyperref[diary:diary.Diary]{\emph{\emph{diary.Diary}}}}) -- the results of the simulation

\item {} 
\textbf{\texttt{keys}} (\emph{list}) -- list of activities to graph

\end{itemize}

\item[{Returns}] \leavevmode


\end{description}\end{quote}

\end{fulllineitems}

\index{plot\_histograms() (in module main)}

\begin{fulllineitems}
\phantomsection\label{main:main.plot_histograms}\pysiglinewithargsret{\code{main.}\bfcode{plot\_histograms}}{\emph{d}, \emph{keys}}{}
This function plots the histograms of start time, end time, and duration for each activity in     the the simulation.
\begin{quote}\begin{description}
\item[{Parameters}] \leavevmode\begin{itemize}
\item {} 
\textbf{\texttt{d}} ({\hyperref[diary:diary.Diary]{\emph{\emph{diary.Diary}}}}) -- the results of the simulation

\item {} 
\textbf{\texttt{keys}} (\emph{list}) -- list of activities to graph

\end{itemize}

\item[{Returns}] \leavevmode


\end{description}\end{quote}

\end{fulllineitems}

\index{save\_output() (in module main)}

\begin{fulllineitems}
\phantomsection\label{main:main.save_output}\pysiglinewithargsret{\code{main.}\bfcode{save\_output}}{\emph{df}, \emph{fname}}{}
This function saves the output of the simulation.
\begin{quote}\begin{description}
\item[{Parameters}] \leavevmode\begin{itemize}
\item {} 
\textbf{\texttt{df}} (\emph{pandas.core.frame.DataFrame}) -- the activity-diary output of the simulation

\item {} 
\textbf{\texttt{fname}} (\emph{str}) -- the file name of the saved file. It must end with ''.csv''

\end{itemize}

\end{description}\end{quote}

\end{fulllineitems}



\section{main\_params module}
\label{main_params::doc}\label{main_params:module-main_params}\label{main_params:main-params-module}\index{main\_params (module)}
This module is responsible for containing parameters that main.py uses to control the simulation.

The user should set the parameters in this module \textbf{before} running the driver (main.py)

\emph{Module author: Dr. Namdi Brandon}
\index{set\_no\_variation() (in module main\_params)}

\begin{fulllineitems}
\phantomsection\label{main_params:main_params.set_no_variation}\pysiglinewithargsret{\code{main\_params.}\bfcode{set\_no\_variation}}{\emph{num\_people}}{}
This function sets the standard deviations in all of the activity-parameters to zero.
\begin{quote}\begin{description}
\item[{Parameters}] \leavevmode
\textbf{\texttt{num\_people}} (\emph{int}) -- the number of people in the simulation

\item[{Returns}] \leavevmode
a tuple of the standard deviations of all of the activity-parameters

\end{description}\end{quote}

\end{fulllineitems}



\section{scenario module}
\label{scenario::doc}\label{scenario:module-scenario}\label{scenario:scenario-module}\index{scenario (module)}
This file contains information to run the Agent-Based Model of Human Activity Patterns (ABMHAP) in in different simulation scenarios in which the agent has a user-defined parametrization.

The following classes are in this module
\begin{enumerate}
\item {} 
{\hyperref[scenario:scenario.Scenario]{\emph{\code{scenario.Scenario}}}}

\item {} 
{\hyperref[scenario:scenario.Solo]{\emph{\code{scenario.Solo}}}}

\item {} 
{\hyperref[scenario:scenario.Duo]{\emph{\code{scenario.Duo}}}}

\end{enumerate}

\emph{Module author: Dr. Namdi Brandon}
\index{Duo (class in scenario)}

\begin{fulllineitems}
\phantomsection\label{scenario:scenario.Duo}\pysiglinewithargsret{\strong{class }\code{scenario.}\bfcode{Duo}}{\emph{hhld\_params}}{}
Bases: {\hyperref[scenario:scenario.Scenario]{\emph{\code{scenario.Scenario}}}}

This class parametrizes / runs a simulation scenario for the cases where two Singleton     ({\hyperref[singleton:singleton.Singleton]{\emph{\code{singleton.Singleton}}}}) persons live in the same residence.

\begin{notice}{note}{Note:}
This scenario is used in order to check for activity conflicts among 2 agents living in         the same household. Currently it is used primarily as a debugging tool.
\end{notice}
\begin{quote}\begin{description}
\item[{Parameters}] \leavevmode
\textbf{\texttt{hhld\_params}} ({\hyperref[params:params.Params]{\emph{\emph{params.Params}}}}) -- the parameters for the household that contain relevant information     for the simulation

\end{description}\end{quote}

\end{fulllineitems}

\index{Scenario (class in scenario)}

\begin{fulllineitems}
\phantomsection\label{scenario:scenario.Scenario}\pysiglinewithargsret{\strong{class }\code{scenario.}\bfcode{Scenario}}{\emph{hhld\_params}}{}
Bases: \code{object}

This class governs what a simulation scenario consists of.
\begin{quote}\begin{description}
\item[{Parameters}] \leavevmode
\textbf{\texttt{hhld\_params}} ({\hyperref[params:params.Params]{\emph{\emph{params.Params}}}}) -- the parameters for the household that contain relevant information     for the simulation

\item[{Variables}] \leavevmode\begin{itemize}
\item {} 
\textbf{\texttt{id}} (\emph{int}) -- the scenario identifier number

\item {} 
\textbf{\texttt{u}} ({\hyperref[universe:universe.Universe]{\emph{\emph{universe.Universe}}}}) -- the universe object for the simulation

\item {} 
\textbf{\texttt{'params'}} ({\hyperref[params:params.Params]{\emph{\emph{params.Params}}}}) -- the parameters needed that control the simulation

\end{itemize}

\end{description}\end{quote}
\index{activity\_diary() (scenario.Scenario method)}

\begin{fulllineitems}
\phantomsection\label{scenario:scenario.Scenario.activity_diary}\pysiglinewithargsret{\bfcode{activity\_diary}}{}{}
This function returns the activity diary for each person

Each person will attain the following tuple
\begin{enumerate}
\item {} 
grouping of the index for each activity

\item {} 
the day, (start-time, end-time), activity code, and location for each activity-event, in a numeric format

\item {} 
the same as above in a string format

\end{enumerate}
\begin{quote}\begin{description}
\item[{Returns}] \leavevmode


\end{description}\end{quote}

\end{fulllineitems}

\index{default\_location() (scenario.Scenario method)}

\begin{fulllineitems}
\phantomsection\label{scenario:scenario.Scenario.default_location}\pysiglinewithargsret{\bfcode{default\_location}}{}{}
Sets the default location for all Person's to be be at the home.

This location may be overridden later in the initialization of Persons.
\begin{quote}\begin{description}
\item[{Returns}] \leavevmode
None

\end{description}\end{quote}

\end{fulllineitems}

\index{initialize() (scenario.Scenario method)}

\begin{fulllineitems}
\phantomsection\label{scenario:scenario.Scenario.initialize}\pysiglinewithargsret{\bfcode{initialize}}{}{}
This function initializes the scenario before the simulation scenario is run

More specifically, the function does the following:
\begin{enumerate}
\item {} 
Sets the state and location for each person

\item {} 
Sets the home

\item {} 
Initialize the initial need-association states for the Person(s) and Home

\end{enumerate}
\begin{quote}\begin{description}
\item[{Returns}] \leavevmode
None

\end{description}\end{quote}

\end{fulllineitems}

\index{run() (scenario.Scenario method)}

\begin{fulllineitems}
\phantomsection\label{scenario:scenario.Scenario.run}\pysiglinewithargsret{\bfcode{run}}{}{}
This function initializes the scenario and then runs the ABMHAP simulation.
\begin{quote}\begin{description}
\item[{Returns}] \leavevmode
None

\end{description}\end{quote}

\end{fulllineitems}

\index{set\_home() (scenario.Scenario method)}

\begin{fulllineitems}
\phantomsection\label{scenario:scenario.Scenario.set_home}\pysiglinewithargsret{\bfcode{set\_home}}{}{}
This function sets aspects of the home in order to run the simulation scenario.

More specifically, the function does the following
\begin{enumerate}
\item {} 
set the home revenue

\item {} 
set the home population

\end{enumerate}
\begin{quote}\begin{description}
\item[{Returns}] \leavevmode
None

\end{description}\end{quote}

\end{fulllineitems}

\index{set\_state() (scenario.Scenario method)}

\begin{fulllineitems}
\phantomsection\label{scenario:scenario.Scenario.set_state}\pysiglinewithargsret{\bfcode{set\_state}}{}{}
This function initializes the scenario in order to run the simulation.

More specifically, this function does the following:
\begin{enumerate}
\item {} \begin{description}
\item[{For each Person, the following is set:}] \leavevmode\begin{enumerate}
\item {} 
identification number

\item {} 
the state

\end{enumerate}

\end{description}

\end{enumerate}
\begin{quote}\begin{description}
\item[{Returns}] \leavevmode
None

\end{description}\end{quote}

\end{fulllineitems}


\end{fulllineitems}

\index{Solo (class in scenario)}

\begin{fulllineitems}
\phantomsection\label{scenario:scenario.Solo}\pysiglinewithargsret{\strong{class }\code{scenario.}\bfcode{Solo}}{\emph{hhld\_params}}{}
Bases: {\hyperref[scenario:scenario.Scenario]{\emph{\code{scenario.Scenario}}}}

This class parametrizes / runs a simulation scenario for the Singleton ({\hyperref[singleton:singleton.Singleton]{\emph{\code{singleton.Singleton}}}}) person.
\begin{quote}\begin{description}
\item[{Parameters}] \leavevmode
\textbf{\texttt{hhld\_params}} ({\hyperref[params:params.Params]{\emph{\emph{params.Params}}}}) -- the parameters for the household that contain relevant information     for the simulation

\end{description}\end{quote}

\end{fulllineitems}



\section{singleton module}
\label{singleton::doc}\label{singleton:module-singleton}\label{singleton:singleton-module}\index{singleton (module)}
This file contains information for creating the default agent that represents a person that lives alone in the home. Singleton will be the name of this type of agent.

This module contains {\hyperref[singleton:singleton.Singleton]{\emph{\code{singleton.Singleton}}}}.

\emph{Module author: Dr. Namdi Brandon}
\index{Singleton (class in singleton)}

\begin{fulllineitems}
\phantomsection\label{singleton:singleton.Singleton}\pysiglinewithargsret{\strong{class }\code{singleton.}\bfcode{Singleton}}{\emph{house}, \emph{clock}, \emph{schedule}}{}
Bases: {\hyperref[person:person.Person]{\emph{\code{person.Person}}}}

Singleton default is a person that has the following characteristics
\begin{enumerate}
\item {} 
female

\item {} 
30 years old

\item {} 
goes to bed at 22:00 and sleeps for 8 hours

\item {} 
lives alone and has no children

\item {} 
works the Standard Job

\item {} 
eats breakfast at 7:30 for 15 minutes, lunch at 12:00 for 30 minutes,     and dinner at 19:00 for 45 minutes

\end{enumerate}
\begin{quote}\begin{description}
\item[{Parameters}] \leavevmode\begin{itemize}
\item {} 
\textbf{\texttt{house}} ({\hyperref[home:home.Home]{\emph{\emph{home.Home}}}}) -- the place of residence

\item {} 
\textbf{\texttt{clock}} ({\hyperref[temporal:temporal.Temporal]{\emph{\emph{temporal.Temporal}}}}) -- the clock running in the simulation

\item {} 
\textbf{\texttt{schedule}} ({\hyperref[scheduler:scheduler.Scheduler]{\emph{\emph{scheduler.Scheduler}}}}) -- the schedule for the agent

\end{itemize}

\end{description}\end{quote}
\index{print\_params() (singleton.Singleton method)}

\begin{fulllineitems}
\phantomsection\label{singleton:singleton.Singleton.print_params}\pysiglinewithargsret{\bfcode{print\_params}}{}{}
This function prints the activity-parameter means in chronological order of start time. This         results in the ability to print the mean daily routine.
\begin{quote}\begin{description}
\item[{Returns}] \leavevmode
a representation of the parameters of the agent in increasing values of         start time

\item[{Return type}] \leavevmode
str

\end{description}\end{quote}

\end{fulllineitems}

\index{set() (singleton.Singleton method)}

\begin{fulllineitems}
\phantomsection\label{singleton:singleton.Singleton.set}\pysiglinewithargsret{\bfcode{set}}{\emph{param}, \emph{idx}}{}
This function sets the Singleton's parameters.

The function does the following
\begin{enumerate}
\item {} 
Sets the biology

\item {} 
Sets the job information

\item {} 
Sets the alarm

\item {} 
Sets the meal information

\end{enumerate}
\begin{quote}\begin{description}
\item[{Parameters}] \leavevmode\begin{itemize}
\item {} 
\textbf{\texttt{param}} ({\hyperref[params:params.Params]{\emph{\emph{params.Params}}}}) -- parameters describing the household

\item {} 
\textbf{\texttt{idx}} (\emph{int}) -- the respective index number of the Person of interest in the household

\end{itemize}

\item[{Returns}] \leavevmode
None

\end{description}\end{quote}

\end{fulllineitems}


\end{fulllineitems}



\chapter{Indices and tables}
\label{index:indices-and-tables}\begin{itemize}
\item {} 
\DUspan{xref,std,std-ref}{genindex}

\item {} 
\DUspan{xref,std,std-ref}{modindex}

\item {} 
\DUspan{xref,std,std-ref}{search}

\end{itemize}


\renewcommand{\indexname}{Python Module Index}
\begin{theindex}
\def\bigletter#1{{\Large\sffamily#1}\nopagebreak\vspace{1mm}}
\bigletter{a}
\item {\texttt{activity}}, \pageref{activity:module-activity}
\item {\texttt{asset}}, \pageref{asset:module-asset}
\indexspace
\bigletter{b}
\item {\texttt{bed}}, \pageref{bed:module-bed}
\item {\texttt{bio}}, \pageref{bio:module-bio}
\indexspace
\bigletter{c}
\item {\texttt{commute}}, \pageref{commute:module-commute}
\indexspace
\bigletter{d}
\item {\texttt{diary}}, \pageref{diary:module-diary}
\indexspace
\bigletter{e}
\item {\texttt{eat}}, \pageref{eat:module-eat}
\indexspace
\bigletter{f}
\item {\texttt{food}}, \pageref{food:module-food}
\indexspace
\bigletter{h}
\item {\texttt{home}}, \pageref{home:module-home}
\item {\texttt{hunger}}, \pageref{hunger:module-hunger}
\indexspace
\bigletter{i}
\item {\texttt{income}}, \pageref{income:module-income}
\item {\texttt{interrupt}}, \pageref{interrupt:module-interrupt}
\item {\texttt{interruption}}, \pageref{interruption:module-interruption}
\indexspace
\bigletter{l}
\item {\texttt{location}}, \pageref{location:module-location}
\indexspace
\bigletter{m}
\item {\texttt{main}}, \pageref{main:module-main}
\item {\texttt{main\_params}}, \pageref{main_params:module-main_params}
\item {\texttt{meal}}, \pageref{meal:module-meal}
\item {\texttt{my\_globals}}, \pageref{my_globals:module-my_globals}
\indexspace
\bigletter{n}
\item {\texttt{need}}, \pageref{need:module-need}
\indexspace
\bigletter{o}
\item {\texttt{occupation}}, \pageref{occupation:module-occupation}
\indexspace
\bigletter{p}
\item {\texttt{params}}, \pageref{params:module-params}
\item {\texttt{person}}, \pageref{person:module-person}
\indexspace
\bigletter{r}
\item {\texttt{rest}}, \pageref{rest:module-rest}
\indexspace
\bigletter{s}
\item {\texttt{scenario}}, \pageref{scenario:module-scenario}
\item {\texttt{scheduler}}, \pageref{scheduler:module-scheduler}
\item {\texttt{singleton}}, \pageref{singleton:module-singleton}
\item {\texttt{sleep}}, \pageref{sleep:module-sleep}
\item {\texttt{social}}, \pageref{social:module-social}
\item {\texttt{state}}, \pageref{state:module-state}
\indexspace
\bigletter{t}
\item {\texttt{temporal}}, \pageref{temporal:module-temporal}
\item {\texttt{transport}}, \pageref{transport:module-transport}
\item {\texttt{travel}}, \pageref{travel:module-travel}
\indexspace
\bigletter{u}
\item {\texttt{universe}}, \pageref{universe:module-universe}
\indexspace
\bigletter{w}
\item {\texttt{work}}, \pageref{work:module-work}
\item {\texttt{workplace}}, \pageref{workplace:module-workplace}
\end{theindex}

\renewcommand{\indexname}{Index}
\printindex
\end{document}
